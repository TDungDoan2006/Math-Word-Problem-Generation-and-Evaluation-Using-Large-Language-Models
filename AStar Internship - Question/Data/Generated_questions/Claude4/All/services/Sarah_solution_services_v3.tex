\documentclass{article}
\usepackage[utf8]{inputenc}
\usepackage{amsmath}
\usepackage{amsfonts}
\usepackage{amssymb}
\usepackage{graphicx}
\usepackage{hyperref}
\title{'Sarah Solutions services v3 CLAUDE '}
\author{Tien Dung Doan}
\begin{document}
\maketitle
\section*{Question 1}
\textbf{Metadata}

\begin{itemize}
  \item Question ID: P3-WNDivRmd3d\_P1-WNCmp\_sonnet4\_Services\_01
  \item Primary KC: WHOLE NUMBERS | Division | dividing whole numbers up to 3 digits by 1 digit with remainder 
  \item Secondary KC: WHOLE NUMBERS | Comparison and ordering | comparing and ordering whole numbers
  \item Topic: Services such as installation, maintenance, repairing, cleaning, laundry, hotel, retail, e-commerce, streaming services, digital services etc.
  \item Grade: Primary 3
\end{itemize}

\textbf{Solution}

To find how many rooms each team had left over, I need to divide and find the remainder for each team.

Team A: $347 \div 4$
$347 = 4 \times 86 + 3$
So Team A has $3$ rooms left over.

Team B: $256 \div 3$
$256 = 3 \times 85 + 1$
So Team B has $1$ room left over.

Team C: $189 \div 2$
$189 = 2 \times 94 + 1$
So Team C has $1$ room left over.

Now I compare the remainders:
Team A: $3$ rooms left over
Team B: $1$ room left over
Team C: $1$ room left over

Comparing: $3 > 1$

Therefore, Team A had the most rooms left over with $3$ rooms.
Team B and Team C both had the least rooms left over with $1$ room each.

\section*{Question 2}
\textbf{Metadata}

\begin{itemize}
  \item Question ID: P3-WNMul3d1d\_P1-WNAdd2nd\_sonnet4\_Services\_01
  \item Primary KC: WHOLE NUMBERS | Multiplication | multiplying whole numbers up to 3 digits by 1 digit
  \item Secondary KC: WHOLE NUMBERS | Addition | adding whole numbers
  \item Topic: Services such as installation, maintenance, repairing, cleaning, laundry, hotel, retail, e-commerce, streaming services, digital services etc.
  \item Grade: Primary 3
\end{itemize}

\textbf{Solution}

Step 1: Find the total amount earned from laptop repairs.
Number of laptops repaired = 4
Charge per laptop repair = \textdollar125
Total earned from laptop repairs = $4 \times 125 = \textdollar500$

Step 2: Find the total amount earned from all repairs.
Amount earned from laptop repairs = \textdollar500
Amount earned from desktop repairs = \textdollar89
Total amount earned = $500 + 89 = \textdollar589$

Therefore, Mr. Tan earned \textdollar589 in total from his computer repair services last week.

\section*{Question 3}
\textbf{Metadata}

\begin{itemize}
  \item Question ID: P3-WNMul3d1d\_P1-WNSub2nd\_sonnet4\_Services\_01
  \item Primary KC: WHOLE NUMBERS | Multiplication | multiplying whole numbers up to 3 digits by 1 digit
  \item Secondary KC: WHOLE NUMBERS | Subtraction | subtracting whole numbers
  \item Topic: Services such as installation, maintenance, repairing, cleaning, laundry, hotel, retail, e-commerce, streaming services, digital services etc.
  \item Grade: Primary 3
\end{itemize}

\textbf{Solution}

First, I need to find how much the company earned on Monday.

On Monday:
- 3 houses with 25 rooms each: $3 \times 25 = 75$ rooms
- 2 houses with 18 rooms each: $2 \times 18 = 36$ rooms  
- 4 houses with 12 rooms each: $4 \times 12 = 48$ rooms

Total rooms cleaned on Monday: $75 + 36 + 48 = 159$ rooms

Money earned on Monday: $159 \times 8 = \textdollar1272$

Next, I need to find how much the company earned on Tuesday.

On Tuesday:
- 5 houses with 23 rooms each: $5 \times 23 = 115$ rooms

Money earned on Tuesday: $115 \times 8 = \textdollar920$

Finally, I need to find the difference.

Difference: $\textdollar1272 - \textdollar920 = \textdollar352$

Therefore, the company earned \textdollar352 more on Monday than on Tuesday.

\section*{Question 4}
\textbf{Metadata}

\begin{itemize}
  \item Question ID: P3-FrAddRl12\_P3-FrSmp\_sonnet4\_Services\_01
  \item Primary KC: FRACTIONS | Addition | adding two related fractions within one whole with denominators of given fractions not exceeding 12
  \item Secondary KC: FRACTIONS | Simplifying | expressing a fraction in its simplest form
  \item Topic: Services such as installation, maintenance, repairing, cleaning, laundry, hotel, retail, e-commerce, streaming services, digital services etc.
  \item Grade: Primary 3
\end{itemize}

\textbf{Solution}

To find the total fraction of laundry orders Sarah completed, I need to add the fractions from the morning and afternoon.

Morning: $\frac{2}{9}$
Afternoon: $\frac{4}{9}$

Since both fractions have the same denominator, I can add them directly:
$\frac{2}{9} + \frac{4}{9} = \frac{2+4}{9} = \frac{6}{9}$

Now I need to express $\frac{6}{9}$ in its simplest form by finding the greatest common factor of 6 and 9.

Factors of 6: 1, 2, 3, 6
Factors of 9: 1, 3, 9

The greatest common factor is 3.

$\frac{6}{9} = \frac{6 \div 3}{9 \div 3} = \frac{2}{3}$

Therefore, Sarah completed $\frac{2}{3}$ of the total laundry orders.

\section*{Question 5}
\textbf{Metadata}

\begin{itemize}
  \item Question ID: P3-FrSubRl12\_P3-FrSmp\_sonnet4\_Services\_01
  \item Primary KC: FRACTIONS | Subtraction | subtracting two related fractions within one whole with denominators of given fractions not exceeding 12
  \item Secondary KC: FRACTIONS | Simplifying | expressing a fraction in its simplest form
  \item Topic: Services such as installation, maintenance, repairing, cleaning, laundry, hotel, retail, e-commerce, streaming services, digital services etc.
  \item Grade: Primary 3
\end{itemize}

\textbf{Solution}

To find how much more of the office building Sarah cleaned on Monday than on Tuesday, I need to subtract the fraction she cleaned on Tuesday from the fraction she cleaned on Monday.

Amount cleaned on Monday: $\frac{5}{6}$
Amount cleaned on Tuesday: $\frac{1}{6}$

Difference = $\frac{5}{6} - \frac{1}{6}$

Since both fractions have the same denominator, I can subtract the numerators:
$\frac{5}{6} - \frac{1}{6} = \frac{5-1}{6} = \frac{4}{6}$

Now I need to express $\frac{4}{6}$ in its simplest form by finding the greatest common factor of 4 and 6.
Factors of 4: 1, 2, 4
Factors of 6: 1, 2, 3, 6
Greatest common factor = 2

$\frac{4}{6} = \frac{4 \div 2}{6 \div 2} = \frac{2}{3}$

Therefore, Sarah cleaned $\frac{2}{3}$ more of the office building on Monday than on Tuesday.

\section*{Question 6}
\textbf{Metadata}

\begin{itemize}
  \item Question ID: P4-WNMul4d1d\_P1-WNSub2nd\_sonnet4\_Services\_01
  \item Primary KC: WHOLE NUMBERS | Multiplication | multiplying whole numbers up to 4 digits by 1 digit or up to 3 digits by 2 digits
  \item Secondary KC: WHOLE NUMBERS | Subtraction | subtracting whole numbers
  \item Topic: Services such as installation, maintenance, repairing, cleaning, laundry, hotel, retail, e-commerce, streaming services, digital services etc.
  \item Grade: Primary 4
\end{itemize}

\textbf{Solution}

Step 1: Calculate the total number of rooms cleaned in January.
Rooms cleaned in January = $24 \times 15 = 360$ rooms

Step 2: Calculate the total number of rooms cleaned in February.
Rooms cleaned in February = $32 \times 12 = 384$ rooms

Step 3: Calculate the total number of rooms cleaned in both months.
Total rooms cleaned = $360 + 384 = 744$ rooms

Step 4: Calculate the total amount charged before refunds.
Total amount charged = $744 \times 18 = \textdollar13,392$

Step 5: Calculate the actual amount received after refunds.
Actual amount received = $\textdollar13,392 - \textdollar2,592 = \textdollar10,800$

Therefore, the cleaning service company actually received \textdollar10,800 after the refunds.

\section*{Question 7}
\textbf{Metadata}

\begin{itemize}
  \item Question ID: P4-WNDiv4d1d\_P1-WNAdd2nd\_sonnet4\_Services\_01
  \item Primary KC: WHOLE NUMBERS | Division | dividing whole numbers up to 4 digits by 1 digit
  \item Secondary KC: WHOLE NUMBERS | Addition | adding whole numbers
  \item Topic: Services such as installation, maintenance, repairing, cleaning, laundry, hotel, retail, e-commerce, streaming services, digital services etc.
  \item Grade: Primary 4
\end{itemize}

\textbf{Solution}

Step 1: Find how many tablets still need software installation.
Tablets remaining = Total tablets - Completed tablets
Tablets remaining = $2856 - 1548 = 1308$ tablets

Step 2: Find how many more hours are needed to finish the remaining tablets.
Since the technicians can install software on 6 tablets per hour:
Hours needed = Tablets remaining $\div$ Rate per hour
Hours needed = $1308 \div 6 = 218$ hours

Therefore, they still need to work on 1308 tablets, and it will take them 218 more hours to complete the remaining installations.

\section*{Question 8}
\textbf{Metadata}

\begin{itemize}
  \item Question ID: P4-FrRepSet\_P3-FrCnvEq\_sonnet4\_Services\_01
  \item Primary KC: FRACTIONS | Representation and concept | expressing a part of a set as a fraction
  \item Secondary KC: FRACTIONS | Conversion to equivalent fractions | Conversion to equivalent fractions (given either the denominator or the numerator)
  \item Topic: Services such as installation, maintenance, repairing, cleaning, laundry, hotel, retail, e-commerce, streaming services, digital services etc.
  \item Grade: Primary 4
\end{itemize}

\textbf{Solution}

Step 1: Find how many laptops are on the first shelf.
Sarah places $\frac{1}{3}$ of the 12 laptops on the first shelf.
Number of laptops = $\frac{1}{3} \times 12 = \frac{12}{3} = 4$ laptops

Step 2: Convert $\frac{1}{3}$ to an equivalent fraction with denominator 9.
To change the denominator from 3 to 9, we multiply by 3.
$\frac{1}{3} = \frac{1 \times 3}{3 \times 3} = \frac{3}{9}$

Step 3: Verify the answer.
$\frac{3}{9}$ of 12 laptops = $\frac{3}{9} \times 12 = \frac{36}{9} = 4$ laptops ✓

Therefore, there are 4 laptops on the first shelf, and the equivalent fraction is $\frac{3}{9}$.

\section*{Question 9}
\textbf{Metadata}

\begin{itemize}
  \item Question ID: P4-FrSubU12\_P2-FrAdd2nd\_sonnet4\_Services\_01
  \item Primary KC: FRACTIONS | Subtraction | subtracting unlike fractions with two different denominators not exceeding 12
  \item Secondary KC: FRACTIONS | Addition | adding fractions
  \item Topic: Services such as installation, maintenance, repairing, cleaning, laundry, hotel, retail, e-commerce, streaming services, digital services etc.
  \item Grade: Primary 4
\end{itemize}

\textbf{Solution}

Step 1: Find the total fraction of shirts Sarah washed during the day.

Sarah washed $\frac{5}{8}$ of the shirts in the morning and $\frac{1}{6}$ in the afternoon.

To add $\frac{5}{8} + \frac{1}{6}$, I need to find a common denominator.

The LCM of 8 and 6 is 24.

$\frac{5}{8} = \frac{5 \times 3}{8 \times 3} = \frac{15}{24}$

$\frac{1}{6} = \frac{1 \times 4}{6 \times 4} = \frac{4}{24}$

$\frac{5}{8} + \frac{1}{6} = \frac{15}{24} + \frac{4}{24} = \frac{19}{24}$

Step 2: Find the fraction of shirts that got stained and need to be rewashed.

The stained shirts represent $\frac{1}{4}$ of the morning's work, which was $\frac{5}{8}$ of the total.

Stained shirts = $\frac{1}{4} \times \frac{5}{8} = \frac{5}{32}$ of the total shirts.

Step 3: Find the fraction of shirts successfully completed.

Since the stained shirts need to be rewashed, they are not successfully completed yet.

Successfully completed shirts = Total washed - Stained shirts

$= \frac{19}{24} - \frac{5}{32}$

To subtract, I need a common denominator. The LCM of 24 and 32 is 96.

$\frac{19}{24} = \frac{19 \times 4}{24 \times 4} = \frac{76}{96}$

$\frac{5}{32} = \frac{5 \times 3}{32 \times 3} = \frac{15}{96}$

$\frac{19}{24} - \frac{5}{32} = \frac{76}{96} - \frac{15}{96} = \frac{61}{96}$

Therefore, Sarah successfully completed $\frac{61}{96}$ of the total shirts by the end of Monday.

\section*{Question 10}
\textbf{Metadata}

\begin{itemize}
  \item Question ID: P4-DcSub2d\_P4-DcCmp3d\_sonnet4\_Services\_01
  \item Primary KC: DECIMALS | Subtraction | subtracting decimals (up to 2 decimal places)
  \item Secondary KC: DECIMALS | Comparison and ordering | comparing and ordering decimals up to 3 decimal places
  \item Topic: Services such as installation, maintenance, repairing, cleaning, laundry, hotel, retail, e-commerce, streaming services, digital services etc.
  \item Grade: Primary 4
\end{itemize}

\textbf{Solution}

Step 1: Find Sarah's total earnings over the three days.
Monday: \textdollar24.75
Tuesday: \textdollar18.50
Wednesday: \textdollar31.25

Total earnings = \textdollar24.75 + \textdollar18.50 + \textdollar31.25 = \textdollar74.50

Step 2: Compare her total earnings with her goal.
Sarah's goal: \textdollar50.00
Sarah's actual earnings: \textdollar74.50

Since \textdollar74.50 > \textdollar50.00, Sarah met her goal.

Step 3: Find by how much she exceeded her goal.
Amount exceeded = \textdollar74.50 - \textdollar50.00 = \textdollar24.50

Therefore, Sarah earned \textdollar74.50 in total. She met her goal and exceeded it by \textdollar24.50.

\section*{Question 11}
\textbf{Metadata}

\begin{itemize}
  \item Question ID: P4-DcMul2d1d\_P4-DcCmp3d\_sonnet4\_Services\_01
  \item Primary KC: DECIMALS | Multiplication | multiplying decimals (up to 2 decimal places) by a 1-digit whole number
  \item Secondary KC: DECIMALS | Comparison and ordering | comparing and ordering decimals up to 3 decimal places
  \item Topic: Services such as installation, maintenance, repairing, cleaning, laundry, hotel, retail, e-commerce, streaming services, digital services etc.
  \item Grade: Primary 4
\end{itemize}

\textbf{Solution}

Step 1: Calculate the cost for cleaning bedrooms.
Cost for bedrooms = \textdollar3.75 × 4 = \textdollar15.00

Step 2: Calculate the cost for cleaning kitchens.
Cost for kitchens = \textdollar4.25 × 3 = \textdollar12.75

Step 3: Calculate the cost for cleaning bathrooms.
Cost for bathrooms = \textdollar2.80 × 6 = \textdollar16.80

Step 4: Compare and order the costs in ascending order.
The three costs are: \textdollar15.00, \textdollar12.75, and \textdollar16.80
Comparing: \textdollar12.75 < \textdollar15.00 < \textdollar16.80

Therefore, the costs in ascending order are:
\textdollar12.75 (kitchens), \textdollar15.00 (bedrooms), \textdollar16.80 (bathrooms)

\section*{Question 12}
\textbf{Metadata}

\begin{itemize}
  \item Question ID: P4-DcMul2d1d\_P4-DcRnd3d\_sonnet4\_Services\_01
  \item Primary KC: DECIMALS | Multiplication | multiplying decimals (up to 2 decimal places) by a 1-digit whole number
  \item Secondary KC: DECIMALS | Rounding | rounding decimals up to 3 decimal places to the nearest whole number, 1 decimal place and 2 decimal places 
  \item Topic: Services such as installation, maintenance, repairing, cleaning, laundry, hotel, retail, e-commerce, streaming services, digital services etc.
  \item Grade: Primary 4
\end{itemize}

\textbf{Solution}

Step 1: Calculate the total amount Mrs. Tan earned.
Amount per hour = \textdollar12.75
Number of hours worked = 6
Total amount = \textdollar12.75 × 6

To multiply \textdollar12.75 × 6:
\textdollar12.75 × 6 = \textdollar76.50

Step 2: Round the answer to the nearest whole number.
\textdollar76.50 rounded to the nearest whole number = \textdollar77
(Since the digit in the tenths place is 5, we round up)

Step 3: Round the answer to 1 decimal place.
\textdollar76.50 rounded to 1 decimal place = \textdollar76.5

Therefore:
- Total amount earned = \textdollar76.50
- Rounded to the nearest whole number = \textdollar77
- Rounded to 1 decimal place = \textdollar76.5

\section*{Question 13}
\textbf{Metadata}

\begin{itemize}
  \item Question ID: P4-DcMul2d1d\_P4-DcSub2nd\_sonnet4\_Services\_01
  \item Primary KC: DECIMALS | Multiplication | multiplying decimals (up to 2 decimal places) by a 1-digit whole number
  \item Secondary KC: DECIMALS | Subtraction | subtracting decimals
  \item Topic: Services such as installation, maintenance, repairing, cleaning, laundry, hotel, retail, e-commerce, streaming services, digital services etc.
  \item Grade: Primary 4
\end{itemize}

\textbf{Solution}

Step 1: Calculate the total cost of the cleaning service.
Cost per hour = \textdollar12.75
Number of hours = 3
Total cost = \textdollar12.75 × 3 = \textdollar38.25

Step 2: Calculate how much money Sarah's mother will have left over.
Budgeted amount = \textdollar45.00
Actual cost = \textdollar38.25
Money left over = \textdollar45.00 - \textdollar38.25 = \textdollar6.75

Therefore, the total cost of the cleaning service is \textdollar38.25, and Sarah's mother will have \textdollar6.75 left over from her budget.

\section*{Question 14}
\textbf{Metadata}

\begin{itemize}
  \item Question ID: P4-DcDiv2d1d\_P4-DcRnd3d\_sonnet4\_Services\_01
  \item Primary KC: DECIMALS | Division | dividing decimals (up to 2 decimal places) by a 1-digit whole number
  \item Secondary KC: DECIMALS | Rounding | rounding decimals up to 3 decimal places to the nearest whole number, 1 decimal place and 2 decimal places 
  \item Topic: Services such as installation, maintenance, repairing, cleaning, laundry, hotel, retail, e-commerce, streaming services, digital services etc.
  \item Grade: Primary 4
\end{itemize}

\textbf{Solution}

To find the cost per room, I need to divide the total cost by the number of rooms.

Cost per room = Total cost $\div$ Number of rooms
Cost per room = $\textdollar12.45 \div 3$

To divide $12.45$ by $3$:
$12.45 \div 3 = 4.15$

So the cost per room is $\textdollar4.15$.

Since $\textdollar4.15$ already has exactly 2 decimal places, rounding to the nearest cent (2 decimal places) gives us $\textdollar4.15$.

Therefore, the cost per room is $\textdollar4.15$.

\section*{Question 15}
\textbf{Metadata}

\begin{itemize}
  \item Question ID: P5-FrAddMix\_P5-FrCnv2Dc\_sonnet4\_Services\_01
  \item Primary KC: FRACTIONS | Addition | adding mixed numbers
  \item Secondary KC: FRACTIONS | Conversion to decimals | expressing fractions as decimals
  \item Topic: Services such as installation, maintenance, repairing, cleaning, laundry, hotel, retail, e-commerce, streaming services, digital services etc.
  \item Grade: Primary 5
\end{itemize}

\textbf{Solution}

First, I need to find the total cost for diagnosis work.
Diagnosis cost = \textdollar$2\frac{3}{4}$ per hour $\times 3$ hours
$2\frac{3}{4} \times 3 = \frac{11}{4} \times 3 = \frac{33}{4} = 8\frac{1}{4}$
So diagnosis cost = \textdollar$8\frac{1}{4}$

Next, I need to find the total cost for repair work.
Repair cost = \textdollar$4\frac{1}{2}$ per hour $\times 2$ hours
$4\frac{1}{2} \times 2 = \frac{9}{2} \times 2 = \frac{18}{2} = 9$
So repair cost = \textdollar$9$

Now I can find the total labor cost by adding the two costs.
Total cost = \textdollar$8\frac{1}{4}$ + \textdollar$9$
$8\frac{1}{4} + 9 = 8\frac{1}{4} + 9\frac{0}{4} = 17\frac{1}{4}$

Finally, I need to express $17\frac{1}{4}$ as a decimal.
$17\frac{1}{4} = 17 + \frac{1}{4} = 17 + 0.25 = 17.25$

Therefore, the total labor cost was \textdollar$17.25$.

\section*{Question 16}
\textbf{Metadata}

\begin{itemize}
  \item Question ID: P5-FrSubMix\_P3-FrSmp\_sonnet4\_Services\_01
  \item Primary KC: FRACTIONS | Subtraction | subtracting mixed numbers
  \item Secondary KC: FRACTIONS | Simplifying | expressing a fraction in its simplest form
  \item Topic: Services such as installation, maintenance, repairing, cleaning, laundry, hotel, retail, e-commerce, streaming services, digital services etc.
  \item Grade: Primary 5
\end{itemize}

\textbf{Solution}

To find how much more solution was used on Monday than Tuesday, I need to subtract the amount used on Tuesday from the amount used on Monday.

Amount used on Monday: $3\frac{2}{5}$ litres
Amount used on Tuesday: $1\frac{3}{4}$ litres

I need to calculate: $3\frac{2}{5} - 1\frac{3}{4}$

First, I'll convert both mixed numbers to improper fractions:
$3\frac{2}{5} = \frac{3 \times 5 + 2}{5} = \frac{15 + 2}{5} = \frac{17}{5}$

$1\frac{3}{4} = \frac{1 \times 4 + 3}{4} = \frac{4 + 3}{4} = \frac{7}{4}$

Now I need to subtract: $\frac{17}{5} - \frac{7}{4}$

To subtract fractions, I need a common denominator. The LCM of 5 and 4 is 20.

$\frac{17}{5} = \frac{17 \times 4}{5 \times 4} = \frac{68}{20}$

$\frac{7}{4} = \frac{7 \times 5}{4 \times 5} = \frac{35}{20}$

Now I can subtract: $\frac{68}{20} - \frac{35}{20} = \frac{68 - 35}{20} = \frac{33}{20}$

To check if $\frac{33}{20}$ is in its simplest form, I need to find the GCD of 33 and 20.
Factors of 33: 1, 3, 11, 33
Factors of 20: 1, 2, 4, 5, 10, 20
The GCD is 1, so $\frac{33}{20}$ is already in its simplest form.

Converting back to a mixed number: $\frac{33}{20} = 1\frac{13}{20}$

Therefore, the cleaning service used $1\frac{13}{20}$ litres more carpet cleaning solution on Monday than on Tuesday.

\section*{Question 17}
\textbf{Metadata}

\begin{itemize}
  \item Question ID: P5-FrSubMix\_P5-FrCnv2Dc\_sonnet4\_Services\_01
  \item Primary KC: FRACTIONS | Subtraction | subtracting mixed numbers
  \item Secondary KC: FRACTIONS | Conversion to decimals | expressing fractions as decimals
  \item Topic: Services such as installation, maintenance, repairing, cleaning, laundry, hotel, retail, e-commerce, streaming services, digital services etc.
  \item Grade: Primary 5
\end{itemize}

\textbf{Solution}

To find how many more floors were cleaned on Monday than on Tuesday, I need to subtract the mixed numbers and then convert to a decimal.

Step 1: Subtract the mixed numbers $3\frac{3}{4} - 2\frac{1}{2}$

First, I'll convert both mixed numbers to improper fractions:
$3\frac{3}{4} = \frac{3 \times 4 + 3}{4} = \frac{15}{4}$
$2\frac{1}{2} = \frac{2 \times 2 + 1}{2} = \frac{5}{2}$

To subtract, I need a common denominator. The LCD of 4 and 2 is 4:
$\frac{5}{2} = \frac{5 \times 2}{2 \times 2} = \frac{10}{4}$

Now I can subtract:
$\frac{15}{4} - \frac{10}{4} = \frac{15-10}{4} = \frac{5}{4}$

Step 2: Convert $\frac{5}{4}$ to a mixed number:
$\frac{5}{4} = 1\frac{1}{4}$

Step 3: Convert $1\frac{1}{4}$ to a decimal:
$\frac{1}{4} = 1 \div 4 = 0.25$

Therefore: $1\frac{1}{4} = 1 + 0.25 = 1.25$

The hotel cleaning service cleaned 1.25 more floors on Monday than on Tuesday.

\section*{Question 18}
\textbf{Metadata}

\begin{itemize}
  \item Question ID: P5-FrMulImN\_P2-FrAdd2nd\_sonnet4\_Services\_01
  \item Primary KC: FRACTIONS | Multiplication | multiplying a proper/improper fraction and a whole number
  \item Secondary KC: FRACTIONS | Addition | adding fractions
  \item Topic: Services such as installation, maintenance, repairing, cleaning, laundry, hotel, retail, e-commerce, streaming services, digital services etc.
  \item Grade: Primary 5
\end{itemize}

\textbf{Solution}

Step 1: Calculate the earnings from Mrs. Lee's house.
Time worked = $\frac{3}{4}$ hour
Earnings = $24 \times \frac{3}{4} = \frac{24 \times 3}{4} = \frac{72}{4} = 18$
So they earned \textdollar18 from Mrs. Lee's house.

Step 2: Calculate the earnings from Mr. Tan's house.
Time worked = $\frac{5}{6}$ hour
Earnings = $24 \times \frac{5}{6} = \frac{24 \times 5}{6} = \frac{120}{6} = 20$
So they earned \textdollar20 from Mr. Tan's house.

Step 3: Find the total time worked.
Total time = $\frac{3}{4} + \frac{5}{6}$
To add these fractions, find the common denominator:
$\frac{3}{4} = \frac{3 \times 3}{4 \times 3} = \frac{9}{12}$
$\frac{5}{6} = \frac{5 \times 2}{6 \times 2} = \frac{10}{12}$
Total time = $\frac{9}{12} + \frac{10}{12} = \frac{19}{12}$ hours

Step 4: Calculate total earnings.
Total earnings = \textdollar18 + \textdollar20 = \textdollar38

Alternatively, we can verify: $24 \times \frac{19}{12} = \frac{24 \times 19}{12} = \frac{456}{12} = 38$

Therefore, the cleaning service company earned \textdollar38 in total on Monday.

\section*{Question 19}
\textbf{Metadata}

\begin{itemize}
  \item Question ID: P5-FrMulImN\_P3-FrSmp\_sonnet4\_Services\_01
  \item Primary KC: FRACTIONS | Multiplication | multiplying a proper/improper fraction and a whole number
  \item Secondary KC: FRACTIONS | Simplifying | expressing a fraction in its simplest form
  \item Topic: Services such as installation, maintenance, repairing, cleaning, laundry, hotel, retail, e-commerce, streaming services, digital services etc.
  \item Grade: Primary 5
\end{itemize}

\textbf{Solution}

To find the total number of bottles needed, I need to multiply the number of bottles per building by the number of buildings.

Number of bottles per building = $\frac{3}{4}$
Number of buildings = 8

Total bottles needed = $\frac{3}{4} \times 8$

To multiply a fraction by a whole number:
$\frac{3}{4} \times 8 = \frac{3 \times 8}{4} = \frac{24}{4}$

Now I need to simplify $\frac{24}{4}$:
$\frac{24}{4} = 6$

Therefore, the cleaning service company will need 6 bottles of floor cleaner in total.

\section*{Question 20}
\textbf{Metadata}

\begin{itemize}
  \item Question ID: P5-FrMulImN\_P5-FrCnv2Dc\_sonnet4\_Services\_01
  \item Primary KC: FRACTIONS | Multiplication | multiplying a proper/improper fraction and a whole number
  \item Secondary KC: FRACTIONS | Conversion to decimals | expressing fractions as decimals
  \item Topic: Services such as installation, maintenance, repairing, cleaning, laundry, hotel, retail, e-commerce, streaming services, digital services etc.
  \item Grade: Primary 5
\end{itemize}

\textbf{Solution}

Step 1: Calculate the earnings for Monday.
Earnings on Monday = $\frac{3}{4} \times 12$
$= \frac{3 \times 12}{4}$
$= \frac{36}{4}$
$= 9$
So the earnings on Monday = \textdollar9

Step 2: Calculate the earnings for Tuesday.
Earnings on Tuesday = $\frac{5}{6} \times 12$
$= \frac{5 \times 12}{6}$
$= \frac{60}{6}$
$= 10$
So the earnings on Tuesday = \textdollar10

Step 3: Find the total earnings.
Total earnings = \textdollar9 + \textdollar10 = \textdollar19

Step 4: Express the answer as a decimal.
Since \textdollar19 is already a whole number, expressed as a decimal it is \textdollar19.00

Therefore, the total amount earned over the two days is \textdollar19.00.

\section*{Question 21}
\textbf{Metadata}

\begin{itemize}
  \item Question ID: P5-FrMulImIm\_P2-FrCmp\_sonnet4\_Services\_01
  \item Primary KC: FRACTIONS | Multiplication | multiplying two improper fractions
  \item Secondary KC: FRACTIONS | Comparison and ordering | comparing and ordering fractions
  \item Topic: Services such as installation, maintenance, repairing, cleaning, laundry, hotel, retail, e-commerce, streaming services, digital services etc.
  \item Grade: Primary 5
\end{itemize}

\textbf{Solution}

First, I need to find how many floors will use each solution.

Total floors = 12

Floors using Solution A = $\frac{8}{5} \times 12 = \frac{8 \times 12}{5} = \frac{96}{5}$ floors

Floors using Solution B = $12 - \frac{96}{5} = \frac{60}{5} - \frac{96}{5} = \frac{60-96}{5} = -\frac{36}{5}$

Wait, let me recalculate this properly.

Floors using Solution A = $\frac{8}{5} \times 12 = \frac{96}{5} = 19\frac{1}{5}$ floors

This is more than 12 floors, which is impossible. Let me reread the problem.

Actually, let me interpret this as: Solution A is used for $\frac{8}{5}$ times as many floors as a base amount.

Let me revise: If Solution A is used for $\frac{5}{8}$ of the total floors instead.

Floors using Solution A = $\frac{5}{8} \times 12 = \frac{5 \times 12}{8} = \frac{60}{8} = \frac{15}{2} = 7\frac{1}{2}$ floors

Floors using Solution B = $12 - 7\frac{1}{2} = 4\frac{1}{2}$ floors

Now I can calculate the concentrate needed:

Concentrate for Solution A = $\frac{5}{3} \times \frac{15}{2} = \frac{5 \times 15}{3 \times 2} = \frac{75}{6} = 12\frac{1}{2}$ litres

Concentrate for Solution B = $\frac{7}{4} \times \frac{9}{2} = \frac{7 \times 9}{4 \times 2} = \frac{63}{8} = 7\frac{7}{8}$ litres

To compare: $12\frac{1}{2} = \frac{25}{2} = \frac{100}{8}$ and $7\frac{7}{8} = \frac{63}{8}$

Since $\frac{100}{8} > \frac{63}{8}$, Solution A requires more concentrate.

Therefore, Solution A needs $12\frac{1}{2}$ litres and Solution B needs $7\frac{7}{8}$ litres. Solution A requires more concentrate in total.

\section*{Question 22}
\textbf{Metadata}

\begin{itemize}
  \item Question ID: P5-FrMulImIm\_P5-FrCnv2Dc\_sonnet4\_Services\_01
  \item Primary KC: FRACTIONS | Multiplication | multiplying two improper fractions
  \item Secondary KC: FRACTIONS | Conversion to decimals | expressing fractions as decimals
  \item Topic: Services such as installation, maintenance, repairing, cleaning, laundry, hotel, retail, e-commerce, streaming services, digital services etc.
  \item Grade: Primary 5
\end{itemize}

\textbf{Solution}

Step 1: Calculate the total amount charged by multiplying the rate per square meter by the area cleaned.

Total amount = Rate per square meter × Area cleaned
Total amount = $\frac{8}{3} \times \frac{9}{4}$

Step 2: Multiply the two improper fractions.
$\frac{8}{3} \times \frac{9}{4} = \frac{8 \times 9}{3 \times 4} = \frac{72}{12}$

Step 3: Simplify the fraction.
$\frac{72}{12} = 6$

Step 4: Convert the result to decimal form.
Since $6 = \frac{6}{1}$, the decimal form is $6.0$ or simply $6$.

Therefore, the total amount charged is \textdollar$6.00$.

\section*{Question 23}
\textbf{Metadata}

\begin{itemize}
  \item Question ID: P5-FrMulMixN\_P2-FrAdd2nd\_sonnet4\_Services\_01
  \item Primary KC: FRACTIONS | Multiplication | multiplying a mixed number and a whole number
  \item Secondary KC: FRACTIONS | Addition | adding fractions
  \item Topic: Services such as installation, maintenance, repairing, cleaning, laundry, hotel, retail, e-commerce, streaming services, digital services etc.
  \item Grade: Primary 5
\end{itemize}

\textbf{Solution}

Step 1: Calculate the earnings for Monday.
Monday hours worked = $2\frac{3}{4}$ hours
Earnings for Monday = $2\frac{3}{4} \times 12$
$= \frac{11}{4} \times 12$
$= \frac{11 \times 12}{4}$
$= \frac{132}{4}$
$= 33$
So the earnings for Monday = \textdollar33

Step 2: Calculate the earnings for Tuesday.
Tuesday hours worked = $1\frac{1}{2}$ hours
Earnings for Tuesday = $1\frac{1}{2} \times 12$
$= \frac{3}{2} \times 12$
$= \frac{3 \times 12}{2}$
$= \frac{36}{2}$
$= 18$
So the earnings for Tuesday = \textdollar18

Step 3: Add the earnings from both days.
Total earnings = \textdollar33 + \textdollar18 = \textdollar51

Therefore, the cleaning service company earned \textdollar51 in total for both days.

\section*{Question 24}
\textbf{Metadata}

\begin{itemize}
  \item Question ID: P5-DcMul3dK\_P4-DcCmp3d\_sonnet4\_Services\_01
  \item Primary KC: DECIMALS | Multiplication | multiplying decimals (up to 3 decimal places) by 10, 100, 1000 and their multiples
  \item Secondary KC: DECIMALS | Comparison and ordering | comparing and ordering decimals up to 3 decimal places
  \item Topic: Services such as installation, maintenance, repairing, cleaning, laundry, hotel, retail, e-commerce, streaming services, digital services etc.
  \item Grade: Primary 5
\end{itemize}

\textbf{Solution}

To find the cost for 1000 minutes of each service, I need to multiply each rate by 1000.

For basic software troubleshooting:
\textdollar0.085 $\times$ 1000 = \textdollar85.000 = \textdollar85

For hardware diagnostics:
\textdollar0.127 $\times$ 1000 = \textdollar127.000 = \textdollar127

For virus removal:
\textdollar0.094 $\times$ 1000 = \textdollar94.000 = \textdollar94

Now I need to arrange these costs in ascending order (from smallest to largest):
\textdollar85 < \textdollar94 < \textdollar127

Therefore, the costs in ascending order are:
1. Basic software troubleshooting: \textdollar85
2. Virus removal: \textdollar94
3. Hardware diagnostics: \textdollar127

\section*{Question 25}
\textbf{Metadata}

\begin{itemize}
  \item Question ID: P5-DcMul3dK\_P4-DcAdd2nd\_sonnet4\_Services\_01
  \item Primary KC: DECIMALS | Multiplication | multiplying decimals (up to 3 decimal places) by 10, 100, 1000 and their multiples
  \item Secondary KC: DECIMALS | Addition | adding decimals
  \item Topic: Services such as installation, maintenance, repairing, cleaning, laundry, hotel, retail, e-commerce, streaming services, digital services etc.
  \item Grade: Primary 5
\end{itemize}

\textbf{Solution}

Step 1: Calculate earnings from Monday's screen repairs.
Earnings from Monday = \textdollar12.75 \times 20 = \textdollar255.00

Step 2: Calculate earnings from Tuesday's screen repairs.
Earnings from Tuesday = \textdollar12.75 \times 100 = \textdollar1275.00

Step 3: Calculate total earnings from screen repairs.
Total from screen repairs = \textdollar255.00 + \textdollar1275.00 = \textdollar1530.00

Step 4: Add earnings from other repair services.
Total earnings = \textdollar1530.00 + \textdollar45.60 = \textdollar1575.60

Therefore, Sarah earned \textdollar1575.60 in total from Monday to Tuesday.

\section*{Question 26}
\textbf{Metadata}

\begin{itemize}
  \item Question ID: P5-DcDiv3dK\_P4-DcCmp3d\_sonnet4\_Services\_01
  \item Primary KC: DECIMALS | Division | dividing decimals (up to 3 decimal places) by 10, 100, 1000 and their multiples
  \item Secondary KC: DECIMALS | Comparison and ordering | comparing and ordering decimals up to 3 decimal places
  \item Topic: Services such as installation, maintenance, repairing, cleaning, laundry, hotel, retail, e-commerce, streaming services, digital services etc.
  \item Grade: Primary 5
\end{itemize}

\textbf{Solution}

First, I need to calculate the new promotional rates by dividing each original rate.

For premium quality:
$4.80 \div 100 = 0.048$
The new premium rate is \textdollar0.048 per hour.

For standard quality:
$2.40 \div 10 = 0.240$
The new standard rate is \textdollar0.240 per hour.

For basic quality:
$1.20 \div 1000 = 0.001$
The new basic rate is \textdollar0.001 per hour.

Next, I need to compare and order these three decimal values: $0.048$, $0.240$, and $0.001$.

Comparing the decimals:
- $0.001$ has $1$ thousandth
- $0.048$ has $48$ thousandths
- $0.240$ has $240$ thousandths

Since $1 < 48 < 240$, we have $0.001 < 0.048 < 0.240$.

Therefore, the three new hourly rates in ascending order are:
Basic quality: \textdollar0.001 per hour
Premium quality: \textdollar0.048 per hour
Standard quality: \textdollar0.240 per hour

\section*{Question 27}
\textbf{Metadata}

\begin{itemize}
  \item Question ID: P5-DcDiv3dK\_P4-DcSub2nd\_sonnet4\_Services\_01
  \item Primary KC: DECIMALS | Division | dividing decimals (up to 3 decimal places) by 10, 100, 1000 and their multiples
  \item Secondary KC: DECIMALS | Subtraction | subtracting decimals
  \item Topic: Services such as installation, maintenance, repairing, cleaning, laundry, hotel, retail, e-commerce, streaming services, digital services etc.
  \item Grade: Primary 5
\end{itemize}

\textbf{Solution}

(a) To find the number of screen replacements:

Number of screen replacements = Total amount from screen replacements $\div$ Cost per screen replacement

Number of screen replacements = \textdollar6840.00 $\div$ \textdollar45.60

To divide by \textdollar45.60, I can divide by 456 first, then multiply by 10:

$6840 \div 456 = 15$

$15 \times 10 = 150$

Therefore, Sarah completed $150$ screen replacements.

(b) To find the total amount collected from battery replacements:

Amount from battery replacements = Total amount collected $-$ Amount from screen replacements

Amount from battery replacements = \textdollar9120.00 $-$ \textdollar6840.00 = \textdollar2280.00

Therefore, the total amount collected from battery replacements was \textdollar2280.00.

\section*{Question 28}
\textbf{Metadata}

\begin{itemize}
  \item Question ID: P5-PcRepWh\_P1-WNAdd2nd\_sonnet4\_Services\_01
  \item Primary KC: PERCENTAGE | Representation and concept | expressing a part of a whole as a percentage
  \item Secondary KC: WHOLE NUMBERS | Addition | adding whole numbers
  \item Topic: Services such as installation, maintenance, repairing, cleaning, laundry, hotel, retail, e-commerce, streaming services, digital services etc.
  \item Grade: Primary 5
\end{itemize}

\textbf{Solution}

Step 1: Find the number of laptops repaired in February.
Laptops repaired in January = $45$
Increase in February = $20\%$ of $45$
$20\%$ of $45 = \frac{20}{100} \times 45 = \frac{20 \times 45}{100} = \frac{900}{100} = 9$
Laptops repaired in February = $45 + 9 = 54$

Step 2: Find the number of desktop computers repaired in February.
Desktop computers repaired in January = $30$
Increase in February = $25\%$ of $30$
$25\%$ of $30 = \frac{25}{100} \times 30 = \frac{25 \times 30}{100} = \frac{750}{100} = 7.5$
Since we cannot repair half a computer, we round to $8$ desktop computers.
Desktop computers repaired in February = $30 + 8 = 38$

Step 3: Find the total number of computers repaired in both months.
Total in January = $45 + 30 = 75$
Total in February = $54 + 38 = 92$
Total for both months = $75 + 92 = 167$

Therefore:
- Sarah repaired $54$ laptops in February
- Sarah repaired $38$ desktop computers in February  
- Sarah repaired a total of $167$ computers in both months

\section*{Question 29}
\textbf{Metadata}

\begin{itemize}
  \item Question ID: P5-PcRepWh\_P1-WNSub2nd\_sonnet4\_Services\_01
  \item Primary KC: PERCENTAGE | Representation and concept | expressing a part of a whole as a percentage
  \item Secondary KC: WHOLE NUMBERS | Subtraction | subtracting whole numbers
  \item Topic: Services such as installation, maintenance, repairing, cleaning, laundry, hotel, retail, e-commerce, streaming services, digital services etc.
  \item Grade: Primary 5
\end{itemize}

\textbf{Solution}

Step 1: Find the number of jobs completed this month.
Jobs completed this month = Jobs completed last month - $18$
Jobs completed this month = $450 - 18 = 432$ jobs

Step 2: Find the monthly target.
Since $432$ jobs represent $72\%$ of the monthly target, we need to find the whole.
Let the monthly target be $x$ jobs.
$72\%$ of $x = 432$
$\frac{72}{100} \times x = 432$
$x = 432 \div \frac{72}{100}$
$x = 432 \times \frac{100}{72}$
$x = \frac{43200}{72}$
$x = 600$

Therefore, the company aimed to complete $600$ jobs this month.

\section*{Question 30}
\textbf{Metadata}

\begin{itemize}
  \item Question ID: P5-RtFndT\_P2-DcCnvD2N\_sonnet4\_Services\_01
  \item Primary KC: RATE | Finding total amount | finding total amount, given rate and number of units
  \item Secondary KC: DECIMALS | Conversion to smaller units | converting a measurement from a larger unit in decimal form to a smaller unit
  \item Topic: Services such as installation, maintenance, repairing, cleaning, laundry, hotel, retail, e-commerce, streaming services, digital services etc.
  \item Grade: Primary 5
\end{itemize}

\textbf{Solution}

Step 1: Find the total amount in dollars.
Rate = \textdollar12.50 per hour
Number of hours = 2.4 hours
Total amount = Rate $\times$ Number of hours
Total amount = \textdollar12.50 $\times$ 2.4 = \textdollar30.00

Step 2: Convert the amount from dollars to cents.
Since 1 dollar = 100 cents
\textdollar30.00 = 30.00 $\times$ 100 = 3000 cents

Therefore, Mrs. Tan paid 3000 cents for the cleaning service.

\section*{Question 31}
\textbf{Metadata}

\begin{itemize}
  \item Question ID: P6-FrDivPN\_P2-FrCmp\_sonnet4\_Services\_01
  \item Primary KC: FRACTIONS | Division | dividing a proper fraction by a whole number
  \item Secondary KC: FRACTIONS | Comparison and ordering | comparing and ordering fractions
  \item Topic: Services such as installation, maintenance, repairing, cleaning, laundry, hotel, retail, e-commerce, streaming services, digital services etc.
  \item Grade: Primary 6
\end{itemize}

\textbf{Solution}

Step 1: Find the allocated time for each cleaning task.
The total cleaning time is $\frac{3}{4}$ hour, divided equally among 3 tasks.
Time per task = $\frac{3}{4} \div 3 = \frac{3}{4} \times \frac{1}{3} = \frac{3}{12} = \frac{1}{4}$ hour

Step 2: Compare the allocated time per task with the actual bathroom cleaning time.
Allocated time per task = $\frac{1}{4}$ hour
Actual bathroom cleaning time = $\frac{1}{6}$ hour

Step 3: To compare these fractions, find a common denominator.
The LCD of 4 and 6 is 12.
$\frac{1}{4} = \frac{3}{12}$
$\frac{1}{6} = \frac{2}{12}$

Step 4: Compare the fractions.
Since $\frac{3}{12} > \frac{2}{12}$, the allocated time per task ($\frac{1}{4}$ hour) is longer than the actual bathroom cleaning time ($\frac{1}{6}$ hour).

Step 5: Find the difference.
Difference = $\frac{1}{4} - \frac{1}{6} = \frac{3}{12} - \frac{2}{12} = \frac{1}{12}$ hour

Therefore, the allocated time per task takes longer than the actual bathroom cleaning time by $\frac{1}{12}$ hour.

\section*{Question 32}
\textbf{Metadata}

\begin{itemize}
  \item Question ID: P6-FrDivPN\_P2-FrSub2nd\_sonnet4\_Services\_01
  \item Primary KC: FRACTIONS | Division | dividing a proper fraction by a whole number
  \item Secondary KC: FRACTIONS | Subtraction | subtracting fractions
  \item Topic: Services such as installation, maintenance, repairing, cleaning, laundry, hotel, retail, e-commerce, streaming services, digital services etc.
  \item Grade: Primary 6
\end{itemize}

\textbf{Solution}

First, I need to find Sarah's actual working time by subtracting the break time from her total time at the house.

Total time at house = $\frac{5}{6}$ hour
Break time = $\frac{1}{4}$ hour

To subtract these fractions, I need a common denominator. The LCM of 6 and 4 is 12.

$\frac{5}{6} = \frac{5 \times 2}{6 \times 2} = \frac{10}{12}$

$\frac{1}{4} = \frac{1 \times 3}{4 \times 3} = \frac{3}{12}$

Actual working time = $\frac{10}{12} - \frac{3}{12} = \frac{7}{12}$ hour

Now I need to calculate how much to charge for $\frac{7}{12}$ hour at \textdollar60 per hour.

This means I need to divide \textdollar60 by 12, then multiply by 7:

$\frac{7}{12} \times \textdollar60 = \frac{7 \times \textdollar60}{12} = \frac{\textdollar420}{12} = \textdollar35$

Therefore, Sarah should charge the Tan family \textdollar35 for her actual cleaning time.

\section*{Question 33}
\textbf{Metadata}

\begin{itemize}
  \item Question ID: P6-FrDivPN\_P5-FrMul2nd\_sonnet4\_Services\_01
  \item Primary KC: FRACTIONS | Division | dividing a proper fraction by a whole number
  \item Secondary KC: FRACTIONS | Multiplication | fraction multiplication
  \item Topic: Services such as installation, maintenance, repairing, cleaning, laundry, hotel, retail, e-commerce, streaming services, digital services etc.
  \item Grade: Primary 6
\end{itemize}

\textbf{Solution}

Step 1: Find how much detergent Sarah uses in each small washing machine.

Sarah divides $\frac{3}{4}$ cup of detergent equally among 6 machines.

Amount per machine = $\frac{3}{4} \div 6$

To divide a fraction by a whole number, we multiply the denominator by the whole number:
$\frac{3}{4} \div 6 = \frac{3}{4 \times 6} = \frac{3}{24} = \frac{1}{8}$ cup

Step 2: Find how much detergent Sarah needs for the stain removal treatment.

She needs $\frac{2}{3}$ times the amount used in each small machine.

Amount needed = $\frac{2}{3} \times \frac{1}{8}$

To multiply fractions, we multiply the numerators together and multiply the denominators together:
$\frac{2}{3} \times \frac{1}{8} = \frac{2 \times 1}{3 \times 8} = \frac{2}{24} = \frac{1}{12}$ cup

Therefore, Sarah needs $\frac{1}{12}$ cup of detergent for the stain removal treatment.

\section*{Question 34}
\textbf{Metadata}

\begin{itemize}
  \item Question ID: P6-FrDivPP\_P2-FrAdd2nd\_sonnet4\_Services\_01
  \item Primary KC: FRACTIONS | Division | dividing a whole number/proper fraction by a proper fraction
  \item Secondary KC: FRACTIONS | Addition | adding fractions
  \item Topic: Services such as installation, maintenance, repairing, cleaning, laundry, hotel, retail, e-commerce, streaming services, digital services etc.
  \item Grade: Primary 6
\end{itemize}

\textbf{Solution}

Step 1: Find how many laptop repairs Sarah can complete.
Number of laptop repairs = $3 \div \frac{3}{4}$
$= 3 \times \frac{4}{3}$
$= \frac{12}{3}$
$= 4$ laptop repairs

Step 2: Find the time spent on laptop repairs.
Time spent on laptop repairs = $4 \times \frac{3}{4}$
$= \frac{12}{4}$
$= 3$ hours

Step 3: Find the remaining time.
Remaining time = $3 - 3 = 0$ hours

Wait, let me recalculate. Since Sarah can do exactly $4$ complete repairs in $3$ hours (as $4 \times \frac{3}{4} = 3$), she uses all her time for laptop repairs.

Therefore, Sarah has $0$ hours remaining, so she can complete $0$ data backup services.

\section*{Question 35}
\textbf{Metadata}

\begin{itemize}
  \item Question ID: P6-FrDivPP\_P5-FrMul2nd\_sonnet4\_Services\_01
  \item Primary KC: FRACTIONS | Division | dividing a whole number/proper fraction by a proper fraction
  \item Secondary KC: FRACTIONS | Multiplication | fraction multiplication
  \item Topic: Services such as installation, maintenance, repairing, cleaning, laundry, hotel, retail, e-commerce, streaming services, digital services etc.
  \item Grade: Primary 6
\end{itemize}

\textbf{Solution}

**Step 1: Find how many rooms Sarah can clean in 1 hour.**

Sarah cleans $\frac{3}{4}$ of a room in $\frac{2}{5}$ hour.

To find how many rooms she can clean in 1 hour, I need to divide $\frac{3}{4}$ by $\frac{2}{5}$:

$\frac{3}{4} \div \frac{2}{5} = \frac{3}{4} \times \frac{5}{2} = \frac{3 \times 5}{4 \times 2} = \frac{15}{8} = 1\frac{7}{8}$

So Sarah can clean $1\frac{7}{8}$ rooms in 1 hour.

**Step 2: Find how many rooms Sarah can clean in $2\frac{1}{3}$ hours.**

Number of rooms = Rate per hour × Time

Number of rooms = $1\frac{7}{8} \times 2\frac{1}{3}$

Converting to improper fractions:
$1\frac{7}{8} = \frac{15}{8}$ and $2\frac{1}{3} = \frac{7}{3}$

Number of rooms = $\frac{15}{8} \times \frac{7}{3} = \frac{15 \times 7}{8 \times 3} = \frac{105}{24} = \frac{35}{8} = 4\frac{3}{8}$

**Step 3: Calculate Sarah's earnings.**

Sarah charges \textdollar45 per room and cleans $4\frac{3}{8}$ rooms.

Earnings = $4\frac{3}{8} \times 45$

$= \frac{35}{8} \times 45 = \frac{35 \times 45}{8} = \frac{1575}{8} = 196\frac{7}{8}$

**Answer:**
- Sarah can clean $1\frac{7}{8}$ rooms in 1 hour.
- Sarah will earn \textdollar$196\frac{7}{8}$ or \textdollar196.88.

\section*{Question 36}
\textbf{Metadata}

\begin{itemize}
  \item Question ID: P6-PcFndWN\_P1-WNMul2nd\_sonnet4\_Services\_01
  \item Primary KC: PERCENTAGE | Finding the whole | finding the whole given a part and the percentage
  \item Secondary KC: WHOLE NUMBERS | Multiplication | multiplying whole numbers
  \item Topic: Services such as installation, maintenance, repairing, cleaning, laundry, hotel, retail, e-commerce, streaming services, digital services etc.
  \item Grade: Primary 6
\end{itemize}

\textbf{Solution}

Step 1: Find the total number of service requests received.
If 48 repair jobs represent 60\% of all service requests, then:
60\% of total requests = 48
$\frac{60}{100} \times \text{total requests} = 48$
$\text{total requests} = 48 \div \frac{60}{100} = 48 \times \frac{100}{60} = 80$

Step 2: Calculate the total earnings from completed repair jobs.
Number of completed repair jobs = 48
Charge per repair job = \textdollar25
Total earnings = $48 \times 25 = \textdollar1200$

Therefore, TechFix Solutions earned \textdollar1200 from all the repair jobs they completed last month.

\section*{Question 37}
\textbf{Metadata}

\begin{itemize}
  \item Question ID: P6-PcFndWN\_P1-WNDiv2nd\_sonnet4\_Services\_01
  \item Primary KC: PERCENTAGE | Finding the whole | finding the whole given a part and the percentage
  \item Secondary KC: WHOLE NUMBERS | Division | dividing whole numbers
  \item Topic: Services such as installation, maintenance, repairing, cleaning, laundry, hotel, retail, e-commerce, streaming services, digital services etc.
  \item Grade: Primary 6
\end{itemize}

\textbf{Solution}

Step 1: Find the total daily revenue of the cleaning service.
Given that \textdollar240 represents 60\% of the total daily revenue.
Let the total daily revenue be $x$.
$60\% \times x = \textdollar240$
$\frac{60}{100} \times x = 240$
$0.6x = 240$
$x = 240 \div 0.6 = 400$
Therefore, the total daily revenue is \textdollar400.

Step 2: Calculate the bonus for each department.
The total daily revenue of \textdollar400 is to be distributed equally among 8 departments.
Bonus per department $= 400 \div 8 = 50$

Therefore, each department will receive a bonus of \textdollar50.

\section*{Question 38}
\textbf{Metadata}

\begin{itemize}
  \item Question ID: P6-PcFndChg\_P1-WNAdd2nd\_sonnet4\_Services\_01
  \item Primary KC: PERCENTAGE | Finding change | finding percentage increase/decrease
  \item Secondary KC: WHOLE NUMBERS | Addition | adding whole numbers
  \item Topic: Services such as installation, maintenance, repairing, cleaning, laundry, hotel, retail, e-commerce, streaming services, digital services etc.
  \item Grade: Primary 6
\end{itemize}

\textbf{Solution}

Step 1: Find the total increase in price.
Total increase = \textdollar12 + \textdollar8 = \textdollar20

Step 2: Find the final price after both increases.
Final price = Original price + Total increase
Final price = \textdollar45 + \textdollar20 = \textdollar65

Step 3: Calculate the percentage increase.
Percentage increase = $\frac{\text{Total increase}}{\text{Original price}} \times 100\%$
Percentage increase = $\frac{\textdollar20}{\textdollar45} \times 100\%$
Percentage increase = $\frac{20}{45} \times 100\%$
Percentage increase = $\frac{4}{9} \times 100\%$
Percentage increase = $44\frac{4}{9}\%$ or $44.44\%$ (rounded to 2 decimal places)

Therefore, the percentage increase in the laptop maintenance price is $44\frac{4}{9}\%$.

\section*{Question 39}
\textbf{Metadata}

\begin{itemize}
  \item Question ID: P6-RoFndDvqWN\_P1-WNSub2nd\_sonnet4\_Services\_01
  \item Primary KC: RATIO | Finding divided quantities | dividing a given quantity in a given ratio
  \item Secondary KC: WHOLE NUMBERS | Subtraction | subtracting whole numbers
  \item Topic: Services such as installation, maintenance, repairing, cleaning, laundry, hotel, retail, e-commerce, streaming services, digital services etc.
  \item Grade: Primary 6
\end{itemize}

\textbf{Solution}

First, I need to find how much money is left after deducting the cost of cleaning supplies.

Money left after deduction = \textdollar840 - \textdollar105 = \textdollar735

Next, I will divide \textdollar735 among Alex, Ben, and Charlie in the ratio $3:5:7$.

The ratio is $3:5:7$, so the total number of parts = $3 + 5 + 7 = 15$ parts

Value of 1 part = \textdollar735 ÷ 15 = \textdollar49

Therefore:
- Alex receives: $3 \times \textdollar49 = \textdollar147$
- Ben receives: $5 \times \textdollar49 = \textdollar245$  
- Charlie receives: $7 \times \textdollar49 = \textdollar343$

Check: \textdollar147 + \textdollar245 + \textdollar343 = \textdollar735 ✓

Alex will receive \textdollar147, Ben will receive \textdollar245, and Charlie will receive \textdollar343.

\section*{Question 40}
\textbf{Metadata}

\begin{itemize}
  \item Question ID: P6-RoFndRoWN\_P1-WNMul2nd\_sonnet4\_Services\_01
  \item Primary KC: RATIO | Finding ratio | finding the ratio of two or three given whole numbers
  \item Secondary KC: WHOLE NUMBERS | Multiplication | multiplying whole numbers
  \item Topic: Services such as installation, maintenance, repairing, cleaning, laundry, hotel, retail, e-commerce, streaming services, digital services etc.
  \item Grade: Primary 6
\end{itemize}

\textbf{Solution}

Step 1: Find the ratio of offices cleaned by Team A to Team B to Team C.
Team A cleaned 24 offices
Team B cleaned 36 offices  
Team C cleaned 48 offices

The ratio is $24 : 36 : 48$

To simplify this ratio, I need to find the highest common factor (HCF) of 24, 36, and 48.
Factors of 24: 1, 2, 3, 4, 6, 8, 12, 24
Factors of 36: 1, 2, 3, 4, 6, 9, 12, 18, 36
Factors of 48: 1, 2, 3, 4, 6, 8, 12, 16, 24, 48

HCF of 24, 36, and 48 = 12

$24 \div 12 = 2$
$36 \div 12 = 3$  
$48 \div 12 = 4$

Therefore, the ratio of offices cleaned by Team A to Team B to Team C is $2 : 3 : 4$.

Step 2: Calculate the total amount earned by all three teams.
Total number of offices cleaned = $24 + 36 + 48 = 108$ offices
Each office cleaning job pays \textdollar15
Total amount earned = $108 \times 15 = \textdollar1620$

Answer: The ratio is $2 : 3 : 4$ and the total amount earned is \textdollar1620.

\section*{Question 41}
\textbf{Metadata}

\begin{itemize}
  \item Question ID: P6-RoFndRoWN\_P6-RoSmpWN\_sonnet4\_Services\_01
  \item Primary KC: RATIO | Finding ratio | finding the ratio of two or three given whole numbers
  \item Secondary KC: RATIO | Simplifying | expressing a ratio in its simplest form
  \item Topic: Services such as installation, maintenance, repairing, cleaning, laundry, hotel, retail, e-commerce, streaming services, digital services etc.
  \item Grade: Primary 6
\end{itemize}

\textbf{Solution}

To find the ratio of time spent on office cleaning to carpet cleaning to window cleaning, I need to write the hours in order and then simplify.

Time spent on:
- Office cleaning: 18 hours
- Carpet cleaning: 24 hours  
- Window cleaning: 30 hours

The ratio is $18 : 24 : 30$

To express this ratio in its simplest form, I need to find the highest common factor (HCF) of 18, 24, and 30.

Factors of 18: 1, 2, 3, 6, 9, 18
Factors of 24: 1, 2, 3, 4, 6, 8, 12, 24
Factors of 30: 1, 2, 3, 5, 6, 10, 15, 30

The common factors are: 1, 2, 3, 6
The highest common factor is 6.

Dividing each term by 6:
$18 \div 6 = 3$
$24 \div 6 = 4$
$30 \div 6 = 5$

Therefore, the ratio of time spent on office cleaning to carpet cleaning to window cleaning in its simplest form is $3 : 4 : 5$.

\section*{Question 42}
\textbf{Metadata}

\begin{itemize}
  \item Question ID: P6-RoFndTmWN\_P1-WNSub2nd\_sonnet4\_Services\_01
  \item Primary KC: RATIO | Finding a missing term | finding the missing term in a pair of equivalent ratios
  \item Secondary KC: WHOLE NUMBERS | Subtraction | subtracting whole numbers
  \item Topic: Services such as installation, maintenance, repairing, cleaning, laundry, hotel, retail, e-commerce, streaming services, digital services etc.
  \item Grade: Primary 6
\end{itemize}

\textbf{Solution}

First, I need to find how many offices Team B cleaned last month using the ratio $3:5$.

Since Team A cleaned $24$ offices and the ratio is $3:5$, I can set up the proportion:
$\frac{3}{5} = \frac{24}{x}$

Cross-multiplying: $3x = 5 \times 24 = 120$

Therefore: $x = 120 \div 3 = 40$

So Team B cleaned $40$ offices last month.

The total number of offices cleaned last month was: $24 + 40 = 64$ offices

This month, the total was reduced by $16$ offices:
Total offices this month $= 64 - 16 = 48$ offices

Since the teams maintained the same ratio $3:5$, the total ratio parts $= 3 + 5 = 8$ parts

Each part represents: $48 \div 8 = 6$ offices

Team B's share this month $= 5 \times 6 = 30$ offices

Therefore, Team B cleaned $30$ offices this month.

\section*{Question 43}
\textbf{Metadata}

\begin{itemize}
  \item Question ID: O1-RoRepFr\_P2-FrAdd2nd\_sonnet4\_Services\_01
  \item Primary KC: RATIO | Representation and concept | ratios involving fractions
  \item Secondary KC: FRACTIONS | Addition | adding fractions
  \item Topic: Services such as installation, maintenance, repairing, cleaning, laundry, hotel, retail, e-commerce, streaming services, digital services etc.
  \item Grade: Secondary O-level 1
\end{itemize}

\textbf{Solution}

(a) To express the ratio $\frac{2}{5} : \frac{1}{3}$ in its simplest form:

First, find the LCM of the denominators 5 and 3, which is 15.

Convert to equivalent fractions: $\frac{2}{5} = \frac{6}{15}$ and $\frac{1}{3} = \frac{5}{15}$

So the ratio becomes $\frac{6}{15} : \frac{5}{15} = 6 : 5$

Therefore, the ratio of contracts completed by Team A to Team B is $6 : 5$.

(b) To find the fraction of contracts completed by Team C:

Team C completes the remaining contracts after Teams A and B.

Fraction completed by Team C = $1 - \frac{2}{5} - \frac{1}{3}$

To subtract these fractions, find the LCM of 5 and 3, which is 15.

$1 - \frac{2}{5} - \frac{1}{3} = \frac{15}{15} - \frac{6}{15} - \frac{5}{15} = \frac{15 - 6 - 5}{15} = \frac{4}{15}$

Therefore, Team C completes $\frac{4}{15}$ of the total weekly cleaning contracts.

(c) To find the ratio of contracts completed by Team A to Team C:

Team A completes $\frac{2}{5}$ and Team C completes $\frac{4}{15}$.

The ratio is $\frac{2}{5} : \frac{4}{15}$

Convert to equivalent fractions with common denominator 15:
$\frac{2}{5} = \frac{6}{15}$

So the ratio becomes $\frac{6}{15} : \frac{4}{15} = 6 : 4 = 3 : 2$

Therefore, the ratio of contracts completed by Team A to Team C is $3 : 2$.

\section*{Question 44}
\textbf{Metadata}

\begin{itemize}
  \item Question ID: O1-RoRepFr\_P5-FrMul2nd\_sonnet4\_Services\_01
  \item Primary KC: RATIO | Representation and concept | ratios involving fractions
  \item Secondary KC: FRACTIONS | Multiplication | fraction multiplication
  \item Topic: Services such as installation, maintenance, repairing, cleaning, laundry, hotel, retail, e-commerce, streaming services, digital services etc.
  \item Grade: Secondary O-level 1
\end{itemize}

\textbf{Solution}

Given ratio: iPhone : Samsung : Huawei = $5\frac{1}{2} : 4\frac{3}{4} : 3\frac{1}{4}$

First, convert mixed numbers to improper fractions:
$5\frac{1}{2} = \frac{11}{2}$, $4\frac{3}{4} = \frac{19}{4}$, $3\frac{1}{4} = \frac{13}{4}$

So the ratio is $\frac{11}{2} : \frac{19}{4} : \frac{13}{4}$

To eliminate fractions, multiply all terms by 4:
$\frac{11}{2} \times 4 : \frac{19}{4} \times 4 : \frac{13}{4} \times 4 = 22 : 19 : 13$

Given that Samsung phone repair costs \textdollar95:
If Samsung corresponds to 19 parts, then 1 part = $\frac{95}{19} = 5$

(a) iPhone repair cost = $22 \times 5 = \textdollar110$

(b) Huawei repair cost = $13 \times 5 = \textdollar65$

(c) Total cost for repairs:
- 2 iPhones: $2 \times \textdollar110 = \textdollar220$
- 3 Samsung phones: $3 \times \textdollar95 = \textdollar285$
- 1 Huawei phone: $1 \times \textdollar65 = \textdollar65$

Total cost = $\textdollar220 + \textdollar285 + \textdollar65 = \textdollar570$

\section*{Question 45}
\textbf{Metadata}

\begin{itemize}
  \item Question ID: O1-RoRepFr\_O1-RoSmpFr\_sonnet4\_Services\_01
  \item Primary KC: RATIO | Representation and concept | ratios involving fractions
  \item Secondary KC: RATIO | Simplifying | converting a ratio involving fractions to its simplest form
  \item Topic: Services such as installation, maintenance, repairing, cleaning, laundry, hotel, retail, e-commerce, streaming services, digital services etc.
  \item Grade: Secondary O-level 1
\end{itemize}

\textbf{Solution}

First, I need to identify the time taken for each package:

Package A: $\frac{3}{4}$ hours

Package B: $1\frac{1}{2}$ hours

To find the ratio, I need to convert the mixed number to an improper fraction:
$1\frac{1}{2} = \frac{3}{2}$

The ratio of Package A to Package B is:
$\frac{3}{4} : \frac{3}{2}$

To simplify this ratio, I can multiply both parts by the LCM of the denominators (4 and 2), which is 4:

$\frac{3}{4} \times 4 : \frac{3}{2} \times 4$

$= 3 : 6$

To express this in simplest form, I find the GCD of 3 and 6, which is 3:

$3 \div 3 : 6 \div 3 = 1 : 2$

Therefore, the ratio of time taken for Package A to Package B is $1 : 2$.

\section*{Question 46}
\textbf{Metadata}

\begin{itemize}
  \item Question ID: O1-RoRepDc\_P4-DcAdd2nd\_sonnet4\_Services\_01
  \item Primary KC: RATIO | Representation and concept | ratios involving decimals
  \item Secondary KC: DECIMALS | Addition | adding decimals
  \item Topic: Services such as installation, maintenance, repairing, cleaning, laundry, hotel, retail, e-commerce, streaming services, digital services etc.
  \item Grade: Secondary O-level 1
\end{itemize}

\textbf{Solution}

**Step 1: Verify the January ratio**

Given ratio = $2.5 : 3.8$
Residential revenue = \textdollar$4875$
Commercial revenue = \textdollar$7410$

Actual ratio = $4875 : 7410 = 2.5 : 3.8$ ✓

**Step 2: Calculate January total revenue**

January total revenue = \textdollar$4875 + \textdollar$7410 = \textdollar$12285$

**Step 3: Calculate February revenues**

February residential revenue = \textdollar$4875 + \textdollar$487.50 = \textdollar$5362.50$
February commercial revenue = \textdollar$7410 + \textdollar$741.00 = \textdollar$8151.00$

**Step 4: Verify February ratio is maintained**

February ratio = $5362.50 : 8151.00 = 2.5 : 3.8$ ✓

**Step 5: Calculate February total revenue**

February total revenue = \textdollar$5362.50 + \textdollar$8151.00 = \textdollar$13513.50$

**Answer:** The company's total revenue in February was \textdollar$13513.50$.

\section*{Question 47}
\textbf{Metadata}

\begin{itemize}
  \item Question ID: O1-RoRepDc\_O1-RoSmpDc\_sonnet4\_Services\_01
  \item Primary KC: RATIO | Representation and concept | ratios involving decimals
  \item Secondary KC: RATIO | Simplifying | converting a ratio involving decimals to its simplest form
  \item Topic: Services such as installation, maintenance, repairing, cleaning, laundry, hotel, retail, e-commerce, streaming services, digital services etc.
  \item Grade: Secondary O-level 1
\end{itemize}

\textbf{Solution}

(a) The ratio of time taken for Package A to Package B is:
$2.5 : 1.8$

(b) To express $2.5 : 1.8$ in its simplest form:
First, convert the decimals to fractions:
$2.5 = \frac{25}{10} = \frac{5}{2}$ and $1.8 = \frac{18}{10} = \frac{9}{5}$

So the ratio becomes $\frac{5}{2} : \frac{9}{5}$

To eliminate fractions, multiply both terms by the LCM of denominators (2 and 5), which is 10:
$\frac{5}{2} \times 10 : \frac{9}{5} \times 10 = 25 : 18$

Since GCD(25, 18) = 1, the simplest form is $25 : 18$

(c) The ratio of cost of Package A to Package B is:
\textdollar$37.50 :$ \textdollar$25.20 = 37.5 : 25.2$

(d) To express $37.5 : 25.2$ in its simplest form:
First, eliminate decimals by multiplying both terms by 10:
$37.5 \times 10 : 25.2 \times 10 = 375 : 252$

Now find GCD(375, 252):
$375 = 3 \times 5^3$ and $252 = 2^2 \times 3^2 \times 7$
GCD(375, 252) = 3

Dividing both terms by 3:
$375 \div 3 : 252 \div 3 = 125 : 84$

Since GCD(125, 84) = 1, the simplest form is $125 : 84$

\section*{Question 48}
\textbf{Metadata}

\begin{itemize}
  \item Question ID: O1-PcRep2q\_O1-PcCnv2Fr\_sonnet4\_Services\_01
  \item Primary KC: PERCENTAGE | Representation and concept | comparing two quantities by percentage
  \item Secondary KC: PERCENTAGE | Conversion to fraction | expressing percentage as a fraction
  \item Topic: Services such as installation, maintenance, repairing, cleaning, laundry, hotel, retail, e-commerce, streaming services, digital services etc.
  \item Grade: Secondary O-level 1
\end{itemize}

\textbf{Solution}

(a) To find what percentage of repairs were battery replacements:
Battery replacements = $\frac{3}{8}$ of total repairs
To convert fraction to percentage: $\frac{3}{8} = \frac{3 \times 12.5}{8 \times 12.5} = \frac{37.5}{100} = 37.5\%$

Therefore, $37.5\%$ of the total repairs were battery replacements.

(b) To find how many more screen replacements than battery replacements:
Screen replacements = $25\%$ of $240 = \frac{25}{100} \times 240 = 60$ repairs
Battery replacements = $\frac{3}{8}$ of $240 = \frac{3}{8} \times 240 = 90$ repairs

Difference = $90 - 60 = 30$

Therefore, there were $30$ more battery replacements than screen replacements.

(c) To find what fraction were other types of fixes:
Screen replacements: $25\% = \frac{1}{4}$
Battery replacements: $\frac{3}{8}$

Other repairs = $1 - \frac{1}{4} - \frac{3}{8}$

Converting to common denominator: $1 - \frac{2}{8} - \frac{3}{8} = \frac{8}{8} - \frac{2}{8} - \frac{3}{8} = \frac{3}{8}$

Therefore, $\frac{3}{8}$ of the total repairs were other types of fixes.

\section*{Question 49}
\textbf{Metadata}

\begin{itemize}
  \item Question ID: O1-PcFndRslt\_P1-WNAdd2nd\_sonnet4\_Services\_01
  \item Primary KC: PERCENTAGE | Finding result after change | increasing/decreasing a quantity by a given percentage
  \item Secondary KC: WHOLE NUMBERS | Addition | adding whole numbers
  \item Topic: Services such as installation, maintenance, repairing, cleaning, laundry, hotel, retail, e-commerce, streaming services, digital services etc.
  \item Grade: Secondary O-level 1
\end{itemize}

\textbf{Solution}

Step 1: Find the increased cleaning rate during holiday season.
Basic cleaning rate = \textdollar80
Increase = 25\% of \textdollar80
Increase = $\frac{25}{100} \times 80 = \textdollar20$

Holiday season cleaning rate = \textdollar80 + \textdollar20 = \textdollar100

Step 2: Add the fixed service fee to find the total amount.
Service fee = \textdollar15
Total amount = Holiday season cleaning rate + Service fee
Total amount = \textdollar100 + \textdollar15 = \textdollar115

Therefore, a customer needs to pay \textdollar115 for a house cleaning service during the holiday season.

\section*{Question 50}
\textbf{Metadata}

\begin{itemize}
  \item Question ID: O1-PcFndRslt\_P1-WNMul2nd\_sonnet4\_Services\_01
  \item Primary KC: PERCENTAGE | Finding result after change | increasing/decreasing a quantity by a given percentage
  \item Secondary KC: WHOLE NUMBERS | Multiplication | multiplying whole numbers
  \item Topic: Services such as installation, maintenance, repairing, cleaning, laundry, hotel, retail, e-commerce, streaming services, digital services etc.
  \item Grade: Secondary O-level 1
\end{itemize}

\textbf{Solution}

Step 1: Find the increased hourly rate during holiday season.

Regular rate = \textdollar25 per hour
Increase = 20\% of \textdollar25
Increase = $\frac{20}{100} \times 25 = \textdollar5$

Holiday season rate = \textdollar25 + \textdollar5 = \textdollar30 per hour

Step 2: Calculate the total cost for 6 hours.

Total cost = Holiday season rate $\times$ Number of hours
Total cost = \textdollar30 $\times$ 6 = \textdollar180

Therefore, the customer will pay \textdollar180 in total.

\section*{Question 51}
\textbf{Metadata}

\begin{itemize}
  \item Question ID: O1-PcRepRvs\_O1-PcCnv2Fr\_sonnet4\_Services\_01
  \item Primary KC: PERCENTAGE | Representation and concept | reverse percentages
  \item Secondary KC: PERCENTAGE | Conversion to fraction | expressing percentage as a fraction
  \item Topic: Services such as installation, maintenance, repairing, cleaning, laundry, hotel, retail, e-commerce, streaming services, digital services etc.
  \item Grade: Secondary O-level 1
\end{itemize}

\textbf{Solution}

Let the original base rate be $x$ dollars.

During peak season, the rate is increased by $25\%$, so the peak season rate is:
$x + 25\% \text{ of } x = x + 0.25x = 1.25x$

We know that the customer paid \textdollar30 during peak season, so:
$1.25x = 30$

Solving for $x$:
$x = \frac{30}{1.25} = \frac{30}{\frac{5}{4}} = 30 \times \frac{4}{5} = \frac{120}{5} = 24$

Therefore, the original base rate was \textdollar24.

To express the original base rate as a fraction of the peak season price:
$\frac{\text{Original base rate}}{\text{Peak season price}} = \frac{24}{30}$

Simplifying this fraction:
$\frac{24}{30} = \frac{24 \div 6}{30 \div 6} = \frac{4}{5}$

Therefore, the original base rate was \textdollar24, which is $\frac{4}{5}$ of the peak season price.

\section*{Question 52}
\textbf{Metadata}

\begin{itemize}
  \item Question ID: O1-AgRepExSq\_O1-AgEvlEx\_sonnet4\_Services\_01
  \item Primary KC: ALGEBRA | Representation and concept | translation of simple real-world situations into quadratic algebraic expressions
  \item Secondary KC: ALGEBRA | Evaluation | evaluation of algebraic expressions and formulae
  \item Topic: Services such as installation, maintenance, repairing, cleaning, laundry, hotel, retail, e-commerce, streaming services, digital services etc.
  \item Grade: Secondary O-level 1
\end{itemize}

\textbf{Solution}

(a) For each phone, the cost is given by $C = 25 + 0.5t^2$

For the first phone (2 hours): $C_1 = 25 + 0.5(2)^2 = 25 + 0.5(4) = 25 + 2$

For the second phone (3 hours): $C_2 = 25 + 0.5(3)^2 = 25 + 0.5(9) = 25 + 4.5$

For the third phone (4 hours): $C_3 = 25 + 0.5(4)^2 = 25 + 0.5(16) = 25 + 8$

Total revenue = $C_1 + C_2 + C_3$

Total revenue = $(25 + 2) + (25 + 4.5) + (25 + 8)$

Total revenue = $25 + 0.5(2^2) + 25 + 0.5(3^2) + 25 + 0.5(4^2)$

Total revenue = $75 + 0.5(2^2 + 3^2 + 4^2)$

(b) Evaluating the expression:

$2^2 + 3^2 + 4^2 = 4 + 9 + 16 = 29$

Total revenue = $75 + 0.5(29) = 75 + 14.5 = 89.5$

Therefore, the total revenue earned by the shop on Monday is \textdollar89.50.

\section*{Question 53}
\textbf{Metadata}

\begin{itemize}
  \item Question ID: O1-AgRepnth\_O1-AgEvlEx\_sonnet4\_Services\_01
  \item Primary KC: ALGEBRA | Representation and concept | recognising and representing patterns/relationships by finding an algebraic expression for the nth term
  \item Secondary KC: ALGEBRA | Evaluation | evaluation of algebraic expressions and formulae
  \item Topic: Services such as installation, maintenance, repairing, cleaning, laundry, hotel, retail, e-commerce, streaming services, digital services etc.
  \item Grade: Secondary O-level 1
\end{itemize}

\textbf{Solution}

(a) The total cost consists of a base fee plus the hourly charge multiplied by the number of hours.

Base fee = \textdollar15
Hourly charge = \textdollar8 per hour
Number of hours = $n$

Therefore, the algebraic expression is:
$C = 15 + 8n$

(b) To find the total cost for 4.5 hours of work, substitute $n = 4.5$ into the expression:

$C = 15 + 8n$
$C = 15 + 8(4.5)$
$C = 15 + 36$
$C = 51$

The total cost for 4.5 hours of repair work is \textdollar51.

(c) To find the number of hours when the total cost is \textdollar71, substitute $C = 71$ into the expression and solve for $n$:

$C = 15 + 8n$
$71 = 15 + 8n$
$71 - 15 = 8n$
$56 = 8n$
$n = \frac{56}{8}$
$n = 7$

The customer who was charged \textdollar71 received 7 hours of repair work.

\section*{Question 54}
\textbf{Metadata}

\begin{itemize}
  \item Question ID: O2-RoRepDP\_P1-WNDiv2nd\_sonnet4\_Services\_01
  \item Primary KC: RATIO | Representation and concept | direct proportion
  \item Secondary KC: WHOLE NUMBERS | Division | dividing whole numbers
  \item Topic: Services such as installation, maintenance, repairing, cleaning, laundry, hotel, retail, e-commerce, streaming services, digital services etc.
  \item Grade: Secondary O-level 2
\end{itemize}

\textbf{Solution}

Step 1: Find the cost per room using the direct proportion.
Since cost is directly proportional to number of rooms:
$\frac{\text{Cost}}{\text{Number of rooms}} = k$ (constant)

From last month's data:
$k = \frac{\textdollar240}{8} = \textdollar30$ per room

Step 2: Verify the office building cost using direct proportion.
For 36 rooms: Cost $= 36 \times \textdollar30 = \textdollar1080$ ✓

Step 3: Find the number of cleaning staff using division.
Total payment = \textdollar1080
Payment per staff = \textdollar45

Number of staff $= \frac{\textdollar1080}{\textdollar45} = 24$

Therefore, 24 cleaning staff worked on the office building project.

\section*{Question 55}
\textbf{Metadata}

\begin{itemize}
  \item Question ID: O2-AgSlvIneq\_O2-AgRepIneq\_sonnet4\_Services\_01
  \item Primary KC: ALGEBRA | Solving | solving simple linear inequalities with one variable
  \item Secondary KC: ALGEBRA | Representation and concept | translation of simple real-world situations to simple linear inequalities with one variable
  \item Topic: Services such as installation, maintenance, repairing, cleaning, laundry, hotel, retail, e-commerce, streaming services, digital services etc.
  \item Grade: Secondary O-level 2
\end{itemize}

\textbf{Solution}

Let $h$ represent the number of hours Sarah works on the repair.

The total cost consists of:
- Fixed service fee: \textdollar15
- Labor cost: \textdollar8 per hour for $h$ hours = \textdollar$8h$

Total cost = \textdollar$15 + 8h$

Since the total cost should not exceed \textdollar75, we can write the inequality:
$15 + 8h \leq 75$

Solving the inequality:
$15 + 8h \leq 75$
$8h \leq 75 - 15$
$8h \leq 60$
$h \leq \frac{60}{8}$
$h \leq 7.5$

Therefore, Sarah can work a maximum of 7.5 hours on this repair job to keep the total cost at or below \textdollar75.

\section*{Question 56}
\textbf{Metadata}

\begin{itemize}
  \item Question ID: O2-AgSlvSq1v\_O1-AgRepEq\_sonnet4\_Services\_01
  \item Primary KC: ALGEBRA | Solving | solving quadratic equations in one variable
  \item Secondary KC: ALGEBRA | Representation and concept | translation of simple real-world situations to equations
  \item Topic: Services such as installation, maintenance, repairing, cleaning, laundry, hotel, retail, e-commerce, streaming services, digital services etc.
  \item Grade: Secondary O-level 2
\end{itemize}

\textbf{Solution}

Let $f$ be the flat fee and $h$ be the hourly rate.

From the given information, I can set up two equations:
For the 3-hour job: $f + 3h = 75$ ... (1)
For the 5-hour job: $f + 5h = 115$ ... (2)

To find the hourly rate, I'll subtract equation (1) from equation (2):
$(f + 5h) - (f + 3h) = 115 - 75$
$2h = 40$
$h = 20$

Substituting $h = 20$ into equation (1):
$f + 3(20) = 75$
$f + 60 = 75$
$f = 15$

So the company charges a flat fee of \textdollar15 plus \textdollar20 per hour.

To earn exactly \textdollar200, let $t$ be the number of hours needed:
$15 + 20t = 200$
$20t = 185$
$t = 9.25$

Therefore, the company needs to work for 9.25 hours (or 9 hours and 15 minutes) to earn exactly \textdollar200 from a single job.

\section*{Question 57}
\textbf{Metadata}

\begin{itemize}
  \item Question ID: O2-SPFndmdn\_O3-SPFndrng\_sonnet4\_Services\_01
  \item Primary KC: STATISTICS AND PROBABILITY | Finding median | Finding median for a set of data
  \item Secondary KC: STATISTICS AND PROBABILITY | Finding range | finding range as measures of spread for a set of data 
  \item Topic: Services such as installation, maintenance, repairing, cleaning, laundry, hotel, retail, e-commerce, streaming services, digital services etc.
  \item Grade: Secondary O-level 2
\end{itemize}

\textbf{Solution}

To find the median and range, I need to organize and analyze the given data.

Given data: 12, 8, 15, 9, 11, 13, 7, 10, 14, 6, 12, 9, 16, 8

Step 1: Find the median
First, I'll arrange the data in ascending order:
6, 7, 8, 8, 9, 9, 10, 11, 12, 12, 13, 14, 15, 16

There are 14 values (even number), so the median is the average of the 7th and 8th values.
The 7th value is 10 and the 8th value is 11.

Median $= \frac{10 + 11}{2} = \frac{21}{2} = 10.5$

Step 2: Find the range
Range = Highest value - Lowest value
Highest value = 16
Lowest value = 6

Range $= 16 - 6 = 10$

Therefore, the median number of devices repaired per day is 10.5, and the range of daily repairs is 10 devices. This means that on average, the shop repairs about 10-11 devices per day, with the daily workload varying by up to 10 devices between the busiest and quietest days.

\section*{Question 58}
\textbf{Metadata}

\begin{itemize}
  \item Question ID: O2-SPFndmean\_O3-BPRepSN\_sonnet4\_Services\_01
  \item Primary KC: STATISTICS AND PROBABILITY | Finding mean deviation | calculation of the mean for a set of data
  \item Secondary KC: BASE AND POWER | Representation and concept  | use of standard form Ax10^n , where n is an integer, and 1<= A<= 10
  \item Topic: Services such as installation, maintenance, repairing, cleaning, laundry, hotel, retail, e-commerce, streaming services, digital services etc.
  \item Grade: Secondary O-level 2
\end{itemize}

\textbf{Solution}

Step 1: Convert all values from standard form to decimal form.
$2.4 \times 10^2 = 240$ GB
$1.8 \times 10^2 = 180$ GB
$3.2 \times 10^2 = 320$ GB
$2.1 \times 10^2 = 210$ GB
$2.8 \times 10^2 = 280$ GB
$1.9 \times 10^2 = 190$ GB
$2.6 \times 10^2 = 260$ GB

Step 2: Calculate the mean daily usage.
Sum of all values = $240 + 180 + 320 + 210 + 280 + 190 + 260 = 1680$ GB
Mean = $\frac{1680}{7} = 240$ GB

Step 3: Calculate the absolute deviation of each value from the mean.
$|240 - 240| = 0$
$|180 - 240| = 60$
$|320 - 240| = 80$
$|210 - 240| = 30$
$|280 - 240| = 40$
$|190 - 240| = 50$
$|260 - 240| = 20$

Step 4: Calculate the mean deviation.
Sum of absolute deviations = $0 + 60 + 80 + 30 + 40 + 50 + 20 = 280$
Mean deviation = $\frac{280}{7} = 40$ GB

Therefore, the mean daily data usage is 240 GB and the mean deviation is 40 GB.

\section*{Question 59}
\textbf{Metadata}

\begin{itemize}
  \item Question ID: O3-BPOpr\_O3-BPRepNegI\_sonnet4\_Services\_01
  \item Primary KC: BASE AND POWER | Operations | laws of indices
  \item Secondary KC: BASE AND POWER | Representation and concept  | negative indices
  \item Topic: Services such as installation, maintenance, repairing, cleaning, laundry, hotel, retail, e-commerce, streaming services, digital services etc.
  \item Grade: Secondary O-level 3/4
\end{itemize}

\textbf{Solution}

(a) Finding the hourly rate for Sarah's virus removal service:
The hourly rate is \textdollar50 $\times$ $2^d$ where $d = 3$
Hourly rate = \textdollar50 $\times$ $2^3$
Using laws of indices: $2^3 = 2 \times 2 \times 2 = 8$
Hourly rate = \textdollar50 $\times$ 8 = \textdollar400 per hour

(b) Finding the hourly rate for Mike's software update service:
The hourly rate is \textdollar50 $\times$ $2^d$ where $d = -2$
Hourly rate = \textdollar50 $\times$ $2^{-2}$
Using negative indices: $2^{-2} = \frac{1}{2^2} = \frac{1}{4} = 0.25$
Hourly rate = \textdollar50 $\times$ 0.25 = \textdollar12.50 per hour

(c) Calculating the total amount both customers will pay:
Sarah's total cost = \textdollar400 per hour $\times$ 2 hours = \textdollar800
Mike's total cost = \textdollar12.50 per hour $\times$ 1.5 hours = \textdollar18.75
Total amount = \textdollar800 + \textdollar18.75 = \textdollar818.75

Therefore, Sarah will pay \textdollar800, Mike will pay \textdollar18.75, and the total amount for both services is \textdollar818.75.

\section*{Question 60}
\textbf{Metadata}

\begin{itemize}
  \item Question ID: O3-MXMulSM\_O3-MXAdd\_sonnet4\_Services\_01
  \item Primary KC: MATRICES | Multiplication | product of a scalar quantity and a matrix
  \item Secondary KC: MATRICES | Addition | addition of matrices
  \item Topic: Services such as installation, maintenance, repairing, cleaning, laundry, hotel, retail, e-commerce, streaming services, digital services etc.
  \item Grade: Secondary O-level 3/4
\end{itemize}

\textbf{Solution}

To solve this problem, I need to:
1. Calculate the doubled labor costs using scalar multiplication
2. Add the material costs using matrix addition

Step 1: Calculate doubled labor costs
The doubled labor costs = $2A$
$2A = 2 \begin{pmatrix} 120 & 180 & 150 \\ 200 & 240 & 220 \\ 80 & 100 & 90 \end{pmatrix}$

$2A = \begin{pmatrix} 2(120) & 2(180) & 2(150) \\ 2(200) & 2(240) & 2(220) \\ 2(80) & 2(100) & 2(90) \end{pmatrix}$

$2A = \begin{pmatrix} 240 & 360 & 300 \\ 400 & 480 & 440 \\ 160 & 200 & 180 \end{pmatrix}$

Step 2: Add material costs
Total cost matrix = $2A + B$

$2A + B = \begin{pmatrix} 240 & 360 & 300 \\ 400 & 480 & 440 \\ 160 & 200 & 180 \end{pmatrix} + \begin{pmatrix} 50 & 60 & 45 \\ 80 & 70 & 85 \\ 30 & 40 & 35 \end{pmatrix}$

$2A + B = \begin{pmatrix} 240+50 & 360+60 & 300+45 \\ 400+80 & 480+70 & 440+85 \\ 160+30 & 200+40 & 180+35 \end{pmatrix}$

$2A + B = \begin{pmatrix} 290 & 420 & 345 \\ 480 & 550 & 525 \\ 190 & 240 & 215 \end{pmatrix}$

Therefore, the total cost matrix for February is $\begin{pmatrix} 290 & 420 & 345 \\ 480 & 550 & 525 \\ 190 & 240 & 215 \end{pmatrix}$ dollars.

\section*{Question 61}
\textbf{Metadata}

\begin{itemize}
  \item Question ID: O3-MXSub\_O3-MXAdd\_sonnet4\_Services\_01
  \item Primary KC: MATRICES | Subtraction | subtraction of matrices
  \item Secondary KC: MATRICES | Addition | addition of matrices
  \item Topic: Services such as installation, maintenance, repairing, cleaning, laundry, hotel, retail, e-commerce, streaming services, digital services etc.
  \item Grade: Secondary O-level 3/4
\end{itemize}

\textbf{Solution}

(a) To find the total number of service calls for January and February combined, we need to add matrices $A$ and $B$:

$A + B = \begin{pmatrix} 45 & 32 & 28 \\ 38 & 41 & 35 \\ 22 & 19 & 25 \end{pmatrix} + \begin{pmatrix} 52 & 29 & 31 \\ 35 & 44 & 38 \\ 18 & 23 & 27 \end{pmatrix}$

$A + B = \begin{pmatrix} 45+52 & 32+29 & 28+31 \\ 38+35 & 41+44 & 35+38 \\ 22+18 & 19+23 & 25+27 \end{pmatrix}$

$A + B = \begin{pmatrix} 97 & 61 & 59 \\ 73 & 85 & 73 \\ 40 & 42 & 52 \end{pmatrix}$

(b) To find the difference between the total calls from part (a) and the March service calls, we subtract matrix $C$ from the result of part (a):

$(A + B) - C = \begin{pmatrix} 97 & 61 & 59 \\ 73 & 85 & 73 \\ 40 & 42 & 52 \end{pmatrix} - \begin{pmatrix} 48 & 35 & 33 \\ 42 & 39 & 32 \\ 25 & 21 & 24 \end{pmatrix}$

$(A + B) - C = \begin{pmatrix} 97-48 & 61-35 & 59-33 \\ 73-42 & 85-39 & 73-32 \\ 40-25 & 42-21 & 52-24 \end{pmatrix}$

$(A + B) - C = \begin{pmatrix} 49 & 26 & 26 \\ 31 & 46 & 41 \\ 15 & 21 & 28 \end{pmatrix}$

(c) The result in part (b) represents how many more service calls were made in the first two months (January and February) combined compared to March alone, broken down by service type and location. For example, there were 49 more hardware repair calls at the Downtown location in January and February combined than in March.

\section*{Question 62}
\textbf{Metadata}

\begin{itemize}
  \item Question ID: O3-SPAddProb\_O2-SPRepPrSE\_sonnet4\_Services\_01
  \item Primary KC: STATISTICS AND PROBABILITY | Addition | addition of probabilities
  \item Secondary KC: STATISTICS AND PROBABILITY | Representation and concept | probability of single events
  \item Topic: Services such as installation, maintenance, repairing, cleaning, laundry, hotel, retail, e-commerce, streaming services, digital services etc.
  \item Grade: Secondary O-level 3/4
\end{itemize}

\textbf{Solution}

Let $H$ be the event that the hardware repair is completed successfully on the first attempt, and $S$ be the event that the software repair is completed successfully on the first attempt.

Given information:
$P(H) = \frac{4}{5}$
$P(S) = \frac{7}{10}$

We need to find $P(H \cup S)$, the probability that at least one repair is completed successfully.

Since the hardware and software repairs are independent events, we can use:
$P(H \cup S) = P(H) + P(S) - P(H \cap S)$

For independent events: $P(H \cap S) = P(H) \times P(S)$

$P(H \cap S) = \frac{4}{5} \times \frac{7}{10} = \frac{28}{50} = \frac{14}{25}$

Therefore:
$P(H \cup S) = \frac{4}{5} + \frac{7}{10} - \frac{14}{25}$

To add these fractions, we need a common denominator. The LCM of 5, 10, and 25 is 50.

$P(H \cup S) = \frac{40}{50} + \frac{35}{50} - \frac{28}{50} = \frac{40 + 35 - 28}{50} = \frac{47}{50}$

Therefore, the probability that at least one of the two repair jobs will be completed successfully on the first attempt is $\frac{47}{50}$ or $0.94$.

\section*{Question 63}
\textbf{Metadata}

\begin{itemize}
  \item Question ID: O3-SPAddProb\_O3-SPFndPrCE\_sonnet4\_Services\_01
  \item Primary KC: STATISTICS AND PROBABILITY | Addition | addition of probabilities
  \item Secondary KC: STATISTICS AND PROBABILITY | Finding probability | probability of simple combined events
  \item Topic: Services such as installation, maintenance, repairing, cleaning, laundry, hotel, retail, e-commerce, streaming services, digital services etc.
  \item Grade: Secondary O-level 3/4
\end{itemize}

\textbf{Solution}

Let $H$ be the event that a laptop has a hardware issue and $S$ be the event that a laptop has a software issue.

Given information:
- $P(H) = 0.6$
- $P(S) = 0.25$ 
- $P(H \cap S) = 0.15$

We need to find $P(H \cup S)$, which is the probability that a laptop has either a hardware issue or a software issue (or both).

Using the addition rule for probabilities:
$P(H \cup S) = P(H) + P(S) - P(H \cap S)$

Substituting the given values:
$P(H \cup S) = 0.6 + 0.25 - 0.15$
$P(H \cup S) = 0.85 - 0.15$
$P(H \cup S) = 0.7$

Therefore, the probability that a randomly selected laptop brought for repair has either a hardware issue or a software issue (or both) is $0.7$ or $70\%$.

\section*{Question 64}
\textbf{Metadata}

\begin{itemize}
  \item Question ID: O3-SPMulProb\_O3-SPFndPrCE\_sonnet4\_Services\_01
  \item Primary KC: STATISTICS AND PROBABILITY | Multiplication | multiplication of probabilities
  \item Secondary KC: STATISTICS AND PROBABILITY | Finding probability | probability of simple combined events
  \item Topic: Services such as installation, maintenance, repairing, cleaning, laundry, hotel, retail, e-commerce, streaming services, digital services etc.
  \item Grade: Secondary O-level 3/4
\end{itemize}

\textbf{Solution}

Let $S$ = event that the problem is software-related
Let $H$ = event that the problem is hardware-related
Let $T$ = event that repair is completed within 2 hours

Given information:
$P(S) = 0.7$
$P(H) = 0.3$
$P(T|S) = 0.9$ (probability of repair within 2 hours given software problem)
$P(T|H) = 0.6$ (probability of repair within 2 hours given hardware problem)

**Part 1:** Find $P(S \cap T)$ (software problem AND repaired within 2 hours)

Using multiplication of probabilities:
$P(S \cap T) = P(S) \times P(T|S) = 0.7 \times 0.9 = 0.63$

**Part 2:** Find $P(T)$ (repaired within 2 hours regardless of problem type)

This is a combined event where repair within 2 hours can happen in two ways:
- Software problem AND repaired within 2 hours: $P(S \cap T) = 0.7 \times 0.9 = 0.63$
- Hardware problem AND repaired within 2 hours: $P(H \cap T) = 0.3 \times 0.6 = 0.18$

Since these events are mutually exclusive:
$P(T) = P(S \cap T) + P(H \cap T) = 0.63 + 0.18 = 0.81$

**Answer:**
The probability that the laptop has a software problem AND can be repaired within 2 hours is $0.63$.
The probability that the laptop can be repaired within 2 hours is $0.81$.

\end{document}
