\documentclass{article}
\usepackage[utf8]{inputenc}
\usepackage{amsmath}
\usepackage{amsfonts}
\usepackage{amssymb}
\usepackage{graphicx}
\usepackage{hyperref}
\title{'TD Solutions services v3 CLAUDE '}
\author{Tien Dung Doan}
\begin{document}
\maketitle
\section*{Question 1}
\textbf{Metadata}

\begin{itemize}
  \item Question ID: P3-WNAdd4d\_P1-WNCmp\_sonnet4\_Services\_03
  \item Primary KC: WHOLE NUMBERS | Addition | adding whole numbers up to 4 digits
  \item Secondary KC: WHOLE NUMBERS | Comparison and ordering | comparing and ordering whole numbers
  \item Topic: Services such as installation, maintenance, repairing, cleaning, laundry, hotel, retail, e-commerce, streaming services, digital services etc.
  \item Grade: Primary 3
\end{itemize}

\textbf{Solution}

(a) To find the total earnings for the four months, I need to add all the monthly earnings together.

January: \textdollar1245
February: \textdollar987
March: \textdollar1568
April: \textdollar1123

Total earnings = \textdollar1245 + \textdollar987 + \textdollar1568 + \textdollar1123

Let me add step by step:
\textdollar1245 + \textdollar987 = \textdollar2232
\textdollar2232 + \textdollar1568 = \textdollar3800
\textdollar3800 + \textdollar1123 = \textdollar4923

The total earnings for the four months was \textdollar4923.

(b) To find the highest and lowest earnings, I need to compare the four amounts:
January: \textdollar1245
February: \textdollar987
March: \textdollar1568
April: \textdollar1123

Comparing the amounts:
\textdollar1568 is the largest amount
\textdollar987 is the smallest amount

Therefore, March had the highest earnings (\textdollar1568) and February had the lowest earnings (\textdollar987).

\section*{Question 2}
\textbf{Metadata}

\begin{itemize}
  \item Question ID: P3-WNSub4d\_P1-WNAdd2nd\_sonnet4\_Services\_01
  \item Primary KC: WHOLE NUMBERS | Subtraction | subtracting whole numbers up to 4 digits
  \item Secondary KC: WHOLE NUMBERS | Addition | adding whole numbers
  \item Topic: Services such as installation, maintenance, repairing, cleaning, laundry, hotel, retail, e-commerce, streaming services, digital services etc.
  \item Grade: Primary 3
\end{itemize}

\textbf{Solution}

Step 1: Find the total number of laptops repaired.
Laptops repaired in first week $= 789$
Laptops repaired in second week $= 456$
Total laptops repaired $= 789 + 456 = 1245$

Step 2: Find how many laptops are still waiting to be repaired.
Total laptops received $= 3456$
Total laptops repaired $= 1245$
Laptops still waiting to be repaired $= 3456 - 1245 = 2211$

Therefore, $2211$ laptops are still waiting to be repaired.

\section*{Question 3}
\textbf{Metadata}

\begin{itemize}
  \item Question ID: P3-WNDivRmd3d\_P1-WNSub2nd\_sonnet4\_Services\_01
  \item Primary KC: WHOLE NUMBERS | Division | dividing whole numbers up to 3 digits by 1 digit with remainder 
  \item Secondary KC: WHOLE NUMBERS | Subtraction | subtracting whole numbers
  \item Topic: Services such as installation, maintenance, repairing, cleaning, laundry, hotel, retail, e-commerce, streaming services, digital services etc.
  \item Grade: Primary 3
\end{itemize}

\textbf{Solution}

Step 1: Find how many loads of washing are needed.
Divide the number of dirty towels by the number of towels per load:
$347 \div 8 = 43$ remainder $3$

Since there are 3 towels remaining, they need 1 more load to wash these towels.
Number of loads needed $= 43 + 1 = 44$ loads

Step 2: Find the total number of towels after washing.
After washing, all 347 dirty towels will become clean towels.
Total towels $= 347 + 128 = 475$ towels

Therefore, the company needs 44 loads of washing and will have 475 towels altogether after washing.

\section*{Question 4}
\textbf{Metadata}

\begin{itemize}
  \item Question ID: P3-WNMul3d1d\_P1-WNCmp\_sonnet4\_Services\_01
  \item Primary KC: WHOLE NUMBERS | Multiplication | multiplying whole numbers up to 3 digits by 1 digit
  \item Secondary KC: WHOLE NUMBERS | Comparison and ordering | comparing and ordering whole numbers
  \item Topic: Services such as installation, maintenance, repairing, cleaning, laundry, hotel, retail, e-commerce, streaming services, digital services etc.
  \item Grade: Primary 3
\end{itemize}

\textbf{Solution}

I need to find the total cost for each package by multiplying the cost per room by the number of rooms, then compare the costs.

Given information:
- Package A: \textdollar8 per room
- Package B: \textdollar6 per room  
- Package C: \textdollar9 per room
- Number of rooms: 7

Step 1: Calculate the total cost for Package A.
Total cost for Package A = \textdollar8 $\times$ 7 = \textdollar56

Step 2: Calculate the total cost for Package B.
Total cost for Package B = \textdollar6 $\times$ 7 = \textdollar42

Step 3: Calculate the total cost for Package C.
Total cost for Package C = \textdollar9 $\times$ 7 = \textdollar63

Step 4: Compare the total costs to find which is most and least expensive.
Package A: \textdollar56
Package B: \textdollar42
Package C: \textdollar63

Comparing: \textdollar42 < \textdollar56 < \textdollar63

Therefore:
- Package A costs \textdollar56
- Package B costs \textdollar42
- Package C costs \textdollar63
- Package C costs the most (\textdollar63)
- Package B costs the least (\textdollar42)

\section*{Question 5}
\textbf{Metadata}

\begin{itemize}
  \item Question ID: P3-WNDiv3d1d\_P1-WNCmp\_sonnet4\_Services\_01
  \item Primary KC: WHOLE NUMBERS | Division | dividing whole numbers up to 3 digits by 1 digit
  \item Secondary KC: WHOLE NUMBERS | Comparison and ordering | comparing and ordering whole numbers
  \item Topic: Services such as installation, maintenance, repairing, cleaning, laundry, hotel, retail, e-commerce, streaming services, digital services etc.
  \item Grade: Primary 3
\end{itemize}

\textbf{Solution}

First, I need to find how many apartments each cleaner cleaned in each team by dividing the total apartments by the number of cleaners.

Team A: $184 \div 8$
$184 \div 8 = 23$
Each cleaner in Team A cleaned 23 apartments.

Team B: $156 \div 6$
$156 \div 6 = 26$
Each cleaner in Team B cleaned 26 apartments.

Team C: $275 \div 5$
$275 \div 5 = 55$
Each cleaner in Team C cleaned 55 apartments.

Now I need to compare and order these numbers: 23, 26, and 55.
$55 > 26 > 23$

Therefore, arranging the teams from most to least apartments cleaned per cleaner:
1st: Team C (55 apartments per cleaner)
2nd: Team B (26 apartments per cleaner)
3rd: Team A (23 apartments per cleaner)

\section*{Question 6}
\textbf{Metadata}

\begin{itemize}
  \item Question ID: P3-FrSubRl12\_P2-FrCmp\_sonnet4\_Services\_01
  \item Primary KC: FRACTIONS | Subtraction | subtracting two related fractions within one whole with denominators of given fractions not exceeding 12
  \item Secondary KC: FRACTIONS | Comparison and ordering | comparing and ordering fractions
  \item Topic: Services such as installation, maintenance, repairing, cleaning, laundry, hotel, retail, e-commerce, streaming services, digital services etc.
  \item Grade: Primary 3
\end{itemize}

\textbf{Solution}

First, I need to compare the fractions to confirm that Sarah has more work than Tom.

Sarah's work: $\frac{7}{8}$ of the main floor
Tom's work: $\frac{3}{8}$ of the main floor

Since both fractions have the same denominator 8, I can compare the numerators:
$7 > 3$, so $\frac{7}{8} > \frac{3}{8}$

This confirms that Sarah has more work to do than Tom.

To find how much more work Sarah has compared to Tom, I need to subtract Tom's work from Sarah's work:

$\frac{7}{8} - \frac{3}{8}$

Since the denominators are the same, I subtract the numerators:
$\frac{7-3}{8} = \frac{4}{8}$

I can simplify this fraction by dividing both numerator and denominator by 4:
$\frac{4}{8} = \frac{4 \div 4}{8 \div 4} = \frac{1}{2}$

Therefore, Sarah needs to clean $\frac{1}{2}$ more of the main floor compared to Tom.

\section*{Question 7}
\textbf{Metadata}

\begin{itemize}
  \item Question ID: P4-WNMul4d1d\_P1-WNCmp\_sonnet4\_Services\_01
  \item Primary KC: WHOLE NUMBERS | Multiplication | multiplying whole numbers up to 4 digits by 1 digit or up to 3 digits by 2 digits
  \item Secondary KC: WHOLE NUMBERS | Comparison and ordering | comparing and ordering whole numbers
  \item Topic: Services such as installation, maintenance, repairing, cleaning, laundry, hotel, retail, e-commerce, streaming services, digital services etc.
  \item Grade: Primary 4
\end{itemize}

\textbf{Solution}

To solve this problem, I need to calculate the revenue for each month using multiplication, then compare the results.

**January Revenue:**
Number of laptops repaired: $248$
Charge per laptop: \textdollar$45$
January revenue = $248 \times 45$
$248 \times 45 = 248 \times (40 + 5) = 248 \times 40 + 248 \times 5$
$= 9920 + 1240 = 11160$
January revenue = \textdollar$11160$

**February Revenue:**
Number of desktops repaired: $186$
Charge per desktop: \textdollar$67$
February revenue = $186 \times 67$
$186 \times 67 = 186 \times (60 + 7) = 186 \times 60 + 186 \times 7$
$= 11160 + 1302 = 12462$
February revenue = \textdollar$12462$

**March Revenue:**
Number of installations: $312$
Charge per installation: \textdollar$28$
March revenue = $312 \times 28$
$312 \times 28 = 312 \times (30 - 2) = 312 \times 30 - 312 \times 2$
$= 9360 - 624 = 8736$
March revenue = \textdollar$8736$

**Comparing the revenues:**
January: \textdollar$11160$
February: \textdollar$12462$
March: \textdollar$8736$

Comparing: $12462 > 11160 > 8736$

Therefore, February had the highest revenue at \textdollar$12462$, and March had the lowest revenue at \textdollar$8736$.

\section*{Question 8}
\textbf{Metadata}

\begin{itemize}
  \item Question ID: P4-WNMul4d1d\_P4-WNRnd5d\_sonnet4\_Services\_01
  \item Primary KC: WHOLE NUMBERS | Multiplication | multiplying whole numbers up to 4 digits by 1 digit or up to 3 digits by 2 digits
  \item Secondary KC: WHOLE NUMBERS | Rounding | rounding whole numbers up to 100000 to the nearest 10, 100 or 1000 
  \item Topic: Services such as installation, maintenance, repairing, cleaning, laundry, hotel, retail, e-commerce, streaming services, digital services etc.
  \item Grade: Primary 4
\end{itemize}

\textbf{Solution}

(a) Total amount earned in January:
$247 \times 18$

Using the standard multiplication method:
$247 \times 18 = 247 \times (10 + 8)$
$= 247 \times 10 + 247 \times 8$
$= 2470 + 1976$
$= 4446$

The company earned \textdollar4446 in January.

(b) Total amount earned in February:
$189 \times 18$

Using the standard multiplication method:
$189 \times 18 = 189 \times (10 + 8)$
$= 189 \times 10 + 189 \times 8$
$= 1890 + 1512$
$= 3402$

The company earned \textdollar3402 in February.

(c) Total amount earned in both months:
$4446 + 3402 = 7848$

To round \textdollar7848 to the nearest hundred:
Look at the tens digit: $4$
Since $4 < 5$, we round down.
$7848$ rounded to the nearest hundred is $7800$.

The total amount earned by the company in both months, rounded to the nearest hundred dollars, is \textdollar7800.

\section*{Question 9}
\textbf{Metadata}

\begin{itemize}
  \item Question ID: P4-WNDiv4d1d\_P1-WNSub2nd\_sonnet4\_Services\_01
  \item Primary KC: WHOLE NUMBERS | Division | dividing whole numbers up to 4 digits by 1 digit
  \item Secondary KC: WHOLE NUMBERS | Subtraction | subtracting whole numbers
  \item Topic: Services such as installation, maintenance, repairing, cleaning, laundry, hotel, retail, e-commerce, streaming services, digital services etc.
  \item Grade: Primary 4
\end{itemize}

\textbf{Solution}

Step 1: Find how many laptops each team needs to repair.
Total laptops = 3456
Number of people per team = 8
Number of laptops per team = $3456 \div 8 = 432$

Step 2: Find how many more laptops the team still needs to repair.
Laptops already repaired by one team = 195
Laptops still needed = $432 - 195 = 237$

Therefore, each team needs to repair 432 laptops, and the team that has already repaired 195 laptops still needs to repair 237 more laptops.

\section*{Question 10}
\textbf{Metadata}

\begin{itemize}
  \item Question ID: P4-WNDiv4d1d\_P4-WNRnd5d\_sonnet4\_Services\_01
  \item Primary KC: WHOLE NUMBERS | Division | dividing whole numbers up to 4 digits by 1 digit
  \item Secondary KC: WHOLE NUMBERS | Rounding | rounding whole numbers up to 100000 to the nearest 10, 100 or 1000 
  \item Topic: Services such as installation, maintenance, repairing, cleaning, laundry, hotel, retail, e-commerce, streaming services, digital services etc.
  \item Grade: Primary 4
\end{itemize}

\textbf{Solution}

Step 1: Round the total hours worked to the nearest hundred.

The total hours worked was 2847 hours.

To round 2847 to the nearest hundred, I look at the tens digit, which is 4.

Since 4 < 5, I round down.

2847 rounded to the nearest hundred = 2800 hours

Step 2: Calculate the approximate monthly earnings.

The shop charges \textdollar8 per hour.

Approximate earnings = 2800 ÷ 1 × \textdollar8 = 2800 × \textdollar8

To find 2800 × 8:
$2800 \times 8 = 22400$

Therefore, the shop's approximate monthly earnings was \textdollar22400.

\section*{Question 11}
\textbf{Metadata}

\begin{itemize}
  \item Question ID: P4-FrAddU12\_P2-FrCmp\_sonnet4\_Services\_01
  \item Primary KC: FRACTIONS | Addition | adding unlike fractions with two different denominators not exceeding 12
  \item Secondary KC: FRACTIONS | Comparison and ordering | comparing and ordering fractions
  \item Topic: Services such as installation, maintenance, repairing, cleaning, laundry, hotel, retail, e-commerce, streaming services, digital services etc.
  \item Grade: Primary 4
\end{itemize}

\textbf{Solution}

To solve this problem, I need to add fractions with different denominators and then compare the totals.

**Step 1: Find the total time spent on Monday**
Monday: $\frac{3}{8} + \frac{1}{6}$

To add these fractions, I need a common denominator.
The LCM of 8 and 6 is 24.

$\frac{3}{8} = \frac{3 \times 3}{8 \times 3} = \frac{9}{24}$

$\frac{1}{6} = \frac{1 \times 4}{6 \times 4} = \frac{4}{24}$

$\frac{9}{24} + \frac{4}{24} = \frac{13}{24}$ hour

**Step 2: Find the total time spent on Tuesday**
Tuesday: $\frac{5}{12} + \frac{1}{4}$

The LCM of 12 and 4 is 12.

$\frac{5}{12} = \frac{5}{12}$ (already has denominator 12)

$\frac{1}{4} = \frac{1 \times 3}{4 \times 3} = \frac{3}{12}$

$\frac{5}{12} + \frac{3}{12} = \frac{8}{12} = \frac{2}{3}$ hour

**Step 3: Compare the totals**
Monday: $\frac{13}{24}$ hour
Tuesday: $\frac{2}{3}$ hour

To compare, I need a common denominator.
$\frac{2}{3} = \frac{2 \times 8}{3 \times 8} = \frac{16}{24}$

Comparing: $\frac{13}{24}$ vs $\frac{16}{24}$

Since $\frac{16}{24} > \frac{13}{24}$, Tuesday's total is greater.

**Answer:** Sarah spent $\frac{13}{24}$ hour on Monday and $\frac{2}{3}$ hour on Tuesday. She spent more time on these tasks on Tuesday.

\section*{Question 12}
\textbf{Metadata}

\begin{itemize}
  \item Question ID: P4-FrSubU12\_P3-FrSmp\_sonnet4\_Services\_01
  \item Primary KC: FRACTIONS | Subtraction | subtracting unlike fractions with two different denominators not exceeding 12
  \item Secondary KC: FRACTIONS | Simplifying | expressing a fraction in its simplest form
  \item Topic: Services such as installation, maintenance, repairing, cleaning, laundry, hotel, retail, e-commerce, streaming services, digital services etc.
  \item Grade: Primary 4
\end{itemize}

\textbf{Solution}

To find how much more of her working day Sarah spent repairing smartphones on Monday than on Tuesday, I need to subtract the fractions: $\frac{5}{8} - \frac{1}{3}$.

Step 1: Find a common denominator for $\frac{5}{8}$ and $\frac{1}{3}$.
The denominators are 8 and 3.
The least common multiple of 8 and 3 is 24.

Step 2: Convert both fractions to equivalent fractions with denominator 24.
$\frac{5}{8} = \frac{5 \times 3}{8 \times 3} = \frac{15}{24}$
$\frac{1}{3} = \frac{1 \times 8}{3 \times 8} = \frac{8}{24}$

Step 3: Subtract the fractions.
$\frac{15}{24} - \frac{8}{24} = \frac{15-8}{24} = \frac{7}{24}$

Step 4: Check if $\frac{7}{24}$ is in its simplest form.
The factors of 7 are: 1, 7
The factors of 24 are: 1, 2, 3, 4, 6, 8, 12, 24
The greatest common factor of 7 and 24 is 1.

Since the greatest common factor is 1, $\frac{7}{24}$ is already in its simplest form.

Therefore, Sarah spent $\frac{7}{24}$ more of her working day repairing smartphones on Monday than on Tuesday.

\section*{Question 13}
\textbf{Metadata}

\begin{itemize}
  \item Question ID: P4-DcAdd2d\_P4-DcCmp3d\_sonnet4\_Services\_01
  \item Primary KC: DECIMALS | Addition | adding decimals (up to 2 decimal places)
  \item Secondary KC: DECIMALS | Comparison and ordering | comparing and ordering decimals up to 3 decimal places
  \item Topic: Services such as installation, maintenance, repairing, cleaning, laundry, hotel, retail, e-commerce, streaming services, digital services etc.
  \item Grade: Primary 4
\end{itemize}

\textbf{Solution}

(a) To find Sarah's total earnings, I need to add the amounts from all three customers.

Total earnings = \textdollar12.75 + \textdollar18.50 + \textdollar9.25

Adding the amounts:
$12.75 + 18.50 + 9.25 = 40.50$

Therefore, Sarah's total earnings was \textdollar40.50.

(b) To arrange the customers from highest to lowest payment, I need to compare the decimal amounts:

• Customer A: \textdollar12.75 = \textdollar12.750
• Customer B: \textdollar18.50 = \textdollar18.500
• Customer C: \textdollar9.25 = \textdollar9.250
• Customer D: \textdollar12.750

Comparing the amounts:
$18.500 > 12.750 > 12.750 > 9.250$

Wait, Customer A paid \textdollar12.750 and Customer D also paid \textdollar12.750, so they paid the same amount.

The correct order from highest to lowest payment is:
1. Customer B: \textdollar18.50
2. Customer A and Customer D (tied): \textdollar12.75
3. Customer C: \textdollar9.25

\section*{Question 14}
\textbf{Metadata}

\begin{itemize}
  \item Question ID: P4-DcSub2d\_P4-DcCnv2Fr\_sonnet4\_Services\_01
  \item Primary KC: DECIMALS | Subtraction | subtracting decimals (up to 2 decimal places)
  \item Secondary KC: DECIMALS | Conversion from decimals to fraction | expressing decimals as fractions
  \item Topic: Services such as installation, maintenance, repairing, cleaning, laundry, hotel, retail, e-commerce, streaming services, digital services etc.
  \item Grade: Primary 4
\end{itemize}

\textbf{Solution}

(a) First, I need to find the total cost of the cleaning service.
Total cost $= \textdollar24.75 + \textdollar8.50 = \textdollar33.25$

Next, I need to find the change received.
Change $= \textdollar50.00 - \textdollar33.25 = \textdollar16.75$

Therefore, Sarah's mother received \textdollar16.75 in change.

(b) To express \textdollar16.75 as a fraction:
\textdollar16.75 $= 16\frac{75}{100}$

Simplifying the fraction part:
$\frac{75}{100} = \frac{75 \div 25}{100 \div 25} = \frac{3}{4}$

Therefore, \textdollar16.75 $= 16\frac{3}{4}$ or $\frac{67}{4}$ as an improper fraction.

The amount of change expressed as a fraction is $\frac{67}{4}$ dollars or $16\frac{3}{4}$ dollars.

\section*{Question 15}
\textbf{Metadata}

\begin{itemize}
  \item Question ID: P4-DcSub2d\_P4-DcRnd3d\_sonnet4\_Services\_01
  \item Primary KC: DECIMALS | Subtraction | subtracting decimals (up to 2 decimal places)
  \item Secondary KC: DECIMALS | Rounding | rounding decimals up to 3 decimal places to the nearest whole number, 1 decimal place and 2 decimal places 
  \item Topic: Services such as installation, maintenance, repairing, cleaning, laundry, hotel, retail, e-commerce, streaming services, digital services etc.
  \item Grade: Primary 4
\end{itemize}

\textbf{Solution}

Step 1: Find the total cost of services.
Total cost = \textdollar24.75 + \textdollar8.30 = \textdollar33.05

Step 2: Calculate the change by subtracting the total cost from the amount paid.
Change = \textdollar50.00 - \textdollar33.05 = \textdollar16.95

Step 3: Round the change to the nearest whole number.
\textdollar16.95 rounded to the nearest whole number = \textdollar17

Therefore, Sarah should give back \textdollar17 in change.

\section*{Question 16}
\textbf{Metadata}

\begin{itemize}
  \item Question ID: P4-DcMul2d1d\_P4-DcCnv2Fr\_sonnet4\_Services\_01
  \item Primary KC: DECIMALS | Multiplication | multiplying decimals (up to 2 decimal places) by a 1-digit whole number
  \item Secondary KC: DECIMALS | Conversion from decimals to fraction | expressing decimals as fractions
  \item Topic: Services such as installation, maintenance, repairing, cleaning, laundry, hotel, retail, e-commerce, streaming services, digital services etc.
  \item Grade: Primary 4
\end{itemize}

\textbf{Solution}

To find the total cost for 6 months, I need to multiply the monthly cost by 6.

Monthly cost = \textdollar2.85
Number of months = 6

Total cost = $2.85 \times 6$

Let me multiply:
$2.85 \times 6 = 17.10$

So the total cost is \textdollar17.10.

Now I need to express \textdollar17.10 as a fraction in its simplest form.

\textdollar17.10 = $\frac{1710}{100}$

To simplify this fraction, I need to find the highest common factor of 1710 and 100.

Factors of 1710: 1, 2, 3, 5, 6, 9, 10, 15, 18, 19, 30, 38, 45, 57, 90, 95, 114, 171, 190, 285, 342, 570, 855, 1710
Factors of 100: 1, 2, 4, 5, 10, 20, 25, 50, 100

The highest common factor is 10.

$\frac{1710}{100} = \frac{1710 \div 10}{100 \div 10} = \frac{171}{10}$

Therefore, Sarah's father will pay $\frac{171}{10}$ dollars for 6 months of the streaming service.

\section*{Question 17}
\textbf{Metadata}

\begin{itemize}
  \item Question ID: P4-DcMul2d1d\_P4-DcAdd2nd\_sonnet4\_Services\_01
  \item Primary KC: DECIMALS | Multiplication | multiplying decimals (up to 2 decimal places) by a 1-digit whole number
  \item Secondary KC: DECIMALS | Addition | adding decimals
  \item Topic: Services such as installation, maintenance, repairing, cleaning, laundry, hotel, retail, e-commerce, streaming services, digital services etc.
  \item Grade: Primary 4
\end{itemize}

\textbf{Solution}

Step 1: Calculate Maria's earnings for Monday.
Maria worked for 3 hours at \textdollar12.75 per hour.
Earnings for Monday = $3 \times 12.75 = 38.25$
So Maria earned \textdollar38.25 on Monday.

Step 2: Calculate Maria's earnings for Tuesday.
Maria worked for 4 hours at \textdollar12.75 per hour.
Earnings for Tuesday = $4 \times 12.75 = 51.00$
So Maria earned \textdollar51.00 on Tuesday.

Step 3: Calculate Maria's total earnings.
Total earnings = Monday earnings + Tuesday earnings + Tips
Total earnings = $38.25 + 51.00 + 8.50 = 97.75$

Therefore, Maria earned \textdollar97.75 in total for the two days.

\section*{Question 18}
\textbf{Metadata}

\begin{itemize}
  \item Question ID: P4-DcDiv2d1d\_P4-DcCmp3d\_sonnet4\_Services\_01
  \item Primary KC: DECIMALS | Division | dividing decimals (up to 2 decimal places) by a 1-digit whole number
  \item Secondary KC: DECIMALS | Comparison and ordering | comparing and ordering decimals up to 3 decimal places
  \item Topic: Services such as installation, maintenance, repairing, cleaning, laundry, hotel, retail, e-commerce, streaming services, digital services etc.
  \item Grade: Primary 4
\end{itemize}

\textbf{Solution}

(a) Total cost for office cleaning = \textdollar12.60 $\times$ 3 = \textdollar37.80
Cost per hour per worker = \textdollar37.80 $\div$ 3 = \textdollar12.600

(b) Total cost for house cleaning = \textdollar15.75 $\times$ 5 = \textdollar78.75
Cost per hour per worker = \textdollar78.75 $\div$ 5 = \textdollar15.750

(c) Total cost for carpet cleaning = \textdollar18.84 $\times$ 4 = \textdollar75.36
Cost per hour per worker = \textdollar75.36 $\div$ 4 = \textdollar18.840

(d) Comparing the costs per hour per worker:
\textdollar12.600, \textdollar15.750, \textdollar18.840

In ascending order: \textdollar12.600, \textdollar15.750, \textdollar18.840

\section*{Question 19}
\textbf{Metadata}

\begin{itemize}
  \item Question ID: P4-DcDiv2d1d\_P4-DcAdd2nd\_sonnet4\_Services\_01
  \item Primary KC: DECIMALS | Division | dividing decimals (up to 2 decimal places) by a 1-digit whole number
  \item Secondary KC: DECIMALS | Addition | adding decimals
  \item Topic: Services such as installation, maintenance, repairing, cleaning, laundry, hotel, retail, e-commerce, streaming services, digital services etc.
  \item Grade: Primary 4
\end{itemize}

\textbf{Solution}

Step 1: Find Sarah's earnings per house on Monday.
Earnings per house on Monday = \textdollar12.48 ÷ 3 = \textdollar4.16

Step 2: Find Sarah's earnings per house on Tuesday.
Earnings per house on Tuesday = \textdollar15.36 ÷ 4 = \textdollar3.84

Step 3: Find Sarah's total earnings over the two days.
Total earnings = \textdollar12.48 + \textdollar15.36 = \textdollar27.84

Therefore, Sarah earned \textdollar27.84 in total over the two days.

\section*{Question 20}
\textbf{Metadata}

\begin{itemize}
  \item Question ID: P5-FrAddMix\_P2-FrCmp\_sonnet4\_Services\_01
  \item Primary KC: FRACTIONS | Addition | adding mixed numbers
  \item Secondary KC: FRACTIONS | Comparison and ordering | comparing and ordering fractions
  \item Topic: Services such as installation, maintenance, repairing, cleaning, laundry, hotel, retail, e-commerce, streaming services, digital services etc.
  \item Grade: Primary 5
\end{itemize}

\textbf{Solution}

To solve this problem, I need to find the total hours Sarah worked each day, then compare them.

**Step 1: Calculate Monday's total hours**
$2\frac{3}{4} + 1\frac{5}{6}$

First, I'll convert to improper fractions:
$2\frac{3}{4} = \frac{11}{4}$ and $1\frac{5}{6} = \frac{11}{6}$

To add these fractions, I need a common denominator. The LCD of 4 and 6 is 12.
$\frac{11}{4} = \frac{33}{12}$ and $\frac{11}{6} = \frac{22}{12}$

$\frac{33}{12} + \frac{22}{12} = \frac{55}{12} = 4\frac{7}{12}$ hours

**Step 2: Calculate Tuesday's total hours**
$3\frac{1}{3} + 2\frac{1}{2}$

Converting to improper fractions:
$3\frac{1}{3} = \frac{10}{3}$ and $2\frac{1}{2} = \frac{5}{2}$

The LCD of 3 and 2 is 6.
$\frac{10}{3} = \frac{20}{6}$ and $\frac{5}{2} = \frac{15}{6}$

$\frac{20}{6} + \frac{15}{6} = \frac{35}{6} = 5\frac{5}{6}$ hours

**Step 3: Compare the totals**
Monday: $4\frac{7}{12}$ hours
Tuesday: $5\frac{5}{6}$ hours

To compare, I'll convert to the same denominator (12):
$5\frac{5}{6} = 5\frac{10}{12}$

Since $5\frac{10}{12} > 4\frac{7}{12}$, Tuesday had more hours.

**Answer:** Sarah worked more hours on Tuesday, working a total of $5\frac{5}{6}$ hours that day.

\section*{Question 21}
\textbf{Metadata}

\begin{itemize}
  \item Question ID: P5-FrSubMix\_P2-FrCmp\_sonnet4\_Services\_01
  \item Primary KC: FRACTIONS | Subtraction | subtracting mixed numbers
  \item Secondary KC: FRACTIONS | Comparison and ordering | comparing and ordering fractions
  \item Topic: Services such as installation, maintenance, repairing, cleaning, laundry, hotel, retail, e-commerce, streaming services, digital services etc.
  \item Grade: Primary 5
\end{itemize}

\textbf{Solution}

(a) To find how much more work Team A completed compared to Team B:
Team A: $4\frac{3}{4}$ hours
Team B: $2\frac{5}{6}$ hours

First, convert to improper fractions:
$4\frac{3}{4} = \frac{19}{4}$ and $2\frac{5}{6} = \frac{17}{6}$

To subtract, find common denominator (LCM of 4 and 6 is 12):
$\frac{19}{4} = \frac{57}{12}$ and $\frac{17}{6} = \frac{34}{12}$

$\frac{57}{12} - \frac{34}{12} = \frac{23}{12} = 1\frac{11}{12}$

Team A completed $1\frac{11}{12}$ hours more work than Team B.

(b) To find how much less work Team C completed compared to their plan:
Planned: $6\frac{1}{2}$ hours
Actual: $3\frac{2}{3}$ hours

Convert to improper fractions:
$6\frac{1}{2} = \frac{13}{2}$ and $3\frac{2}{3} = \frac{11}{3}$

Find common denominator (LCM of 2 and 3 is 6):
$\frac{13}{2} = \frac{39}{6}$ and $\frac{11}{3} = \frac{22}{6}$

$\frac{39}{6} - \frac{22}{6} = \frac{17}{6} = 2\frac{5}{6}$

Team C completed $2\frac{5}{6}$ hours less work than planned.

(c) To compare the actual work completed:
Team A: $4\frac{3}{4} = \frac{57}{12}$ hours
Team B: $2\frac{5}{6} = \frac{34}{12}$ hours
Team C: $3\frac{2}{3} = \frac{44}{12}$ hours

Comparing: $\frac{57}{12} > \frac{44}{12} > \frac{34}{12}$

Therefore, Team A completed the most work and Team B completed the least work.

\section*{Question 22}
\textbf{Metadata}

\begin{itemize}
  \item Question ID: P5-FrMulImN\_P2-FrCmp\_sonnet4\_Services\_01
  \item Primary KC: FRACTIONS | Multiplication | multiplying a proper/improper fraction and a whole number
  \item Secondary KC: FRACTIONS | Comparison and ordering | comparing and ordering fractions
  \item Topic: Services such as installation, maintenance, repairing, cleaning, laundry, hotel, retail, e-commerce, streaming services, digital services etc.
  \item Grade: Primary 5
\end{itemize}

\textbf{Solution}

First, I need to find how many loads Sarah washed each day by multiplying fractions with whole numbers.

Monday: $\frac{3}{4} \times 24 = \frac{3 \times 24}{4} = \frac{72}{4} = 18$ loads

Tuesday: $\frac{2}{3} \times 18 = \frac{2 \times 18}{3} = \frac{36}{3} = 12$ loads

Wednesday: $\frac{5}{6} \times 30 = \frac{5 \times 30}{6} = \frac{150}{6} = 25$ loads

Now I need to compare the number of loads washed each day:
Monday: 18 loads
Tuesday: 12 loads
Wednesday: 25 loads

Comparing these numbers: $25 > 18 > 12$

Therefore, Sarah washed the most loads of laundry on Wednesday (25 loads) and the least loads of laundry on Tuesday (12 loads).

\section*{Question 23}
\textbf{Metadata}

\begin{itemize}
  \item Question ID: P5-FrMulPIm\_P2-FrCmp\_sonnet4\_Services\_01
  \item Primary KC: FRACTIONS | Multiplication | multiplying a proper fraction and a proper/improper fractions
  \item Secondary KC: FRACTIONS | Comparison and ordering | comparing and ordering fractions
  \item Topic: Services such as installation, maintenance, repairing, cleaning, laundry, hotel, retail, e-commerce, streaming services, digital services etc.
  \item Grade: Primary 5
\end{itemize}

\textbf{Solution}

(a) To find the fraction of all hotel rooms each team actually cleaned, I need to multiply the fraction they completed by the fraction they were assigned.

Team A: $\frac{3}{4} \times \frac{2}{3} = \frac{3 \times 2}{4 \times 3} = \frac{6}{12} = \frac{1}{2}$

Team B: $\frac{5}{6} \times \frac{1}{4} = \frac{5 \times 1}{6 \times 4} = \frac{5}{24}$

Team C: $\frac{4}{5} \times \frac{1}{12} = \frac{4 \times 1}{5 \times 12} = \frac{4}{60} = \frac{1}{15}$

(b) To compare the fractions $\frac{1}{2}$, $\frac{5}{24}$, and $\frac{1}{15}$, I need to find a common denominator.

The LCM of 2, 24, and 15 is 120.

$\frac{1}{2} = \frac{60}{120}$

$\frac{5}{24} = \frac{25}{120}$

$\frac{1}{15} = \frac{8}{120}$

Comparing: $\frac{60}{120} > \frac{25}{120} > \frac{8}{120}$

Therefore: $\frac{1}{2} > \frac{5}{24} > \frac{1}{15}$

Team A cleaned the most rooms in the hotel, and Team C cleaned the least rooms.

\section*{Question 24}
\textbf{Metadata}

\begin{itemize}
  \item Question ID: P5-FrMulPIm\_P2-FrAdd2nd\_sonnet4\_Services\_01
  \item Primary KC: FRACTIONS | Multiplication | multiplying a proper fraction and a proper/improper fractions
  \item Secondary KC: FRACTIONS | Addition | adding fractions
  \item Topic: Services such as installation, maintenance, repairing, cleaning, laundry, hotel, retail, e-commerce, streaming services, digital services etc.
  \item Grade: Primary 5
\end{itemize}

\textbf{Solution}

Let me solve this step by step.

Step 1: Find how much work Sarah completed on Tuesday.
On Monday, Sarah completed $\frac{3}{4}$ of the job.
Remaining work after Monday = $1 - \frac{3}{4} = \frac{1}{4}$ of the original job.

On Tuesday, she completed $\frac{2}{3}$ of the remaining work:
Tuesday's work = $\frac{2}{3} \times \frac{1}{4} = \frac{2}{12} = \frac{1}{6}$ of the original job.

Step 2: Find how much work Sarah completed on Wednesday.
Remaining work after Tuesday = $\frac{1}{4} - \frac{1}{6}$
To subtract these fractions, find a common denominator:
$\frac{1}{4} - \frac{1}{6} = \frac{3}{12} - \frac{2}{12} = \frac{1}{12}$ of the original job.

On Wednesday, she completed $\frac{1}{2}$ of what was left:
Wednesday's work = $\frac{1}{2} \times \frac{1}{12} = \frac{1}{24}$ of the original job.

Step 3: Find the total work completed over three days.
Total work = Monday's work + Tuesday's work + Wednesday's work
Total work = $\frac{3}{4} + \frac{1}{6} + \frac{1}{24}$

To add these fractions, find a common denominator of 24:
$\frac{3}{4} = \frac{18}{24}$
$\frac{1}{6} = \frac{4}{24}$
$\frac{1}{24} = \frac{1}{24}$

Total work = $\frac{18}{24} + \frac{4}{24} + \frac{1}{24} = \frac{23}{24}$

Therefore, Sarah completed $\frac{23}{24}$ of the original house cleaning job over the three days.

\section*{Question 25}
\textbf{Metadata}

\begin{itemize}
  \item Question ID: P5-FrMulPIm\_P3-FrSmp\_sonnet4\_Services\_01
  \item Primary KC: FRACTIONS | Multiplication | multiplying a proper fraction and a proper/improper fractions
  \item Secondary KC: FRACTIONS | Simplifying | expressing a fraction in its simplest form
  \item Topic: Services such as installation, maintenance, repairing, cleaning, laundry, hotel, retail, e-commerce, streaming services, digital services etc.
  \item Grade: Primary 5
\end{itemize}

\textbf{Solution}

To find how much floor cleaner will be used, I need to multiply $\frac{3}{4}$ by $\frac{4}{5}$.

$\frac{3}{4} \times \frac{4}{5} = \frac{3 \times 4}{4 \times 5} = \frac{12}{20}$

Now I need to express $\frac{12}{20}$ in its simplest form by finding the highest common factor (HCF) of 12 and 20.

Factors of 12: 1, 2, 3, 4, 6, 12
Factors of 20: 1, 2, 4, 5, 10, 20

The HCF of 12 and 20 is 4.

$\frac{12}{20} = \frac{12 \div 4}{20 \div 4} = \frac{3}{5}$

Therefore, the cleaning service will use $\frac{3}{5}$ of a bottle of floor cleaner to clean one office.

\section*{Question 26}
\textbf{Metadata}

\begin{itemize}
  \item Question ID: P5-FrMulImIm\_P2-FrAdd2nd\_sonnet4\_Services\_01
  \item Primary KC: FRACTIONS | Multiplication | multiplying two improper fractions
  \item Secondary KC: FRACTIONS | Addition | adding fractions
  \item Topic: Services such as installation, maintenance, repairing, cleaning, laundry, hotel, retail, e-commerce, streaming services, digital services etc.
  \item Grade: Primary 5
\end{itemize}

\textbf{Solution}

To find the total earnings, I need to calculate the earnings from regular cleaning and deep cleaning separately, then add them together.

Step 1: Calculate earnings from regular cleaning.
Earnings from regular cleaning = Rate per hour × Hours worked
= $\frac{9}{4} \times \frac{8}{3}$
= $\frac{9 \times 8}{4 \times 3}$
= $\frac{72}{12}$
= $6$

Step 2: Calculate earnings from deep cleaning.
Earnings from deep cleaning = Rate per hour × Hours worked
= $\frac{7}{3} \times \frac{5}{2}$
= $\frac{7 \times 5}{3 \times 2}$
= $\frac{35}{6}$

Step 3: Add the two earnings together.
Total earnings = $6 + \frac{35}{6}$

To add these, I need to convert $6$ to a fraction with denominator $6$:
$6 = \frac{36}{6}$

Total earnings = $\frac{36}{6} + \frac{35}{6}$
= $\frac{36 + 35}{6}$
= $\frac{71}{6}$
= $11\frac{5}{6}$

Therefore, the company earned \textdollar$11\frac{5}{6}$ in total on Monday.

\section*{Question 27}
\textbf{Metadata}

\begin{itemize}
  \item Question ID: P5-FrMulImIm\_P2-FrSub2nd\_sonnet4\_Services\_01
  \item Primary KC: FRACTIONS | Multiplication | multiplying two improper fractions
  \item Secondary KC: FRACTIONS | Subtraction | subtracting fractions
  \item Topic: Services such as installation, maintenance, repairing, cleaning, laundry, hotel, retail, e-commerce, streaming services, digital services etc.
  \item Grade: Primary 5
\end{itemize}

\textbf{Solution}

Step 1: Find the total cleaning solution used for one office building.
Cleaning solution per building = $\frac{7}{4} + \frac{5}{3}$
To add these fractions, find a common denominator: LCM of 4 and 3 is 12
$\frac{7}{4} = \frac{7 \times 3}{4 \times 3} = \frac{21}{12}$
$\frac{5}{3} = \frac{5 \times 4}{3 \times 4} = \frac{20}{12}$
Cleaning solution per building = $\frac{21}{12} + \frac{20}{12} = \frac{41}{12}$ bottles

Step 2: Find the total cleaning solution used for 2 office buildings.
This requires multiplying two improper fractions: $\frac{41}{12} \times \frac{2}{1}$
Total cleaning solution = $\frac{41 \times 2}{12 \times 1} = \frac{82}{12}$ bottles

Step 3: Find the amount of expired cleaning solution.
Expired solution = $\frac{1}{6} \times \frac{82}{12}$
Multiplying the improper fractions: $\frac{1 \times 82}{6 \times 12} = \frac{82}{72} = \frac{41}{36}$ bottles

Step 4: Find the actual cleaning solution used after removing expired portion.
Actual solution used = $\frac{82}{12} - \frac{41}{36}$
To subtract, find a common denominator: LCM of 12 and 36 is 36
$\frac{82}{12} = \frac{82 \times 3}{12 \times 3} = \frac{246}{36}$
Actual solution used = $\frac{246}{36} - \frac{41}{36} = \frac{205}{36}$ bottles

Therefore, they actually used $\frac{205}{36}$ bottles of cleaning solution.

\section*{Question 28}
\textbf{Metadata}

\begin{itemize}
  \item Question ID: P5-FrMulImIm\_P3-FrSmp\_sonnet4\_Services\_01
  \item Primary KC: FRACTIONS | Multiplication | multiplying two improper fractions
  \item Secondary KC: FRACTIONS | Simplifying | expressing a fraction in its simplest form
  \item Topic: Services such as installation, maintenance, repairing, cleaning, laundry, hotel, retail, e-commerce, streaming services, digital services etc.
  \item Grade: Primary 5
\end{itemize}

\textbf{Solution}

To find the total number of bottles needed, I need to multiply the number of bottles per building by the number of buildings.

Number of bottles per building = $\frac{7}{4}$
Number of buildings = $\frac{9}{5}$

Total bottles needed = $\frac{7}{4} \times \frac{9}{5}$

To multiply fractions, I multiply the numerators together and multiply the denominators together:
$\frac{7}{4} \times \frac{9}{5} = \frac{7 \times 9}{4 \times 5} = \frac{63}{20}$

Now I need to check if $\frac{63}{20}$ can be simplified by finding the highest common factor (HCF) of 63 and 20.

Factors of 63: 1, 3, 7, 9, 21, 63
Factors of 20: 1, 2, 4, 5, 10, 20

The only common factor is 1, so the HCF of 63 and 20 is 1.

Since the HCF is 1, $\frac{63}{20}$ is already in its simplest form.

Therefore, the cleaning service company will need $\frac{63}{20}$ bottles of cleaning solution.

\section*{Question 29}
\textbf{Metadata}

\begin{itemize}
  \item Question ID: P5-FrMulMixN\_P2-FrSub2nd\_sonnet4\_Services\_01
  \item Primary KC: FRACTIONS | Multiplication | multiplying a mixed number and a whole number
  \item Secondary KC: FRACTIONS | Subtraction | subtracting fractions
  \item Topic: Services such as installation, maintenance, repairing, cleaning, laundry, hotel, retail, e-commerce, streaming services, digital services etc.
  \item Grade: Primary 5
\end{itemize}

\textbf{Solution}

Step 1: Find the total hours Sarah worked on Monday.
Sarah worked $3\frac{1}{4}$ hours at each of 6 houses.
Total hours on Monday = $3\frac{1}{4} \times 6$

To multiply a mixed number by a whole number:
$3\frac{1}{4} \times 6 = \frac{13}{4} \times 6 = \frac{13 \times 6}{4} = \frac{78}{4} = 19\frac{2}{4} = 19\frac{1}{2}$ hours

Step 2: Find how many more hours Sarah worked on Monday than on Tuesday.
Hours worked on Monday = $19\frac{1}{2}$ hours
Hours worked on Tuesday = $8\frac{3}{4}$ hours

Difference = $19\frac{1}{2} - 8\frac{3}{4}$

To subtract fractions with different denominators:
$19\frac{1}{2} = 19\frac{2}{4}$

$19\frac{2}{4} - 8\frac{3}{4}$

Since $\frac{2}{4} < \frac{3}{4}$, we need to borrow from the whole number:
$19\frac{2}{4} = 18\frac{6}{4}$

$18\frac{6}{4} - 8\frac{3}{4} = (18-8) + (\frac{6}{4} - \frac{3}{4}) = 10 + \frac{3}{4} = 10\frac{3}{4}$

Therefore, Sarah worked $10\frac{3}{4}$ hours more on Monday than on Tuesday.

\section*{Question 30}
\textbf{Metadata}

\begin{itemize}
  \item Question ID: P5-DcMul3dK\_P4-DcCnv2Fr\_sonnet4\_Services\_01
  \item Primary KC: DECIMALS | Multiplication | multiplying decimals (up to 3 decimal places) by 10, 100, 1000 and their multiples
  \item Secondary KC: DECIMALS | Conversion from decimals to fraction | expressing decimals as fractions
  \item Topic: Services such as installation, maintenance, repairing, cleaning, laundry, hotel, retail, e-commerce, streaming services, digital services etc.
  \item Grade: Primary 5
\end{itemize}

\textbf{Solution}

(a) First, I need to find Sarah's total earnings.

Sarah charges \textdollar0.125 per square meter.

Total area to be cleaned = $200 + 800 + 1000 = 2000$ square meters

To find the total earnings, I multiply the charge per square meter by the total area:
Total earnings = $0.125 \times 2000$

Using the multiplication rule for decimals by multiples of 10:
$0.125 \times 2000 = 0.125 \times 2 \times 1000 = 0.25 \times 1000 = 250$

Therefore, Sarah will earn \textdollar250 from cleaning all three offices.

(b) To express \textdollar0.125 as a fraction:

$0.125 = \frac{125}{1000}$

To simplify this fraction, I need to find the highest common factor of 125 and 1000:
$125 = 5^3$
$1000 = 2^3 \times 5^3$

The highest common factor is $5^3 = 125$

$\frac{125}{1000} = \frac{125 \div 125}{1000 \div 125} = \frac{1}{8}$

Therefore, Sarah's charge per square meter expressed as a fraction in its simplest form is $\frac{1}{8}$ dollar per square meter.

\section*{Question 31}
\textbf{Metadata}

\begin{itemize}
  \item Question ID: P5-DcMul3dK\_P4-DcSub2nd\_sonnet4\_Services\_01
  \item Primary KC: DECIMALS | Multiplication | multiplying decimals (up to 3 decimal places) by 10, 100, 1000 and their multiples
  \item Secondary KC: DECIMALS | Subtraction | subtracting decimals
  \item Topic: Services such as installation, maintenance, repairing, cleaning, laundry, hotel, retail, e-commerce, streaming services, digital services etc.
  \item Grade: Primary 5
\end{itemize}

\textbf{Solution}

Step 1: Calculate the original cleaning cost before discount.
Original cost = \textdollar0.085 $\times$ 1200

To multiply 0.085 by 1200:
0.085 $\times$ 1200 = 0.085 $\times$ 12 $\times$ 100 = 1.02 $\times$ 100 = \textdollar102.00

Step 2: Apply the promotional discount.
Final amount = Original cost - Discount
Final amount = \textdollar102.00 - \textdollar8.75 = \textdollar93.25

Therefore, the company will charge \textdollar93.25 for their cleaning services this month.

\section*{Question 32}
\textbf{Metadata}

\begin{itemize}
  \item Question ID: P5-DcDiv3dK\_P4-DcRnd3d\_sonnet4\_Services\_01
  \item Primary KC: DECIMALS | Division | dividing decimals (up to 3 decimal places) by 10, 100, 1000 and their multiples
  \item Secondary KC: DECIMALS | Rounding | rounding decimals up to 3 decimal places to the nearest whole number, 1 decimal place and 2 decimal places 
  \item Topic: Services such as installation, maintenance, repairing, cleaning, laundry, hotel, retail, e-commerce, streaming services, digital services etc.
  \item Grade: Primary 5
\end{itemize}

\textbf{Solution}

Step 1: Find how many square meters each team will clean.
Total area to clean = $48,000$ square meters
Number of teams = $200$
Area per team = $48,000 \div 200 = 240$ square meters

Step 2: Calculate the total cost for the entire contract.
Cost per square meter = \textdollar$2.456$
Total area = $48,000$ square meters
Total cost = $2.456 \times 48,000$

To calculate this, we can use the fact that $48,000 = 48 \times 1,000$:
$2.456 \times 48,000 = 2.456 \times 48 \times 1,000$
$= 117.888 \times 1,000$
$= 117,888$

So the exact total cost is \textdollar$117,888.000$

Step 3: Round the total cost to the nearest dollar.
\textdollar$117,888.000$ rounded to the nearest dollar is \textdollar$117,888$

Therefore:
- Each team will clean $240$ square meters
- The total cost for the entire contract, rounded to the nearest dollar, is \textdollar$117,888$

\section*{Question 33}
\textbf{Metadata}

\begin{itemize}
  \item Question ID: P5-DcDiv3dK\_P4-DcAdd2nd\_sonnet4\_Services\_01
  \item Primary KC: DECIMALS | Division | dividing decimals (up to 3 decimal places) by 10, 100, 1000 and their multiples
  \item Secondary KC: DECIMALS | Addition | adding decimals
  \item Topic: Services such as installation, maintenance, repairing, cleaning, laundry, hotel, retail, e-commerce, streaming services, digital services etc.
  \item Grade: Primary 5
\end{itemize}

\textbf{Solution}

Step 1: Find Sarah's total earnings for the month.
Total earnings = \textdollar428.50 + \textdollar156.75 = \textdollar585.25

Step 2: Calculate Sarah's average hourly rate.
Average hourly rate = Total earnings ÷ Total hours worked
Average hourly rate = \textdollar585.25 ÷ 300
Average hourly rate = \textdollar1.951

Therefore, Sarah's average hourly rate for her cleaning services was \textdollar1.951 per hour.

\section*{Question 34}
\textbf{Metadata}

\begin{itemize}
  \item Question ID: P5-PcRepWh\_P1-WNMul2nd\_sonnet4\_Services\_01
  \item Primary KC: PERCENTAGE | Representation and concept | expressing a part of a whole as a percentage
  \item Secondary KC: WHOLE NUMBERS | Multiplication | multiplying whole numbers
  \item Topic: Services such as installation, maintenance, repairing, cleaning, laundry, hotel, retail, e-commerce, streaming services, digital services etc.
  \item Grade: Primary 5
\end{itemize}

\textbf{Solution}

Step 1: Find the total earnings from room cleaning.
Total earnings from cleaning = \textdollar8 $\times$ 45 = \textdollar360

Step 2: Find the amount of tips received.
Tips = 15\% of \textdollar360
Tips = $\frac{15}{100} \times 360$
Tips = $15 \times \frac{360}{100}$
Tips = $15 \times 3.6$
Tips = \textdollar54

Step 3: Find the total earnings including tips.
Total earnings = \textdollar360 + \textdollar54 = \textdollar414

Therefore, the company earned \textdollar414 in total last month including tips.

\section*{Question 35}
\textbf{Metadata}

\begin{itemize}
  \item Question ID: P5-PcRepWh\_P1-WNDiv2nd\_sonnet4\_Services\_01
  \item Primary KC: PERCENTAGE | Representation and concept | expressing a part of a whole as a percentage
  \item Secondary KC: WHOLE NUMBERS | Division | dividing whole numbers
  \item Topic: Services such as installation, maintenance, repairing, cleaning, laundry, hotel, retail, e-commerce, streaming services, digital services etc.
  \item Grade: Primary 5
\end{itemize}

\textbf{Solution}

To find the percentage of laptops that were successfully repaired, I need to:

1) Find what fraction of the laptops were repaired
2) Convert this fraction to a percentage

Step 1: Find the fraction of laptops repaired
Number of laptops repaired = 180
Total number of laptops = 240
Fraction repaired = $\frac{180}{240}$

Step 2: Simplify the fraction by dividing both numerator and denominator
$180 \div 60 = 3$
$240 \div 60 = 4$
So $\frac{180}{240} = \frac{3}{4}$

Step 3: Convert the fraction to a percentage
$\frac{3}{4} = \frac{3 \times 25}{4 \times 25} = \frac{75}{100} = 75\%$

Therefore, $75\%$ of the laptops were successfully repaired.

\section*{Question 36}
\textbf{Metadata}

\begin{itemize}
  \item Question ID: P5-RtFndU\_P2-DcCnvN2D\_sonnet4\_Services\_01
  \item Primary KC: RATE | Finding number of unit | finding number of units given rate and total amount
  \item Secondary KC: DECIMALS | Conversion to larger units | converting a measurement from a smaller unit to a larger unit in decimal form
  \item Topic: Services such as installation, maintenance, repairing, cleaning, laundry, hotel, retail, e-commerce, streaming services, digital services etc.
  \item Grade: Primary 5
\end{itemize}

\textbf{Solution}

To find the number of hours the cleaning service worked, I need to divide the total amount paid by the rate per hour.

Number of hours = Total amount ÷ Rate per hour
Number of hours = \textdollar78 ÷ \textdollar12 per hour
Number of hours = $6.5$ hours

Now I need to convert $6.5$ hours to hours and minutes.
$6.5$ hours = $6$ hours + $0.5$ hours

To convert $0.5$ hours to minutes:
$0.5$ hours = $0.5 \times 60$ minutes = $30$ minutes

Therefore, the cleaning service worked for $6$ hours and $30$ minutes.

\section*{Question 37}
\textbf{Metadata}

\begin{itemize}
  \item Question ID: P6-FrDivPN\_P2-FrAdd2nd\_sonnet4\_Services\_01
  \item Primary KC: FRACTIONS | Division | dividing a proper fraction by a whole number
  \item Secondary KC: FRACTIONS | Addition | adding fractions
  \item Topic: Services such as installation, maintenance, repairing, cleaning, laundry, hotel, retail, e-commerce, streaming services, digital services etc.
  \item Grade: Primary 6
\end{itemize}

\textbf{Solution}

Step 1: Find the time Sarah spent cleaning on Monday.
Sarah cleaned 6 rooms, spending $\frac{3}{4}$ hour on each room.
Time on Monday = $6 \times \frac{3}{4} = \frac{6 \times 3}{4} = \frac{18}{4} = \frac{9}{2}$ hours

Step 2: Find the time Sarah spent cleaning on Tuesday.
Sarah cleaned 4 rooms, spending $\frac{3}{4}$ hour on each room.
Time on Tuesday = $4 \times \frac{3}{4} = \frac{4 \times 3}{4} = \frac{12}{4} = 3$ hours

Step 3: Find the total time spent cleaning over both days.
Total time = Time on Monday + Time on Tuesday
Total time = $\frac{9}{2} + 3$
Total time = $\frac{9}{2} + \frac{6}{2} = \frac{15}{2} = 7\frac{1}{2}$ hours

Therefore, Sarah spent $7\frac{1}{2}$ hours cleaning over both days.

\section*{Question 38}
\textbf{Metadata}

\begin{itemize}
  \item Question ID: P6-FrDivPN\_P5-FrCnv2Dc\_sonnet4\_Services\_01
  \item Primary KC: FRACTIONS | Division | dividing a proper fraction by a whole number
  \item Secondary KC: FRACTIONS | Conversion to decimals | expressing fractions as decimals
  \item Topic: Services such as installation, maintenance, repairing, cleaning, laundry, hotel, retail, e-commerce, streaming services, digital services etc.
  \item Grade: Primary 6
\end{itemize}

\textbf{Solution}

To find how much cleaning solution each station receives, I need to divide $\frac{3}{4}$ litre by 6.

$\frac{3}{4} \div 6 = \frac{3}{4} \times \frac{1}{6} = \frac{3}{24} = \frac{1}{8}$

Now I need to express $\frac{1}{8}$ as a decimal.

$\frac{1}{8} = 1 \div 8 = 0.125$

Therefore, each cleaning station will receive $0.125$ litres of cleaning solution.

\section*{Question 39}
\textbf{Metadata}

\begin{itemize}
  \item Question ID: P6-FrDivPP\_P2-FrSub2nd\_sonnet4\_Services\_01
  \item Primary KC: FRACTIONS | Division | dividing a whole number/proper fraction by a proper fraction
  \item Secondary KC: FRACTIONS | Subtraction | subtracting fractions
  \item Topic: Services such as installation, maintenance, repairing, cleaning, laundry, hotel, retail, e-commerce, streaming services, digital services etc.
  \item Grade: Primary 6
\end{itemize}

\textbf{Solution}

Step 1: Find how much adhesive tape Sarah has remaining after the morning session.

Remaining tape = $\frac{3}{4} - \frac{1}{6}$

To subtract these fractions, I need a common denominator. The LCM of 4 and 6 is 12.

$\frac{3}{4} = \frac{3 \times 3}{4 \times 3} = \frac{9}{12}$

$\frac{1}{6} = \frac{1 \times 2}{6 \times 2} = \frac{2}{12}$

Remaining tape = $\frac{9}{12} - \frac{2}{12} = \frac{7}{12}$

Step 2: Find how many repairs like the morning session can be completed with the remaining tape.

Since each repair session uses $\frac{1}{6}$ of the roll, I need to divide the remaining tape by the amount used per session:

Number of repair sessions = $\frac{7}{12} \div \frac{1}{6}$

To divide by a fraction, I multiply by its reciprocal:

$\frac{7}{12} \div \frac{1}{6} = \frac{7}{12} \times \frac{6}{1} = \frac{7 \times 6}{12 \times 1} = \frac{42}{12} = \frac{7}{2} = 3\frac{1}{2}$

Therefore, Sarah can complete $3\frac{1}{2}$ repair sessions like the morning session with her remaining adhesive tape.

\section*{Question 40}
\textbf{Metadata}

\begin{itemize}
  \item Question ID: P6-FrDivPP\_P5-FrCnv2Dc\_sonnet4\_Services\_01
  \item Primary KC: FRACTIONS | Division | dividing a whole number/proper fraction by a proper fraction
  \item Secondary KC: FRACTIONS | Conversion to decimals | expressing fractions as decimals
  \item Topic: Services such as installation, maintenance, repairing, cleaning, laundry, hotel, retail, e-commerce, streaming services, digital services etc.
  \item Grade: Primary 6
\end{itemize}

\textbf{Solution}

To find how many offices can be cleaned, I need to divide the total bottles by the amount used per office.

Total bottles available = $6$
Amount used per office = $\frac{3}{4}$ bottle

Number of offices = $6 \div \frac{3}{4}$

To divide by a fraction, I multiply by its reciprocal:
$6 \div \frac{3}{4} = 6 \times \frac{4}{3}$

$= \frac{6 \times 4}{3}$

$= \frac{24}{3}$

$= 8$

Now I need to express this as a decimal:
$8 = 8.0$

Therefore, the cleaning service company can clean $8.0$ offices with their current stock of cleaning solution.

\section*{Question 41}
\textbf{Metadata}

\begin{itemize}
  \item Question ID: P6-PcFndWN\_P1-WNAdd2nd\_sonnet4\_Services\_01
  \item Primary KC: PERCENTAGE | Finding the whole | finding the whole given a part and the percentage
  \item Secondary KC: WHOLE NUMBERS | Addition | adding whole numbers
  \item Topic: Services such as installation, maintenance, repairing, cleaning, laundry, hotel, retail, e-commerce, streaming services, digital services etc.
  \item Grade: Primary 6
\end{itemize}

\textbf{Solution}

Step 1: Find the service fee for the first laptop repair.
Since parts cost is 75\% of the total bill, the service fee is 25\% of the total bill.
Service fee for first laptop = 25\% of \textdollar180
$= \frac{25}{100} \times 180 = \frac{1}{4} \times 180 = \textdollar45$

Step 2: Find the service fee for the second laptop repair.
Service fee for second laptop = 25\% of \textdollar240
$= \frac{25}{100} \times 240 = \frac{1}{4} \times 240 = \textdollar60$

Step 3: Find the service fee for the third laptop repair.
Service fee for third laptop = 25\% of \textdollar300
$= \frac{25}{100} \times 300 = \frac{1}{4} \times 300 = \textdollar75$

Step 4: Add all the service fees together.
Total service fees = \textdollar45 + \textdollar60 + \textdollar75
$= \textdollar180$

Therefore, the total service fees collected from all three laptop repairs is \textdollar180.

\section*{Question 42}
\textbf{Metadata}

\begin{itemize}
  \item Question ID: P6-PcFndChg\_P1-WNDiv2nd\_sonnet4\_Services\_01
  \item Primary KC: PERCENTAGE | Finding change | finding percentage increase/decrease
  \item Secondary KC: WHOLE NUMBERS | Division | dividing whole numbers
  \item Topic: Services such as installation, maintenance, repairing, cleaning, laundry, hotel, retail, e-commerce, streaming services, digital services etc.
  \item Grade: Primary 6
\end{itemize}

\textbf{Solution}

Step 1: Find the increase in repair charges.
Increase = \textdollar600 - \textdollar480 = \textdollar120

Step 2: Find the percentage increase.
Percentage increase = $\frac{\text{Increase}}{\text{Original amount}} \times 100\%$
Percentage increase = $\frac{120}{480} \times 100\%$
Percentage increase = $\frac{1}{4} \times 100\% = 25\%$

Step 3: Find the average repair charge per laptop in February.
Average repair charge = Total charges ÷ Number of repairs
Average repair charge = \textdollar600 ÷ 15 = \textdollar40

Therefore, the repair charges increased by 25% from January to February, and the average repair charge per laptop in February was \textdollar40.

\section*{Question 43}
\textbf{Metadata}

\begin{itemize}
  \item Question ID: P6-RoFndRoWN\_P1-WNAdd2nd\_sonnet4\_Services\_01
  \item Primary KC: RATIO | Finding ratio | finding the ratio of two or three given whole numbers
  \item Secondary KC: WHOLE NUMBERS | Addition | adding whole numbers
  \item Topic: Services such as installation, maintenance, repairing, cleaning, laundry, hotel, retail, e-commerce, streaming services, digital services etc.
  \item Grade: Primary 6
\end{itemize}

\textbf{Solution}

First, I need to find how many jobs of each type they plan to complete this month.

Last month's jobs:
- House cleaning: $45$ jobs
- Office cleaning: $30$ jobs  
- Carpet cleaning: $15$ jobs

Additional jobs planned for this month:
- House cleaning: $20$ more jobs
- Office cleaning: $15$ more jobs
- Carpet cleaning: $10$ more jobs

This month's planned jobs:
- House cleaning: $45 + 20 = 65$ jobs
- Office cleaning: $30 + 15 = 45$ jobs
- Carpet cleaning: $15 + 10 = 25$ jobs

Now I need to find the ratio of house cleaning : office cleaning : carpet cleaning jobs.
The ratio is $65 : 45 : 25$.

To simplify this ratio, I need to find the greatest common divisor (GCD) of $65$, $45$, and $25$.

Factors of $65$: $1, 5, 13, 65$
Factors of $45$: $1, 3, 5, 9, 15, 45$
Factors of $25$: $1, 5, 25$

The GCD of $65$, $45$, and $25$ is $5$.

Dividing each term by $5$:
$65 \div 5 = 13$
$45 \div 5 = 9$
$25 \div 5 = 5$

Therefore, the ratio of house cleaning jobs to office cleaning jobs to carpet cleaning jobs is $13 : 9 : 5$.

\section*{Question 44}
\textbf{Metadata}

\begin{itemize}
  \item Question ID: P6-AgRepLrEx\_P6-AgSmpLrEx\_sonnet4\_Services\_01
  \item Primary KC: ALGEBRA | Representation and concept | translation of real-world situations into linear algebraic expressions
  \item Secondary KC: ALGEBRA | Simplifying | simplifying linear expressions
  \item Topic: Services such as installation, maintenance, repairing, cleaning, laundry, hotel, retail, e-commerce, streaming services, digital services etc.
  \item Grade: Primary 6
\end{itemize}

\textbf{Solution}

Step 1: Identify what Sarah charges.
Sarah charges \textdollar15 basic fee per visit + \textdollar8 per hour of cleaning.

Step 2: Write expressions for each house visit.
For Mrs. Chen's house: $15 + 8h$ dollars
For Mr. Tan's house: $15 + 8(h + 2)$ dollars

Step 3: Write the total amount earned.
Total amount = $(15 + 8h) + (15 + 8(h + 2))$

Step 4: Simplify the expression.
$(15 + 8h) + (15 + 8(h + 2))$
$= 15 + 8h + 15 + 8h + 16$
$= 15 + 15 + 16 + 8h + 8h$
$= 46 + 16h$

Therefore, Sarah earned $(46 + 16h)$ dollars on Monday.

\section*{Question 45}
\textbf{Metadata}

\begin{itemize}
  \item Question ID: O1-PcFndRslt\_P1-WNSub2nd\_sonnet4\_Services\_01
  \item Primary KC: PERCENTAGE | Finding result after change | increasing/decreasing a quantity by a given percentage
  \item Secondary KC: WHOLE NUMBERS | Subtraction | subtracting whole numbers
  \item Topic: Services such as installation, maintenance, repairing, cleaning, laundry, hotel, retail, e-commerce, streaming services, digital services etc.
  \item Grade: Secondary O-level 1
\end{itemize}

\textbf{Solution}

Step 1: Calculate the discount amount.
Discount = $15\%$ of \textdollar180
Discount = $\frac{15}{100} \times 180 = \textdollar27$

Step 2: Find the price after applying the discount.
Price after discount = Original price $-$ Discount amount
Price after discount = \textdollar180 $-$ \textdollar27 = \textdollar153

Step 3: Add the service fee to get the final amount.
Final amount = Price after discount $+$ Service fee
Final amount = \textdollar153 $+$ \textdollar12 = \textdollar165

Therefore, the customer needs to pay \textdollar165 for the computer repair service during the promotion.

\section*{Question 46}
\textbf{Metadata}

\begin{itemize}
  \item Question ID: O1-PcRepRvs\_O1-PcCnv2Dc\_sonnet4\_Services\_01
  \item Primary KC: PERCENTAGE | Representation and concept | reverse percentages
  \item Secondary KC: PERCENTAGE | Conversion to decimals | expressing percentage as a decimal
  \item Topic: Services such as installation, maintenance, repairing, cleaning, laundry, hotel, retail, e-commerce, streaming services, digital services etc.
  \item Grade: Secondary O-level 1
\end{itemize}

\textbf{Solution}

Let the original base rate be $x$ dollars.

Since Sarah received a 15\% discount, she paid 85\% of the original price.

First, I need to express 85\% as a decimal:
85\% = $\frac{85}{100} = 0.85$

Since Sarah paid \textdollar68 after the discount:
$0.85x = 68$

To find the original base rate, I need to solve for $x$:
$x = \frac{68}{0.85}$
$x = \frac{68}{0.85} = \frac{6800}{85} = 80$

Therefore, the original base rate that the cleaning service normally charges is \textdollar80.

To verify: 15\% of \textdollar80 = $0.15 \times 80 = 12$
Amount paid after discount = $80 - 12 = 68$ ✓

\section*{Question 47}
\textbf{Metadata}

\begin{itemize}
  \item Question ID: O2-RoRepDP\_P1-WNMul2nd\_sonnet4\_Services\_01
  \item Primary KC: RATIO | Representation and concept | direct proportion
  \item Secondary KC: WHOLE NUMBERS | Multiplication | multiplying whole numbers
  \item Topic: Services such as installation, maintenance, repairing, cleaning, laundry, hotel, retail, e-commerce, streaming services, digital services etc.
  \item Grade: Secondary O-level 2
\end{itemize}

\textbf{Solution}

Let the charges for cleaning one bedroom, one bathroom, and one living area be $3x$, $2x$, and $5x$ dollars respectively, where $x$ is a constant.

The total number of rooms cleaned:
- Bedrooms: $12$
- Bathrooms: $8$ 
- Living areas: $20$

The total earnings can be calculated as:
Total earnings $= 12 \times 3x + 8 \times 2x + 20 \times 5x$

Simplifying:
Total earnings $= 36x + 16x + 100x = 152x$

Since the total earnings was \textdollar$1800$:
$152x = 1800$

Solving for $x$:
$x = \frac{1800}{152} = \frac{225}{19}$

Therefore, the charge per bedroom cleaned is:
$3x = 3 \times \frac{225}{19} = \frac{675}{19} \approx 35.53$

The charge per bedroom cleaned is \textdollar$\frac{675}{19}$ or approximately \textdollar$35.53$.

\section*{Question 48}
\textbf{Metadata}

\begin{itemize}
  \item Question ID: O3-MXMul\_O3-MXAdd\_sonnet4\_Services\_01
  \item Primary KC: MATRICES | Multiplication | multiplication of matrices
  \item Secondary KC: MATRICES | Addition | addition of matrices
  \item Topic: Services such as installation, maintenance, repairing, cleaning, laundry, hotel, retail, e-commerce, streaming services, digital services etc.
  \item Grade: Secondary O-level 3/4
\end{itemize}

\textbf{Solution}

**Solution:**

**(a) Finding the total cost matrix $C_{total}$**

To find the total costs for January and February combined, we add the two cost matrices:
$$C_{total} = C_1 + C_2 = \begin{pmatrix} 12 & 8 \\ 18 & 15 \\ 6 & 9 \end{pmatrix} + \begin{pmatrix} 3 & 2 \\ 4 & 5 \\ 2 & 1 \end{pmatrix}$$

$$C_{total} = \begin{pmatrix} 12+3 & 8+2 \\ 18+4 & 15+5 \\ 6+2 & 9+1 \end{pmatrix} = \begin{pmatrix} 15 & 10 \\ 22 & 20 \\ 8 & 10 \end{pmatrix}$$

**(b) Calculating the revenue matrix $R$**

The revenue matrix is found by element-wise multiplication of total costs and pricing factors:
$$R = C_{total} \circ P = \begin{pmatrix} 15 & 10 \\ 22 & 20 \\ 8 & 10 \end{pmatrix} \circ \begin{pmatrix} 1.5 & 1.8 \\ 2.0 & 1.6 \\ 1.4 & 2.2 \end{pmatrix}$$

$$R = \begin{pmatrix} 15 \times 1.5 & 10 \times 1.8 \\ 22 \times 2.0 & 20 \times 1.6 \\ 8 \times 1.4 & 10 \times 2.2 \end{pmatrix} = \begin{pmatrix} 22.5 & 18 \\ 44 & 32 \\ 11.2 & 22 \end{pmatrix}$$

**(c) Computing total revenue for March**

The total revenue is calculated by matrix multiplication $R \times D$:
$$R \times D = \begin{pmatrix} 22.5 & 18 \\ 44 & 32 \\ 11.2 & 22 \end{pmatrix} \times \begin{pmatrix} 4 \\ 6 \end{pmatrix}$$

$$= \begin{pmatrix} 22.5 \times 4 + 18 \times 6 \\ 44 \times 4 + 32 \times 6 \\ 11.2 \times 4 + 22 \times 6 \end{pmatrix} = \begin{pmatrix} 90 + 108 \\ 176 + 192 \\ 44.8 + 132 \end{pmatrix} = \begin{pmatrix} 198 \\ 368 \\ 176.8 \end{pmatrix}$$

**Final Answer:**
- Total cost matrix: $C_{total} = \begin{pmatrix} 15 & 10 \\ 22 & 20 \\ 8 & 10 \end{pmatrix}$ (hundreds of dollars)
- Revenue matrix: $R = \begin{pmatrix} 22.5 & 18 \\ 44 & 32 \\ 11.2 & 22 \end{pmatrix}$ (hundreds of dollars)
- March revenue by service type: $\begin{pmatrix} 198 \\ 368 \\ 176.8 \end{pmatrix}$ (hundreds of dollars)
- Total March revenue: $198 + 368 + 176.8 = \textdollar74,280$

\section*{Question 49}
\textbf{Metadata}

\begin{itemize}
  \item Question ID: O3-SPFndstd\_O2-SPFndmean\_sonnet4\_Services\_01
  \item Primary KC: STATISTICS AND PROBABILITY | Finding standard deviation | calculation of the standard deviation for a set of data
  \item Secondary KC: STATISTICS AND PROBABILITY | Finding mean deviation | calculation of the mean for a set of data
  \item Topic: Services such as installation, maintenance, repairing, cleaning, laundry, hotel, retail, e-commerce, streaming services, digital services etc.
  \item Grade: Secondary O-level 3/4
\end{itemize}

\textbf{Solution}

Given data: 6, 8, 5, 9, 7, 6, 8, 10, 5, 6 hours

(a) Finding the mean:
Mean $= \frac{\text{Sum of all values}}{\text{Number of values}}$

Sum $= 6 + 8 + 5 + 9 + 7 + 6 + 8 + 10 + 5 + 6 = 70$ hours

Number of values $= 10$ days

Mean $= \frac{70}{10} = 7$ hours per day

(b) Finding the standard deviation:
First, find the deviations from the mean:
$(6-7)^2 = (-1)^2 = 1$
$(8-7)^2 = (1)^2 = 1$
$(5-7)^2 = (-2)^2 = 4$
$(9-7)^2 = (2)^2 = 4$
$(7-7)^2 = (0)^2 = 0$
$(6-7)^2 = (-1)^2 = 1$
$(8-7)^2 = (1)^2 = 1$
$(10-7)^2 = (3)^2 = 9$
$(5-7)^2 = (-2)^2 = 4$
$(6-7)^2 = (-1)^2 = 1$

Sum of squared deviations $= 1 + 1 + 4 + 4 + 0 + 1 + 1 + 9 + 4 + 1 = 26$

Variance $= \frac{\text{Sum of squared deviations}}{n} = \frac{26}{10} = 2.6$

Standard deviation $= \sqrt{\text{Variance}} = \sqrt{2.6} = 1.61$ hours (to 2 decimal places)

Therefore:
(a) The mean number of hours spent on repairs per day is 7 hours
(b) The standard deviation of the daily repair hours is 1.61 hours

\end{document}
