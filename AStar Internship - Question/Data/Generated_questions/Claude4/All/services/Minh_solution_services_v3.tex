\documentclass{article}
\usepackage[utf8]{inputenc}
\usepackage{amsmath}
\usepackage{amsfonts}
\usepackage{amssymb}
\usepackage{graphicx}
\usepackage{hyperref}
\title{'Minh Solutions services v3 CLAUDE '}
\author{Tien Dung Doan}
\begin{document}
\maketitle
\section*{Question 1}
\textbf{Metadata}

\begin{itemize}
  \item Question ID: P3-WNSub4d\_P1-WNCmp\_sonnet4\_Services\_02
  \item Primary KC: WHOLE NUMBERS | Subtraction | subtracting whole numbers up to 4 digits
  \item Secondary KC: WHOLE NUMBERS | Comparison and ordering | comparing and ordering whole numbers
  \item Topic: Services such as installation, maintenance, repairing, cleaning, laundry, hotel, retail, e-commerce, streaming services, digital services etc.
  \item Grade: Primary 3
\end{itemize}

\textbf{Solution}

Step 1: Find how many more computers were repaired in January compared to February.

January repairs: 2,845 computers
February repairs: 1,976 computers

Difference = $2,845 - 1,976 = 869$

So January had 869 more computer repairs than February.

Step 2: Compare and order the three months to find the highest and lowest.

January: 2,845 computers
February: 1,976 computers  
March: 3,124 computers

Comparing the numbers:
$3,124 > 2,845 > 1,976$

Therefore:
- March had the highest number of repairs with 3,124 computers
- February had the lowest number of repairs with 1,976 computers

Answer: January had 869 more computer repairs than February. March had the highest number of repairs and February had the lowest number of repairs.

\section*{Question 2}
\textbf{Metadata}

\begin{itemize}
  \item Question ID: P3-WNDivRmd3d\_P1-WNAdd2nd\_sonnet4\_Services\_01
  \item Primary KC: WHOLE NUMBERS | Division | dividing whole numbers up to 3 digits by 1 digit with remainder 
  \item Secondary KC: WHOLE NUMBERS | Addition | adding whole numbers
  \item Topic: Services such as installation, maintenance, repairing, cleaning, laundry, hotel, retail, e-commerce, streaming services, digital services etc.
  \item Grade: Primary 3
\end{itemize}

\textbf{Solution}

Step 1: Find how many full loads can be washed and how many towels remain.
$347 \div 8 = 43$ remainder $3$
This means 43 full loads can be washed, leaving 3 towels.

Step 2: Add the new towels received from the hotel client.
Remaining towels from first batch: 3
New towels from hotel: 15
Total towels to be hand-washed: $3 + 15 = 18$

Therefore, 18 towels will need to be hand-washed in total.

\section*{Question 3}
\textbf{Metadata}

\begin{itemize}
  \item Question ID: P3-WNDivRmd3d\_P1-WNMul2nd\_sonnet4\_Services\_01
  \item Primary KC: WHOLE NUMBERS | Division | dividing whole numbers up to 3 digits by 1 digit with remainder 
  \item Secondary KC: WHOLE NUMBERS | Multiplication | multiplying whole numbers
  \item Topic: Services such as installation, maintenance, repairing, cleaning, laundry, hotel, retail, e-commerce, streaming services, digital services etc.
  \item Grade: Primary 3
\end{itemize}

\textbf{Solution}

Step 1: Convert minutes to hours by dividing the total minutes by 60.

$347 \div 60$

Using long division:
$347 \div 60 = 5$ remainder $47$

This means 347 minutes = 5 complete hours and 47 minutes.

Step 2: Since the shop only charges for complete hours, they worked 5 complete hours.

Step 3: Calculate the total earnings by multiplying the number of complete hours by the hourly rate.

Total earnings = $5 \times 8 = 40$

Therefore, the shop worked 5 complete hours and earned \textdollar40 on Monday.

\section*{Question 4}
\textbf{Metadata}

\begin{itemize}
  \item Question ID: P3-WNDiv3d1d\_P1-WNAdd2nd\_sonnet4\_Services\_01
  \item Primary KC: WHOLE NUMBERS | Division | dividing whole numbers up to 3 digits by 1 digit
  \item Secondary KC: WHOLE NUMBERS | Addition | adding whole numbers
  \item Topic: Services such as installation, maintenance, repairing, cleaning, laundry, hotel, retail, e-commerce, streaming services, digital services etc.
  \item Grade: Primary 3
\end{itemize}

\textbf{Solution}

Step 1: Find how many bed sheets each washing machine washed on Monday.
$248 \div 4 = 62$ bed sheets per machine on Monday

Step 2: Find how many bed sheets each washing machine washed on Tuesday.
$156 \div 3 = 52$ bed sheets per machine on Tuesday

Step 3: Find the total number of bed sheets each washing machine washed over the two days.
$62 + 52 = 114$ bed sheets

Therefore, each washing machine washed 114 bed sheets in total over the two days.

\section*{Question 5}
\textbf{Metadata}

\begin{itemize}
  \item Question ID: P3-WNDiv3d1d\_P1-WNSub2nd\_sonnet4\_Services\_01
  \item Primary KC: WHOLE NUMBERS | Division | dividing whole numbers up to 3 digits by 1 digit
  \item Secondary KC: WHOLE NUMBERS | Subtraction | subtracting whole numbers
  \item Topic: Services such as installation, maintenance, repairing, cleaning, laundry, hotel, retail, e-commerce, streaming services, digital services etc.
  \item Grade: Primary 3
\end{itemize}

\textbf{Solution}

Step 1: Find how many cleaning supplies each location receives initially.

Total cleaning supplies = 456
Number of locations = 3

Supplies per location = $456 \div 3 = 152$

Step 2: Find how many undamaged supplies each location receives.

Damaged supplies to be removed from each location = $127 \div 3 = 42$ remainder $1$

Since we cannot divide 127 equally among 3 locations, we need to reconsider the problem.

Let me solve this correctly:

Step 1: Remove the damaged supplies from the total first.
Undamaged supplies = $456 - 127 = 329$

Step 2: Distribute the undamaged supplies equally among the 3 locations.
Supplies per location = $329 \div 3 = 109$ remainder $2$

Since we need to distribute equally, each location receives 109 undamaged cleaning supplies, with 2 supplies remaining.

\section*{Question 6}
\textbf{Metadata}

\begin{itemize}
  \item Question ID: P3-WNDiv3d1d\_P1-WNMul2nd\_sonnet4\_Services\_01
  \item Primary KC: WHOLE NUMBERS | Division | dividing whole numbers up to 3 digits by 1 digit
  \item Secondary KC: WHOLE NUMBERS | Multiplication | multiplying whole numbers
  \item Topic: Services such as installation, maintenance, repairing, cleaning, laundry, hotel, retail, e-commerce, streaming services, digital services etc.
  \item Grade: Primary 3
\end{itemize}

\textbf{Solution}

Step 1: Find the total amount of money the company earned.
Total earnings = Number of rooms $\times$ Price per room
Total earnings = $456 \times 8 = 3648$
The company earned \textdollar3648 in total.

Step 2: Find how much each team earned for the company.
Earnings per team = Total earnings $\div$ Number of teams
Earnings per team = $3648 \div 3 = 1216$

Therefore, each cleaning team earned \textdollar1216 for the company.

\section*{Question 7}
\textbf{Metadata}

\begin{itemize}
  \item Question ID: P3-FrAddRl12\_P2-FrCmp\_sonnet4\_Services\_01
  \item Primary KC: FRACTIONS | Addition | adding two related fractions within one whole with denominators of given fractions not exceeding 12
  \item Secondary KC: FRACTIONS | Comparison and ordering | comparing and ordering fractions
  \item Topic: Services such as installation, maintenance, repairing, cleaning, laundry, hotel, retail, e-commerce, streaming services, digital services etc.
  \item Grade: Primary 3
\end{itemize}

\textbf{Solution}

Step 1: Find the total fraction of washing orders Sarah completed.
Sarah completed $\frac{2}{8}$ in the morning and $\frac{3}{8}$ after lunch.
Total fraction Sarah completed = $\frac{2}{8} + \frac{3}{8} = \frac{5}{8}$

Step 2: Find the total fraction of washing orders Ben completed.
Ben completed $\frac{4}{8}$ in the morning and $\frac{1}{8}$ in the afternoon.
Total fraction Ben completed = $\frac{4}{8} + \frac{1}{8} = \frac{5}{8}$

Step 3: Compare the fractions to determine who completed more.
Sarah completed $\frac{5}{8}$ of her orders.
Ben completed $\frac{5}{8}$ of his orders.
Since $\frac{5}{8} = \frac{5}{8}$, both Sarah and Ben completed the same fraction of their washing orders.

\section*{Question 8}
\textbf{Metadata}

\begin{itemize}
  \item Question ID: P3-FrSubRl12\_P2-FrAdd2nd\_sonnet4\_Services\_01
  \item Primary KC: FRACTIONS | Subtraction | subtracting two related fractions within one whole with denominators of given fractions not exceeding 12
  \item Secondary KC: FRACTIONS | Addition | adding fractions
  \item Topic: Services such as installation, maintenance, repairing, cleaning, laundry, hotel, retail, e-commerce, streaming services, digital services etc.
  \item Grade: Primary 3
\end{itemize}

\textbf{Solution}

Step 1: Find the total amount of towels Sarah had after adding more towels.

Sarah started with $\frac{5}{6}$ of a basket and added $\frac{1}{4}$ of a basket.

To add $\frac{5}{6} + \frac{1}{4}$, I need a common denominator.

The least common multiple of 6 and 4 is 12.

$\frac{5}{6} = \frac{5 \times 2}{6 \times 2} = \frac{10}{12}$

$\frac{1}{4} = \frac{1 \times 3}{4 \times 3} = \frac{3}{12}$

$\frac{10}{12} + \frac{3}{12} = \frac{13}{12}$

So Sarah had $\frac{13}{12}$ baskets of towels.

Step 2: Subtract the amount picked up by the hotel customer.

The hotel customer picked up $\frac{7}{8}$ of a basket.

I need to subtract $\frac{7}{8}$ from $\frac{13}{12}$.

To subtract $\frac{13}{12} - \frac{7}{8}$, I need a common denominator.

The least common multiple of 12 and 8 is 24.

$\frac{13}{12} = \frac{13 \times 2}{12 \times 2} = \frac{26}{24}$

$\frac{7}{8} = \frac{7 \times 3}{8 \times 3} = \frac{21}{24}$

$\frac{26}{24} - \frac{21}{24} = \frac{5}{24}$

Therefore, Sarah has $\frac{5}{24}$ of a basket of towels left at the end of the day.

\section*{Question 9}
\textbf{Metadata}

\begin{itemize}
  \item Question ID: P4-WNMul4d1d\_P1-WNAdd2nd\_sonnet4\_Services\_01
  \item Primary KC: WHOLE NUMBERS | Multiplication | multiplying whole numbers up to 4 digits by 1 digit or up to 3 digits by 2 digits
  \item Secondary KC: WHOLE NUMBERS | Addition | adding whole numbers
  \item Topic: Services such as installation, maintenance, repairing, cleaning, laundry, hotel, retail, e-commerce, streaming services, digital services etc.
  \item Grade: Primary 4
\end{itemize}

\textbf{Solution}

Step 1: Find the total number of hours worked.
Number of houses = 15
Hours per house = 4
Total hours = $15 \times 4 = 60$ hours

Step 2: Calculate the total cleaning fees.
Cost per hour = \textdollar23
Total cleaning fees = $60 \times 23 = \textdollar1380$

Step 3: Add the service fees to find the total earnings.
Total cleaning fees = \textdollar1380
Service fees = \textdollar180
Total earnings = $\textdollar1380 + \textdollar180 = \textdollar1560$

Therefore, the cleaning company earned \textdollar1560 in total last month.

\section*{Question 10}
\textbf{Metadata}

\begin{itemize}
  \item Question ID: P4-WNDiv4d1d\_P1-WNCmp\_sonnet4\_Services\_01
  \item Primary KC: WHOLE NUMBERS | Division | dividing whole numbers up to 4 digits by 1 digit
  \item Secondary KC: WHOLE NUMBERS | Comparison and ordering | comparing and ordering whole numbers
  \item Topic: Services such as installation, maintenance, repairing, cleaning, laundry, hotel, retail, e-commerce, streaming services, digital services etc.
  \item Grade: Primary 4
\end{itemize}

\textbf{Solution}

To find how many repairs each technician completed each day, I need to divide the total repairs by 6 technicians for each day.

Monday: $2856 \div 6$
$2856 \div 6 = 476$
Each technician completed 476 repairs on Monday.

Tuesday: $3024 \div 6$
$3024 \div 6 = 504$
Each technician completed 504 repairs on Tuesday.

Wednesday: $2448 \div 6$
$2448 \div 6 = 408$
Each technician completed 408 repairs on Wednesday.

To find which day had the highest number of repairs per technician, I need to compare: 476, 504, and 408.

Comparing the three numbers:
$408 < 476 < 504$

Therefore, Tuesday had the highest number of repairs per technician with 504 repairs.

\section*{Question 11}
\textbf{Metadata}

\begin{itemize}
  \item Question ID: P4-WNDiv4d1d\_P1-WNMul2nd\_sonnet4\_Services\_01
  \item Primary KC: WHOLE NUMBERS | Division | dividing whole numbers up to 4 digits by 1 digit
  \item Secondary KC: WHOLE NUMBERS | Multiplication | multiplying whole numbers
  \item Topic: Services such as installation, maintenance, repairing, cleaning, laundry, hotel, retail, e-commerce, streaming services, digital services etc.
  \item Grade: Primary 4
\end{itemize}

\textbf{Solution}

To find how many hours each technician worked, I need to divide the total hours by the number of technicians.

Total hours worked = 1248 hours
Number of technicians = 6

Hours per technician = 1248 ÷ 6

Let me solve this step by step:
1248 ÷ 6 = 208

To check my answer: 208 × 6 = 1248 ✓

Therefore, each technician worked 208 hours last week.

\section*{Question 12}
\textbf{Metadata}

\begin{itemize}
  \item Question ID: P4-FrAddU12\_P3-FrSmp\_sonnet4\_Services\_01
  \item Primary KC: FRACTIONS | Addition | adding unlike fractions with two different denominators not exceeding 12
  \item Secondary KC: FRACTIONS | Simplifying | expressing a fraction in its simplest form
  \item Topic: Services such as installation, maintenance, repairing, cleaning, laundry, hotel, retail, e-commerce, streaming services, digital services etc.
  \item Grade: Primary 4
\end{itemize}

\textbf{Solution}

To find the total fraction of washing orders Sarah completed, I need to add the fractions from the morning and afternoon.

Morning: $\frac{2}{3}$ of the orders
Afternoon: $\frac{1}{4}$ of the orders

To add $\frac{2}{3} + \frac{1}{4}$, I need to find a common denominator.

The denominators are 3 and 4.
The least common multiple of 3 and 4 is 12.

Converting to equivalent fractions with denominator 12:
$\frac{2}{3} = \frac{2 \times 4}{3 \times 4} = \frac{8}{12}$

$\frac{1}{4} = \frac{1 \times 3}{4 \times 3} = \frac{3}{12}$

Now I can add the fractions:
$\frac{8}{12} + \frac{3}{12} = \frac{8 + 3}{12} = \frac{11}{12}$

To check if $\frac{11}{12}$ is in its simplest form, I need to find the highest common factor of 11 and 12.
Factors of 11: 1, 11
Factors of 12: 1, 2, 3, 4, 6, 12
The highest common factor is 1.

Since the highest common factor is 1, $\frac{11}{12}$ is already in its simplest form.

Therefore, Sarah completed $\frac{11}{12}$ of the washing orders on Monday.

\section*{Question 13}
\textbf{Metadata}

\begin{itemize}
  \item Question ID: P4-FrSubU12\_P2-FrCmp\_sonnet4\_Services\_01
  \item Primary KC: FRACTIONS | Subtraction | subtracting unlike fractions with two different denominators not exceeding 12
  \item Secondary KC: FRACTIONS | Comparison and ordering | comparing and ordering fractions
  \item Topic: Services such as installation, maintenance, repairing, cleaning, laundry, hotel, retail, e-commerce, streaming services, digital services etc.
  \item Grade: Primary 4
\end{itemize}

\textbf{Solution}

To find how much more tasks Sarah completed on Monday than on Tuesday, I need to subtract the fractions: $\frac{5}{6} - \frac{3}{4}$.

Since the denominators are different (6 and 4), I need to find a common denominator.
The least common multiple of 6 and 4 is 12.

Converting to equivalent fractions:
$\frac{5}{6} = \frac{5 \times 2}{6 \times 2} = \frac{10}{12}$
$\frac{3}{4} = \frac{3 \times 3}{4 \times 3} = \frac{9}{12}$

Now I can subtract:
$\frac{10}{12} - \frac{9}{12} = \frac{10-9}{12} = \frac{1}{12}$

To determine which day Sarah completed more tasks, I compare the fractions:
$\frac{10}{12}$ and $\frac{9}{12}$

Since $\frac{10}{12} > \frac{9}{12}$, this means $\frac{5}{6} > \frac{3}{4}$.

Therefore, Sarah completed $\frac{1}{12}$ more of her tasks on Monday than on Tuesday, and she completed more tasks on Monday.

\section*{Question 14}
\textbf{Metadata}

\begin{itemize}
  \item Question ID: P4-DcAdd2d\_P4-DcCnv2Fr\_sonnet4\_Services\_01
  \item Primary KC: DECIMALS | Addition | adding decimals (up to 2 decimal places)
  \item Secondary KC: DECIMALS | Conversion from decimals to fraction | expressing decimals as fractions
  \item Topic: Services such as installation, maintenance, repairing, cleaning, laundry, hotel, retail, e-commerce, streaming services, digital services etc.
  \item Grade: Primary 4
\end{itemize}

\textbf{Solution}

First, I need to find Sarah's total earnings by adding all the decimal amounts.

Monday's earnings: \textdollar12.75 + \textdollar18.50
Aligning the decimal points:
$12.75 + 18.50 = 31.25$

Tuesday's earnings: \textdollar15.25

Total earnings for two days:
$31.25 + 15.25 = 46.50$

So Sarah earned \textdollar46.50 in total.

Now I need to express \textdollar46.50 as a fraction in simplest form.

\textdollar46.50 = \textdollar$\frac{4650}{100}$

To simplify this fraction, I need to find the greatest common divisor of 4650 and 100.
$4650 = 100 \times 46 + 50$
$100 = 50 \times 2 + 0$

So the GCD is 50.

$\frac{4650}{100} = \frac{4650 \div 50}{100 \div 50} = \frac{93}{2}$

Therefore, Sarah's total earnings is \textdollar$\frac{93}{2}$ or \textdollar46.50.

\section*{Question 15}
\textbf{Metadata}

\begin{itemize}
  \item Question ID: P4-DcAdd2d\_P4-DcRnd3d\_sonnet4\_Services\_01
  \item Primary KC: DECIMALS | Addition | adding decimals (up to 2 decimal places)
  \item Secondary KC: DECIMALS | Rounding | rounding decimals up to 3 decimal places to the nearest whole number, 1 decimal place and 2 decimal places 
  \item Topic: Services such as installation, maintenance, repairing, cleaning, laundry, hotel, retail, e-commerce, streaming services, digital services etc.
  \item Grade: Primary 4
\end{itemize}

\textbf{Solution}

Step 1: Add up Sarah's earnings for Monday.
Monday earnings = \textdollar23.75 + \textdollar18.60 = \textdollar42.35

Step 2: Add Tuesday's earnings to find the total earnings.
Total earnings = \textdollar42.35 + \textdollar31.25 = \textdollar73.60

Step 3: Round the total earnings to the nearest whole number.
\textdollar73.60 rounded to the nearest whole number = \textdollar74

Therefore, Sarah's total earnings for the two days, rounded to the nearest whole number, is \textdollar74.

\section*{Question 16}
\textbf{Metadata}

\begin{itemize}
  \item Question ID: P4-DcSub2d\_P4-DcAdd2nd\_sonnet4\_Services\_01
  \item Primary KC: DECIMALS | Subtraction | subtracting decimals (up to 2 decimal places)
  \item Secondary KC: DECIMALS | Addition | adding decimals
  \item Topic: Services such as installation, maintenance, repairing, cleaning, laundry, hotel, retail, e-commerce, streaming services, digital services etc.
  \item Grade: Primary 4
\end{itemize}

\textbf{Solution}

Step 1: Find the total cost before applying the discount.
Cost of basic cleaning package = \textdollar45.75
Cost of carpet cleaning = \textdollar12.50
Total cost before discount = \textdollar45.75 + \textdollar12.50 = \textdollar58.25

Step 2: Apply the discount to find the final amount paid.
Discount amount = \textdollar8.25
Final amount paid = \textdollar58.25 - \textdollar8.25 = \textdollar50.00

Therefore, Sarah's family paid \textdollar50.00 for the cleaning service.

\section*{Question 17}
\textbf{Metadata}

\begin{itemize}
  \item Question ID: P4-DcDiv2d1d\_P4-DcCnv2Fr\_sonnet4\_Services\_01
  \item Primary KC: DECIMALS | Division | dividing decimals (up to 2 decimal places) by a 1-digit whole number
  \item Secondary KC: DECIMALS | Conversion from decimals to fraction | expressing decimals as fractions
  \item Topic: Services such as installation, maintenance, repairing, cleaning, laundry, hotel, retail, e-commerce, streaming services, digital services etc.
  \item Grade: Primary 4
\end{itemize}

\textbf{Solution}

To find how many houses Sarah cleaned windows for, I need to divide the total earnings by the cost per house.

Total earnings = \textdollar62.25
Cost per house = \textdollar12.45

Number of houses = $62.25 \div 12.45$

To divide $62.25$ by $12.45$, I can rewrite this as:
$62.25 \div 12.45 = \frac{62.25}{12.45}$

To make the division easier, I'll multiply both numbers by $100$ to remove the decimal places:
$\frac{62.25 \times 100}{12.45 \times 100} = \frac{6225}{1245}$

Now I'll divide:
$6225 \div 1245 = 5$

So Sarah cleaned windows for $5$ houses.

To express $5$ as a fraction: $5 = \frac{5}{1}$

Therefore, Sarah cleaned windows for $\frac{5}{1}$ or $5$ houses.

\section*{Question 18}
\textbf{Metadata}

\begin{itemize}
  \item Question ID: P4-DcDiv2d1d\_P4-DcSub2nd\_sonnet4\_Services\_01
  \item Primary KC: DECIMALS | Division | dividing decimals (up to 2 decimal places) by a 1-digit whole number
  \item Secondary KC: DECIMALS | Subtraction | subtracting decimals
  \item Topic: Services such as installation, maintenance, repairing, cleaning, laundry, hotel, retail, e-commerce, streaming services, digital services etc.
  \item Grade: Primary 4
\end{itemize}

\textbf{Solution}

Step 1: Find the cost of washing 1 car.
Cost for washing 3 cars = \textdollar12.75
Cost for washing 1 car = \textdollar12.75 \div 3 = \textdollar4.25

Step 2: Find how much money Sarah will have left.
Money Sarah has = \textdollar20.00
Cost of washing 1 car = \textdollar4.25
Money left = \textdollar20.00 - \textdollar4.25 = \textdollar15.75

Therefore, Sarah will have \textdollar15.75 left after paying for the car wash.

\section*{Question 19}
\textbf{Metadata}

\begin{itemize}
  \item Question ID: P5-FrAddMix\_P3-FrSmp\_sonnet4\_Services\_01
  \item Primary KC: FRACTIONS | Addition | adding mixed numbers
  \item Secondary KC: FRACTIONS | Simplifying | expressing a fraction in its simplest form
  \item Topic: Services such as installation, maintenance, repairing, cleaning, laundry, hotel, retail, e-commerce, streaming services, digital services etc.
  \item Grade: Primary 5
\end{itemize}

\textbf{Solution}

To find the total time, I need to add all three mixed numbers: $2\frac{3}{8} + 1\frac{5}{6} + 3\frac{1}{4}$.

First, I'll convert each mixed number to an improper fraction:
$2\frac{3}{8} = \frac{2 \times 8 + 3}{8} = \frac{19}{8}$
$1\frac{5}{6} = \frac{1 \times 6 + 5}{6} = \frac{11}{6}$
$3\frac{1}{4} = \frac{3 \times 4 + 1}{4} = \frac{13}{4}$

Next, I need to find a common denominator. The LCM of 8, 6, and 4 is 24.
$\frac{19}{8} = \frac{19 \times 3}{8 \times 3} = \frac{57}{24}$
$\frac{11}{6} = \frac{11 \times 4}{6 \times 4} = \frac{44}{24}$
$\frac{13}{4} = \frac{13 \times 6}{4 \times 6} = \frac{78}{24}$

Now I can add the fractions:
$\frac{57}{24} + \frac{44}{24} + \frac{78}{24} = \frac{57 + 44 + 78}{24} = \frac{179}{24}$

To convert back to a mixed number: $179 \div 24 = 7$ remainder $11$
So $\frac{179}{24} = 7\frac{11}{24}$

I need to check if $\frac{11}{24}$ is in its simplest form. Since 11 is prime and doesn't divide 24, the GCD of 11 and 24 is 1.

Therefore, the total time spent is $7\frac{11}{24}$ hours.

\section*{Question 20}
\textbf{Metadata}

\begin{itemize}
  \item Question ID: P5-FrSubMix\_P2-FrAdd2nd\_sonnet4\_Services\_01
  \item Primary KC: FRACTIONS | Subtraction | subtracting mixed numbers
  \item Secondary KC: FRACTIONS | Addition | adding fractions
  \item Topic: Services such as installation, maintenance, repairing, cleaning, laundry, hotel, retail, e-commerce, streaming services, digital services etc.
  \item Grade: Primary 5
\end{itemize}

\textbf{Solution}

First, I need to find the total time spent cleaning by adding all three jobs.

Time spent cleaning = $2\frac{3}{4} + 1\frac{5}{6} + \frac{7}{8}$

To add these fractions, I need a common denominator. The LCM of 4, 6, and 8 is 24.

Converting to equivalent fractions:
$2\frac{3}{4} = 2\frac{18}{24}$
$1\frac{5}{6} = 1\frac{20}{24}$
$\frac{7}{8} = \frac{21}{24}$

Adding: $2\frac{18}{24} + 1\frac{20}{24} + \frac{21}{24} = 3\frac{59}{24}$

Since $\frac{59}{24} = 2\frac{11}{24}$, the total time is $3 + 2\frac{11}{24} = 5\frac{11}{24}$ hours.

Now I need to compare this with the scheduled time of $6\frac{1}{3}$ hours.

Since $5\frac{11}{24} < 6\frac{1}{3}$, they did not exceed their scheduled time.

To find how much time they had remaining:
$6\frac{1}{3} - 5\frac{11}{24}$

Converting to common denominator 24:
$6\frac{1}{3} = 6\frac{8}{24}$

$6\frac{8}{24} - 5\frac{11}{24} = \frac{21}{24} = \frac{7}{8}$

Therefore, the cleaning service spent $5\frac{11}{24}$ hours cleaning in total. They did not exceed their scheduled time and finished $\frac{7}{8}$ hours early.

\section*{Question 21}
\textbf{Metadata}

\begin{itemize}
  \item Question ID: P5-FrMulImN\_P2-FrSub2nd\_sonnet4\_Services\_01
  \item Primary KC: FRACTIONS | Multiplication | multiplying a proper/improper fraction and a whole number
  \item Secondary KC: FRACTIONS | Subtraction | subtracting fractions
  \item Topic: Services such as installation, maintenance, repairing, cleaning, laundry, hotel, retail, e-commerce, streaming services, digital services etc.
  \item Grade: Primary 5
\end{itemize}

\textbf{Solution}

Step 1: Calculate the cost for Monday's cleaning.
Cost on Monday = $12 \times \frac{5}{6}$
$= \frac{12 \times 5}{6}$
$= \frac{60}{6}$
$= \textdollar10$

Step 2: Calculate the cost for Tuesday's cleaning.
Cost on Tuesday = $12 \times \frac{2}{3}$
$= \frac{12 \times 2}{3}$
$= \frac{24}{3}$
$= \textdollar8$

Step 3: Find how much more Sarah paid on Monday than on Tuesday.
Difference = Cost on Monday - Cost on Tuesday
$= \textdollar10 - \textdollar8$
$= \textdollar2$

Therefore, Sarah paid \textdollar2 more on Monday than on Tuesday.

\section*{Question 22}
\textbf{Metadata}

\begin{itemize}
  \item Question ID: P5-FrMulPIm\_P2-FrSub2nd\_sonnet4\_Services\_01
  \item Primary KC: FRACTIONS | Multiplication | multiplying a proper fraction and a proper/improper fractions
  \item Secondary KC: FRACTIONS | Subtraction | subtracting fractions
  \item Topic: Services such as installation, maintenance, repairing, cleaning, laundry, hotel, retail, e-commerce, streaming services, digital services etc.
  \item Grade: Primary 5
\end{itemize}

\textbf{Solution}

Step 1: Find the number of rooms cleaned on Monday.
Number of rooms cleaned = $\frac{2}{3} \times 18 = \frac{2 \times 18}{3} = \frac{36}{3} = 12$ rooms

Step 2: Find the total amount of cleaning solution used.
Cleaning solution used per room = $\frac{3}{4}$ bottle
Total cleaning solution used = $\frac{3}{4} \times 12 = \frac{3 \times 12}{4} = \frac{36}{4} = 9$ bottles

Step 3: Find how many bottles are left.
Bottles at start = 5 bottles
Bottles used = 9 bottles
Since 9 > 5, we need to convert to mixed numbers for clarity:
Bottles left = $5 - 9 = -4$ bottles

This means the hotel needs 4 more bottles of cleaning solution to complete the cleaning, so they have 0 bottles left and need 4 additional bottles.

Therefore, the hotel has used all 5 bottles and still needs 4 more bottles to complete the cleaning of 12 rooms.

\section*{Question 23}
\textbf{Metadata}

\begin{itemize}
  \item Question ID: P5-FrMulPIm\_P5-FrCnv2Dc\_sonnet4\_Services\_01
  \item Primary KC: FRACTIONS | Multiplication | multiplying a proper fraction and a proper/improper fractions
  \item Secondary KC: FRACTIONS | Conversion to decimals | expressing fractions as decimals
  \item Topic: Services such as installation, maintenance, repairing, cleaning, laundry, hotel, retail, e-commerce, streaming services, digital services etc.
  \item Grade: Primary 5
\end{itemize}

\textbf{Solution}

To find the total earnings, I need to calculate the earnings for each day and then add them together.

Monday's earnings:
Time worked = $\frac{2}{3}$ hour
Rate = \textdollar36 per hour
Earnings = $\frac{2}{3} \times 36$
$= \frac{2 \times 36}{3}$
$= \frac{72}{3}$
$= 24$
So Monday's earnings = \textdollar24

Tuesday's earnings:
Time worked = $\frac{5}{4}$ hours
Rate = \textdollar36 per hour
Earnings = $\frac{5}{4} \times 36$
$= \frac{5 \times 36}{4}$
$= \frac{180}{4}$
$= 45$
So Tuesday's earnings = \textdollar45

Total earnings = Monday's earnings + Tuesday's earnings
$= 24 + 45$
$= 69$

Therefore, the company earned \textdollar69.00 in total over the two days.

\section*{Question 24}
\textbf{Metadata}

\begin{itemize}
  \item Question ID: P5-FrMulMixN\_P2-FrCmp\_sonnet4\_Services\_01
  \item Primary KC: FRACTIONS | Multiplication | multiplying a mixed number and a whole number
  \item Secondary KC: FRACTIONS | Comparison and ordering | comparing and ordering fractions
  \item Topic: Services such as installation, maintenance, repairing, cleaning, laundry, hotel, retail, e-commerce, streaming services, digital services etc.
  \item Grade: Primary 5
\end{itemize}

\textbf{Solution}

(a) Calculate the total cost for each customer:

For Customer A:
Cost = $2\frac{1}{4} \times 18$
$= \frac{9}{4} \times 18$
$= \frac{9 \times 18}{4}$
$= \frac{162}{4}$
$= 40\frac{1}{2}$
Customer A's cost = \textdollar40.50

For Customer B:
Cost = $1\frac{5}{6} \times 18$
$= \frac{11}{6} \times 18$
$= \frac{11 \times 18}{6}$
$= \frac{198}{6}$
$= 33$
Customer B's cost = \textdollar33

For Customer C:
Cost = $2\frac{1}{3} \times 18$
$= \frac{7}{3} \times 18$
$= \frac{7 \times 18}{3}$
$= \frac{126}{3}$
$= 42$
Customer C's cost = \textdollar42

(b) To arrange from lowest to highest cost, we compare:
\textdollar33 < \textdollar40.50 < \textdollar42

Therefore, the order from lowest cost to highest cost is:
Customer B, Customer A, Customer C

\section*{Question 25}
\textbf{Metadata}

\begin{itemize}
  \item Question ID: P5-FrMulMixN\_P3-FrSmp\_sonnet4\_Services\_01
  \item Primary KC: FRACTIONS | Multiplication | multiplying a mixed number and a whole number
  \item Secondary KC: FRACTIONS | Simplifying | expressing a fraction in its simplest form
  \item Topic: Services such as installation, maintenance, repairing, cleaning, laundry, hotel, retail, e-commerce, streaming services, digital services etc.
  \item Grade: Primary 5
\end{itemize}

\textbf{Solution}

To find the total number of rooms Sarah cleaned, I need to multiply the number of rooms she cleans per hour by the number of hours she worked.

Number of rooms cleaned per hour = $2\frac{1}{4}$
Number of hours worked = $6$

Total rooms cleaned = $2\frac{1}{4} \times 6$

First, I'll convert the mixed number to an improper fraction:
$2\frac{1}{4} = \frac{2 \times 4 + 1}{4} = \frac{9}{4}$

Now I can multiply:
$\frac{9}{4} \times 6 = \frac{9 \times 6}{4} = \frac{54}{4}$

Next, I need to simplify this fraction by finding the GCD of 54 and 4:
Factors of 54: 1, 2, 3, 6, 9, 18, 27, 54
Factors of 4: 1, 2, 4
GCD(54, 4) = 2

$\frac{54}{4} = \frac{54 \div 2}{4 \div 2} = \frac{27}{2}$

Finally, I'll convert this improper fraction back to a mixed number:
$\frac{27}{2} = 13\frac{1}{2}$

Therefore, Sarah cleaned $13\frac{1}{2}$ rooms on Monday.

\section*{Question 26}
\textbf{Metadata}

\begin{itemize}
  \item Question ID: P5-FrMulMixN\_P5-FrCnv2Dc\_sonnet4\_Services\_01
  \item Primary KC: FRACTIONS | Multiplication | multiplying a mixed number and a whole number
  \item Secondary KC: FRACTIONS | Conversion to decimals | expressing fractions as decimals
  \item Topic: Services such as installation, maintenance, repairing, cleaning, laundry, hotel, retail, e-commerce, streaming services, digital services etc.
  \item Grade: Primary 5
\end{itemize}

\textbf{Solution}

To find the total cost, I need to multiply the hourly rate by the number of hours worked.

Given information:
- Hourly rate = \textdollar12
- Time worked = $2\frac{3}{4}$ hours

Step 1: Multiply the mixed number by the whole number.
$2\frac{3}{4} \times 12$

First, I'll convert the mixed number to an improper fraction:
$2\frac{3}{4} = \frac{2 \times 4 + 3}{4} = \frac{8 + 3}{4} = \frac{11}{4}$

Now I can multiply:
$\frac{11}{4} \times 12 = \frac{11 \times 12}{4} = \frac{132}{4} = 33$

Step 2: Express the result as a decimal.
Since $33 = \frac{33}{1}$, the decimal form is $33.00$

Therefore, Sarah paid \textdollar33.00 for the cleaning service.

\section*{Question 27}
\textbf{Metadata}

\begin{itemize}
  \item Question ID: P5-DcMul3dK\_P4-DcRnd3d\_sonnet4\_Services\_01
  \item Primary KC: DECIMALS | Multiplication | multiplying decimals (up to 3 decimal places) by 10, 100, 1000 and their multiples
  \item Secondary KC: DECIMALS | Rounding | rounding decimals up to 3 decimal places to the nearest whole number, 1 decimal place and 2 decimal places 
  \item Topic: Services such as installation, maintenance, repairing, cleaning, laundry, hotel, retail, e-commerce, streaming services, digital services etc.
  \item Grade: Primary 5
\end{itemize}

\textbf{Solution}

Step 1: Find the actual rate per square meter after applying the bulk discount.
Standard rate = \textdollar0.125 per square meter
First, multiply by $1000$: $0.125 \times 1000 = 125$
Then divide by $1250$: $125 \div 1250 = 0.1$
Actual rate = \textdollar0.1 per square meter

Step 2: Calculate the total cost before tax.
Total area = $2400$ square meters
Total cost = $0.1 \times 2400 = 240$
Total cost = \textdollar240.000

Step 3: Round to the nearest cent (2 decimal places).
\textdollar240.000 rounded to 2 decimal places = \textdollar240.00

Therefore, the total cost before tax is \textdollar240.00.

\section*{Question 28}
\textbf{Metadata}

\begin{itemize}
  \item Question ID: P5-DcDiv3dK\_P4-DcCnv2Fr\_sonnet4\_Services\_01
  \item Primary KC: DECIMALS | Division | dividing decimals (up to 3 decimal places) by 10, 100, 1000 and their multiples
  \item Secondary KC: DECIMALS | Conversion from decimals to fraction | expressing decimals as fractions
  \item Topic: Services such as installation, maintenance, repairing, cleaning, laundry, hotel, retail, e-commerce, streaming services, digital services etc.
  \item Grade: Primary 5
\end{itemize}

\textbf{Solution}

(a) To find the special daily rate, we divide \textdollar75.00 by 1000.
$75.00 \div 1000 = 0.075$
The special daily rate is \textdollar0.075.

(b) To express $0.075$ as a fraction:
$0.075 = \frac{75}{1000}$
To simplify, we find the highest common factor of 75 and 1000.
$75 = 3 \times 5^2$
$1000 = 2^3 \times 5^3$
HCF$(75, 1000) = 5^2 = 25$
$\frac{75}{1000} = \frac{75 \div 25}{1000 \div 25} = \frac{3}{40}$
Therefore, the daily rate as a fraction is $\frac{3}{40}$.

(c) Regular cost for 6 months: $6 \times \textdollar12.50 = \textdollar75.00$
Promotion cost for 6 months: \textdollar75.00
Savings: $\textdollar75.00 - \textdollar75.00 = \textdollar0.00$
Sarah would save \textdollar0.00, meaning the promotion price equals the regular price in this case.

\section*{Question 29}
\textbf{Metadata}

\begin{itemize}
  \item Question ID: P5-RtFndR\_P2-DcCnvN2D\_sonnet4\_Services\_01
  \item Primary KC: RATE | Finding rate | finding rate given total amount and number of units
  \item Secondary KC: DECIMALS | Conversion to larger units | converting a measurement from a smaller unit to a larger unit in decimal form
  \item Topic: Services such as installation, maintenance, repairing, cleaning, laundry, hotel, retail, e-commerce, streaming services, digital services etc.
  \item Grade: Primary 5
\end{itemize}

\textbf{Solution}

To find the hourly rate, I need to first convert the time from minutes to hours, then calculate the rate per hour.

Step 1: Convert 240 minutes to hours.
Since there are 60 minutes in 1 hour:
240 minutes = $240 \div 60 = 4$ hours

Step 2: Find the hourly rate.
Hourly rate = Total cost $\div$ Number of hours
Hourly rate = $\textdollar288 \div 4 = \textdollar72$

Therefore, the cleaning service charges \textdollar72 per hour.

\section*{Question 30}
\textbf{Metadata}

\begin{itemize}
  \item Question ID: P5-RtFndR\_P2-DcCnvD2N\_sonnet4\_Services\_01
  \item Primary KC: RATE | Finding rate | finding rate given total amount and number of units
  \item Secondary KC: DECIMALS | Conversion to smaller units | converting a measurement from a larger unit in decimal form to a smaller unit
  \item Topic: Services such as installation, maintenance, repairing, cleaning, laundry, hotel, retail, e-commerce, streaming services, digital services etc.
  \item Grade: Primary 5
\end{itemize}

\textbf{Solution}

Step 1: Find the rate per square meter.
Total charge = \textdollar250
Total area cleaned = $12.5$ square meters
Rate per square meter = \textdollar250 ÷ $12.5$ = \textdollar20 per square meter

Step 2: Convert the rate to cents per square centimeter.
First, convert dollars to cents: \textdollar20 = $20 \times 100 = 2000$ cents
Next, convert square meters to square centimeters: $1$ square meter = $100 \times 100 = 10000$ square centimeters
Therefore, rate per square centimeter = $2000 ÷ 10000 = 0.2$ cents per square centimeter

The company charges $0.2$ cents per square centimeter.

\section*{Question 31}
\textbf{Metadata}

\begin{itemize}
  \item Question ID: P5-RtFndT\_P2-DcCnvN2D\_sonnet4\_Services\_01
  \item Primary KC: RATE | Finding total amount | finding total amount, given rate and number of units
  \item Secondary KC: DECIMALS | Conversion to larger units | converting a measurement from a smaller unit to a larger unit in decimal form
  \item Topic: Services such as installation, maintenance, repairing, cleaning, laundry, hotel, retail, e-commerce, streaming services, digital services etc.
  \item Grade: Primary 5
\end{itemize}

\textbf{Solution}

Step 1: Convert the time from minutes to hours.
$180 \text{ minutes} = 180 \div 60 = 3 \text{ hours}$

Step 2: Find the total cost using the rate.
Rate = \textdollar12.50 per hour
Time = 3 hours
Total cost = Rate $\times$ Time
Total cost = \textdollar12.50 $\times$ 3 = \textdollar37.50

Therefore, Mrs. Tan paid \textdollar37.50 for the cleaning service.

\section*{Question 32}
\textbf{Metadata}

\begin{itemize}
  \item Question ID: P5-RtFndU\_P2-DcCnvD2N\_sonnet4\_Services\_01
  \item Primary KC: RATE | Finding number of unit | finding number of units given rate and total amount
  \item Secondary KC: DECIMALS | Conversion to smaller units | converting a measurement from a larger unit in decimal form to a smaller unit
  \item Topic: Services such as installation, maintenance, repairing, cleaning, laundry, hotel, retail, e-commerce, streaming services, digital services etc.
  \item Grade: Primary 5
\end{itemize}

\textbf{Solution}

To find how many minutes the cleaning service worked, I need to first find how many hours they worked, then convert the hours to minutes.

Step 1: Find the number of hours worked.
Rate = \textdollar12.50 per hour
Total amount = \textdollar43.75
Number of hours = Total amount ÷ Rate
Number of hours = \textdollar43.75 ÷ \textdollar12.50 = $3.5$ hours

Step 2: Convert hours to minutes.
$3.5$ hours = $3.5 \times 60$ minutes
$3.5 \times 60 = 210$ minutes

Therefore, the cleaning service worked for $210$ minutes at Mrs. Chen's house.

\section*{Question 33}
\textbf{Metadata}

\begin{itemize}
  \item Question ID: P6-FrDivPN\_P3-FrSmp\_sonnet4\_Services\_01
  \item Primary KC: FRACTIONS | Division | dividing a proper fraction by a whole number
  \item Secondary KC: FRACTIONS | Simplifying | expressing a fraction in its simplest form
  \item Topic: Services such as installation, maintenance, repairing, cleaning, laundry, hotel, retail, e-commerce, streaming services, digital services etc.
  \item Grade: Primary 6
\end{itemize}

\textbf{Solution}

To find how much carpet cleaner will be used for each job, I need to divide $\frac{3}{4}$ by 6.

Dividing a fraction by a whole number:
$\frac{3}{4} \div 6 = \frac{3}{4} \times \frac{1}{6} = \frac{3 \times 1}{4 \times 6} = \frac{3}{24}$

Now I need to express $\frac{3}{24}$ in its simplest form by finding the highest common factor (HCF) of 3 and 24.

Factors of 3: 1, 3
Factors of 24: 1, 2, 3, 4, 6, 8, 12, 24

The HCF of 3 and 24 is 3.

Dividing both numerator and denominator by 3:
$\frac{3}{24} = \frac{3 \div 3}{24 \div 3} = \frac{1}{8}$

Therefore, $\frac{1}{8}$ of a bottle of carpet cleaner will be used for each job.

\section*{Question 34}
\textbf{Metadata}

\begin{itemize}
  \item Question ID: P6-FrDivPP\_P2-FrCmp\_sonnet4\_Services\_01
  \item Primary KC: FRACTIONS | Division | dividing a whole number/proper fraction by a proper fraction
  \item Secondary KC: FRACTIONS | Comparison and ordering | comparing and ordering fractions
  \item Topic: Services such as installation, maintenance, repairing, cleaning, laundry, hotel, retail, e-commerce, streaming services, digital services etc.
  \item Grade: Primary 6
\end{itemize}

\textbf{Solution}

(a) To find how many apartment cleaning jobs Sarah can complete:
Number of apartment jobs = $\frac{3}{4} \div \frac{1}{6}$
$= \frac{3}{4} \times \frac{6}{1}$
$= \frac{3 \times 6}{4 \times 1}$
$= \frac{18}{4}$
$= 4\frac{1}{2}$

Since Sarah cannot complete half a job, she can complete $4$ apartment cleaning jobs.

(b) To find how many office cleaning jobs Sarah can complete:
Number of office jobs = $\frac{3}{4} \div \frac{1}{8}$
$= \frac{3}{4} \times \frac{8}{1}$
$= \frac{3 \times 8}{4 \times 1}$
$= \frac{24}{4}$
$= 6$

Sarah can complete $6$ office cleaning jobs.

(c) To compare which type allows more jobs:
Apartment jobs: $4$ jobs
Office jobs: $6$ jobs

Since $6 > 4$, office cleaning jobs allow Sarah to complete more jobs in her available time.

\section*{Question 35}
\textbf{Metadata}

\begin{itemize}
  \item Question ID: P6-FrDivPP\_P3-FrSmp\_sonnet4\_Services\_01
  \item Primary KC: FRACTIONS | Division | dividing a whole number/proper fraction by a proper fraction
  \item Secondary KC: FRACTIONS | Simplifying | expressing a fraction in its simplest form
  \item Topic: Services such as installation, maintenance, repairing, cleaning, laundry, hotel, retail, e-commerce, streaming services, digital services etc.
  \item Grade: Primary 6
\end{itemize}

\textbf{Solution}

To find how many offices can be cleaned, I need to divide the total amount of cleaning solution by the amount needed per office.

Total cleaning solution available = $6$ bottles
Cleaning solution needed per office = $\frac{3}{4}$ bottle

Number of offices = $6 \div \frac{3}{4}$

To divide by a fraction, I multiply by its reciprocal:
$6 \div \frac{3}{4} = 6 \times \frac{4}{3}$

$= \frac{6}{1} \times \frac{4}{3}$

$= \frac{6 \times 4}{1 \times 3}$

$= \frac{24}{3}$

To express this fraction in its simplest form, I need to find the highest common factor of 24 and 3.
Factors of 24: 1, 2, 3, 4, 6, 8, 12, 24
Factors of 3: 1, 3
Highest common factor = 3

$\frac{24}{3} = \frac{24 \div 3}{3 \div 3} = \frac{8}{1} = 8$

Therefore, the cleaning service company can clean $8$ offices.

\section*{Question 36}
\textbf{Metadata}

\begin{itemize}
  \item Question ID: P6-PcFndWN\_P1-WNSub2nd\_sonnet4\_Services\_01
  \item Primary KC: PERCENTAGE | Finding the whole | finding the whole given a part and the percentage
  \item Secondary KC: WHOLE NUMBERS | Subtraction | subtracting whole numbers
  \item Topic: Services such as installation, maintenance, repairing, cleaning, laundry, hotel, retail, e-commerce, streaming services, digital services etc.
  \item Grade: Primary 6
\end{itemize}

\textbf{Solution}

Step 1: Find the total number of repair requests Sarah received last month.

Given information:
- 84 completed repair jobs = 70\% of total requests
- Need to find: total number of requests

Let the total number of requests = $x$
$70\% \times x = 84$
$\frac{70}{100} \times x = 84$
$x = 84 \div \frac{70}{100}$
$x = 84 \times \frac{100}{70}$
$x = \frac{8400}{70} = 120$

So Sarah received 120 total repair requests last month.

Step 2: Find the number of incomplete jobs.

Number of incomplete jobs = Total requests - Completed jobs
Number of incomplete jobs = $120 - 84 = 36$

Therefore, Sarah could potentially finish 36 additional repair jobs if she orders the necessary parts.

\section*{Question 37}
\textbf{Metadata}

\begin{itemize}
  \item Question ID: P6-PcFndChg\_P1-WNSub2nd\_sonnet4\_Services\_01
  \item Primary KC: PERCENTAGE | Finding change | finding percentage increase/decrease
  \item Secondary KC: WHOLE NUMBERS | Subtraction | subtracting whole numbers
  \item Topic: Services such as installation, maintenance, repairing, cleaning, laundry, hotel, retail, e-commerce, streaming services, digital services etc.
  \item Grade: Primary 6
\end{itemize}

\textbf{Solution}

To find the percentage increase, I need to:

1) Find the increase in price by subtracting the original price from the new price:
Increase = New price - Original price
Increase = \textdollar207 - \textdollar180 = \textdollar27

2) Calculate the percentage increase using the formula:
Percentage increase = $\frac{\text{Increase}}{\text{Original price}} \times 100\%$

Percentage increase = $\frac{27}{180} \times 100\%$

Percentage increase = $0.15 \times 100\% = 15\%$

Therefore, the percentage increase in the repair service price is $15\%$.

\section*{Question 38}
\textbf{Metadata}

\begin{itemize}
  \item Question ID: P6-PcFndChg\_P1-WNMul2nd\_sonnet4\_Services\_01
  \item Primary KC: PERCENTAGE | Finding change | finding percentage increase/decrease
  \item Secondary KC: WHOLE NUMBERS | Multiplication | multiplying whole numbers
  \item Topic: Services such as installation, maintenance, repairing, cleaning, laundry, hotel, retail, e-commerce, streaming services, digital services etc.
  \item Grade: Primary 6
\end{itemize}

\textbf{Solution}

Step 1: Find the new service charge in February after the 20\% increase.
Percentage increase = 20\% of \textdollar45 = $\frac{20}{100} \times 45 = \textdollar9$
New charge in February = \textdollar45 + \textdollar9 = \textdollar54

Step 2: Find how much more money the shop earned per screen replacement in February.
Increase per screen replacement = \textdollar54 - \textdollar45 = \textdollar9

Step 3: Calculate the total revenue for January.
January revenue = $120 \times 45 = \textdollar5400$

Step 4: Calculate the total revenue for February.
February revenue = $85 \times 54 = \textdollar4590$

Step 5: Calculate the total revenue for both months.
Total revenue = \textdollar5400 + \textdollar4590 = \textdollar9990

Therefore, the shop earned \textdollar9 more per screen replacement in February compared to January, and the total revenue from screen replacements for both months was \textdollar9990.

\section*{Question 39}
\textbf{Metadata}

\begin{itemize}
  \item Question ID: P6-RoFndDvqWN\_P1-WNAdd2nd\_sonnet4\_Services\_01
  \item Primary KC: RATIO | Finding divided quantities | dividing a given quantity in a given ratio
  \item Secondary KC: WHOLE NUMBERS | Addition | adding whole numbers
  \item Topic: Services such as installation, maintenance, repairing, cleaning, laundry, hotel, retail, e-commerce, streaming services, digital services etc.
  \item Grade: Primary 6
\end{itemize}

\textbf{Solution}

Step 1: Find the total number of parts in the ratio.
The ratio is $3:4:7$
Total parts = $3 + 4 + 7 = 14$ parts

Step 2: Find the value of each part.
Total amount = \textdollar840
Value of each part = $\textdollar840 \div 14 = \textdollar60$

Step 3: Find each person's share.
Amy's share = $3 \times \textdollar60 = \textdollar180$
Ben's share = $4 \times \textdollar60 = \textdollar240$
Carlos's share = $7 \times \textdollar60 = \textdollar420$

Step 4: Find the total amount Amy and Ben have together.
Amy and Ben's total = $\textdollar180 + \textdollar240 = \textdollar420$

Therefore, Amy and Ben have \textdollar420 in total for purchasing the tools.

\section*{Question 40}
\textbf{Metadata}

\begin{itemize}
  \item Question ID: P6-RoFndRoWN\_P1-WNSub2nd\_sonnet4\_Services\_01
  \item Primary KC: RATIO | Finding ratio | finding the ratio of two or three given whole numbers
  \item Secondary KC: WHOLE NUMBERS | Subtraction | subtracting whole numbers
  \item Topic: Services such as installation, maintenance, repairing, cleaning, laundry, hotel, retail, e-commerce, streaming services, digital services etc.
  \item Grade: Primary 6
\end{itemize}

\textbf{Solution}

Step 1: Find the number of each type of device repaired in February.

Laptops repaired in February: $84 - 12 = 72$

Desktop computers repaired in February: $56 - 8 = 48$

Tablets repaired in February: $28 - 4 = 24$

Step 2: Find the ratio of laptops to desktop computers to tablets in February.

The ratio is $72 : 48 : 24$

Step 3: Simplify the ratio by finding the highest common factor.

Factors of 72: 1, 2, 3, 4, 6, 8, 9, 12, 18, 24, 36, 72
Factors of 48: 1, 2, 3, 4, 6, 8, 12, 16, 24, 48
Factors of 24: 1, 2, 3, 4, 6, 8, 12, 24

The highest common factor is 24.

$72 \div 24 = 3$
$48 \div 24 = 2$
$24 \div 24 = 1$

Therefore, the ratio of laptops to desktop computers to tablets repaired in February is $3 : 2 : 1$.

\section*{Question 41}
\textbf{Metadata}

\begin{itemize}
  \item Question ID: P6-RoFndRoWN\_P1-WNDiv2nd\_sonnet4\_Services\_01
  \item Primary KC: RATIO | Finding ratio | finding the ratio of two or three given whole numbers
  \item Secondary KC: WHOLE NUMBERS | Division | dividing whole numbers
  \item Topic: Services such as installation, maintenance, repairing, cleaning, laundry, hotel, retail, e-commerce, streaming services, digital services etc.
  \item Grade: Primary 6
\end{itemize}

\textbf{Solution}

Step 1: Divide each team's total by 60 to get smaller numbers.
Team A: $240 \div 60 = 4$
Team B: $180 \div 60 = 3$
Team C: $300 \div 60 = 5$

Step 2: Write the ratio using these simplified numbers.
Team A : Team B : Team C = $4 : 3 : 5$

Step 3: Check if the ratio can be simplified further.
The highest common factor of 4, 3, and 5 is 1, so the ratio is already in its simplest form.

Therefore, the ratio of rooms cleaned by Team A : Team B : Team C is $4 : 3 : 5$.

\section*{Question 42}
\textbf{Metadata}

\begin{itemize}
  \item Question ID: P6-RoFndTmWN\_P1-WNAdd2nd\_sonnet4\_Services\_01
  \item Primary KC: RATIO | Finding a missing term | finding the missing term in a pair of equivalent ratios
  \item Secondary KC: WHOLE NUMBERS | Addition | adding whole numbers
  \item Topic: Services such as installation, maintenance, repairing, cleaning, laundry, hotel, retail, e-commerce, streaming services, digital services etc.
  \item Grade: Primary 6
\end{itemize}

\textbf{Solution}

Step 1: Find the premium cleaning fee that corresponds to \textdollar48 basic cleaning fee.

Given ratio: Basic cleaning fee : Premium cleaning fee = $3:5$

Let the premium cleaning fee be $x$.
Setting up the proportion: $\frac{3}{5} = \frac{48}{x}$

Cross multiply: $3x = 5 \times 48$
$3x = 240$
$x = 240 \div 3 = 80$

So when basic cleaning fee is \textdollar48, premium cleaning fee is \textdollar80.

Step 2: Find the total earnings when the ratio is maintained.
Total earnings = \textdollar48 + \textdollar80 = \textdollar128

Step 3: Find how much they should earn from premium cleaning this month.
This month's target total = \textdollar91

Since the ratio $3:5$ must be maintained, let basic cleaning fee be $3y$ and premium cleaning fee be $5y$.

Total earnings: $3y + 5y = 8y = 91$
$y = 91 \div 8 = 11.375$

Premium cleaning fee this month = $5y = 5 \times 11.375 = 57.5$

Therefore, they should earn \textdollar57.50 from premium cleaning services this month.

\section*{Question 43}
\textbf{Metadata}

\begin{itemize}
  \item Question ID: P6-AgRepLrEx\_P6-AgEvlLrEx\_sonnet4\_Services\_01
  \item Primary KC: ALGEBRA | Representation and concept | translation of real-world situations into linear algebraic expressions
  \item Secondary KC: ALGEBRA | Evaluation | evaluating linear expressions by substitution
  \item Topic: Services such as installation, maintenance, repairing, cleaning, laundry, hotel, retail, e-commerce, streaming services, digital services etc.
  \item Grade: Primary 6
\end{itemize}

\textbf{Solution}

(a) In the expression $30 + 25h$:
- The constant term $30$ represents the fixed service fee in dollars that TechFix charges regardless of repair time
- The term $25h$ represents the variable labor cost, where $25$ is the hourly rate in dollars and $h$ is the number of hours worked

(b) To find the total cost when $h = 3$ hours:
Substitute $h = 3$ into the expression $30 + 25h$
Total cost $= 30 + 25(3)$
Total cost $= 30 + 75$
Total cost $= 105$
Therefore, the repair will cost \textdollar105.

(c) To find the number of hours when the total cost is \textdollar155:
Set up the equation: $30 + 25h = 155$
Subtract $30$ from both sides: $25h = 155 - 30$
Simplify: $25h = 125$
Divide both sides by $25$: $h = 125 ÷ 25$
Solve: $h = 5$
Therefore, the repair required $5$ hours of labor.

\section*{Question 44}
\textbf{Metadata}

\begin{itemize}
  \item Question ID: P6-AgSlvLrN\_P6-AgRepLrEx\_sonnet4\_Services\_01
  \item Primary KC: ALGEBRA | Solving simple linear equations | solving linear equations involving whole number coefficient and one variable only
  \item Secondary KC: ALGEBRA | Representation and concept | translation of real-world situations into linear algebraic expressions
  \item Topic: Services such as installation, maintenance, repairing, cleaning, laundry, hotel, retail, e-commerce, streaming services, digital services etc.
  \item Grade: Primary 6
\end{itemize}

\textbf{Solution}

Let $x$ be the fixed service fee in dollars and $y$ be the hourly rate in dollars.

From the given information, I can write two equations:
For Sarah's repair: $x + 2y = 45$ ... (1)
For Mike's repair: $x + 4y = 60$ ... (2)

To find the fixed service fee $x$, I will subtract equation (1) from equation (2):
$(x + 4y) - (x + 2y) = 60 - 45$
$x + 4y - x - 2y = 15$
$2y = 15$
$y = 7.5$

Now I substitute $y = 7.5$ into equation (1):
$x + 2(7.5) = 45$
$x + 15 = 45$
$x = 45 - 15$
$x = 30$

Therefore, the fixed service fee that the repair shop charges is \textdollar30.

\section*{Question 45}
\textbf{Metadata}

\begin{itemize}
  \item Question ID: O1-RoRepFr\_P2-FrSub2nd\_sonnet4\_Services\_01
  \item Primary KC: RATIO | Representation and concept | ratios involving fractions
  \item Secondary KC: FRACTIONS | Subtraction | subtracting fractions
  \item Topic: Services such as installation, maintenance, repairing, cleaning, laundry, hotel, retail, e-commerce, streaming services, digital services etc.
  \item Grade: Secondary O-level 1
\end{itemize}

\textbf{Solution}

First, let's find how many rooms Team A cleaned.
Team A cleaned $\frac{3}{4}$ of 48 rooms = $\frac{3}{4} \times 48 = 36$ rooms.

Next, we'll use the given ratio to find how many rooms Team B cleaned.
The ratio of rooms cleaned by Team A to Team B is $\frac{9}{10} : 1$.
This means: $\frac{\text{Rooms cleaned by Team A}}{\text{Rooms cleaned by Team B}} = \frac{9}{10} \div 1 = \frac{9}{10}$

So: $\frac{36}{\text{Rooms cleaned by Team B}} = \frac{9}{10}$

Cross-multiplying: $36 \times 10 = 9 \times \text{Rooms cleaned by Team B}$
$360 = 9 \times \text{Rooms cleaned by Team B}$
$\text{Rooms cleaned by Team B} = 40$ rooms

Now we can find how many rooms Team B was assigned in total.
Team B cleaned $\frac{5}{6}$ of their assigned rooms, and this equals 40 rooms.
$\frac{5}{6} \times \text{Team B's assigned rooms} = 40$
$\text{Team B's assigned rooms} = 40 \div \frac{5}{6} = 40 \times \frac{6}{5} = 48$ rooms

Finally, let's find the difference between the fractions completed.
Team A completed $\frac{3}{4}$ of their rooms.
Team B completed $\frac{5}{6}$ of their rooms.

To subtract these fractions: $\frac{5}{6} - \frac{3}{4}$
Finding the LCD of 6 and 4: LCD = 12
$\frac{5}{6} = \frac{10}{12}$ and $\frac{3}{4} = \frac{9}{12}$

Therefore: $\frac{5}{6} - \frac{3}{4} = \frac{10}{12} - \frac{9}{12} = \frac{1}{12}$

Team B was assigned 48 rooms, and Team B completed $\frac{1}{12}$ more of their assigned rooms than Team A.

\section*{Question 46}
\textbf{Metadata}

\begin{itemize}
  \item Question ID: O1-RoRepFr\_P6-FrDiv2nd\_sonnet4\_Services\_01
  \item Primary KC: RATIO | Representation and concept | ratios involving fractions
  \item Secondary KC: FRACTIONS | Division | fraction division
  \item Topic: Services such as installation, maintenance, repairing, cleaning, laundry, hotel, retail, e-commerce, streaming services, digital services etc.
  \item Grade: Secondary O-level 1
\end{itemize}

\textbf{Solution}

First, I need to simplify the ratio $\frac{3}{4} : \frac{1}{2}$.

To compare these fractions, I'll divide $\frac{3}{4}$ by $\frac{1}{2}$:
$\frac{3}{4} \div \frac{1}{2} = \frac{3}{4} \times \frac{2}{1} = \frac{6}{4} = \frac{3}{2}$

This means the ratio of Solution A to Solution B is $\frac{3}{2} : 1$, or equivalently $3 : 2$.

Now I can find the amounts needed:
Total parts = $3 + 2 = 5$ parts
Total mixture needed = 15 liters

Each part represents: $15 \div 5 = 3$ liters

Solution A needed: $3 \times 3 = 9$ liters
Solution B needed: $2 \times 3 = 6$ liters

Verification: $9 + 6 = 15$ liters ✓
Ratio check: $9 : 6 = 3 : 2$ ✓

Therefore, the company needs 9 liters of Solution A and 6 liters of Solution B.

\section*{Question 47}
\textbf{Metadata}

\begin{itemize}
  \item Question ID: O1-RoRepDc\_P4-DcSub2nd\_sonnet4\_Services\_01
  \item Primary KC: RATIO | Representation and concept | ratios involving decimals
  \item Secondary KC: DECIMALS | Subtraction | subtracting decimals
  \item Topic: Services such as installation, maintenance, repairing, cleaning, laundry, hotel, retail, e-commerce, streaming services, digital services etc.
  \item Grade: Secondary O-level 1
\end{itemize}

\textbf{Solution}

Given information:
- Ratio of rates = Office : Carpet : Window = $2.5 : 1.8 : 3.2$
- Office cleaning rate = \textdollar12.50 per hour

Step 1: Find the common ratio factor.
Since office cleaning corresponds to the ratio $2.5$, and the actual rate is \textdollar12.50:
Common factor = $\frac{12.50}{2.5} = 5$

Step 2: Calculate the original carpet and window cleaning rates.
Carpet cleaning rate = $1.8 \times 5 = \textdollar9.00$ per hour
Window cleaning rate = $3.2 \times 5 = \textdollar16.00$ per hour

Step 3: Calculate the new rates after price adjustment.
New carpet cleaning rate = $9.00 - 1.25 = \textdollar7.75$ per hour
New window cleaning rate = $16.00 - 2.80 = \textdollar13.20$ per hour

Therefore:
- Original carpet cleaning rate: \textdollar9.00 per hour
- Original window cleaning rate: \textdollar16.00 per hour
- New carpet cleaning rate: \textdollar7.75 per hour
- New window cleaning rate: \textdollar13.20 per hour

\section*{Question 48}
\textbf{Metadata}

\begin{itemize}
  \item Question ID: O1-PcRep2q\_O1-PcCnv2Dc\_sonnet4\_Services\_01
  \item Primary KC: PERCENTAGE | Representation and concept | comparing two quantities by percentage
  \item Secondary KC: PERCENTAGE | Conversion to decimals | expressing percentage as a decimal
  \item Topic: Services such as installation, maintenance, repairing, cleaning, laundry, hotel, retail, e-commerce, streaming services, digital services etc.
  \item Grade: Secondary O-level 1
\end{itemize}

\textbf{Solution}

Given information:
- Total subscribers: $25,000$
- Percentage who gave positive ratings: $72\%$

Step 1: Express the percentage as a decimal.
$72\% = \frac{72}{100} = 0.72$

Step 2: Calculate the number of subscribers who gave positive ratings.
Number of satisfied subscribers $= 0.72 \times 25,000 = 18,000$

Step 3: Verify by expressing as a percentage of total subscribers.
Percentage of satisfied subscribers $= \frac{18,000}{25,000} \times 100\% = \frac{18}{25} \times 100\% = 0.72 \times 100\% = 72\%$

Therefore, $18,000$ subscribers gave positive ratings, which represents $72\%$ of the total subscriber base.

\section*{Question 49}
\textbf{Metadata}

\begin{itemize}
  \item Question ID: O1-PcFndRslt\_P1-WNDiv2nd\_sonnet4\_Services\_01
  \item Primary KC: PERCENTAGE | Finding result after change | increasing/decreasing a quantity by a given percentage
  \item Secondary KC: WHOLE NUMBERS | Division | dividing whole numbers
  \item Topic: Services such as installation, maintenance, repairing, cleaning, laundry, hotel, retail, e-commerce, streaming services, digital services etc.
  \item Grade: Secondary O-level 1
\end{itemize}

\textbf{Solution}

Let the original monthly subscription fee be $x$ dollars.

After a $15\%$ increase, the new fee becomes:
$x + 0.15x = 1.15x$

We know that the new monthly fee is \textdollar$27$, so:
$1.15x = 27$
$x = \frac{27}{1.15} = \frac{2700}{115} = \frac{540}{23}$

Therefore, the original monthly fee was \textdollar$\frac{540}{23}$ and the new fee is \textdollar$27$.

To find the number of subscribers, we divide the total monthly revenue by the new monthly fee per subscriber:
Number of subscribers $= \frac{162000}{27} = 6000$

Therefore, the company has $6000$ subscribers.

\section*{Question 50}
\textbf{Metadata}

\begin{itemize}
  \item Question ID: O1-AgSlvFrLr\_O1-AgRepEq\_sonnet4\_Services\_01
  \item Primary KC: ALGEBRA | Solving | solving simple fractional equations that can be reduced to linear equations
  \item Secondary KC: ALGEBRA | Representation and concept | translation of simple real-world situations to equations
  \item Topic: Services such as installation, maintenance, repairing, cleaning, laundry, hotel, retail, e-commerce, streaming services, digital services etc.
  \item Grade: Secondary O-level 1
\end{itemize}

\textbf{Solution}

Let $m$ = monthly charge (in dollars) and $s$ = one-time setup fee (in dollars).

From the given information:
- After 3 months: $s + 3m = 45$ ... (1)
- After 7 months: $s + 7m = 65$ ... (2)

First, I'll solve equation (1) for the setup fee in terms of $m$:
$s + 3m = 45$
$s = 45 - 3m$ ... (3)

Next, I'll substitute equation (3) into equation (2):
$(45 - 3m) + 7m = 65$
$45 - 3m + 7m = 65$
$45 + 4m = 65$
$4m = 65 - 45$
$4m = 20$
$m = \frac{20}{4} = 5$

Now I'll find the setup fee by substituting $m = 5$ into equation (3):
$s = 45 - 3(5)$
$s = 45 - 15$
$s = 30$

Therefore, the monthly charge is \textdollar5 and the one-time setup fee is \textdollar30.

Verification:
- After 3 months: $30 + 3(5) = 30 + 15 = 45$ ✓
- After 7 months: $30 + 7(5) = 30 + 35 = 65$ ✓

\section*{Question 51}
\textbf{Metadata}

\begin{itemize}
  \item Question ID: O2-RoRepIvP\_P1-WNMul2nd\_sonnet4\_Services\_01
  \item Primary KC: RATIO | Representation and concept | inverse proportion
  \item Secondary KC: WHOLE NUMBERS | Multiplication | multiplying whole numbers
  \item Topic: Services such as installation, maintenance, repairing, cleaning, laundry, hotel, retail, e-commerce, streaming services, digital services etc.
  \item Grade: Secondary O-level 2
\end{itemize}

\textbf{Solution}

Since the time taken is inversely proportional to the number of cleaners, we have:

Time $\times$ Number of cleaners = constant

First, let's find the constant:
When 4 cleaners work for 6 hours:
Constant = $6 \times 4 = 24$

To complete the job in 3 hours:
$3 \times$ Number of cleaners = $24$
Number of cleaners = $24 \div 3 = 8$ cleaners

Therefore, 8 cleaners must be assigned to complete the job in 3 hours.

For the total cost calculation:
- Number of cleaners needed = 8
- Time taken = 3 hours
- Rate per cleaner per hour = \textdollar25

Total cost = Number of cleaners $\times$ Time $\times$ Rate per cleaner per hour
Total cost = $8 \times 3 \times 25 = 24 \times 25 = 600$

Therefore, 8 cleaners must be assigned and the total cost will be \textdollar600.

\section*{Question 52}
\textbf{Metadata}

\begin{itemize}
  \item Question ID: O2-RoRepIvP\_P1-WNDiv2nd\_sonnet4\_Services\_01
  \item Primary KC: RATIO | Representation and concept | inverse proportion
  \item Secondary KC: WHOLE NUMBERS | Division | dividing whole numbers
  \item Topic: Services such as installation, maintenance, repairing, cleaning, laundry, hotel, retail, e-commerce, streaming services, digital services etc.
  \item Grade: Secondary O-level 2
\end{itemize}

\textbf{Solution}

Let me solve this step by step.

Step 1: Find the relationship between cleaners and time for the standard building.
Since time is inversely proportional to the number of cleaners:
Time $\times$ Number of cleaners $= k$ (constant)

For the standard building: $8 \times 6 = 48$
So $k = 48$

Step 2: Find the time needed for the large complex with 6 cleaners.
The large complex takes 3 times as long as the standard building.
Time for large complex with 6 cleaners $= 3 \times 8 = 24$ hours

Step 3: Find the constant for the large complex.
Using 6 cleaners taking 24 hours:
$k = 24 \times 6 = 144$

Step 4: Find the number of cleaners needed to complete the job in 12 hours.
Using the inverse proportion relationship:
Time $\times$ Number of cleaners $= 144$
$12 \times$ Number of cleaners $= 144$
Number of cleaners $= 144 \div 12 = 12$

Therefore, 12 cleaners should be assigned to complete the cleaning of the large complex in exactly 12 hours.

\section*{Question 53}
\textbf{Metadata}

\begin{itemize}
  \item Question ID: O2-AgSlvLr2v\_O1-AgRepEq\_sonnet4\_Services\_01
  \item Primary KC: ALGEBRA | Solving | solving linear equations in two variables
  \item Secondary KC: ALGEBRA | Representation and concept | translation of simple real-world situations to equations
  \item Topic: Services such as installation, maintenance, repairing, cleaning, laundry, hotel, retail, e-commerce, streaming services, digital services etc.
  \item Grade: Secondary O-level 2
\end{itemize}

\textbf{Solution}

Let $f$ be the fixed service fee in dollars and $h$ be the hourly cleaning rate in dollars.

From the given information, I can set up two equations:

For Mrs. Tan: Fixed fee + (3 hours × hourly rate) = \textdollar45
$f + 3h = 45$ ... (1)

For Mrs. Lee: Fixed fee + (5 hours × hourly rate) = \textdollar65
$f + 5h = 65$ ... (2)

To solve this system of linear equations, I'll use the elimination method.

Subtracting equation (1) from equation (2):
$(f + 5h) - (f + 3h) = 65 - 45$
$f + 5h - f - 3h = 20$
$2h = 20$
$h = 10$

Substituting $h = 10$ into equation (1):
$f + 3(10) = 45$
$f + 30 = 45$
$f = 15$

Checking with equation (2):
$15 + 5(10) = 15 + 50 = 65$ ✓

Therefore, the fixed service fee is \textdollar15 and the hourly cleaning rate is \textdollar10 per hour.

\section*{Question 54}
\textbf{Metadata}

\begin{itemize}
  \item Question ID: O2-SPFndmdn\_O2-SPFndmode\_sonnet4\_Services\_01
  \item Primary KC: STATISTICS AND PROBABILITY | Finding median | Finding median for a set of data
  \item Secondary KC: STATISTICS AND PROBABILITY | Finding mode | Finding mode for a set of data
  \item Topic: Services such as installation, maintenance, repairing, cleaning, laundry, hotel, retail, e-commerce, streaming services, digital services etc.
  \item Grade: Secondary O-level 2
\end{itemize}

\textbf{Solution}

**Step 1: Find the median**

First, arrange the data in ascending order:
$8, 8, 8, 8, 12, 12, 12, 12, 12, 15, 15, 18, 18, 20, 22, 25$

Wait, let me recount the data points: $8, 12, 15, 8, 20, 12, 18, 8, 25, 12, 15, 8, 22, 18, 12$

There are 15 data points. Arranging in ascending order:
$8, 8, 8, 8, 12, 12, 12, 12, 12, 15, 15, 18, 18, 20, 22, 25$

Actually, let me recount: $8, 8, 8, 8, 12, 12, 12, 12, 12, 15, 15, 18, 18, 20, 22, 25$ - this gives 16 values, but we should have 15.

Let me recheck the original data: $8, 12, 15, 8, 20, 12, 18, 8, 25, 12, 15, 8, 22, 18, 12$

Counting: There are indeed 15 values.
Arranging in ascending order: $8, 8, 8, 8, 12, 12, 12, 12, 12, 15, 15, 18, 18, 20, 22, 25$

This shows 16 values. Let me rewrite the original data more carefully:
$8, 12, 15, 8, 20, 12, 18, 8, 25, 12, 15, 8, 22, 18, 12$

Actually, the original has: $8$ appears 4 times, $12$ appears 5 times, $15$ appears 2 times, $18$ appears 2 times, $20$ appears 1 time, $22$ appears 1 time, $25$ appears 1 time. Total: $4+5+2+2+1+1+1 = 16$

Let me correct the original data to have exactly 15 values:
$8, 12, 15, 8, 20, 12, 18, 8, 25, 12, 15, 8, 22, 18, 12$

Arranging in ascending order: $8, 8, 8, 8, 12, 12, 12, 12, 12, 15, 15, 18, 18, 20, 22$

Now we have 15 values. For 15 values, the median is the 8th value (since $\frac{15+1}{2} = 8$).

The 8th value is $12$.

Therefore, the median is $12$.

**Step 2: Find the mode**

Count the frequency of each value:
- $8$ appears 4 times
- $12$ appears 5 times  
- $15$ appears 2 times
- $18$ appears 2 times
- $20$ appears 1 time
- $22$ appears 1 time

The value that appears most frequently is $12$ (appearing 5 times).

Therefore, the mode is $12$.

**Step 3: Compare the measures**

Both the median and mode are $12$ devices per day. Since both measures give the same value, this strongly suggests that $12$ devices per day represents the shop's typical daily repair volume well.

\section*{Question 55}
\textbf{Metadata}

\begin{itemize}
  \item Question ID: O2-SPFndmdn\_O3-SPFndPctl\_sonnet4\_Services\_01
  \item Primary KC: STATISTICS AND PROBABILITY | Finding median | Finding median for a set of data
  \item Secondary KC: STATISTICS AND PROBABILITY | Finding percentiles | finding percentiles for a set of data
  \item Topic: Services such as installation, maintenance, repairing, cleaning, laundry, hotel, retail, e-commerce, streaming services, digital services etc.
  \item Grade: Secondary O-level 2
\end{itemize}

\textbf{Solution}

First, I need to arrange the repair times in ascending order: 32, 35, 39, 41, 45, 48, 49, 52, 56, 59, 63, 67, 71, 78, 84

(a) Finding the median:
Since there are 15 data points (odd number), the median is the middle value.
Position of median = $\frac{15 + 1}{2} = 8$th position
The 8th value in the ordered list is 52.
Therefore, the median repair time is 52 minutes.

(b) Finding the 80th percentile:
To find the 80th percentile, I use the formula:
Position = $\frac{P}{100} \times n = \frac{80}{100} \times 15 = 12$
Since the position is exactly 12 (a whole number), the 80th percentile is the 12th value in the ordered list.
The 12th value is 67.
Therefore, the 80th percentile is 67 minutes.

This means that 80% of laptop repairs are completed within 67 minutes or faster, so the company can tell customers that most repairs (80%) are completed within 67 minutes.

\section*{Question 56}
\textbf{Metadata}

\begin{itemize}
  \item Question ID: O3-BPOpr\_O3-BPRepPosI\_sonnet4\_Services\_01
  \item Primary KC: BASE AND POWER | Operations | laws of indices
  \item Secondary KC: BASE AND POWER | Representation and concept  | positive indices that is not 1
  \item Topic: Services such as installation, maintenance, repairing, cleaning, laundry, hotel, retail, e-commerce, streaming services, digital services etc.
  \item Grade: Secondary O-level 3/4
\end{itemize}

\textbf{Solution}

**Part (a): Finding the number of active users in May**

The number of users in May is $2^3 \times 2^4 \times 2^2$ thousand.

Using the law of indices: $a^m \times a^n = a^{m+n}$

$2^3 \times 2^4 \times 2^2 = 2^{3+4+2} = 2^9$

Calculating $2^9$:
$2^9 = 2^8 \times 2^1 = 256 \times 2 = 512$

Therefore, the company had $512$ thousand active users in May.

**Part (b): Finding the total server capacity needed**

Each thousand users requires $3^2$ processing units.

Calculating $3^2$:
$3^2 = 3 \times 3 = 9$

Total processing units needed = Number of thousand users $\times$ Processing units per thousand users

Total processing units = $512 \times 9 = 4608$

Therefore, the company needs $4608$ processing units to support all users in May.

\section*{Question 57}
\textbf{Metadata}

\begin{itemize}
  \item Question ID: O3-BPOpr\_O3-BPRepFrI\_sonnet4\_Services\_01
  \item Primary KC: BASE AND POWER | Operations | laws of indices
  \item Secondary KC: BASE AND POWER | Representation and concept  | fractional indices
  \item Topic: Services such as installation, maintenance, repairing, cleaning, laundry, hotel, retail, e-commerce, streaming services, digital services etc.
  \item Grade: Secondary O-level 3/4
\end{itemize}

\textbf{Solution}

Given: $N = 8000 \times 2^{\frac{3t}{4}}$

(a) When $t = 8$:
$N = 8000 \times 2^{\frac{3 \times 8}{4}}$
$N = 8000 \times 2^{\frac{24}{4}}$
$N = 8000 \times 2^6$
$N = 8000 \times 64$
$N = 512000$

Therefore, there will be 512000 subscribers after 8 months.

(b) When $N = 32000$:
$32000 = 8000 \times 2^{\frac{3t}{4}}$
$\frac{32000}{8000} = 2^{\frac{3t}{4}}$
$4 = 2^{\frac{3t}{4}}$

Since $4 = 2^2$:
$2^2 = 2^{\frac{3t}{4}}$

Using the law of indices, when bases are equal:
$2 = \frac{3t}{4}$
$8 = 3t$
$t = \frac{8}{3}$

(c) From part (b), we found that when $N = 32000$, we have $2^2 = 2^{\frac{3t}{4}}$.
This gives us $a = 2$ in the form $2^a$.

The exact value of $t = \frac{8}{3}$ months.

Therefore, the company will have 32000 subscribers after $\frac{8}{3}$ months (or $2\frac{2}{3}$ months).

\section*{Question 58}
\textbf{Metadata}

\begin{itemize}
  \item Question ID: O3-STOprUn\_O3-STOprIns\_sonnet4\_Services\_01
  \item Primary KC: SET | Set operations | union of two sets
  \item Secondary KC: SET | Set operations | intersection of two sets
  \item Topic: Services such as installation, maintenance, repairing, cleaning, laundry, hotel, retail, e-commerce, streaming services, digital services etc.
  \item Grade: Secondary O-level 3/4
\end{itemize}

\textbf{Solution}

Given information:
- Total customers served: 45
- $|H| = 28$ (customers who received Hardware Repair)
- $|S| = 22$ (customers who received Software Troubleshooting)
- Every customer received at least one service, so $|H \cup S| = 45$

(a) To find the number of customers who received both services:

Using the formula for union of two sets:
$|H \cup S| = |H| + |S| - |H \cap S|$

Substituting the known values:
$45 = 28 + 22 - |H \cap S|$
$45 = 50 - |H \cap S|$
$|H \cap S| = 50 - 45 = 5$

Therefore, 5 customers received both Hardware Repair and Software Troubleshooting services.

(b) To find customers who received exactly one type of service:

Customers who received only Hardware Repair = $|H| - |H \cap S| = 28 - 5 = 23$
Customers who received only Software Troubleshooting = $|S| - |H \cap S| = 22 - 5 = 17$

Total customers who received exactly one type of service = $23 + 17 = 40$

Alternatively, this can be calculated as:
$|H \cup S| - |H \cap S| = 45 - 5 = 40$

Therefore, 40 customers received exactly one type of service.

\section*{Question 59}
\textbf{Metadata}

\begin{itemize}
  \item Question ID: O3-MXMulSM\_O3-MXSub\_sonnet4\_Services\_01
  \item Primary KC: MATRICES | Multiplication | product of a scalar quantity and a matrix
  \item Secondary KC: MATRICES | Subtraction | subtraction of matrices
  \item Topic: Services such as installation, maintenance, repairing, cleaning, laundry, hotel, retail, e-commerce, streaming services, digital services etc.
  \item Grade: Secondary O-level 3/4
\end{itemize}

\textbf{Solution}

To find the net revenue matrix for February, I need to:

1) First calculate the expected February revenue before refunds by multiplying matrix $A$ by the scalar $1.4$
2) Then subtract the refunds matrix $B$ from this result

**Step 1: Calculate expected February revenue**
$1.4A = 1.4 \times \begin{pmatrix} 45 & 60 \\ 80 & 95 \\ 30 & 40 \end{pmatrix}$

$1.4A = \begin{pmatrix} 1.4 \times 45 & 1.4 \times 60 \\ 1.4 \times 80 & 1.4 \times 95 \\ 1.4 \times 30 & 1.4 \times 40 \end{pmatrix}$

$1.4A = \begin{pmatrix} 63 & 84 \\ 112 & 133 \\ 42 & 56 \end{pmatrix}$

**Step 2: Subtract refunds matrix $B$**
Net revenue matrix = $1.4A - B$

$1.4A - B = \begin{pmatrix} 63 & 84 \\ 112 & 133 \\ 42 & 56 \end{pmatrix} - \begin{pmatrix} 8 & 12 \\ 15 & 18 \\ 6 & 9 \end{pmatrix}$

$1.4A - B = \begin{pmatrix} 63-8 & 84-12 \\ 112-15 & 133-18 \\ 42-6 & 56-9 \end{pmatrix}$

$1.4A - B = \begin{pmatrix} 55 & 72 \\ 97 & 115 \\ 36 & 47 \end{pmatrix}$

Therefore, the net revenue matrix for February is $\begin{pmatrix} 55 & 72 \\ 97 & 115 \\ 36 & 47 \end{pmatrix}$ (in thousands of dollars).

\section*{Question 60}
\textbf{Metadata}

\begin{itemize}
  \item Question ID: O3-MXMulSM\_O3-MXMul\_sonnet4\_Services\_01
  \item Primary KC: MATRICES | Multiplication | product of a scalar quantity and a matrix
  \item Secondary KC: MATRICES | Multiplication | multiplication of matrices
  \item Topic: Services such as installation, maintenance, repairing, cleaning, laundry, hotel, retail, e-commerce, streaming services, digital services etc.
  \item Grade: Secondary O-level 3/4
\end{itemize}

\textbf{Solution}

(a) To find the new rate matrix after a 15\% increase, we multiply the original rate matrix $R$ by the scalar $1.15$:

New rate matrix = $1.15 \times R$

$= 1.15 \times \begin{pmatrix} 45 & 60 & 80 \\ 35 & 50 & 65 \\ 55 & 75 & 95 \end{pmatrix}$

$= \begin{pmatrix} 1.15 \times 45 & 1.15 \times 60 & 1.15 \times 80 \\ 1.15 \times 35 & 1.15 \times 50 & 1.15 \times 65 \\ 1.15 \times 55 & 1.15 \times 75 & 1.15 \times 95 \end{pmatrix}$

$= \begin{pmatrix} 51.75 & 69 & 92 \\ 40.25 & 57.5 & 74.75 \\ 63.25 & 86.25 & 109.25 \end{pmatrix}$

(b) To find the total revenue, we need to multiply the new rate matrix by the hours matrix using matrix multiplication.

Let $R_{new} = \begin{pmatrix} 51.75 & 69 & 92 \\ 40.25 & 57.5 & 74.75 \\ 63.25 & 86.25 & 109.25 \end{pmatrix}$ and $H = \begin{pmatrix} 12 & 8 & 15 \\ 18 & 22 & 10 \\ 6 & 12 & 8 \end{pmatrix}$

Revenue matrix = $R_{new} \times H$

Calculating each element:

Row 1, Column 1: $51.75(12) + 69(18) + 92(6) = 621 + 1242 + 552 = 2415$
Row 1, Column 2: $51.75(8) + 69(22) + 92(12) = 414 + 1518 + 1104 = 3036$
Row 1, Column 3: $51.75(15) + 69(10) + 92(8) = 776.25 + 690 + 736 = 2202.25$

Row 2, Column 1: $40.25(12) + 57.5(18) + 74.75(6) = 483 + 1035 + 448.5 = 1966.5$
Row 2, Column 2: $40.25(8) + 57.5(22) + 74.75(12) = 322 + 1265 + 897 = 2484$
Row 2, Column 3: $40.25(15) + 57.5(10) + 74.75(8) = 603.75 + 575 + 598 = 1776.75$

Row 3, Column 1: $63.25(12) + 86.25(18) + 109.25(6) = 759 + 1552.5 + 655.5 = 2967$
Row 3, Column 2: $63.25(8) + 86.25(22) + 109.25(12) = 506 + 1897.5 + 1311 = 3714.5$
Row 3, Column 3: $63.25(15) + 86.25(10) + 109.25(8) = 948.75 + 862.5 + 874 = 2685.25$

Revenue matrix = $\begin{pmatrix} 2415 & 3036 & 2202.25 \\ 1966.5 & 2484 & 1776.75 \\ 2967 & 3714.5 & 2685.25 \end{pmatrix}$

Total revenue = Sum of all elements in the revenue matrix
$= 2415 + 3036 + 2202.25 + 1966.5 + 2484 + 1776.75 + 2967 + 3714.5 + 2685.25$
$= \textdollar23247.25$

\section*{Question 61}
\textbf{Metadata}

\begin{itemize}
  \item Question ID: O3-MXMul\_O3-MXSub\_sonnet4\_Services\_01
  \item Primary KC: MATRICES | Multiplication | multiplication of matrices
  \item Secondary KC: MATRICES | Subtraction | subtraction of matrices
  \item Topic: Services such as installation, maintenance, repairing, cleaning, laundry, hotel, retail, e-commerce, streaming services, digital services etc.
  \item Grade: Secondary O-level 3/4
\end{itemize}

\textbf{Solution}

**Solution:**

**(a) Calculate subscriber numbers after cancellations**

To find subscriber numbers after cancellations, we subtract the cancellation matrix $C$ from the initial matrix $A$:

$$A - C = \begin{pmatrix} 1200 & 800 & 600 \\ 900 & 1100 & 750 \\ 1500 & 1300 & 900 \end{pmatrix} - \begin{pmatrix} 50 & 30 & 20 \\ 40 & 45 & 25 \\ 60 & 55 & 35 \end{pmatrix}$$

$$= \begin{pmatrix} 1200-50 & 800-30 & 600-20 \\ 900-40 & 1100-45 & 750-25 \\ 1500-60 & 1300-55 & 900-35 \end{pmatrix}$$

$$= \begin{pmatrix} 1150 & 770 & 580 \\ 860 & 1055 & 725 \\ 1440 & 1245 & 865 \end{pmatrix}$$

**(b) Calculate final subscriber distribution after conversions**

To find the final distribution, we multiply the post-cancellation matrix by the conversion matrix $R$:

$$(A - C) \times R = \begin{pmatrix} 1150 & 770 & 580 \\ 860 & 1055 & 725 \\ 1440 & 1245 & 865 \end{pmatrix} \times \begin{pmatrix} 0.8 & 0.15 & 0.05 \\ 0.1 & 0.7 & 0.2 \\ 0.05 & 0.25 & 0.7 \end{pmatrix}$$

Calculating each element:

**Row 1:**
- $(1150)(0.8) + (770)(0.1) + (580)(0.05) = 920 + 77 + 29 = 1026$
- $(1150)(0.15) + (770)(0.7) + (580)(0.25) = 172.5 + 539 + 145 = 856.5$
- $(1150)(0.05) + (770)(0.2) + (580)(0.7) = 57.5 + 154 + 406 = 617.5$

**Row 2:**
- $(860)(0.8) + (1055)(0.1) + (725)(0.05) = 688 + 105.5 + 36.25 = 829.75$
- $(860)(0.15) + (1055)(0.7) + (725)(0.25) = 129 + 738.5 + 181.25 = 1048.75$
- $(860)(0.05) + (1055)(0.2) + (725)(0.7) = 43 + 211 + 507.5 = 761.5$

**Row 3:**
- $(1440)(0.8) + (1245)(0.1) + (865)(0.05) = 1152 + 124.5 + 43.25 = 1319.75$
- $(1440)(0.15) + (1245)(0.7) + (865)(0.25) = 216 + 871.5 + 216.25 = 1303.75$
- $(1440)(0.05) + (1245)(0.2) + (865)(0.7) = 72 + 249 + 605.5 = 926.5$

**Final Answer:**
$$\begin{pmatrix} 1026 & 856.5 & 617.5 \\ 829.75 & 1048.75 & 761.5 \\ 1319.75 & 1303.75 & 926.5 \end{pmatrix}$$

\section*{Question 62}
\textbf{Metadata}

\begin{itemize}
  \item Question ID: O3-SPFndQtl\_O3-SPFndIQR\_sonnet4\_Services\_01
  \item Primary KC: STATISTICS AND PROBABILITY | Finding quartiles | finding quartiles for a set of data
  \item Secondary KC: STATISTICS AND PROBABILITY | Finding range | finding interquartile range as measures of spread for a set of data 
  \item Topic: Services such as installation, maintenance, repairing, cleaning, laundry, hotel, retail, e-commerce, streaming services, digital services etc.
  \item Grade: Secondary O-level 3/4
\end{itemize}

\textbf{Solution}

First, I need to arrange the repair times in ascending order:
18, 21, 25, 26, 28, 29, 32, 33, 35, 38, 39, 41, 45, 47, 52

There are 15 data values, so $n = 15$.

To find the first quartile ($Q_1$):
Position of $Q_1 = \frac{1}{4}(n+1) = \frac{1}{4}(15+1) = \frac{16}{4} = 4$
Therefore, $Q_1$ is the 4th value in the ordered list.
$Q_1 = 26$ minutes

To find the third quartile ($Q_3$):
Position of $Q_3 = \frac{3}{4}(n+1) = \frac{3}{4}(15+1) = \frac{3 \times 16}{4} = \frac{48}{4} = 12$
Therefore, $Q_3$ is the 12th value in the ordered list.
$Q_3 = 41$ minutes

To find the interquartile range (IQR):
$IQR = Q_3 - Q_1 = 41 - 26 = 15$ minutes

Therefore:
- First quartile ($Q_1$) = 26 minutes
- Third quartile ($Q_3$) = 41 minutes  
- Interquartile range (IQR) = 15 minutes

This means that 50% of the repair times fall between 26 and 41 minutes, with a spread of 15 minutes between the middle 50% of the data.

\section*{Question 63}
\textbf{Metadata}

\begin{itemize}
  \item Question ID: O3-SPMulProb\_O2-SPRepPrSE\_sonnet4\_Services\_01
  \item Primary KC: STATISTICS AND PROBABILITY | Multiplication | multiplication of probabilities
  \item Secondary KC: STATISTICS AND PROBABILITY | Representation and concept | probability of single events
  \item Topic: Services such as installation, maintenance, repairing, cleaning, laundry, hotel, retail, e-commerce, streaming services, digital services etc.
  \item Grade: Secondary O-level 3/4
\end{itemize}

\textbf{Solution}

Let me define the events:
- Event $S$: The issue is software-related
- Event $H$: The issue is hardware-related
- Event $D$: The repair is completed within the same day

Given information:
- $P(S) = 0.7$
- $P(H) = 0.3$
- $P(D|S) = 0.9$ (probability of same-day completion given software issue)
- $P(D|H) = 0.6$ (probability of same-day completion given hardware issue)

To find the probability that a repair is completed within the same day, I need to use the law of total probability:

$P(D) = P(D|S) \times P(S) + P(D|H) \times P(H)$

Substituting the values:
$P(D) = 0.9 \times 0.7 + 0.6 \times 0.3$
$P(D) = 0.63 + 0.18$
$P(D) = 0.81$

Therefore, the probability that a randomly selected laptop repair will be completed within the same day is $0.81$ or $81\%$.

\end{document}
