\documentclass{article}
\usepackage[utf8]{inputenc}
\usepackage{amsmath}
\usepackage{amsfonts}
\usepackage{amssymb}
\usepackage{graphicx}
\usepackage{hyperref}
\title{'Minh Solutions recreation v3 CLAUDE '}
\author{Tien Dung Doan}
\begin{document}
\maketitle
\section*{Question 1}
\textbf{Metadata}

\begin{itemize}
  \item Question ID: P3-WNSub4d\_P1-WNCmp\_sonnet4\_Recreation\_04
  \item Primary KC: WHOLE NUMBERS | Subtraction | subtracting whole numbers up to 4 digits
  \item Secondary KC: WHOLE NUMBERS | Comparison and ordering | comparing and ordering whole numbers
  \item Topic: Recreation such as sports, games, exercises, music, movie, dancing, painting, fishing and other recreation activities
  \item Grade: Primary 3
\end{itemize}

\textbf{Solution}

(a) To find who scored the highest, I need to compare the three scores:
- Alex: $8,524$ points
- Ben: $6,789$ points  
- Claire: $7,356$ points

Comparing the thousands place: Alex has $8$ thousand, while Ben and Claire both have $6$ thousand and $7$ thousand respectively.

Since $8 > 7 > 6$, Alex scored the highest with $8,524$ points.

(b) From part (a), Alex scored the highest with $8,524$ points.

Now I need to find who scored the lowest by comparing Ben's and Claire's scores:
- Ben: $6,789$ points
- Claire: $7,356$ points

Since $7,356 > 6,789$, Ben scored the lowest with $6,789$ points.

To find how many more points the highest scorer got compared to the lowest scorer:
$8,524 - 6,789$

Using column subtraction:
```
  8524
- 6789
------
  1735
```

Therefore:
(a) Alex scored the highest among the three friends.
(b) The highest scorer got $1,735$ more points than the lowest scorer.

\section*{Question 2}
\textbf{Metadata}

\begin{itemize}
  \item Question ID: P3-WNDivRmd3d\_P1-WNAdd2nd\_sonnet4\_Recreation\_04
  \item Primary KC: WHOLE NUMBERS | Division | dividing whole numbers up to 3 digits by 1 digit with remainder 
  \item Secondary KC: WHOLE NUMBERS | Addition | adding whole numbers
  \item Topic: Recreation such as sports, games, exercises, music, movie, dancing, painting, fishing and other recreation activities
  \item Grade: Primary 3
\end{itemize}

\textbf{Solution}

To find how many complete groups can be formed, I need to divide $847$ by $6$.

$847 \div 6 = 141$ remainder $1$

Let me check: $141 \times 6 = 846$
$847 - 846 = 1$

So $141$ complete groups can be formed, and $1$ student will be left without a complete group.

To find the total number of students participating, I need to add the students in complete groups and the remaining student:
$846 + 1 = 847$

Therefore:
- $141$ complete groups can be formed
- $1$ student will be left without a complete group
- $847$ students will participate in total

\section*{Question 3}
\textbf{Metadata}

\begin{itemize}
  \item Question ID: P3-WNDivRmd3d\_P1-WNMul2nd\_sonnet4\_Recreation\_04
  \item Primary KC: WHOLE NUMBERS | Division | dividing whole numbers up to 3 digits by 1 digit with remainder 
  \item Secondary KC: WHOLE NUMBERS | Multiplication | multiplying whole numbers
  \item Topic: Recreation such as sports, games, exercises, music, movie, dancing, painting, fishing and other recreation activities
  \item Grade: Primary 3
\end{itemize}

\textbf{Solution}

Step 1: Find how many paintbrushes each group gets and how many are left over.

Divide $345$ by $8$:
$345 \div 8 = 43$ remainder $1$

To check: $43 \times 8 = 344$
$345 - 344 = 1$

So each group gets $43$ paintbrushes and there is $1$ paintbrush left over.

Step 2: Find how many sheets of paper each group receives.

Each group gets $7$ sheets of paper per paintbrush.
Number of sheets per group $= 43 \times 7 = 301$

Therefore, each group will get $43$ paintbrushes with $1$ paintbrush left over, and each group will receive $301$ sheets of paper in total.

\section*{Question 4}
\textbf{Metadata}

\begin{itemize}
  \item Question ID: P3-WNDiv3d1d\_P1-WNAdd2nd\_sonnet4\_Recreation\_04
  \item Primary KC: WHOLE NUMBERS | Division | dividing whole numbers up to 3 digits by 1 digit
  \item Secondary KC: WHOLE NUMBERS | Addition | adding whole numbers
  \item Topic: Recreation such as sports, games, exercises, music, movie, dancing, painting, fishing and other recreation activities
  \item Grade: Primary 3
\end{itemize}

\textbf{Solution}

Step 1: Find how many basketballs each team gets by dividing the total basketballs equally among 4 teams.
$276 \div 4 = 69$

Step 2: Find the total number of basketballs Sarah has after the division (this remains the same as the original amount).
Total basketballs after division = $276$

Step 3: Add the donated basketballs to find the final total.
Final total = $276 + 48 = 324$

Therefore, Sarah has $324$ basketballs in total after receiving the donation.

\section*{Question 5}
\textbf{Metadata}

\begin{itemize}
  \item Question ID: P3-WNDiv3d1d\_P1-WNSub2nd\_sonnet4\_Recreation\_04
  \item Primary KC: WHOLE NUMBERS | Division | dividing whole numbers up to 3 digits by 1 digit
  \item Secondary KC: WHOLE NUMBERS | Subtraction | subtracting whole numbers
  \item Topic: Recreation such as sports, games, exercises, music, movie, dancing, painting, fishing and other recreation activities
  \item Grade: Primary 3
\end{itemize}

\textbf{Solution}

Step 1: Find how many sheets of painting paper are usable after removing the damaged ones.

Usable sheets = Total sheets - Damaged sheets
Usable sheets = $864 - 48 = 816$ sheets

Step 2: Divide the usable sheets equally among the $6$ art stations.

Sheets per station = $816 \div 6$

To solve $816 \div 6$:
$816 \div 6 = 136$

Therefore, each art station will receive $136$ sheets of painting paper.

\section*{Question 6}
\textbf{Metadata}

\begin{itemize}
  \item Question ID: P3-WNDiv3d1d\_P1-WNMul2nd\_sonnet4\_Recreation\_04
  \item Primary KC: WHOLE NUMBERS | Division | dividing whole numbers up to 3 digits by 1 digit
  \item Secondary KC: WHOLE NUMBERS | Multiplication | multiplying whole numbers
  \item Topic: Recreation such as sports, games, exercises, music, movie, dancing, painting, fishing and other recreation activities
  \item Grade: Primary 3
\end{itemize}

\textbf{Solution}

Step 1: Find how many basketballs each team will receive.
Total basketballs = $432$
Number of teams = $6$
Basketballs per team = $432 \div 6 = 72$

Step 2: Find how many practice sessions each team can have.
Each practice session needs $4$ basketballs.
Number of practice sessions per team = $72 \div 4 = 18$

Therefore, each team will receive $72$ basketballs and can have $18$ practice sessions.

\section*{Question 7}
\textbf{Metadata}

\begin{itemize}
  \item Question ID: P3-FrAddRl12\_P2-FrCmp\_sonnet4\_Recreation\_04
  \item Primary KC: FRACTIONS | Addition | adding two related fractions within one whole with denominators of given fractions not exceeding 12
  \item Secondary KC: FRACTIONS | Comparison and ordering | comparing and ordering fractions
  \item Topic: Recreation such as sports, games, exercises, music, movie, dancing, painting, fishing and other recreation activities
  \item Grade: Primary 3
\end{itemize}

\textbf{Solution}

(a) Time practiced on Monday and Tuesday combined:
$\frac{2}{6} + \frac{3}{6} = \frac{2+3}{6} = \frac{5}{6}$ of an hour

(b) Time practiced on Tuesday and Wednesday combined:
$\frac{3}{6} + \frac{1}{6} = \frac{3+1}{6} = \frac{4}{6}$ of an hour

(c) Comparing the two results:
Monday and Tuesday combined: $\frac{5}{6}$ of an hour
Tuesday and Wednesday combined: $\frac{4}{6}$ of an hour

Since $\frac{5}{6} > \frac{4}{6}$, Sarah practiced for a longer time on Monday and Tuesday combined.

\section*{Question 8}
\textbf{Metadata}

\begin{itemize}
  \item Question ID: P3-FrSubRl12\_P2-FrAdd2nd\_sonnet4\_Recreation\_04
  \item Primary KC: FRACTIONS | Subtraction | subtracting two related fractions within one whole with denominators of given fractions not exceeding 12
  \item Secondary KC: FRACTIONS | Addition | adding fractions
  \item Topic: Recreation such as sports, games, exercises, music, movie, dancing, painting, fishing and other recreation activities
  \item Grade: Primary 3
\end{itemize}

\textbf{Solution}

Step 1: Find the combined practice time for Monday and Tuesday.
Monday: $\frac{3}{8}$ hour
Tuesday: $\frac{1}{4}$ hour

To add these fractions, I need a common denominator. The LCD of 8 and 4 is 8.
$\frac{1}{4} = \frac{2}{8}$

Combined time = $\frac{3}{8} + \frac{2}{8} = \frac{5}{8}$ hour

Step 2: Find how much longer Wednesday's practice was compared to Monday and Tuesday combined.
Wednesday: $\frac{5}{12}$ hour
Monday and Tuesday combined: $\frac{5}{8}$ hour

To subtract these fractions, I need a common denominator. The LCD of 12 and 8 is 24.
$\frac{5}{12} = \frac{10}{24}$
$\frac{5}{8} = \frac{15}{24}$

Since $\frac{15}{24} > \frac{10}{24}$, the combined time of Monday and Tuesday is longer than Wednesday.

Difference = $\frac{15}{24} - \frac{10}{24} = \frac{5}{24}$ hour

Therefore:
- Sarah practiced for $\frac{5}{8}$ hour on Monday and Tuesday combined.
- She practiced $\frac{5}{24}$ hour less on Wednesday compared to the combined time of Monday and Tuesday.

\section*{Question 9}
\textbf{Metadata}

\begin{itemize}
  \item Question ID: P4-WNMul4d1d\_P1-WNAdd2nd\_sonnet4\_Recreation\_04
  \item Primary KC: WHOLE NUMBERS | Multiplication | multiplying whole numbers up to 4 digits by 1 digit or up to 3 digits by 2 digits
  \item Secondary KC: WHOLE NUMBERS | Addition | adding whole numbers
  \item Topic: Recreation such as sports, games, exercises, music, movie, dancing, painting, fishing and other recreation activities
  \item Grade: Primary 4
\end{itemize}

\textbf{Solution}

Step 1: Find the total number of packs Sarah needs to buy.
Number of packs = $15 + 23 = 38$ packs

Step 2: Find the total cost for all the packs.
Total cost = $38 \times 8 = \textdollar304$

Therefore, Sarah needs to spend \textdollar304 in total.

\section*{Question 10}
\textbf{Metadata}

\begin{itemize}
  \item Question ID: P4-WNDiv4d1d\_P1-WNCmp\_sonnet4\_Recreation\_04
  \item Primary KC: WHOLE NUMBERS | Division | dividing whole numbers up to 4 digits by 1 digit
  \item Secondary KC: WHOLE NUMBERS | Comparison and ordering | comparing and ordering whole numbers
  \item Topic: Recreation such as sports, games, exercises, music, movie, dancing, painting, fishing and other recreation activities
  \item Grade: Primary 4
\end{itemize}

\textbf{Solution}

(a) To find the number of groups for each school, I need to divide the number of students by 8.

Sunny Primary School: $2856 \div 8 = 357$ groups

Rainbow Primary School: $1944 \div 8 = 243$ groups

Green Valley Primary School: $3528 \div 8 = 441$ groups

(b) To determine which school has the most and fewest groups, I need to compare the results:

Comparing the number of groups: $441 > 357 > 243$

Therefore, Green Valley Primary School will have the most groups with 441 groups, and Rainbow Primary School will have the fewest groups with 243 groups.

\section*{Question 11}
\textbf{Metadata}

\begin{itemize}
  \item Question ID: P4-WNDiv4d1d\_P1-WNMul2nd\_sonnet4\_Recreation\_04
  \item Primary KC: WHOLE NUMBERS | Division | dividing whole numbers up to 4 digits by 1 digit
  \item Secondary KC: WHOLE NUMBERS | Multiplication | multiplying whole numbers
  \item Topic: Recreation such as sports, games, exercises, music, movie, dancing, painting, fishing and other recreation activities
  \item Grade: Primary 4
\end{itemize}

\textbf{Solution}

Step 1: Find the number of participants by dividing the total registration fees by the fee per participant.

Number of participants = \textdollar2856 ÷ \textdollar6

To solve $2856 ÷ 6$:
$2856 ÷ 6 = 476$

So there are 476 participants.

Step 2: Find the total cost of prizes by multiplying the number of participants by the cost per prize.

Total cost of prizes = 476 × \textdollar4

To solve $476 × 4$:
$476 × 4 = 1904$

Therefore, Sarah will need \textdollar1904 to buy prizes for all the participants.

\section*{Question 12}
\textbf{Metadata}

\begin{itemize}
  \item Question ID: P4-FrAddU12\_P3-FrSmp\_sonnet4\_Recreation\_04
  \item Primary KC: FRACTIONS | Addition | adding unlike fractions with two different denominators not exceeding 12
  \item Secondary KC: FRACTIONS | Simplifying | expressing a fraction in its simplest form
  \item Topic: Recreation such as sports, games, exercises, music, movie, dancing, painting, fishing and other recreation activities
  \item Grade: Primary 4
\end{itemize}

\textbf{Solution}

To find the total practice time, I need to add $\frac{3}{4}$ and $\frac{5}{6}$.

Step 1: Find a common denominator for $\frac{3}{4}$ and $\frac{5}{6}$.
The denominators are 4 and 6.
Multiples of 4: 4, 8, 12, 16, ...
Multiples of 6: 6, 12, 18, ...
The least common multiple is 12.

Step 2: Convert both fractions to equivalent fractions with denominator 12.
$\frac{3}{4} = \frac{3 \times 3}{4 \times 3} = \frac{9}{12}$
$\frac{5}{6} = \frac{5 \times 2}{6 \times 2} = \frac{10}{12}$

Step 3: Add the fractions.
$\frac{9}{12} + \frac{10}{12} = \frac{9 + 10}{12} = \frac{19}{12}$

Step 4: Check if $\frac{19}{12}$ is in its simplest form.
Since 19 is a prime number and does not share any common factors with 12, $\frac{19}{12}$ is already in its simplest form.

Therefore, Sarah practiced piano for $\frac{19}{12}$ hours or $1\frac{7}{12}$ hours in total over the two days.

\section*{Question 13}
\textbf{Metadata}

\begin{itemize}
  \item Question ID: P4-FrSubU12\_P2-FrCmp\_sonnet4\_Recreation\_04
  \item Primary KC: FRACTIONS | Subtraction | subtracting unlike fractions with two different denominators not exceeding 12
  \item Secondary KC: FRACTIONS | Comparison and ordering | comparing and ordering fractions
  \item Topic: Recreation such as sports, games, exercises, music, movie, dancing, painting, fishing and other recreation activities
  \item Grade: Primary 4
\end{itemize}

\textbf{Solution}

(a) Finding how much longer Sarah practiced than Tom yesterday:
Sarah's practice time yesterday: $\frac{5}{6}$ hour
Tom's practice time yesterday: $\frac{3}{8}$ hour

To subtract $\frac{5}{6} - \frac{3}{8}$, I need to find a common denominator.
The LCM of 6 and 8 is 24.

$\frac{5}{6} = \frac{5 \times 4}{6 \times 4} = \frac{20}{24}$

$\frac{3}{8} = \frac{3 \times 3}{8 \times 3} = \frac{9}{24}$

$\frac{5}{6} - \frac{3}{8} = \frac{20}{24} - \frac{9}{24} = \frac{11}{24}$

Sarah practiced $\frac{11}{24}$ hour longer than Tom yesterday.

(b) Finding how much longer Sarah practiced than Tom today:
Sarah's practice time today: $\frac{7}{12}$ hour
Tom's practice time today: $\frac{1}{4}$ hour

To subtract $\frac{7}{12} - \frac{1}{4}$, I need to find a common denominator.
The LCM of 12 and 4 is 12.

$\frac{7}{12} = \frac{7}{12}$

$\frac{1}{4} = \frac{1 \times 3}{4 \times 3} = \frac{3}{12}$

$\frac{7}{12} - \frac{1}{4} = \frac{7}{12} - \frac{3}{12} = \frac{4}{12} = \frac{1}{3}$

Sarah practiced $\frac{1}{3}$ hour longer than Tom today.

(c) Comparing the differences:
Yesterday's difference: $\frac{11}{24}$
Today's difference: $\frac{1}{3}$

To compare $\frac{11}{24}$ and $\frac{1}{3}$, I need a common denominator.
The LCM of 24 and 3 is 24.

$\frac{1}{3} = \frac{1 \times 8}{3 \times 8} = \frac{8}{24}$

Comparing: $\frac{11}{24}$ and $\frac{8}{24}$

Since $\frac{11}{24} > \frac{8}{24}$, the difference in their practice time was greater yesterday.

\section*{Question 14}
\textbf{Metadata}

\begin{itemize}
  \item Question ID: P4-DcAdd2d\_P4-DcCnv2Fr\_sonnet4\_Recreation\_04
  \item Primary KC: DECIMALS | Addition | adding decimals (up to 2 decimal places)
  \item Secondary KC: DECIMALS | Conversion from decimals to fraction | expressing decimals as fractions
  \item Topic: Recreation such as sports, games, exercises, music, movie, dancing, painting, fishing and other recreation activities
  \item Grade: Primary 4
\end{itemize}

\textbf{Solution}

To find the total distance Sarah swam, I need to add the distances from all three days.

Monday: $2.75$ km
Tuesday: $1.8$ km  
Wednesday: $3.25$ km

First, I'll add the decimals:
$2.75 + 1.8 + 3.25$

To add these decimals, I need to align the decimal places:
$2.75 + 1.80 + 3.25 = 7.80$ km

Now I need to express $7.80$ as a mixed number.

$7.80 = 7 + 0.80$

To convert $0.80$ to a fraction:
$0.80 = \frac{80}{100}$

I can simplify this fraction by dividing both numerator and denominator by their greatest common factor, which is $20$:
$\frac{80}{100} = \frac{80 \div 20}{100 \div 20} = \frac{4}{5}$

Therefore, $7.80 = 7\frac{4}{5}$

Sarah swam a total distance of $7\frac{4}{5}$ km over the three days.

\section*{Question 15}
\textbf{Metadata}

\begin{itemize}
  \item Question ID: P4-DcAdd2d\_P4-DcRnd3d\_sonnet4\_Recreation\_04
  \item Primary KC: DECIMALS | Addition | adding decimals (up to 2 decimal places)
  \item Secondary KC: DECIMALS | Rounding | rounding decimals up to 3 decimal places to the nearest whole number, 1 decimal place and 2 decimal places 
  \item Topic: Recreation such as sports, games, exercises, music, movie, dancing, painting, fishing and other recreation activities
  \item Grade: Primary 4
\end{itemize}

\textbf{Solution}

Step 1: Find the total distance Sarah swam on Monday.
Distance on Monday = $2.75 + 1.48 = 4.23$ km

Step 2: Find the total distance Sarah swam over the two days.
Total distance = Distance on Monday + Distance on Tuesday
Total distance = $4.23 + 3.126 = 7.356$ km

Step 3: Round the answer to $1$ decimal place.
$7.356$ km rounded to $1$ decimal place = $7.4$ km

Therefore, Sarah swam a total distance of $7.4$ km over the two days.

\section*{Question 16}
\textbf{Metadata}

\begin{itemize}
  \item Question ID: P4-DcSub2d\_P4-DcAdd2nd\_sonnet4\_Recreation\_04
  \item Primary KC: DECIMALS | Subtraction | subtracting decimals (up to 2 decimal places)
  \item Secondary KC: DECIMALS | Addition | adding decimals
  \item Topic: Recreation such as sports, games, exercises, music, movie, dancing, painting, fishing and other recreation activities
  \item Grade: Primary 4
\end{itemize}

\textbf{Solution}

Step 1: Find the total distance Sarah swam on Monday.
Distance on Monday = Morning distance + Evening distance
Distance on Monday = $2.75 + 1.80 = 4.55$ km

Step 2: Find how much more Sarah swam on Monday than on Tuesday.
Difference = Monday's distance - Tuesday's distance
Difference = $4.55 - 3.90 = 0.65$ km

Therefore, Sarah swam $0.65$ km more on Monday than on Tuesday.

\section*{Question 17}
\textbf{Metadata}

\begin{itemize}
  \item Question ID: P4-DcDiv2d1d\_P4-DcCnv2Fr\_sonnet4\_Recreation\_04
  \item Primary KC: DECIMALS | Division | dividing decimals (up to 2 decimal places) by a 1-digit whole number
  \item Secondary KC: DECIMALS | Conversion from decimals to fraction | expressing decimals as fractions
  \item Topic: Recreation such as sports, games, exercises, music, movie, dancing, painting, fishing and other recreation activities
  \item Grade: Primary 4
\end{itemize}

\textbf{Solution}

To find how many hours Sarah practiced each day, I need to divide the total practice time by the number of days.

Total practice time = $2.4$ hours
Number of days = $3$

Hours practiced each day = $2.4 \div 3$

To divide $2.4$ by $3$:
$2.4 \div 3 = 0.8$

Now I need to express $0.8$ as a fraction in its simplest form.

$0.8 = \frac{8}{10}$

To simplify $\frac{8}{10}$, I find the greatest common factor of $8$ and $10$.
The factors of $8$ are: $1, 2, 4, 8$
The factors of $10$ are: $1, 2, 5, 10$
The greatest common factor is $2$.

$\frac{8}{10} = \frac{8 \div 2}{10 \div 2} = \frac{4}{5}$

Therefore, Sarah practiced $\frac{4}{5}$ hours each day.

\section*{Question 18}
\textbf{Metadata}

\begin{itemize}
  \item Question ID: P4-DcDiv2d1d\_P4-DcSub2nd\_sonnet4\_Recreation\_04
  \item Primary KC: DECIMALS | Division | dividing decimals (up to 2 decimal places) by a 1-digit whole number
  \item Secondary KC: DECIMALS | Subtraction | subtracting decimals
  \item Topic: Recreation such as sports, games, exercises, music, movie, dancing, painting, fishing and other recreation activities
  \item Grade: Primary 4
\end{itemize}

\textbf{Solution}

Step 1: Find how much money Sarah spent in total.
Money spent = \textdollar12.48 + \textdollar2.76 = \textdollar15.24

Step 2: Find how much money Sarah had originally.
Since she had \textdollar8.91 left after spending \textdollar15.24:
Original amount = \textdollar15.24 + \textdollar8.91 = \textdollar24.15

Step 3: Find how much money she will put into each fund.
Sarah wants to divide her remaining \textdollar8.91 equally among 3 funds.
Money for each fund = \textdollar8.91 ÷ 3 = \textdollar2.97

Therefore, Sarah will put \textdollar2.97 into each art supply fund.

\section*{Question 19}
\textbf{Metadata}

\begin{itemize}
  \item Question ID: P5-FrAddMix\_P3-FrSmp\_sonnet4\_Recreation\_04
  \item Primary KC: FRACTIONS | Addition | adding mixed numbers
  \item Secondary KC: FRACTIONS | Simplifying | expressing a fraction in its simplest form
  \item Topic: Recreation such as sports, games, exercises, music, movie, dancing, painting, fishing and other recreation activities
  \item Grade: Primary 5
\end{itemize}

\textbf{Solution}

To find the total distance Sarah ran, I need to add all four distances together.

Monday's total distance:
$2\frac{3}{8} + 1\frac{5}{8}$

Since the denominators are the same, I can add the whole numbers and fractions separately:
$= (2 + 1) + (\frac{3}{8} + \frac{5}{8})$
$= 3 + \frac{8}{8}$
$= 3 + 1$
$= 4$ km

Tuesday's total distance:
$3\frac{1}{4} + 2\frac{3}{4}$

Since the denominators are the same, I can add the whole numbers and fractions separately:
$= (3 + 2) + (\frac{1}{4} + \frac{3}{4})$
$= 5 + \frac{4}{4}$
$= 5 + 1$
$= 6$ km

Total distance over two days:
$4 + 6 = 10$ km

Alternatively, I can add all four distances at once:
$2\frac{3}{8} + 1\frac{5}{8} + 3\frac{1}{4} + 2\frac{3}{4}$

First, I need to convert $\frac{1}{4}$ and $\frac{3}{4}$ to eighths:
$\frac{1}{4} = \frac{2}{8}$ and $\frac{3}{4} = \frac{6}{8}$

So: $2\frac{3}{8} + 1\frac{5}{8} + 3\frac{2}{8} + 2\frac{6}{8}$
$= (2 + 1 + 3 + 2) + (\frac{3}{8} + \frac{5}{8} + \frac{2}{8} + \frac{6}{8})$
$= 8 + \frac{16}{8}$
$= 8 + 2$
$= 10$ km

Since 10 is already a whole number, it is in its simplest form.

Therefore, Sarah ran a total of 10 km over the two days.

\section*{Question 20}
\textbf{Metadata}

\begin{itemize}
  \item Question ID: P5-FrSubMix\_P2-FrAdd2nd\_sonnet4\_Recreation\_04
  \item Primary KC: FRACTIONS | Subtraction | subtracting mixed numbers
  \item Secondary KC: FRACTIONS | Addition | adding fractions
  \item Topic: Recreation such as sports, games, exercises, music, movie, dancing, painting, fishing and other recreation activities
  \item Grade: Primary 5
\end{itemize}

\textbf{Solution}

First, I need to find Sarah's total running time for each week.

Last week's total running time:
$4\frac{3}{4} + 2\frac{1}{6}$

To add these mixed numbers, I'll convert them to improper fractions:
$4\frac{3}{4} = \frac{19}{4}$ and $2\frac{1}{6} = \frac{13}{6}$

To add $\frac{19}{4} + \frac{13}{6}$, I need a common denominator. The LCM of 4 and 6 is 12.
$\frac{19}{4} = \frac{19 \times 3}{4 \times 3} = \frac{57}{12}$
$\frac{13}{6} = \frac{13 \times 2}{6 \times 2} = \frac{26}{12}$

$\frac{57}{12} + \frac{26}{12} = \frac{83}{12} = 6\frac{11}{12}$ hours

This week's total running time:
$3\frac{1}{3} + 1\frac{5}{6}$

Converting to improper fractions:
$3\frac{1}{3} = \frac{10}{3}$ and $1\frac{5}{6} = \frac{11}{6}$

To add $\frac{10}{3} + \frac{11}{6}$, I need a common denominator of 6:
$\frac{10}{3} = \frac{20}{6}$

$\frac{20}{6} + \frac{11}{6} = \frac{31}{6} = 5\frac{1}{6}$ hours

Now I need to find the difference by subtracting this week's time from last week's time:
$6\frac{11}{12} - 5\frac{1}{6}$

Converting to improper fractions:
$6\frac{11}{12} = \frac{83}{12}$ and $5\frac{1}{6} = \frac{31}{6}$

To subtract, I need a common denominator of 12:
$\frac{31}{6} = \frac{62}{12}$

$\frac{83}{12} - \frac{62}{12} = \frac{21}{12} = \frac{7}{4} = 1\frac{3}{4}$ hours

Therefore, Sarah spent $1\frac{3}{4}$ hours less time running this week compared to last week.

\section*{Question 21}
\textbf{Metadata}

\begin{itemize}
  \item Question ID: P5-FrMulImN\_P2-FrSub2nd\_sonnet4\_Recreation\_04
  \item Primary KC: FRACTIONS | Multiplication | multiplying a proper/improper fraction and a whole number
  \item Secondary KC: FRACTIONS | Subtraction | subtracting fractions
  \item Topic: Recreation such as sports, games, exercises, music, movie, dancing, painting, fishing and other recreation activities
  \item Grade: Primary 5
\end{itemize}

\textbf{Solution}

Step 1: Find how long Sarah practiced on Monday.
Sarah practiced for her full planned time on Monday = $2\frac{1}{4}$ hours

Step 2: Find how long Sarah practiced on Tuesday.
On Tuesday, she practiced for $\frac{5}{6}$ of her planned time.
Time practiced on Tuesday = $\frac{5}{6} \times 2\frac{1}{4}$

First, convert $2\frac{1}{4}$ to an improper fraction:
$2\frac{1}{4} = \frac{9}{4}$

Then multiply:
$\frac{5}{6} \times \frac{9}{4} = \frac{5 \times 9}{6 \times 4} = \frac{45}{24}$

Simplify $\frac{45}{24}$:
$\frac{45}{24} = \frac{15}{8} = 1\frac{7}{8}$ hours

Step 3: Find the difference between Monday and Tuesday practice times.
Difference = Monday time - Tuesday time
$= 2\frac{1}{4} - 1\frac{7}{8}$
$= \frac{9}{4} - \frac{15}{8}$

Convert to common denominator:
$\frac{9}{4} = \frac{18}{8}$

Subtract:
$\frac{18}{8} - \frac{15}{8} = \frac{3}{8}$

Therefore, Sarah practiced $\frac{3}{8}$ hours longer on Monday than on Tuesday.

\section*{Question 22}
\textbf{Metadata}

\begin{itemize}
  \item Question ID: P5-FrMulPIm\_P2-FrSub2nd\_sonnet4\_Recreation\_04
  \item Primary KC: FRACTIONS | Multiplication | multiplying a proper fraction and a proper/improper fractions
  \item Secondary KC: FRACTIONS | Subtraction | subtracting fractions
  \item Topic: Recreation such as sports, games, exercises, music, movie, dancing, painting, fishing and other recreation activities
  \item Grade: Primary 5
\end{itemize}

\textbf{Solution}

Step 1: Find how many hours Sarah practiced on Tuesday.
Tuesday practice time = $\frac{3}{5} \times 2\frac{1}{4}$

First, convert $2\frac{1}{4}$ to an improper fraction:
$2\frac{1}{4} = \frac{9}{4}$

Tuesday practice time = $\frac{3}{5} \times \frac{9}{4} = \frac{3 \times 9}{5 \times 4} = \frac{27}{20}$

Convert to mixed number: $\frac{27}{20} = 1\frac{7}{20}$ hours

Step 2: Find how many hours Sarah practiced on Wednesday.
Wednesday practice time = Tuesday practice time - $\frac{1}{6}$
Wednesday practice time = $\frac{27}{20} - \frac{1}{6}$

To subtract, find a common denominator. The LCM of 20 and 6 is 60.
$\frac{27}{20} = \frac{27 \times 3}{20 \times 3} = \frac{81}{60}$
$\frac{1}{6} = \frac{1 \times 10}{6 \times 10} = \frac{10}{60}$

Wednesday practice time = $\frac{81}{60} - \frac{10}{60} = \frac{71}{60}$

Convert to mixed number: $\frac{71}{60} = 1\frac{11}{60}$ hours

Therefore, Sarah practiced for $1\frac{11}{60}$ hours on Wednesday.

\section*{Question 23}
\textbf{Metadata}

\begin{itemize}
  \item Question ID: P5-FrMulPIm\_P5-FrCnv2Dc\_sonnet4\_Recreation\_04
  \item Primary KC: FRACTIONS | Multiplication | multiplying a proper fraction and a proper/improper fractions
  \item Secondary KC: FRACTIONS | Conversion to decimals | expressing fractions as decimals
  \item Topic: Recreation such as sports, games, exercises, music, movie, dancing, painting, fishing and other recreation activities
  \item Grade: Primary 5
\end{itemize}

\textbf{Solution}

To find how long Sarah practiced on Tuesday, I need to multiply $\frac{2}{3}$ by $\frac{3}{4}$.

Time practiced on Tuesday = $\frac{2}{3} \times \frac{3}{4}$

To multiply fractions, I multiply the numerators together and multiply the denominators together:
$\frac{2}{3} \times \frac{3}{4} = \frac{2 \times 3}{3 \times 4} = \frac{6}{12}$

I can simplify this fraction by dividing both numerator and denominator by their greatest common factor, which is 6:
$\frac{6}{12} = \frac{6 \div 6}{12 \div 6} = \frac{1}{2}$

Now I need to express $\frac{1}{2}$ as a decimal:
$\frac{1}{2} = 1 \div 2 = 0.5$

Therefore, Sarah practiced for $\frac{1}{2}$ hour or $0.5$ hours on Tuesday.

\section*{Question 24}
\textbf{Metadata}

\begin{itemize}
  \item Question ID: P5-FrMulMixN\_P2-FrCmp\_sonnet4\_Recreation\_04
  \item Primary KC: FRACTIONS | Multiplication | multiplying a mixed number and a whole number
  \item Secondary KC: FRACTIONS | Comparison and ordering | comparing and ordering fractions
  \item Topic: Recreation such as sports, games, exercises, music, movie, dancing, painting, fishing and other recreation activities
  \item Grade: Primary 5
\end{itemize}

\textbf{Solution}

First, I need to find the total hours Sarah practiced.
Sarah's total practice time = $2\frac{3}{4} \times 8$

To multiply the mixed number by a whole number:
$2\frac{3}{4} \times 8 = \frac{11}{4} \times 8 = \frac{11 \times 8}{4} = \frac{88}{4} = 22$ hours

Next, I need to find the total hours Emma practiced.
Emma's total practice time = $1\frac{5}{6} \times 12$

To multiply the mixed number by a whole number:
$1\frac{5}{6} \times 12 = \frac{11}{6} \times 12 = \frac{11 \times 12}{6} = \frac{132}{6} = 22$ hours

Now I need to compare the total practice times:
Sarah's total: 22 hours
Emma's total: 22 hours

Since $22 = 22$, both Sarah and Emma have practiced for the same amount of time.

Therefore, Sarah and Emma have practiced for the same number of hours (22 hours each), so neither has practiced more than the other.

\section*{Question 25}
\textbf{Metadata}

\begin{itemize}
  \item Question ID: P5-FrMulMixN\_P3-FrSmp\_sonnet4\_Recreation\_04
  \item Primary KC: FRACTIONS | Multiplication | multiplying a mixed number and a whole number
  \item Secondary KC: FRACTIONS | Simplifying | expressing a fraction in its simplest form
  \item Topic: Recreation such as sports, games, exercises, music, movie, dancing, painting, fishing and other recreation activities
  \item Grade: Primary 5
\end{itemize}

\textbf{Solution}

To find the total hours Sarah practices, I need to multiply the number of hours she practices each day by the number of days.

Hours per day = $2\frac{1}{4}$ hours
Number of days = 6 days

Total hours = $2\frac{1}{4} \times 6$

First, I'll convert the mixed number to an improper fraction:
$2\frac{1}{4} = \frac{2 \times 4 + 1}{4} = \frac{9}{4}$

Now I can multiply:
$\frac{9}{4} \times 6 = \frac{9 \times 6}{4} = \frac{54}{4}$

Next, I need to express this fraction in its simplest form by finding the highest common factor of 54 and 4:
Factors of 54: 1, 2, 3, 6, 9, 18, 27, 54
Factors of 4: 1, 2, 4
Highest common factor = 2

$\frac{54}{4} = \frac{54 \div 2}{4 \div 2} = \frac{27}{2}$

Converting back to a mixed number:
$\frac{27}{2} = 13\frac{1}{2}$

Therefore, Sarah practices for $13\frac{1}{2}$ hours in total during the week.

\section*{Question 26}
\textbf{Metadata}

\begin{itemize}
  \item Question ID: P5-FrMulMixN\_P5-FrCnv2Dc\_sonnet4\_Recreation\_04
  \item Primary KC: FRACTIONS | Multiplication | multiplying a mixed number and a whole number
  \item Secondary KC: FRACTIONS | Conversion to decimals | expressing fractions as decimals
  \item Topic: Recreation such as sports, games, exercises, music, movie, dancing, painting, fishing and other recreation activities
  \item Grade: Primary 5
\end{itemize}

\textbf{Solution}

To find the total number of laps Sarah completes, I need to multiply her rate by the time.

Given information:
- Rate: $2\frac{3}{4}$ laps per minute
- Time: 8 minutes

Step 1: Multiply the mixed number by the whole number.
$2\frac{3}{4} \times 8$

First, I'll convert the mixed number to an improper fraction:
$2\frac{3}{4} = \frac{2 \times 4 + 3}{4} = \frac{8 + 3}{4} = \frac{11}{4}$

Now I can multiply:
$\frac{11}{4} \times 8 = \frac{11 \times 8}{4} = \frac{88}{4} = 22$

Step 2: Express the answer as a decimal.
Since $22 = 22.0$, the decimal form is $22.0$.

Therefore, Sarah will complete $22.0$ laps in total.

\section*{Question 27}
\textbf{Metadata}

\begin{itemize}
  \item Question ID: P5-DcMul3dK\_P4-DcRnd3d\_sonnet4\_Recreation\_04
  \item Primary KC: DECIMALS | Multiplication | multiplying decimals (up to 3 decimal places) by 10, 100, 1000 and their multiples
  \item Secondary KC: DECIMALS | Rounding | rounding decimals up to 3 decimal places to the nearest whole number, 1 decimal place and 2 decimal places 
  \item Topic: Recreation such as sports, games, exercises, music, movie, dancing, painting, fishing and other recreation activities
  \item Grade: Primary 5
\end{itemize}

\textbf{Solution}

**Solution:**

**(a) Finding Sarah's total time for 30 laps:**

Sarah's time for 1 lap = $23.456$ seconds

To find her time for $30$ laps, I need to multiply:
$23.456 \times 30$

I can rewrite $30$ as $3 \times 10$:
$23.456 \times 30 = 23.456 \times 3 \times 10$

First, $23.456 \times 3 = 70.368$

Then, $70.368 \times 10 = 703.68$

Therefore, Sarah's total time for $30$ laps would be $703.68$ seconds.

**(b) Rounding to the nearest whole number:**

$703.68$ seconds

Looking at the first decimal place: $6$
Since $6 \geq 5$, I round up.

$703.68$ rounded to the nearest whole number = $704$ seconds

**(c) Rounding to 1 decimal place:**

$703.68$ seconds

Looking at the second decimal place: $8$
Since $8 \geq 5$, I round up the first decimal place from $6$ to $7$.

$703.68$ rounded to $1$ decimal place = $703.7$ seconds

\section*{Question 28}
\textbf{Metadata}

\begin{itemize}
  \item Question ID: P5-DcDiv3dK\_P4-DcCnv2Fr\_sonnet4\_Recreation\_04
  \item Primary KC: DECIMALS | Division | dividing decimals (up to 3 decimal places) by 10, 100, 1000 and their multiples
  \item Secondary KC: DECIMALS | Conversion from decimals to fraction | expressing decimals as fractions
  \item Topic: Recreation such as sports, games, exercises, music, movie, dancing, painting, fishing and other recreation activities
  \item Grade: Primary 5
\end{itemize}

\textbf{Solution}

First, I need to convert Sarah's practice time from seconds to minutes by dividing by $60$.

Sarah's practice time in seconds: $142.8$ seconds

To convert to minutes: $142.8 \div 60$

I can rewrite this as: $142.8 \div 60 = 142.8 \div (6 \times 10) = 142.8 \div 6 \div 10$

First, let me calculate $142.8 \div 6$:
$142.8 \div 6 = 23.8$

Then divide by $10$:
$23.8 \div 10 = 2.38$

So Sarah practiced for $2.38$ minutes.

Now I need to express $2.38$ as a fraction in its simplest form.

$2.38 = 2\frac{38}{100}$

First, I'll simplify the fractional part $\frac{38}{100}$:
$\frac{38}{100} = \frac{38 \div 2}{100 \div 2} = \frac{19}{50}$

Since $19$ is prime and doesn't share any common factors with $50$, this fraction is in its simplest form.

Therefore: $2.38 = 2\frac{19}{50}$

Sarah practiced for $2.38$ minutes, which can be expressed as the mixed number $2\frac{19}{50}$ minutes.

\section*{Question 29}
\textbf{Metadata}

\begin{itemize}
  \item Question ID: P5-RtFndR\_P2-DcCnvN2D\_sonnet4\_Recreation\_04
  \item Primary KC: RATE | Finding rate | finding rate given total amount and number of units
  \item Secondary KC: DECIMALS | Conversion to larger units | converting a measurement from a smaller unit to a larger unit in decimal form
  \item Topic: Recreation such as sports, games, exercises, music, movie, dancing, painting, fishing and other recreation activities
  \item Grade: Primary 5
\end{itemize}

\textbf{Solution}

To find the rate at which Sarah covers distance per lap, I need to divide the total distance by the number of laps.

Total distance = 2400 meters
Number of laps = 48 laps

Rate per lap = $\frac{\text{Total distance}}{\text{Number of laps}}$

Rate per lap = $\frac{2400}{48} = 50$ meters per lap

Now I need to convert 50 meters to kilometers.

Since 1 kilometer = 1000 meters:
50 meters = $\frac{50}{1000}$ kilometers = 0.05 kilometers

Therefore, Sarah covers distance at a rate of 0.05 kilometers per lap.

\section*{Question 30}
\textbf{Metadata}

\begin{itemize}
  \item Question ID: P5-RtFndR\_P2-DcCnvD2N\_sonnet4\_Recreation\_04
  \item Primary KC: RATE | Finding rate | finding rate given total amount and number of units
  \item Secondary KC: DECIMALS | Conversion to smaller units | converting a measurement from a larger unit in decimal form to a smaller unit
  \item Topic: Recreation such as sports, games, exercises, music, movie, dancing, painting, fishing and other recreation activities
  \item Grade: Primary 5
\end{itemize}

\textbf{Solution}

To find Sarah's average swimming speed in metres per minute, I need to find the rate and convert the units.

Step 1: Find the swimming speed in kilometres per hour.
Speed = Total distance ÷ Total time
Speed = $12.5 \div 2.5 = 5$ km/h

Step 2: Convert kilometres per hour to metres per minute.
First, convert km/h to m/h:
$5$ km/h = $5 \times 1000 = 5000$ m/h

Then, convert m/h to m/min:
$5000$ m/h = $5000 \div 60 = 83.33$ m/min (rounded to 2 decimal places)

Alternatively, we can calculate directly:
$5$ km/h = $5 \times \frac{1000}{60}$ m/min = $5 \times 16.67$ m/min = $83.33$ m/min

Therefore, Sarah's average swimming speed is $83.33$ metres per minute.

\section*{Question 31}
\textbf{Metadata}

\begin{itemize}
  \item Question ID: P5-RtFndT\_P2-DcCnvN2D\_sonnet4\_Recreation\_04
  \item Primary KC: RATE | Finding total amount | finding total amount, given rate and number of units
  \item Secondary KC: DECIMALS | Conversion to larger units | converting a measurement from a smaller unit to a larger unit in decimal form
  \item Topic: Recreation such as sports, games, exercises, music, movie, dancing, painting, fishing and other recreation activities
  \item Grade: Primary 5
\end{itemize}

\textbf{Solution}

To find the total distance Sarah ran, I need to use the formula: Total distance = Rate × Time.

Given information:
- Rate = $1.25$ km per minute
- Time = $48$ minutes

Step 1: Find the total distance in kilometers.
Total distance = Rate × Time
Total distance = $1.25 \times 48$
Total distance = $60$ km

Step 2: Convert the distance from kilometers to meters.
Since $1$ km = $1000$ m:
Total distance in meters = $60 \times 1000 = 60000$ m

Therefore, Sarah ran $60000$ meters in total.

\section*{Question 32}
\textbf{Metadata}

\begin{itemize}
  \item Question ID: P5-RtFndU\_P2-DcCnvD2N\_sonnet4\_Recreation\_04
  \item Primary KC: RATE | Finding number of unit | finding number of units given rate and total amount
  \item Secondary KC: DECIMALS | Conversion to smaller units | converting a measurement from a larger unit in decimal form to a smaller unit
  \item Topic: Recreation such as sports, games, exercises, music, movie, dancing, painting, fishing and other recreation activities
  \item Grade: Primary 5
\end{itemize}

\textbf{Solution}

To find the number of laps Sarah completed, I need to use the relationship between rate, time, and total amount.

Given information:
- Rate: $0.8$ laps per minute
- Time: $21.6$ minutes
- Need to find: Number of laps

First, I'll convert the decimal time to make the calculation easier.
$21.6$ minutes $= 21.6 \times 60 = 1296$ seconds

However, since the rate is given in laps per minute, I'll work directly with minutes.

Using the formula: Number of units = Rate $\times$ Time

Number of laps $= 0.8 \times 21.6$

To multiply these decimals:
$0.8 \times 21.6 = 0.8 \times (20 + 1.6)$
$= 0.8 \times 20 + 0.8 \times 1.6$
$= 16 + 1.28$
$= 17.28$

Therefore, Sarah completed $17.28$ laps during her training session.

\section*{Question 33}
\textbf{Metadata}

\begin{itemize}
  \item Question ID: P6-FrDivPN\_P3-FrSmp\_sonnet4\_Recreation\_04
  \item Primary KC: FRACTIONS | Division | dividing a proper fraction by a whole number
  \item Secondary KC: FRACTIONS | Simplifying | expressing a fraction in its simplest form
  \item Topic: Recreation such as sports, games, exercises, music, movie, dancing, painting, fishing and other recreation activities
  \item Grade: Primary 6
\end{itemize}

\textbf{Solution}

To find how much string each bracelet will use, I need to divide $\frac{3}{4}$ meters by 6.

$\frac{3}{4} \div 6$

To divide a fraction by a whole number, I multiply the fraction by the reciprocal of the whole number:

$\frac{3}{4} \div 6 = \frac{3}{4} \times \frac{1}{6}$

$= \frac{3 \times 1}{4 \times 6}$

$= \frac{3}{24}$

Now I need to express $\frac{3}{24}$ in its simplest form by finding the greatest common factor (GCF) of 3 and 24.

Factors of 3: 1, 3
Factors of 24: 1, 2, 3, 4, 6, 8, 12, 24

The GCF of 3 and 24 is 3.

$\frac{3}{24} = \frac{3 \div 3}{24 \div 3} = \frac{1}{8}$

Therefore, each bracelet will use $\frac{1}{8}$ meters of string.

\section*{Question 34}
\textbf{Metadata}

\begin{itemize}
  \item Question ID: P6-FrDivPP\_P2-FrCmp\_sonnet4\_Recreation\_04
  \item Primary KC: FRACTIONS | Division | dividing a whole number/proper fraction by a proper fraction
  \item Secondary KC: FRACTIONS | Comparison and ordering | comparing and ordering fractions
  \item Topic: Recreation such as sports, games, exercises, music, movie, dancing, painting, fishing and other recreation activities
  \item Grade: Primary 6
\end{itemize}

\textbf{Solution}

First, I need to find how many practice sessions Sarah had each day by dividing her total practice time by the length of each session.

**Monday:**
Total practice time = $2\frac{1}{4}$ hours = $\frac{9}{4}$ hours
Length of each session = $\frac{3}{8}$ hour

Number of sessions = $\frac{9}{4} \div \frac{3}{8}$
$= \frac{9}{4} \times \frac{8}{3}$
$= \frac{9 \times 8}{4 \times 3}$
$= \frac{72}{12}$
$= 6$ sessions

**Tuesday:**
Total practice time = $1\frac{5}{6}$ hours = $\frac{11}{6}$ hours
Length of each session = $\frac{2}{9}$ hour

Number of sessions = $\frac{11}{6} \div \frac{2}{9}$
$= \frac{11}{6} \times \frac{9}{2}$
$= \frac{11 \times 9}{6 \times 2}$
$= \frac{99}{12}$
$= 8\frac{1}{4}$ sessions

Since we cannot have a fraction of a session, Tuesday had 8 complete sessions.

**Comparing the number of sessions:**
Monday: 6 sessions
Tuesday: 8 sessions

Since $8 > 6$, Tuesday had more practice sessions.

Difference = $8 - 6 = 2$ sessions

Therefore, Sarah had more practice sessions on Tuesday, with 2 more sessions than Monday.

\section*{Question 35}
\textbf{Metadata}

\begin{itemize}
  \item Question ID: P6-FrDivPP\_P3-FrSmp\_sonnet4\_Recreation\_04
  \item Primary KC: FRACTIONS | Division | dividing a whole number/proper fraction by a proper fraction
  \item Secondary KC: FRACTIONS | Simplifying | expressing a fraction in its simplest form
  \item Topic: Recreation such as sports, games, exercises, music, movie, dancing, painting, fishing and other recreation activities
  \item Grade: Primary 6
\end{itemize}

\textbf{Solution}

To find how many pieces Sarah can practice, I need to divide the total practice time by the time spent on each piece.

Total practice time = $\frac{3}{4}$ hour
Time per piece = $\frac{1}{8}$ hour

Number of pieces = $\frac{3}{4} \div \frac{1}{8}$

To divide fractions, I multiply by the reciprocal:
$\frac{3}{4} \div \frac{1}{8} = \frac{3}{4} \times \frac{8}{1}$

$= \frac{3 \times 8}{4 \times 1}$

$= \frac{24}{4}$

$= 6$

Since 6 is a whole number, it is already in its simplest form.

Therefore, Sarah can practice 6 pieces.

\section*{Question 36}
\textbf{Metadata}

\begin{itemize}
  \item Question ID: P6-PcFndWN\_P1-WNSub2nd\_sonnet4\_Recreation\_04
  \item Primary KC: PERCENTAGE | Finding the whole | finding the whole given a part and the percentage
  \item Secondary KC: WHOLE NUMBERS | Subtraction | subtracting whole numbers
  \item Topic: Recreation such as sports, games, exercises, music, movie, dancing, painting, fishing and other recreation activities
  \item Grade: Primary 6
\end{itemize}

\textbf{Solution}

Step 1: Find Sarah's team's total score for the entire game.
Since 42 points represents 60\% of the total score:
Let the total score be $x$ points.
$60\% \times x = 42$
$\frac{60}{100} \times x = 42$
$0.6x = 42$
$x = 42 \div 0.6 = 70$

Sarah's team scored 70 points in total.

Step 2: Find how many points the opponents scored in the second half.
The opponents scored 23 points fewer than Sarah's team's total score.
Opponents' score in second half $= 70 - 23 = 47$ points

Therefore, the opponents scored 47 points in the second half.

\section*{Question 37}
\textbf{Metadata}

\begin{itemize}
  \item Question ID: P6-PcFndChg\_P1-WNSub2nd\_sonnet4\_Recreation\_04
  \item Primary KC: PERCENTAGE | Finding change | finding percentage increase/decrease
  \item Secondary KC: WHOLE NUMBERS | Subtraction | subtracting whole numbers
  \item Topic: Recreation such as sports, games, exercises, music, movie, dancing, painting, fishing and other recreation activities
  \item Grade: Primary 6
\end{itemize}

\textbf{Solution}

Step 1: Find Sarah's free throw accuracy for last month.
Last month's accuracy = $\frac{84}{120} \times 100\% = 70\%$

Step 2: Find Sarah's free throw accuracy for this month.
This month's accuracy = $\frac{105}{150} \times 100\% = 70\%$

Step 3: Find the change in accuracy.
Change in accuracy = This month's accuracy - Last month's accuracy
Change in accuracy = $70\% - 70\% = 0\%$

Step 4: Since there is no change in accuracy, there is no percentage decrease.
Percentage decrease = $0\%$

Therefore, there is no percentage decrease in Sarah's free throw accuracy from last month to this month. Her accuracy remained the same at $70\%$.

\section*{Question 38}
\textbf{Metadata}

\begin{itemize}
  \item Question ID: P6-PcFndChg\_P1-WNMul2nd\_sonnet4\_Recreation\_04
  \item Primary KC: PERCENTAGE | Finding change | finding percentage increase/decrease
  \item Secondary KC: WHOLE NUMBERS | Multiplication | multiplying whole numbers
  \item Topic: Recreation such as sports, games, exercises, music, movie, dancing, painting, fishing and other recreation activities
  \item Grade: Primary 6
\end{itemize}

\textbf{Solution}

Step 1: Find how many more successful free throws Sarah made this month compared to last month.
Increase in successful free throws = $63 - 45 = 18$

Step 2: Calculate the points earned last month.
Points earned last month = $45 \times 3 = 135$ points

Step 3: Calculate the points earned this month.
Points earned this month = $63 \times 3 = 189$ points

Step 4: Find how many more points Sarah earned this month.
Increase in points = $189 - 135 = 54$ points

Step 5: Calculate the percentage increase in successful free throws.
Percentage increase = $\frac{\text{Increase}}{\text{Original amount}} \times 100\%$
Percentage increase = $\frac{18}{45} \times 100\% = \frac{2}{5} \times 100\% = 40\%$

Therefore, Sarah earned $54$ more points this month compared to last month, and her successful free throws increased by $40\%$.

\section*{Question 39}
\textbf{Metadata}

\begin{itemize}
  \item Question ID: P6-RoFndDvqWN\_P1-WNAdd2nd\_sonnet4\_Recreation\_04
  \item Primary KC: RATIO | Finding divided quantities | dividing a given quantity in a given ratio
  \item Secondary KC: WHOLE NUMBERS | Addition | adding whole numbers
  \item Topic: Recreation such as sports, games, exercises, music, movie, dancing, painting, fishing and other recreation activities
  \item Grade: Primary 6
\end{itemize}

\textbf{Solution}

Step 1: Find the initial number of participants in each game by dividing the total in the given ratio.

The ratio is chess : checkers : card games = $3:5:4$
Total ratio parts = $3 + 5 + 4 = 12$

Initial participants in chess = $\frac{3}{12} \times 84 = \frac{1}{4} \times 84 = 21$

Initial participants in checkers = $\frac{5}{12} \times 84 = 5 \times 7 = 35$

Initial participants in card games = $\frac{4}{12} \times 84 = \frac{1}{3} \times 84 = 28$

Step 2: Add the additional participants to each game.

Final participants in chess = $21 + 15 = 36$

Final participants in checkers = $35 + 12 = 47$

Final participants in card games = $28 + 9 = 37$

Therefore, after the additional participants join, there are $36$ participants in chess, $47$ participants in checkers, and $37$ participants in card games.

\section*{Question 40}
\textbf{Metadata}

\begin{itemize}
  \item Question ID: P6-RoFndRoWN\_P1-WNSub2nd\_sonnet4\_Recreation\_04
  \item Primary KC: RATIO | Finding ratio | finding the ratio of two or three given whole numbers
  \item Secondary KC: WHOLE NUMBERS | Subtraction | subtracting whole numbers
  \item Topic: Recreation such as sports, games, exercises, music, movie, dancing, painting, fishing and other recreation activities
  \item Grade: Primary 6
\end{itemize}

\textbf{Solution}

First, I need to find the number of remaining items after the damaged ones are removed.

Remaining paint brushes = $120 - 15 = 105$
Remaining paint palettes = $80 - 12 = 68$
Remaining easels = $40 - 8 = 32$

Next, I need to find the ratio of paint brushes to paint palettes to easels.
The ratio is $105 : 68 : 32$

To simplify this ratio, I need to find the greatest common divisor (GCD) of $105$, $68$, and $32$.

Factors of $105$: $1, 3, 5, 7, 15, 21, 35, 105$
Factors of $68$: $1, 2, 4, 17, 34, 68$
Factors of $32$: $1, 2, 4, 8, 16, 32$

The only common factor of all three numbers is $1$.

Therefore, the ratio $105 : 68 : 32$ is already in its simplest form.

The ratio of the remaining paint brushes to paint palettes to easels is $105 : 68 : 32$.

\section*{Question 41}
\textbf{Metadata}

\begin{itemize}
  \item Question ID: P6-RoFndRoWN\_P1-WNDiv2nd\_sonnet4\_Recreation\_04
  \item Primary KC: RATIO | Finding ratio | finding the ratio of two or three given whole numbers
  \item Secondary KC: WHOLE NUMBERS | Division | dividing whole numbers
  \item Topic: Recreation such as sports, games, exercises, music, movie, dancing, painting, fishing and other recreation activities
  \item Grade: Primary 6
\end{itemize}

\textbf{Solution}

To find the ratio of Team A's time to Team B's time to Team C's time, I need to write the times as a ratio and simplify it.

Step 1: Write the ratio of the three times.
Team A : Team B : Team C = $120 : 180 : 240$

Step 2: Find the greatest common divisor (GCD) of 120, 180, and 240.
To find the GCD, I will use division to find the common factors:
- $120 = 2^3 \times 3 \times 5$
- $180 = 2^2 \times 3^2 \times 5$  
- $240 = 2^4 \times 3 \times 5$

The GCD is $2^2 \times 3 \times 5 = 4 \times 3 \times 5 = 60$

Step 3: Divide each number by the GCD to simplify the ratio.
$120 \div 60 = 2$
$180 \div 60 = 3$
$240 \div 60 = 4$

Therefore, the ratio of Team A's time to Team B's time to Team C's time in its simplest form is $2 : 3 : 4$.

\section*{Question 42}
\textbf{Metadata}

\begin{itemize}
  \item Question ID: P6-RoFndTmWN\_P1-WNAdd2nd\_sonnet4\_Recreation\_04
  \item Primary KC: RATIO | Finding a missing term | finding the missing term in a pair of equivalent ratios
  \item Secondary KC: WHOLE NUMBERS | Addition | adding whole numbers
  \item Topic: Recreation such as sports, games, exercises, music, movie, dancing, painting, fishing and other recreation activities
  \item Grade: Primary 6
\end{itemize}

\textbf{Solution}

First, I need to find how many games Team A has lost so far using the given ratio.

Given ratio of wins to losses = $3:5$
Team A has won $18$ games.

To find the missing term (number of losses), I'll set up equivalent ratios:
$\frac{3}{5} = \frac{18}{\text{losses}}$

Cross multiply: $3 \times \text{losses} = 5 \times 18$
$3 \times \text{losses} = 90$
$\text{losses} = 90 \div 3 = 30$

So Team A has lost $30$ games.

Total games played so far = wins + losses = $18 + 30 = 48$ games

Team A then plays $4$ more games, so:
Total games played = $48 + 4 = 52$ games

Therefore, Team A will have played a total of $52$ games.

\section*{Question 43}
\textbf{Metadata}

\begin{itemize}
  \item Question ID: P6-AgRepLrEx\_P6-AgEvlLrEx\_sonnet4\_Recreation\_04
  \item Primary KC: ALGEBRA | Representation and concept | translation of real-world situations into linear algebraic expressions
  \item Secondary KC: ALGEBRA | Evaluation | evaluating linear expressions by substitution
  \item Topic: Recreation such as sports, games, exercises, music, movie, dancing, painting, fishing and other recreation activities
  \item Grade: Primary 6
\end{itemize}

\textbf{Solution}

Step 1: Identify the number of laps each day.

Monday: Morning = $a$ laps, Evening = $b$ laps
Total for Monday = $a + b$

Tuesday: Morning = $2a$ laps, Evening = $b + 5$ laps
Total for Tuesday = $2a + (b + 5) = 2a + b + 5$

Wednesday: Morning = $3a$ laps, Evening = $b$ laps
Total for Wednesday = $3a + b$

Step 2: Write the algebraic expression for total laps over three days.

Total laps = Monday + Tuesday + Wednesday
Total laps = $(a + b) + (2a + b + 5) + (3a + b)$
Total laps = $a + b + 2a + b + 5 + 3a + b$
Total laps = $6a + 3b + 5$

Step 3: Substitute $a = 12$ and $b = 8$ to find the total number of laps.

Total laps = $6(12) + 3(8) + 5$
Total laps = $72 + 24 + 5$
Total laps = $101$

Therefore, the algebraic expression is $6a + 3b + 5$, and Sarah swam a total of $101$ laps over the three days.

\section*{Question 44}
\textbf{Metadata}

\begin{itemize}
  \item Question ID: P6-AgSlvLrN\_P6-AgRepLrEx\_sonnet4\_Recreation\_04
  \item Primary KC: ALGEBRA | Solving simple linear equations | solving linear equations involving whole number coefficient and one variable only
  \item Secondary KC: ALGEBRA | Representation and concept | translation of real-world situations into linear algebraic expressions
  \item Topic: Recreation such as sports, games, exercises, music, movie, dancing, painting, fishing and other recreation activities
  \item Grade: Primary 6
\end{itemize}

\textbf{Solution}

Let $x$ be the number of coins Sarah collected in the first level.

From the problem:
- First level: $x$ coins
- Second level: $3x$ coins (3 times as many as first level)
- Third level: $x + 15$ coins (15 more than first level)

The total number of coins collected is 195, so we can write the equation:
$x + 3x + (x + 15) = 195$

Simplifying the left side:
$x + 3x + x + 15 = 195$
$5x + 15 = 195$

Subtracting 15 from both sides:
$5x = 195 - 15$
$5x = 180$

Dividing both sides by 5:
$x = 36$

Therefore:
- First level: $36$ coins
- Second level: $3 \times 36 = 108$ coins
- Third level: $36 + 15 = 51$ coins

Checking: $36 + 108 + 51 = 195$ ✓

Sarah collected 36 coins in the first level, 108 coins in the second level, and 51 coins in the third level.

\section*{Question 45}
\textbf{Metadata}

\begin{itemize}
  \item Question ID: O1-RoRepFr\_P2-FrSub2nd\_sonnet4\_Recreation\_04
  \item Primary KC: RATIO | Representation and concept | ratios involving fractions
  \item Secondary KC: FRACTIONS | Subtraction | subtracting fractions
  \item Topic: Recreation such as sports, games, exercises, music, movie, dancing, painting, fishing and other recreation activities
  \item Grade: Secondary O-level 1
\end{itemize}

\textbf{Solution}

First, I need to find each swimmer's time in the second practice session.

Sarah's time in second session:
Sarah's first session time: $\frac{3}{4}$ hours
Reduction: $\frac{1}{6}$ hours
Sarah's second session time = $\frac{3}{4} - \frac{1}{6}$

To subtract these fractions, I need a common denominator:
LCM of 4 and 6 is 12
$\frac{3}{4} = \frac{9}{12}$ and $\frac{1}{6} = \frac{2}{12}$

Sarah's second session time = $\frac{9}{12} - \frac{2}{12} = \frac{7}{12}$ hours

Emma's time in second session:
Emma's first session time: $\frac{5}{6}$ hours
Reduction: $\frac{1}{4}$ hours
Emma's second session time = $\frac{5}{6} - \frac{1}{4}$

To subtract these fractions, I need a common denominator:
LCM of 6 and 4 is 12
$\frac{5}{6} = \frac{10}{12}$ and $\frac{1}{4} = \frac{3}{12}$

Emma's second session time = $\frac{10}{12} - \frac{3}{12} = \frac{7}{12}$ hours

Now I can find the ratio of Sarah's time to Emma's time in the second session:
Ratio = Sarah's time : Emma's time = $\frac{7}{12} : \frac{7}{12}$

To simplify this ratio involving fractions:
$\frac{7}{12} : \frac{7}{12} = \frac{7}{12} \div \frac{7}{12} = \frac{7}{12} \times \frac{12}{7} = 1$

Therefore, the ratio is $1:1$.

The ratio of Sarah's swimming time to Emma's swimming time in the second practice session is $1:1$.

\section*{Question 46}
\textbf{Metadata}

\begin{itemize}
  \item Question ID: O1-RoRepFr\_P6-FrDiv2nd\_sonnet4\_Recreation\_04
  \item Primary KC: RATIO | Representation and concept | ratios involving fractions
  \item Secondary KC: FRACTIONS | Division | fraction division
  \item Topic: Recreation such as sports, games, exercises, music, movie, dancing, painting, fishing and other recreation activities
  \item Grade: Secondary O-level 1
\end{itemize}

\textbf{Solution}

Step 1: Calculate Sarah's total practice time in one week.
Sarah practices $\frac{3}{4}$ hours daily for 7 days.
Sarah's total time = $\frac{3}{4} \times 7 = \frac{21}{4}$ hours

Step 2: Calculate Emma's total practice time in one week.
Emma practices $\frac{1}{2}$ hour daily for 7 days.
Emma's total time = $\frac{1}{2} \times 7 = \frac{7}{2}$ hours

Step 3: Find the ratio of Sarah's total practice time to Emma's total practice time.
Ratio = Sarah's time : Emma's time = $\frac{21}{4} : \frac{7}{2}$

To simplify this ratio, we divide the first term by the second term:
$\frac{21}{4} \div \frac{7}{2} = \frac{21}{4} \times \frac{2}{7} = \frac{42}{28} = \frac{3}{2}$

Therefore, the ratio is $\frac{3}{2} : 1$ or $3 : 2$

Step 4: Calculate how much time Sarah will spend on each piece.
Sarah needs to divide her total practice time of $\frac{21}{4}$ hours equally among 3 pieces.
Time per piece = $\frac{21}{4} \div 3 = \frac{21}{4} \times \frac{1}{3} = \frac{21}{12} = \frac{7}{4}$ hours

Therefore, the ratio of Sarah's practice time to Emma's practice time is $3 : 2$, and Sarah will spend $\frac{7}{4}$ hours or $1\frac{3}{4}$ hours practicing each piece.

\section*{Question 47}
\textbf{Metadata}

\begin{itemize}
  \item Question ID: O1-RoRepDc\_P4-DcSub2nd\_sonnet4\_Recreation\_04
  \item Primary KC: RATIO | Representation and concept | ratios involving decimals
  \item Secondary KC: DECIMALS | Subtraction | subtracting decimals
  \item Topic: Recreation such as sports, games, exercises, music, movie, dancing, painting, fishing and other recreation activities
  \item Grade: Secondary O-level 1
\end{itemize}

\textbf{Solution}

Step 1: Calculate each player's final score after the deduction.

Alex's final score = $24.8 - 5.4 = 19.4$ points

Ben's final score = $18.6 - 5.4 = 13.2$ points

Charlie's final score = $31.2 - 5.4 = 25.8$ points

Step 2: Express the ratio of their final scores.

The ratio is Alex : Ben : Charlie = $19.4 : 13.2 : 25.8$

Step 3: Simplify the ratio by converting decimals to whole numbers.

Multiply all terms by 10 to eliminate one decimal place:
$19.4 : 13.2 : 25.8 = 194 : 132 : 258$

Step 4: Find the greatest common divisor (GCD) of 194, 132, and 258.

$194 = 2 \times 97$
$132 = 2^2 \times 3 \times 11$
$258 = 2 \times 3 \times 43$

The GCD is $2$.

Step 5: Divide each term by the GCD.

$194 \div 2 = 97$
$132 \div 2 = 66$
$258 \div 2 = 129$

Therefore, the ratio of Alex's final score to Ben's final score to Charlie's final score is $97 : 66 : 129$.

\section*{Question 48}
\textbf{Metadata}

\begin{itemize}
  \item Question ID: O1-PcRep2q\_O1-PcCnv2Dc\_sonnet4\_Recreation\_04
  \item Primary KC: PERCENTAGE | Representation and concept | comparing two quantities by percentage
  \item Secondary KC: PERCENTAGE | Conversion to decimals | expressing percentage as a decimal
  \item Topic: Recreation such as sports, games, exercises, music, movie, dancing, painting, fishing and other recreation activities
  \item Grade: Secondary O-level 1
\end{itemize}

\textbf{Solution}

To find each player's success rate as a percentage:

Sarah's success rate:
Success rate = $\frac{\text{shots made}}{\text{total attempts}} \times 100\%$
Sarah's success rate = $\frac{24}{40} \times 100\% = 0.6 \times 100\% = 60\%$

Mike's success rate:
Mike's success rate = $\frac{18}{30} \times 100\% = 0.6 \times 100\% = 60\%$

Converting percentages to decimals:
Sarah's success rate as decimal: $60\% = \frac{60}{100} = 0.6$
Mike's success rate as decimal: $60\% = \frac{60}{100} = 0.6$

Comparison: Both Sarah and Mike have the same success rate of $60\%$ or $0.6$ as a decimal. Therefore, they performed equally well in the basketball shooting competition.

\section*{Question 49}
\textbf{Metadata}

\begin{itemize}
  \item Question ID: O1-PcFndRslt\_P1-WNDiv2nd\_sonnet4\_Recreation\_04
  \item Primary KC: PERCENTAGE | Finding result after change | increasing/decreasing a quantity by a given percentage
  \item Secondary KC: WHOLE NUMBERS | Division | dividing whole numbers
  \item Topic: Recreation such as sports, games, exercises, music, movie, dancing, painting, fishing and other recreation activities
  \item Grade: Secondary O-level 1
\end{itemize}

\textbf{Solution}

Step 1: Find the average points per game in the first 12 games.
Average points per game = $480 \div 12 = 40$ points per game

Step 2: Find the points per game in the next 8 games after a 25% increase.
25% of 40 = $40 \times \frac{25}{100} = 40 \times 0.25 = 10$ points
New points per game = $40 + 10 = 50$ points per game

Step 3: Find the total points scored in the next 8 games.
Total points in next 8 games = $50 \times 8 = 400$ points

Step 4: Find the total points scored in all 20 games.
Total points = Points from first 12 games + Points from next 8 games
Total points = $480 + 400 = 880$ points

Therefore, the team scored 880 points in all 20 games.

\section*{Question 50}
\textbf{Metadata}

\begin{itemize}
  \item Question ID: O1-AgSlvFrLr\_O1-AgRepEq\_sonnet4\_Recreation\_04
  \item Primary KC: ALGEBRA | Solving | solving simple fractional equations that can be reduced to linear equations
  \item Secondary KC: ALGEBRA | Representation and concept | translation of simple real-world situations to equations
  \item Topic: Recreation such as sports, games, exercises, music, movie, dancing, painting, fishing and other recreation activities
  \item Grade: Secondary O-level 1
\end{itemize}

\textbf{Solution}

Let me solve this step by step.

First, I need to find Sarah's swimming speed.

Let $d$ be the total planned distance.
From the given information: Sarah completes $\frac{2}{3}d$ in 24 minutes.

So her speed is: $\text{speed} = \frac{\frac{2}{3}d}{24} = \frac{2d}{3 \times 24} = \frac{2d}{72} = \frac{d}{36}$ distance per minute.

Next, I'll find what fraction she should complete in 12 minutes.

If Sarah wants to complete the entire distance $d$ in 30 minutes, her required speed should be:
$\text{required speed} = \frac{d}{30}$ distance per minute.

However, her actual speed is $\frac{d}{36}$ distance per minute, which is slower than required.

At her actual speed of $\frac{d}{36}$ distance per minute, in 12 minutes she will complete:
$\text{distance in 12 minutes} = 12 \times \frac{d}{36} = \frac{12d}{36} = \frac{d}{3}$

Therefore, Sarah should complete $\frac{1}{3}$ of her planned distance in the first 12 minutes to maintain her current pace.

Note: At this pace, she will actually need 36 minutes to complete the full distance, not her target of 30 minutes.

\section*{Question 51}
\textbf{Metadata}

\begin{itemize}
  \item Question ID: O2-RoRepIvP\_P1-WNMul2nd\_sonnet4\_Recreation\_04
  \item Primary KC: RATIO | Representation and concept | inverse proportion
  \item Secondary KC: WHOLE NUMBERS | Multiplication | multiplying whole numbers
  \item Topic: Recreation such as sports, games, exercises, music, movie, dancing, painting, fishing and other recreation activities
  \item Grade: Secondary O-level 2
\end{itemize}

\textbf{Solution}

Let me solve this step by step.

**Step 1: Understanding the inverse proportion relationship**

Since the number of practice hours per day is inversely proportional to the number of days left, we have:
$h \times d = k$ (where $k$ is a constant)

Where $h$ = hours per day and $d$ = days left

**Step 2: Find the constant of proportionality**

When $d = 20$ days, $h = 3$ hours per day:
$k = h \times d = 3 \times 20 = 60$

So the relationship is: $h \times d = 60$

**Step 3: Find when they should start the 12-hour daily practice**

When $h = 12$ hours per day:
$12 \times d = 60$
$d = \frac{60}{12} = 5$

Therefore, they should start the intensive 12-hour daily practice when they have 5 days left before the concert.

**Step 4: Calculate the total practice hours**

Using the constant $k = 60$, the total practice hours for any period following this inverse proportion is:
Total hours = $k = 60$ hours

Alternatively, we can verify: $12 \text{ hours/day} \times 5 \text{ days} = 60$ hours

**Answer:**
- They should start the 12-hour daily practice schedule when they have 5 days left before the concert
- The total number of practice hours for their entire preparation period will be 60 hours

\section*{Question 52}
\textbf{Metadata}

\begin{itemize}
  \item Question ID: O2-RoRepIvP\_P1-WNDiv2nd\_sonnet4\_Recreation\_04
  \item Primary KC: RATIO | Representation and concept | inverse proportion
  \item Secondary KC: WHOLE NUMBERS | Division | dividing whole numbers
  \item Topic: Recreation such as sports, games, exercises, music, movie, dancing, painting, fishing and other recreation activities
  \item Grade: Secondary O-level 2
\end{itemize}

\textbf{Solution}

Step 1: Understand the inverse proportion relationship.
Since the prize money per team is inversely proportional to the number of teams, we have:
Prize money per team $\times$ Number of teams $= k$ (constant)

Step 2: Find the constant $k$ using the first tournament data.
For the first tournament: $450 \times 8 = k$
$k = 3600$

Step 3: Calculate the prize money per team in the second tournament.
For the second tournament with 12 teams:
Prize money per team $\times 12 = 3600$
Prize money per team $= 3600 \div 12 = 300$

Step 4: Calculate the money each team member receives.
Each team receives \textdollar300
Each team has 6 members
Money per team member $= 300 \div 6 = 50$

Therefore, each team in the second tournament received \textdollar300, and each team member received \textdollar50.

\section*{Question 53}
\textbf{Metadata}

\begin{itemize}
  \item Question ID: O2-AgSlvLr2v\_O1-AgRepEq\_sonnet4\_Recreation\_04
  \item Primary KC: ALGEBRA | Solving | solving linear equations in two variables
  \item Secondary KC: ALGEBRA | Representation and concept | translation of simple real-world situations to equations
  \item Topic: Recreation such as sports, games, exercises, music, movie, dancing, painting, fishing and other recreation activities
  \item Grade: Secondary O-level 2
\end{itemize}

\textbf{Solution}

Let $x$ be the cost of the gaming headset in dollars and $y$ be the cost of the gaming chair in dollars.

From the given information, we can set up two equations:

Equation 1: The total cost is \textdollar350
$x + y = 350$

Equation 2: The gaming chair costs \textdollar50 more than twice the price of the gaming headset
$y = 2x + 50$

Now we solve this system of linear equations by substitution.

Substitute equation 2 into equation 1:
$x + (2x + 50) = 350$
$x + 2x + 50 = 350$
$3x + 50 = 350$
$3x = 300$
$x = 100$

Substitute $x = 100$ back into equation 2:
$y = 2(100) + 50$
$y = 200 + 50$
$y = 250$

Therefore, the gaming headset costs \textdollar100 and the gaming chair costs \textdollar250.

Verification:
- Total cost: $100 + 250 = 350$ ✓
- Chair cost check: $2(100) + 50 = 250$ ✓

\section*{Question 54}
\textbf{Metadata}

\begin{itemize}
  \item Question ID: O2-SPFndmdn\_O2-SPFndmode\_sonnet4\_Recreation\_04
  \item Primary KC: STATISTICS AND PROBABILITY | Finding median | Finding median for a set of data
  \item Secondary KC: STATISTICS AND PROBABILITY | Finding mode | Finding mode for a set of data
  \item Topic: Recreation such as sports, games, exercises, music, movie, dancing, painting, fishing and other recreation activities
  \item Grade: Secondary O-level 2
\end{itemize}

\textbf{Solution}

(a) Finding the median score:

First, arrange the scores in ascending order:
$78, 78, 78, 82, 82, 85, 85, 88, 91, 91, 94$

Since there are $11$ scores (odd number), the median is the middle value, which is the $6$th value.

Therefore, the median score is $85$ points.

(b) Finding the mode:

Count the frequency of each score:
- $78$ appears $3$ times
- $82$ appears $2$ times  
- $85$ appears $2$ times
- $88$ appears $1$ time
- $91$ appears $2$ times
- $94$ appears $1$ time

The score $78$ appears most frequently ($3$ times).

Therefore, the mode is $78$ points.

(c) Finding the lowest possible score for the next game:

For the mode to remain $78$, the score $78$ must still appear most frequently after adding the new score.

Currently, $78$ appears $3$ times, and all other scores appear at most $2$ times.

For the mode to remain unchanged, the new score cannot be $82$, $85$, or $91$ (as these would then appear $3$ times each, creating multiple modes including values other than $78$).

The median of the current $11$ scores is $85$. For the next game's score to be higher than the median, it must be greater than $85$.

The possible scores greater than $85$ that won't change the mode are: $86, 87, 88, 89, 90, 92, 93, 94$, or any score higher than $94$.

Among these options, the lowest possible score is $86$ points.

\section*{Question 55}
\textbf{Metadata}

\begin{itemize}
  \item Question ID: O2-SPFndmdn\_O3-SPFndPctl\_sonnet4\_Recreation\_04
  \item Primary KC: STATISTICS AND PROBABILITY | Finding median | Finding median for a set of data
  \item Secondary KC: STATISTICS AND PROBABILITY | Finding percentiles | finding percentiles for a set of data
  \item Topic: Recreation such as sports, games, exercises, music, movie, dancing, painting, fishing and other recreation activities
  \item Grade: Secondary O-level 2
\end{itemize}

\textbf{Solution}

**Step 1: Find the median height**

First, arrange the heights in ascending order:
$165, 168, 169, 170, 171, 172, 173, 174, 175, 176, 177, 178, 179, 180, 182$

Since there are $n = 15$ players (odd number), the median is the middle value.
Position of median = $\frac{n+1}{2} = \frac{15+1}{2} = 8$th position

The 8th value in the ordered list is $174$ cm.

Therefore, the median height is $174$ cm.

**Step 2: Find the percentile for a player with height 175 cm**

To find the percentile of a player with height $175$ cm:

Count the number of players with heights less than $175$ cm:
$165, 168, 169, 170, 171, 172, 173, 174$ → $8$ players

Count the number of players with height equal to $175$ cm:
$175$ → $1$ player

Using the percentile formula:
Percentile = $\frac{\text{Number of values below} + 0.5 \times \text{Number of values equal}}{\text{Total number of values}} \times 100\%$

Percentile = $\frac{8 + 0.5 \times 1}{15} \times 100\% = \frac{8.5}{15} \times 100\% = 56.67\%$

**Answer:**
- The median height of the team is $174$ cm
- A player with height $175$ cm falls into the $56.67$th percentile (approximately $57$th percentile)

\section*{Question 56}
\textbf{Metadata}

\begin{itemize}
  \item Question ID: O3-BPOpr\_O3-BPRepPosI\_sonnet4\_Recreation\_04
  \item Primary KC: BASE AND POWER | Operations | laws of indices
  \item Secondary KC: BASE AND POWER | Representation and concept  | positive indices that is not 1
  \item Topic: Recreation such as sports, games, exercises, music, movie, dancing, painting, fishing and other recreation activities
  \item Grade: Secondary O-level 3/4
\end{itemize}

\textbf{Solution}

To find Sarah's final score, I need to track her score through each round using the laws of indices.

Starting score: $3$ points = $3^1$ points

After first round: $3^4$ points

After second round: Her score is multiplied by $3^6$
Using the law $a^m \times a^n = a^{m+n}$:
Score = $3^4 \times 3^6 = 3^{4+6} = 3^{10}$ points

After third round: Her score is divided by $3^2$
Using the law $a^m \div a^n = a^{m-n}$:
Score = $3^{10} \div 3^2 = 3^{10-2} = 3^8$ points

Therefore, Sarah's final score is $3^8$ points, so $n = 8$.

\section*{Question 57}
\textbf{Metadata}

\begin{itemize}
  \item Question ID: O3-BPOpr\_O3-BPRepFrI\_sonnet4\_Recreation\_04
  \item Primary KC: BASE AND POWER | Operations | laws of indices
  \item Secondary KC: BASE AND POWER | Representation and concept  | fractional indices
  \item Topic: Recreation such as sports, games, exercises, music, movie, dancing, painting, fishing and other recreation activities
  \item Grade: Secondary O-level 3/4
\end{itemize}

\textbf{Solution}

Given information:
- At 16 meters: intensity = $I_0 \cdot 2^{-8}$
- At any distance: intensity = $I_0 \cdot 2^{-2x}$ where $x$ is the distance factor
- Amplification factor = $8^{\frac{2}{3}}$

(a) Finding the pattern for $x$:
At 16 meters: $I_0 \cdot 2^{-2x} = I_0 \cdot 2^{-8}$
Therefore: $-2x = -8$
So: $x = 4$

This means $x = \frac{\text{distance}}{4}$

At 4 meters: $x = \frac{4}{4} = 1$

(b) Sound intensity at 4 meters before amplification:
Intensity = $I_0 \cdot 2^{-2x} = I_0 \cdot 2^{-2(1)} = I_0 \cdot 2^{-2} = \frac{I_0}{4}$

(c) Calculating the amplification factor:
$8^{\frac{2}{3}} = (2^3)^{\frac{2}{3}} = 2^{3 \cdot \frac{2}{3}} = 2^2 = 4$

Final sound intensity after amplification:
$\frac{I_0}{4} \times 4 = I_0$

Therefore:
(a) $x = 1$ at 4 meters distance
(b) Sound intensity before amplification = $\frac{I_0}{4}$ units
(c) Final sound intensity after amplification = $I_0$ units

\section*{Question 58}
\textbf{Metadata}

\begin{itemize}
  \item Question ID: O3-STOprUn\_O3-STOprIns\_sonnet4\_Recreation\_04
  \item Primary KC: SET | Set operations | union of two sets
  \item Secondary KC: SET | Set operations | intersection of two sets
  \item Topic: Recreation such as sports, games, exercises, music, movie, dancing, painting, fishing and other recreation activities
  \item Grade: Secondary O-level 3/4
\end{itemize}

\textbf{Solution}

Given information:
- Total members surveyed: 50
- Members who played basketball: $|B| = 28$
- Members who played soccer: $|S| = 22$
- Members who played both sports: $|B \cap S| = 12$

(a) Finding $|B \cup S|$ (union of sets B and S):
Using the formula: $|B \cup S| = |B| + |S| - |B \cap S|$
$|B \cup S| = 28 + 22 - 12 = 38$

Therefore, 38 members played at least one of these sports.

(b) Finding eligible members for the tournament:
The tournament requires players with experience in both sports, which corresponds to the intersection $B \cap S$.
$|B \cap S| = 12$

Therefore, 12 members are eligible for the tournament.

(c) Finding members eligible for the recreational league:
We need members who played only basketball or only soccer (but not both).
- Members who played only basketball: $|B| - |B \cap S| = 28 - 12 = 16$
- Members who played only soccer: $|S| - |B \cap S| = 22 - 12 = 10$
- Total eligible for recreational league: $16 + 10 = 26$

Alternatively, this can be calculated as: $|B \cup S| - |B \cap S| = 38 - 12 = 26$

Therefore, 26 members would be eligible for the recreational league.

\section*{Question 59}
\textbf{Metadata}

\begin{itemize}
  \item Question ID: O3-MXMulSM\_O3-MXSub\_sonnet4\_Recreation\_04
  \item Primary KC: MATRICES | Multiplication | product of a scalar quantity and a matrix
  \item Secondary KC: MATRICES | Subtraction | subtraction of matrices
  \item Topic: Recreation such as sports, games, exercises, music, movie, dancing, painting, fishing and other recreation activities
  \item Grade: Secondary O-level 3/4
\end{itemize}

\textbf{Solution}

(a) To find the adjusted performance matrix, we subtract the penalty matrix $Q$ from the original performance matrix $P$:

Adjusted matrix = $P - Q$

$= \begin{pmatrix} 85 & 78 & 92 \\ 90 & 82 & 88 \\ 76 & 85 & 90 \\ 88 & 80 & 85 \end{pmatrix} - \begin{pmatrix} 5 & 3 & 4 \\ 2 & 6 & 3 \\ 4 & 2 & 5 \\ 3 & 4 & 2 \end{pmatrix}$

$= \begin{pmatrix} 85-5 & 78-3 & 92-4 \\ 90-2 & 82-6 & 88-3 \\ 76-4 & 85-2 & 90-5 \\ 88-3 & 80-4 & 85-2 \end{pmatrix}$

$= \begin{pmatrix} 80 & 75 & 88 \\ 88 & 76 & 85 \\ 72 & 83 & 85 \\ 85 & 76 & 83 \end{pmatrix}$

(b) To find the final performance matrix, we multiply the adjusted matrix by the scalar $1.2$:

Final matrix = $1.2 \times \begin{pmatrix} 80 & 75 & 88 \\ 88 & 76 & 85 \\ 72 & 83 & 85 \\ 85 & 76 & 83 \end{pmatrix}$

$= \begin{pmatrix} 1.2 \times 80 & 1.2 \times 75 & 1.2 \times 88 \\ 1.2 \times 88 & 1.2 \times 76 & 1.2 \times 85 \\ 1.2 \times 72 & 1.2 \times 83 & 1.2 \times 85 \\ 1.2 \times 85 & 1.2 \times 76 & 1.2 \times 83 \end{pmatrix}$

$= \begin{pmatrix} 96 & 90 & 105.6 \\ 105.6 & 91.2 & 102 \\ 86.4 & 99.6 & 102 \\ 102 & 91.2 & 99.6 \end{pmatrix}$

Therefore, the final performance matrix after applying both the penalty deduction and bonus multiplier is $\begin{pmatrix} 96 & 90 & 105.6 \\ 105.6 & 91.2 & 102 \\ 86.4 & 99.6 & 102 \\ 102 & 91.2 & 99.6 \end{pmatrix}$.

\section*{Question 60}
\textbf{Metadata}

\begin{itemize}
  \item Question ID: O3-MXMulSM\_O3-MXMul\_sonnet4\_Recreation\_04
  \item Primary KC: MATRICES | Multiplication | product of a scalar quantity and a matrix
  \item Secondary KC: MATRICES | Multiplication | multiplication of matrices
  \item Topic: Recreation such as sports, games, exercises, music, movie, dancing, painting, fishing and other recreation activities
  \item Grade: Secondary O-level 3/4
\end{itemize}

\textbf{Solution}

To solve this problem, I need to first apply the scalar multiplication to matrix $A$, then multiply the result by matrix $B$.

Step 1: Apply the championship factor to the base scores.
The championship scoring matrix is $1.5A$:

$1.5A = 1.5 \begin{pmatrix} 100 & 150 & 200 \\ 80 & 120 & 180 \\ 60 & 100 & 160 \end{pmatrix} = \begin{pmatrix} 150 & 225 & 300 \\ 120 & 180 & 270 \\ 90 & 150 & 240 \end{pmatrix}$

Step 2: Multiply the championship scoring matrix by the player participation matrix.
I need to calculate $(1.5A) \times B$:

$(1.5A) \times B = \begin{pmatrix} 150 & 225 & 300 \\ 120 & 180 & 270 \\ 90 & 150 & 240 \end{pmatrix} \times \begin{pmatrix} 12 & 8 \\ 15 & 10 \\ 18 & 12 \end{pmatrix}$

Calculating each element:

For Strategy games:
- Round 1: $150 \times 12 + 225 \times 15 + 300 \times 18 = 1800 + 3375 + 5400 = 10575$
- Round 2: $150 \times 8 + 225 \times 10 + 300 \times 12 = 1200 + 2250 + 3600 = 7050$

For Action games:
- Round 1: $120 \times 12 + 180 \times 15 + 270 \times 18 = 1440 + 2700 + 4860 = 9000$
- Round 2: $120 \times 8 + 180 \times 10 + 270 \times 12 = 960 + 1800 + 3240 = 6000$

For Puzzle games:
- Round 1: $90 \times 12 + 150 \times 15 + 240 \times 18 = 1080 + 2250 + 4320 = 7650$
- Round 2: $90 \times 8 + 150 \times 10 + 240 \times 12 = 720 + 1500 + 2880 = 5100$

Therefore: $(1.5A) \times B = \begin{pmatrix} 10575 & 7050 \\ 9000 & 6000 \\ 7650 & 5100 \end{pmatrix}$

The total points that can be awarded are:
- Strategy games: $10575$ points in Round 1, $7050$ points in Round 2
- Action games: $9000$ points in Round 1, $6000$ points in Round 2
- Puzzle games: $7650$ points in Round 1, $5100$ points in Round 2

\section*{Question 61}
\textbf{Metadata}

\begin{itemize}
  \item Question ID: O3-MXMul\_O3-MXSub\_sonnet4\_Recreation\_04
  \item Primary KC: MATRICES | Multiplication | multiplication of matrices
  \item Secondary KC: MATRICES | Subtraction | subtraction of matrices
  \item Topic: Recreation such as sports, games, exercises, music, movie, dancing, painting, fishing and other recreation activities
  \item Grade: Secondary O-level 3/4
\end{itemize}

\textbf{Solution}

Step 1: Find the actual participants matrix by subtracting withdrawals from registrations.

Actual participants matrix $A = R - W$:
$A = \begin{pmatrix} 15 & 12 & 18 \\ 20 & 16 & 14 \\ 22 & 19 & 25 \end{pmatrix} - \begin{pmatrix} 3 & 2 & 4 \\ 5 & 3 & 2 \\ 4 & 6 & 5 \end{pmatrix}$

$A = \begin{pmatrix} 12 & 10 & 14 \\ 15 & 13 & 12 \\ 18 & 13 & 20 \end{pmatrix}$

Step 2: Extract the first two columns of matrix $A$ (Strategy Games and Action Games only).

$A_{Strategy,Action} = \begin{pmatrix} 12 & 10 \\ 15 & 13 \\ 18 & 13 \end{pmatrix}$

Step 3: Multiply the actual participants matrix by the scoring weight matrix.

Total weighted score matrix $T = A_{Strategy,Action} \times S$:
$T = \begin{pmatrix} 12 & 10 \\ 15 & 13 \\ 18 & 13 \end{pmatrix} \times \begin{pmatrix} 2 & 3 \\ 4 & 2 \end{pmatrix}$

$T = \begin{pmatrix} 12(2) + 10(4) & 12(3) + 10(2) \\ 15(2) + 13(4) & 15(3) + 13(2) \\ 18(2) + 13(4) & 18(3) + 13(2) \end{pmatrix}$

$T = \begin{pmatrix} 24 + 40 & 36 + 20 \\ 30 + 52 & 45 + 26 \\ 36 + 52 & 54 + 26 \end{pmatrix}$

$T = \begin{pmatrix} 64 & 56 \\ 82 & 71 \\ 88 & 80 \end{pmatrix}$

Step 4: Calculate the total weighted score by summing all elements in matrix $T$.

Total weighted score $= 64 + 56 + 82 + 71 + 88 + 80 = 441$

Therefore, the total weighted score for actual participants is $441$ points.

\section*{Question 62}
\textbf{Metadata}

\begin{itemize}
  \item Question ID: O3-SPFndQtl\_O3-SPFndIQR\_sonnet4\_Recreation\_04
  \item Primary KC: STATISTICS AND PROBABILITY | Finding quartiles | finding quartiles for a set of data
  \item Secondary KC: STATISTICS AND PROBABILITY | Finding range | finding interquartile range as measures of spread for a set of data 
  \item Topic: Recreation such as sports, games, exercises, music, movie, dancing, painting, fishing and other recreation activities
  \item Grade: Secondary O-level 3/4
\end{itemize}

\textbf{Solution}

**Solution:**

**(a) Finding the quartiles:**

First, we arrange the data in ascending order (already given):
2, 4, 6, 8, 9, 10, 12, 14, 15, 16, 18, 20, 22, 24, 26

With $n = 15$ data points:

**Finding $Q_2$ (median):**
Position of $Q_2 = \frac{n+1}{2} = \frac{15+1}{2} = 8$th position
$Q_2 = 14$ points

**Finding $Q_1$:**
For the lower half (positions 1-7): 2, 4, 6, 8, 9, 10, 12
Position of $Q_1 = \frac{7+1}{2} = 4$th position in the lower half
$Q_1 = 8$ points

**Finding $Q_3$:**
For the upper half (positions 9-15): 15, 16, 18, 20, 22, 24, 26
Position of $Q_3 = \frac{7+1}{2} = 4$th position in the upper half
$Q_3 = 20$ points

Therefore: $Q_1 = 8$, $Q_2 = 14$, $Q_3 = 20$

**(b) Calculating the interquartile range:**

$\text{IQR} = Q_3 - Q_1 = 20 - 8 = 12$ points

**Interpretation:**
The interquartile range of 12 points tells us that the middle 50% of players scored within a range of 12 points. This indicates moderate variability in the scoring performance among the central group of players on the team. A smaller IQR would suggest more consistent scoring, while a larger IQR would indicate greater variation in performance among the middle-performing players.

\section*{Question 63}
\textbf{Metadata}

\begin{itemize}
  \item Question ID: O3-SPMulProb\_O2-SPRepPrSE\_sonnet4\_Recreation\_04
  \item Primary KC: STATISTICS AND PROBABILITY | Multiplication | multiplication of probabilities
  \item Secondary KC: STATISTICS AND PROBABILITY | Representation and concept | probability of single events
  \item Topic: Recreation such as sports, games, exercises, music, movie, dancing, painting, fishing and other recreation activities
  \item Grade: Secondary O-level 3/4
\end{itemize}

\textbf{Solution}

To find the probability that Sarah wins the prize, I need to find the probability that she succeeds in both challenges.

Given information:
• Probability of hitting bullseye in first challenge = $\frac{2}{5}$
• Probability of landing on "WIN" in second challenge = $\frac{3}{8}$
• Sarah must succeed in both challenges to win the prize

Since the two challenges are independent events, and Sarah needs to succeed in both challenges, I multiply the individual probabilities:

Probability of winning the prize = P(hitting bullseye) × P(landing on "WIN")

Probability of winning the prize = $\frac{2}{5} \times \frac{3}{8}$

Probability of winning the prize = $\frac{2 \times 3}{5 \times 8} = \frac{6}{40}$

Simplifying the fraction: $\frac{6}{40} = \frac{3}{20}$

Therefore, the probability that Sarah wins the prize is $\frac{3}{20}$ or $0.15$ or $15\%$.

\end{document}
