\documentclass{article}
\usepackage[utf8]{inputenc}
\usepackage{amsmath}
\usepackage{amsfonts}
\usepackage{amssymb}
\usepackage{graphicx}
\usepackage{hyperref}
\title{'Sarah Solutions recreation v3 CLAUDE '}
\author{Tien Dung Doan}
\begin{document}
\maketitle
\section*{Question 1}
\textbf{Metadata}

\begin{itemize}
  \item Question ID: P3-WNDivRmd3d\_P1-WNCmp\_sonnet4\_Recreation\_04
  \item Primary KC: WHOLE NUMBERS | Division | dividing whole numbers up to 3 digits by 1 digit with remainder 
  \item Secondary KC: WHOLE NUMBERS | Comparison and ordering | comparing and ordering whole numbers
  \item Topic: Recreation such as sports, games, exercises, music, movie, dancing, painting, fishing and other recreation activities
  \item Grade: Primary 3
\end{itemize}

\textbf{Solution}

(a) To find how many paintbrushes each group receives, I need to divide $247$ by $8$.

$247 \div 8$

$8 \times 30 = 240$
$247 - 240 = 7$

So $247 \div 8 = 30$ remainder $7$

Each group will receive $30$ paintbrushes and there will be $7$ paintbrushes left over.

(b) To find how many tubes of paint each group receives, I need to divide $185$ by $8$.

$185 \div 8$

$8 \times 23 = 184$
$185 - 184 = 1$

So $185 \div 8 = 23$ remainder $1$

Each group will receive $23$ tubes of paint and there will be $1$ tube left over.

(c) From parts (a) and (b):
- Paintbrushes left over: $7$
- Tubes of paint left over: $1$

Comparing the leftovers: $7 > 1$

Paintbrushes have more leftovers.

The difference is: $7 - 1 = 6$

Therefore, paintbrushes have $6$ more leftovers than tubes of paint.

\section*{Question 2}
\textbf{Metadata}

\begin{itemize}
  \item Question ID: P3-WNMul3d1d\_P1-WNAdd2nd\_sonnet4\_Recreation\_04
  \item Primary KC: WHOLE NUMBERS | Multiplication | multiplying whole numbers up to 3 digits by 1 digit
  \item Secondary KC: WHOLE NUMBERS | Addition | adding whole numbers
  \item Topic: Recreation such as sports, games, exercises, music, movie, dancing, painting, fishing and other recreation activities
  \item Grade: Primary 3
\end{itemize}

\textbf{Solution}

Step 1: Find the total number of dancers in the 4 teams.
Number of dancers per team = 126
Number of teams = 4
Total dancers in teams = $126 \times 4$

$126 \times 4 = 504$

Step 2: Add the additional backup dancers.
Backup dancers = 89
Total dancers = $504 + 89$

$504 + 89 = 593$

Therefore, there are 593 dancers in total at the dance competition.

\section*{Question 3}
\textbf{Metadata}

\begin{itemize}
  \item Question ID: P3-WNMul3d1d\_P1-WNSub2nd\_sonnet4\_Recreation\_04
  \item Primary KC: WHOLE NUMBERS | Multiplication | multiplying whole numbers up to 3 digits by 1 digit
  \item Secondary KC: WHOLE NUMBERS | Subtraction | subtracting whole numbers
  \item Topic: Recreation such as sports, games, exercises, music, movie, dancing, painting, fishing and other recreation activities
  \item Grade: Primary 3
\end{itemize}

\textbf{Solution}

Step 1: Find the total number of players in the tournament.
Number of teams = 8
Number of players per team = 125
Total number of players = $8 \times 125$

$8 \times 125 = 8 \times (100 + 25)$
$= 8 \times 100 + 8 \times 25$
$= 800 + 200$
$= 1000$

Step 2: Find how many more medals Sarah needs to buy.
Total medals needed = 1000
Medals already ordered = 1200

Since Sarah ordered 1200 medals but only needs 1000 medals:
$1200 - 1000 = 200$

Sarah ordered 200 more medals than needed, so she does not need to buy any more medals. In fact, she has 200 extra medals.

\section*{Question 4}
\textbf{Metadata}

\begin{itemize}
  \item Question ID: P3-FrAddRl12\_P3-FrSmp\_sonnet4\_Recreation\_04
  \item Primary KC: FRACTIONS | Addition | adding two related fractions within one whole with denominators of given fractions not exceeding 12
  \item Secondary KC: FRACTIONS | Simplifying | expressing a fraction in its simplest form
  \item Topic: Recreation such as sports, games, exercises, music, movie, dancing, painting, fishing and other recreation activities
  \item Grade: Primary 3
\end{itemize}

\textbf{Solution}

To find the total time Sarah practiced piano, I need to add the fractions from Monday and Tuesday.

Monday: $\frac{2}{6}$ hour
Tuesday: $\frac{3}{6}$ hour

Since both fractions have the same denominator, I can add the numerators:
$\frac{2}{6} + \frac{3}{6} = \frac{2+3}{6} = \frac{5}{6}$

Now I need to check if $\frac{5}{6}$ is in its simplest form by finding the greatest common factor of 5 and 6.

Factors of 5: 1, 5
Factors of 6: 1, 2, 3, 6

The greatest common factor is 1, so $\frac{5}{6}$ is already in its simplest form.

Therefore, Sarah practiced piano for $\frac{5}{6}$ of an hour in total over the two days.

\section*{Question 5}
\textbf{Metadata}

\begin{itemize}
  \item Question ID: P3-FrSubRl12\_P3-FrSmp\_sonnet4\_Recreation\_04
  \item Primary KC: FRACTIONS | Subtraction | subtracting two related fractions within one whole with denominators of given fractions not exceeding 12
  \item Secondary KC: FRACTIONS | Simplifying | expressing a fraction in its simplest form
  \item Topic: Recreation such as sports, games, exercises, music, movie, dancing, painting, fishing and other recreation activities
  \item Grade: Primary 3
\end{itemize}

\textbf{Solution}

To find how much longer Sarah practiced on Monday than on Tuesday, I need to subtract the time she practiced on Tuesday from the time she practiced on Monday.

Time practiced on Monday = $\frac{5}{6}$ hour
Time practiced on Tuesday = $\frac{1}{6}$ hour

Difference = $\frac{5}{6} - \frac{1}{6}$

Since both fractions have the same denominator, I can subtract the numerators:
$\frac{5}{6} - \frac{1}{6} = \frac{5-1}{6} = \frac{4}{6}$

Now I need to express $\frac{4}{6}$ in its simplest form by finding the greatest common factor of 4 and 6.
Factors of 4: 1, 2, 4
Factors of 6: 1, 2, 3, 6
Greatest common factor = 2

$\frac{4}{6} = \frac{4 \div 2}{6 \div 2} = \frac{2}{3}$

Therefore, Sarah practiced $\frac{2}{3}$ of an hour longer on Monday than on Tuesday.

\section*{Question 6}
\textbf{Metadata}

\begin{itemize}
  \item Question ID: P4-WNMul4d1d\_P1-WNSub2nd\_sonnet4\_Recreation\_04
  \item Primary KC: WHOLE NUMBERS | Multiplication | multiplying whole numbers up to 4 digits by 1 digit or up to 3 digits by 2 digits
  \item Secondary KC: WHOLE NUMBERS | Subtraction | subtracting whole numbers
  \item Topic: Recreation such as sports, games, exercises, music, movie, dancing, painting, fishing and other recreation activities
  \item Grade: Primary 4
\end{itemize}

\textbf{Solution}

Step 1: Find how many complete boxes can be made.
Divide the total tennis balls by the number of tennis balls per box:
$2{,}340 \div 24 = 97$ boxes with remainder $12$

This means $97 \times 24 = 2{,}328$ tennis balls are packed in complete boxes.
The remaining tennis balls are: $2{,}340 - 2{,}328 = 12$ tennis balls.

Step 2: Calculate the total tennis balls before considering damage.
Total tennis balls that were initially available: $2{,}340$ tennis balls

Step 3: Subtract the damaged tennis balls.
Tennis balls available for sale: $2{,}340 - 156 = 2{,}184$ tennis balls

Therefore, $2{,}184$ tennis balls are available for sale.

\section*{Question 7}
\textbf{Metadata}

\begin{itemize}
  \item Question ID: P4-WNDiv4d1d\_P1-WNAdd2nd\_sonnet4\_Recreation\_04
  \item Primary KC: WHOLE NUMBERS | Division | dividing whole numbers up to 4 digits by 1 digit
  \item Secondary KC: WHOLE NUMBERS | Addition | adding whole numbers
  \item Topic: Recreation such as sports, games, exercises, music, movie, dancing, painting, fishing and other recreation activities
  \item Grade: Primary 4
\end{itemize}

\textbf{Solution}

Step 1: Find the number of dancers who registered.
Number of dancers = Total registration fees ÷ Registration fee per dancer
Number of dancers = $3248 ÷ 8 = 406$ dancers

Step 2: Find the total cost of certificates.
Cost per certificate = \textdollar3
Total cost = Number of dancers × Cost per certificate
Total cost = $406 × 3 = 1218$

Therefore, Sarah will need \textdollar1218 in total to buy certificates for all the dancers.

\section*{Question 8}
\textbf{Metadata}

\begin{itemize}
  \item Question ID: P4-FrRepSet\_P3-FrCnvEq\_sonnet4\_Recreation\_04
  \item Primary KC: FRACTIONS | Representation and concept | expressing a part of a set as a fraction
  \item Secondary KC: FRACTIONS | Conversion to equivalent fractions | Conversion to equivalent fractions (given either the denominator or the numerator)
  \item Topic: Recreation such as sports, games, exercises, music, movie, dancing, painting, fishing and other recreation activities
  \item Grade: Primary 4
\end{itemize}

\textbf{Solution}

Step 1: Find the number of paintbrushes used in the first session.
Total paintbrushes = 24
Fraction used in first session = $\frac{3}{8}$
Number of paintbrushes used = $\frac{3}{8} \times 24 = \frac{3 \times 24}{8} = \frac{72}{8} = 9$ paintbrushes

Step 2: Express 9 paintbrushes out of 24 as a fraction with denominator 12.
Fraction = $\frac{9}{24}$
To convert to denominator 12, we need to find the equivalent fraction.
Since $24 \div 2 = 12$, we divide both numerator and denominator by 2.
$\frac{9}{24} = \frac{9 \div 2}{24 \div 2} = \frac{4.5}{12}$

Since we need a whole number in the numerator, let's work differently.
$\frac{9}{24} = \frac{9 \times 1}{24 \times 1} = \frac{9}{24}$
To get denominator 12: $\frac{9}{24} = \frac{9 \div 2}{24 \div 2} = \frac{4.5}{12}$

Actually, let me recalculate: $\frac{9}{24}$ in simplest form is $\frac{3}{8}$.
To convert $\frac{3}{8}$ to denominator 12, we need $8 \times ? = 12$, but 12 is not a multiple of 8.

Let me reconsider: We want to express $\frac{9}{24}$ with denominator 12.
$\frac{9}{24} = \frac{9 \div 2}{24 \div 2} = \frac{4.5}{12}$

Since fractions should have whole numbers, the fraction $\frac{9}{24}$ converts to $\frac{9}{24} \times \frac{12}{24} = \frac{9 \times 12}{24 \times 24} = \frac{108}{576}$

Let me restart: $\frac{9}{24}$ needs to be converted to twelfths.
$\frac{9}{24} = \frac{9}{24} \times \frac{1/2}{1/2} = \frac{4.5}{12}$

Since we need whole numbers: $9$ out of $24$ equals how many out of $12$?
$\frac{9}{24} = \frac{x}{12}$
$9 \times 12 = 24 \times x$
$108 = 24x$
$x = 4.5$

This suggests we should adjust our problem. Let me recalculate properly:
$\frac{9}{24} = \frac{9 \div 2}{24 \div 2} = \frac{4.5}{12}$

For Primary 4, we should have whole numbers. So $\frac{9}{24} = \frac{9}{24} \times \frac{0.5}{0.5} = \frac{4.5}{12}$

But since $4.5$ is not a whole number, the equivalent fraction is $\frac{9}{24} = \frac{9 \times 1}{24 \times 1}$. To get denominator 12: $\frac{9 \times 12}{24 \times 12} \div \frac{24}{24} = \frac{108}{288} ÷ 24 = \frac{108 ÷ 24}{288 ÷ 24} = \frac{4.5}{12}$

The answer should be: 9 paintbrushes were used in the first session, and this can be expressed as $\frac{9}{24}$, which equals $\frac{4.5}{12}$ or more appropriately $\frac{9}{24}$ cannot be expressed as a fraction with denominator 12 using whole numbers.

Actual solution: 9 paintbrushes were used in the first session. The fraction $\frac{9}{24}$ simplifies to $\frac{3}{8}$, and since 12 is not a multiple of 8, we cannot express this as an equivalent fraction with denominator 12 using whole numbers.

\section*{Question 9}
\textbf{Metadata}

\begin{itemize}
  \item Question ID: P4-FrSubU12\_P2-FrAdd2nd\_sonnet4\_Recreation\_04
  \item Primary KC: FRACTIONS | Subtraction | subtracting unlike fractions with two different denominators not exceeding 12
  \item Secondary KC: FRACTIONS | Addition | adding fractions
  \item Topic: Recreation such as sports, games, exercises, music, movie, dancing, painting, fishing and other recreation activities
  \item Grade: Primary 4
\end{itemize}

\textbf{Solution}

**Part 1: How much longer did Sarah practice on Monday than on Tuesday?**

Monday practice time: $\frac{3}{4}$ hour
Tuesday practice time: $\frac{1}{3}$ hour

To find how much longer Monday was than Tuesday, I need to subtract:
$\frac{3}{4} - \frac{1}{3}$

To subtract fractions with different denominators, I need to find a common denominator.
The LCM of 4 and 3 is 12.

$\frac{3}{4} = \frac{3 \times 3}{4 \times 3} = \frac{9}{12}$

$\frac{1}{3} = \frac{1 \times 4}{3 \times 4} = \frac{4}{12}$

$\frac{9}{12} - \frac{4}{12} = \frac{5}{12}$

Sarah practiced $\frac{5}{12}$ hour longer on Monday than on Tuesday.

**Part 2: What was her total practice time for all three days?**

Total practice time = Monday + Tuesday + Wednesday
$= \frac{3}{4} + \frac{1}{3} + \frac{5}{6}$

To add fractions with different denominators, I need a common denominator.
The LCM of 4, 3, and 6 is 12.

$\frac{3}{4} = \frac{9}{12}$

$\frac{1}{3} = \frac{4}{12}$

$\frac{5}{6} = \frac{5 \times 2}{6 \times 2} = \frac{10}{12}$

$\frac{9}{12} + \frac{4}{12} + \frac{10}{12} = \frac{23}{12}$

$\frac{23}{12} = 1\frac{11}{12}$

Sarah's total practice time for all three days was $1\frac{11}{12}$ hours.

\section*{Question 10}
\textbf{Metadata}

\begin{itemize}
  \item Question ID: P4-DcSub2d\_P4-DcCmp3d\_sonnet4\_Recreation\_04
  \item Primary KC: DECIMALS | Subtraction | subtracting decimals (up to 2 decimal places)
  \item Secondary KC: DECIMALS | Comparison and ordering | comparing and ordering decimals up to 3 decimal places
  \item Topic: Recreation such as sports, games, exercises, music, movie, dancing, painting, fishing and other recreation activities
  \item Grade: Primary 4
\end{itemize}

\textbf{Solution}

Step 1: Calculate Amy's corrected time by subtracting the penalty.
Amy's original recorded time: $25.847$ seconds
Penalty to subtract: $0.180$ seconds (written with 3 decimal places for alignment)
Amy's corrected time: $25.847 - 0.180 = 25.667$ seconds

Step 2: List all swimmers' final times and compare them to the qualifying time of $26.050$ seconds.
Amy's corrected time: $25.667$ seconds
Ben's time: $26.120$ seconds
Chloe's time: $25.903$ seconds
Qualifying time: $26.050$ seconds (to advance, time must be $26.050$ seconds or faster)

Step 3: Compare each swimmer's time to the qualifying standard.
- Amy: $25.667 < 26.050$ ✓ (qualifies - faster than qualifying time)
- Ben: $26.120 > 26.050$ ✗ (does not qualify - slower than qualifying time)
- Chloe: $25.903 < 26.050$ ✓ (qualifies - faster than qualifying time)

Therefore, Amy's corrected time is $25.667$ seconds, and Amy and Chloe qualified for the next round.

\section*{Question 11}
\textbf{Metadata}

\begin{itemize}
  \item Question ID: P4-DcMul2d1d\_P4-DcCmp3d\_sonnet4\_Recreation\_04
  \item Primary KC: DECIMALS | Multiplication | multiplying decimals (up to 2 decimal places) by a 1-digit whole number
  \item Secondary KC: DECIMALS | Comparison and ordering | comparing and ordering decimals up to 3 decimal places
  \item Topic: Recreation such as sports, games, exercises, music, movie, dancing, painting, fishing and other recreation activities
  \item Grade: Primary 4
\end{itemize}

\textbf{Solution}

First, I need to find the total practice time for each day by multiplying the time per lap by the number of laps.

**Monday's total time:**
$2.45 \times 6$
$= 14.70$ seconds

**Tuesday's total time:**
$2.38 \times 6$
$= 14.28$ seconds

**Wednesday's total time:**
$2.406 \times 6$
$= 14.436$ seconds

Now I need to compare and order the total times: $14.70$, $14.28$, and $14.436$ seconds.

To compare these decimals, I'll align them by decimal places:
- Monday: $14.700$ seconds
- Tuesday: $14.280$ seconds  
- Wednesday: $14.436$ seconds

Comparing from left to right:
- All have $14$ in the ones place
- In the tenths place: $2 < 4 < 7$

Therefore: $14.280 < 14.436 < 14.700$

**Answer:** 
- Monday's total time: $14.70$ seconds
- Tuesday's total time: $14.28$ seconds
- Wednesday's total time: $14.436$ seconds

Order from shortest to longest total practice time: Tuesday, Wednesday, Monday

\section*{Question 12}
\textbf{Metadata}

\begin{itemize}
  \item Question ID: P4-DcMul2d1d\_P4-DcRnd3d\_sonnet4\_Recreation\_04
  \item Primary KC: DECIMALS | Multiplication | multiplying decimals (up to 2 decimal places) by a 1-digit whole number
  \item Secondary KC: DECIMALS | Rounding | rounding decimals up to 3 decimal places to the nearest whole number, 1 decimal place and 2 decimal places 
  \item Topic: Recreation such as sports, games, exercises, music, movie, dancing, painting, fishing and other recreation activities
  \item Grade: Primary 4
\end{itemize}

\textbf{Solution}

To find the total distance Sarah swims, I need to multiply her speed by the time.

Speed = $3.25$ meters per second
Time = $4$ seconds

Total distance = Speed $\times$ Time
Total distance = $3.25 \times 4$

To multiply $3.25 \times 4$:
$3.25 \times 4 = 13.00 = 13$ meters

Now I need to round the answer:
- To the nearest whole number: $13.00$ rounds to $13$
- To $1$ decimal place: $13.00$ rounds to $13.0$

Therefore, Sarah will swim $13$ meters in total. Rounded to the nearest whole number, the answer is $13$ meters. Rounded to $1$ decimal place, the answer is $13.0$ meters.

\section*{Question 13}
\textbf{Metadata}

\begin{itemize}
  \item Question ID: P4-DcMul2d1d\_P4-DcSub2nd\_sonnet4\_Recreation\_04
  \item Primary KC: DECIMALS | Multiplication | multiplying decimals (up to 2 decimal places) by a 1-digit whole number
  \item Secondary KC: DECIMALS | Subtraction | subtracting decimals
  \item Topic: Recreation such as sports, games, exercises, music, movie, dancing, painting, fishing and other recreation activities
  \item Grade: Primary 4
\end{itemize}

\textbf{Solution}

Step 1: Find how many hours Sarah practiced on Tuesday.
Sarah practiced $3$ times as long on Tuesday as she did on Monday.
Hours practiced on Tuesday $= 2.75 \times 3 = 8.25$ hours

Step 2: Find how many hours Sarah practiced on Wednesday.
Sarah practiced $1.5$ hours less on Wednesday than she did on Tuesday.
Hours practiced on Wednesday $= 8.25 - 1.5 = 6.75$ hours

Therefore, Sarah practiced for $6.75$ hours on Wednesday.

\section*{Question 14}
\textbf{Metadata}

\begin{itemize}
  \item Question ID: P4-DcDiv2d1d\_P4-DcRnd3d\_sonnet4\_Recreation\_04
  \item Primary KC: DECIMALS | Division | dividing decimals (up to 2 decimal places) by a 1-digit whole number
  \item Secondary KC: DECIMALS | Rounding | rounding decimals up to 3 decimal places to the nearest whole number, 1 decimal place and 2 decimal places 
  \item Topic: Recreation such as sports, games, exercises, music, movie, dancing, painting, fishing and other recreation activities
  \item Grade: Primary 4
\end{itemize}

\textbf{Solution}

To find the length of each lap, I need to divide the total distance by the number of laps.

Length of each lap $= 47.28 \div 6$

Let me perform the division:
$47.28 \div 6 = 7.88$

So each lap is $7.88$ meters long.

Now I need to round this answer:

1) To the nearest whole number:
   Looking at $7.88$, the digit in the tenths place is $8$.
   Since $8 \geq 5$, I round up.
   $7.88$ rounded to the nearest whole number is $8$ meters.

2) To the nearest tenth (1 decimal place):
   Looking at $7.88$, the digit in the hundredths place is $8$.
   Since $8 \geq 5$, I round up the tenths place.
   $7.88$ rounded to the nearest tenth is $7.9$ meters.

Therefore, each lap is $7.88$ meters long. Rounded to the nearest whole number, it is $8$ meters, and rounded to the nearest tenth, it is $7.9$ meters.

\section*{Question 15}
\textbf{Metadata}

\begin{itemize}
  \item Question ID: P5-FrAddMix\_P5-FrCnv2Dc\_sonnet4\_Recreation\_04
  \item Primary KC: FRACTIONS | Addition | adding mixed numbers
  \item Secondary KC: FRACTIONS | Conversion to decimals | expressing fractions as decimals
  \item Topic: Recreation such as sports, games, exercises, music, movie, dancing, painting, fishing and other recreation activities
  \item Grade: Primary 5
\end{itemize}

\textbf{Solution}

To find the total training time, I need to add all the mixed numbers together, then convert the result to a decimal.

Step 1: Add the swimming times.
Monday swimming: $2\frac{3}{4}$ hours
Tuesday swimming: $1\frac{5}{8}$ hours

First, I'll convert to improper fractions with a common denominator:
$2\frac{3}{4} = \frac{11}{4} = \frac{22}{8}$
$1\frac{5}{8} = \frac{13}{8}$

Total swimming time = $\frac{22}{8} + \frac{13}{8} = \frac{35}{8} = 4\frac{3}{8}$ hours

Step 2: Add the running times.
Monday running: $1\frac{1}{2}$ hours
Tuesday running: $2\frac{1}{4}$ hours

Converting to improper fractions with a common denominator:
$1\frac{1}{2} = \frac{3}{2} = \frac{6}{4}$
$2\frac{1}{4} = \frac{9}{4}$

Total running time = $\frac{6}{4} + \frac{9}{4} = \frac{15}{4} = 3\frac{3}{4}$ hours

Step 3: Add the total swimming and running times.
Total swimming: $4\frac{3}{8}$ hours
Total running: $3\frac{3}{4}$ hours

Converting to improper fractions with a common denominator:
$4\frac{3}{8} = \frac{35}{8}$
$3\frac{3}{4} = \frac{15}{4} = \frac{30}{8}$

Total training time = $\frac{35}{8} + \frac{30}{8} = \frac{65}{8} = 8\frac{1}{8}$ hours

Step 4: Convert to decimal.
$8\frac{1}{8} = 8 + \frac{1}{8} = 8 + 0.125 = 8.125$ hours

Therefore, Sarah spent a total of $8.125$ hours training over the two days.

\section*{Question 16}
\textbf{Metadata}

\begin{itemize}
  \item Question ID: P5-FrSubMix\_P3-FrSmp\_sonnet4\_Recreation\_04
  \item Primary KC: FRACTIONS | Subtraction | subtracting mixed numbers
  \item Secondary KC: FRACTIONS | Simplifying | expressing a fraction in its simplest form
  \item Topic: Recreation such as sports, games, exercises, music, movie, dancing, painting, fishing and other recreation activities
  \item Grade: Primary 5
\end{itemize}

\textbf{Solution}

To find how many kilometers Sarah ran on Tuesday, I need to subtract $2\frac{5}{6}$ km from $5\frac{3}{4}$ km.

First, I'll convert both mixed numbers to improper fractions:
$5\frac{3}{4} = \frac{5 \times 4 + 3}{4} = \frac{23}{4}$
$2\frac{5}{6} = \frac{2 \times 6 + 5}{6} = \frac{17}{6}$

Now I need to subtract: $\frac{23}{4} - \frac{17}{6}$

To subtract fractions, I need a common denominator. The LCM of 4 and 6 is 12.
$\frac{23}{4} = \frac{23 \times 3}{4 \times 3} = \frac{69}{12}$
$\frac{17}{6} = \frac{17 \times 2}{6 \times 2} = \frac{34}{12}$

Now I can subtract:
$\frac{69}{12} - \frac{34}{12} = \frac{69-34}{12} = \frac{35}{12}$

To check if $\frac{35}{12}$ is in its simplest form, I need to find the GCD of 35 and 12.
Factors of 35: 1, 5, 7, 35
Factors of 12: 1, 2, 3, 4, 6, 12
The GCD is 1, so $\frac{35}{12}$ is already in its simplest form.

Therefore, Sarah ran $\frac{35}{12}$ km on Tuesday, which can also be written as $2\frac{11}{12}$ km.

\section*{Question 17}
\textbf{Metadata}

\begin{itemize}
  \item Question ID: P5-FrSubMix\_P5-FrCnv2Dc\_sonnet4\_Recreation\_04
  \item Primary KC: FRACTIONS | Subtraction | subtracting mixed numbers
  \item Secondary KC: FRACTIONS | Conversion to decimals | expressing fractions as decimals
  \item Topic: Recreation such as sports, games, exercises, music, movie, dancing, painting, fishing and other recreation activities
  \item Grade: Primary 5
\end{itemize}

\textbf{Solution}

First, I need to find the total distance Sarah ran on Monday.

Monday's total distance = Morning distance + Evening distance
Monday's total distance = $4\frac{3}{4} + 2\frac{1}{2}$

To add these mixed numbers, I'll convert them to improper fractions with a common denominator:
$4\frac{3}{4} = \frac{19}{4}$ and $2\frac{1}{2} = 2\frac{2}{4} = \frac{10}{4}$

Monday's total distance = $\frac{19}{4} + \frac{10}{4} = \frac{29}{4} = 7\frac{1}{4}$ km

Now I need to find how much further Sarah ran on Monday than on Tuesday.

Difference = Monday's distance - Tuesday's distance
Difference = $7\frac{1}{4} - 6\frac{2}{5}$

To subtract these mixed numbers, I need a common denominator. The LCD of 4 and 5 is 20.

$7\frac{1}{4} = 7\frac{5}{20}$ and $6\frac{2}{5} = 6\frac{8}{20}$

Difference = $7\frac{5}{20} - 6\frac{8}{20}$

Since $\frac{5}{20} < \frac{8}{20}$, I need to borrow 1 from the whole number:
$7\frac{5}{20} = 6\frac{25}{20}$

Difference = $6\frac{25}{20} - 6\frac{8}{20} = \frac{17}{20}$ km

Now I need to express $\frac{17}{20}$ as a decimal:
$\frac{17}{20} = \frac{17 \times 5}{20 \times 5} = \frac{85}{100} = 0.85$

Therefore, Sarah ran 0.85 km further on Monday than on Tuesday.

\section*{Question 18}
\textbf{Metadata}

\begin{itemize}
  \item Question ID: P5-FrMulImN\_P2-FrAdd2nd\_sonnet4\_Recreation\_04
  \item Primary KC: FRACTIONS | Multiplication | multiplying a proper/improper fraction and a whole number
  \item Secondary KC: FRACTIONS | Addition | adding fractions
  \item Topic: Recreation such as sports, games, exercises, music, movie, dancing, painting, fishing and other recreation activities
  \item Grade: Primary 5
\end{itemize}

\textbf{Solution}

Step 1: Find how long Sarah practiced ballet on Tuesday.
On Monday, Sarah practiced ballet for $\frac{3}{4}$ hour.
On Tuesday, she practiced 3 times as long as Monday.
Tuesday's practice time = $3 \times \frac{3}{4} = \frac{3 \times 3}{4} = \frac{9}{4}$ hours

Step 2: Find the total ballet practice time over Monday and Tuesday.
Total ballet time = Monday + Tuesday
Total ballet time = $\frac{3}{4} + \frac{9}{4} = \frac{3 + 9}{4} = \frac{12}{4} = 3$ hours

Step 3: Find the total practice time over all three days.
Total practice time = Ballet time + Jazz time
Total practice time = $3 + \frac{5}{6}$
To add these, convert 3 to a fraction with denominator 6:
$3 = \frac{18}{6}$
Total practice time = $\frac{18}{6} + \frac{5}{6} = \frac{18 + 5}{6} = \frac{23}{6} = 3\frac{5}{6}$ hours

Therefore, Sarah spent 3 hours practicing ballet over Monday and Tuesday, and $3\frac{5}{6}$ hours practicing all dance routines over the three days.

\section*{Question 19}
\textbf{Metadata}

\begin{itemize}
  \item Question ID: P5-FrMulImN\_P3-FrSmp\_sonnet4\_Recreation\_04
  \item Primary KC: FRACTIONS | Multiplication | multiplying a proper/improper fraction and a whole number
  \item Secondary KC: FRACTIONS | Simplifying | expressing a fraction in its simplest form
  \item Topic: Recreation such as sports, games, exercises, music, movie, dancing, painting, fishing and other recreation activities
  \item Grade: Primary 5
\end{itemize}

\textbf{Solution}

To find how many hours Sarah practiced on Tuesday, I need to multiply her Monday practice time by 3.

Monday practice time: $2\frac{1}{4}$ hours
Tuesday practice time: $3 \times 2\frac{1}{4}$ hours

First, I'll convert the mixed number to an improper fraction:
$2\frac{1}{4} = \frac{2 \times 4 + 1}{4} = \frac{9}{4}$

Now I can multiply:
$3 \times \frac{9}{4} = \frac{3 \times 9}{4} = \frac{27}{4}$

Next, I'll convert this improper fraction back to a mixed number:
$\frac{27}{4} = 6\frac{3}{4}$ (since $27 \div 4 = 6$ remainder $3$)

I need to check if $\frac{3}{4}$ is in its simplest form:
Since 3 and 4 share no common factors other than 1, $\frac{3}{4}$ is already in its simplest form.

Therefore, Sarah practiced for $6\frac{3}{4}$ hours on Tuesday.

\section*{Question 20}
\textbf{Metadata}

\begin{itemize}
  \item Question ID: P5-FrMulImN\_P5-FrCnv2Dc\_sonnet4\_Recreation\_04
  \item Primary KC: FRACTIONS | Multiplication | multiplying a proper/improper fraction and a whole number
  \item Secondary KC: FRACTIONS | Conversion to decimals | expressing fractions as decimals
  \item Topic: Recreation such as sports, games, exercises, music, movie, dancing, painting, fishing and other recreation activities
  \item Grade: Primary 5
\end{itemize}

\textbf{Solution}

To find the total number of laps Sarah swims, I need to multiply the fraction of a lap she swims per stroke cycle by the number of stroke cycles.

Total laps = $\frac{3}{4} \times 16$

To multiply a fraction by a whole number, I multiply the numerator by the whole number:
$\frac{3}{4} \times 16 = \frac{3 \times 16}{4} = \frac{48}{4} = 12$

So Sarah swims 12 laps in total.

To express this as a decimal: $12 = 12.0$

Therefore, Sarah swims 12.0 laps during her training session.

\section*{Question 21}
\textbf{Metadata}

\begin{itemize}
  \item Question ID: P5-FrMulImIm\_P2-FrCmp\_sonnet4\_Recreation\_04
  \item Primary KC: FRACTIONS | Multiplication | multiplying two improper fractions
  \item Secondary KC: FRACTIONS | Comparison and ordering | comparing and ordering fractions
  \item Topic: Recreation such as sports, games, exercises, music, movie, dancing, painting, fishing and other recreation activities
  \item Grade: Primary 5
\end{itemize}

\textbf{Solution}

To find the total energy used by each dancer, I need to multiply the time danced by the fraction of energy used.

For Sarah:
Total energy used = $\frac{7}{4} \times \frac{5}{3}$
$= \frac{7 \times 5}{4 \times 3}$
$= \frac{35}{12}$ energy-hours

For Emma:
Total energy used = $\frac{9}{5} \times \frac{4}{3}$
$= \frac{9 \times 4}{5 \times 3}$
$= \frac{36}{15}$
$= \frac{12}{5}$ energy-hours

Now I need to compare $\frac{35}{12}$ and $\frac{12}{5}$ to determine who used more energy.

To compare these fractions, I'll find a common denominator. The LCM of 12 and 5 is 60.

$\frac{35}{12} = \frac{35 \times 5}{12 \times 5} = \frac{175}{60}$

$\frac{12}{5} = \frac{12 \times 12}{5 \times 12} = \frac{144}{60}$

Since $\frac{175}{60} > \frac{144}{60}$, Sarah used more energy.

Therefore, Sarah used $\frac{35}{12}$ energy-hours and Emma used $\frac{12}{5}$ energy-hours. Sarah used more energy during her practice session.

\section*{Question 22}
\textbf{Metadata}

\begin{itemize}
  \item Question ID: P5-FrMulImIm\_P5-FrCnv2Dc\_sonnet4\_Recreation\_04
  \item Primary KC: FRACTIONS | Multiplication | multiplying two improper fractions
  \item Secondary KC: FRACTIONS | Conversion to decimals | expressing fractions as decimals
  \item Topic: Recreation such as sports, games, exercises, music, movie, dancing, painting, fishing and other recreation activities
  \item Grade: Primary 5
\end{itemize}

\textbf{Solution}

To find how many hours Sarah practiced on Tuesday, I need to multiply the time she practiced on Monday by $\frac{7}{4}$.

Time practiced on Monday = $\frac{5}{3}$ hours
Time practiced on Tuesday = $\frac{5}{3} \times \frac{7}{4}$ hours

To multiply these improper fractions:
$\frac{5}{3} \times \frac{7}{4} = \frac{5 \times 7}{3 \times 4} = \frac{35}{12}$

Now I need to express $\frac{35}{12}$ as a decimal.
Dividing 35 by 12:
$35 \div 12 = 2.916\overline{6}$

Since we typically round to a reasonable number of decimal places for time, $\frac{35}{12} = 2.92$ hours (rounded to 2 decimal places).

Therefore, Sarah practiced for 2.92 hours on Tuesday.

\section*{Question 23}
\textbf{Metadata}

\begin{itemize}
  \item Question ID: P5-FrMulMixN\_P2-FrAdd2nd\_sonnet4\_Recreation\_04
  \item Primary KC: FRACTIONS | Multiplication | multiplying a mixed number and a whole number
  \item Secondary KC: FRACTIONS | Addition | adding fractions
  \item Topic: Recreation such as sports, games, exercises, music, movie, dancing, painting, fishing and other recreation activities
  \item Grade: Primary 5
\end{itemize}

\textbf{Solution}

To find the total number of laps Sarah swims over two days, I need to calculate the laps for each day separately, then add them together.

**Monday:**
Laps per minute: $2\frac{3}{4}$
Time: 8 minutes
Total laps = $2\frac{3}{4} \times 8$

First, I'll convert the mixed number to an improper fraction:
$2\frac{3}{4} = \frac{2 \times 4 + 3}{4} = \frac{11}{4}$

Now multiply:
$\frac{11}{4} \times 8 = \frac{11 \times 8}{4} = \frac{88}{4} = 22$ laps

**Tuesday:**
Laps per minute: $1\frac{5}{6}$
Time: 6 minutes
Total laps = $1\frac{5}{6} \times 6$

First, I'll convert the mixed number to an improper fraction:
$1\frac{5}{6} = \frac{1 \times 6 + 5}{6} = \frac{11}{6}$

Now multiply:
$\frac{11}{6} \times 6 = \frac{11 \times 6}{6} = \frac{66}{6} = 11$ laps

**Total laps over two days:**
$22 + 11 = 33$ laps

Therefore, Sarah swam 33 laps in total over the two days.

\section*{Question 24}
\textbf{Metadata}

\begin{itemize}
  \item Question ID: P5-DcMul3dK\_P4-DcCmp3d\_sonnet4\_Recreation\_04
  \item Primary KC: DECIMALS | Multiplication | multiplying decimals (up to 3 decimal places) by 10, 100, 1000 and their multiples
  \item Secondary KC: DECIMALS | Comparison and ordering | comparing and ordering decimals up to 3 decimal places
  \item Topic: Recreation such as sports, games, exercises, music, movie, dancing, painting, fishing and other recreation activities
  \item Grade: Primary 5
\end{itemize}

\textbf{Solution}

First, I need to calculate each day's lap time by multiplying the Monday time by the given multiples.

Monday's lap time: $0.875$ seconds

Tuesday's lap time: $0.875 \times 100 = 87.5$ seconds

Wednesday's lap time: $0.875 \times 1000 = 875$ seconds

Thursday's lap time: $0.875 \times 200 = 175$ seconds

Now I have the four lap times:
- Monday: $0.875$ seconds
- Tuesday: $87.5$ seconds  
- Wednesday: $875$ seconds
- Thursday: $175$ seconds

To arrange from fastest to slowest, I need to order from smallest to largest time (since smaller time means faster speed):

Comparing the decimals:
$0.875 < 87.5 < 175 < 875$

Therefore, Sarah's daily average lap times from fastest to slowest are:
1. Monday: $0.875$ seconds (fastest)
2. Tuesday: $87.5$ seconds
3. Thursday: $175$ seconds  
4. Wednesday: $875$ seconds (slowest)

\section*{Question 25}
\textbf{Metadata}

\begin{itemize}
  \item Question ID: P5-DcMul3dK\_P4-DcAdd2nd\_sonnet4\_Recreation\_04
  \item Primary KC: DECIMALS | Multiplication | multiplying decimals (up to 3 decimal places) by 10, 100, 1000 and their multiples
  \item Secondary KC: DECIMALS | Addition | adding decimals
  \item Topic: Recreation such as sports, games, exercises, music, movie, dancing, painting, fishing and other recreation activities
  \item Grade: Primary 5
\end{itemize}

\textbf{Solution}

To find the total distance Sarah covers, I need to calculate the distance for each stroke and then add them together.

**Step 1: Calculate distance for freestyle**
Freestyle distance = $0.025 \times 20 = 0.5$ km

**Step 2: Calculate distance for backstroke**
Backstroke distance = $0.018 \times 30 = 0.54$ km

**Step 3: Calculate distance for breaststroke**
Breaststroke distance = $0.022 \times 40 = 0.88$ km

**Step 4: Add all distances together**
Total distance in km = $0.5 + 0.54 + 0.88 = 1.92$ km

**Step 5: Convert to meters**
Since the question asks for the answer in meters, I need to multiply by $1000$:
Total distance in meters = $1.92 \times 1000 = 1920$ meters

Therefore, Sarah covers a total distance of $1920$ meters during her training session.

\section*{Question 26}
\textbf{Metadata}

\begin{itemize}
  \item Question ID: P5-DcDiv3dK\_P4-DcCmp3d\_sonnet4\_Recreation\_04
  \item Primary KC: DECIMALS | Division | dividing decimals (up to 3 decimal places) by 10, 100, 1000 and their multiples
  \item Secondary KC: DECIMALS | Comparison and ordering | comparing and ordering decimals up to 3 decimal places
  \item Topic: Recreation such as sports, games, exercises, music, movie, dancing, painting, fishing and other recreation activities
  \item Grade: Primary 5
\end{itemize}

\textbf{Solution}

(a) To find the original number, I need to work backwards from the given times.

Looking at the times:
- Alex: $23.450$ seconds
- Ben: $234.50$ seconds  
- Chris: $2345.0$ seconds

I can see that:
- $23.450 \times 100 = 2345.0$
- $234.50 \times 10 = 2345.0$
- $2345.0 \times 1 = 2345.0$

So the original number is $2345$.

To verify:
- Alex's time: $2345 \div 100 = 23.450$ seconds
- Ben's time: $2345 \div 10 = 234.50$ seconds
- Chris's time: $2345 \div 1 = 2345.0$ seconds

(b) To arrange the times in ascending order (fastest to slowest), I need to compare the decimal numbers:

$23.450 < 234.50 < 2345.0$

Therefore, the order from fastest to slowest is:
1st place: Alex with $23.450$ seconds
2nd place: Ben with $234.50$ seconds
3rd place: Chris with $2345.0$ seconds

(c) The difference between the fastest and slowest swimmer:
$2345.0 - 23.450 = 2321.550$ seconds

Therefore, the difference in time between the fastest and slowest swimmer is $2321.550$ seconds.

\section*{Question 27}
\textbf{Metadata}

\begin{itemize}
  \item Question ID: P5-DcDiv3dK\_P4-DcSub2nd\_sonnet4\_Recreation\_04
  \item Primary KC: DECIMALS | Division | dividing decimals (up to 3 decimal places) by 10, 100, 1000 and their multiples
  \item Secondary KC: DECIMALS | Subtraction | subtracting decimals
  \item Topic: Recreation such as sports, games, exercises, music, movie, dancing, painting, fishing and other recreation activities
  \item Grade: Primary 5
\end{itemize}

\textbf{Solution}

(a) To find Sarah's average time per kilometer in her first practice run:
Average time per kilometer = Total time ÷ Distance
Average time per kilometer = $28.750 \div 5$
Average time per kilometer = $5.750$ minutes per kilometer

(b) To find Sarah's average time per kilometer in her second practice run:
Average time per kilometer = Total time ÷ Distance
Average time per kilometer = $26.425 \div 5$
Average time per kilometer = $5.285$ minutes per kilometer

(c) To find the improvement per kilometer:
Improvement = First run average time per km - Second run average time per km
Improvement = $5.750 - 5.285$
Improvement = $0.465$ minutes per kilometer

Therefore, Sarah improved by $0.465$ minutes per kilometer between her first and second practice runs.

\section*{Question 28}
\textbf{Metadata}

\begin{itemize}
  \item Question ID: P5-PcRepWh\_P1-WNAdd2nd\_sonnet4\_Recreation\_04
  \item Primary KC: PERCENTAGE | Representation and concept | expressing a part of a whole as a percentage
  \item Secondary KC: WHOLE NUMBERS | Addition | adding whole numbers
  \item Topic: Recreation such as sports, games, exercises, music, movie, dancing, painting, fishing and other recreation activities
  \item Grade: Primary 5
\end{itemize}

\textbf{Solution}

Step 1: Find the total number of players in the tournament.
Players from first round = $45$
Players who joined in second round = $27$
Total players = $45 + 27 = 72$

Step 2: Find what percentage of total players are from the first round.
Players from first round = $45$
Total players = $72$
Percentage = $\frac{45}{72} \times 100\%$

Step 3: Simplify the fraction.
$\frac{45}{72} = \frac{45 \div 9}{72 \div 9} = \frac{5}{8}$

Step 4: Convert to percentage.
$\frac{5}{8} \times 100\% = \frac{500}{8}\% = 62.5\%$

Therefore, $62.5\%$ of the total players in the tournament are from the first round.

\section*{Question 29}
\textbf{Metadata}

\begin{itemize}
  \item Question ID: P5-PcRepWh\_P1-WNSub2nd\_sonnet4\_Recreation\_04
  \item Primary KC: PERCENTAGE | Representation and concept | expressing a part of a whole as a percentage
  \item Secondary KC: WHOLE NUMBERS | Subtraction | subtracting whole numbers
  \item Topic: Recreation such as sports, games, exercises, music, movie, dancing, painting, fishing and other recreation activities
  \item Grade: Primary 5
\end{itemize}

\textbf{Solution}

Step 1: Find the number of games the Lions team lost.
Number of games lost = Total games - Games won
Number of games lost = $80 - 68 = 12$ games

Step 2: Express the number of games lost as a percentage of the total games.
Percentage of games lost = $\frac{\text{Number of games lost}}{\text{Total games}} \times 100\%$
Percentage of games lost = $\frac{12}{80} \times 100\%$
Percentage of games lost = $0.15 \times 100\% = 15\%$

Therefore, the Lions team lost $15\%$ of their games.

\section*{Question 30}
\textbf{Metadata}

\begin{itemize}
  \item Question ID: P5-RtFndT\_P2-DcCnvD2N\_sonnet4\_Recreation\_04
  \item Primary KC: RATE | Finding total amount | finding total amount, given rate and number of units
  \item Secondary KC: DECIMALS | Conversion to smaller units | converting a measurement from a larger unit in decimal form to a smaller unit
  \item Topic: Recreation such as sports, games, exercises, music, movie, dancing, painting, fishing and other recreation activities
  \item Grade: Primary 5
\end{itemize}

\textbf{Solution}

To find the total amount Sarah raised, I need to multiply the rate per kilometer by the distance she ran.

Given information:
- Rate: \textdollar12.50 per kilometer
- Distance: 2.4 kilometers

First, I'll convert 2.4 kilometers to a smaller unit to make the calculation clearer:
$2.4 \text{ km} = 2.4 \times 1000 = 2400 \text{ meters}$

However, since the rate is given per kilometer, I'll work directly with kilometers:

Total amount raised = Rate per kilometer $\times$ Number of kilometers
Total amount raised = \textdollar12.50 $\times$ 2.4

To multiply \textdollar12.50 by 2.4:
$12.50 \times 2.4 = 12.50 \times (2 + 0.4)$
$= 12.50 \times 2 + 12.50 \times 0.4$
$= 25.00 + 5.00$
$= 30.00$

Therefore, Sarah raised \textdollar30.00 for charity during her training session.

\section*{Question 31}
\textbf{Metadata}

\begin{itemize}
  \item Question ID: P6-FrDivPN\_P2-FrCmp\_sonnet4\_Recreation\_04
  \item Primary KC: FRACTIONS | Division | dividing a proper fraction by a whole number
  \item Secondary KC: FRACTIONS | Comparison and ordering | comparing and ordering fractions
  \item Topic: Recreation such as sports, games, exercises, music, movie, dancing, painting, fishing and other recreation activities
  \item Grade: Primary 6
\end{itemize}

\textbf{Solution}

(a) Time for each segment in the first round:
$\frac{3}{4} \div 6 = \frac{3}{4} \times \frac{1}{6} = \frac{3}{24} = \frac{1}{8}$ hour

(b) Time for each segment in the second round:
$\frac{2}{3} \div 4 = \frac{2}{3} \times \frac{1}{4} = \frac{2}{12} = \frac{1}{6}$ hour

(c) To compare $\frac{1}{8}$ and $\frac{1}{6}$:
The LCM of 8 and 6 is 24.
$\frac{1}{8} = \frac{3}{24}$ and $\frac{1}{6} = \frac{4}{24}$

Since $\frac{4}{24} > \frac{3}{24}$, we have $\frac{1}{6} > \frac{1}{8}$.

The second round gives more time per segment.

Difference: $\frac{1}{6} - \frac{1}{8} = \frac{4}{24} - \frac{3}{24} = \frac{1}{24}$ hour

Therefore, the second round gives $\frac{1}{24}$ hour more per segment than the first round.

\section*{Question 32}
\textbf{Metadata}

\begin{itemize}
  \item Question ID: P6-FrDivPN\_P2-FrSub2nd\_sonnet4\_Recreation\_04
  \item Primary KC: FRACTIONS | Division | dividing a proper fraction by a whole number
  \item Secondary KC: FRACTIONS | Subtraction | subtracting fractions
  \item Topic: Recreation such as sports, games, exercises, music, movie, dancing, painting, fishing and other recreation activities
  \item Grade: Primary 6
\end{itemize}

\textbf{Solution}

Step 1: Find the remaining practice time after warming up.
Total practice time = $\frac{3}{4}$ hour
Warm-up time = $\frac{1}{6}$ hour

To subtract these fractions, I need a common denominator.
The LCM of 4 and 6 is 12.
$\frac{3}{4} = \frac{3 \times 3}{4 \times 3} = \frac{9}{12}$
$\frac{1}{6} = \frac{1 \times 2}{6 \times 2} = \frac{2}{12}$

Remaining practice time = $\frac{9}{12} - \frac{2}{12} = \frac{7}{12}$ hour

Step 2: Divide the remaining time equally among 3 dance sequences.
Time for each sequence = $\frac{7}{12} \div 3$
= $\frac{7}{12} \times \frac{1}{3}$
= $\frac{7}{36}$ hour

Therefore, Sarah will spend $\frac{7}{36}$ of an hour practicing each dance sequence.

\section*{Question 33}
\textbf{Metadata}

\begin{itemize}
  \item Question ID: P6-FrDivPN\_P5-FrMul2nd\_sonnet4\_Recreation\_04
  \item Primary KC: FRACTIONS | Division | dividing a proper fraction by a whole number
  \item Secondary KC: FRACTIONS | Multiplication | fraction multiplication
  \item Topic: Recreation such as sports, games, exercises, music, movie, dancing, painting, fishing and other recreation activities
  \item Grade: Primary 6
\end{itemize}

\textbf{Solution}

Step 1: Find how much paint each student receives.
Divide $\frac{3}{4}$ liters by 6 students:
$\frac{3}{4} \div 6 = \frac{3}{4} \times \frac{1}{6} = \frac{3}{24} = \frac{1}{8}$ liters per student

Step 2: Find how much paint each student uses for the sky.
Each student uses $\frac{2}{3}$ of their allocated $\frac{1}{8}$ liters:
$\frac{2}{3} \times \frac{1}{8} = \frac{2}{24} = \frac{1}{12}$ liters

Therefore, each student will use $\frac{1}{12}$ liters of paint for painting the sky.

\section*{Question 34}
\textbf{Metadata}

\begin{itemize}
  \item Question ID: P6-FrDivPP\_P2-FrAdd2nd\_sonnet4\_Recreation\_04
  \item Primary KC: FRACTIONS | Division | dividing a whole number/proper fraction by a proper fraction
  \item Secondary KC: FRACTIONS | Addition | adding fractions
  \item Topic: Recreation such as sports, games, exercises, music, movie, dancing, painting, fishing and other recreation activities
  \item Grade: Primary 6
\end{itemize}

\textbf{Solution}

Step 1: Find Sarah's total practice time by adding the fractions.
Monday's practice time: $\frac{3}{4}$ hour
Tuesday's practice time: $\frac{1}{2}$ hour

To add these fractions, I need a common denominator.
$\frac{3}{4} + \frac{1}{2} = \frac{3}{4} + \frac{2}{4} = \frac{5}{4}$ hours

Step 2: Find how many pieces Sarah will practice.
Sarah has 6 pieces total and wants to practice $\frac{2}{3}$ of them.
$6 \times \frac{2}{3} = \frac{12}{3} = 4$ pieces

Step 3: Divide the total practice time by the fraction representing the portion of pieces.
I need to divide $\frac{5}{4}$ hours by $\frac{2}{3}$ (since $\frac{2}{3}$ represents the fraction of total pieces).

$\frac{5}{4} \div \frac{2}{3} = \frac{5}{4} \times \frac{3}{2} = \frac{15}{8}$ hours

Step 4: Convert to mixed number and verify.
$\frac{15}{8} = 1\frac{7}{8}$ hours per piece

Therefore, Sarah will spend $1\frac{7}{8}$ hours practicing each selected piece.

\section*{Question 35}
\textbf{Metadata}

\begin{itemize}
  \item Question ID: P6-FrDivPP\_P5-FrMul2nd\_sonnet4\_Recreation\_04
  \item Primary KC: FRACTIONS | Division | dividing a whole number/proper fraction by a proper fraction
  \item Secondary KC: FRACTIONS | Multiplication | fraction multiplication
  \item Topic: Recreation such as sports, games, exercises, music, movie, dancing, painting, fishing and other recreation activities
  \item Grade: Primary 6
\end{itemize}

\textbf{Solution}

Step 1: Find how many containers Sarah filled with blue paint.
To find the number of containers, I need to divide the total amount of blue paint by the amount in each container.
Number of containers for blue paint = $\frac{3}{4} \div \frac{1}{8}$

To divide fractions, I multiply by the reciprocal:
$\frac{3}{4} \div \frac{1}{8} = \frac{3}{4} \times \frac{8}{1} = \frac{3 \times 8}{4 \times 1} = \frac{24}{4} = 6$

Sarah filled 6 containers with blue paint.

Step 2: Find how many containers she needs for red paint.
The number of containers for red paint = $\frac{2}{3}$ times the number of containers for blue paint
Number of containers for red paint = $\frac{2}{3} \times 6 = \frac{2 \times 6}{3} = \frac{12}{3} = 4$

Therefore, Sarah filled 6 containers with blue paint and needs 4 containers for red paint.

\section*{Question 36}
\textbf{Metadata}

\begin{itemize}
  \item Question ID: P6-PcFndWN\_P1-WNMul2nd\_sonnet4\_Recreation\_04
  \item Primary KC: PERCENTAGE | Finding the whole | finding the whole given a part and the percentage
  \item Secondary KC: WHOLE NUMBERS | Multiplication | multiplying whole numbers
  \item Topic: Recreation such as sports, games, exercises, music, movie, dancing, painting, fishing and other recreation activities
  \item Grade: Primary 6
\end{itemize}

\textbf{Solution}

Step 1: Find the total number of students who participated in the talent show.

Given information:
- $45$ students performed dance routines
- These dancers represent $25\%$ of all participants

Let the total number of participants be $x$.

Since $25\%$ of $x$ equals $45$:
$25\% \times x = 45$
$\frac{25}{100} \times x = 45$
$0.25x = 45$
$x = 45 \div 0.25$
$x = 45 \times 4$
$x = 180$

Therefore, $180$ students participated in the talent show in total.

Step 2: Calculate the total number of performance slots needed.

Since each of the $180$ students performed $3$ times:
Total performance slots $= 180 \times 3 = 540$

Answer: $180$ students participated in the talent show in total, and $540$ performance slots were needed for the evening.

\section*{Question 37}
\textbf{Metadata}

\begin{itemize}
  \item Question ID: P6-PcFndWN\_P1-WNDiv2nd\_sonnet4\_Recreation\_04
  \item Primary KC: PERCENTAGE | Finding the whole | finding the whole given a part and the percentage
  \item Secondary KC: WHOLE NUMBERS | Division | dividing whole numbers
  \item Topic: Recreation such as sports, games, exercises, music, movie, dancing, painting, fishing and other recreation activities
  \item Grade: Primary 6
\end{itemize}

\textbf{Solution}

Step 1: Find the total number of free throws Sarah attempted.

Let the total number of free throws attempted = $x$

Given that 36 successful free throws = 60\% of total attempts

$60\% \times x = 36$

$\frac{60}{100} \times x = 36$

$x = 36 \div \frac{60}{100}$

$x = 36 \times \frac{100}{60}$

$x = \frac{3600}{60} = 60$

Sarah attempted 60 free throws in total.

Step 2: Divide the total attempts equally among 5 practice drills.

Number of free throws per drill = $60 \div 5 = 12$

Therefore, Sarah should allocate 12 free throws to each drill.

\section*{Question 38}
\textbf{Metadata}

\begin{itemize}
  \item Question ID: P6-PcFndChg\_P1-WNAdd2nd\_sonnet4\_Recreation\_04
  \item Primary KC: PERCENTAGE | Finding change | finding percentage increase/decrease
  \item Secondary KC: WHOLE NUMBERS | Addition | adding whole numbers
  \item Topic: Recreation such as sports, games, exercises, music, movie, dancing, painting, fishing and other recreation activities
  \item Grade: Primary 6
\end{itemize}

\textbf{Solution}

Step 1: Find Team Tigers' percentage increase.
Team Tigers scored 45 points in the first game and 63 points in the second game.
Increase in points = $63 - 45 = 18$ points
Percentage increase = $\frac{18}{45} \times 100\% = \frac{1800}{45}\% = 40\%$

Step 2: Find Team Eagles' percentage increase.
Team Eagles scored 40 points in the first game and 58 points in the second game.
Increase in points = $58 - 40 = 18$ points
Percentage increase = $\frac{18}{40} \times 100\% = \frac{1800}{40}\% = 45\%$

Step 3: Compare the percentage increases.
Team Tigers: $40\%$ increase
Team Eagles: $45\%$ increase

Therefore, Team Eagles had a greater percentage increase of $45\%$ compared to Team Tigers' $40\%$ increase.

\section*{Question 39}
\textbf{Metadata}

\begin{itemize}
  \item Question ID: P6-RoFndDvqWN\_P1-WNSub2nd\_sonnet4\_Recreation\_04
  \item Primary KC: RATIO | Finding divided quantities | dividing a given quantity in a given ratio
  \item Secondary KC: WHOLE NUMBERS | Subtraction | subtracting whole numbers
  \item Topic: Recreation such as sports, games, exercises, music, movie, dancing, painting, fishing and other recreation activities
  \item Grade: Primary 6
\end{itemize}

\textbf{Solution}

Step 1: Find how Sarah and Marcus share the 480 points in the ratio 7:5.

Total ratio parts = $7 + 5 = 12$ parts

Value of 1 part = $480 \div 12 = 40$ points

Sarah's share = $7 \times 40 = 280$ points

Marcus's share = $5 \times 40 = 200$ points

Step 2: Find how many points Sarah has left after spending 85 points on power-ups.

Points Sarah has left = $280 - 85 = 195$ points

Therefore, Sarah has 195 points left after buying the power-ups.

\section*{Question 40}
\textbf{Metadata}

\begin{itemize}
  \item Question ID: P6-RoFndRoWN\_P1-WNMul2nd\_sonnet4\_Recreation\_04
  \item Primary KC: RATIO | Finding ratio | finding the ratio of two or three given whole numbers
  \item Secondary KC: WHOLE NUMBERS | Multiplication | multiplying whole numbers
  \item Topic: Recreation such as sports, games, exercises, music, movie, dancing, painting, fishing and other recreation activities
  \item Grade: Primary 6
\end{itemize}

\textbf{Solution}

First, I need to find the total number of laps completed by each class.

For Class 6A:
Number of students = $12$
Laps per student = $3$
Total laps by Class 6A = $12 \times 3 = 36$ laps

For Class 6B:
Number of students = $18$
Laps per student = $3$
Total laps by Class 6B = $18 \times 3 = 54$ laps

For Class 6C:
Number of students = $15$
Laps per student = $3$
Total laps by Class 6C = $15 \times 3 = 45$ laps

Now I need to find the ratio of total laps completed by Class 6A : Class 6B : Class 6C
Ratio = $36 : 54 : 45$

To simplify this ratio, I need to find the highest common factor (HCF) of $36$, $54$, and $45$.

Factors of $36$: $1, 2, 3, 4, 6, 9, 12, 18, 36$
Factors of $54$: $1, 2, 3, 6, 9, 18, 27, 54$
Factors of $45$: $1, 3, 5, 9, 15, 45$

The HCF of $36$, $54$, and $45$ is $9$.

Dividing each number by $9$:
$36 \div 9 = 4$
$54 \div 9 = 6$
$45 \div 9 = 5$

Therefore, the ratio of total laps completed by Class 6A to Class 6B to Class 6C is $4 : 6 : 5$.

\section*{Question 41}
\textbf{Metadata}

\begin{itemize}
  \item Question ID: P6-RoFndRoWN\_P6-RoSmpWN\_sonnet4\_Recreation\_04
  \item Primary KC: RATIO | Finding ratio | finding the ratio of two or three given whole numbers
  \item Secondary KC: RATIO | Simplifying | expressing a ratio in its simplest form
  \item Topic: Recreation such as sports, games, exercises, music, movie, dancing, painting, fishing and other recreation activities
  \item Grade: Primary 6
\end{itemize}

\textbf{Solution}

To find the ratio of Red House : Blue House : Green House points, I need to write the ratio using their scores and then simplify it.

Step 1: Write the ratio using the given points.
Red House : Blue House : Green House = $48 : 72 : 60$

Step 2: Find the highest common factor (HCF) of 48, 72, and 60.
Factors of 48: 1, 2, 3, 4, 6, 8, 12, 16, 24, 48
Factors of 72: 1, 2, 3, 4, 6, 8, 9, 12, 18, 24, 36, 72
Factors of 60: 1, 2, 3, 4, 5, 6, 10, 12, 15, 20, 30, 60

The HCF of 48, 72, and 60 is 12.

Step 3: Divide each number by the HCF to simplify the ratio.
$48 \div 12 = 4$
$72 \div 12 = 6$
$60 \div 12 = 5$

Therefore, the ratio of Red House's points to Blue House's points to Green House's points in its simplest form is $4 : 6 : 5$.

\section*{Question 42}
\textbf{Metadata}

\begin{itemize}
  \item Question ID: P6-RoFndTmWN\_P1-WNSub2nd\_sonnet4\_Recreation\_04
  \item Primary KC: RATIO | Finding a missing term | finding the missing term in a pair of equivalent ratios
  \item Secondary KC: WHOLE NUMBERS | Subtraction | subtracting whole numbers
  \item Topic: Recreation such as sports, games, exercises, music, movie, dancing, painting, fishing and other recreation activities
  \item Grade: Primary 6
\end{itemize}

\textbf{Solution}

First, I need to find how many successful free throws Team A made using the given ratio.

For Team A, the ratio of successful free throws to total attempts is $3:5$.
This means for every $5$ attempts, $3$ are successful.

Since Team A attempted $40$ free throws in total:
$\frac{\text{successful free throws}}{\text{total attempts}} = \frac{3}{5}$

Let $x$ be the number of successful free throws by Team A.
$\frac{x}{40} = \frac{3}{5}$

Cross multiply: $5x = 3 \times 40$
$5x = 120$
$x = 24$

So Team A made $24$ successful free throws.

Team B made $72$ successful free throws (given).

To find how many more successful free throws Team B made compared to Team A:
$72 - 24 = 48$

Therefore, Team B made $48$ more successful free throws than Team A.

\section*{Question 43}
\textbf{Metadata}

\begin{itemize}
  \item Question ID: O1-RoRepFr\_P2-FrAdd2nd\_sonnet4\_Recreation\_04
  \item Primary KC: RATIO | Representation and concept | ratios involving fractions
  \item Secondary KC: FRACTIONS | Addition | adding fractions
  \item Topic: Recreation such as sports, games, exercises, music, movie, dancing, painting, fishing and other recreation activities
  \item Grade: Secondary O-level 1
\end{itemize}

\textbf{Solution}

First, I need to find the actual ratio by working with the given fractions $\frac{2}{3} : \frac{1}{2} : \frac{5}{6}$.

To work with this ratio more easily, I'll find a common denominator for all three fractions. The LCD of 3, 2, and 6 is 6.

$\frac{2}{3} = \frac{4}{6}$, $\frac{1}{2} = \frac{3}{6}$, $\frac{5}{6} = \frac{5}{6}$

So the ratio is $\frac{4}{6} : \frac{3}{6} : \frac{5}{6}$ or $4 : 3 : 5$.

Now I need to check if Sarah's current amounts maintain this ratio.

Strawberry juice used: $\frac{4}{5}$ liters
Mango juice used: $\frac{3}{10}$ liters

First, let me add these amounts: $\frac{4}{5} + \frac{3}{10}$

To add these fractions, I need a common denominator. The LCD of 5 and 10 is 10.
$\frac{4}{5} = \frac{8}{10}$

$\frac{8}{10} + \frac{3}{10} = \frac{11}{10}$ liters of strawberry and mango juice combined.

Now I need to find the scaling factor. From the ratio $4 : 3 : 5$, the strawberry to mango ratio should be $4 : 3$.

Let me check: $\frac{\frac{4}{5}}{\frac{3}{10}} = \frac{4}{5} \times \frac{10}{3} = \frac{40}{15} = \frac{8}{3}$

The ideal ratio should be $\frac{4}{3}$, but we have $\frac{8}{3}$. This means Sarah used twice as much strawberry juice as the ratio requires.

Since the amounts don't match the original ratio, I'll work with what Sarah has and find the orange juice needed.

From the ratio $4 : 3 : 5$, if strawberry is 4 parts and Sarah used $\frac{4}{5}$ liters:
Each part = $\frac{4}{5} \div 4 = \frac{4}{5} \times \frac{1}{4} = \frac{1}{5}$ liter

For mango juice (3 parts): $3 \times \frac{1}{5} = \frac{3}{5}$ liters should be used.
But Sarah used $\frac{3}{10}$ liters.

Let me recalculate using Sarah's mango juice amount instead.
If mango juice is 3 parts and Sarah used $\frac{3}{10}$ liters:
Each part = $\frac{3}{10} \div 3 = \frac{3}{10} \times \frac{1}{3} = \frac{1}{10}$ liter

For orange juice (5 parts): $5 \times \frac{1}{10} = \frac{5}{10} = \frac{1}{2}$ liter

Therefore, Sarah should add $\frac{1}{2}$ liter of orange juice to maintain the correct ratio.

\section*{Question 44}
\textbf{Metadata}

\begin{itemize}
  \item Question ID: O1-RoRepFr\_P5-FrMul2nd\_sonnet4\_Recreation\_04
  \item Primary KC: RATIO | Representation and concept | ratios involving fractions
  \item Secondary KC: FRACTIONS | Multiplication | fraction multiplication
  \item Topic: Recreation such as sports, games, exercises, music, movie, dancing, painting, fishing and other recreation activities
  \item Grade: Secondary O-level 1
\end{itemize}

\textbf{Solution}

First, I need to find how much time is spent on violin practice using the given ratio.

The ratio of violin practice to piano practice is $\frac{3}{4} : \frac{1}{2}$.

To work with this ratio, I'll convert it to its simplest form by dividing both parts by their common factor.

$\frac{3}{4} \div \frac{1}{2} = \frac{3}{4} \times \frac{2}{1} = \frac{6}{4} = \frac{3}{2}$

So the ratio is $\frac{3}{2} : 1$, which means for every $1$ hour of piano practice, there are $\frac{3}{2}$ hours of violin practice.

Since the student spends $2$ hours on piano practice:
Violin practice time $= 2 \times \frac{3}{2} = 3$ hours

Next, I need to find how much time is spent on violin scales.
The student spends $\frac{2}{3}$ of their violin practice time on scales.

Time spent on violin scales $= \frac{2}{3} \times 3 = \frac{6}{3} = 2$ hours

Therefore, the student spends $2$ hours on violin scales.

\section*{Question 45}
\textbf{Metadata}

\begin{itemize}
  \item Question ID: O1-RoRepFr\_O1-RoSmpFr\_sonnet4\_Recreation\_04
  \item Primary KC: RATIO | Representation and concept | ratios involving fractions
  \item Secondary KC: RATIO | Simplifying | converting a ratio involving fractions to its simplest form
  \item Topic: Recreation such as sports, games, exercises, music, movie, dancing, painting, fishing and other recreation activities
  \item Grade: Secondary O-level 1
\end{itemize}

\textbf{Solution}

First, I need to identify the practice times for each song:
- Song A: $\frac{3}{4}$ hour
- Song B: $1\frac{1}{2} = \frac{3}{2}$ hours
- Song C: $\frac{5}{6}$ hour

The ratio of Song A to Song B to Song C is:
$\frac{3}{4} : \frac{3}{2} : \frac{5}{6}$

To convert this ratio to its simplest form, I need to eliminate the fractions. I'll find the least common multiple (LCM) of the denominators 4, 2, and 6.

The LCM of 4, 2, and 6 is 12.

Multiplying each term by 12:
- First term: $\frac{3}{4} \times 12 = 9$
- Second term: $\frac{3}{2} \times 12 = 18$
- Third term: $\frac{5}{6} \times 12 = 10$

So the ratio becomes $9 : 18 : 10$.

Now I need to check if this ratio can be simplified further by finding the greatest common divisor (GCD) of 9, 18, and 10.

Factors of 9: 1, 3, 9
Factors of 18: 1, 2, 3, 6, 9, 18
Factors of 10: 1, 2, 5, 10

The GCD of 9, 18, and 10 is 1.

Therefore, the ratio of practice time for Song A to Song B to Song C in its simplest form is $9 : 18 : 10$.

\section*{Question 46}
\textbf{Metadata}

\begin{itemize}
  \item Question ID: O1-RoRepDc\_P4-DcAdd2nd\_sonnet4\_Recreation\_04
  \item Primary KC: RATIO | Representation and concept | ratios involving decimals
  \item Secondary KC: DECIMALS | Addition | adding decimals
  \item Topic: Recreation such as sports, games, exercises, music, movie, dancing, painting, fishing and other recreation activities
  \item Grade: Secondary O-level 1
\end{itemize}

\textbf{Solution}

First, I need to find the total exercise time for each person by adding their swimming and running times.

Sarah's total exercise time:
Swimming time + Running time = $12.5 + 6.25 = 18.75$ minutes

Tom's total exercise time:
Swimming time + Running time = $8.75 + 4.5 = 13.25$ minutes

Now I need to find the ratio of Sarah's total time to Tom's total time:
Ratio = Sarah's time : Tom's time = $18.75 : 13.25$

To simplify this ratio, I'll multiply both terms by 100 to eliminate the decimal places:
$18.75 \times 100 : 13.25 \times 100 = 1875 : 1325$

Next, I'll find the greatest common divisor (GCD) of 1875 and 1325:
$1875 = 3 \times 5^4$
$1325 = 5^2 \times 53$

The GCD is $5^2 = 25$

Dividing both terms by 25:
$1875 \div 25 : 1325 \div 25 = 75 : 53$

Therefore, the ratio of Sarah's total exercise time to Tom's total exercise time is $75 : 53$.

\section*{Question 47}
\textbf{Metadata}

\begin{itemize}
  \item Question ID: O1-RoRepDc\_O1-RoSmpDc\_sonnet4\_Recreation\_04
  \item Primary KC: RATIO | Representation and concept | ratios involving decimals
  \item Secondary KC: RATIO | Simplifying | converting a ratio involving decimals to its simplest form
  \item Topic: Recreation such as sports, games, exercises, music, movie, dancing, painting, fishing and other recreation activities
  \item Grade: Secondary O-level 1
\end{itemize}

\textbf{Solution}

Step 1: Identify the given information.
Sarah made 12.5 successful free throws.
David made 8.4 successful free throws.

Step 2: Write the ratio of Sarah's successful free throws to David's successful free throws.
Ratio = Sarah's successful free throws : David's successful free throws
Ratio = $12.5 : 8.4$

Step 3: Convert the decimal ratio to its simplest form.
To eliminate decimals, multiply both terms by 10:
$12.5 \times 10 : 8.4 \times 10 = 125 : 84$

Step 4: Find the highest common factor (HCF) of 125 and 84.
Factors of 125: $125 = 5^3$
Factors of 84: $84 = 2^2 \times 3 \times 7$

Since 125 and 84 share no common factors, their HCF is 1.

Step 5: Simplify the ratio.
Since the HCF is 1, the ratio $125 : 84$ is already in its simplest form.

Therefore, the ratio of Sarah's successful free throws to David's successful free throws in its simplest form is $125 : 84$.

\section*{Question 48}
\textbf{Metadata}

\begin{itemize}
  \item Question ID: O1-PcRep2q\_O1-PcCnv2Fr\_sonnet4\_Recreation\_04
  \item Primary KC: PERCENTAGE | Representation and concept | comparing two quantities by percentage
  \item Secondary KC: PERCENTAGE | Conversion to fraction | expressing percentage as a fraction
  \item Topic: Recreation such as sports, games, exercises, music, movie, dancing, painting, fishing and other recreation activities
  \item Grade: Secondary O-level 1
\end{itemize}

\textbf{Solution}

To find Team B's score as a percentage of Team A's score:

Percentage = $\frac{\text{Team B's score}}{\text{Team A's score}} \times 100\%$

Percentage = $\frac{70}{84} \times 100\%$

First, simplify the fraction $\frac{70}{84}$:
GCD of $70$ and $84$ is $14$
$\frac{70}{84} = \frac{70 \div 14}{84 \div 14} = \frac{5}{6}$

Therefore: Percentage = $\frac{5}{6} \times 100\% = \frac{500}{6}\% = 83\frac{1}{3}\%$

To convert this percentage to a fraction:
$83\frac{1}{3}\% = \frac{250}{3}\% = \frac{250}{3} \times \frac{1}{100} = \frac{250}{300} = \frac{5}{6}$

Therefore, Team B's score is $83\frac{1}{3}\%$ of Team A's score, which as a fraction in simplest form is $\frac{5}{6}$.

\section*{Question 49}
\textbf{Metadata}

\begin{itemize}
  \item Question ID: O1-PcFndRslt\_P1-WNAdd2nd\_sonnet4\_Recreation\_04
  \item Primary KC: PERCENTAGE | Finding result after change | increasing/decreasing a quantity by a given percentage
  \item Secondary KC: WHOLE NUMBERS | Addition | adding whole numbers
  \item Topic: Recreation such as sports, games, exercises, music, movie, dancing, painting, fishing and other recreation activities
  \item Grade: Secondary O-level 1
\end{itemize}

\textbf{Solution}

Step 1: Find the increase in Sarah's free throws due to the $15\%$ improvement.

Increase $= 15\% \times 180 = \frac{15}{100} \times 180 = 27$ free throws

Step 2: Find the number of free throws from her regular training routine this month.

Regular free throws this month $= 180 + 27 = 207$ free throws

Step 3: Add the extra free throws from practice sessions to find the total.

Total free throws this month $= 207 + 45 = 252$ free throws

Therefore, Sarah successfully made $252$ free throws in total this month.

\section*{Question 50}
\textbf{Metadata}

\begin{itemize}
  \item Question ID: O1-PcFndRslt\_P1-WNMul2nd\_sonnet4\_Recreation\_04
  \item Primary KC: PERCENTAGE | Finding result after change | increasing/decreasing a quantity by a given percentage
  \item Secondary KC: WHOLE NUMBERS | Multiplication | multiplying whole numbers
  \item Topic: Recreation such as sports, games, exercises, music, movie, dancing, painting, fishing and other recreation activities
  \item Grade: Secondary O-level 1
\end{itemize}

\textbf{Solution}

Step 1: Find Sarah's daily push-ups in February after the $25\%$ increase.

Increase in push-ups = $25\%$ of $24$
$= \frac{25}{100} \times 24$
$= 0.25 \times 24$
$= 6$ push-ups

Daily push-ups in February = $24 + 6 = 30$ push-ups

Step 2: Calculate total push-ups in February.

Number of exercise days in February = $6$ days per week $\times 4$ weeks = $24$ days

Total push-ups in February = $30 \times 24 = 720$ push-ups

Therefore, Sarah will complete $720$ push-ups in total during February.

\section*{Question 51}
\textbf{Metadata}

\begin{itemize}
  \item Question ID: O1-PcRepRvs\_O1-PcCnv2Fr\_sonnet4\_Recreation\_04
  \item Primary KC: PERCENTAGE | Representation and concept | reverse percentages
  \item Secondary KC: PERCENTAGE | Conversion to fraction | expressing percentage as a fraction
  \item Topic: Recreation such as sports, games, exercises, music, movie, dancing, painting, fishing and other recreation activities
  \item Grade: Secondary O-level 1
\end{itemize}

\textbf{Solution}

First, I need to express $25\%$ as a fraction in its simplest form.

$25\% = \frac{25}{100}$

To simplify this fraction, I find the greatest common divisor of $25$ and $100$:
$\gcd(25, 100) = 25$

$\frac{25}{100} = \frac{25 \div 25}{100 \div 25} = \frac{1}{4}$

So $25\% = \frac{1}{4}$

Now I need to find the total practice session duration using reverse percentages.

If the warm-up routine represents $\frac{1}{4}$ of the total practice session, and the warm-up lasted $45$ minutes, then:

$\frac{1}{4} \times \text{total time} = 45$ minutes

To find the total time, I multiply both sides by $4$:
$\text{total time} = 45 \times 4 = 180$ minutes

Therefore, Sarah's entire practice session is $180$ minutes long, and $25\%$ expressed as a fraction in simplest form is $\frac{1}{4}$.

\section*{Question 52}
\textbf{Metadata}

\begin{itemize}
  \item Question ID: O1-AgRepExSq\_O1-AgEvlEx\_sonnet4\_Recreation\_04
  \item Primary KC: ALGEBRA | Representation and concept | translation of simple real-world situations into quadratic algebraic expressions
  \item Secondary KC: ALGEBRA | Evaluation | evaluation of algebraic expressions and formulae
  \item Topic: Recreation such as sports, games, exercises, music, movie, dancing, painting, fishing and other recreation activities
  \item Grade: Secondary O-level 1
\end{itemize}

\textbf{Solution}

(a) To find the total number of successful free throws over the first 3 days, we need to add the expressions for day 1, day 2, and day 3.

For day 1: $n = 1$, so the expression is $2(1)^2 + 3(1) - 5 = 2 + 3 - 5$
For day 2: $n = 2$, so the expression is $2(2)^2 + 3(2) - 5 = 8 + 6 - 5$
For day 3: $n = 3$, so the expression is $2(3)^2 + 3(3) - 5 = 18 + 9 - 5$

Total expression = $(2 + 3 - 5) + (8 + 6 - 5) + (18 + 9 - 5)$
                = $2 + 3 - 5 + 8 + 6 - 5 + 18 + 9 - 5$
                = $(2 + 8 + 18) + (3 + 6 + 9) + (-5 - 5 - 5)$
                = $28 + 18 - 15$
                = $31$

(b) For day 3, we substitute $n = 3$ into the expression $2n^2 + 3n - 5$:
$2(3)^2 + 3(3) - 5$
$= 2(9) + 9 - 5$
$= 18 + 9 - 5$
$= 22$

Therefore, he made 22 successful free throws on day 3.

(c) Using the result from part (a), the total number of successful free throws made over the first 3 days is 31.

Alternatively, we can verify by calculating each day separately:
Day 1: $2(1)^2 + 3(1) - 5 = 2 + 3 - 5 = 0$
Day 2: $2(2)^2 + 3(2) - 5 = 8 + 6 - 5 = 9$
Day 3: $2(3)^2 + 3(3) - 5 = 18 + 9 - 5 = 22$
Total: $0 + 9 + 22 = 31$

\section*{Question 53}
\textbf{Metadata}

\begin{itemize}
  \item Question ID: O1-AgRepnth\_O1-AgEvlEx\_sonnet4\_Recreation\_04
  \item Primary KC: ALGEBRA | Representation and concept | recognising and representing patterns/relationships by finding an algebraic expression for the nth term
  \item Secondary KC: ALGEBRA | Evaluation | evaluation of algebraic expressions and formulae
  \item Topic: Recreation such as sports, games, exercises, music, movie, dancing, painting, fishing and other recreation activities
  \item Grade: Secondary O-level 1
\end{itemize}

\textbf{Solution}

(a) Let's identify the pattern:
- Week 1: 15 km
- Week 2: 15 + 3 = 18 km
- Week 3: 18 + 3 = 21 km
- Week 4: 21 + 3 = 24 km

This is an arithmetic sequence with first term $a = 15$ and common difference $d = 3$.

Using the formula for the $n$th term of an arithmetic sequence:
$T_n = a + (n-1)d$
$T_n = 15 + (n-1) \times 3$
$T_n = 15 + 3n - 3$
$T_n = 12 + 3n$

Therefore, the distance Sarah runs in the $n$th week is $(12 + 3n)$ km.

(b) To find the distance in the 8th week, substitute $n = 8$ into the expression:
$T_8 = 12 + 3(8)$
$T_8 = 12 + 24$
$T_8 = 36$

Sarah will run 36 km in the 8th week.

(c) To find when Sarah first runs more than 40 km, we need to solve:
$12 + 3n > 40$
$3n > 40 - 12$
$3n > 28$
$n > \frac{28}{3}$
$n > 9.33...$

Since $n$ must be a whole number, the smallest value is $n = 10$.

Let's verify: $T_{10} = 12 + 3(10) = 12 + 30 = 42$ km

Therefore, Sarah will first run more than 40 km in the 10th week.

\section*{Question 54}
\textbf{Metadata}

\begin{itemize}
  \item Question ID: O2-RoRepDP\_P1-WNDiv2nd\_sonnet4\_Recreation\_04
  \item Primary KC: RATIO | Representation and concept | direct proportion
  \item Secondary KC: WHOLE NUMBERS | Division | dividing whole numbers
  \item Topic: Recreation such as sports, games, exercises, music, movie, dancing, painting, fishing and other recreation activities
  \item Grade: Secondary O-level 2
\end{itemize}

\textbf{Solution}

First, I need to find the shooting accuracy ratio from the initial practice session.

Shooting accuracy = $\frac{\text{successful shots}}{\text{total shots}} = \frac{72}{120}$

Simplifying this ratio: $\frac{72}{120} = \frac{72 \div 24}{120 \div 24} = \frac{3}{5}$

So the team has a shooting accuracy of $\frac{3}{5}$ or 60\%.

Since the shooting accuracy remains constant (direct proportion), I can set up the proportion:
$\frac{\text{successful shots}}{\text{total shots}} = \frac{3}{5}$

For the next practice session where they want 108 successful shots:
$\frac{108}{\text{total shots}} = \frac{3}{5}$

Cross-multiplying: $3 \times \text{total shots} = 108 \times 5$
$3 \times \text{total shots} = 540$
$\text{total shots} = 540 \div 3 = 180$

Therefore, they must attempt 180 total shots to achieve 108 successful shots.

To find how many shots each player takes:
Total shots per player = $180 \div 9 = 20$

Therefore, the team must attempt 180 total shots, and each player will take 20 shots.

\section*{Question 55}
\textbf{Metadata}

\begin{itemize}
  \item Question ID: O2-AgSlvIneq\_O2-AgRepIneq\_sonnet4\_Recreation\_04
  \item Primary KC: ALGEBRA | Solving | solving simple linear inequalities with one variable
  \item Secondary KC: ALGEBRA | Representation and concept | translation of simple real-world situations to simple linear inequalities with one variable
  \item Topic: Recreation such as sports, games, exercises, music, movie, dancing, painting, fishing and other recreation activities
  \item Grade: Secondary O-level 2
\end{itemize}

\textbf{Solution}

Let $t$ be the total time Sarah needs to practice (in minutes).

First, I need to find Sarah's rate of making successful free throws.
In 45 minutes, she made 28 successful free throws.
Rate = $\frac{28}{45}$ free throws per minute

Next, I'll set up the inequality based on the qualification requirement.
Number of successful free throws in $t$ minutes = $\frac{28}{45} \times t$

For qualification, she needs at least 50 successful free throws:
$\frac{28}{45} \times t \geq 50$

Now I'll solve the inequality:
$\frac{28t}{45} \geq 50$

Multiply both sides by 45:
$28t \geq 50 \times 45$
$28t \geq 2250$

Divide both sides by 28:
$t \geq \frac{2250}{28}$
$t \geq 80.357...$

Since time must be a whole number of minutes, the minimum total time is 81 minutes.

Therefore, Sarah needs to practice for at least 81 minutes to meet the qualification requirement.

\section*{Question 56}
\textbf{Metadata}

\begin{itemize}
  \item Question ID: O2-AgSlvSq1v\_O1-AgRepEq\_sonnet4\_Recreation\_04
  \item Primary KC: ALGEBRA | Solving | solving quadratic equations in one variable
  \item Secondary KC: ALGEBRA | Representation and concept | translation of simple real-world situations to equations
  \item Topic: Recreation such as sports, games, exercises, music, movie, dancing, painting, fishing and other recreation activities
  \item Grade: Secondary O-level 2
\end{itemize}

\textbf{Solution}

(a) We need to find when the basketball is at a height of 3.6 meters.

Set up the equation: $h = 3.6$
$-5t^2 + 6t + 2.1 = 3.6$
$-5t^2 + 6t + 2.1 - 3.6 = 0$
$-5t^2 + 6t - 1.5 = 0$

Multiply by $-1$: $5t^2 - 6t + 1.5 = 0$

Multiply by 2 to eliminate decimals: $10t^2 - 12t + 3 = 0$

Using the quadratic formula: $t = \frac{12 \pm \sqrt{144 - 120}}{20} = \frac{12 \pm \sqrt{24}}{20} = \frac{12 \pm 2\sqrt{6}}{20} = \frac{6 \pm \sqrt{6}}{10}$

$t_1 = \frac{6 - \sqrt{6}}{10} \approx 0.355$ seconds
$t_2 = \frac{6 + \sqrt{6}}{10} \approx 0.845$ seconds

The basketball will be at 3.6 meters height at approximately 0.355 seconds (going up) and 0.845 seconds (coming down).

(b) We need to find when the basketball hits the ground.

Set up the equation: $h = 0$
$-5t^2 + 6t + 2.1 = 0$

Multiply by $-1$: $5t^2 - 6t - 2.1 = 0$

Multiply by 10: $50t^2 - 60t - 21 = 0$

Using the quadratic formula: $t = \frac{60 \pm \sqrt{3600 + 4200}}{100} = \frac{60 \pm \sqrt{7800}}{100} = \frac{60 \pm 10\sqrt{78}}{100} = \frac{6 \pm \sqrt{78}}{10}$

Since time must be positive: $t = \frac{6 + \sqrt{78}}{10} \approx 1.48$ seconds

The basketball will hit the ground after approximately 1.48 seconds.

\section*{Question 57}
\textbf{Metadata}

\begin{itemize}
  \item Question ID: O2-SPFndmdn\_O3-SPFndrng\_sonnet4\_Recreation\_04
  \item Primary KC: STATISTICS AND PROBABILITY | Finding median | Finding median for a set of data
  \item Secondary KC: STATISTICS AND PROBABILITY | Finding range | finding range as measures of spread for a set of data 
  \item Topic: Recreation such as sports, games, exercises, music, movie, dancing, painting, fishing and other recreation activities
  \item Grade: Secondary O-level 2
\end{itemize}

\textbf{Solution}

Step 1: Arrange the data in ascending order to find the median.
Data: 15, 22, 18, 25, 20, 17, 23, 19, 21
Arranged in ascending order: 15, 17, 18, 19, 20, 21, 22, 23, 25

Step 2: Find the median.
Since there are 9 values (odd number), the median is the middle value, which is the 5th value.
Median = 20 shuttlecocks

Step 3: Find the range.
Range = Highest value - Lowest value
Range = 25 - 15 = 10 shuttlecocks

Step 4: Calculate shuttlecocks needed for 10 training sessions based on median usage.
Shuttlecocks needed = Median × Number of sessions
Shuttlecocks needed = 20 × 10 = 200 shuttlecocks

Therefore:
- The median number of shuttlecocks used is 20
- The range of shuttlecocks used is 10
- The coach should order 200 shuttlecocks for 10 training sessions

\section*{Question 58}
\textbf{Metadata}

\begin{itemize}
  \item Question ID: O2-SPFndmean\_O3-BPRepSN\_sonnet4\_Recreation\_04
  \item Primary KC: STATISTICS AND PROBABILITY | Finding mean deviation | calculation of the mean for a set of data
  \item Secondary KC: BASE AND POWER | Representation and concept  | use of standard form Ax10^n , where n is an integer, and 1<= A<= 10
  \item Topic: Recreation such as sports, games, exercises, music, movie, dancing, painting, fishing and other recreation activities
  \item Grade: Secondary O-level 2
\end{itemize}

\textbf{Solution}

Step 1: Convert all reaction times from standard form to decimal form.
$1.2 \times 10^{-1} = 0.12$ seconds
$1.5 \times 10^{-1} = 0.15$ seconds
$9.8 \times 10^{-2} = 0.098$ seconds
$1.1 \times 10^{-1} = 0.11$ seconds
$1.3 \times 10^{-1} = 0.13$ seconds
$1.4 \times 10^{-1} = 0.14$ seconds
$1.0 \times 10^{-1} = 0.10$ seconds
$1.6 \times 10^{-1} = 0.16$ seconds

Step 2: Calculate the mean reaction time.
Sum of all reaction times = $0.12 + 0.15 + 0.098 + 0.11 + 0.13 + 0.14 + 0.10 + 0.16 = 1.018$ seconds
Mean = $\frac{1.018}{8} = 0.12725$ seconds

Step 3: Calculate the absolute deviations from the mean.
$|0.12 - 0.12725| = 0.00725$
$|0.15 - 0.12725| = 0.02275$
$|0.098 - 0.12725| = 0.02925$
$|0.11 - 0.12725| = 0.01725$
$|0.13 - 0.12725| = 0.00275$
$|0.14 - 0.12725| = 0.01275$
$|0.10 - 0.12725| = 0.02725$
$|0.16 - 0.12725| = 0.03275$

Step 4: Calculate the mean deviation.
Sum of absolute deviations = $0.00725 + 0.02275 + 0.02925 + 0.01725 + 0.00275 + 0.01275 + 0.02725 + 0.03275 = 0.152$ seconds
Mean deviation = $\frac{0.152}{8} = 0.019$ seconds

Therefore, the mean reaction time is $0.12725$ seconds and the mean deviation is $0.019$ seconds.

\section*{Question 59}
\textbf{Metadata}

\begin{itemize}
  \item Question ID: O3-BPOpr\_O3-BPRepNegI\_sonnet4\_Recreation\_04
  \item Primary KC: BASE AND POWER | Operations | laws of indices
  \item Secondary KC: BASE AND POWER | Representation and concept  | negative indices
  \item Topic: Recreation such as sports, games, exercises, music, movie, dancing, painting, fishing and other recreation activities
  \item Grade: Secondary O-level 3/4
\end{itemize}

\textbf{Solution}

To find Alex's final score, I need to multiply his three round scores and then divide by $2^4$.

First, let me identify the scores from each round:
- Round 1: $2^5$ points
- Round 2: $2^{-2}$ points  
- Round 3: $2^3$ points

Next, I'll multiply all three scores together:
$2^5 \times 2^{-2} \times 2^3$

Using the law of indices for multiplication (when multiplying powers with the same base, add the exponents):
$2^5 \times 2^{-2} \times 2^3 = 2^{5 + (-2) + 3} = 2^{5 - 2 + 3} = 2^6$

Now I need to divide this result by $2^4$:
$\frac{2^6}{2^4}$

Using the law of indices for division (when dividing powers with the same base, subtract the exponents):
$\frac{2^6}{2^4} = 2^{6-4} = 2^2$

Therefore, Alex's final score is $2^2$ points.

To verify: $2^2 = 4$ points, which makes sense as a reasonable final score for a gaming tournament.

\section*{Question 60}
\textbf{Metadata}

\begin{itemize}
  \item Question ID: O3-MXMulSM\_O3-MXAdd\_sonnet4\_Recreation\_04
  \item Primary KC: MATRICES | Multiplication | product of a scalar quantity and a matrix
  \item Secondary KC: MATRICES | Addition | addition of matrices
  \item Topic: Recreation such as sports, games, exercises, music, movie, dancing, painting, fishing and other recreation activities
  \item Grade: Secondary O-level 3/4
\end{itemize}

\textbf{Solution}

(a) To find the total bonus points matrix for Week 1, we multiply matrix $A$ by the scalar 3:

Bonus points matrix = $3A = 3 \begin{pmatrix} 12 & 8 & 15 \\ 20 & 25 & 18 \\ 6 & 10 & 8 \end{pmatrix}$

$= \begin{pmatrix} 3(12) & 3(8) & 3(15) \\ 3(20) & 3(25) & 3(18) \\ 3(6) & 3(10) & 3(8) \end{pmatrix}$

$= \begin{pmatrix} 36 & 24 & 45 \\ 60 & 75 & 54 \\ 18 & 30 & 24 \end{pmatrix}$

(b) To find the combined participation matrix, we add matrices $A$ and $B$:

Combined matrix = $A + B = \begin{pmatrix} 12 & 8 & 15 \\ 20 & 25 & 18 \\ 6 & 10 & 8 \end{pmatrix} + \begin{pmatrix} 15 & 12 & 18 \\ 22 & 28 & 20 \\ 8 & 12 & 10 \end{pmatrix}$

$= \begin{pmatrix} 12+15 & 8+12 & 15+18 \\ 20+22 & 25+28 & 18+20 \\ 6+8 & 10+12 & 8+10 \end{pmatrix}$

$= \begin{pmatrix} 27 & 20 & 33 \\ 42 & 53 & 38 \\ 14 & 22 & 18 \end{pmatrix}$

(c) From the bonus points matrix calculated in part (a), the tennis players (third row) earned bonus points as follows:
Morning: 18 points, Afternoon: 30 points, Evening: 24 points

Total bonus points for tennis players = $18 + 30 + 24 = 72$ points

\section*{Question 61}
\textbf{Metadata}

\begin{itemize}
  \item Question ID: O3-MXSub\_O3-MXAdd\_sonnet4\_Recreation\_04
  \item Primary KC: MATRICES | Subtraction | subtraction of matrices
  \item Secondary KC: MATRICES | Addition | addition of matrices
  \item Topic: Recreation such as sports, games, exercises, music, movie, dancing, painting, fishing and other recreation activities
  \item Grade: Secondary O-level 3/4
\end{itemize}

\textbf{Solution}

**Solution:**

**(a) Finding the change in scores matrix $C$:**

To find the change from Week 1 to Week 2, we calculate $C = B - A$:

$C = \begin{pmatrix} 88 & 89 & 82 \\ 80 & 85 & 92 \\ 87 & 86 & 90 \end{pmatrix} - \begin{pmatrix} 85 & 92 & 78 \\ 76 & 88 & 95 \\ 90 & 82 & 87 \end{pmatrix}$

$C = \begin{pmatrix} 88-85 & 89-92 & 82-78 \\ 80-76 & 85-88 & 92-95 \\ 87-90 & 86-82 & 90-87 \end{pmatrix}$

$C = \begin{pmatrix} 3 & -3 & 4 \\ 4 & -3 & -3 \\ -3 & 4 & 3 \end{pmatrix}$

**(b) Finding the total scores matrix $D$:**

To find the total scores for both weeks, we calculate $D = A + B$:

$D = \begin{pmatrix} 85 & 92 & 78 \\ 76 & 88 & 95 \\ 90 & 82 & 87 \end{pmatrix} + \begin{pmatrix} 88 & 89 & 82 \\ 80 & 85 & 92 \\ 87 & 86 & 90 \end{pmatrix}$

$D = \begin{pmatrix} 85+88 & 92+89 & 78+82 \\ 76+80 & 88+85 & 95+92 \\ 90+87 & 82+86 & 87+90 \end{pmatrix}$

$D = \begin{pmatrix} 173 & 181 & 160 \\ 156 & 173 & 187 \\ 177 & 168 & 177 \end{pmatrix}$

**(c) Finding the final scores matrix $F$ with bonus points:**

To find the final scores with bonus points, we calculate $F = D + E$:

$F = \begin{pmatrix} 173 & 181 & 160 \\ 156 & 173 & 187 \\ 177 & 168 & 177 \end{pmatrix} + \begin{pmatrix} 5 & 3 & 4 \\ 2 & 6 & 1 \\ 3 & 4 & 5 \end{pmatrix}$

$F = \begin{pmatrix} 173+5 & 181+3 & 160+4 \\ 156+2 & 173+6 & 187+1 \\ 177+3 & 168+4 & 177+5 \end{pmatrix}$

$F = \begin{pmatrix} 178 & 184 & 164 \\ 158 & 179 & 188 \\ 180 & 172 & 182 \end{pmatrix}$

\section*{Question 62}
\textbf{Metadata}

\begin{itemize}
  \item Question ID: O3-SPAddProb\_O2-SPRepPrSE\_sonnet4\_Recreation\_04
  \item Primary KC: STATISTICS AND PROBABILITY | Addition | addition of probabilities
  \item Secondary KC: STATISTICS AND PROBABILITY | Representation and concept | probability of single events
  \item Topic: Recreation such as sports, games, exercises, music, movie, dancing, painting, fishing and other recreation activities
  \item Grade: Secondary O-level 3/4
\end{itemize}

\textbf{Solution}

To find the probability that Sarah wins a prize, we need to find the probability that she gets either a red section on the first wheel OR a green section on the second wheel (or both).

First, let's find the individual probabilities:

Probability of getting red on first wheel:
$P(\text{Red}) = \frac{3}{8}$

Probability of getting green on second wheel:
$P(\text{Green}) = \frac{4}{10} = \frac{2}{5}$

Since the two wheel spins are independent events, we can find the probability of getting both red AND green:
$P(\text{Red and Green}) = P(\text{Red}) \times P(\text{Green}) = \frac{3}{8} \times \frac{2}{5} = \frac{6}{40} = \frac{3}{20}$

Using the addition rule for probabilities:
$P(\text{Red or Green}) = P(\text{Red}) + P(\text{Green}) - P(\text{Red and Green})$

$P(\text{Red or Green}) = \frac{3}{8} + \frac{2}{5} - \frac{3}{20}$

To add these fractions, we need a common denominator. The LCM of 8, 5, and 20 is 40.

$P(\text{Red or Green}) = \frac{15}{40} + \frac{16}{40} - \frac{6}{40} = \frac{25}{40} = \frac{5}{8}$

Therefore, the probability that Sarah wins a prize is $\frac{5}{8}$ or $0.625$.

\section*{Question 63}
\textbf{Metadata}

\begin{itemize}
  \item Question ID: O3-SPAddProb\_O3-SPFndPrCE\_sonnet4\_Recreation\_04
  \item Primary KC: STATISTICS AND PROBABILITY | Addition | addition of probabilities
  \item Secondary KC: STATISTICS AND PROBABILITY | Finding probability | probability of simple combined events
  \item Topic: Recreation such as sports, games, exercises, music, movie, dancing, painting, fishing and other recreation activities
  \item Grade: Secondary O-level 3/4
\end{itemize}

\textbf{Solution}

Let me define the events:
- Event A: Getting red on the first wheel
- Event B: Getting green on the second wheel

First, I'll find the individual probabilities:
$P(A) = \frac{3}{8}$ (3 red sections out of 8 total sections)
$P(B) = \frac{2}{6} = \frac{1}{3}$ (2 green sections out of 6 total sections)

Since the wheels are independent:
$P(A \cap B) = P(A) \times P(B) = \frac{3}{8} \times \frac{1}{3} = \frac{1}{8}$

Sarah wins a small prize if she gets red OR green (or both), which is the event $(A \cup B)$.

Using the addition rule for probabilities:
$P(A \cup B) = P(A) + P(B) - P(A \cap B)$
$P(A \cup B) = \frac{3}{8} + \frac{1}{3} - \frac{1}{8}$
$P(A \cup B) = \frac{2}{8} + \frac{1}{3}$
$P(A \cup B) = \frac{1}{4} + \frac{1}{3}$

To add these fractions, I need a common denominator:
$P(A \cup B) = \frac{3}{12} + \frac{4}{12} = \frac{7}{12}$

Therefore, the probability that Sarah wins a small prize is $\frac{7}{12}$.

\section*{Question 64}
\textbf{Metadata}

\begin{itemize}
  \item Question ID: O3-SPMulProb\_O3-SPFndPrCE\_sonnet4\_Recreation\_04
  \item Primary KC: STATISTICS AND PROBABILITY | Multiplication | multiplication of probabilities
  \item Secondary KC: STATISTICS AND PROBABILITY | Finding probability | probability of simple combined events
  \item Topic: Recreation such as sports, games, exercises, music, movie, dancing, painting, fishing and other recreation activities
  \item Grade: Secondary O-level 3/4
\end{itemize}

\textbf{Solution}

Let's define the events:
- Event A: Sarah hits the target in the dart challenge
- Event B: Sarah draws a red ball

Given information:
- $P(A) = \frac{3}{5}$
- For event B: There are 4 red balls out of 12 total balls, so $P(B) = \frac{4}{12} = \frac{1}{3}$
- $P(A') = 1 - \frac{3}{5} = \frac{2}{5}$
- $P(B') = 1 - \frac{1}{3} = \frac{2}{3}$

(a) To win the main prize, Sarah must succeed in both challenges:
$P(\text{main prize}) = P(A \text{ and } B) = P(A) \times P(B)$
$P(\text{main prize}) = \frac{3}{5} \times \frac{1}{3} = \frac{3}{15} = \frac{1}{5}$

(b) To win the consolation prize, Sarah must succeed in exactly one challenge:
$P(\text{consolation prize}) = P(A \text{ and } B') + P(A' \text{ and } B)$
$P(A \text{ and } B') = P(A) \times P(B') = \frac{3}{5} \times \frac{2}{3} = \frac{6}{15} = \frac{2}{5}$
$P(A' \text{ and } B) = P(A') \times P(B) = \frac{2}{5} \times \frac{1}{3} = \frac{2}{15}$
$P(\text{consolation prize}) = \frac{2}{5} + \frac{2}{15} = \frac{6}{15} + \frac{2}{15} = \frac{8}{15}$

Therefore:
(a) The probability that Sarah wins the main prize is $\frac{1}{5}$
(b) The probability that Sarah wins the consolation prize is $\frac{8}{15}$

\end{document}
