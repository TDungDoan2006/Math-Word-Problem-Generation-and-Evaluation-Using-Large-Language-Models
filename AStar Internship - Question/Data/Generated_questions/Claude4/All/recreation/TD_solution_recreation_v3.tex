\documentclass{article}
\usepackage[utf8]{inputenc}
\usepackage{amsmath}
\usepackage{amsfonts}
\usepackage{amssymb}
\usepackage{graphicx}
\usepackage{hyperref}
\title{'TD Solutions recreation v3 CLAUDE '}
\author{Tien Dung Doan}
\begin{document}
\maketitle
\section*{Question 1}
\textbf{Metadata}

\begin{itemize}
  \item Question ID: P3-WNAdd4d\_P1-WNCmp\_sonnet4\_Recreation\_05
  \item Primary KC: WHOLE NUMBERS | Addition | adding whole numbers up to 4 digits
  \item Secondary KC: WHOLE NUMBERS | Comparison and ordering | comparing and ordering whole numbers
  \item Topic: Recreation such as sports, games, exercises, music, movie, dancing, painting, fishing and other recreation activities
  \item Grade: Primary 3
\end{itemize}

\textbf{Solution}

Step 1: Calculate the total score for Team A.
Team A's total score = $1,245 + 876 = 2,121$ points

Step 2: Calculate the total score for Team B.
Team B's total score = $987 + 1,298 = 2,285$ points

Step 3: Calculate the total score for Team C.
Team C's total score = $1,156 + 945 = 2,101$ points

Step 4: Compare the total scores to determine the winner.
Team A: $2,121$ points
Team B: $2,285$ points
Team C: $2,101$ points

Since $2,285 > 2,121 > 2,101$, Team B has the highest score.

Therefore, Team B won the tournament with $2,285$ points.

\section*{Question 2}
\textbf{Metadata}

\begin{itemize}
  \item Question ID: P3-WNSub4d\_P1-WNAdd2nd\_sonnet4\_Recreation\_04
  \item Primary KC: WHOLE NUMBERS | Subtraction | subtracting whole numbers up to 4 digits
  \item Secondary KC: WHOLE NUMBERS | Addition | adding whole numbers
  \item Topic: Recreation such as sports, games, exercises, music, movie, dancing, painting, fishing and other recreation activities
  \item Grade: Primary 3
\end{itemize}

\textbf{Solution}

Step 1: Find the total number of stickers given out on the first day.
Stickers given out in the morning: $678$
Stickers given out in the afternoon: $529$
Total stickers given out: $678 + 529 = 1,207$

Step 2: Find how many stickers Sarah has left.
Original number of stickers: $2,450$
Stickers given out: $1,207$
Stickers left: $2,450 - 1,207 = 1,243$

Therefore, Sarah has $1,243$ stickers left.

\section*{Question 3}
\textbf{Metadata}

\begin{itemize}
  \item Question ID: P3-WNDivRmd3d\_P1-WNSub2nd\_sonnet4\_Recreation\_04
  \item Primary KC: WHOLE NUMBERS | Division | dividing whole numbers up to 3 digits by 1 digit with remainder 
  \item Secondary KC: WHOLE NUMBERS | Subtraction | subtracting whole numbers
  \item Topic: Recreation such as sports, games, exercises, music, movie, dancing, painting, fishing and other recreation activities
  \item Grade: Primary 3
\end{itemize}

\textbf{Solution}

First, I need to find how many complete teams can be formed and how many students will be left without a team.

Dividing the total number of students by the number of players per team:
$312 \div 5 = 62$ remainder $2$

This means $62$ complete teams can be formed, and $2$ students will be left without a team.

To verify: $62 \times 5 = 310$ students are in complete teams
$312 - 310 = 2$ students are left without a team

Next, I need to find how many helper roles will be left after assigning them to the $2$ students who cannot join teams.

Helper roles remaining = Total helper roles - Helper roles assigned
Helper roles remaining = $350 - 2 = 348$

Therefore, there will be $348$ helper roles left.

\section*{Question 4}
\textbf{Metadata}

\begin{itemize}
  \item Question ID: P3-WNMul3d1d\_P1-WNCmp\_sonnet4\_Recreation\_04
  \item Primary KC: WHOLE NUMBERS | Multiplication | multiplying whole numbers up to 3 digits by 1 digit
  \item Secondary KC: WHOLE NUMBERS | Comparison and ordering | comparing and ordering whole numbers
  \item Topic: Recreation such as sports, games, exercises, music, movie, dancing, painting, fishing and other recreation activities
  \item Grade: Primary 3
\end{itemize}

\textbf{Solution}

To find how many stickers each person will have, I need to multiply the number of albums by the number of stickers each album can hold.

Sarah's stickers:
$126 \times 4$
$= 126 \times 4$
$= 504$ stickers

Tom's stickers:
$148 \times 3$
$= 148 \times 3$
$= 444$ stickers

Lisa's stickers:
$235 \times 2$
$= 235 \times 2$
$= 470$ stickers

Now I need to compare the three totals to find who has the most:
Sarah: $504$ stickers
Tom: $444$ stickers
Lisa: $470$ stickers

Comparing the numbers: $504 > 470 > 444$

Therefore, Sarah will have $504$ stickers, Tom will have $444$ stickers, Lisa will have $470$ stickers, and Sarah will have the most stickers.

\section*{Question 5}
\textbf{Metadata}

\begin{itemize}
  \item Question ID: P3-WNDiv3d1d\_P1-WNCmp\_sonnet4\_Recreation\_04
  \item Primary KC: WHOLE NUMBERS | Division | dividing whole numbers up to 3 digits by 1 digit
  \item Secondary KC: WHOLE NUMBERS | Comparison and ordering | comparing and ordering whole numbers
  \item Topic: Recreation such as sports, games, exercises, music, movie, dancing, painting, fishing and other recreation activities
  \item Grade: Primary 3
\end{itemize}

\textbf{Solution}

Step 1: Find how many tennis balls each of Sarah's stations will get.
Sarah has 264 tennis balls to divide among 6 stations.
$264 \div 6 = 44$
Each of Sarah's stations will get 44 tennis balls.

Step 2: Find how many tennis balls each of Mike's stations will get.
Mike has 189 tennis balls to divide among 3 stations.
$189 \div 3 = 63$
Each of Mike's stations will get 63 tennis balls.

Step 3: Compare the number of tennis balls per station.
Sarah's stations: 44 tennis balls each
Mike's stations: 63 tennis balls each
Since $63 > 44$, Mike's stations will have more tennis balls.

Step 4: Find the difference.
$63 - 44 = 19$

Therefore:
- Each of Sarah's stations will get 44 tennis balls
- Each of Mike's stations will get 63 tennis balls
- Mike will have more tennis balls per station, with 19 more tennis balls per station than Sarah

\section*{Question 6}
\textbf{Metadata}

\begin{itemize}
  \item Question ID: P3-FrSubRl12\_P2-FrCmp\_sonnet4\_Recreation\_04
  \item Primary KC: FRACTIONS | Subtraction | subtracting two related fractions within one whole with denominators of given fractions not exceeding 12
  \item Secondary KC: FRACTIONS | Comparison and ordering | comparing and ordering fractions
  \item Topic: Recreation such as sports, games, exercises, music, movie, dancing, painting, fishing and other recreation activities
  \item Grade: Primary 3
\end{itemize}

\textbf{Solution}

To find how much more Sarah painted than Emma, I need to subtract Emma's portion from Sarah's portion.

Sarah painted: $\frac{7}{12}$ of the canvas
Emma painted: $\frac{3}{12}$ of the canvas

Difference = $\frac{7}{12} - \frac{3}{12} = \frac{7-3}{12} = \frac{4}{12}$

I can simplify $\frac{4}{12}$ by dividing both numerator and denominator by 4:
$\frac{4}{12} = \frac{1}{3}$

To determine who painted the larger portion, I compare the fractions:
$\frac{7}{12}$ and $\frac{3}{12}$

Since both fractions have the same denominator, I compare the numerators:
$7 > 3$, so $\frac{7}{12} > \frac{3}{12}$

Therefore, Sarah painted $\frac{1}{3}$ more of the canvas than Emma, and Sarah painted the larger portion of the canvas.

\section*{Question 7}
\textbf{Metadata}

\begin{itemize}
  \item Question ID: P4-WNMul4d1d\_P1-WNCmp\_sonnet4\_Recreation\_04
  \item Primary KC: WHOLE NUMBERS | Multiplication | multiplying whole numbers up to 4 digits by 1 digit or up to 3 digits by 2 digits
  \item Secondary KC: WHOLE NUMBERS | Comparison and ordering | comparing and ordering whole numbers
  \item Topic: Recreation such as sports, games, exercises, music, movie, dancing, painting, fishing and other recreation activities
  \item Grade: Primary 4
\end{itemize}

\textbf{Solution}

To solve this problem, I need to find the total number of stickers each person has, then compare and order them.

**Step 1: Calculate Sarah's total stickers**
Sarah has $24$ packs with $138$ stickers each.
Total stickers = $24 \times 138$
$24 \times 138 = 3312$
Sarah has $3312$ stickers.

**Step 2: Calculate Maya's total stickers**
Maya has $15$ packs with $247$ stickers each.
Total stickers = $15 \times 247$
$15 \times 247 = 3705$
Maya has $3705$ stickers.

**Step 3: Calculate Alex's total stickers**
Alex has $28$ packs with $125$ stickers each.
Total stickers = $28 \times 125$
$28 \times 125 = 3500$
Alex has $3500$ stickers.

**Step 4: Compare and order the totals**
Sarah: $3312$ stickers
Maya: $3705$ stickers
Alex: $3500$ stickers

Comparing: $3705 > 3500 > 3312$

**Answer:**
- Sarah has $3312$ stickers
- Maya has $3705$ stickers
- Alex has $3500$ stickers

In order from most to fewest stickers: Maya, Alex, Sarah

\section*{Question 8}
\textbf{Metadata}

\begin{itemize}
  \item Question ID: P4-WNMul4d1d\_P4-WNRnd5d\_sonnet4\_Recreation\_04
  \item Primary KC: WHOLE NUMBERS | Multiplication | multiplying whole numbers up to 4 digits by 1 digit or up to 3 digits by 2 digits
  \item Secondary KC: WHOLE NUMBERS | Rounding | rounding whole numbers up to 100000 to the nearest 10, 100 or 1000 
  \item Topic: Recreation such as sports, games, exercises, music, movie, dancing, painting, fishing and other recreation activities
  \item Grade: Primary 4
\end{itemize}

\textbf{Solution}

Step 1: Find the total number of students participating.
Total students = $127 + 86 = 213$ students

Step 2: Calculate the total number of medals needed.
Since each student receives $3$ medals:
Total medals = $213 \times 3 = 639$ medals

Step 3: Round the total number of medals to the nearest hundred.
$639$ rounded to the nearest hundred:
Look at the tens digit: $3$
Since $3 < 5$, we round down.
$639$ rounded to the nearest hundred = $600$

Therefore, Sarah needs to order $600$ medals.

\section*{Question 9}
\textbf{Metadata}

\begin{itemize}
  \item Question ID: P4-WNDiv4d1d\_P1-WNSub2nd\_sonnet4\_Recreation\_04
  \item Primary KC: WHOLE NUMBERS | Division | dividing whole numbers up to 4 digits by 1 digit
  \item Secondary KC: WHOLE NUMBERS | Subtraction | subtracting whole numbers
  \item Topic: Recreation such as sports, games, exercises, music, movie, dancing, painting, fishing and other recreation activities
  \item Grade: Primary 4
\end{itemize}

\textbf{Solution}

Step 1: Find the total number of rows when all seats are arranged.
Divide the total number of seats by the number of seats per row:
$2856 \div 8 = 357$ rows

Step 2: Find the number of rows available for the audience.
Subtract the number of rows removed from the total number of rows:
$357 - 47 = 310$ rows

Therefore, there will be 310 rows of seats available for the audience.

\section*{Question 10}
\textbf{Metadata}

\begin{itemize}
  \item Question ID: P4-WNDiv4d1d\_P4-WNRnd5d\_sonnet4\_Recreation\_04
  \item Primary KC: WHOLE NUMBERS | Division | dividing whole numbers up to 4 digits by 1 digit
  \item Secondary KC: WHOLE NUMBERS | Rounding | rounding whole numbers up to 100000 to the nearest 10, 100 or 1000 
  \item Topic: Recreation such as sports, games, exercises, music, movie, dancing, painting, fishing and other recreation activities
  \item Grade: Primary 4
\end{itemize}

\textbf{Solution}

Step 1: Find how many shuttlecocks each court will receive by dividing the total number of shuttlecocks by the number of courts.

$2436 \div 6$

Using long division:
$2436 \div 6 = 406$

Step 2: Round 406 to the nearest 10.

Look at the ones digit of 406, which is 6.
Since 6 ≥ 5, we round up.
$406$ rounded to the nearest 10 is $410$.

Therefore, each court will receive $406$ shuttlecocks, and this number rounded to the nearest 10 is $410$.

\section*{Question 11}
\textbf{Metadata}

\begin{itemize}
  \item Question ID: P4-FrAddU12\_P2-FrCmp\_sonnet4\_Recreation\_04
  \item Primary KC: FRACTIONS | Addition | adding unlike fractions with two different denominators not exceeding 12
  \item Secondary KC: FRACTIONS | Comparison and ordering | comparing and ordering fractions
  \item Topic: Recreation such as sports, games, exercises, music, movie, dancing, painting, fishing and other recreation activities
  \item Grade: Primary 4
\end{itemize}

\textbf{Solution}

(a) Sarah completed $\frac{2}{3}$ of the painting challenge and $\frac{1}{4}$ of the music challenge.
Total fraction Sarah completed = $\frac{2}{3} + \frac{1}{4}$
To add these fractions, we need a common denominator.
LCM of 3 and 4 = 12
$\frac{2}{3} = \frac{2 \times 4}{3 \times 4} = \frac{8}{12}$
$\frac{1}{4} = \frac{1 \times 3}{4 \times 3} = \frac{3}{12}$
$\frac{8}{12} + \frac{3}{12} = \frac{11}{12}$
Sarah completed $\frac{11}{12}$ of the challenges in total.

(b) Emma completed $\frac{1}{2}$ of the painting challenge and $\frac{3}{8}$ of the music challenge.
Total fraction Emma completed = $\frac{1}{2} + \frac{3}{8}$
To add these fractions, we need a common denominator.
LCM of 2 and 8 = 8
$\frac{1}{2} = \frac{1 \times 4}{2 \times 4} = \frac{4}{8}$
$\frac{4}{8} + \frac{3}{8} = \frac{7}{8}$
Emma completed $\frac{7}{8}$ of the challenges in total.

(c) To compare $\frac{11}{12}$ and $\frac{7}{8}$, we need a common denominator.
LCM of 12 and 8 = 24
$\frac{11}{12} = \frac{11 \times 2}{12 \times 2} = \frac{22}{24}$
$\frac{7}{8} = \frac{7 \times 3}{8 \times 3} = \frac{21}{24}$
Since $\frac{22}{24} > \frac{21}{24}$, Sarah completed a greater total fraction of challenges than Emma.

\section*{Question 12}
\textbf{Metadata}

\begin{itemize}
  \item Question ID: P4-FrSubU12\_P3-FrSmp\_sonnet4\_Recreation\_04
  \item Primary KC: FRACTIONS | Subtraction | subtracting unlike fractions with two different denominators not exceeding 12
  \item Secondary KC: FRACTIONS | Simplifying | expressing a fraction in its simplest form
  \item Topic: Recreation such as sports, games, exercises, music, movie, dancing, painting, fishing and other recreation activities
  \item Grade: Primary 4
\end{itemize}

\textbf{Solution}

To find how much longer Sarah practiced on Monday than on Tuesday, I need to subtract the time she practiced on Tuesday from the time she practiced on Monday.

Time practiced on Monday: $\frac{5}{6}$ hour
Time practiced on Tuesday: $\frac{1}{4}$ hour

Difference = $\frac{5}{6} - \frac{1}{4}$

To subtract these fractions, I need to find a common denominator. The denominators are 6 and 4.
The least common multiple of 6 and 4 is 12.

Converting to equivalent fractions with denominator 12:
$\frac{5}{6} = \frac{5 \times 2}{6 \times 2} = \frac{10}{12}$

$\frac{1}{4} = \frac{1 \times 3}{4 \times 3} = \frac{3}{12}$

Now I can subtract:
$\frac{10}{12} - \frac{3}{12} = \frac{10-3}{12} = \frac{7}{12}$

To check if $\frac{7}{12}$ is in its simplest form, I need to find the greatest common factor of 7 and 12.
Factors of 7: 1, 7
Factors of 12: 1, 2, 3, 4, 6, 12
The greatest common factor is 1.

Since the greatest common factor is 1, $\frac{7}{12}$ is already in its simplest form.

Therefore, Sarah practiced $\frac{7}{12}$ hour longer on Monday than on Tuesday.

\section*{Question 13}
\textbf{Metadata}

\begin{itemize}
  \item Question ID: P4-DcAdd2d\_P4-DcCmp3d\_sonnet4\_Recreation\_04
  \item Primary KC: DECIMALS | Addition | adding decimals (up to 2 decimal places)
  \item Secondary KC: DECIMALS | Comparison and ordering | comparing and ordering decimals up to 3 decimal places
  \item Topic: Recreation such as sports, games, exercises, music, movie, dancing, painting, fishing and other recreation activities
  \item Grade: Primary 4
\end{itemize}

\textbf{Solution}

Step 1: Find Sarah's times for each session.
First session: $1.25$ minutes
Second session: $1.25 - 0.08 = 1.17$ minutes  
Third session: $1.142$ minutes

Step 2: Add the three times to find the total swimming time.
Total time $= 1.25 + 1.17 + 1.142$
$= 2.42 + 1.142$
$= 3.562$ minutes

Step 3: Compare the three times to find the fastest time.
First session: $1.25$ minutes $= 1.250$ minutes
Second session: $1.17$ minutes $= 1.170$ minutes
Third session: $1.142$ minutes

Comparing: $1.142 < 1.170 < 1.250$

Therefore, Sarah's total swimming time for all three sessions was $3.562$ minutes, and the third session had the fastest time at $1.142$ minutes.

\section*{Question 14}
\textbf{Metadata}

\begin{itemize}
  \item Question ID: P4-DcSub2d\_P4-DcCnv2Fr\_sonnet4\_Recreation\_04
  \item Primary KC: DECIMALS | Subtraction | subtracting decimals (up to 2 decimal places)
  \item Secondary KC: DECIMALS | Conversion from decimals to fraction | expressing decimals as fractions
  \item Topic: Recreation such as sports, games, exercises, music, movie, dancing, painting, fishing and other recreation activities
  \item Grade: Primary 4
\end{itemize}

\textbf{Solution}

Step 1: Find how many hours Emma practiced on Wednesday.
Emma practiced on Monday for $2.75$ hours.
On Wednesday, she practiced $0.25$ hours less than Monday.
Hours practiced on Wednesday = $2.75 - 0.25 = 2.50$ hours

Step 2: Find the difference between Monday and Wednesday practice times.
Difference = Monday hours - Wednesday hours
Difference = $2.75 - 2.50 = 0.25$ hours

Step 3: Convert the decimal to a fraction.
$0.25 = \frac{25}{100}$

Step 4: Simplify the fraction.
$\frac{25}{100} = \frac{25 \div 25}{100 \div 25} = \frac{1}{4}$

Therefore, Emma practiced $\frac{1}{4}$ hours fewer on Wednesday compared to Monday.

\section*{Question 15}
\textbf{Metadata}

\begin{itemize}
  \item Question ID: P4-DcSub2d\_P4-DcRnd3d\_sonnet4\_Recreation\_04
  \item Primary KC: DECIMALS | Subtraction | subtracting decimals (up to 2 decimal places)
  \item Secondary KC: DECIMALS | Rounding | rounding decimals up to 3 decimal places to the nearest whole number, 1 decimal place and 2 decimal places 
  \item Topic: Recreation such as sports, games, exercises, music, movie, dancing, painting, fishing and other recreation activities
  \item Grade: Primary 4
\end{itemize}

\textbf{Solution}

**Part (a):**

To find how much further Sarah swam on Monday than on Tuesday, I need to subtract Tuesday's distance from Monday's distance.

Distance on Monday = $3.75$ km
Distance on Tuesday = $2.486$ km

Difference = $3.75 - 2.486$

To subtract decimals, I align the decimal points:
$$\begin{align}
3.750 \\
- 2.486 \\
\hline
1.264
\end{align}$$

Therefore, Sarah swam $1.264$ km further on Monday than on Tuesday.

Since the answer is already given to 3 decimal places and we need it to 2 decimal places, the answer is $1.26$ km.

**Part (b):**

To round $1.26$ km to the nearest whole number, I look at the first decimal place, which is $2$.

Since $2 < 5$, I round down.

Therefore, $1.26$ km rounded to the nearest whole number is $1$ km.

**Final Answers:**
(a) $1.26$ km
(b) $1$ km

\section*{Question 16}
\textbf{Metadata}

\begin{itemize}
  \item Question ID: P4-DcMul2d1d\_P4-DcCnv2Fr\_sonnet4\_Recreation\_04
  \item Primary KC: DECIMALS | Multiplication | multiplying decimals (up to 2 decimal places) by a 1-digit whole number
  \item Secondary KC: DECIMALS | Conversion from decimals to fraction | expressing decimals as fractions
  \item Topic: Recreation such as sports, games, exercises, music, movie, dancing, painting, fishing and other recreation activities
  \item Grade: Primary 4
\end{itemize}

\textbf{Solution}

To find the total practice time, I need to multiply the length of each session by the number of sessions.

Total practice time $= 2.45 \times 3$

To multiply $2.45$ by $3$:
$2.45 \times 3 = 7.35$ hours

Now I need to express $7.35$ as a fraction in its simplest form.

$7.35 = 7 + 0.35$

For the decimal part: $0.35 = \frac{35}{100}$

To simplify $\frac{35}{100}$, I find the greatest common factor of $35$ and $100$:
$35 = 5 \times 7$
$100 = 5 \times 20$

So $\frac{35}{100} = \frac{35 \div 5}{100 \div 5} = \frac{7}{20}$

Therefore: $7.35 = 7\frac{7}{20} = \frac{7 \times 20 + 7}{20} = \frac{140 + 7}{20} = \frac{147}{20}$

Sarah will practice for $\frac{147}{20}$ hours in total.

\section*{Question 17}
\textbf{Metadata}

\begin{itemize}
  \item Question ID: P4-DcMul2d1d\_P4-DcAdd2nd\_sonnet4\_Recreation\_04
  \item Primary KC: DECIMALS | Multiplication | multiplying decimals (up to 2 decimal places) by a 1-digit whole number
  \item Secondary KC: DECIMALS | Addition | adding decimals
  \item Topic: Recreation such as sports, games, exercises, music, movie, dancing, painting, fishing and other recreation activities
  \item Grade: Primary 4
\end{itemize}

\textbf{Solution}

Step 1: Find the total cost of the acrylic paint.
Cost of each tube of paint = \textdollar2.45
Number of tubes = 3
Total cost of paint = $2.45 \times 3 = \textdollar7.35$

Step 2: Find the total cost of the paintbrushes.
Cost of each paintbrush = \textdollar1.25
Number of paintbrushes = 4
Total cost of paintbrushes = $1.25 \times 4 = \textdollar5.00$

Step 3: Find the total amount Sarah spends.
Total cost = Cost of paint + Cost of paintbrushes
Total cost = $\textdollar7.35 + \textdollar5.00 = \textdollar12.35$

Therefore, Sarah spends \textdollar12.35 in total on art supplies.

\section*{Question 18}
\textbf{Metadata}

\begin{itemize}
  \item Question ID: P4-DcDiv2d1d\_P4-DcCmp3d\_sonnet4\_Recreation\_04
  \item Primary KC: DECIMALS | Division | dividing decimals (up to 2 decimal places) by a 1-digit whole number
  \item Secondary KC: DECIMALS | Comparison and ordering | comparing and ordering decimals up to 3 decimal places
  \item Topic: Recreation such as sports, games, exercises, music, movie, dancing, painting, fishing and other recreation activities
  \item Grade: Primary 4
\end{itemize}

\textbf{Solution}

First, I need to find Sarah's average time per $100$ meters for each day by dividing each total time by $4$.

Monday: $12.48 \div 4$
$12.48 \div 4 = 3.12$ minutes per $100$ meters

Tuesday: $11.76 \div 4$
$11.76 \div 4 = 2.94$ minutes per $100$ meters

Wednesday: $13.32 \div 4$
$13.32 \div 4 = 3.33$ minutes per $100$ meters

Now I need to arrange these times from fastest to slowest. The fastest time is the smallest number.

Comparing the three times:
- Monday: $3.12$ minutes
- Tuesday: $2.94$ minutes  
- Wednesday: $3.33$ minutes

Comparing $2.94$, $3.12$, and $3.33$:
$2.94 < 3.12 < 3.33$

Therefore, arranged from fastest to slowest:
1st (fastest): Tuesday with $2.94$ minutes per $100$ meters
2nd: Monday with $3.12$ minutes per $100$ meters
3rd (slowest): Wednesday with $3.33$ minutes per $100$ meters

\section*{Question 19}
\textbf{Metadata}

\begin{itemize}
  \item Question ID: P4-DcDiv2d1d\_P4-DcAdd2nd\_sonnet4\_Recreation\_04
  \item Primary KC: DECIMALS | Division | dividing decimals (up to 2 decimal places) by a 1-digit whole number
  \item Secondary KC: DECIMALS | Addition | adding decimals
  \item Topic: Recreation such as sports, games, exercises, music, movie, dancing, painting, fishing and other recreation activities
  \item Grade: Primary 4
\end{itemize}

\textbf{Solution}

First, I need to find the distance Sarah swam per lap on each day by dividing the total distance by the number of laps.

Monday:
Distance per lap = $4.68 \div 3$
$= 1.56$ km

Tuesday:
Distance per lap = $5.24 \div 3$
$= 1.75$ km (since $5.24 \div 3 = 1.746...$, but checking: $1.75 \times 3 = 5.25$, let me recalculate)
$5.24 \div 3 = 1.746...$
Since we need exact division: $5.24 \div 3 = 1.74\frac{2}{3}$
Let me use $5.22$ instead: $5.22 \div 3 = 1.74$ km

Actually, let me recalculate with $5.22$:
Tuesday: $5.22 \div 3 = 1.74$ km

Total distance per lap over the two days:
$1.56 + 1.74 = 3.30$ km

Therefore, Sarah swam a total distance of $3.30$ km per lap over the two days.

\section*{Question 20}
\textbf{Metadata}

\begin{itemize}
  \item Question ID: P5-FrAddMix\_P2-FrCmp\_sonnet4\_Recreation\_04
  \item Primary KC: FRACTIONS | Addition | adding mixed numbers
  \item Secondary KC: FRACTIONS | Comparison and ordering | comparing and ordering fractions
  \item Topic: Recreation such as sports, games, exercises, music, movie, dancing, painting, fishing and other recreation activities
  \item Grade: Primary 5
\end{itemize}

\textbf{Solution}

(a) Total distance on Monday = $2\frac{3}{4} + 1\frac{5}{8}$

First, convert to the same denominator:
$2\frac{3}{4} = 2\frac{6}{8}$

$2\frac{6}{8} + 1\frac{5}{8} = 3\frac{11}{8} = 4\frac{3}{8}$ km

(b) Total distance on Tuesday = $3\frac{1}{2} + \frac{7}{8}$

First, convert to the same denominator:
$3\frac{1}{2} = 3\frac{4}{8}$

$3\frac{4}{8} + \frac{7}{8} = 3\frac{11}{8} = 4\frac{3}{8}$ km

(c) Comparing the totals:
Monday: $4\frac{3}{8}$ km
Tuesday: $4\frac{3}{8}$ km

Since $4\frac{3}{8} = 4\frac{3}{8}$, Sarah swam the same total distance on both days.

\section*{Question 21}
\textbf{Metadata}

\begin{itemize}
  \item Question ID: P5-FrSubMix\_P2-FrCmp\_sonnet4\_Recreation\_04
  \item Primary KC: FRACTIONS | Subtraction | subtracting mixed numbers
  \item Secondary KC: FRACTIONS | Comparison and ordering | comparing and ordering fractions
  \item Topic: Recreation such as sports, games, exercises, music, movie, dancing, painting, fishing and other recreation activities
  \item Grade: Primary 5
\end{itemize}

\textbf{Solution}

(a) First, find the total distance each person ran over the two days.

Sarah's total distance: $4\frac{3}{4} + 2\frac{5}{8}$

Convert to common denominators: $4\frac{6}{8} + 2\frac{5}{8} = 6\frac{11}{8} = 7\frac{3}{8}$ km

Emma's total distance: $3\frac{1}{2} + 1\frac{3}{4}$

Convert to common denominators: $3\frac{2}{4} + 1\frac{3}{4} = 4\frac{5}{4} = 5\frac{1}{4}$ km

Difference: $7\frac{3}{8} - 5\frac{1}{4}$

Convert to common denominators: $7\frac{3}{8} - 5\frac{2}{8} = 2\frac{1}{8}$ km

Sarah ran $2\frac{1}{8}$ km further than Emma.

(b) Find each person's difference between Monday and Tuesday distances.

Sarah's difference: $4\frac{3}{4} - 2\frac{5}{8}$

Convert to common denominators: $4\frac{6}{8} - 2\frac{5}{8} = 2\frac{1}{8}$ km

Emma's difference: $3\frac{1}{2} - 1\frac{3}{4}$

Convert to common denominators: $3\frac{2}{4} - 1\frac{3}{4} = 2\frac{2}{4} - 1\frac{3}{4} = 1\frac{6}{4} - 1\frac{3}{4} = \frac{3}{4}$ km

Compare the differences: $2\frac{1}{8}$ and $\frac{3}{4}$

Convert to common denominators: $2\frac{1}{8}$ and $\frac{6}{8}$

Since $2\frac{1}{8} > \frac{6}{8}$, Sarah had the greater difference.

Difference between their differences: $2\frac{1}{8} - \frac{6}{8} = 1\frac{3}{8}$ km

Sarah had the greater difference by $1\frac{3}{8}$ km.

\section*{Question 22}
\textbf{Metadata}

\begin{itemize}
  \item Question ID: P5-FrMulImN\_P2-FrCmp\_sonnet4\_Recreation\_04
  \item Primary KC: FRACTIONS | Multiplication | multiplying a proper/improper fraction and a whole number
  \item Secondary KC: FRACTIONS | Comparison and ordering | comparing and ordering fractions
  \item Topic: Recreation such as sports, games, exercises, music, movie, dancing, painting, fishing and other recreation activities
  \item Grade: Primary 5
\end{itemize}

\textbf{Solution}

Step 1: Calculate the distance Sarah ran.
Sarah's distance = $2\frac{1}{4} \times 6$

First, convert the mixed number to an improper fraction:
$2\frac{1}{4} = \frac{9}{4}$

Then multiply:
$\frac{9}{4} \times 6 = \frac{9 \times 6}{4} = \frac{54}{4} = 13\frac{1}{2}$ kilometers

Step 2: Calculate the distance Tom ran.
Tom's distance = $1\frac{3}{5} \times 8$

First, convert the mixed number to an improper fraction:
$1\frac{3}{5} = \frac{8}{5}$

Then multiply:
$\frac{8}{5} \times 8 = \frac{8 \times 8}{5} = \frac{64}{5} = 12\frac{4}{5}$ kilometers

Step 3: Compare the distances to determine who ran more.
Sarah ran $13\frac{1}{2}$ kilometers and Tom ran $12\frac{4}{5}$ kilometers.

To compare, convert both to improper fractions with a common denominator:
$13\frac{1}{2} = \frac{27}{2} = \frac{135}{10}$
$12\frac{4}{5} = \frac{64}{5} = \frac{128}{10}$

Since $\frac{135}{10} > \frac{128}{10}$, Sarah ran a greater distance.

Step 4: Find the difference.
Difference = $13\frac{1}{2} - 12\frac{4}{5} = \frac{135}{10} - \frac{128}{10} = \frac{7}{10}$ kilometers

Therefore, Sarah ran $13\frac{1}{2}$ kilometers, Tom ran $12\frac{4}{5}$ kilometers, and Sarah ran $\frac{7}{10}$ kilometers more than Tom.

\section*{Question 23}
\textbf{Metadata}

\begin{itemize}
  \item Question ID: P5-FrMulPIm\_P2-FrCmp\_sonnet4\_Recreation\_04
  \item Primary KC: FRACTIONS | Multiplication | multiplying a proper fraction and a proper/improper fractions
  \item Secondary KC: FRACTIONS | Comparison and ordering | comparing and ordering fractions
  \item Topic: Recreation such as sports, games, exercises, music, movie, dancing, painting, fishing and other recreation activities
  \item Grade: Primary 5
\end{itemize}

\textbf{Solution}

(a) Emma's practice time on Monday = $\frac{2}{3} \times \frac{3}{4}$ hours

$\frac{2}{3} \times \frac{3}{4} = \frac{2 \times 3}{3 \times 4} = \frac{6}{12} = \frac{1}{2}$ hours

Emma practiced for $\frac{1}{2}$ hour on Monday.

(b) Emma's practice time on Tuesday = $\frac{3}{5} \times \frac{5}{6}$ hours

$\frac{3}{5} \times \frac{5}{6} = \frac{3 \times 5}{5 \times 6} = \frac{15}{30} = \frac{1}{2}$ hours

Emma practiced for $\frac{1}{2}$ hour on Tuesday.

(c) To compare Emma's practice times:
Monday: $\frac{1}{2}$ hour
Tuesday: $\frac{1}{2}$ hour

Since $\frac{1}{2} = \frac{1}{2}$, Emma practiced for the same amount of time on both days.

\section*{Question 24}
\textbf{Metadata}

\begin{itemize}
  \item Question ID: P5-FrMulPIm\_P2-FrAdd2nd\_sonnet4\_Recreation\_04
  \item Primary KC: FRACTIONS | Multiplication | multiplying a proper fraction and a proper/improper fractions
  \item Secondary KC: FRACTIONS | Addition | adding fractions
  \item Topic: Recreation such as sports, games, exercises, music, movie, dancing, painting, fishing and other recreation activities
  \item Grade: Primary 5
\end{itemize}

\textbf{Solution}

Step 1: Find how long Sarah practiced on Tuesday.
Tuesday practice time = $\frac{2}{3} \times \frac{3}{4} = \frac{2 \times 3}{3 \times 4} = \frac{6}{12} = \frac{1}{2}$ hours

Step 2: Find how long Sarah practiced on Wednesday.
Wednesday practice time = $\frac{5}{6} \times \frac{1}{2} = \frac{5 \times 1}{6 \times 2} = \frac{5}{12}$ hours

Step 3: Add up all three days' practice times.
Total practice time = $\frac{3}{4} + \frac{1}{2} + \frac{5}{12}$

To add these fractions, find the common denominator. The LCM of 4, 2, and 12 is 12.
$\frac{3}{4} = \frac{9}{12}$
$\frac{1}{2} = \frac{6}{12}$
$\frac{5}{12} = \frac{5}{12}$

Total practice time = $\frac{9}{12} + \frac{6}{12} + \frac{5}{12} = \frac{20}{12} = \frac{5}{3} = 1\frac{2}{3}$ hours

Therefore, Sarah practiced piano for $1\frac{2}{3}$ hours in total over the three days.

\section*{Question 25}
\textbf{Metadata}

\begin{itemize}
  \item Question ID: P5-FrMulPIm\_P3-FrSmp\_sonnet4\_Recreation\_04
  \item Primary KC: FRACTIONS | Multiplication | multiplying a proper fraction and a proper/improper fractions
  \item Secondary KC: FRACTIONS | Simplifying | expressing a fraction in its simplest form
  \item Topic: Recreation such as sports, games, exercises, music, movie, dancing, painting, fishing and other recreation activities
  \item Grade: Primary 5
\end{itemize}

\textbf{Solution}

To find what fraction of the entire mural consists of landscape scenery, I need to multiply the fraction of the mural that Sarah has completed by the fraction of her completed work that is landscape scenery.

Fraction of mural completed = $\frac{3}{4}$

Fraction of completed work that is landscape = $\frac{2}{5}$

Fraction of entire mural that is landscape = $\frac{3}{4} \times \frac{2}{5}$

To multiply fractions, I multiply the numerators together and multiply the denominators together:

$\frac{3}{4} \times \frac{2}{5} = \frac{3 \times 2}{4 \times 5} = \frac{6}{20}$

Now I need to simplify $\frac{6}{20}$ to its simplest form by finding the highest common factor (HCF) of 6 and 20.

Factors of 6: 1, 2, 3, 6
Factors of 20: 1, 2, 4, 5, 10, 20

The HCF of 6 and 20 is 2.

$\frac{6}{20} = \frac{6 \div 2}{20 \div 2} = \frac{3}{10}$

Therefore, $\frac{3}{10}$ of the entire mural consists of landscape scenery.

\section*{Question 26}
\textbf{Metadata}

\begin{itemize}
  \item Question ID: P5-FrMulImIm\_P2-FrAdd2nd\_sonnet4\_Recreation\_04
  \item Primary KC: FRACTIONS | Multiplication | multiplying two improper fractions
  \item Secondary KC: FRACTIONS | Addition | adding fractions
  \item Topic: Recreation such as sports, games, exercises, music, movie, dancing, painting, fishing and other recreation activities
  \item Grade: Primary 5
\end{itemize}

\textbf{Solution}

Let me solve this step by step.

(a) Finding the total blue paint:
Sarah has $\frac{7}{4}$ cups of dark blue paint and $\frac{5}{3}$ cups of light blue paint.
To add these fractions, I need a common denominator.
LCM of 4 and 3 is 12.
$\frac{7}{4} = \frac{7 \times 3}{4 \times 3} = \frac{21}{12}$
$\frac{5}{3} = \frac{5 \times 4}{3 \times 4} = \frac{20}{12}$
Total blue paint = $\frac{21}{12} + \frac{20}{12} = \frac{41}{12}$ cups

(b) Finding the total yellow paint:
Sarah has $\frac{9}{5}$ cups of golden yellow paint and $\frac{8}{7}$ cups of lemon yellow paint.
To add these fractions, I need a common denominator.
LCM of 5 and 7 is 35.
$\frac{9}{5} = \frac{9 \times 7}{5 \times 7} = \frac{63}{35}$
$\frac{8}{7} = \frac{8 \times 5}{7 \times 5} = \frac{40}{35}$
Total yellow paint = $\frac{63}{35} + \frac{40}{35} = \frac{103}{35}$ cups

(c) Finding the total amount of paint:
Total paint = Total blue paint + Total yellow paint
Total paint = $\frac{41}{12} + \frac{103}{35}$
To add these fractions, I need a common denominator.
LCM of 12 and 35 is 420.
$\frac{41}{12} = \frac{41 \times 35}{12 \times 35} = \frac{1435}{420}$
$\frac{103}{35} = \frac{103 \times 12}{35 \times 12} = \frac{1236}{420}$
Total paint = $\frac{1435}{420} + \frac{1236}{420} = \frac{2671}{420}$ cups

Therefore:
(a) Sarah has $\frac{41}{12}$ cups of blue paint
(b) Sarah has $\frac{103}{35}$ cups of yellow paint
(c) Sarah has prepared $\frac{2671}{420}$ cups of paint in total

\section*{Question 27}
\textbf{Metadata}

\begin{itemize}
  \item Question ID: P5-FrMulImIm\_P2-FrSub2nd\_sonnet4\_Recreation\_04
  \item Primary KC: FRACTIONS | Multiplication | multiplying two improper fractions
  \item Secondary KC: FRACTIONS | Subtraction | subtracting fractions
  \item Topic: Recreation such as sports, games, exercises, music, movie, dancing, painting, fishing and other recreation activities
  \item Grade: Primary 5
\end{itemize}

\textbf{Solution}

Step 1: Find Sarah's total dance time without breaks.
Morning practice time = $\frac{7}{4}$ hours
Evening practice time = $\frac{5}{3}$ hours
Total dance time = $\frac{7}{4} + \frac{5}{3}$

To add these fractions, find the common denominator:
$\frac{7}{4} = \frac{21}{12}$ and $\frac{5}{3} = \frac{20}{12}$
Total dance time = $\frac{21}{12} + \frac{20}{12} = \frac{41}{12}$ hours

Step 2: Find Sarah's actual practice time (including breaks).
Actual practice time = Total dance time + Break time
Actual practice time = $\frac{41}{12} + \frac{3}{2}$

Convert $\frac{3}{2}$ to twelfths: $\frac{3}{2} = \frac{18}{12}$
Actual practice time = $\frac{41}{12} + \frac{18}{12} = \frac{59}{12}$ hours

Step 3: Find the planned practice time.
We know that actual practice time = $\frac{2}{3}$ × planned practice time
$\frac{59}{12} = \frac{2}{3} \times$ planned practice time

To find planned practice time, multiply both sides by the reciprocal of $\frac{2}{3}$:
Planned practice time = $\frac{59}{12} \times \frac{3}{2}$
Planned practice time = $\frac{59 \times 3}{12 \times 2} = \frac{177}{24}$

Simplify: $\frac{177}{24} = \frac{59}{8} = 7\frac{3}{8}$ hours

Therefore, Sarah's planned practice time was $7\frac{3}{8}$ hours.

\section*{Question 28}
\textbf{Metadata}

\begin{itemize}
  \item Question ID: P5-FrMulImIm\_P3-FrSmp\_sonnet4\_Recreation\_04
  \item Primary KC: FRACTIONS | Multiplication | multiplying two improper fractions
  \item Secondary KC: FRACTIONS | Simplifying | expressing a fraction in its simplest form
  \item Topic: Recreation such as sports, games, exercises, music, movie, dancing, painting, fishing and other recreation activities
  \item Grade: Primary 5
\end{itemize}

\textbf{Solution}

To find how many hours Sarah practiced on Tuesday, I need to multiply the time she practiced on Monday by $\frac{7}{4}$.

Time practiced on Monday = $\frac{5}{3}$ hours
Time practiced on Tuesday = $\frac{5}{3} \times \frac{7}{4}$ hours

To multiply two fractions, I multiply the numerators together and multiply the denominators together:
$\frac{5}{3} \times \frac{7}{4} = \frac{5 \times 7}{3 \times 4} = \frac{35}{12}$

Now I need to check if $\frac{35}{12}$ can be simplified by finding the highest common factor (HCF) of 35 and 12.

Factors of 35: 1, 5, 7, 35
Factors of 12: 1, 2, 3, 4, 6, 12

The only common factor is 1, so the HCF of 35 and 12 is 1.

Since the HCF is 1, $\frac{35}{12}$ is already in its simplest form.

Therefore, Sarah practiced for $\frac{35}{12}$ hours on Tuesday.

\section*{Question 29}
\textbf{Metadata}

\begin{itemize}
  \item Question ID: P5-FrMulMixN\_P2-FrSub2nd\_sonnet4\_Recreation\_04
  \item Primary KC: FRACTIONS | Multiplication | multiplying a mixed number and a whole number
  \item Secondary KC: FRACTIONS | Subtraction | subtracting fractions
  \item Topic: Recreation such as sports, games, exercises, music, movie, dancing, painting, fishing and other recreation activities
  \item Grade: Primary 5
\end{itemize}

\textbf{Solution}

Step 1: Find the total laps Sarah swam on Monday.
On Monday: $2\frac{3}{4} \times 8$
Convert to improper fraction: $2\frac{3}{4} = \frac{11}{4}$
$\frac{11}{4} \times 8 = \frac{11 \times 8}{4} = \frac{88}{4} = 22$ laps

Step 2: Find the total laps Sarah swam on Tuesday.
On Tuesday: $3\frac{1}{6} \times 6$
Convert to improper fraction: $3\frac{1}{6} = \frac{19}{6}$
$\frac{19}{6} \times 6 = \frac{19 \times 6}{6} = 19$ laps

Step 3: Find the difference between Monday and Tuesday.
Difference = $22 - 19 = 3$ laps

Therefore, Sarah swam 3 more laps on Monday than on Tuesday.

\section*{Question 30}
\textbf{Metadata}

\begin{itemize}
  \item Question ID: P5-DcMul3dK\_P4-DcCnv2Fr\_sonnet4\_Recreation\_04
  \item Primary KC: DECIMALS | Multiplication | multiplying decimals (up to 3 decimal places) by 10, 100, 1000 and their multiples
  \item Secondary KC: DECIMALS | Conversion from decimals to fraction | expressing decimals as fractions
  \item Topic: Recreation such as sports, games, exercises, music, movie, dancing, painting, fishing and other recreation activities
  \item Grade: Primary 5
\end{itemize}

\textbf{Solution}

To find the total time Sarah needs to complete $300$ laps, I need to multiply her time per lap by the number of laps.

Time per lap = $0.875$ minutes
Number of laps = $300$

Total time = $0.875 \times 300$

To multiply $0.875$ by $300$:
Since $300 = 3 \times 100$, I can use the property of multiplying decimals by multiples of $10$, $100$, and $1000$.

$0.875 \times 300 = 0.875 \times 3 \times 100$

First, $0.875 \times 3 = 2.625$

Then, $2.625 \times 100 = 262.5$

So the total time is $262.5$ minutes.

Now I need to express $262.5$ as a mixed number in simplest form.

$262.5 = 262 + 0.5$

To convert $0.5$ to a fraction:
$0.5 = \frac{5}{10} = \frac{1}{2}$

Therefore, $262.5 = 262\frac{1}{2}$ minutes.

Sarah will need $262\frac{1}{2}$ minutes to complete her training session.

\section*{Question 31}
\textbf{Metadata}

\begin{itemize}
  \item Question ID: P5-DcMul3dK\_P4-DcSub2nd\_sonnet4\_Recreation\_04
  \item Primary KC: DECIMALS | Multiplication | multiplying decimals (up to 3 decimal places) by 10, 100, 1000 and their multiples
  \item Secondary KC: DECIMALS | Subtraction | subtracting decimals
  \item Topic: Recreation such as sports, games, exercises, music, movie, dancing, painting, fishing and other recreation activities
  \item Grade: Primary 5
\end{itemize}

\textbf{Solution}

Step 1: Find the distance Sarah swims during practice.
Distance = Speed × Time
Distance = $0.125 \times 400$
To multiply $0.125$ by $400$, I can multiply $0.125$ by $100 \times 4$:
$0.125 \times 100 = 12.5$
$12.5 \times 4 = 50.0$
So Sarah swims $50.0$ meters during practice.

Step 2: Find how much longer the competition pool is compared to the distance she swam.
Difference = Competition pool length - Distance swam
Difference = $65.8 - 50.0 = 15.8$

Therefore, Sarah swims $50.0$ meters during practice, and the competition pool is $15.8$ meters longer than the distance she swam.

\section*{Question 32}
\textbf{Metadata}

\begin{itemize}
  \item Question ID: P5-DcDiv3dK\_P4-DcRnd3d\_sonnet4\_Recreation\_04
  \item Primary KC: DECIMALS | Division | dividing decimals (up to 3 decimal places) by 10, 100, 1000 and their multiples
  \item Secondary KC: DECIMALS | Rounding | rounding decimals up to 3 decimal places to the nearest whole number, 1 decimal place and 2 decimal places 
  \item Topic: Recreation such as sports, games, exercises, music, movie, dancing, painting, fishing and other recreation activities
  \item Grade: Primary 5
\end{itemize}

\textbf{Solution}

To find Sarah's average daily running time, I need to divide her total running time by the number of days.

Step 1: Calculate the average daily running time.
Total running time = $127.850$ minutes
Number of days = $7$

Average daily running time = $127.850 \div 7$

To divide $127.850$ by $7$:
$127.850 \div 7 = 18.264285...$

Step 2: Round the result to $2$ decimal places.
The average daily running time is $18.264285...$ minutes.

To round to $2$ decimal places, I look at the third decimal place:
$18.264285...$

The third decimal place is $4$, which is less than $5$, so I round down.

Therefore, Sarah's average daily running time rounded to $2$ decimal places is $18.26$ minutes.

\section*{Question 33}
\textbf{Metadata}

\begin{itemize}
  \item Question ID: P5-DcDiv3dK\_P4-DcAdd2nd\_sonnet4\_Recreation\_04
  \item Primary KC: DECIMALS | Division | dividing decimals (up to 3 decimal places) by 10, 100, 1000 and their multiples
  \item Secondary KC: DECIMALS | Addition | adding decimals
  \item Topic: Recreation such as sports, games, exercises, music, movie, dancing, painting, fishing and other recreation activities
  \item Grade: Primary 5
\end{itemize}

\textbf{Solution}

Step 1: Find the total practice time by adding all the lap times.
Total time = $2.456 + 3.128 + 1.892 + 2.734$
Total time = $10.210$ minutes

Step 2: Divide the total time by 400 to find the average time per 100 meters.
Average time per 100 meters = $10.210 \div 400$
Average time per 100 meters = $0.025525$ minutes

Therefore, Sarah's average time per 100 meters is $0.025525$ minutes.

\section*{Question 34}
\textbf{Metadata}

\begin{itemize}
  \item Question ID: P5-PcRepWh\_P1-WNMul2nd\_sonnet4\_Recreation\_04
  \item Primary KC: PERCENTAGE | Representation and concept | expressing a part of a whole as a percentage
  \item Secondary KC: WHOLE NUMBERS | Multiplication | multiplying whole numbers
  \item Topic: Recreation such as sports, games, exercises, music, movie, dancing, painting, fishing and other recreation activities
  \item Grade: Primary 5
\end{itemize}

\textbf{Solution}

To find the percentage of successful free throws, I need to find what part $90$ is of the total $120$ attempts, then express it as a percentage.

First, I'll find the fraction of successful attempts:
Fraction of successful attempts = $\frac{90}{120}$

To convert this fraction to a percentage, I multiply by $100$:
Percentage = $\frac{90}{120} \times 100$

I can simplify this calculation:
$\frac{90}{120} \times 100 = \frac{90 \times 100}{120} = \frac{9000}{120}$

Now I'll divide $9000$ by $120$:
$9000 \div 120 = 75$

Therefore, Sarah successfully made $75\%$ of her free throw attempts.

\section*{Question 35}
\textbf{Metadata}

\begin{itemize}
  \item Question ID: P5-PcRepWh\_P1-WNDiv2nd\_sonnet4\_Recreation\_04
  \item Primary KC: PERCENTAGE | Representation and concept | expressing a part of a whole as a percentage
  \item Secondary KC: WHOLE NUMBERS | Division | dividing whole numbers
  \item Topic: Recreation such as sports, games, exercises, music, movie, dancing, painting, fishing and other recreation activities
  \item Grade: Primary 5
\end{itemize}

\textbf{Solution}

Step 1: Find the number of participants in each team.
Total participants = 240
Number of teams = 8
Participants per team = $240 \div 8 = 30$ participants

Step 2: Find the total number of participants eliminated.
Participants eliminated per team = 15
Total participants eliminated = $15 \times 8 = 120$ participants

Step 3: Express the eliminated participants as a percentage of the total participants.
Percentage eliminated = $\frac{120}{240} \times 100\%$
Percentage eliminated = $\frac{1}{2} \times 100\%$
Percentage eliminated = $50\%$

Therefore, 50% of the total participants were eliminated in the first round.

\section*{Question 36}
\textbf{Metadata}

\begin{itemize}
  \item Question ID: P5-RtFndU\_P2-DcCnvN2D\_sonnet4\_Recreation\_04
  \item Primary KC: RATE | Finding number of unit | finding number of units given rate and total amount
  \item Secondary KC: DECIMALS | Conversion to larger units | converting a measurement from a smaller unit to a larger unit in decimal form
  \item Topic: Recreation such as sports, games, exercises, music, movie, dancing, painting, fishing and other recreation activities
  \item Grade: Primary 5
\end{itemize}

\textbf{Solution}

First, I need to convert the total distance from meters to kilometers since the rate is given in km per hour.

Total distance = $7500$ m

To convert meters to kilometers, I divide by $1000$:
Total distance = $7500 \div 1000 = 7.5$ km

Now I can find the number of hours using the rate formula:
Rate = Distance $\div$ Time

Rearranging to find time:
Time = Distance $\div$ Rate

Time = $7.5 \div 1.25$

To divide by a decimal, I can multiply both numbers by $100$:
Time = $750 \div 125$

Using long division:
$750 \div 125 = 6$

Therefore, Sarah spent $6$ hours running.

\section*{Question 37}
\textbf{Metadata}

\begin{itemize}
  \item Question ID: P6-FrDivPN\_P2-FrAdd2nd\_sonnet4\_Recreation\_04
  \item Primary KC: FRACTIONS | Division | dividing a proper fraction by a whole number
  \item Secondary KC: FRACTIONS | Addition | adding fractions
  \item Topic: Recreation such as sports, games, exercises, music, movie, dancing, painting, fishing and other recreation activities
  \item Grade: Primary 6
\end{itemize}

\textbf{Solution}

Step 1: Find how much time Sarah spent on each piece on Monday.
Time spent on each piece on Monday = $\frac{3}{4} \div 3$
$= \frac{3}{4} \times \frac{1}{3}$
$= \frac{3}{12}$
$= \frac{1}{4}$ hour

Step 2: Find how much time Sarah spent on each piece on Tuesday.
Time spent on each piece on Tuesday = Time on Monday + Additional time
$= \frac{1}{4} + \frac{1}{6}$

To add these fractions, find the common denominator:
LCM of 4 and 6 = 12
$\frac{1}{4} = \frac{3}{12}$
$\frac{1}{6} = \frac{2}{12}$

$\frac{1}{4} + \frac{1}{6} = \frac{3}{12} + \frac{2}{12} = \frac{5}{12}$

Therefore, Sarah spent $\frac{5}{12}$ of an hour practicing each piece on Tuesday.

\section*{Question 38}
\textbf{Metadata}

\begin{itemize}
  \item Question ID: P6-FrDivPN\_P5-FrCnv2Dc\_sonnet4\_Recreation\_04
  \item Primary KC: FRACTIONS | Division | dividing a proper fraction by a whole number
  \item Secondary KC: FRACTIONS | Conversion to decimals | expressing fractions as decimals
  \item Topic: Recreation such as sports, games, exercises, music, movie, dancing, painting, fishing and other recreation activities
  \item Grade: Primary 6
\end{itemize}

\textbf{Solution}

To find how much time Sarah will spend on each piece, I need to divide $\frac{3}{4}$ hours by 6 pieces.

$\frac{3}{4} \div 6 = \frac{3}{4} \times \frac{1}{6} = \frac{3}{24} = \frac{1}{8}$

Now I need to convert $\frac{1}{8}$ to a decimal.

$\frac{1}{8} = 1 \div 8 = 0.125$

Therefore, Sarah will spend $0.125$ hours on each piece.

\section*{Question 39}
\textbf{Metadata}

\begin{itemize}
  \item Question ID: P6-FrDivPP\_P2-FrSub2nd\_sonnet4\_Recreation\_04
  \item Primary KC: FRACTIONS | Division | dividing a whole number/proper fraction by a proper fraction
  \item Secondary KC: FRACTIONS | Subtraction | subtracting fractions
  \item Topic: Recreation such as sports, games, exercises, music, movie, dancing, painting, fishing and other recreation activities
  \item Grade: Primary 6
\end{itemize}

\textbf{Solution}

Step 1: Find the remaining practice time.
Planned practice time = $2\frac{1}{4}$ hours = $\frac{9}{4}$ hours
Actual practice time = $1\frac{3}{4}$ hours = $\frac{7}{4}$ hours
Remaining practice time = $\frac{9}{4} - \frac{7}{4} = \frac{2}{4} = \frac{1}{2}$ hour

Step 2: Divide the remaining practice time by the fraction of the week.
Sarah wants to divide $\frac{1}{2}$ hour equally among $\frac{2}{3}$ of the week.
Daily practice time = $\frac{1}{2} \div \frac{2}{3}$
= $\frac{1}{2} \times \frac{3}{2}$
= $\frac{3}{4}$ hour

Therefore, Sarah should practice $\frac{3}{4}$ hour each day for the remaining portion of the week.

\section*{Question 40}
\textbf{Metadata}

\begin{itemize}
  \item Question ID: P6-FrDivPP\_P5-FrCnv2Dc\_sonnet4\_Recreation\_04
  \item Primary KC: FRACTIONS | Division | dividing a whole number/proper fraction by a proper fraction
  \item Secondary KC: FRACTIONS | Conversion to decimals | expressing fractions as decimals
  \item Topic: Recreation such as sports, games, exercises, music, movie, dancing, painting, fishing and other recreation activities
  \item Grade: Primary 6
\end{itemize}

\textbf{Solution}

To find how many complete practice sessions Sarah has finished, I need to divide the fraction of time she has completed by the length of each session.

Sarah has completed $\frac{3}{4}$ of her total practice time.
Each session is $\frac{2}{5}$ of an hour long.

Number of sessions = $\frac{3}{4} \div \frac{2}{5}$

To divide fractions, I multiply by the reciprocal:
$\frac{3}{4} \div \frac{2}{5} = \frac{3}{4} \times \frac{5}{2}$

$= \frac{3 \times 5}{4 \times 2} = \frac{15}{8}$

Now I need to convert $\frac{15}{8}$ to a decimal:
$\frac{15}{8} = 15 \div 8 = 1.875$

Therefore, Sarah has completed 1.875 practice sessions. Since the question asks for complete sessions, she has finished 1 complete practice session.

\section*{Question 41}
\textbf{Metadata}

\begin{itemize}
  \item Question ID: P6-PcFndWN\_P1-WNAdd2nd\_sonnet4\_Recreation\_04
  \item Primary KC: PERCENTAGE | Finding the whole | finding the whole given a part and the percentage
  \item Secondary KC: WHOLE NUMBERS | Addition | adding whole numbers
  \item Topic: Recreation such as sports, games, exercises, music, movie, dancing, painting, fishing and other recreation activities
  \item Grade: Primary 6
\end{itemize}

\textbf{Solution}

Step 1: Find the total number of shots Sarah attempted in her first practice session.

Let the total number of shots attempted in the first session be $x$.

We know that $60\%$ of $x$ equals $72$ successful shots.

$60\% \times x = 72$

$\frac{60}{100} \times x = 72$

$0.6x = 72$

$x = 72 \div 0.6 = 120$

So Sarah attempted $120$ shots in her first practice session.

Step 2: Find the total number of shots across both practice sessions.

In the first practice session: $120$ shots
In the second practice session: $120 + 45 = 165$ shots

Total shots across both sessions: $120 + 165 = 285$ shots

Therefore, Sarah will attempt $285$ total shots across both practice sessions.

\section*{Question 42}
\textbf{Metadata}

\begin{itemize}
  \item Question ID: P6-PcFndChg\_P1-WNDiv2nd\_sonnet4\_Recreation\_04
  \item Primary KC: PERCENTAGE | Finding change | finding percentage increase/decrease
  \item Secondary KC: WHOLE NUMBERS | Division | dividing whole numbers
  \item Topic: Recreation such as sports, games, exercises, music, movie, dancing, painting, fishing and other recreation activities
  \item Grade: Primary 6
\end{itemize}

\textbf{Solution}

Step 1: Find the number of successful shots last month and this month.
Last month: $180$ successful shots
This month: $216$ successful shots

Step 2: Find the increase in the number of successful shots.
Increase = $216 - 180 = 36$ shots

Step 3: Calculate the percentage increase.
Percentage increase = $\frac{\text{Increase}}{\text{Original amount}} \times 100\%$
Percentage increase = $\frac{36}{180} \times 100\%$

Step 4: Simplify the fraction by dividing.
$\frac{36}{180} = \frac{36 \div 36}{180 \div 36} = \frac{1}{5}$

Step 5: Convert to percentage.
$\frac{1}{5} \times 100\% = \frac{100}{5}\% = 20\%$

Therefore, the percentage increase in the number of successful shots Sarah made is $20\%$.

\section*{Question 43}
\textbf{Metadata}

\begin{itemize}
  \item Question ID: P6-RoFndRoWN\_P1-WNAdd2nd\_sonnet4\_Recreation\_04
  \item Primary KC: RATIO | Finding ratio | finding the ratio of two or three given whole numbers
  \item Secondary KC: WHOLE NUMBERS | Addition | adding whole numbers
  \item Topic: Recreation such as sports, games, exercises, music, movie, dancing, painting, fishing and other recreation activities
  \item Grade: Primary 6
\end{itemize}

\textbf{Solution}

First, I need to find the total points collected by all three teams.

Red team: 45 points
Blue team: 36 points  
Yellow team: 54 points

Total points = $45 + 36 + 54 = 135$ points

Now I need to find the ratio of Red : Blue : Yellow
Ratio = $45 : 36 : 54$

To simplify this ratio, I need to find the highest common factor (HCF) of 45, 36, and 54.

Factors of 45: 1, 3, 5, 9, 15, 45
Factors of 36: 1, 2, 3, 4, 6, 9, 12, 18, 36
Factors of 54: 1, 2, 3, 6, 9, 18, 27, 54

The HCF of 45, 36, and 54 is 9.

Dividing each number by 9:
$45 \div 9 = 5$
$36 \div 9 = 4$
$54 \div 9 = 6$

Therefore, the ratio of points collected by the Red team to the Blue team to the Yellow team is $5 : 4 : 6$.

\section*{Question 44}
\textbf{Metadata}

\begin{itemize}
  \item Question ID: P6-AgRepLrEx\_P6-AgSmpLrEx\_sonnet4\_Recreation\_04
  \item Primary KC: ALGEBRA | Representation and concept | translation of real-world situations into linear algebraic expressions
  \item Secondary KC: ALGEBRA | Simplifying | simplifying linear expressions
  \item Topic: Recreation such as sports, games, exercises, music, movie, dancing, painting, fishing and other recreation activities
  \item Grade: Primary 6
\end{itemize}

\textbf{Solution}

Let me set up the problem step by step.

Given information:
- Points from free throws = $x$
- Points from 2-point shots = twice the free throw points = $2x$
- Points from 3-point shots = 5 more than free throw points = $x + 5$
- Total points = 89

Step 1: Write an algebraic expression for the total points.
Total points = Points from free throws + Points from 2-point shots + Points from 3-point shots
Total points = $x + 2x + (x + 5)$

Step 2: Simplify the linear expression.
$x + 2x + (x + 5) = x + 2x + x + 5 = 4x + 5$

Step 3: Set up and solve the equation.
Since the total points is 89:
$4x + 5 = 89$
$4x = 89 - 5$
$4x = 84$
$x = 21$

Step 4: Find the points from each type of shot.
- Points from free throws = $x = 21$ points
- Points from 2-point shots = $2x = 2(21) = 42$ points
- Points from 3-point shots = $x + 5 = 21 + 5 = 26$ points

Step 5: Verify the answer.
$21 + 42 + 26 = 89$ ✓

Therefore, $x = 21$, and Sarah's team scored 21 points from free throws, 42 points from 2-point shots, and 26 points from 3-point shots.

\section*{Question 45}
\textbf{Metadata}

\begin{itemize}
  \item Question ID: O1-PcFndRslt\_P1-WNSub2nd\_sonnet4\_Recreation\_04
  \item Primary KC: PERCENTAGE | Finding result after change | increasing/decreasing a quantity by a given percentage
  \item Secondary KC: WHOLE NUMBERS | Subtraction | subtracting whole numbers
  \item Topic: Recreation such as sports, games, exercises, music, movie, dancing, painting, fishing and other recreation activities
  \item Grade: Secondary O-level 1
\end{itemize}

\textbf{Solution}

**Step 1: Find the number of participants after the first round**

Initial participants = 480
Decrease in first round = 25\% of 480
$25\% \times 480 = \frac{25}{100} \times 480 = 120$

Participants remaining after first round = $480 - 120 = 360$

**Step 2: Find the number of participants after the second round**

Decrease in second round = 15\% of 360
$15\% \times 360 = \frac{15}{100} \times 360 = 54$

Participants remaining after second round = $360 - 54 = 306$

**Step 3: Find the total number of participants eliminated**

Total participants eliminated = Initial participants - Final participants
Total participants eliminated = $480 - 306 = 174$

Therefore, 174 participants were eliminated between the start of the tournament and the end of the second round.

\section*{Question 46}
\textbf{Metadata}

\begin{itemize}
  \item Question ID: O1-PcRepRvs\_O1-PcCnv2Dc\_sonnet4\_Recreation\_04
  \item Primary KC: PERCENTAGE | Representation and concept | reverse percentages
  \item Secondary KC: PERCENTAGE | Conversion to decimals | expressing percentage as a decimal
  \item Topic: Recreation such as sports, games, exercises, music, movie, dancing, painting, fishing and other recreation activities
  \item Grade: Secondary O-level 1
\end{itemize}

\textbf{Solution}

To find the total number of free throws Sarah attempted, I need to use reverse percentages since I know that 36 successful throws represents 72\% of her total attempts.

First, I'll convert the percentage to a decimal:
$72\% = \frac{72}{100} = 0.72$

Now I can set up the equation. If 36 throws represent 72\% of the total, then:
$0.72 \times \text{total attempts} = 36$

To find the total attempts, I'll divide both sides by 0.72:
$\text{total attempts} = \frac{36}{0.72}$

$\text{total attempts} = \frac{36}{0.72} = \frac{36}{\frac{72}{100}} = 36 \times \frac{100}{72} = \frac{3600}{72} = 50$

Therefore, Sarah attempted 50 free throws in total during that practice session.

To verify: $50 \times 0.72 = 36$ ✓

\section*{Question 47}
\textbf{Metadata}

\begin{itemize}
  \item Question ID: O2-RoRepDP\_P1-WNMul2nd\_sonnet4\_Recreation\_04
  \item Primary KC: RATIO | Representation and concept | direct proportion
  \item Secondary KC: WHOLE NUMBERS | Multiplication | multiplying whole numbers
  \item Topic: Recreation such as sports, games, exercises, music, movie, dancing, painting, fishing and other recreation activities
  \item Grade: Secondary O-level 2
\end{itemize}

\textbf{Solution}

(a) Finding the original total number of participants:

Let the original number of yoga participants be $3x$ and the original number of aerobics participants be $5x$, where the ratio is $3:5$.

After multiplying by $4$:
- New number of yoga participants = $3x \times 4 = 12x$
- New number of aerobics participants = $5x \times 4 = 20x$

Total participants after expansion = $12x + 20x = 32x = 192$

Solving for $x$:
$32x = 192$
$x = 192 \div 32 = 6$

Original total number of participants = $3x + 5x = 8x = 8 \times 6 = 48$

(b) Finding the original number of yoga participants:
Original number of yoga participants = $3x = 3 \times 6 = 18$

(c) Finding the original number of aerobics participants:
Original number of aerobics participants = $5x = 5 \times 6 = 30$

Verification: $18 + 30 = 48$ participants originally, and $(18 \times 4) + (30 \times 4) = 72 + 120 = 192$ participants after expansion. ✓

\section*{Question 48}
\textbf{Metadata}

\begin{itemize}
  \item Question ID: O3-MXMul\_O3-MXAdd\_sonnet4\_Recreation\_04
  \item Primary KC: MATRICES | Multiplication | multiplication of matrices
  \item Secondary KC: MATRICES | Addition | addition of matrices
  \item Topic: Recreation such as sports, games, exercises, music, movie, dancing, painting, fishing and other recreation activities
  \item Grade: Secondary O-level 3/4
\end{itemize}

\textbf{Solution}

(a) To find $AB$, we multiply the $2 \times 3$ matrix $A$ by the $3 \times 2$ matrix $B$:

$AB = \begin{pmatrix} 4 & 3 & 5 \\ 2 & 6 & 4 \end{pmatrix} \begin{pmatrix} 2 & 1 \\ 3 & 2 \\ 1 & 4 \end{pmatrix}$

For the first row, first column: $4(2) + 3(3) + 5(1) = 8 + 9 + 5 = 22$
For the first row, second column: $4(1) + 3(2) + 5(4) = 4 + 6 + 20 = 30$
For the second row, first column: $2(2) + 6(3) + 4(1) = 4 + 18 + 4 = 26$
For the second row, second column: $2(1) + 6(2) + 4(4) = 2 + 12 + 16 = 30$

Therefore: $AB = \begin{pmatrix} 22 & 30 \\ 26 & 30 \end{pmatrix}$

(b) To find $AB + C$, we add the corresponding elements:

$AB + C = \begin{pmatrix} 22 & 30 \\ 26 & 30 \end{pmatrix} + \begin{pmatrix} 1 & 2 \\ 3 & 1 \end{pmatrix}$

$AB + C = \begin{pmatrix} 22+1 & 30+2 \\ 26+3 & 30+1 \end{pmatrix} = \begin{pmatrix} 23 & 32 \\ 29 & 31 \end{pmatrix}$

(c) The total combined score is the sum of all elements in the final matrix:
$23 + 32 + 29 + 31 = 115$

Since the scores are in hundreds, the total combined score is $115 \times 100 = 11,500$ points.

\section*{Question 49}
\textbf{Metadata}

\begin{itemize}
  \item Question ID: O3-SPFndstd\_O2-SPFndmean\_sonnet4\_Recreation\_04
  \item Primary KC: STATISTICS AND PROBABILITY | Finding standard deviation | calculation of the standard deviation for a set of data
  \item Secondary KC: STATISTICS AND PROBABILITY | Finding mean deviation | calculation of the mean for a set of data
  \item Topic: Recreation such as sports, games, exercises, music, movie, dancing, painting, fishing and other recreation activities
  \item Grade: Secondary O-level 3/4
\end{itemize}

\textbf{Solution}

**Solution:**

**(a) Finding the mean**

Given data: 12, 18, 15, 22, 16, 20, 14, 19

Number of games, $n = 8$

Mean = $\frac{\text{Sum of all values}}{\text{Number of values}}$

Sum = $12 + 18 + 15 + 22 + 16 + 20 + 14 + 19 = 136$

Mean = $\frac{136}{8} = 17$ points per game

**(b) Finding the standard deviation**

Using the mean from part (a), $\bar{x} = 17$

We need to find the deviations from the mean and their squares:

\begin{align}
(12 - 17)^2 &= (-5)^2 = 25 \\
(18 - 17)^2 &= (1)^2 = 1 \\
(15 - 17)^2 &= (-2)^2 = 4 \\
(22 - 17)^2 &= (5)^2 = 25 \\
(16 - 17)^2 &= (-1)^2 = 1 \\
(20 - 17)^2 &= (3)^2 = 9 \\
(14 - 17)^2 &= (-3)^2 = 9 \\
(19 - 17)^2 &= (2)^2 = 4
\end{align}

Sum of squared deviations = $25 + 1 + 4 + 25 + 1 + 9 + 9 + 4 = 78$

Variance = $\frac{\text{Sum of squared deviations}}{n} = \frac{78}{8} = 9.75$

Standard deviation = $\sqrt{\text{Variance}} = \sqrt{9.75} = 3.12$ points (rounded to 2 decimal places)

**Answer:**
(a) The mean number of points scored per game is 17 points.
(b) The standard deviation is 3.12 points, indicating that the player's performance typically varies by about 3.12 points from the average of 17 points per game.

\end{document}
