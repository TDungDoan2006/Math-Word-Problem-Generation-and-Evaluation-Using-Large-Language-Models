\documentclass{article}
\usepackage[utf8]{inputenc}
\usepackage{amsmath}
\usepackage{amsfonts}
\usepackage{amssymb}
\usepackage{graphicx}
\usepackage{hyperref}
\title{'Sarah Solutions household finance v3 CLAUDE '}
\author{Tien Dung Doan}
\begin{document}
\maketitle
\section*{Question 1}
\textbf{Metadata}

\begin{itemize}
  \item Question ID: P3-WNDivRmd3d\_P1-WNCmp\_sonnet4\_Household Finance\_01
  \item Primary KC: WHOLE NUMBERS | Division | dividing whole numbers up to 3 digits by 1 digit with remainder 
  \item Secondary KC: WHOLE NUMBERS | Comparison and ordering | comparing and ordering whole numbers
  \item Topic: Household finance such as income, utility bills, money, interest, savings, instalment, mortgage, financial planning etc.
  \item Grade: Primary 3
\end{itemize}

\textbf{Solution}

Step 1: Divide \textdollar287 by 4 to find how much money goes in each jar.

$287 \div 4 = 71$ remainder $3$

Let me check: $71 \times 4 = 284$
$287 - 284 = 3$

So each jar will have \textdollar71 and there will be \textdollar3 left over.

Step 2: Compare the amount in each jar with the amount left over.

Amount in each jar: \textdollar71
Amount left over: \textdollar3

Comparing: $71 > 3$

Therefore:
- Each jar will have \textdollar71
- \textdollar3 will be left over
- The money in each jar (\textdollar71) is greater than the money left over (\textdollar3)

\section*{Question 2}
\textbf{Metadata}

\begin{itemize}
  \item Question ID: P3-WNMul3d1d\_P1-WNAdd2nd\_sonnet4\_Household Finance\_01
  \item Primary KC: WHOLE NUMBERS | Multiplication | multiplying whole numbers up to 3 digits by 1 digit
  \item Secondary KC: WHOLE NUMBERS | Addition | adding whole numbers
  \item Topic: Household finance such as income, utility bills, money, interest, savings, instalment, mortgage, financial planning etc.
  \item Grade: Primary 3
\end{itemize}

\textbf{Solution}

To find the total amount Mrs. Tan paid for electricity in 4 months, I need to find the cost for February, March, and April first, then add January's bill.

Step 1: Find the total cost for February, March, and April.
Since each of these 3 months had a bill of \textdollar89:
$3 \times 89 = 267$
So February, March, and April cost \textdollar267 altogether.

Step 2: Add January's bill to get the total for all 4 months.
January's bill was \textdollar126.
Total for 4 months = \textdollar267 + \textdollar126
$267 + 126 = 393$

Therefore, Mrs. Tan paid \textdollar393 for electricity in these 4 months altogether.

\section*{Question 3}
\textbf{Metadata}

\begin{itemize}
  \item Question ID: P3-WNMul3d1d\_P1-WNSub2nd\_sonnet4\_Household Finance\_01
  \item Primary KC: WHOLE NUMBERS | Multiplication | multiplying whole numbers up to 3 digits by 1 digit
  \item Secondary KC: WHOLE NUMBERS | Subtraction | subtracting whole numbers
  \item Topic: Household finance such as income, utility bills, money, interest, savings, instalment, mortgage, financial planning etc.
  \item Grade: Primary 3
\end{itemize}

\textbf{Solution}

First, I need to find out how much more money Mrs. Chen needs to save.

Total amount needed: \textdollar324
Amount already saved: \textdollar29

Amount still needed = \textdollar324 - \textdollar29 = \textdollar295

Next, I need to find out how much she needs to save each week.
She will save for 4 weeks, so I need to divide the remaining amount by 4.

However, since this involves division which is beyond Primary 3 level, I can think of this as: What number multiplied by 4 gives \textdollar295?

Let me check: $295 \div 4 = 73$ remainder $3$

This means $4 \times 73 = 292$
Let me verify: $4 \times 73 = 292$

Since $295 - 292 = 3$, Mrs. Chen would be \textdollar3 short.

Therefore, she needs to save $73 + 1 = 74$ dollars each week.

Let me check: $4 \times 74 = 296$
Total she would have: \textdollar296 + \textdollar29 = \textdollar325

Since \textdollar325 > \textdollar324, this is sufficient.

Therefore, Mrs. Chen needs to save \textdollar74 each week.

\section*{Question 4}
\textbf{Metadata}

\begin{itemize}
  \item Question ID: P3-FrAddRl12\_P3-FrSmp\_sonnet4\_Household Finance\_01
  \item Primary KC: FRACTIONS | Addition | adding two related fractions within one whole with denominators of given fractions not exceeding 12
  \item Secondary KC: FRACTIONS | Simplifying | expressing a fraction in its simplest form
  \item Topic: Household finance such as income, utility bills, money, interest, savings, instalment, mortgage, financial planning etc.
  \item Grade: Primary 3
\end{itemize}

\textbf{Solution}

To find the total fraction of the monthly budget spent on groceries and utility bills, I need to add the two fractions.

Fraction spent on groceries = $\frac{3}{8}$
Fraction spent on utility bills = $\frac{1}{8}$

Since both fractions have the same denominator, I can add them directly:
$\frac{3}{8} + \frac{1}{8} = \frac{3+1}{8} = \frac{4}{8}$

Now I need to express $\frac{4}{8}$ in its simplest form by finding the greatest common factor of 4 and 8.
The factors of 4 are: 1, 2, 4
The factors of 8 are: 1, 2, 4, 8
The greatest common factor is 4.

Dividing both numerator and denominator by 4:
$\frac{4}{8} = \frac{4 \div 4}{8 \div 4} = \frac{1}{2}$

Therefore, Sarah's family spent $\frac{1}{2}$ of their monthly budget on groceries and utility bills altogether.

\section*{Question 5}
\textbf{Metadata}

\begin{itemize}
  \item Question ID: P3-FrSubRl12\_P3-FrSmp\_sonnet4\_Household Finance\_01
  \item Primary KC: FRACTIONS | Subtraction | subtracting two related fractions within one whole with denominators of given fractions not exceeding 12
  \item Secondary KC: FRACTIONS | Simplifying | expressing a fraction in its simplest form
  \item Topic: Household finance such as income, utility bills, money, interest, savings, instalment, mortgage, financial planning etc.
  \item Grade: Primary 3
\end{itemize}

\textbf{Solution}

To find how much more they spend on household expenses than on food and groceries, I need to subtract the fraction spent on food from the fraction spent on household expenses.

Fraction spent on household expenses = $\frac{5}{8}$

Fraction spent on food and groceries = $\frac{3}{8}$

Difference = $\frac{5}{8} - \frac{3}{8}$

Since both fractions have the same denominator, I can subtract the numerators directly:

$\frac{5}{8} - \frac{3}{8} = \frac{5-3}{8} = \frac{2}{8}$

Now I need to express $\frac{2}{8}$ in its simplest form by finding the greatest common factor of 2 and 8.

Factors of 2: 1, 2
Factors of 8: 1, 2, 4, 8

The greatest common factor is 2.

$\frac{2}{8} = \frac{2 \div 2}{8 \div 2} = \frac{1}{4}$

Therefore, Mrs. Tan's family spends $\frac{1}{4}$ more of their monthly income on household expenses than on food and groceries.

\section*{Question 6}
\textbf{Metadata}

\begin{itemize}
  \item Question ID: P4-WNMul4d1d\_P1-WNSub2nd\_sonnet4\_Household Finance\_01
  \item Primary KC: WHOLE NUMBERS | Multiplication | multiplying whole numbers up to 4 digits by 1 digit or up to 3 digits by 2 digits
  \item Secondary KC: WHOLE NUMBERS | Subtraction | subtracting whole numbers
  \item Topic: Household finance such as income, utility bills, money, interest, savings, instalment, mortgage, financial planning etc.
  \item Grade: Primary 4
\end{itemize}

\textbf{Solution}

Step 1: Calculate the total monthly income of the Chen family.
Total income = Mr. Chen's income + Mrs. Chen's income
Total income = \textdollar3,245 + \textdollar2,850 = \textdollar6,095

Step 2: Calculate the total utility bills.
Total utility bills = Electricity + Water + Internet
Total utility bills = \textdollar125 + \textdollar68 + \textdollar89 = \textdollar282

Step 3: Calculate the total fixed expenses (utility bills + savings).
Total fixed expenses = Total utility bills + Monthly savings
Total fixed expenses = \textdollar282 + \textdollar800 = \textdollar1,082

Step 4: Calculate the money left for other expenses.
Money left = Total income - Total fixed expenses
Money left = \textdollar6,095 - \textdollar1,082 = \textdollar5,013

Therefore, the Chen family will have \textdollar5,013 left for other expenses.

\section*{Question 7}
\textbf{Metadata}

\begin{itemize}
  \item Question ID: P4-WNDiv4d1d\_P1-WNAdd2nd\_sonnet4\_Household Finance\_01
  \item Primary KC: WHOLE NUMBERS | Division | dividing whole numbers up to 4 digits by 1 digit
  \item Secondary KC: WHOLE NUMBERS | Addition | adding whole numbers
  \item Topic: Household finance such as income, utility bills, money, interest, savings, instalment, mortgage, financial planning etc.
  \item Grade: Primary 4
\end{itemize}

\textbf{Solution}

Step 1: Find the monthly utility bill amount.
Total utility bills over 8 months = \textdollar3264
Monthly utility bill = \textdollar3264 ÷ 8 = \textdollar408

Step 2: Find the total amount paid including all fees.
Total utility bills = \textdollar3264
Security deposit = \textdollar156
Connection fee = \textdollar72
Total amount = \textdollar3264 + \textdollar156 + \textdollar72 = \textdollar3492

Therefore, the Lim family pays a total of \textdollar3492 for utilities including all additional fees.

\section*{Question 8}
\textbf{Metadata}

\begin{itemize}
  \item Question ID: P4-FrRepSet\_P3-FrCnvEq\_sonnet4\_Household Finance\_01
  \item Primary KC: FRACTIONS | Representation and concept | expressing a part of a set as a fraction
  \item Secondary KC: FRACTIONS | Conversion to equivalent fractions | Conversion to equivalent fractions (given either the denominator or the numerator)
  \item Topic: Household finance such as income, utility bills, money, interest, savings, instalment, mortgage, financial planning etc.
  \item Grade: Primary 4
\end{itemize}

\textbf{Solution}

To compare the fractions $\frac{3}{8}$ and $\frac{9}{24}$, I need to convert one fraction to have the same denominator as the other.

I will convert $\frac{3}{8}$ to have denominator 24.

To change the denominator from 8 to 24: $24 \div 8 = 3$

So I multiply both numerator and denominator by 3:
$\frac{3}{8} = \frac{3 \times 3}{8 \times 3} = \frac{9}{24}$

Comparing the fractions: $\frac{9}{24} = \frac{9}{24}$

Therefore, Mrs. Tan and Mrs. Lim save the same fraction of their salaries.

To find how much money Mrs. Tan saves each month:
Mrs. Tan saves $\frac{3}{8}$ of \textdollar2400

$\frac{3}{8} \times 2400 = \frac{3 \times 2400}{8} = \frac{7200}{8} = 900$

Therefore, Mrs. Tan saves \textdollar900 each month.

\section*{Question 9}
\textbf{Metadata}

\begin{itemize}
  \item Question ID: P4-FrSubU12\_P2-FrAdd2nd\_sonnet4\_Household Finance\_01
  \item Primary KC: FRACTIONS | Subtraction | subtracting unlike fractions with two different denominators not exceeding 12
  \item Secondary KC: FRACTIONS | Addition | adding fractions
  \item Topic: Household finance such as income, utility bills, money, interest, savings, instalment, mortgage, financial planning etc.
  \item Grade: Primary 4
\end{itemize}

\textbf{Solution}

To find how much more they spend on household expenses than on savings, I need to subtract the fractions:

$\frac{5}{8} - \frac{1}{6}$

Since the denominators are different (8 and 6), I need to find a common denominator.
The least common multiple of 8 and 6 is 24.

$\frac{5}{8} = \frac{5 \times 3}{8 \times 3} = \frac{15}{24}$

$\frac{1}{6} = \frac{1 \times 4}{6 \times 4} = \frac{4}{24}$

Now I can subtract: $\frac{15}{24} - \frac{4}{24} = \frac{11}{24}$

So they spend $\frac{11}{24}$ more of their monthly income on household expenses than on savings.

To find the total fraction spent on household expenses and savings combined, I need to add the fractions:

$\frac{5}{8} + \frac{1}{6}$

Using the same common denominator of 24:
$\frac{15}{24} + \frac{4}{24} = \frac{19}{24}$

Therefore, they spend $\frac{19}{24}$ of their monthly income on household expenses and savings combined.

\section*{Question 10}
\textbf{Metadata}

\begin{itemize}
  \item Question ID: P4-DcSub2d\_P4-DcCmp3d\_sonnet4\_Household Finance\_01
  \item Primary KC: DECIMALS | Subtraction | subtracting decimals (up to 2 decimal places)
  \item Secondary KC: DECIMALS | Comparison and ordering | comparing and ordering decimals up to 3 decimal places
  \item Topic: Household finance such as income, utility bills, money, interest, savings, instalment, mortgage, financial planning etc.
  \item Grade: Primary 4
\end{itemize}

\textbf{Solution}

(a) To find how much the electricity bill decreased:
January electricity bill - February electricity bill = \textdollar68.75 - \textdollar52.30 = \textdollar16.45

(b) To find how much the water bill decreased:
January water bill - February water bill = \textdollar23.40 - \textdollar19.85 = \textdollar3.55

(c) To compare the decreases:
Electricity bill decrease: \textdollar16.45
Water bill decrease: \textdollar3.55
Since $16.45 > 3.55$, the electricity bill had a greater decrease.
The difference between the decreases: \textdollar16.45 - \textdollar3.55 = \textdollar12.90

Therefore, the electricity bill had a greater decrease by \textdollar12.90.

\section*{Question 11}
\textbf{Metadata}

\begin{itemize}
  \item Question ID: P4-DcMul2d1d\_P4-DcCmp3d\_sonnet4\_Household Finance\_01
  \item Primary KC: DECIMALS | Multiplication | multiplying decimals (up to 2 decimal places) by a 1-digit whole number
  \item Secondary KC: DECIMALS | Comparison and ordering | comparing and ordering decimals up to 3 decimal places
  \item Topic: Household finance such as income, utility bills, money, interest, savings, instalment, mortgage, financial planning etc.
  \item Grade: Primary 4
\end{itemize}

\textbf{Solution}

**Solution:**

**(a) Finding the February electricity bill:**

First, I need to find 3 times \textdollar2.65:
$3 \times 2.65 = 7.95$

The February bill = January bill + increase
February bill = \textdollar45.80 + \textdollar7.95 = \textdollar53.75

**(b) Arranging the bills in ascending order:**

The three electricity bills are:
- January: \textdollar45.80 (which is \textdollar45.800)
- February: \textdollar53.75 (which is \textdollar53.750)
- March: \textdollar53.750

Comparing the decimals:
- \textdollar45.800 is the smallest
- \textdollar53.750 = \textdollar53.750 (February and March are equal)

Therefore, in ascending order: \textdollar45.80, \textdollar53.75, \textdollar53.750

Or written as: January, February, March

\section*{Question 12}
\textbf{Metadata}

\begin{itemize}
  \item Question ID: P4-DcMul2d1d\_P4-DcRnd3d\_sonnet4\_Household Finance\_01
  \item Primary KC: DECIMALS | Multiplication | multiplying decimals (up to 2 decimal places) by a 1-digit whole number
  \item Secondary KC: DECIMALS | Rounding | rounding decimals up to 3 decimal places to the nearest whole number, 1 decimal place and 2 decimal places 
  \item Topic: Household finance such as income, utility bills, money, interest, savings, instalment, mortgage, financial planning etc.
  \item Grade: Primary 4
\end{itemize}

\textbf{Solution}

Step 1: Calculate the exact electricity bill.
Electricity bill = Number of units × Rate per unit
Electricity bill = $147.6 \times 0.28$

Using the standard multiplication method:
$147.6 \times 0.28 = 147.6 \times 0.2 + 147.6 \times 0.08$
$= 29.52 + 11.808$
$= 41.328$

The exact electricity bill is \textdollar41.328.

Step 2: Round to the nearest dollar (nearest whole number).
$41.328$ rounded to the nearest dollar is \textdollar41.
(Since the digit in the tenths place is 3, which is less than 5, we round down)

Step 3: Round to the nearest cent (nearest 2 decimal places).
$41.328$ rounded to the nearest cent is \textdollar41.33.
(Since the digit in the thousandths place is 8, which is greater than 5, we round up)

Therefore:
- Exact bill: \textdollar41.328
- Rounded to nearest dollar: \textdollar41
- Rounded to nearest cent: \textdollar41.33

\section*{Question 13}
\textbf{Metadata}

\begin{itemize}
  \item Question ID: P4-DcMul2d1d\_P4-DcSub2nd\_sonnet4\_Household Finance\_01
  \item Primary KC: DECIMALS | Multiplication | multiplying decimals (up to 2 decimal places) by a 1-digit whole number
  \item Secondary KC: DECIMALS | Subtraction | subtracting decimals
  \item Topic: Household finance such as income, utility bills, money, interest, savings, instalment, mortgage, financial planning etc.
  \item Grade: Primary 4
\end{itemize}

\textbf{Solution}

Step 1: Find the April electricity bill before rebate.
April bill = 3 × March bill
April bill = 3 × \textdollar42.85 = \textdollar128.55

Step 2: Find the actual amount paid after rebate.
Actual amount paid = April bill - Government rebate
Actual amount paid = \textdollar128.55 - \textdollar18.60 = \textdollar109.95

Therefore, Mrs. Tan's family actually paid \textdollar109.95 for their April electricity bill after the rebate.

\section*{Question 14}
\textbf{Metadata}

\begin{itemize}
  \item Question ID: P4-DcDiv2d1d\_P4-DcRnd3d\_sonnet4\_Household Finance\_01
  \item Primary KC: DECIMALS | Division | dividing decimals (up to 2 decimal places) by a 1-digit whole number
  \item Secondary KC: DECIMALS | Rounding | rounding decimals up to 3 decimal places to the nearest whole number, 1 decimal place and 2 decimal places 
  \item Topic: Household finance such as income, utility bills, money, interest, savings, instalment, mortgage, financial planning etc.
  \item Grade: Primary 4
\end{itemize}

\textbf{Solution}

Step 1: Find the total utility bill amount.
Total utility bill = \textdollar89.64 + \textdollar45.28 + \textdollar37.16 = \textdollar172.08

Step 2: Divide the total amount equally among 4 children.
Amount per child = \textdollar172.08 ÷ 4

To divide \textdollar172.08 by 4:
$172.08 ÷ 4 = 43.02$

Step 3: Round the answer to the nearest cent (2 decimal places).
\textdollar43.02 is already rounded to 2 decimal places.

Therefore, each child needs to contribute \textdollar43.02 towards the monthly utility bills.

\section*{Question 15}
\textbf{Metadata}

\begin{itemize}
  \item Question ID: P5-FrAddMix\_P5-FrCnv2Dc\_sonnet4\_Household Finance\_01
  \item Primary KC: FRACTIONS | Addition | adding mixed numbers
  \item Secondary KC: FRACTIONS | Conversion to decimals | expressing fractions as decimals
  \item Topic: Household finance such as income, utility bills, money, interest, savings, instalment, mortgage, financial planning etc.
  \item Grade: Primary 5
\end{itemize}

\textbf{Solution}

(a) Total utilities in January:
$2\frac{3}{4} + 3\frac{1}{2}$

First, convert to improper fractions or add whole numbers and fractions separately:
$2\frac{3}{4} + 3\frac{1}{2} = (2 + 3) + (\frac{3}{4} + \frac{1}{2})$
$= 5 + (\frac{3}{4} + \frac{2}{4})$
$= 5 + \frac{5}{4}$
$= 5 + 1\frac{1}{4}$
$= 6\frac{1}{4}$

Total utilities in January = \textdollar$6\frac{1}{4}$

(b) Total utilities in February:
$2\frac{1}{4} + 3\frac{3}{4}$
$= (2 + 3) + (\frac{1}{4} + \frac{3}{4})$
$= 5 + \frac{4}{4}$
$= 5 + 1$
$= 6$

Total utilities in February = \textdollar$6$

(c) Total utility expenses for both months:
$6\frac{1}{4} + 6 = 12\frac{1}{4}$

To express $12\frac{1}{4}$ as a decimal:
$\frac{1}{4} = 0.25$

Therefore, $12\frac{1}{4} = 12.25$

The total utility expenses for both months is \textdollar$12.25$

\section*{Question 16}
\textbf{Metadata}

\begin{itemize}
  \item Question ID: P5-FrSubMix\_P3-FrSmp\_sonnet4\_Household Finance\_01
  \item Primary KC: FRACTIONS | Subtraction | subtracting mixed numbers
  \item Secondary KC: FRACTIONS | Simplifying | expressing a fraction in its simplest form
  \item Topic: Household finance such as income, utility bills, money, interest, savings, instalment, mortgage, financial planning etc.
  \item Grade: Primary 5
\end{itemize}

\textbf{Solution}

To find how much money the Lim family saved, I need to subtract the actual bill from the budgeted amount.

Budgeted amount: \textdollar$4\frac{3}{4}$
Actual bill: \textdollar$2\frac{5}{6}$

Amount saved = $4\frac{3}{4} - 2\frac{5}{6}$

First, I'll convert the mixed numbers to improper fractions:
$4\frac{3}{4} = \frac{4 \times 4 + 3}{4} = \frac{19}{4}$
$2\frac{5}{6} = \frac{2 \times 6 + 5}{6} = \frac{17}{6}$

To subtract fractions, I need a common denominator. The LCM of 4 and 6 is 12.

$\frac{19}{4} = \frac{19 \times 3}{4 \times 3} = \frac{57}{12}$
$\frac{17}{6} = \frac{17 \times 2}{6 \times 2} = \frac{34}{12}$

Now I can subtract:
$\frac{57}{12} - \frac{34}{12} = \frac{57 - 34}{12} = \frac{23}{12}$

To check if $\frac{23}{12}$ is in its simplest form, I need to find the GCD of 23 and 12.
Since 23 is a prime number and doesn't divide 12, the GCD is 1.

Therefore, $\frac{23}{12}$ is already in its simplest form.

Converting back to a mixed number: $\frac{23}{12} = 1\frac{11}{12}$

The Lim family saved \textdollar$1\frac{11}{12}$ compared to their budget.

\section*{Question 17}
\textbf{Metadata}

\begin{itemize}
  \item Question ID: P5-FrSubMix\_P5-FrCnv2Dc\_sonnet4\_Household Finance\_01
  \item Primary KC: FRACTIONS | Subtraction | subtracting mixed numbers
  \item Secondary KC: FRACTIONS | Conversion to decimals | expressing fractions as decimals
  \item Topic: Household finance such as income, utility bills, money, interest, savings, instalment, mortgage, financial planning etc.
  \item Grade: Primary 5
\end{itemize}

\textbf{Solution}

To find how much time the Tan family had left, I need to subtract the time they spent from their total budgeted time.

Total budgeted time: $3\frac{3}{4}$ hours
Time spent reviewing bills: $1\frac{1}{2}$ hours

First, I'll subtract the mixed numbers:
$3\frac{3}{4} - 1\frac{1}{2}$

To subtract mixed numbers, I need a common denominator for the fractions.
$3\frac{3}{4} - 1\frac{2}{4}$

Subtracting the whole numbers: $3 - 1 = 2$
Subtracting the fractions: $\frac{3}{4} - \frac{2}{4} = \frac{1}{4}$

So the remaining time is $2\frac{1}{4}$ hours.

Now I need to express this as a decimal:
$2\frac{1}{4} = 2 + \frac{1}{4}$

To convert $\frac{1}{4}$ to a decimal: $1 \div 4 = 0.25$

Therefore: $2\frac{1}{4} = 2 + 0.25 = 2.25$

The Tan family had $2.25$ hours left for other financial planning activities.

\section*{Question 18}
\textbf{Metadata}

\begin{itemize}
  \item Question ID: P5-FrMulImN\_P2-FrAdd2nd\_sonnet4\_Household Finance\_01
  \item Primary KC: FRACTIONS | Multiplication | multiplying a proper/improper fraction and a whole number
  \item Secondary KC: FRACTIONS | Addition | adding fractions
  \item Topic: Household finance such as income, utility bills, money, interest, savings, instalment, mortgage, financial planning etc.
  \item Grade: Primary 5
\end{itemize}

\textbf{Solution}

Step 1: Find February's electricity bill.
February's bill = $\frac{5}{6} \times 120$
$= \frac{5 \times 120}{6}$
$= \frac{600}{6}$
$= 100$
So February's bill was \textdollar100.

Step 2: Find March's electricity bill.
March's bill = $\frac{3}{4} \times 120$
$= \frac{3 \times 120}{4}$
$= \frac{360}{4}$
$= 90$
So March's bill was \textdollar90.

Step 3: Find the total amount paid over three months.
Total = January's bill + February's bill + March's bill
Total = \textdollar120 + \textdollar100 + \textdollar90
Total = \textdollar310

Therefore, Mrs. Tan paid a total of \textdollar310 for electricity over the three months.

\section*{Question 19}
\textbf{Metadata}

\begin{itemize}
  \item Question ID: P5-FrMulImN\_P3-FrSmp\_sonnet4\_Household Finance\_01
  \item Primary KC: FRACTIONS | Multiplication | multiplying a proper/improper fraction and a whole number
  \item Secondary KC: FRACTIONS | Simplifying | expressing a fraction in its simplest form
  \item Topic: Household finance such as income, utility bills, money, interest, savings, instalment, mortgage, financial planning etc.
  \item Grade: Primary 5
\end{itemize}

\textbf{Solution}

To find how much Mrs. Tan spends on household expenses, I need to multiply her monthly income by the fraction she spends.

Monthly income = \textdollar4800
Fraction spent on household expenses = $\frac{5}{8}$

Amount spent = $\frac{5}{8} \times 4800$

To multiply a fraction by a whole number:
$\frac{5}{8} \times 4800 = \frac{5 \times 4800}{8} = \frac{24000}{8}$

Now I need to simplify this fraction:
$\frac{24000}{8} = 3000$

Since $3000 = \frac{3000}{1}$, this is already in its simplest form as a whole number.

Therefore, Mrs. Tan spends \textdollar3000 on household expenses each month.

\section*{Question 20}
\textbf{Metadata}

\begin{itemize}
  \item Question ID: P5-FrMulImN\_P5-FrCnv2Dc\_sonnet4\_Household Finance\_01
  \item Primary KC: FRACTIONS | Multiplication | multiplying a proper/improper fraction and a whole number
  \item Secondary KC: FRACTIONS | Conversion to decimals | expressing fractions as decimals
  \item Topic: Household finance such as income, utility bills, money, interest, savings, instalment, mortgage, financial planning etc.
  \item Grade: Primary 5
\end{itemize}

\textbf{Solution}

Step 1: Find the amount of reduction in the electricity bill.
Reduction = $\frac{3}{8} \times 240$
Reduction = $\frac{3 \times 240}{8}$
Reduction = $\frac{720}{8}$
Reduction = \textdollar90

Step 2: Find the new monthly electricity bill.
New bill = Original bill - Reduction
New bill = \textdollar240 - \textdollar90
New bill = \textdollar150

Step 3: Express the new bill as a decimal.
Since \textdollar150 is already a whole number, when expressed as a decimal it is \textdollar150.00 or simply \textdollar150.0

Therefore, Mrs. Tan's new monthly electricity bill is \textdollar150.0

\section*{Question 21}
\textbf{Metadata}

\begin{itemize}
  \item Question ID: P5-FrMulImIm\_P2-FrCmp\_sonnet4\_Household Finance\_01
  \item Primary KC: FRACTIONS | Multiplication | multiplying two improper fractions
  \item Secondary KC: FRACTIONS | Comparison and ordering | comparing and ordering fractions
  \item Topic: Household finance such as income, utility bills, money, interest, savings, instalment, mortgage, financial planning etc.
  \item Grade: Primary 5
\end{itemize}

\textbf{Solution}

Step 1: Find the amount spent on housing expenses.
Housing expenses = $\frac{7}{4}$ of monthly income
Housing expenses = $\frac{7}{4} \times 2400$
Housing expenses = $\frac{7 \times 2400}{4} = \frac{16800}{4} = 4200$
So they spend \textdollar4200 on housing expenses.

Step 2: Find the amount spent on mortgage payment.
Mortgage payment = $\frac{8}{5}$ of housing expenses
Mortgage payment = $\frac{8}{5} \times 4200$
Mortgage payment = $\frac{8 \times 4200}{5} = \frac{33600}{5} = 6720$
So they spend \textdollar6720 on their mortgage payment each month.

Step 3: Compare the mortgage payment with half of their total monthly income.
Half of monthly income = $\frac{1}{2} \times 2400 = 1200$
So half of their monthly income is \textdollar1200.

Step 4: Compare the fractions to determine which is larger.
Mortgage payment as a fraction of income = $\frac{6720}{2400} = \frac{28}{10} = \frac{14}{5}$
Half of income as a fraction of income = $\frac{1200}{2400} = \frac{1}{2}$

To compare $\frac{14}{5}$ and $\frac{1}{2}$:
$\frac{14}{5} = \frac{28}{10}$ and $\frac{1}{2} = \frac{5}{10}$

Since $\frac{28}{10} > \frac{5}{10}$, we have $\frac{14}{5} > \frac{1}{2}$

Therefore, their mortgage payment of \textdollar6720 is more than half of their total monthly income.

\section*{Question 22}
\textbf{Metadata}

\begin{itemize}
  \item Question ID: P5-FrMulImIm\_P5-FrCnv2Dc\_sonnet4\_Household Finance\_01
  \item Primary KC: FRACTIONS | Multiplication | multiplying two improper fractions
  \item Secondary KC: FRACTIONS | Conversion to decimals | expressing fractions as decimals
  \item Topic: Household finance such as income, utility bills, money, interest, savings, instalment, mortgage, financial planning etc.
  \item Grade: Primary 5
\end{itemize}

\textbf{Solution}

Step 1: Find Mrs. Tan's original electricity bill.
Original electricity bill = \textdollar120

Step 2: Calculate the increase in the electricity bill.
Increase in electricity bill = $\frac{7}{4} \times \frac{120}{1}$
$= \frac{7 \times 120}{4 \times 1}$
$= \frac{840}{4}$
$= \textdollar210$

Step 3: Find the original water bill.
Original water bill = $\frac{3}{2} \times \textdollar120$
$= \frac{3 \times 120}{2}$
$= \frac{360}{2}$
$= \textdollar180$

Step 4: Calculate the increase in the water bill.
Increase in water bill = $\frac{7}{4} \times \frac{180}{1}$
$= \frac{7 \times 180}{4 \times 1}$
$= \frac{1260}{4}$
$= \textdollar315$

Step 5: Find the total increase in both bills.
Total increase = \textdollar210 + \textdollar315 = \textdollar525

Step 6: Express the answer as a decimal.
$525 = 525.0$

Therefore, the total increase in both bills combined is \textdollar525.0.

\section*{Question 23}
\textbf{Metadata}

\begin{itemize}
  \item Question ID: P5-FrMulMixN\_P2-FrAdd2nd\_sonnet4\_Household Finance\_01
  \item Primary KC: FRACTIONS | Multiplication | multiplying a mixed number and a whole number
  \item Secondary KC: FRACTIONS | Addition | adding fractions
  \item Topic: Household finance such as income, utility bills, money, interest, savings, instalment, mortgage, financial planning etc.
  \item Grade: Primary 5
\end{itemize}

\textbf{Solution}

Step 1: Calculate January's electricity bill.
January bill = $2\frac{1}{4} \times \textdollar180$

First, convert the mixed number to an improper fraction:
$2\frac{1}{4} = \frac{9}{4}$

January bill = $\frac{9}{4} \times 180 = \frac{9 \times 180}{4} = \frac{1620}{4} = 405$

So January's bill was \textdollar405.

Step 2: Calculate February's electricity bill.
February bill = $\frac{2}{3} \times \textdollar180 = \frac{2 \times 180}{3} = \frac{360}{3} = 120$

So February's bill was \textdollar120.

Step 3: Add the two months' bills to find the total.
Total = \textdollar405 + \textdollar120

To add these amounts, we can express them as fractions with a common denominator:
\textdollar405 = $\frac{405}{1} = \frac{405}{1}$
\textdollar120 = $\frac{120}{1} = \frac{120}{1}$

Total = $\frac{405}{1} + \frac{120}{1} = \frac{405 + 120}{1} = \frac{525}{1} = 525$

Therefore, Sarah's family paid \textdollar525 for electricity over the two months.

\section*{Question 24}
\textbf{Metadata}

\begin{itemize}
  \item Question ID: P5-DcMul3dK\_P4-DcCmp3d\_sonnet4\_Household Finance\_01
  \item Primary KC: DECIMALS | Multiplication | multiplying decimals (up to 3 decimal places) by 10, 100, 1000 and their multiples
  \item Secondary KC: DECIMALS | Comparison and ordering | comparing and ordering decimals up to 3 decimal places
  \item Topic: Household finance such as income, utility bills, money, interest, savings, instalment, mortgage, financial planning etc.
  \item Grade: Primary 5
\end{itemize}

\textbf{Solution}

Step 1: Calculate the cost for 1000 units of electricity for each month.

January: $0.248 \times 1000 = \textdollar248.000 = \textdollar248$

February: $0.315 \times 1000 = \textdollar315.000 = \textdollar315$

March: $0.287 \times 1000 = \textdollar287.000 = \textdollar287$

Step 2: Compare and order the costs from lowest to highest.

Comparing the three amounts:
$248 < 287 < 315$

Therefore, arranging from lowest to highest:
January: \textdollar248, March: \textdollar287, February: \textdollar315

January had the most affordable electricity rate.

\section*{Question 25}
\textbf{Metadata}

\begin{itemize}
  \item Question ID: P5-DcMul3dK\_P4-DcAdd2nd\_sonnet4\_Household Finance\_01
  \item Primary KC: DECIMALS | Multiplication | multiplying decimals (up to 3 decimal places) by 10, 100, 1000 and their multiples
  \item Secondary KC: DECIMALS | Addition | adding decimals
  \item Topic: Household finance such as income, utility bills, money, interest, savings, instalment, mortgage, financial planning etc.
  \item Grade: Primary 5
\end{itemize}

\textbf{Solution}

To find the total utility bill, I need to calculate each bill separately and then add them together.

**Step 1: Calculate the electricity bill**
Electricity cost = \textdollar0.285 × 400
When multiplying a decimal by 100 (since 400 = 4 × 100):
\textdollar0.285 × 400 = \textdollar0.285 × 4 × 100 = \textdollar1.140 × 100 = \textdollar114.00

**Step 2: Calculate the water bill**
Water cost = \textdollar0.147 × 300
When multiplying a decimal by 100 (since 300 = 3 × 100):
\textdollar0.147 × 300 = \textdollar0.147 × 3 × 100 = \textdollar0.441 × 100 = \textdollar44.10

**Step 3: Add all three bills together**
Total utility bill = Electricity bill + Water bill + Internet bill
Total utility bill = \textdollar114.00 + \textdollar44.10 + \textdollar39.90
Total utility bill = \textdollar198.00

Therefore, the Tan family needs to pay \textdollar198.00 for their utilities this month.

\section*{Question 26}
\textbf{Metadata}

\begin{itemize}
  \item Question ID: P5-DcDiv3dK\_P4-DcCmp3d\_sonnet4\_Household Finance\_01
  \item Primary KC: DECIMALS | Division | dividing decimals (up to 3 decimal places) by 10, 100, 1000 and their multiples
  \item Secondary KC: DECIMALS | Comparison and ordering | comparing and ordering decimals up to 3 decimal places
  \item Topic: Household finance such as income, utility bills, money, interest, savings, instalment, mortgage, financial planning etc.
  \item Grade: Primary 5
\end{itemize}

\textbf{Solution}

Step 1: Calculate March's electricity bill.
March's bill = January's bill ÷ 100
March's bill = \textdollar45.600 ÷ 100 = \textdollar0.456

Step 2: List all three bills:
- January: \textdollar45.600
- February: \textdollar0.046
- March: \textdollar0.456

Step 3: Compare and order the bills from lowest to highest.
Comparing the decimal values:
- \textdollar0.046 (February)
- \textdollar0.456 (March) 
- \textdollar45.600 (January)

Therefore, March's electricity bill is \textdollar0.456, and the bills arranged from lowest to highest are: February (\textdollar0.046), March (\textdollar0.456), January (\textdollar45.600).

\section*{Question 27}
\textbf{Metadata}

\begin{itemize}
  \item Question ID: P5-DcDiv3dK\_P4-DcSub2nd\_sonnet4\_Household Finance\_01
  \item Primary KC: DECIMALS | Division | dividing decimals (up to 3 decimal places) by 10, 100, 1000 and their multiples
  \item Secondary KC: DECIMALS | Subtraction | subtracting decimals
  \item Topic: Household finance such as income, utility bills, money, interest, savings, instalment, mortgage, financial planning etc.
  \item Grade: Primary 5
\end{itemize}

\textbf{Solution}

Step 1: Find the total subsidy amount Mrs. Tan received.
Subsidy amount = Original bill - Final bill after subsidy
Subsidy amount = \textdollar65.320 - \textdollar48.750 = \textdollar16.570

Step 2: Find how much subsidy each household received.
Since the subsidy is distributed equally among 100 households:
Subsidy per household = \textdollar16.570 ÷ 100 = \textdollar0.165

Therefore, each household received \textdollar0.165 in subsidy.

\section*{Question 28}
\textbf{Metadata}

\begin{itemize}
  \item Question ID: P5-PcRepWh\_P1-WNAdd2nd\_sonnet4\_Household Finance\_01
  \item Primary KC: PERCENTAGE | Representation and concept | expressing a part of a whole as a percentage
  \item Secondary KC: WHOLE NUMBERS | Addition | adding whole numbers
  \item Topic: Household finance such as income, utility bills, money, interest, savings, instalment, mortgage, financial planning etc.
  \item Grade: Primary 5
\end{itemize}

\textbf{Solution}

Step 1: Find the total monthly expenses by adding all the expense categories.
Total expenses = \textdollar480 + \textdollar320 + \textdollar200 = \textdollar1000

Step 2: Calculate what percentage of their income they spend on expenses.
Percentage of income spent = $\frac{\text{Total expenses}}{\text{Total income}} \times 100\%$
Percentage of income spent = $\frac{\textdollar1000}{\textdollar2500} \times 100\%$
Percentage of income spent = $\frac{1000}{2500} \times 100\%$
Percentage of income spent = $0.4 \times 100\% = 40\%$

Therefore, the Tan family's total monthly expenses are \textdollar1000, which represents 40% of their monthly income.

\section*{Question 29}
\textbf{Metadata}

\begin{itemize}
  \item Question ID: P5-PcRepWh\_P1-WNSub2nd\_sonnet4\_Household Finance\_01
  \item Primary KC: PERCENTAGE | Representation and concept | expressing a part of a whole as a percentage
  \item Secondary KC: WHOLE NUMBERS | Subtraction | subtracting whole numbers
  \item Topic: Household finance such as income, utility bills, money, interest, savings, instalment, mortgage, financial planning etc.
  \item Grade: Primary 5
\end{itemize}

\textbf{Solution}

Step 1: Find the amount saved by the Chen family.
Amount saved = Total budget - Amount spent
Amount saved = \textdollar3600 - \textdollar2880 = \textdollar720

Step 2: Express the amount saved as a percentage of the total budget.
Percentage saved = $\frac{\text{Amount saved}}{\text{Total budget}} \times 100\%$
Percentage saved = $\frac{720}{3600} \times 100\%$
Percentage saved = $\frac{1}{5} \times 100\%$
Percentage saved = $20\%$

Therefore, the Chen family saved $20\%$ of their monthly budget in January.

\section*{Question 30}
\textbf{Metadata}

\begin{itemize}
  \item Question ID: P5-RtFndT\_P2-DcCnvD2N\_sonnet4\_Household Finance\_01
  \item Primary KC: RATE | Finding total amount | finding total amount, given rate and number of units
  \item Secondary KC: DECIMALS | Conversion to smaller units | converting a measurement from a larger unit in decimal form to a smaller unit
  \item Topic: Household finance such as income, utility bills, money, interest, savings, instalment, mortgage, financial planning etc.
  \item Grade: Primary 5
\end{itemize}

\textbf{Solution}

Step 1: Convert the electricity usage from thousand kWh to kWh.
$2.8 \text{ thousand kWh} = 2.8 \times 1000 = 2800 \text{ kWh}$

Step 2: Find the total electricity bill using the rate.
Rate per kWh = \textdollar0.25
Total usage = $2800$ kWh
Total cost = Rate $\times$ Usage
Total cost = $\textdollar0.25 \times 2800 = \textdollar700$

Therefore, Mrs. Chen paid \textdollar700 for electricity last month.

\section*{Question 31}
\textbf{Metadata}

\begin{itemize}
  \item Question ID: P6-FrDivPN\_P2-FrCmp\_sonnet4\_Household Finance\_01
  \item Primary KC: FRACTIONS | Division | dividing a proper fraction by a whole number
  \item Secondary KC: FRACTIONS | Comparison and ordering | comparing and ordering fractions
  \item Topic: Household finance such as income, utility bills, money, interest, savings, instalment, mortgage, financial planning etc.
  \item Grade: Primary 6
\end{itemize}

\textbf{Solution}

Step 1: Find the amount Mrs. Tan withdraws from her savings.
Amount withdrawn = $\frac{3}{4} \times \textdollar240 = \textdollar180$

Step 2: Find how much each child receives per week.
Total amount for 3 children for 4 weeks = \textdollar180
Amount each child receives per week = $\frac{\textdollar180}{3 \times 4} = \frac{\textdollar180}{12} = \textdollar15$

Alternatively, we can think of it as dividing the fraction:
Each child gets $\frac{3}{4} \div 3 = \frac{3}{4} \times \frac{1}{3} = \frac{1}{4}$ of the total savings per week.
So each child receives $\frac{1}{4} \times \textdollar240 = \textdollar60$ over 4 weeks.
Therefore, each child receives $\textdollar60 \div 4 = \textdollar15$ per week.

Step 3: Find each child's total allowance over 4 weeks.
Each child's total allowance = $\textdollar15 \times 4 = \textdollar60$

Step 4: Find the remaining amount in Mrs. Tan's account.
Remaining amount = $\textdollar240 - \textdollar180 = \textdollar60$

Step 5: Compare each child's total allowance with the remaining amount.
Each child's total allowance = \textdollar60
Remaining amount in account = \textdollar60
Therefore, each child's total 4-week allowance is equal to the money remaining in Mrs. Tan's account.

Answer: Each child receives \textdollar15 per week. Each child's total 4-week allowance is equal to the money remaining in Mrs. Tan's account (both are \textdollar60).

\section*{Question 32}
\textbf{Metadata}

\begin{itemize}
  \item Question ID: P6-FrDivPN\_P2-FrSub2nd\_sonnet4\_Household Finance\_01
  \item Primary KC: FRACTIONS | Division | dividing a proper fraction by a whole number
  \item Secondary KC: FRACTIONS | Subtraction | subtracting fractions
  \item Topic: Household finance such as income, utility bills, money, interest, savings, instalment, mortgage, financial planning etc.
  \item Grade: Primary 6
\end{itemize}

\textbf{Solution}

Step 1: Find the fraction of monthly income spent on rent and groceries.
Since the family spends $\frac{3}{4}$ of their income on various expenses, and $\frac{1}{3}$ remains for other expenses, the fraction spent on rent and groceries is:
$\frac{3}{4} - \frac{1}{3}$

To subtract these fractions, find a common denominator:
$\frac{3}{4} = \frac{9}{12}$ and $\frac{1}{3} = \frac{4}{12}$

$\frac{9}{12} - \frac{4}{12} = \frac{5}{12}$

So $\frac{5}{12}$ of their monthly income is spent on rent and groceries.

Step 2: Find the fraction allocated to each of the 6 categories.
The remaining $\frac{1}{3}$ of their monthly income needs to be divided equally among 6 categories.

This means we need to calculate: $\frac{1}{3} \div 6$

To divide a fraction by a whole number, we multiply the fraction by the reciprocal of the whole number:
$\frac{1}{3} \div 6 = \frac{1}{3} \times \frac{1}{6} = \frac{1}{18}$

Therefore, $\frac{1}{18}$ of their monthly income is allocated to each category.

\section*{Question 33}
\textbf{Metadata}

\begin{itemize}
  \item Question ID: P6-FrDivPN\_P5-FrMul2nd\_sonnet4\_Household Finance\_01
  \item Primary KC: FRACTIONS | Division | dividing a proper fraction by a whole number
  \item Secondary KC: FRACTIONS | Multiplication | fraction multiplication
  \item Topic: Household finance such as income, utility bills, money, interest, savings, instalment, mortgage, financial planning etc.
  \item Grade: Primary 6
\end{itemize}

\textbf{Solution}

Step 1: Find the amount of reduction in the electricity bill.
Reduction = $\frac{2}{5} \times \textdollar240 = \frac{2 \times 240}{5} = \frac{480}{5} = \textdollar96$

Step 2: Find the new reduced bill amount.
New bill amount = Original bill - Reduction
New bill amount = $\textdollar240 - \textdollar96 = \textdollar144$

Step 3: Divide the new bill amount equally among 4 family members.
Amount per person = $\frac{\textdollar144}{4} = \textdollar36$

Therefore, each family member will pay \textdollar36.

\section*{Question 34}
\textbf{Metadata}

\begin{itemize}
  \item Question ID: P6-FrDivPP\_P2-FrAdd2nd\_sonnet4\_Household Finance\_01
  \item Primary KC: FRACTIONS | Division | dividing a whole number/proper fraction by a proper fraction
  \item Secondary KC: FRACTIONS | Addition | adding fractions
  \item Topic: Household finance such as income, utility bills, money, interest, savings, instalment, mortgage, financial planning etc.
  \item Grade: Primary 6
\end{itemize}

\textbf{Solution}

Step 1: Find the total monthly utility expenses.
Since the electricity bill of \textdollar240 represents $\frac{3}{5}$ of the total utility expenses, we need to divide \textdollar240 by $\frac{3}{5}$.

Total utility expenses = $240 \div \frac{3}{5}$
$= 240 \times \frac{5}{3}$
$= \frac{240 \times 5}{3}$
$= \frac{1200}{3}$
$= 400$

So the total monthly utility expenses are \textdollar400.

Step 2: Find the amount for water bill.
Water bill = $\frac{1}{4}$ of total utility expenses
$= \frac{1}{4} \times 400$
$= 100$

So the water bill is \textdollar100.

Step 3: Find the amount for gas bill.
We know that electricity bill + water bill + gas bill = total utility expenses
Gas bill = Total utility expenses - electricity bill - water bill
$= 400 - 240 - 100$
$= 60$

So the gas bill is \textdollar60.

Step 4: Find the combined amount for water and gas bills.
Water and gas bills combined = Water bill + gas bill
$= 100 + 60$
$= 160$

Therefore, Sarah's family pays \textdollar160 for their water and gas bills combined each month.

\section*{Question 35}
\textbf{Metadata}

\begin{itemize}
  \item Question ID: P6-FrDivPP\_P5-FrMul2nd\_sonnet4\_Household Finance\_01
  \item Primary KC: FRACTIONS | Division | dividing a whole number/proper fraction by a proper fraction
  \item Secondary KC: FRACTIONS | Multiplication | fraction multiplication
  \item Topic: Household finance such as income, utility bills, money, interest, savings, instalment, mortgage, financial planning etc.
  \item Grade: Primary 6
\end{itemize}

\textbf{Solution}

Step 1: Find Mrs. Tan's monthly savings.
Monthly savings = $\frac{1}{6} \times \textdollar4800 = \textdollar800$

Step 2: Find the amount deposited in each investment account.
Since the savings are divided equally among 3 accounts:
Amount per account = $\textdollar800 \div 3 = \frac{\textdollar800}{1} \div \frac{3}{1} = \frac{\textdollar800}{1} \times \frac{1}{3} = \frac{\textdollar800}{3}$

Step 3: Find the interest earned from one investment account.
Interest from one account = $\frac{3}{4} \times \frac{\textdollar800}{3}$
Interest = $\frac{3 \times \textdollar800}{4 \times 3} = \frac{\textdollar2400}{12} = \textdollar200$

Therefore, Mrs. Tan will earn \textdollar200 interest from one investment account after one year.

\section*{Question 36}
\textbf{Metadata}

\begin{itemize}
  \item Question ID: P6-PcFndWN\_P1-WNMul2nd\_sonnet4\_Household Finance\_01
  \item Primary KC: PERCENTAGE | Finding the whole | finding the whole given a part and the percentage
  \item Secondary KC: WHOLE NUMBERS | Multiplication | multiplying whole numbers
  \item Topic: Household finance such as income, utility bills, money, interest, savings, instalment, mortgage, financial planning etc.
  \item Grade: Primary 6
\end{itemize}

\textbf{Solution}

Step 1: Find the total yearly electricity bill.
Given that \textdollar84 represents 30\% of the total yearly bill:
30\% of total bill = \textdollar84
$\frac{30}{100} \times \text{total bill} = 84$
$\text{total bill} = 84 \div \frac{30}{100} = 84 \times \frac{100}{30} = 84 \times \frac{10}{3} = 280$

The total yearly electricity bill is \textdollar280.

Step 2: Find the monthly payment amount.
Since the yearly bill is \textdollar280 and it's paid over 12 months:
Monthly payment = $280 \div 12 = \textdollar23.33$ (rounded to nearest cent)

Step 3: Find the total amount paid over 12 months.
Total amount paid over 12 months = Monthly payment $\times$ 12
Total amount paid = $23.33 \times 12 = \textdollar280$

Therefore, Mrs. Tan pays \textdollar280 in total for her electricity bill over 12 months.

\section*{Question 37}
\textbf{Metadata}

\begin{itemize}
  \item Question ID: P6-PcFndWN\_P1-WNDiv2nd\_sonnet4\_Household Finance\_01
  \item Primary KC: PERCENTAGE | Finding the whole | finding the whole given a part and the percentage
  \item Secondary KC: WHOLE NUMBERS | Division | dividing whole numbers
  \item Topic: Household finance such as income, utility bills, money, interest, savings, instalment, mortgage, financial planning etc.
  \item Grade: Primary 6
\end{itemize}

\textbf{Solution}

Step 1: Find the total electricity bill.
If \textdollar168 represents 30\% of the total bill, then:
30\% of total bill = \textdollar168
$\frac{30}{100} \times \text{total bill} = 168$
$\text{total bill} = 168 \div \frac{30}{100} = 168 \times \frac{100}{30} = 168 \times \frac{10}{3} = 560$

The total electricity bill is \textdollar560.

Step 2: Find the remaining amount to be paid.
Remaining amount = Total bill - Amount already paid
Remaining amount = \textdollar560 - \textdollar168 = \textdollar392

Step 3: Find the amount for each remaining instalment.
The remaining \textdollar392 will be paid in 4 equal instalments.
Amount per instalment = \textdollar392 $\div$ 4 = \textdollar98

Therefore, Mrs. Lim needs to pay \textdollar98 for each of the remaining instalments.

\section*{Question 38}
\textbf{Metadata}

\begin{itemize}
  \item Question ID: P6-PcFndChg\_P1-WNAdd2nd\_sonnet4\_Household Finance\_01
  \item Primary KC: PERCENTAGE | Finding change | finding percentage increase/decrease
  \item Secondary KC: WHOLE NUMBERS | Addition | adding whole numbers
  \item Topic: Household finance such as income, utility bills, money, interest, savings, instalment, mortgage, financial planning etc.
  \item Grade: Primary 6
\end{itemize}

\textbf{Solution}

**Step 1: Find the percentage decrease from January to February**

January bill = \textdollar180
February bill = \textdollar135

Decrease in bill = \textdollar180 - \textdollar135 = \textdollar45

Percentage decrease = $\frac{\text{Decrease}}{\text{Original amount}} \times 100\%$

Percentage decrease = $\frac{45}{180} \times 100\%$

Percentage decrease = $\frac{1}{4} \times 100\% = 25\%$

**Step 2: Find the percentage increase from February to March**

February bill = \textdollar135
March bill = \textdollar189

Increase in bill = \textdollar189 - \textdollar135 = \textdollar54

Percentage increase = $\frac{\text{Increase}}{\text{Original amount}} \times 100\%$

Percentage increase = $\frac{54}{135} \times 100\%$

Percentage increase = $\frac{2}{5} \times 100\% = 40\%$

**Answer:**
The percentage decrease from January to February is $25\%$.
The percentage increase from February to March is $40\%$.

\section*{Question 39}
\textbf{Metadata}

\begin{itemize}
  \item Question ID: P6-RoFndDvqWN\_P1-WNSub2nd\_sonnet4\_Household Finance\_01
  \item Primary KC: RATIO | Finding divided quantities | dividing a given quantity in a given ratio
  \item Secondary KC: WHOLE NUMBERS | Subtraction | subtracting whole numbers
  \item Topic: Household finance such as income, utility bills, money, interest, savings, instalment, mortgage, financial planning etc.
  \item Grade: Primary 6
\end{itemize}

\textbf{Solution}

Step 1: Find the total ratio parts.
Total ratio parts = 5 + 3 + 2 = 10 parts

Step 2: Find the value of each ratio part.
Value of each part = \textdollar4200 ÷ 10 = \textdollar420

Step 3: Find the amount allocated for food.
Food = 5 parts = 5 × \textdollar420 = \textdollar2100

Step 4: Find the amount allocated for utilities.
Utilities = 3 parts = 3 × \textdollar420 = \textdollar1260

Step 5: Find the total amount spent on food and utilities.
Total spent = \textdollar2100 + \textdollar1260 = \textdollar3360

Step 6: Find the amount left for savings.
Amount left for savings = \textdollar4200 - \textdollar3360 = \textdollar840

Therefore, the Tan family has \textdollar840 left for savings.

\section*{Question 40}
\textbf{Metadata}

\begin{itemize}
  \item Question ID: P6-RoFndRoWN\_P1-WNMul2nd\_sonnet4\_Household Finance\_01
  \item Primary KC: RATIO | Finding ratio | finding the ratio of two or three given whole numbers
  \item Secondary KC: WHOLE NUMBERS | Multiplication | multiplying whole numbers
  \item Topic: Household finance such as income, utility bills, money, interest, savings, instalment, mortgage, financial planning etc.
  \item Grade: Primary 6
\end{itemize}

\textbf{Solution}

Step 1: Calculate February expenses for each category by multiplying January amounts by 3.

Utilities in February: $180 \times 3 = \textdollar540$

Groceries in February: $420 \times 3 = \textdollar840$

Transportation in February: $240 \times 3 = \textdollar720$

Step 2: Find the ratio of utilities to groceries to transportation for February.

Ratio = $540 : 840 : 720$

Step 3: Simplify the ratio by finding the highest common factor.

Find HCF of 540, 840, and 720:
$540 = 2^2 \times 3^3 \times 5$
$840 = 2^3 \times 3 \times 5 \times 7$
$720 = 2^4 \times 3^2 \times 5$

HCF = $2^2 \times 3 \times 5 = 60$

Divide each term by 60:
$540 \div 60 = 9$
$840 \div 60 = 14$
$720 \div 60 = 12$

Therefore, the ratio of utilities to groceries to transportation expenses for February is $9 : 14 : 12$.

\section*{Question 41}
\textbf{Metadata}

\begin{itemize}
  \item Question ID: P6-RoFndRoWN\_P6-RoSmpWN\_sonnet4\_Household Finance\_01
  \item Primary KC: RATIO | Finding ratio | finding the ratio of two or three given whole numbers
  \item Secondary KC: RATIO | Simplifying | expressing a ratio in its simplest form
  \item Topic: Household finance such as income, utility bills, money, interest, savings, instalment, mortgage, financial planning etc.
  \item Grade: Primary 6
\end{itemize}

\textbf{Solution}

(a) The utilities expenses were \textdollar240, groceries expenses were \textdollar360, and transportation expenses were \textdollar120.

The ratio of utilities : groceries : transportation = $240 : 360 : 120$

(b) To express the ratio in its simplest form, I need to find the highest common factor (HCF) of 240, 360, and 120.

Factors of 240: $240 = 2^4 \times 3 \times 5$
Factors of 360: $360 = 2^3 \times 3^2 \times 5$
Factors of 120: $120 = 2^3 \times 3 \times 5$

HCF of 240, 360, and 120 = $2^3 \times 3 \times 5 = 120$

Dividing each term by 120:
$240 \div 120 = 2$
$360 \div 120 = 3$
$120 \div 120 = 1$

Therefore, the ratio in its simplest form is $2 : 3 : 1$.

This means that for every \textdollar2 spent on utilities, the family spends \textdollar3 on groceries and \textdollar1 on transportation.

\section*{Question 42}
\textbf{Metadata}

\begin{itemize}
  \item Question ID: P6-RoFndTmWN\_P1-WNSub2nd\_sonnet4\_Household Finance\_01
  \item Primary KC: RATIO | Finding a missing term | finding the missing term in a pair of equivalent ratios
  \item Secondary KC: WHOLE NUMBERS | Subtraction | subtracting whole numbers
  \item Topic: Household finance such as income, utility bills, money, interest, savings, instalment, mortgage, financial planning etc.
  \item Grade: Primary 6
\end{itemize}

\textbf{Solution}

First, I need to find what Mrs. Tan's total monthly income should be to save \textdollar1200.

Given ratio of savings to expenses = $3:7$
This means the ratio of savings to total income = $3:(3+7) = 3:10$

Using the ratio to find the required total income:
If savings : total income = $3:10$
Then \textdollar1200 : total income = $3:10$

Let the required total income be $x$.
$\frac{3}{10} = \frac{1200}{x}$

Cross multiply: $3x = 1200 \times 10$
$3x = 12000$
$x = 4000$

So Mrs. Tan needs a total monthly income of \textdollar4000 to save \textdollar1200.

Since her current monthly income is \textdollar400 less than what she needs:
Current monthly income = \textdollar4000 $-$ \textdollar400 = \textdollar3600

Therefore, Mrs. Tan's current monthly income is \textdollar3600.

\section*{Question 43}
\textbf{Metadata}

\begin{itemize}
  \item Question ID: O1-RoRepFr\_P2-FrAdd2nd\_sonnet4\_Household Finance\_01
  \item Primary KC: RATIO | Representation and concept | ratios involving fractions
  \item Secondary KC: FRACTIONS | Addition | adding fractions
  \item Topic: Household finance such as income, utility bills, money, interest, savings, instalment, mortgage, financial planning etc.
  \item Grade: Secondary O-level 1
\end{itemize}

\textbf{Solution}

Given ratio: utilities : groceries : savings = $\frac{2}{3} : \frac{3}{4} : \frac{1}{2}$

First, I need to convert this ratio to whole numbers by finding the LCM of denominators 3, 4, and 2.
LCM(3, 4, 2) = 12

Multiplying each fraction by 12:
$\frac{2}{3} \times 12 = 8$
$\frac{3}{4} \times 12 = 9$
$\frac{1}{2} \times 12 = 6$

So the ratio becomes utilities : groceries : savings = 8 : 9 : 6

Given that utilities = \textdollar900
Since utilities corresponds to 8 parts in the ratio:
1 part = $\frac{\textdollar900}{8} = \textdollar112.50$

(a) Groceries = 9 parts = $9 \times \textdollar112.50 = \textdollar1012.50$

(b) Savings = 6 parts = $6 \times \textdollar112.50 = \textdollar675$

(c) Total amount for groceries and savings combined:
$\textdollar1012.50 + \textdollar675 = \textdollar1687.50$

Therefore:
(a) The family spends \textdollar1012.50 on groceries each month.
(b) The family saves \textdollar675 each month.
(c) The total amount spent on groceries and savings combined is \textdollar1687.50.

\section*{Question 44}
\textbf{Metadata}

\begin{itemize}
  \item Question ID: O1-RoRepFr\_P5-FrMul2nd\_sonnet4\_Household Finance\_01
  \item Primary KC: RATIO | Representation and concept | ratios involving fractions
  \item Secondary KC: FRACTIONS | Multiplication | fraction multiplication
  \item Topic: Household finance such as income, utility bills, money, interest, savings, instalment, mortgage, financial planning etc.
  \item Grade: Secondary O-level 1
\end{itemize}

\textbf{Solution}

**Solution:**

**(a) Finding the allocation for each category**

Given ratio: $3\frac{1}{2} : 2\frac{1}{4} : 1\frac{1}{3}$

First, convert the mixed numbers to improper fractions:
- $3\frac{1}{2} = \frac{7}{2}$
- $2\frac{1}{4} = \frac{9}{4}$
- $1\frac{1}{3} = \frac{4}{3}$

So the ratio becomes $\frac{7}{2} : \frac{9}{4} : \frac{4}{3}$

To work with this ratio, find the LCM of denominators 2, 4, and 3:
LCM(2, 4, 3) = 12

Multiply each fraction by 12:
- $\frac{7}{2} \times 12 = 42$
- $\frac{9}{4} \times 12 = 27$
- $\frac{4}{3} \times 12 = 16$

So the simplified ratio is $42 : 27 : 16$

Total parts = $42 + 27 + 16 = 85$ parts

Total income = \textdollar4800

Value of each part = $\frac{\textdollar4800}{85} = \textdollar\frac{4800}{85} = \textdollar\frac{960}{17}$

**Housing allocation:** $42 \times \frac{960}{17} = \frac{40320}{17} = \textdollar2400$

**Food allocation:** $27 \times \frac{960}{17} = \frac{25920}{17} = \textdollar1525.88$ (rounded to nearest cent)

**Savings allocation:** $16 \times \frac{960}{17} = \frac{15360}{17} = \textdollar904.12$ (rounded to nearest cent)

**(b) Finding the new food budget**

Original food allocation = \textdollar1525.88

Increase = $\frac{1}{6} \times \textdollar1525.88 = \textdollar254.31$

New food budget = \textdollar1525.88 + \textdollar254.31 = \textdollar1780.19

\section*{Question 45}
\textbf{Metadata}

\begin{itemize}
  \item Question ID: O1-RoRepFr\_O1-RoSmpFr\_sonnet4\_Household Finance\_01
  \item Primary KC: RATIO | Representation and concept | ratios involving fractions
  \item Secondary KC: RATIO | Simplifying | converting a ratio involving fractions to its simplest form
  \item Topic: Household finance such as income, utility bills, money, interest, savings, instalment, mortgage, financial planning etc.
  \item Grade: Secondary O-level 1
\end{itemize}

\textbf{Solution}

Let the total monthly income be represented as 1 unit.

Step 1: Find the fraction allocated to each category.
Savings = $\frac{1}{4}$ of income
Household expenses = $\frac{2}{5}$ of income
Entertainment = Remaining amount = $1 - \frac{1}{4} - \frac{2}{5}$

Step 2: Calculate the entertainment fraction.
To subtract the fractions, find a common denominator for $\frac{1}{4}$ and $\frac{2}{5}$.
LCM of 4 and 5 = 20
$\frac{1}{4} = \frac{5}{20}$ and $\frac{2}{5} = \frac{8}{20}$

Entertainment = $1 - \frac{5}{20} - \frac{8}{20} = \frac{20}{20} - \frac{5}{20} - \frac{8}{20} = \frac{7}{20}$

Step 3: Write the ratio.
Savings : Household expenses : Entertainment = $\frac{1}{4} : \frac{2}{5} : \frac{7}{20}$

Step 4: Convert to simplest form.
To eliminate fractions, multiply all terms by the LCM of denominators 4, 5, and 20.
LCM of 4, 5, and 20 = 20

$\frac{1}{4} \times 20 : \frac{2}{5} \times 20 : \frac{7}{20} \times 20 = 5 : 8 : 7$

Step 5: Check if the ratio can be simplified further.
GCD of 5, 8, and 7 = 1 (since these numbers share no common factors other than 1)

Therefore, the ratio of savings to household expenses to entertainment in its simplest form is $5 : 8 : 7$.

\section*{Question 46}
\textbf{Metadata}

\begin{itemize}
  \item Question ID: O1-RoRepDc\_P4-DcAdd2nd\_sonnet4\_Household Finance\_01
  \item Primary KC: RATIO | Representation and concept | ratios involving decimals
  \item Secondary KC: DECIMALS | Addition | adding decimals
  \item Topic: Household finance such as income, utility bills, money, interest, savings, instalment, mortgage, financial planning etc.
  \item Grade: Secondary O-level 1
\end{itemize}

\textbf{Solution}

Step 1: Find the amount allocated to expenses and savings.
The ratio of expenses to savings is $3.5 : 1.5$
Total ratio parts = $3.5 + 1.5 = 5.0$
Amount for expenses = $\frac{3.5}{5.0} \times \textdollar4200 = \textdollar2940$
Amount for savings = $\frac{1.5}{5.0} \times \textdollar4200 = \textdollar1260$

Step 2: Find the original amounts for rent, utilities, and groceries.
The ratio of rent : utilities : groceries is $2.8 : 0.7 : 1.5$
Total ratio parts = $2.8 + 0.7 + 1.5 = 5.0$
Rent = $\frac{2.8}{5.0} \times \textdollar2940 = \textdollar1646.40$
Utilities = $\frac{0.7}{5.0} \times \textdollar2940 = \textdollar411.60$
Groceries = $\frac{1.5}{5.0} \times \textdollar2940 = \textdollar882.00$

Step 3: Calculate the new utility bill after the increase.
New utility bill = $\textdollar411.60 + \textdollar45.50 = \textdollar457.10$

Step 4: Calculate the new total monthly expenses.
New total expenses = Rent + New utilities + Groceries
New total expenses = $\textdollar1646.40 + \textdollar457.10 + \textdollar882.00 = \textdollar2985.50$

Therefore, Mrs. Lim's new total monthly expenses is \textdollar2985.50.

\section*{Question 47}
\textbf{Metadata}

\begin{itemize}
  \item Question ID: O1-RoRepDc\_O1-RoSmpDc\_sonnet4\_Household Finance\_01
  \item Primary KC: RATIO | Representation and concept | ratios involving decimals
  \item Secondary KC: RATIO | Simplifying | converting a ratio involving decimals to its simplest form
  \item Topic: Household finance such as income, utility bills, money, interest, savings, instalment, mortgage, financial planning etc.
  \item Grade: Secondary O-level 1
\end{itemize}

\textbf{Solution}

To find the ratio of grocery expenses to utility expenses to mortgage payments, I need to write the amounts as a ratio and then simplify it.

Step 1: Write the ratio using the given amounts.
Grocery expenses : Utility expenses : Mortgage payments = \textdollar1200.50 : \textdollar800.25 : \textdollar2400.75

Step 2: Remove the dollar signs and work with the decimal numbers.
$1200.50 : 800.25 : 2400.75$

Step 3: Convert the decimals to whole numbers by multiplying all terms by 100 (since all numbers have 2 decimal places).
$1200.50 \times 100 : 800.25 \times 100 : 2400.75 \times 100$
$= 120050 : 80025 : 240075$

Step 4: Find the greatest common divisor (GCD) of 120050, 80025, and 240075.
First, I'll find the prime factorization or use the Euclidean algorithm.
$120050 = 2 \times 5^2 \times 7^2 \times 7^2 = 2 \times 25 \times 49 \times 49$
Let me find GCD systematically:
$120050 \div 25 = 4802$
$80025 \div 25 = 3201$
$240075 \div 25 = 9603$

So we have $4802 : 3201 : 9603$

Checking if these can be simplified further:
$4802 \div 7 = 686$
$3201 \div 7 = 457.3...$ (not divisible)

Let me recalculate more carefully:
$\gcd(120050, 80025, 240075) = 25$

Step 5: Divide all terms by their GCD.
$120050 \div 25 : 80025 \div 25 : 240075 \div 25$
$= 4802 : 3201 : 9603$

Step 6: Check if this can be simplified further.
Since $\gcd(4802, 3201, 9603) = 1$, the ratio cannot be simplified further.

Therefore, the ratio of grocery expenses to utility expenses to mortgage payments in its simplest form is $4802 : 3201 : 9603$.

\section*{Question 48}
\textbf{Metadata}

\begin{itemize}
  \item Question ID: O1-PcRep2q\_O1-PcCnv2Fr\_sonnet4\_Household Finance\_01
  \item Primary KC: PERCENTAGE | Representation and concept | comparing two quantities by percentage
  \item Secondary KC: PERCENTAGE | Conversion to fraction | expressing percentage as a fraction
  \item Topic: Household finance such as income, utility bills, money, interest, savings, instalment, mortgage, financial planning etc.
  \item Grade: Secondary O-level 1
\end{itemize}

\textbf{Solution}

Step 1: Calculate the percentage of income spent on rent.
Rent percentage $= \frac{\text{Rent}}{\text{Total Income}} \times 100\%$
Rent percentage $= \frac{1440}{4800} \times 100\%$
Rent percentage $= 0.3 \times 100\% = 30\%$

Step 2: Calculate the percentage of income spent on food.
Food percentage $= \frac{\text{Food expenses}}{\text{Total Income}} \times 100\%$
Food percentage $= \frac{960}{4800} \times 100\%$
Food percentage $= 0.2 \times 100\% = 20\%$

Step 3: Express the rent percentage as a fraction in simplest form.
$30\% = \frac{30}{100} = \frac{3}{10}$

Step 4: Express the food percentage as a fraction in simplest form.
$20\% = \frac{20}{100} = \frac{1}{5}$

Therefore, $30\%$ of their total income is spent on rent, which equals $\frac{3}{10}$ as a fraction. $20\%$ of their total income is spent on food, which equals $\frac{1}{5}$ as a fraction.

\section*{Question 49}
\textbf{Metadata}

\begin{itemize}
  \item Question ID: O1-PcFndRslt\_P1-WNAdd2nd\_sonnet4\_Household Finance\_01
  \item Primary KC: PERCENTAGE | Finding result after change | increasing/decreasing a quantity by a given percentage
  \item Secondary KC: WHOLE NUMBERS | Addition | adding whole numbers
  \item Topic: Household finance such as income, utility bills, money, interest, savings, instalment, mortgage, financial planning etc.
  \item Grade: Secondary O-level 1
\end{itemize}

\textbf{Solution}

Step 1: Calculate the total January expenses by adding all three categories.
Utilities: \textdollar180
Groceries: \textdollar450
Transportation: \textdollar320
Total January expenses = \textdollar180 + \textdollar450 + \textdollar320 = \textdollar950

Step 2: Calculate the $15\%$ increase in expenses.
Increase = $15\% \times \textdollar950$
Increase = $0.15 \times \textdollar950 = \textdollar142.50$

Step 3: Calculate the total February budget.
February budget = January expenses + Increase
February budget = \textdollar950 + \textdollar142.50 = \textdollar1092.50

Therefore, the Tan family should budget \textdollar1092.50 for these three expense categories in February.

\section*{Question 50}
\textbf{Metadata}

\begin{itemize}
  \item Question ID: O1-PcFndRslt\_P1-WNMul2nd\_sonnet4\_Household Finance\_01
  \item Primary KC: PERCENTAGE | Finding result after change | increasing/decreasing a quantity by a given percentage
  \item Secondary KC: WHOLE NUMBERS | Multiplication | multiplying whole numbers
  \item Topic: Household finance such as income, utility bills, money, interest, savings, instalment, mortgage, financial planning etc.
  \item Grade: Secondary O-level 1
\end{itemize}

\textbf{Solution}

Step 1: Calculate the bill after reducing consumption by $15\%$.

Reduction in bill = $15\% \times \textdollar180 = \frac{15}{100} \times 180 = \textdollar27$

Bill after reduction = $\textdollar180 - \textdollar27 = \textdollar153$

Step 2: Calculate the final bill after the $8\%$ rate increase.

Increase due to rate hike = $8\% \times \textdollar153 = \frac{8}{100} \times 153 = \textdollar12.24$

Final monthly bill = $\textdollar153 + \textdollar12.24 = \textdollar165.24$

Step 3: Calculate the total bill for $6$ months.

Total bill for $6$ months = $\textdollar165.24 \times 6 = \textdollar991.44$

Therefore, Sarah's family will pay a total of \textdollar991.44 for their electricity bills over the next $6$ months.

\section*{Question 51}
\textbf{Metadata}

\begin{itemize}
  \item Question ID: O1-PcRepRvs\_O1-PcCnv2Fr\_sonnet4\_Household Finance\_01
  \item Primary KC: PERCENTAGE | Representation and concept | reverse percentages
  \item Secondary KC: PERCENTAGE | Conversion to fraction | expressing percentage as a fraction
  \item Topic: Household finance such as income, utility bills, money, interest, savings, instalment, mortgage, financial planning etc.
  \item Grade: Secondary O-level 1
\end{itemize}

\textbf{Solution}

**Step 1: Find Mrs. Tan's total monthly income using reverse percentages**

Given that \textdollar480 represents 30\% of her total monthly income:

Let the total monthly income be $x$

$30\% \text{ of } x = \textdollar480$

$\frac{30}{100} \times x = 480$

$0.3x = 480$

$x = \frac{480}{0.3} = \textdollar1600$

Therefore, Mrs. Tan's total monthly income is \textdollar1600.

**Step 2: Express the percentage for utilities and housing as a fraction**

Mrs. Tan allocates 25\% of her income for utilities and housing costs.

To convert 25\% to a fraction:

$25\% = \frac{25}{100} = \frac{1}{4}$

**Answer:**
- Mrs. Tan's total monthly income is \textdollar1600
- She spends $\frac{1}{4}$ of her income on utilities and housing costs combined

\section*{Question 52}
\textbf{Metadata}

\begin{itemize}
  \item Question ID: O1-AgRepExSq\_O1-AgEvlEx\_sonnet4\_Household Finance\_01
  \item Primary KC: ALGEBRA | Representation and concept | translation of simple real-world situations into quadratic algebraic expressions
  \item Secondary KC: ALGEBRA | Evaluation | evaluation of algebraic expressions and formulae
  \item Topic: Household finance such as income, utility bills, money, interest, savings, instalment, mortgage, financial planning etc.
  \item Grade: Secondary O-level 1
\end{itemize}

\textbf{Solution}

(a) To find the quadratic expression for Sarah's total savings after $n$ months:

Initial amount: \textdollar200

Amount saved each month: \textdollar50 per month for $n$ months = \textdollar$50n$

Bank fees lost: \textdollar2 in month 1, \textdollar4 in month 2, ..., \textdollar$2n$ in month $n$
Total fees = \textdollar$(2 + 4 + 6 + ... + 2n) = \textdollar$2(1 + 2 + 3 + ... + n) = \textdollar$2 \cdot \frac{n(n+1)}{2} = \textdollar$n(n+1) = \textdollar$(n^2 + n)$

Total amount after $n$ months = Initial amount + Monthly savings - Total fees
Total amount = $200 + 50n - (n^2 + n) = 200 + 50n - n^2 - n = 200 + 49n - n^2$

Therefore, the quadratic expression is $200 + 49n - n^2$ dollars.

(b) To find the amount after 6 months, substitute $n = 6$ into the expression:

Total amount = $200 + 49(6) - (6)^2$
$= 200 + 294 - 36$
$= 494 - 36$
$= 458$

Therefore, Sarah will have \textdollar458 after 6 months.

\section*{Question 53}
\textbf{Metadata}

\begin{itemize}
  \item Question ID: O1-AgRepnth\_O1-AgEvlEx\_sonnet4\_Household Finance\_01
  \item Primary KC: ALGEBRA | Representation and concept | recognising and representing patterns/relationships by finding an algebraic expression for the nth term
  \item Secondary KC: ALGEBRA | Evaluation | evaluation of algebraic expressions and formulae
  \item Topic: Household finance such as income, utility bills, money, interest, savings, instalment, mortgage, financial planning etc.
  \item Grade: Secondary O-level 1
\end{itemize}

\textbf{Solution}

(a) Let me identify the pattern in Sarah's monthly income:
- Month 1 (January): \textdollar800
- Month 2 (February): \textdollar950
- Month 3 (March): \textdollar1100
- Month 4 (April): \textdollar1250

The differences between consecutive months are:
- \textdollar950 - \textdollar800 = \textdollar150
- \textdollar1100 - \textdollar950 = \textdollar150
- \textdollar1250 - \textdollar1100 = \textdollar150

This is an arithmetic sequence with first term $a = 800$ and common difference $d = 150$.

Using the formula for the $n$th term of an arithmetic sequence: $T_n = a + (n-1)d$

Sarah's income in the $n$th month = $800 + (n-1) \times 150$
= $800 + 150n - 150$
= $650 + 150n$

Therefore, Sarah's income in the $n$th month is $\textdollar(650 + 150n)$.

(b) Sarah saves 20% of her income each month.
Her monthly savings in the $n$th month = $0.2 \times (650 + 150n)$
= $0.2 \times 650 + 0.2 \times 150n$
= $130 + 30n$

Therefore, Sarah's savings in the $n$th month is $\textdollar(130 + 30n)$.

(c) To find Sarah's savings in the 8th month (August), I substitute $n = 8$ into the expression:
Savings in 8th month = $130 + 30(8)$
= $130 + 240$
= $370$

Therefore, Sarah will save \textdollar370 in August.

\section*{Question 54}
\textbf{Metadata}

\begin{itemize}
  \item Question ID: O2-RoRepDP\_P1-WNDiv2nd\_sonnet4\_Household Finance\_01
  \item Primary KC: RATIO | Representation and concept | direct proportion
  \item Secondary KC: WHOLE NUMBERS | Division | dividing whole numbers
  \item Topic: Household finance such as income, utility bills, money, interest, savings, instalment, mortgage, financial planning etc.
  \item Grade: Secondary O-level 2
\end{itemize}

\textbf{Solution}

**Step 1: Find the spending on each category with the original budget**

The ratio for food : utilities : entertainment is $3:2:1$

Total ratio parts = $3 + 2 + 1 = 6$ parts

Total monthly expenses = \textdollar1800

Value of 1 part = \textdollar1800 ÷ 6 = \textdollar300

Therefore:
- Food expenses = $3 × \textdollar300 = \textdollar900$
- Utilities expenses = $2 × \textdollar300 = \textdollar600$
- Entertainment expenses = $1 × \textdollar300 = \textdollar300$

**Step 2: Find the new budget after reducing food expenses**

New food expenses = \textdollar900 - \textdollar150 = \textdollar750

Since the ratio $3:2:1$ is maintained, and food still represents 3 parts:

Value of 1 part in the new budget = \textdollar750 ÷ 3 = \textdollar250

Therefore:
- New utilities expenses = $2 × \textdollar250 = \textdollar500$
- New entertainment expenses = $1 × \textdollar250 = \textdollar250$

**Answer:**
Original monthly spending: Food = \textdollar900, Utilities = \textdollar600, Entertainment = \textdollar300

New monthly budget: Utilities = \textdollar500, Entertainment = \textdollar250

\section*{Question 55}
\textbf{Metadata}

\begin{itemize}
  \item Question ID: O2-AgSlvIneq\_O2-AgRepIneq\_sonnet4\_Household Finance\_01
  \item Primary KC: ALGEBRA | Solving | solving simple linear inequalities with one variable
  \item Secondary KC: ALGEBRA | Representation and concept | translation of simple real-world situations to simple linear inequalities with one variable
  \item Topic: Household finance such as income, utility bills, money, interest, savings, instalment, mortgage, financial planning etc.
  \item Grade: Secondary O-level 2
\end{itemize}

\textbf{Solution}

Let $x$ represent the monthly cost of internet service in dollars.

Step 1: Translate the real-world situation into a linear inequality.
Sarah's total monthly utility expenses = Electricity + Water + Internet
Total expenses = $45 + 25 + x = 70 + x$

Since Sarah wants to keep her total expenses below \textdollar120, we write:
$70 + x < 120$

Step 2: Solve the linear inequality.
$70 + x < 120$
$x < 120 - 70$
$x < 50$

Step 3: Interpret the solution.
Sarah can spend at most \textdollar49.99 on internet service to keep her total monthly utility expenses below \textdollar120.

Therefore, the maximum amount she can spend on internet service is any amount less than \textdollar50.

\section*{Question 56}
\textbf{Metadata}

\begin{itemize}
  \item Question ID: O2-AgSlvSq1v\_O1-AgRepEq\_sonnet4\_Household Finance\_01
  \item Primary KC: ALGEBRA | Solving | solving quadratic equations in one variable
  \item Secondary KC: ALGEBRA | Representation and concept | translation of simple real-world situations to equations
  \item Topic: Household finance such as income, utility bills, money, interest, savings, instalment, mortgage, financial planning etc.
  \item Grade: Secondary O-level 2
\end{itemize}

\textbf{Solution}

Let $x$ be the amount Sarah allocates for monthly savings (in dollars).

From the problem, we can identify the following relationships:
- Savings amount: $x$
- Discretionary spending: $2x$ (twice the savings amount)
- Emergency fund: $x + 20$ (\textdollar20 more than savings)
- Total available after fixed expenses: \textdollar480

We are given that savings plus discretionary spending equals \textdollar320:
$x + 2x = 320$
$3x = 320$
$x = \frac{320}{3}$

However, we also know that all three allocations (savings, discretionary spending, and emergency fund) must total \textdollar480:
$x + 2x + (x + 20) = 480$
$4x + 20 = 480$
$4x = 460$
$x = 115$

Let's verify this solution:
- Savings: \textdollar115
- Discretionary spending: $2(115) = $ \textdollar230
- Emergency fund: $115 + 20 = $ \textdollar135
- Total: $115 + 230 + 135 = $ \textdollar480 ✓
- Savings + Discretionary: $115 + 230 = $ \textdollar345

Wait, this doesn't match the given condition. Let me reconsider the problem.

The problem states that savings plus discretionary spending equals \textdollar320, and the total available is \textdollar480. This means the emergency fund should be $480 - 320 = $ \textdollar160.

So we have:
$x + 2x = 320$ ... (1)
$x + 20 = 160$ ... (2)

From equation (2): $x = 140$

Checking with equation (1): $140 + 2(140) = 140 + 280 = 420 \neq 320$

This creates a contradiction. Let me set up the problem correctly using the constraint that the emergency fund is \textdollar20 more than savings:

Let $x$ = savings amount
Discretionary spending = $2x$
Emergency fund = $x + 20$

Given conditions:
1) $x + 2x = 320$ (savings + discretionary = \textdollar320)
2) $x + 2x + (x + 20) = 480$ (total available = \textdollar480)

From condition 1: $3x = 320$, so $x = \frac{320}{3} ≈ 106.67$

From condition 2: $4x + 20 = 480$, so $4x = 460$, and $x = 115$

Since we have conflicting constraints, let me check if there's an error in the problem setup. The problem as stated creates an inconsistent system.

Let me resolve this by using the primary constraint and checking:
If $x + 2x = 320$, then $x = \frac{320}{3}$

But since we're dealing with money, let's assume there might be a small adjustment needed. Let me set up the problem so it has a clean integer solution:

Using the total constraint: $x + 2x + (x + 20) = 480$
$4x + 20 = 480$
$4x = 460$
$x = 115$

Therefore, Sarah should allocate \textdollar115 for her monthly savings.

\section*{Question 57}
\textbf{Metadata}

\begin{itemize}
  \item Question ID: O2-SPFndmdn\_O3-SPFndrng\_sonnet4\_Household Finance\_01
  \item Primary KC: STATISTICS AND PROBABILITY | Finding median | Finding median for a set of data
  \item Secondary KC: STATISTICS AND PROBABILITY | Finding range | finding range as measures of spread for a set of data 
  \item Topic: Household finance such as income, utility bills, money, interest, savings, instalment, mortgage, financial planning etc.
  \item Grade: Secondary O-level 2
\end{itemize}

\textbf{Solution}

(a) To find the median, first arrange the data in ascending order:
\textdollar78, \textdollar82, \textdollar85, \textdollar88, \textdollar90, \textdollar92, \textdollar95, \textdollar98, \textdollar105

Since there are 9 values (odd number), the median is the middle value, which is the 5th value.
Median = \textdollar90

(b) To find the range:
Range = Highest value - Lowest value
Range = \textdollar105 - \textdollar78 = \textdollar27

(c) Mrs. Tan's monthly budget for electricity:
Budget = Median + $\frac{1}{2} \times$ Range
Budget = \textdollar90 + $\frac{1}{2} \times$ \textdollar27
Budget = \textdollar90 + \textdollar13.50
Budget = \textdollar103.50

Therefore, Mrs. Tan should budget \textdollar103.50 monthly for electricity expenses.

\section*{Question 58}
\textbf{Metadata}

\begin{itemize}
  \item Question ID: O2-SPFndmean\_O3-BPRepSN\_sonnet4\_Household Finance\_01
  \item Primary KC: STATISTICS AND PROBABILITY | Finding mean deviation | calculation of the mean for a set of data
  \item Secondary KC: BASE AND POWER | Representation and concept  | use of standard form Ax10^n , where n is an integer, and 1<= A<= 10
  \item Topic: Household finance such as income, utility bills, money, interest, savings, instalment, mortgage, financial planning etc.
  \item Grade: Secondary O-level 2
\end{itemize}

\textbf{Solution}

(a) To find the mean monthly expense:

Mean = $\frac{\text{Sum of all expenses}}{\text{Number of months}}$

Sum = $12000 + 15000 + 9000 + 18000 + 11000 + 13000 = 78000$

Mean = $\frac{78000}{6} = 13000$

The mean monthly expense is \textdollar13000.

(b) To find the mean deviation:

First, find the absolute deviations from the mean:
$|12000 - 13000| = 1000$
$|15000 - 13000| = 2000$
$|9000 - 13000| = 4000$
$|18000 - 13000| = 5000$
$|11000 - 13000| = 2000$
$|13000 - 13000| = 0$

Mean deviation = $\frac{\text{Sum of absolute deviations}}{\text{Number of values}}$

Mean deviation = $\frac{1000 + 2000 + 4000 + 5000 + 2000 + 0}{6}$

Mean deviation = $\frac{14000}{6} = 2333.33...$

The mean deviation is \textdollar2333.33 (rounded to 2 decimal places).

(c) To express the mean deviation in standard form:

$2333.33 = 2.33333 \times 10^3$

Since we need $1 \leq A < 10$, we have $A = 2.33$ (rounded to 2 decimal places) and $n = 3$.

Therefore, the mean deviation in standard form is $2.33 \times 10^3$ dollars.

\section*{Question 59}
\textbf{Metadata}

\begin{itemize}
  \item Question ID: O3-BPOpr\_O3-BPRepNegI\_sonnet4\_Household Finance\_01
  \item Primary KC: BASE AND POWER | Operations | laws of indices
  \item Secondary KC: BASE AND POWER | Representation and concept  | negative indices
  \item Topic: Household finance such as income, utility bills, money, interest, savings, instalment, mortgage, financial planning etc.
  \item Grade: Secondary O-level 3/4
\end{itemize}

\textbf{Solution}

(a) Finding the electricity bill in the 3rd month:
$E = 120 \times 2^{-n}$ where $n = 3$
$E = 120 \times 2^{-3}$
$E = 120 \times \frac{1}{2^3}$
$E = 120 \times \frac{1}{8}$
$E = \frac{120}{8} = 15$
The electricity bill is \textdollar15.

(b) Finding the water bill in the 3rd month:
$W = 80 \times 3^{2-n}$ where $n = 3$
$W = 80 \times 3^{2-3}$
$W = 80 \times 3^{-1}$
$W = 80 \times \frac{1}{3^1}$
$W = 80 \times \frac{1}{3}$
$W = \frac{80}{3}$
The water bill is \textdollar$\frac{80}{3}$.

(c) Finding the total utility bill:
Total = Electricity bill + Water bill
Total = $15 + \frac{80}{3}$
Total = $\frac{45}{3} + \frac{80}{3}$
Total = $\frac{125}{3}$

The total utility bill for the 3rd month is \textdollar$\frac{125}{3}$.

\section*{Question 60}
\textbf{Metadata}

\begin{itemize}
  \item Question ID: O3-MXMulSM\_O3-MXAdd\_sonnet4\_Household Finance\_01
  \item Primary KC: MATRICES | Multiplication | product of a scalar quantity and a matrix
  \item Secondary KC: MATRICES | Addition | addition of matrices
  \item Topic: Household finance such as income, utility bills, money, interest, savings, instalment, mortgage, financial planning etc.
  \item Grade: Secondary O-level 3/4
\end{itemize}

\textbf{Solution}

(a) To find the expected expenses after a 15% increase, we multiply matrix $A$ by the scalar $1.15$:

$1.15A = 1.15 \times \begin{pmatrix} 120 & 450 & 200 \\ 135 & 480 & 220 \\ 110 & 420 & 180 \end{pmatrix}$

$= \begin{pmatrix} 1.15 \times 120 & 1.15 \times 450 & 1.15 \times 200 \\ 1.15 \times 135 & 1.15 \times 480 & 1.15 \times 220 \\ 1.15 \times 110 & 1.15 \times 420 & 1.15 \times 180 \end{pmatrix}$

$= \begin{pmatrix} 138 & 517.5 & 230 \\ 155.25 & 552 & 253 \\ 126.5 & 483 & 207 \end{pmatrix}$

Therefore, the expected expenses after 15% increase are \textdollar138, \textdollar517.50, \textdollar230 for January; \textdollar155.25, \textdollar552, \textdollar253 for February; and \textdollar126.50, \textdollar483, \textdollar207 for March respectively.

(b) The additional monthly savings can be represented as matrix $B$:

$B = \begin{pmatrix} 50 & 100 & 30 \\ 50 & 100 & 30 \\ 50 & 100 & 30 \end{pmatrix}$

The total monthly budget is found by adding the expected expenses to the additional savings:

$1.15A + B = \begin{pmatrix} 138 & 517.5 & 230 \\ 155.25 & 552 & 253 \\ 126.5 & 483 & 207 \end{pmatrix} + \begin{pmatrix} 50 & 100 & 30 \\ 50 & 100 & 30 \\ 50 & 100 & 30 \end{pmatrix}$

$= \begin{pmatrix} 138+50 & 517.5+100 & 230+30 \\ 155.25+50 & 552+100 & 253+30 \\ 126.5+50 & 483+100 & 207+30 \end{pmatrix}$

$= \begin{pmatrix} 188 & 617.5 & 260 \\ 205.25 & 652 & 283 \\ 176.5 & 583 & 237 \end{pmatrix}$

Therefore, the total monthly budget for the next quarter will be \textdollar188, \textdollar617.50, \textdollar260 for the first month; \textdollar205.25, \textdollar652, \textdollar283 for the second month; and \textdollar176.50, \textdollar583, \textdollar237 for the third month respectively.

\section*{Question 61}
\textbf{Metadata}

\begin{itemize}
  \item Question ID: O3-MXSub\_O3-MXAdd\_sonnet4\_Household Finance\_01
  \item Primary KC: MATRICES | Subtraction | subtraction of matrices
  \item Secondary KC: MATRICES | Addition | addition of matrices
  \item Topic: Household finance such as income, utility bills, money, interest, savings, instalment, mortgage, financial planning etc.
  \item Grade: Secondary O-level 3/4
\end{itemize}

\textbf{Solution}

(a) To find the total expenses for February and March combined, we need to add matrices $B$ and $C$:

$B + C = \begin{pmatrix} 165 & 420 & 340 \\ 185 & 400 & 310 \end{pmatrix} + \begin{pmatrix} 175 & 465 & 305 \\ 195 & 415 & 285 \end{pmatrix}$

$B + C = \begin{pmatrix} 165+175 & 420+465 & 340+305 \\ 185+195 & 400+415 & 310+285 \end{pmatrix}$

$B + C = \begin{pmatrix} 340 & 885 & 645 \\ 380 & 815 & 595 \end{pmatrix}$

(b) To find the difference between the combined February-March expenses and January expenses, we subtract matrix $A$ from matrix $(B + C)$:

$(B + C) - A = \begin{pmatrix} 340 & 885 & 645 \\ 380 & 815 & 595 \end{pmatrix} - \begin{pmatrix} 180 & 450 & 320 \\ 200 & 380 & 290 \end{pmatrix}$

$(B + C) - A = \begin{pmatrix} 340-180 & 885-450 & 645-320 \\ 380-200 & 815-380 & 595-290 \end{pmatrix}$

$(B + C) - A = \begin{pmatrix} 160 & 435 & 325 \\ 180 & 435 & 305 \end{pmatrix}$

This result shows that the combined February-March expenses exceeded January expenses in all categories during both time periods. The family spent significantly more on groceries (\textdollar435 more in each period) and substantial increases in utilities and transportation costs, indicating they need to budget for higher expenses in the second quarter.

\section*{Question 62}
\textbf{Metadata}

\begin{itemize}
  \item Question ID: O3-SPAddProb\_O2-SPRepPrSE\_sonnet4\_Household Finance\_01
  \item Primary KC: STATISTICS AND PROBABILITY | Addition | addition of probabilities
  \item Secondary KC: STATISTICS AND PROBABILITY | Representation and concept | probability of single events
  \item Topic: Household finance such as income, utility bills, money, interest, savings, instalment, mortgage, financial planning etc.
  \item Grade: Secondary O-level 3/4
\end{itemize}

\textbf{Solution}

To find the probability that at least one expense category exceeds the budget, I need to add the probabilities of the three mutually exclusive events.

Given information:
- Probability that utility bill exceeds budget = $P(U) = \frac{1}{4}$
- Probability that grocery expenses exceed budget = $P(G) = \frac{1}{3}$
- Probability that transportation costs exceed budget = $P(T) = \frac{1}{6}$
- The events are mutually exclusive

Since the events are mutually exclusive, the probability that at least one category exceeds the budget is:
$P(U \text{ or } G \text{ or } T) = P(U) + P(G) + P(T)$

$P(U \text{ or } G \text{ or } T) = \frac{1}{4} + \frac{1}{3} + \frac{1}{6}$

To add these fractions, I need a common denominator. The LCD of 4, 3, and 6 is 12.

$\frac{1}{4} = \frac{3}{12}$
$\frac{1}{3} = \frac{4}{12}$
$\frac{1}{6} = \frac{2}{12}$

Therefore:
$P(U \text{ or } G \text{ or } T) = \frac{3}{12} + \frac{4}{12} + \frac{2}{12} = \frac{9}{12} = \frac{3}{4}$

The probability that at least one expense category will exceed her budget in a given month is $\frac{3}{4}$ or 0.75 or 75%.

\section*{Question 63}
\textbf{Metadata}

\begin{itemize}
  \item Question ID: O3-SPAddProb\_O3-SPFndPrCE\_sonnet4\_Household Finance\_01
  \item Primary KC: STATISTICS AND PROBABILITY | Addition | addition of probabilities
  \item Secondary KC: STATISTICS AND PROBABILITY | Finding probability | probability of simple combined events
  \item Topic: Household finance such as income, utility bills, money, interest, savings, instalment, mortgage, financial planning etc.
  \item Grade: Secondary O-level 3/4
\end{itemize}

\textbf{Solution}

Let $E$ = event that electricity bill is paid late
Let $W$ = event that water bill is paid late

Given: $P(E) = \frac{1}{6}$, $P(W) = \frac{1}{8}$, and $E$ and $W$ are independent.

**(a) Finding the probability of paying at least one bill late:**

We need to find $P(E \cup W)$.

Using the addition rule for probabilities:
$P(E \cup W) = P(E) + P(W) - P(E \cap W)$

Since $E$ and $W$ are independent:
$P(E \cap W) = P(E) \times P(W) = \frac{1}{6} \times \frac{1}{8} = \frac{1}{48}$

Therefore:
$P(E \cup W) = \frac{1}{6} + \frac{1}{8} - \frac{1}{48}$

Finding common denominator (LCM of 6, 8, 48 is 48):
$P(E \cup W) = \frac{8}{48} + \frac{6}{48} - \frac{1}{48} = \frac{13}{48}$

**(b) With automatic payments:**

New probabilities:
$P(E_{new}) = \frac{1}{2} \times \frac{1}{6} = \frac{1}{12}$
$P(W_{new}) = \frac{1}{2} \times \frac{1}{8} = \frac{1}{16}$

Since the bills remain independent:
$P(E_{new} \cap W_{new}) = \frac{1}{12} \times \frac{1}{16} = \frac{1}{192}$

Using the addition rule:
$P(E_{new} \cup W_{new}) = \frac{1}{12} + \frac{1}{16} - \frac{1}{192}$

Finding common denominator (LCM of 12, 16, 192 is 192):
$P(E_{new} \cup W_{new}) = \frac{16}{192} + \frac{12}{192} - \frac{1}{192} = \frac{27}{192} = \frac{9}{64}$

**Final Answers:**
(a) The probability of paying at least one bill late is $\frac{13}{48}$
(b) With automatic payments, the probability of paying at least one bill late is $\frac{9}{64}$

\section*{Question 64}
\textbf{Metadata}

\begin{itemize}
  \item Question ID: O3-SPMulProb\_O3-SPFndPrCE\_sonnet4\_Household Finance\_01
  \item Primary KC: STATISTICS AND PROBABILITY | Multiplication | multiplication of probabilities
  \item Secondary KC: STATISTICS AND PROBABILITY | Finding probability | probability of simple combined events
  \item Topic: Household finance such as income, utility bills, money, interest, savings, instalment, mortgage, financial planning etc.
  \item Grade: Secondary O-level 3/4
\end{itemize}

\textbf{Solution}

Given information:
- Probability that electricity bill exceeds \textdollar120: $P(E) = \frac{3}{4}$
- Probability that water bill exceeds \textdollar80: $P(W) = \frac{2}{5}$
- The two events are independent

(a) Finding the probability that both bills exceed their thresholds:

Since the events are independent, we use the multiplication rule for independent events:
$P(E \text{ and } W) = P(E) \times P(W)$

$P(E \text{ and } W) = \frac{3}{4} \times \frac{2}{5} = \frac{6}{20} = \frac{3}{10}$

Therefore, the probability that both bills exceed their thresholds is $\frac{3}{10}$ or $0.3$.

(b) Finding the probability that at least one bill exceeds its threshold:

We can solve this using the complement rule:
$P(\text{at least one}) = 1 - P(\text{neither})$

First, find the probability that neither bill exceeds its threshold:
- $P(\text{electricity does not exceed \textdollar120}) = 1 - \frac{3}{4} = \frac{1}{4}$
- $P(\text{water does not exceed \textdollar80}) = 1 - \frac{2}{5} = \frac{3}{5}$

Since the events are independent:
$P(\text{neither}) = \frac{1}{4} \times \frac{3}{5} = \frac{3}{20}$

Therefore:
$P(\text{at least one}) = 1 - \frac{3}{20} = \frac{20}{20} - \frac{3}{20} = \frac{17}{20}$

The probability that at least one bill exceeds its threshold is $\frac{17}{20}$ or $0.85$.

\end{document}
