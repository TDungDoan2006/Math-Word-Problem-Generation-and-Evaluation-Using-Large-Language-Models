\documentclass{article}
\usepackage[utf8]{inputenc}
\usepackage{amsmath}
\usepackage{amsfonts}
\usepackage{amssymb}
\usepackage{graphicx}
\usepackage{hyperref}
\title{'Sarah Questions household finance v3 CLAUDE '}
\author{Tien Dung Doan}
\begin{document}
\maketitle
\section*{Question 1}
\textbf{Metadata}

\begin{itemize}
  \item Question ID: P3-WNDivRmd3d\_P1-WNCmp\_sonnet4\_Household Finance\_01
  \item Primary KC: WHOLE NUMBERS | Division | dividing whole numbers up to 3 digits by 1 digit with remainder 
  \item Secondary KC: WHOLE NUMBERS | Comparison and ordering | comparing and ordering whole numbers
  \item Topic: Household finance such as income, utility bills, money, interest, savings, instalment, mortgage, financial planning etc.
  \item Grade: Primary 3
\end{itemize}

\textbf{Question}

Mrs. Tan wants to save money for her family's vacation. She has \textdollar287 in her savings account. She plans to divide this money equally among 4 different saving jars for different vacation expenses: accommodation, food, transport, and activities. After dividing the money equally, she will compare the amounts in each jar to see how much money is left over. How much money will be in each jar? How much money will be left over? Which amount is greater: the money in each jar or the money left over?

\section*{Question 2}
\textbf{Metadata}

\begin{itemize}
  \item Question ID: P3-WNMul3d1d\_P1-WNAdd2nd\_sonnet4\_Household Finance\_01
  \item Primary KC: WHOLE NUMBERS | Multiplication | multiplying whole numbers up to 3 digits by 1 digit
  \item Secondary KC: WHOLE NUMBERS | Addition | adding whole numbers
  \item Topic: Household finance such as income, utility bills, money, interest, savings, instalment, mortgage, financial planning etc.
  \item Grade: Primary 3
\end{itemize}

\textbf{Question}

Mrs. Tan pays her electricity bill every month. In January, her electricity bill was \textdollar126. In February, March, and April, her electricity bills were each \textdollar89. How much did Mrs. Tan pay for electricity in these 4 months altogether?

\section*{Question 3}
\textbf{Metadata}

\begin{itemize}
  \item Question ID: P3-WNMul3d1d\_P1-WNSub2nd\_sonnet4\_Household Finance\_01
  \item Primary KC: WHOLE NUMBERS | Multiplication | multiplying whole numbers up to 3 digits by 1 digit
  \item Secondary KC: WHOLE NUMBERS | Subtraction | subtracting whole numbers
  \item Topic: Household finance such as income, utility bills, money, interest, savings, instalment, mortgage, financial planning etc.
  \item Grade: Primary 3
\end{itemize}

\textbf{Question}

Mrs. Chen is planning her family's monthly budget. She needs to pay \textdollar324 for electricity bills each month. She decides to save money for this by setting aside the same amount every week for 4 weeks. However, she already has \textdollar29 saved from last month. How much does Mrs. Chen need to save each week to have enough money for her electricity bill?

\section*{Question 4}
\textbf{Metadata}

\begin{itemize}
  \item Question ID: P3-FrAddRl12\_P3-FrSmp\_sonnet4\_Household Finance\_01
  \item Primary KC: FRACTIONS | Addition | adding two related fractions within one whole with denominators of given fractions not exceeding 12
  \item Secondary KC: FRACTIONS | Simplifying | expressing a fraction in its simplest form
  \item Topic: Household finance such as income, utility bills, money, interest, savings, instalment, mortgage, financial planning etc.
  \item Grade: Primary 3
\end{itemize}

\textbf{Question}

Sarah helps her mother manage the household expenses. In January, they spent $\frac{3}{8}$ of their monthly budget on groceries and $\frac{1}{8}$ of their monthly budget on utility bills. What fraction of their monthly budget did they spend on groceries and utility bills altogether? Express your answer in its simplest form.

\section*{Question 5}
\textbf{Metadata}

\begin{itemize}
  \item Question ID: P3-FrSubRl12\_P3-FrSmp\_sonnet4\_Household Finance\_01
  \item Primary KC: FRACTIONS | Subtraction | subtracting two related fractions within one whole with denominators of given fractions not exceeding 12
  \item Secondary KC: FRACTIONS | Simplifying | expressing a fraction in its simplest form
  \item Topic: Household finance such as income, utility bills, money, interest, savings, instalment, mortgage, financial planning etc.
  \item Grade: Primary 3
\end{itemize}

\textbf{Question}

Mrs. Tan's family spends $\frac{5}{8}$ of their monthly income on household expenses. They spend $\frac{3}{8}$ of their monthly income on food and groceries. How much more of their monthly income do they spend on household expenses than on food and groceries? Express your answer in its simplest form.

\section*{Question 6}
\textbf{Metadata}

\begin{itemize}
  \item Question ID: P4-WNMul4d1d\_P1-WNSub2nd\_sonnet4\_Household Finance\_01
  \item Primary KC: WHOLE NUMBERS | Multiplication | multiplying whole numbers up to 4 digits by 1 digit or up to 3 digits by 2 digits
  \item Secondary KC: WHOLE NUMBERS | Subtraction | subtracting whole numbers
  \item Topic: Household finance such as income, utility bills, money, interest, savings, instalment, mortgage, financial planning etc.
  \item Grade: Primary 4
\end{itemize}

\textbf{Question}

The Chen family is planning their monthly budget. Mr. Chen earns \textdollar3,245 per month and Mrs. Chen earns \textdollar2,850 per month. They need to pay their utility bills which include electricity costing \textdollar125, water costing \textdollar68, and internet costing \textdollar89. They also want to save \textdollar800 each month for their children's education fund. After paying all bills and setting aside their savings, how much money will the Chen family have left for other expenses?

\section*{Question 7}
\textbf{Metadata}

\begin{itemize}
  \item Question ID: P4-WNDiv4d1d\_P1-WNAdd2nd\_sonnet4\_Household Finance\_01
  \item Primary KC: WHOLE NUMBERS | Division | dividing whole numbers up to 4 digits by 1 digit
  \item Secondary KC: WHOLE NUMBERS | Addition | adding whole numbers
  \item Topic: Household finance such as income, utility bills, money, interest, savings, instalment, mortgage, financial planning etc.
  \item Grade: Primary 4
\end{itemize}

\textbf{Question}

The Lim family pays a total of \textdollar3264 for their household utilities over 8 months. They pay the same amount each month. In addition to their monthly utility bill, they also pay a one-time security deposit of \textdollar156 and a one-time connection fee of \textdollar72. What is the total amount the Lim family pays for utilities including all the additional fees?

\section*{Question 8}
\textbf{Metadata}

\begin{itemize}
  \item Question ID: P4-FrRepSet\_P3-FrCnvEq\_sonnet4\_Household Finance\_01
  \item Primary KC: FRACTIONS | Representation and concept | expressing a part of a set as a fraction
  \item Secondary KC: FRACTIONS | Conversion to equivalent fractions | Conversion to equivalent fractions (given either the denominator or the numerator)
  \item Topic: Household finance such as income, utility bills, money, interest, savings, instalment, mortgage, financial planning etc.
  \item Grade: Primary 4
\end{itemize}

\textbf{Question}

Mrs. Tan receives her monthly salary of \textdollar2400. She saves $\frac{3}{8}$ of her salary each month. Mrs. Tan wants to compare her savings with her friend Mrs. Lim, who saves $\frac{9}{24}$ of her salary each month. Are Mrs. Tan and Mrs. Lim saving the same fraction of their salaries? Show your working by converting one of the fractions to have the same denominator as the other. How much money does Mrs. Tan save each month?

\section*{Question 9}
\textbf{Metadata}

\begin{itemize}
  \item Question ID: P4-FrSubU12\_P2-FrAdd2nd\_sonnet4\_Household Finance\_01
  \item Primary KC: FRACTIONS | Subtraction | subtracting unlike fractions with two different denominators not exceeding 12
  \item Secondary KC: FRACTIONS | Addition | adding fractions
  \item Topic: Household finance such as income, utility bills, money, interest, savings, instalment, mortgage, financial planning etc.
  \item Grade: Primary 4
\end{itemize}

\textbf{Question}

Sarah's family is planning their monthly budget. They spend $\frac{5}{8}$ of their monthly income on household expenses and $\frac{1}{6}$ of their monthly income on savings. How much more of their monthly income do they spend on household expenses than on savings? What fraction of their monthly income is spent on household expenses and savings combined?

\section*{Question 10}
\textbf{Metadata}

\begin{itemize}
  \item Question ID: P4-DcSub2d\_P4-DcCmp3d\_sonnet4\_Household Finance\_01
  \item Primary KC: DECIMALS | Subtraction | subtracting decimals (up to 2 decimal places)
  \item Secondary KC: DECIMALS | Comparison and ordering | comparing and ordering decimals up to 3 decimal places
  \item Topic: Household finance such as income, utility bills, money, interest, savings, instalment, mortgage, financial planning etc.
  \item Grade: Primary 4
\end{itemize}

\textbf{Question}

The Lim family is reviewing their monthly utility bills to plan their budget. In January, their electricity bill was \textdollar68.75 and their water bill was \textdollar23.40. In February, their electricity bill was \textdollar52.30 and their water bill was \textdollar19.85. (a) How much did the electricity bill decrease from January to February? (b) How much did the water bill decrease from January to February? (c) Which utility bill had a greater decrease? By how much more?

\section*{Question 11}
\textbf{Metadata}

\begin{itemize}
  \item Question ID: P4-DcMul2d1d\_P4-DcCmp3d\_sonnet4\_Household Finance\_01
  \item Primary KC: DECIMALS | Multiplication | multiplying decimals (up to 2 decimal places) by a 1-digit whole number
  \item Secondary KC: DECIMALS | Comparison and ordering | comparing and ordering decimals up to 3 decimal places
  \item Topic: Household finance such as income, utility bills, money, interest, savings, instalment, mortgage, financial planning etc.
  \item Grade: Primary 4
\end{itemize}

\textbf{Question}

Mrs. Tan is comparing the monthly electricity bills for her family over three months. In January, her electricity bill was \textdollar45.80. In February, the bill increased by 3 times the amount of \textdollar2.65. In March, the bill was \textdollar53.750. Help Mrs. Tan by answering the following questions:

(a) What was the electricity bill in February?
(b) Arrange the three months' electricity bills in ascending order.

\section*{Question 12}
\textbf{Metadata}

\begin{itemize}
  \item Question ID: P4-DcMul2d1d\_P4-DcRnd3d\_sonnet4\_Household Finance\_01
  \item Primary KC: DECIMALS | Multiplication | multiplying decimals (up to 2 decimal places) by a 1-digit whole number
  \item Secondary KC: DECIMALS | Rounding | rounding decimals up to 3 decimal places to the nearest whole number, 1 decimal place and 2 decimal places 
  \item Topic: Household finance such as income, utility bills, money, interest, savings, instalment, mortgage, financial planning etc.
  \item Grade: Primary 4
\end{itemize}

\textbf{Question}

Mrs. Tan is calculating her monthly electricity bill. The electricity rate is \textdollar0.28 per unit. Last month, her family used 147.6 units of electricity. She wants to estimate her bill by rounding the exact amount to the nearest dollar and to the nearest cent. Calculate the exact electricity bill, then round it to the nearest dollar and to the nearest cent.

\section*{Question 13}
\textbf{Metadata}

\begin{itemize}
  \item Question ID: P4-DcMul2d1d\_P4-DcSub2nd\_sonnet4\_Household Finance\_01
  \item Primary KC: DECIMALS | Multiplication | multiplying decimals (up to 2 decimal places) by a 1-digit whole number
  \item Secondary KC: DECIMALS | Subtraction | subtracting decimals
  \item Topic: Household finance such as income, utility bills, money, interest, savings, instalment, mortgage, financial planning etc.
  \item Grade: Primary 4
\end{itemize}

\textbf{Question}

Mrs. Tan's family electricity bill for March was \textdollar42.85. In April, their electricity usage increased and the bill became 3 times the March bill. However, they received a government rebate of \textdollar18.60 for the April bill. How much did Mrs. Tan's family actually pay for their April electricity bill after the rebate?

\section*{Question 14}
\textbf{Metadata}

\begin{itemize}
  \item Question ID: P4-DcDiv2d1d\_P4-DcRnd3d\_sonnet4\_Household Finance\_01
  \item Primary KC: DECIMALS | Division | dividing decimals (up to 2 decimal places) by a 1-digit whole number
  \item Secondary KC: DECIMALS | Rounding | rounding decimals up to 3 decimal places to the nearest whole number, 1 decimal place and 2 decimal places 
  \item Topic: Household finance such as income, utility bills, money, interest, savings, instalment, mortgage, financial planning etc.
  \item Grade: Primary 4
\end{itemize}

\textbf{Question}

Mrs. Tan wants to divide her monthly utility bill equally among her 4 children as part of teaching them about household expenses. Her electricity bill is \textdollar89.64, water bill is \textdollar45.28, and gas bill is \textdollar37.16. Find the total utility bill amount. Then calculate how much each child needs to contribute. Round your final answer to the nearest cent.

\section*{Question 15}
\textbf{Metadata}

\begin{itemize}
  \item Question ID: P5-FrAddMix\_P5-FrCnv2Dc\_sonnet4\_Household Finance\_01
  \item Primary KC: FRACTIONS | Addition | adding mixed numbers
  \item Secondary KC: FRACTIONS | Conversion to decimals | expressing fractions as decimals
  \item Topic: Household finance such as income, utility bills, money, interest, savings, instalment, mortgage, financial planning etc.
  \item Grade: Primary 5
\end{itemize}

\textbf{Question}

Mrs. Tan is tracking her family's monthly utility expenses. In January, she paid \textdollar$2\frac{3}{4}$ for the water bill and \textdollar$3\frac{1}{2}$ for the electricity bill. In February, she paid \textdollar$2\frac{1}{4}$ for the water bill and \textdollar$3\frac{3}{4}$ for the electricity bill. 

(a) What was the total amount Mrs. Tan paid for utilities in January?
(b) What was the total amount Mrs. Tan paid for utilities in February?
(c) Express the total utility expenses for both months as a decimal.

\section*{Question 16}
\textbf{Metadata}

\begin{itemize}
  \item Question ID: P5-FrSubMix\_P3-FrSmp\_sonnet4\_Household Finance\_01
  \item Primary KC: FRACTIONS | Subtraction | subtracting mixed numbers
  \item Secondary KC: FRACTIONS | Simplifying | expressing a fraction in its simplest form
  \item Topic: Household finance such as income, utility bills, money, interest, savings, instalment, mortgage, financial planning etc.
  \item Grade: Primary 5
\end{itemize}

\textbf{Question}

The Lim family budgets \textdollar$4\frac{3}{4}$ for their monthly electricity bill. In January, their actual electricity bill was \textdollar$2\frac{5}{6}$. How much money did they save compared to their budget? Express your answer as a fraction in its simplest form.

\section*{Question 17}
\textbf{Metadata}

\begin{itemize}
  \item Question ID: P5-FrSubMix\_P5-FrCnv2Dc\_sonnet4\_Household Finance\_01
  \item Primary KC: FRACTIONS | Subtraction | subtracting mixed numbers
  \item Secondary KC: FRACTIONS | Conversion to decimals | expressing fractions as decimals
  \item Topic: Household finance such as income, utility bills, money, interest, savings, instalment, mortgage, financial planning etc.
  \item Grade: Primary 5
\end{itemize}

\textbf{Question}

The Tan family budgets $3\frac{3}{4}$ hours each week for financial planning activities. Last week, they spent $1\frac{1}{2}$ hours reviewing their monthly utility bills and mortgage payments. How much time did they have left for other financial planning activities? Express your answer as a decimal.

\section*{Question 18}
\textbf{Metadata}

\begin{itemize}
  \item Question ID: P5-FrMulImN\_P2-FrAdd2nd\_sonnet4\_Household Finance\_01
  \item Primary KC: FRACTIONS | Multiplication | multiplying a proper/improper fraction and a whole number
  \item Secondary KC: FRACTIONS | Addition | adding fractions
  \item Topic: Household finance such as income, utility bills, money, interest, savings, instalment, mortgage, financial planning etc.
  \item Grade: Primary 5
\end{itemize}

\textbf{Question}

Mrs. Tan pays her electricity bill monthly. In January, she paid \textdollar120. In February, her electricity bill was $\frac{5}{6}$ of January's bill. In March, her electricity bill was $\frac{3}{4}$ of January's bill. What was the total amount Mrs. Tan paid for electricity over these three months?

\section*{Question 19}
\textbf{Metadata}

\begin{itemize}
  \item Question ID: P5-FrMulImN\_P3-FrSmp\_sonnet4\_Household Finance\_01
  \item Primary KC: FRACTIONS | Multiplication | multiplying a proper/improper fraction and a whole number
  \item Secondary KC: FRACTIONS | Simplifying | expressing a fraction in its simplest form
  \item Topic: Household finance such as income, utility bills, money, interest, savings, instalment, mortgage, financial planning etc.
  \item Grade: Primary 5
\end{itemize}

\textbf{Question}

Mrs. Tan's monthly household income is \textdollar4800. She spends $\frac{5}{8}$ of her income on household expenses. How much money does she spend on household expenses each month? Express your answer as a mixed number in its simplest form.

\section*{Question 20}
\textbf{Metadata}

\begin{itemize}
  \item Question ID: P5-FrMulImN\_P5-FrCnv2Dc\_sonnet4\_Household Finance\_01
  \item Primary KC: FRACTIONS | Multiplication | multiplying a proper/improper fraction and a whole number
  \item Secondary KC: FRACTIONS | Conversion to decimals | expressing fractions as decimals
  \item Topic: Household finance such as income, utility bills, money, interest, savings, instalment, mortgage, financial planning etc.
  \item Grade: Primary 5
\end{itemize}

\textbf{Question}

Mrs. Tan pays \textdollar240 for her monthly electricity bill. Due to energy-saving measures, she managed to reduce her electricity usage by $\frac{3}{8}$ of the original amount. After the reduction, she wants to know her new monthly electricity bill as a decimal number. What is Mrs. Tan's new monthly electricity bill in dollars, expressed as a decimal?

\section*{Question 21}
\textbf{Metadata}

\begin{itemize}
  \item Question ID: P5-FrMulImIm\_P2-FrCmp\_sonnet4\_Household Finance\_01
  \item Primary KC: FRACTIONS | Multiplication | multiplying two improper fractions
  \item Secondary KC: FRACTIONS | Comparison and ordering | comparing and ordering fractions
  \item Topic: Household finance such as income, utility bills, money, interest, savings, instalment, mortgage, financial planning etc.
  \item Grade: Primary 5
\end{itemize}

\textbf{Question}

Sarah's family spends $\frac{7}{4}$ of their monthly income on housing expenses. Of this housing expense amount, they spend $\frac{8}{5}$ on their mortgage payment. If Sarah's family has a monthly income of \textdollar2400, how much do they spend on their mortgage payment each month? Is their mortgage payment more than half of their total monthly income?

\section*{Question 22}
\textbf{Metadata}

\begin{itemize}
  \item Question ID: P5-FrMulImIm\_P5-FrCnv2Dc\_sonnet4\_Household Finance\_01
  \item Primary KC: FRACTIONS | Multiplication | multiplying two improper fractions
  \item Secondary KC: FRACTIONS | Conversion to decimals | expressing fractions as decimals
  \item Topic: Household finance such as income, utility bills, money, interest, savings, instalment, mortgage, financial planning etc.
  \item Grade: Primary 5
\end{itemize}

\textbf{Question}

Mrs. Tan's monthly electricity bill is \textdollar120. Due to increased usage during the school holidays, her bill increased by $\frac{7}{4}$ times the original amount. At the same time, her water bill, which was $\frac{3}{2}$ of her original electricity bill, also increased by the same factor. Calculate the total increase in both bills combined and express your answer as a decimal.

\section*{Question 23}
\textbf{Metadata}

\begin{itemize}
  \item Question ID: P5-FrMulMixN\_P2-FrAdd2nd\_sonnet4\_Household Finance\_01
  \item Primary KC: FRACTIONS | Multiplication | multiplying a mixed number and a whole number
  \item Secondary KC: FRACTIONS | Addition | adding fractions
  \item Topic: Household finance such as income, utility bills, money, interest, savings, instalment, mortgage, financial planning etc.
  \item Grade: Primary 5
\end{itemize}

\textbf{Question}

Sarah's family pays \textdollar180 for their monthly electricity bill. In January, they used $2\frac{1}{4}$ times their usual amount of electricity due to air conditioning during a heat wave. In February, they managed to reduce their usage and paid $\frac{2}{3}$ of their usual bill. What was the total amount Sarah's family paid for electricity over these two months?

\section*{Question 24}
\textbf{Metadata}

\begin{itemize}
  \item Question ID: P5-DcMul3dK\_P4-DcCmp3d\_sonnet4\_Household Finance\_01
  \item Primary KC: DECIMALS | Multiplication | multiplying decimals (up to 3 decimal places) by 10, 100, 1000 and their multiples
  \item Secondary KC: DECIMALS | Comparison and ordering | comparing and ordering decimals up to 3 decimal places
  \item Topic: Household finance such as income, utility bills, money, interest, savings, instalment, mortgage, financial planning etc.
  \item Grade: Primary 5
\end{itemize}

\textbf{Question}

Mrs. Tan is comparing electricity bills from three different months to plan her household budget. In January, her electricity bill was \textdollar0.248 per unit. In February, the rate increased to \textdollar0.315 per unit. In March, the rate was \textdollar0.287 per unit. To better understand the costs, she wants to calculate what these rates would be if multiplied by 1000 units of electricity. After calculating, she wants to arrange the three monthly costs from the lowest to the highest to see which month had the most affordable electricity rate.

\section*{Question 25}
\textbf{Metadata}

\begin{itemize}
  \item Question ID: P5-DcMul3dK\_P4-DcAdd2nd\_sonnet4\_Household Finance\_01
  \item Primary KC: DECIMALS | Multiplication | multiplying decimals (up to 3 decimal places) by 10, 100, 1000 and their multiples
  \item Secondary KC: DECIMALS | Addition | adding decimals
  \item Topic: Household finance such as income, utility bills, money, interest, savings, instalment, mortgage, financial planning etc.
  \item Grade: Primary 5
\end{itemize}

\textbf{Question}

The Tan family is planning their monthly budget. Their electricity bill is \textdollar0.285 per unit, and they used 400 units last month. Their water bill is \textdollar0.147 per unit, and they used 300 units. Additionally, they have a fixed internet charge of \textdollar39.90 per month. What is the total amount the Tan family needs to pay for these three utilities this month?

\section*{Question 26}
\textbf{Metadata}

\begin{itemize}
  \item Question ID: P5-DcDiv3dK\_P4-DcCmp3d\_sonnet4\_Household Finance\_01
  \item Primary KC: DECIMALS | Division | dividing decimals (up to 3 decimal places) by 10, 100, 1000 and their multiples
  \item Secondary KC: DECIMALS | Comparison and ordering | comparing and ordering decimals up to 3 decimal places
  \item Topic: Household finance such as income, utility bills, money, interest, savings, instalment, mortgage, financial planning etc.
  \item Grade: Primary 5
\end{itemize}

\textbf{Question}

Mrs. Lim is comparing electricity bills from three different months to find the most cost-effective usage pattern. In January, her total electricity bill was \textdollar45.600. In February, she used 1000 times less electricity and paid \textdollar0.046. In March, she used 100 times less electricity than January and paid accordingly. Calculate March's electricity bill and arrange the three months' bills from lowest to highest.

\section*{Question 27}
\textbf{Metadata}

\begin{itemize}
  \item Question ID: P5-DcDiv3dK\_P4-DcSub2nd\_sonnet4\_Household Finance\_01
  \item Primary KC: DECIMALS | Division | dividing decimals (up to 3 decimal places) by 10, 100, 1000 and their multiples
  \item Secondary KC: DECIMALS | Subtraction | subtracting decimals
  \item Topic: Household finance such as income, utility bills, money, interest, savings, instalment, mortgage, financial planning etc.
  \item Grade: Primary 5
\end{itemize}

\textbf{Question}

Mrs. Tan received her monthly electricity bill of \textdollar48.750. She noticed that this amount includes a government subsidy. Without the subsidy, her original bill would have been \textdollar65.320. The subsidy amount she received is distributed equally among 100 households in her neighborhood. How much subsidy did each household receive?

\section*{Question 28}
\textbf{Metadata}

\begin{itemize}
  \item Question ID: P5-PcRepWh\_P1-WNAdd2nd\_sonnet4\_Household Finance\_01
  \item Primary KC: PERCENTAGE | Representation and concept | expressing a part of a whole as a percentage
  \item Secondary KC: WHOLE NUMBERS | Addition | adding whole numbers
  \item Topic: Household finance such as income, utility bills, money, interest, savings, instalment, mortgage, financial planning etc.
  \item Grade: Primary 5
\end{itemize}

\textbf{Question}

The Tan family tracks their monthly household expenses. In January, they spent \textdollar480 on groceries, \textdollar320 on utilities, and \textdollar200 on transportation. Their total monthly income is \textdollar2500. What is their total monthly expenses? What percentage of their monthly income do they spend on expenses?

\section*{Question 29}
\textbf{Metadata}

\begin{itemize}
  \item Question ID: P5-PcRepWh\_P1-WNSub2nd\_sonnet4\_Household Finance\_01
  \item Primary KC: PERCENTAGE | Representation and concept | expressing a part of a whole as a percentage
  \item Secondary KC: WHOLE NUMBERS | Subtraction | subtracting whole numbers
  \item Topic: Household finance such as income, utility bills, money, interest, savings, instalment, mortgage, financial planning etc.
  \item Grade: Primary 5
\end{itemize}

\textbf{Question}

The Chen family's monthly household budget is \textdollar3600. In January, they spent \textdollar2880 on various expenses including groceries, utilities, and transportation. The remaining amount was saved. What percentage of their monthly budget did the Chen family save in January?

\section*{Question 30}
\textbf{Metadata}

\begin{itemize}
  \item Question ID: P5-RtFndT\_P2-DcCnvD2N\_sonnet4\_Household Finance\_01
  \item Primary KC: RATE | Finding total amount | finding total amount, given rate and number of units
  \item Secondary KC: DECIMALS | Conversion to smaller units | converting a measurement from a larger unit in decimal form to a smaller unit
  \item Topic: Household finance such as income, utility bills, money, interest, savings, instalment, mortgage, financial planning etc.
  \item Grade: Primary 5
\end{itemize}

\textbf{Question}

Mrs. Chen pays \textdollar0.25 per kilowatt-hour (kWh) for electricity in her home. Last month, her family used 2.8 thousand kWh of electricity. How much did Mrs. Chen pay for electricity last month?

\section*{Question 31}
\textbf{Metadata}

\begin{itemize}
  \item Question ID: P6-FrDivPN\_P2-FrCmp\_sonnet4\_Household Finance\_01
  \item Primary KC: FRACTIONS | Division | dividing a proper fraction by a whole number
  \item Secondary KC: FRACTIONS | Comparison and ordering | comparing and ordering fractions
  \item Topic: Household finance such as income, utility bills, money, interest, savings, instalment, mortgage, financial planning etc.
  \item Grade: Primary 6
\end{itemize}

\textbf{Question}

Mrs. Tan has \textdollar240 in her savings account. She decides to withdraw $\frac{3}{4}$ of her savings and divide it equally among her 3 children as their weekly allowance. Each child will receive the same amount for 4 weeks. After giving out the allowances, she compares the remaining amount in her account with each child's total allowance over the 4 weeks. How much money does each child receive per week? How does each child's total 4-week allowance compare to the money remaining in Mrs. Tan's account?

\section*{Question 32}
\textbf{Metadata}

\begin{itemize}
  \item Question ID: P6-FrDivPN\_P2-FrSub2nd\_sonnet4\_Household Finance\_01
  \item Primary KC: FRACTIONS | Division | dividing a proper fraction by a whole number
  \item Secondary KC: FRACTIONS | Subtraction | subtracting fractions
  \item Topic: Household finance such as income, utility bills, money, interest, savings, instalment, mortgage, financial planning etc.
  \item Grade: Primary 6
\end{itemize}

\textbf{Question}

Mrs. Tan's family spends $\frac{3}{4}$ of their monthly income on various expenses. After paying for rent and groceries, they have $\frac{1}{3}$ of their monthly income remaining for other expenses. The remaining amount needs to be divided equally among 6 different categories: utilities, transportation, insurance, entertainment, clothing, and emergency fund. What fraction of their monthly income is allocated to each category?

\section*{Question 33}
\textbf{Metadata}

\begin{itemize}
  \item Question ID: P6-FrDivPN\_P5-FrMul2nd\_sonnet4\_Household Finance\_01
  \item Primary KC: FRACTIONS | Division | dividing a proper fraction by a whole number
  \item Secondary KC: FRACTIONS | Multiplication | fraction multiplication
  \item Topic: Household finance such as income, utility bills, money, interest, savings, instalment, mortgage, financial planning etc.
  \item Grade: Primary 6
\end{itemize}

\textbf{Question}

Sarah's family has a monthly electricity bill of \textdollar240. They decide to reduce their electricity usage by $\frac{2}{5}$ of the original amount. The reduced bill amount will be split equally among 4 family members to pay. How much will each family member pay?

\section*{Question 34}
\textbf{Metadata}

\begin{itemize}
  \item Question ID: P6-FrDivPP\_P2-FrAdd2nd\_sonnet4\_Household Finance\_01
  \item Primary KC: FRACTIONS | Division | dividing a whole number/proper fraction by a proper fraction
  \item Secondary KC: FRACTIONS | Addition | adding fractions
  \item Topic: Household finance such as income, utility bills, money, interest, savings, instalment, mortgage, financial planning etc.
  \item Grade: Primary 6
\end{itemize}

\textbf{Question}

Sarah's family pays \textdollar240 for their monthly electricity bill. This amount represents $\frac{3}{5}$ of their total monthly utility expenses. The water bill makes up $\frac{1}{4}$ of the total utility expenses, and the gas bill makes up the remaining portion. How much does Sarah's family pay for their water and gas bills combined each month?

\section*{Question 35}
\textbf{Metadata}

\begin{itemize}
  \item Question ID: P6-FrDivPP\_P5-FrMul2nd\_sonnet4\_Household Finance\_01
  \item Primary KC: FRACTIONS | Division | dividing a whole number/proper fraction by a proper fraction
  \item Secondary KC: FRACTIONS | Multiplication | fraction multiplication
  \item Topic: Household finance such as income, utility bills, money, interest, savings, instalment, mortgage, financial planning etc.
  \item Grade: Primary 6
\end{itemize}

\textbf{Question}

Mrs. Tan receives a monthly salary of \textdollar4800. She saves $\frac{1}{6}$ of her salary each month. She decides to divide her monthly savings equally among 3 investment accounts. If each investment account earns $\frac{3}{4}$ times the amount deposited as interest after one year, how much interest will she earn from one investment account after one year?

\section*{Question 36}
\textbf{Metadata}

\begin{itemize}
  \item Question ID: P6-PcFndWN\_P1-WNMul2nd\_sonnet4\_Household Finance\_01
  \item Primary KC: PERCENTAGE | Finding the whole | finding the whole given a part and the percentage
  \item Secondary KC: WHOLE NUMBERS | Multiplication | multiplying whole numbers
  \item Topic: Household finance such as income, utility bills, money, interest, savings, instalment, mortgage, financial planning etc.
  \item Grade: Primary 6
\end{itemize}

\textbf{Question}

Mrs. Tan pays her electricity bill through monthly instalments. This month, she paid \textdollar84, which represents 30\% of her total electricity bill for the year. If she pays the same amount every month, how much does she pay in total for her electricity bill over 12 months?

\section*{Question 37}
\textbf{Metadata}

\begin{itemize}
  \item Question ID: P6-PcFndWN\_P1-WNDiv2nd\_sonnet4\_Household Finance\_01
  \item Primary KC: PERCENTAGE | Finding the whole | finding the whole given a part and the percentage
  \item Secondary KC: WHOLE NUMBERS | Division | dividing whole numbers
  \item Topic: Household finance such as income, utility bills, money, interest, savings, instalment, mortgage, financial planning etc.
  \item Grade: Primary 6
\end{itemize}

\textbf{Question}

Mrs. Lim pays her electricity bill using an instalment plan. She has already paid \textdollar168 in instalments, which represents 30\% of her total electricity bill. If she decides to pay the remaining amount in 4 equal instalments, how much does she need to pay for each of the remaining instalments?

\section*{Question 38}
\textbf{Metadata}

\begin{itemize}
  \item Question ID: P6-PcFndChg\_P1-WNAdd2nd\_sonnet4\_Household Finance\_01
  \item Primary KC: PERCENTAGE | Finding change | finding percentage increase/decrease
  \item Secondary KC: WHOLE NUMBERS | Addition | adding whole numbers
  \item Topic: Household finance such as income, utility bills, money, interest, savings, instalment, mortgage, financial planning etc.
  \item Grade: Primary 6
\end{itemize}

\textbf{Question}

The Tan family's monthly electricity bill was \textdollar180 in January. In February, they installed energy-saving LED bulbs and their electricity bill decreased to \textdollar135. In March, due to hot weather, they used the air conditioner more frequently and their electricity bill increased to \textdollar189. Find the percentage decrease in their electricity bill from January to February. Then, find the percentage increase in their electricity bill from February to March.

\section*{Question 39}
\textbf{Metadata}

\begin{itemize}
  \item Question ID: P6-RoFndDvqWN\_P1-WNSub2nd\_sonnet4\_Household Finance\_01
  \item Primary KC: RATIO | Finding divided quantities | dividing a given quantity in a given ratio
  \item Secondary KC: WHOLE NUMBERS | Subtraction | subtracting whole numbers
  \item Topic: Household finance such as income, utility bills, money, interest, savings, instalment, mortgage, financial planning etc.
  \item Grade: Primary 6
\end{itemize}

\textbf{Question}

The Tan family has a monthly household budget of \textdollar4200. They allocate their budget for food, utilities, and savings in the ratio 5:3:2. After paying for food and utilities, how much money do they have left for savings?

\section*{Question 40}
\textbf{Metadata}

\begin{itemize}
  \item Question ID: P6-RoFndRoWN\_P1-WNMul2nd\_sonnet4\_Household Finance\_01
  \item Primary KC: RATIO | Finding ratio | finding the ratio of two or three given whole numbers
  \item Secondary KC: WHOLE NUMBERS | Multiplication | multiplying whole numbers
  \item Topic: Household finance such as income, utility bills, money, interest, savings, instalment, mortgage, financial planning etc.
  \item Grade: Primary 6
\end{itemize}

\textbf{Question}

The Tan family's monthly household expenses consist of three main categories: utilities, groceries, and transportation. In January, they spent \textdollar180 on utilities, \textdollar420 on groceries, and \textdollar240 on transportation. Due to increased usage, their February expenses were 3 times their January amounts for each category. Find the ratio of utilities to groceries to transportation expenses for February.

\section*{Question 41}
\textbf{Metadata}

\begin{itemize}
  \item Question ID: P6-RoFndRoWN\_P6-RoSmpWN\_sonnet4\_Household Finance\_01
  \item Primary KC: RATIO | Finding ratio | finding the ratio of two or three given whole numbers
  \item Secondary KC: RATIO | Simplifying | expressing a ratio in its simplest form
  \item Topic: Household finance such as income, utility bills, money, interest, savings, instalment, mortgage, financial planning etc.
  \item Grade: Primary 6
\end{itemize}

\textbf{Question}

The Lim family's monthly household expenses consist of three main categories: utilities, groceries, and transportation. In January, they spent \textdollar240 on utilities, \textdollar360 on groceries, and \textdollar120 on transportation. Mrs. Lim wants to understand the spending pattern by finding the ratio of their expenses.

(a) Find the ratio of utilities expenses to groceries expenses to transportation expenses.
(b) Express this ratio in its simplest form.

\section*{Question 42}
\textbf{Metadata}

\begin{itemize}
  \item Question ID: P6-RoFndTmWN\_P1-WNSub2nd\_sonnet4\_Household Finance\_01
  \item Primary KC: RATIO | Finding a missing term | finding the missing term in a pair of equivalent ratios
  \item Secondary KC: WHOLE NUMBERS | Subtraction | subtracting whole numbers
  \item Topic: Household finance such as income, utility bills, money, interest, savings, instalment, mortgage, financial planning etc.
  \item Grade: Primary 6
\end{itemize}

\textbf{Question}

Mrs. Tan is planning her monthly household budget. She allocates her monthly income in the ratio of savings to expenses as $3:7$. Last month, she saved \textdollar900. This month, she wants to increase her total monthly income so that she can save \textdollar1200 while keeping the same ratio of savings to expenses. However, her current monthly income is \textdollar400 less than what she needs for this goal. What is Mrs. Tan's current monthly income?

\section*{Question 43}
\textbf{Metadata}

\begin{itemize}
  \item Question ID: O1-RoRepFr\_P2-FrAdd2nd\_sonnet4\_Household Finance\_01
  \item Primary KC: RATIO | Representation and concept | ratios involving fractions
  \item Secondary KC: FRACTIONS | Addition | adding fractions
  \item Topic: Household finance such as income, utility bills, money, interest, savings, instalment, mortgage, financial planning etc.
  \item Grade: Secondary O-level 1
\end{itemize}

\textbf{Question}

The Tan family's monthly household expenses are divided into three categories: utilities, groceries, and savings. The ratio of utilities to groceries to savings is $\frac{2}{3} : \frac{3}{4} : \frac{1}{2}$. If the family spends \textdollar900 on utilities each month, find:

(a) How much the family spends on groceries each month.

(b) How much the family saves each month.

(c) The total amount spent on groceries and savings combined.

\section*{Question 44}
\textbf{Metadata}

\begin{itemize}
  \item Question ID: O1-RoRepFr\_P5-FrMul2nd\_sonnet4\_Household Finance\_01
  \item Primary KC: RATIO | Representation and concept | ratios involving fractions
  \item Secondary KC: FRACTIONS | Multiplication | fraction multiplication
  \item Topic: Household finance such as income, utility bills, money, interest, savings, instalment, mortgage, financial planning etc.
  \item Grade: Secondary O-level 1
\end{itemize}

\textbf{Question}

The Chen family allocates their monthly income for household expenses in the ratio $3\frac{1}{2} : 2\frac{1}{4} : 1\frac{1}{3}$ for housing, food, and savings respectively. If the family's total monthly income is \textdollar4800, find:

(a) How much money is allocated for each category?

(b) If the family decides to increase their food budget by $\frac{1}{6}$ of the original food allocation, what will be the new food budget?

\section*{Question 45}
\textbf{Metadata}

\begin{itemize}
  \item Question ID: O1-RoRepFr\_O1-RoSmpFr\_sonnet4\_Household Finance\_01
  \item Primary KC: RATIO | Representation and concept | ratios involving fractions
  \item Secondary KC: RATIO | Simplifying | converting a ratio involving fractions to its simplest form
  \item Topic: Household finance such as income, utility bills, money, interest, savings, instalment, mortgage, financial planning etc.
  \item Grade: Secondary O-level 1
\end{itemize}

\textbf{Question}

The Chen family plans their monthly budget by allocating their income in the following ratio for three categories: savings, household expenses, and entertainment. They save $\frac{1}{4}$ of their income, spend $\frac{2}{5}$ of their income on household expenses, and use the remaining amount for entertainment. Find the ratio of savings to household expenses to entertainment in its simplest form.

\section*{Question 46}
\textbf{Metadata}

\begin{itemize}
  \item Question ID: O1-RoRepDc\_P4-DcAdd2nd\_sonnet4\_Household Finance\_01
  \item Primary KC: RATIO | Representation and concept | ratios involving decimals
  \item Secondary KC: DECIMALS | Addition | adding decimals
  \item Topic: Household finance such as income, utility bills, money, interest, savings, instalment, mortgage, financial planning etc.
  \item Grade: Secondary O-level 1
\end{itemize}

\textbf{Question}

Mrs. Lim is planning her monthly household budget. Her monthly income is \textdollar4200. She allocates her income in the ratio of expenses to savings as $3.5 : 1.5$. Her monthly expenses consist of rent, utilities, and groceries in the ratio $2.8 : 0.7 : 1.5$. If her utility bill increased by \textdollar45.50 this month due to higher electricity usage, what is her new total monthly expenses?

\section*{Question 47}
\textbf{Metadata}

\begin{itemize}
  \item Question ID: O1-RoRepDc\_O1-RoSmpDc\_sonnet4\_Household Finance\_01
  \item Primary KC: RATIO | Representation and concept | ratios involving decimals
  \item Secondary KC: RATIO | Simplifying | converting a ratio involving decimals to its simplest form
  \item Topic: Household finance such as income, utility bills, money, interest, savings, instalment, mortgage, financial planning etc.
  \item Grade: Secondary O-level 1
\end{itemize}

\textbf{Question}

The Lim family's monthly household expenses are distributed among different categories. They spend \textdollar1200.50 on groceries, \textdollar800.25 on utilities, and \textdollar2400.75 on mortgage payments. Find the ratio of their grocery expenses to utility expenses to mortgage payments, and express this ratio in its simplest form.

\section*{Question 48}
\textbf{Metadata}

\begin{itemize}
  \item Question ID: O1-PcRep2q\_O1-PcCnv2Fr\_sonnet4\_Household Finance\_01
  \item Primary KC: PERCENTAGE | Representation and concept | comparing two quantities by percentage
  \item Secondary KC: PERCENTAGE | Conversion to fraction | expressing percentage as a fraction
  \item Topic: Household finance such as income, utility bills, money, interest, savings, instalment, mortgage, financial planning etc.
  \item Grade: Secondary O-level 1
\end{itemize}

\textbf{Question}

Sarah's family is planning their monthly budget. Their total monthly income is \textdollar4800. They spend \textdollar1440 on rent and \textdollar960 on food expenses. Sarah wants to compare these two major expenses as percentages of their total income and then express these percentages as fractions in their simplest form. What percentage of their total income is spent on rent, and what percentage is spent on food? Express both percentages as fractions in their simplest form.

\section*{Question 49}
\textbf{Metadata}

\begin{itemize}
  \item Question ID: O1-PcFndRslt\_P1-WNAdd2nd\_sonnet4\_Household Finance\_01
  \item Primary KC: PERCENTAGE | Finding result after change | increasing/decreasing a quantity by a given percentage
  \item Secondary KC: WHOLE NUMBERS | Addition | adding whole numbers
  \item Topic: Household finance such as income, utility bills, money, interest, savings, instalment, mortgage, financial planning etc.
  \item Grade: Secondary O-level 1
\end{itemize}

\textbf{Question}

The Tan family's monthly household expenses consist of utilities, groceries, and transportation costs. In January, their utilities bill was \textdollar180, groceries cost \textdollar450, and transportation expenses were \textdollar320. Due to rising costs, the family expects their February expenses to increase by $15\%$ across all categories. Calculate the total amount the Tan family should budget for these three expense categories in February.

\section*{Question 50}
\textbf{Metadata}

\begin{itemize}
  \item Question ID: O1-PcFndRslt\_P1-WNMul2nd\_sonnet4\_Household Finance\_01
  \item Primary KC: PERCENTAGE | Finding result after change | increasing/decreasing a quantity by a given percentage
  \item Secondary KC: WHOLE NUMBERS | Multiplication | multiplying whole numbers
  \item Topic: Household finance such as income, utility bills, money, interest, savings, instalment, mortgage, financial planning etc.
  \item Grade: Secondary O-level 1
\end{itemize}

\textbf{Question}

Sarah's family has a monthly electricity bill of \textdollar180. Due to energy conservation efforts, they managed to reduce their electricity consumption by $15\%$ this month. However, the utility company also increased the electricity rate by $8\%$ for all customers. If Sarah's family maintains their reduced consumption level for the next $6$ months, what will be their total electricity bill for these $6$ months?

\section*{Question 51}
\textbf{Metadata}

\begin{itemize}
  \item Question ID: O1-PcRepRvs\_O1-PcCnv2Fr\_sonnet4\_Household Finance\_01
  \item Primary KC: PERCENTAGE | Representation and concept | reverse percentages
  \item Secondary KC: PERCENTAGE | Conversion to fraction | expressing percentage as a fraction
  \item Topic: Household finance such as income, utility bills, money, interest, savings, instalment, mortgage, financial planning etc.
  \item Grade: Secondary O-level 1
\end{itemize}

\textbf{Question}

Mrs. Tan's monthly household budget includes various expenses. She spends \textdollar480 on groceries, which represents 30\% of her total monthly income. She also allocates 25\% of her total monthly income for utilities and housing costs. What is Mrs. Tan's total monthly income, and what fraction of her income does she spend on utilities and housing costs combined?

\section*{Question 52}
\textbf{Metadata}

\begin{itemize}
  \item Question ID: O1-AgRepExSq\_O1-AgEvlEx\_sonnet4\_Household Finance\_01
  \item Primary KC: ALGEBRA | Representation and concept | translation of simple real-world situations into quadratic algebraic expressions
  \item Secondary KC: ALGEBRA | Evaluation | evaluation of algebraic expressions and formulae
  \item Topic: Household finance such as income, utility bills, money, interest, savings, instalment, mortgage, financial planning etc.
  \item Grade: Secondary O-level 1
\end{itemize}

\textbf{Question}

Sarah is planning to save money for a new laptop. She starts with an initial amount of \textdollar200 in her savings account. Each month, she saves \textdollar50 from her allowance. However, due to bank fees, she loses \textdollar2 for every month that has passed (so \textdollar2 in the first month, \textdollar4 in the second month, etc.). Let $n$ represent the number of months since she started saving.

(a) Write a quadratic expression to represent the total amount of money Sarah has after $n$ months.

(b) How much money will Sarah have after 6 months?

\section*{Question 53}
\textbf{Metadata}

\begin{itemize}
  \item Question ID: O1-AgRepnth\_O1-AgEvlEx\_sonnet4\_Household Finance\_01
  \item Primary KC: ALGEBRA | Representation and concept | recognising and representing patterns/relationships by finding an algebraic expression for the nth term
  \item Secondary KC: ALGEBRA | Evaluation | evaluation of algebraic expressions and formulae
  \item Topic: Household finance such as income, utility bills, money, interest, savings, instalment, mortgage, financial planning etc.
  \item Grade: Secondary O-level 1
\end{itemize}

\textbf{Question}

Sarah is planning her monthly savings based on her income pattern. She notices that her monthly income follows a specific pattern. In January, she earned \textdollar800. In February, she earned \textdollar950. In March, she earned \textdollar1100. In April, she earned \textdollar1250. This pattern continues throughout the year.

(a) Find an algebraic expression for Sarah's income in the $n$th month.

(b) Sarah plans to save 20% of her income each month. Find an algebraic expression for her monthly savings in the $n$th month.

(c) Calculate how much Sarah will save in the 8th month (August).

\section*{Question 54}
\textbf{Metadata}

\begin{itemize}
  \item Question ID: O2-RoRepDP\_P1-WNDiv2nd\_sonnet4\_Household Finance\_01
  \item Primary KC: RATIO | Representation and concept | direct proportion
  \item Secondary KC: WHOLE NUMBERS | Division | dividing whole numbers
  \item Topic: Household finance such as income, utility bills, money, interest, savings, instalment, mortgage, financial planning etc.
  \item Grade: Secondary O-level 2
\end{itemize}

\textbf{Question}

The Chen family's monthly household expenses are divided into three categories in the ratio $3:2:1$ for food, utilities, and entertainment respectively. If they spend a total of \textdollar1800 per month on these expenses, how much do they spend on each category? Additionally, the family decides to save money by reducing their food expenses. If they reduce their food expenses by \textdollar150 per month and want to maintain the same ratio for the remaining expenses, what should be their new monthly budget for utilities and entertainment?

\section*{Question 55}
\textbf{Metadata}

\begin{itemize}
  \item Question ID: O2-AgSlvIneq\_O2-AgRepIneq\_sonnet4\_Household Finance\_01
  \item Primary KC: ALGEBRA | Solving | solving simple linear inequalities with one variable
  \item Secondary KC: ALGEBRA | Representation and concept | translation of simple real-world situations to simple linear inequalities with one variable
  \item Topic: Household finance such as income, utility bills, money, interest, savings, instalment, mortgage, financial planning etc.
  \item Grade: Secondary O-level 2
\end{itemize}

\textbf{Question}

Sarah is planning her monthly budget for utilities. Her electricity bill is \textdollar45 per month, and her water bill is \textdollar25 per month. She also needs to pay for internet service, which costs \textdollar$x$ per month. If Sarah wants to keep her total monthly utility expenses below \textdollar120, what is the maximum amount she can spend on internet service? Write an inequality to represent this situation and solve it.

\section*{Question 56}
\textbf{Metadata}

\begin{itemize}
  \item Question ID: O2-AgSlvSq1v\_O1-AgRepEq\_sonnet4\_Household Finance\_01
  \item Primary KC: ALGEBRA | Solving | solving quadratic equations in one variable
  \item Secondary KC: ALGEBRA | Representation and concept | translation of simple real-world situations to equations
  \item Topic: Household finance such as income, utility bills, money, interest, savings, instalment, mortgage, financial planning etc.
  \item Grade: Secondary O-level 2
\end{itemize}

\textbf{Question}

Sarah is planning her monthly budget for household expenses. She knows that her total monthly income minus her fixed expenses (rent, utilities, and groceries) leaves her with \textdollar480 for savings and discretionary spending. She wants to save a certain amount each month, and spend twice that amount on discretionary items like entertainment and dining out. However, she also needs to set aside an additional \textdollar20 more than her savings amount for an emergency fund. If the sum of her savings amount and her discretionary spending equals \textdollar320, how much should Sarah allocate for her monthly savings?

\section*{Question 57}
\textbf{Metadata}

\begin{itemize}
  \item Question ID: O2-SPFndmdn\_O3-SPFndrng\_sonnet4\_Household Finance\_01
  \item Primary KC: STATISTICS AND PROBABILITY | Finding median | Finding median for a set of data
  \item Secondary KC: STATISTICS AND PROBABILITY | Finding range | finding range as measures of spread for a set of data 
  \item Topic: Household finance such as income, utility bills, money, interest, savings, instalment, mortgage, financial planning etc.
  \item Grade: Secondary O-level 2
\end{itemize}

\textbf{Question}

Mrs. Tan tracks her family's monthly electricity bills for the past 9 months to better plan their household budget. The monthly bills (in dollars) are: \textdollar85, \textdollar92, \textdollar78, \textdollar105, \textdollar88, \textdollar95, \textdollar82, \textdollar90, \textdollar98. 

(a) Find the median monthly electricity bill.

(b) Find the range of the monthly electricity bills.

(c) Mrs. Tan wants to set aside money each month for electricity expenses. She decides to budget an amount equal to the median plus half the range. How much should she budget monthly for electricity?

\section*{Question 58}
\textbf{Metadata}

\begin{itemize}
  \item Question ID: O2-SPFndmean\_O3-BPRepSN\_sonnet4\_Household Finance\_01
  \item Primary KC: STATISTICS AND PROBABILITY | Finding mean deviation | calculation of the mean for a set of data
  \item Secondary KC: BASE AND POWER | Representation and concept  | use of standard form Ax10^n , where n is an integer, and 1<= A<= 10
  \item Topic: Household finance such as income, utility bills, money, interest, savings, instalment, mortgage, financial planning etc.
  \item Grade: Secondary O-level 2
\end{itemize}

\textbf{Question}

The Chan family is analyzing their monthly household expenses over the past 6 months to better plan their budget. Their monthly expenses (in dollars) were: \textdollar12000, \textdollar15000, \textdollar9000, \textdollar18000, \textdollar11000, and \textdollar13000.

(a) Calculate the mean monthly expense for the Chan family.

(b) Find the mean deviation of their monthly expenses.

(c) Express the mean deviation in standard form.

\section*{Question 59}
\textbf{Metadata}

\begin{itemize}
  \item Question ID: O3-BPOpr\_O3-BPRepNegI\_sonnet4\_Household Finance\_01
  \item Primary KC: BASE AND POWER | Operations | laws of indices
  \item Secondary KC: BASE AND POWER | Representation and concept  | negative indices
  \item Topic: Household finance such as income, utility bills, money, interest, savings, instalment, mortgage, financial planning etc.
  \item Grade: Secondary O-level 3/4
\end{itemize}

\textbf{Question}

Sarah is planning her household budget and notices that her monthly expenses follow certain patterns. Her electricity bill decreases each month according to the formula $E = 120 \times 2^{-n}$ dollars, where $n$ is the number of months since she installed solar panels. Her water bill follows the pattern $W = 80 \times 3^{2-n}$ dollars for the same time period. 

(a) Calculate her electricity bill in the 3rd month after installing solar panels.

(b) Calculate her water bill in the 3rd month after installing solar panels.

(c) Find her total utility bill (electricity + water) for the 3rd month. Express your answer as a fraction in its simplest form.

\section*{Question 60}
\textbf{Metadata}

\begin{itemize}
  \item Question ID: O3-MXMulSM\_O3-MXAdd\_sonnet4\_Household Finance\_01
  \item Primary KC: MATRICES | Multiplication | product of a scalar quantity and a matrix
  \item Secondary KC: MATRICES | Addition | addition of matrices
  \item Topic: Household finance such as income, utility bills, money, interest, savings, instalment, mortgage, financial planning etc.
  \item Grade: Secondary O-level 3/4
\end{itemize}

\textbf{Question}

The Tan family tracks their monthly household expenses using matrices. Their expenses for utilities, groceries, and transportation over the past 3 months are recorded in matrix $A$:

$A = \begin{pmatrix} 120 & 450 & 200 \\ 135 & 480 & 220 \\ 110 & 420 & 180 \end{pmatrix}$

where each row represents a month (January, February, March) and each column represents utilities, groceries, and transportation expenses in dollars respectively.

Due to inflation, the family expects all their expenses to increase by 15% next quarter. Additionally, they plan to add fixed monthly savings of \textdollar50 for utilities (emergency fund), \textdollar100 for groceries (bulk buying), and \textdollar30 for transportation (maintenance fund) to their budget.

(a) Find the matrix representing their expected expenses after the 15% increase.

(b) Find the matrix representing their total monthly budget (expected expenses plus additional savings) for the next quarter.

\section*{Question 61}
\textbf{Metadata}

\begin{itemize}
  \item Question ID: O3-MXSub\_O3-MXAdd\_sonnet4\_Household Finance\_01
  \item Primary KC: MATRICES | Subtraction | subtraction of matrices
  \item Secondary KC: MATRICES | Addition | addition of matrices
  \item Topic: Household finance such as income, utility bills, money, interest, savings, instalment, mortgage, financial planning etc.
  \item Grade: Secondary O-level 3/4
\end{itemize}

\textbf{Question}

The Tan family tracks their monthly household expenses using matrices. They record their expenses for utilities, groceries, and transportation costs for the first quarter of the year.

In January, their expenses were represented by matrix $A = \begin{pmatrix} 180 & 450 & 320 \\ 200 & 380 & 290 \end{pmatrix}$, where the first row represents Week 1-2 expenses and the second row represents Week 3-4 expenses. The three columns represent utilities, groceries, and transportation costs respectively (all in dollars).

In February, their expenses were represented by matrix $B = \begin{pmatrix} 165 & 420 & 340 \\ 185 & 400 & 310 \end{pmatrix}$.

In March, their expenses were represented by matrix $C = \begin{pmatrix} 175 & 465 & 305 \\ 195 & 415 & 285 \end{pmatrix}$.

The family wants to analyze their spending patterns to plan their budget for the second quarter.

(a) Find the total expenses for February and March combined.

(b) Find the difference between the combined February-March expenses and January expenses. What does this result tell the family about their spending changes?

\section*{Question 62}
\textbf{Metadata}

\begin{itemize}
  \item Question ID: O3-SPAddProb\_O2-SPRepPrSE\_sonnet4\_Household Finance\_01
  \item Primary KC: STATISTICS AND PROBABILITY | Addition | addition of probabilities
  \item Secondary KC: STATISTICS AND PROBABILITY | Representation and concept | probability of single events
  \item Topic: Household finance such as income, utility bills, money, interest, savings, instalment, mortgage, financial planning etc.
  \item Grade: Secondary O-level 3/4
\end{itemize}

\textbf{Question}

Mrs. Tan is analyzing her monthly household expenses to better manage her family's finances. She categorizes her expenses into utilities, groceries, and transportation. Based on her past records, she finds that in any given month, the probability that her utility bill exceeds her budget is $\frac{1}{4}$, the probability that her grocery expenses exceed her budget is $\frac{1}{3}$, and the probability that her transportation costs exceed her budget is $\frac{1}{6}$. These events are mutually exclusive as she carefully manages each category separately. What is the probability that at least one of these three expense categories will exceed her budget in a given month?

\section*{Question 63}
\textbf{Metadata}

\begin{itemize}
  \item Question ID: O3-SPAddProb\_O3-SPFndPrCE\_sonnet4\_Household Finance\_01
  \item Primary KC: STATISTICS AND PROBABILITY | Addition | addition of probabilities
  \item Secondary KC: STATISTICS AND PROBABILITY | Finding probability | probability of simple combined events
  \item Topic: Household finance such as income, utility bills, money, interest, savings, instalment, mortgage, financial planning etc.
  \item Grade: Secondary O-level 3/4
\end{itemize}

\textbf{Question}

Mrs. Tan is analyzing her family's monthly utility bill payments over the past year. She found that the probability of paying the electricity bill late is $\frac{1}{6}$, and the probability of paying the water bill late is $\frac{1}{8}$. The two bill payments are independent of each other.

(a) Find the probability that Mrs. Tan pays at least one of the two bills late in any given month.

(b) Mrs. Tan wants to set up automatic payments to reduce late payments. If automatic payment reduces the probability of late payment for each bill by half, find the new probability that she pays at least one bill late.

\section*{Question 64}
\textbf{Metadata}

\begin{itemize}
  \item Question ID: O3-SPMulProb\_O3-SPFndPrCE\_sonnet4\_Household Finance\_01
  \item Primary KC: STATISTICS AND PROBABILITY | Multiplication | multiplication of probabilities
  \item Secondary KC: STATISTICS AND PROBABILITY | Finding probability | probability of simple combined events
  \item Topic: Household finance such as income, utility bills, money, interest, savings, instalment, mortgage, financial planning etc.
  \item Grade: Secondary O-level 3/4
\end{itemize}

\textbf{Question}

The Tan family is reviewing their monthly household expenses to create a better financial plan. They have identified that in any given month, there is a $\frac{3}{4}$ probability that their electricity bill will exceed \textdollar120, and a $\frac{2}{5}$ probability that their water bill will exceed \textdollar80. Based on their past records, these two events are independent of each other.

(a) What is the probability that in a randomly selected month, both their electricity bill will exceed \textdollar120 AND their water bill will exceed \textdollar80?

(b) What is the probability that in a randomly selected month, at least one of these bills (electricity or water) will exceed their respective thresholds?

\end{document}
