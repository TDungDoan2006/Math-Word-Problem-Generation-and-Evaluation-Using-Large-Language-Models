\documentclass{article}
\usepackage[utf8]{inputenc}
\usepackage{amsmath}
\usepackage{amsfonts}
\usepackage{amssymb}
\usepackage{graphicx}
\usepackage{hyperref}
\title{'TD Solutions household finance v3 CLAUDE '}
\author{Tien Dung Doan}
\begin{document}
\maketitle
\section*{Question 1}
\textbf{Metadata}

\begin{itemize}
  \item Question ID: P3-WNAdd4d\_P1-WNCmp\_sonnet4\_Household Finance\_01
  \item Primary KC: WHOLE NUMBERS | Addition | adding whole numbers up to 4 digits
  \item Secondary KC: WHOLE NUMBERS | Comparison and ordering | comparing and ordering whole numbers
  \item Topic: Household finance such as income, utility bills, money, interest, savings, instalment, mortgage, financial planning etc.
  \item Grade: Primary 3
\end{itemize}

\textbf{Solution}

(a) To find the total monthly income, I need to add Mr. Tan's income and Mrs. Tan's income.

Total monthly income = \textdollar2845 + \textdollar1967

$\begin{align}
2845 \\
+ 1967 \\
\hline
4812
\end{align}$

Total monthly income = \textdollar4812

(b) To find the total monthly expenses, I need to add all the expenses.

Total monthly expenses = \textdollar1250 + \textdollar185 + \textdollar720 + \textdollar395

$\begin{align}
1250 \\
185 \\
720 \\
+ 395 \\
\hline
2550
\end{align}$

Total monthly expenses = \textdollar2550

(c) To compare their income and expenses, I need to compare \textdollar4812 and \textdollar2550.

Since $4812 > 2550$, their total monthly income is greater than their total monthly expenses.

Yes, they have enough income to cover all their expenses.

\section*{Question 2}
\textbf{Metadata}

\begin{itemize}
  \item Question ID: P3-WNSub4d\_P1-WNAdd2nd\_sonnet4\_Household Finance\_01
  \item Primary KC: WHOLE NUMBERS | Subtraction | subtracting whole numbers up to 4 digits
  \item Secondary KC: WHOLE NUMBERS | Addition | adding whole numbers
  \item Topic: Household finance such as income, utility bills, money, interest, savings, instalment, mortgage, financial planning etc.
  \item Grade: Primary 3
\end{itemize}

\textbf{Solution}

Step 1: Find the total income of the Tan family.
Mr. Tan's income + Mrs. Tan's income = \textdollar3250 + \textdollar2180 = \textdollar5430

Step 2: Find the total expenses of the Tan family.
Rent + Groceries + Utilities + Transport = \textdollar1820 + \textdollar650 + \textdollar380 + \textdollar290 = \textdollar3140

Step 3: Find how much money is left for savings.
Total income - Total expenses = \textdollar5430 - \textdollar3140 = \textdollar2290

Therefore, the Tan family has \textdollar2290 left for savings.

\section*{Question 3}
\textbf{Metadata}

\begin{itemize}
  \item Question ID: P3-WNDivRmd3d\_P1-WNSub2nd\_sonnet4\_Household Finance\_01
  \item Primary KC: WHOLE NUMBERS | Division | dividing whole numbers up to 3 digits by 1 digit with remainder 
  \item Secondary KC: WHOLE NUMBERS | Subtraction | subtracting whole numbers
  \item Topic: Household finance such as income, utility bills, money, interest, savings, instalment, mortgage, financial planning etc.
  \item Grade: Primary 3
\end{itemize}

\textbf{Solution}

First, I need to find out how much money Mrs. Lim has left after buying oranges.

Money left after buying oranges = \textdollar348 - \textdollar25 = \textdollar323

Next, I need to divide this remaining money equally among 7 grandchildren.

\textdollar323 ÷ 7 = \textdollar46 remainder \textdollar1

Let me check: $46 × 7 = 322$, and $323 - 322 = 1$

Since there is a remainder of \textdollar1, Mrs. Lim cannot divide the money exactly equally. Each grandchild will receive \textdollar46, and there will be \textdollar1 left over.

Therefore, each grandchild will receive \textdollar46.

\section*{Question 4}
\textbf{Metadata}

\begin{itemize}
  \item Question ID: P3-WNMul3d1d\_P1-WNCmp\_sonnet4\_Household Finance\_01
  \item Primary KC: WHOLE NUMBERS | Multiplication | multiplying whole numbers up to 3 digits by 1 digit
  \item Secondary KC: WHOLE NUMBERS | Comparison and ordering | comparing and ordering whole numbers
  \item Topic: Household finance such as income, utility bills, money, interest, savings, instalment, mortgage, financial planning etc.
  \item Grade: Primary 3
\end{itemize}

\textbf{Solution}

First, I need to find the electricity bill for Apartment B.

Apartment B's bill = 3 times Apartment A's bill
Apartment B's bill = $3 \times 138$
Apartment B's bill = \textdollar414

Now I can compare all three apartments:
- Apartment A: \textdollar138
- Apartment B: \textdollar414  
- Apartment C: \textdollar425

Comparing the amounts: $138 < 414 < 425$

Therefore:
- Apartment C has the highest electricity bill at \textdollar425
- Apartment A has the lowest electricity bill at \textdollar138
- Apartment B's electricity bill is \textdollar414

\section*{Question 5}
\textbf{Metadata}

\begin{itemize}
  \item Question ID: P3-WNDiv3d1d\_P1-WNCmp\_sonnet4\_Household Finance\_01
  \item Primary KC: WHOLE NUMBERS | Division | dividing whole numbers up to 3 digits by 1 digit
  \item Secondary KC: WHOLE NUMBERS | Comparison and ordering | comparing and ordering whole numbers
  \item Topic: Household finance such as income, utility bills, money, interest, savings, instalment, mortgage, financial planning etc.
  \item Grade: Primary 3
\end{itemize}

\textbf{Solution}

Step 1: Find how much Mrs. Chen allocates for each utility bill.
Amount saved for utilities = \textdollar648
Number of utility bills = 3
Amount for each utility bill = \textdollar648 ÷ 3 = \textdollar216

Step 2: Find how much Mrs. Chen allocates for groceries each week.
Amount saved for groceries = \textdollar756
Number of weeks = 4
Amount for groceries each week = \textdollar756 ÷ 4 = \textdollar189

Step 3: Compare the weekly expenses.
Total weekly utilities = \textdollar216 × 3 ÷ 4 = \textdollar648 ÷ 4 = \textdollar162
Weekly groceries = \textdollar189

Since \textdollar189 > \textdollar162, groceries cost more per week.

Answer: Mrs. Chen will allocate \textdollar216 for each utility bill and \textdollar189 for groceries each week. The weekly grocery expense is higher than the weekly utility expense.

\section*{Question 6}
\textbf{Metadata}

\begin{itemize}
  \item Question ID: P3-FrSubRl12\_P2-FrCmp\_sonnet4\_Household Finance\_01
  \item Primary KC: FRACTIONS | Subtraction | subtracting two related fractions within one whole with denominators of given fractions not exceeding 12
  \item Secondary KC: FRACTIONS | Comparison and ordering | comparing and ordering fractions
  \item Topic: Household finance such as income, utility bills, money, interest, savings, instalment, mortgage, financial planning etc.
  \item Grade: Primary 3
\end{itemize}

\textbf{Solution}

Step 1: Find how much of her income Mrs. Tan has left from essential expenses after buying groceries.

Amount left = Essential expenses allocation - Grocery expenses
Amount left = $\frac{5}{8} - \frac{3}{8} = \frac{5-3}{8} = \frac{2}{8}$

Simplifying: $\frac{2}{8} = \frac{1}{4}$

Step 2: Compare the remaining amount with grocery expenses.

Remaining amount = $\frac{1}{4}$
Grocery expenses = $\frac{3}{8}$

To compare $\frac{1}{4}$ and $\frac{3}{8}$, convert to the same denominator:
$\frac{1}{4} = \frac{2}{8}$

Comparing $\frac{2}{8}$ and $\frac{3}{8}$:
Since $2 < 3$, we have $\frac{2}{8} < \frac{3}{8}$

Therefore, Mrs. Tan has $\frac{1}{4}$ of her monthly income left from the essential expenses allocation after buying groceries, and the grocery expenses are larger than the remaining amount.

\section*{Question 7}
\textbf{Metadata}

\begin{itemize}
  \item Question ID: P4-WNMul4d1d\_P1-WNCmp\_sonnet4\_Household Finance\_01
  \item Primary KC: WHOLE NUMBERS | Multiplication | multiplying whole numbers up to 4 digits by 1 digit or up to 3 digits by 2 digits
  \item Secondary KC: WHOLE NUMBERS | Comparison and ordering | comparing and ordering whole numbers
  \item Topic: Household finance such as income, utility bills, money, interest, savings, instalment, mortgage, financial planning etc.
  \item Grade: Primary 4
\end{itemize}

\textbf{Solution}

**Part (a): Calculate the total electricity cost for each month**

**January:**
- Daily usage: $24$ units
- Number of days: $31$ days
- Total units used: $24 \times 31 = 744$ units
- Cost per unit: \textdollar3
- Total cost: $744 \times 3 = 2232$
- January electricity cost: \textdollar2232

**February:**
- Daily usage: $28$ units
- Number of days: $28$ days
- Total units used: $28 \times 28 = 784$ units
- Cost per unit: \textdollar3
- Total cost: $784 \times 3 = 2352$
- February electricity cost: \textdollar2352

**March:**
- Daily usage: $26$ units
- Number of days: $31$ days
- Total units used: $26 \times 31 = 806$ units
- Cost per unit: \textdollar3
- Total cost: $806 \times 3 = 2418$
- March electricity cost: \textdollar2418

**Part (b): Arrange the months from highest to lowest electricity cost**

Comparing the three costs:
- January: \textdollar2232
- February: \textdollar2352
- March: \textdollar2418

Since $2418 > 2352 > 2232$, the order from highest to lowest electricity cost is:
**March, February, January**

\section*{Question 8}
\textbf{Metadata}

\begin{itemize}
  \item Question ID: P4-WNMul4d1d\_P4-WNRnd5d\_sonnet4\_Household Finance\_01
  \item Primary KC: WHOLE NUMBERS | Multiplication | multiplying whole numbers up to 4 digits by 1 digit or up to 3 digits by 2 digits
  \item Secondary KC: WHOLE NUMBERS | Rounding | rounding whole numbers up to 100000 to the nearest 10, 100 or 1000 
  \item Topic: Household finance such as income, utility bills, money, interest, savings, instalment, mortgage, financial planning etc.
  \item Grade: Primary 4
\end{itemize}

\textbf{Solution}

Step 1: Round the electricity bill to the nearest \textdollar10.
\textdollar127 rounded to the nearest \textdollar10 is \textdollar130.

Step 2: Round the water bill to the nearest \textdollar10.
\textdollar89 rounded to the nearest \textdollar10 is \textdollar90.

Step 3: Find the total estimated monthly utility cost.
Estimated monthly utility cost $= \textdollar130 + \textdollar90 = \textdollar220$

Step 4: Calculate the estimated yearly utility cost.
Estimated yearly utility cost $= \textdollar220 \times 12 = \textdollar2640$

Therefore, Mrs. Lim's estimated total utility cost for the year is \textdollar2640.

\section*{Question 9}
\textbf{Metadata}

\begin{itemize}
  \item Question ID: P4-WNDiv4d1d\_P1-WNSub2nd\_sonnet4\_Household Finance\_01
  \item Primary KC: WHOLE NUMBERS | Division | dividing whole numbers up to 4 digits by 1 digit
  \item Secondary KC: WHOLE NUMBERS | Subtraction | subtracting whole numbers
  \item Topic: Household finance such as income, utility bills, money, interest, savings, instalment, mortgage, financial planning etc.
  \item Grade: Primary 4
\end{itemize}

\textbf{Solution}

Step 1: Find out how much money is available for investment.
Money available for investment = Total savings - Emergency expenses
Money available for investment = \textdollar8736 - \textdollar528 = \textdollar8208

Step 2: Divide the investment money equally among 6 funds.
Money invested in each fund = \textdollar8208 ÷ 6

Using long division:
$8208 ÷ 6 = 1368$

Therefore, the Tan family will invest \textdollar1368 in each fund.

\section*{Question 10}
\textbf{Metadata}

\begin{itemize}
  \item Question ID: P4-WNDiv4d1d\_P4-WNRnd5d\_sonnet4\_Household Finance\_01
  \item Primary KC: WHOLE NUMBERS | Division | dividing whole numbers up to 4 digits by 1 digit
  \item Secondary KC: WHOLE NUMBERS | Rounding | rounding whole numbers up to 100000 to the nearest 10, 100 or 1000 
  \item Topic: Household finance such as income, utility bills, money, interest, savings, instalment, mortgage, financial planning etc.
  \item Grade: Primary 4
\end{itemize}

\textbf{Solution}

Step 1: Find the exact monthly savings amount before rounding.
To find how much they should save each month, I need to divide the total amount by the number of months.
$2436 \div 6 = 406$

So the Wong family should save \textdollar406 each month before rounding.

Step 2: Round the monthly amount to the nearest \textdollar10.
To round \textdollar406 to the nearest \textdollar10, I look at the ones digit.
The ones digit is 6.
Since 6 is greater than or equal to 5, I round up.
$406$ rounded to the nearest $10 = 410$

Therefore, the Wong family should save \textdollar406 each month before rounding, and \textdollar410 each month after rounding to the nearest \textdollar10.

\section*{Question 11}
\textbf{Metadata}

\begin{itemize}
  \item Question ID: P4-FrAddU12\_P2-FrCmp\_sonnet4\_Household Finance\_01
  \item Primary KC: FRACTIONS | Addition | adding unlike fractions with two different denominators not exceeding 12
  \item Secondary KC: FRACTIONS | Comparison and ordering | comparing and ordering fractions
  \item Topic: Household finance such as income, utility bills, money, interest, savings, instalment, mortgage, financial planning etc.
  \item Grade: Primary 4
\end{itemize}

\textbf{Solution}

To find the fraction of monthly income spent on groceries and utility bills combined, I need to add $\frac{1}{3}$ and $\frac{1}{4}$.

Step 1: Find a common denominator for $\frac{1}{3}$ and $\frac{1}{4}$.
The denominators are 3 and 4. The least common multiple of 3 and 4 is 12.

Step 2: Convert both fractions to equivalent fractions with denominator 12.
$\frac{1}{3} = \frac{1 \times 4}{3 \times 4} = \frac{4}{12}$
$\frac{1}{4} = \frac{1 \times 3}{4 \times 3} = \frac{3}{12}$

Step 3: Add the fractions.
$\frac{4}{12} + \frac{3}{12} = \frac{7}{12}$

Therefore, Mrs. Tan spends $\frac{7}{12}$ of her monthly income on groceries and utility bills combined.

Step 4: Compare $\frac{7}{12}$ (groceries and utility bills) with $\frac{7}{12}$ (mortgage payment).
Since both fractions have the same denominator and the same numerator, $\frac{7}{12} = \frac{7}{12}$.

Therefore, Mrs. Tan spends exactly the same amount on groceries and utility bills combined as she does on her mortgage payment.

\section*{Question 12}
\textbf{Metadata}

\begin{itemize}
  \item Question ID: P4-FrSubU12\_P3-FrSmp\_sonnet4\_Household Finance\_01
  \item Primary KC: FRACTIONS | Subtraction | subtracting unlike fractions with two different denominators not exceeding 12
  \item Secondary KC: FRACTIONS | Simplifying | expressing a fraction in its simplest form
  \item Topic: Household finance such as income, utility bills, money, interest, savings, instalment, mortgage, financial planning etc.
  \item Grade: Primary 4
\end{itemize}

\textbf{Solution}

To find the fraction of income left, I need to subtract the fractions used for essential expenses and education fund from the whole income.

Fraction used for essential expenses = $\frac{5}{8}$
Fraction used for education fund = $\frac{1}{6}$

Total fraction used = $\frac{5}{8} + \frac{1}{6}$

To add these fractions, I need a common denominator. The LCM of 8 and 6 is 24.

$\frac{5}{8} = \frac{5 \times 3}{8 \times 3} = \frac{15}{24}$

$\frac{1}{6} = \frac{1 \times 4}{6 \times 4} = \frac{4}{24}$

Total fraction used = $\frac{15}{24} + \frac{4}{24} = \frac{19}{24}$

Fraction left = $1 - \frac{19}{24} = \frac{24}{24} - \frac{19}{24} = \frac{5}{24}$

Since the greatest common factor of 5 and 24 is 1, $\frac{5}{24}$ is already in its simplest form.

Therefore, Mrs. Tan has $\frac{5}{24}$ of her monthly income left for other expenses and savings.

\section*{Question 13}
\textbf{Metadata}

\begin{itemize}
  \item Question ID: P4-DcAdd2d\_P4-DcCmp3d\_sonnet4\_Household Finance\_01
  \item Primary KC: DECIMALS | Addition | adding decimals (up to 2 decimal places)
  \item Secondary KC: DECIMALS | Comparison and ordering | comparing and ordering decimals up to 3 decimal places
  \item Topic: Household finance such as income, utility bills, money, interest, savings, instalment, mortgage, financial planning etc.
  \item Grade: Primary 4
\end{itemize}

\textbf{Solution}

(a) Finding the total utility bill for each month:

January: 
$45.60 + 23.40 = 69.00$
Total utility bill for January = \textdollar69.00

February:
$48.75 + 21.85 = 70.60$
Total utility bill for February = \textdollar70.60

March:
$44.20 + 25.30 = 69.50$
Total utility bill for March = \textdollar69.50

(b) Comparing the total utility bills:
January: \textdollar69.00
February: \textdollar70.60
March: \textdollar69.50

Comparing $69.00$, $70.60$, and $69.50$:
$69.00 < 69.50 < 70.60$

Therefore, arranging from lowest to highest total utility bill:
January, March, February

\section*{Question 14}
\textbf{Metadata}

\begin{itemize}
  \item Question ID: P4-DcSub2d\_P4-DcCnv2Fr\_sonnet4\_Household Finance\_01
  \item Primary KC: DECIMALS | Subtraction | subtracting decimals (up to 2 decimal places)
  \item Secondary KC: DECIMALS | Conversion from decimals to fraction | expressing decimals as fractions
  \item Topic: Household finance such as income, utility bills, money, interest, savings, instalment, mortgage, financial planning etc.
  \item Grade: Primary 4
\end{itemize}

\textbf{Solution}

To find how much money Mrs. Tan saved, I need to subtract the amount she spent from her budgeted amount.

Amount budgeted for electricity = \textdollar45.80
Amount actually spent = \textdollar32.65

Amount saved = \textdollar45.80 - \textdollar32.65

Let me subtract:
$45.80 - 32.65$
$= 13.15$

So Mrs. Tan saved \textdollar13.15.

Now I need to express \textdollar13.15 as a fraction in its simplest form.

\textdollar13.15 = $\frac{1315}{100}$

To simplify this fraction, I need to find the highest common factor of 1315 and 100.
Factors of 1315: 1, 5, 263, 1315
Factors of 100: 1, 2, 4, 5, 10, 20, 25, 50, 100

The highest common factor is 5.

$\frac{1315}{100} = \frac{1315 \div 5}{100 \div 5} = \frac{263}{20}$

Therefore, Mrs. Tan saved $\frac{263}{20}$ dollars from her electricity budget.

\section*{Question 15}
\textbf{Metadata}

\begin{itemize}
  \item Question ID: P4-DcSub2d\_P4-DcRnd3d\_sonnet4\_Household Finance\_01
  \item Primary KC: DECIMALS | Subtraction | subtracting decimals (up to 2 decimal places)
  \item Secondary KC: DECIMALS | Rounding | rounding decimals up to 3 decimal places to the nearest whole number, 1 decimal place and 2 decimal places 
  \item Topic: Household finance such as income, utility bills, money, interest, savings, instalment, mortgage, financial planning etc.
  \item Grade: Primary 4
\end{itemize}

\textbf{Solution}

First, I need to find the total amount Mrs. Tan spent on utility bills by adding all three bills together.

Electricity bill: \textdollar68.75
Water bill: \textdollar29.38
Gas bill: \textdollar45.92

Total spent = $68.75 + 29.38 + 45.92$
$= 98.13 + 45.92$
$= 144.05$

So Mrs. Tan spent \textdollar144.05 on utility bills.

Next, I need to find how much money she has left from her budget by subtracting the total spent from her budget.

Money left = Budget - Total spent
Money left = $150.00 - 144.05$
Money left = $5.95$

Finally, I need to round \textdollar5.95 to the nearest whole number.

Since the digit in the tenths place is 9, which is 5 or greater, I round up.

\textdollar5.95 rounded to the nearest whole number = \textdollar6

Therefore, Mrs. Tan has \textdollar6 left from her utility budget.

\section*{Question 16}
\textbf{Metadata}

\begin{itemize}
  \item Question ID: P4-DcMul2d1d\_P4-DcCnv2Fr\_sonnet4\_Household Finance\_01
  \item Primary KC: DECIMALS | Multiplication | multiplying decimals (up to 2 decimal places) by a 1-digit whole number
  \item Secondary KC: DECIMALS | Conversion from decimals to fraction | expressing decimals as fractions
  \item Topic: Household finance such as income, utility bills, money, interest, savings, instalment, mortgage, financial planning etc.
  \item Grade: Primary 4
\end{itemize}

\textbf{Solution}

To find the total electricity bill, I need to multiply the cost per unit by the number of units used.

Cost per unit = \textdollar0.75
Number of units = 8

Total cost = $0.75 \times 8$

To multiply $0.75 \times 8$:
$0.75 \times 8 = 6.00$

So Mrs. Tan paid \textdollar6.00 for electricity.

Now I need to express \textdollar6.00 as a fraction.

\textdollar6.00 = \textdollar$\frac{6}{1}$ = \textdollar$\frac{6}{1}$

Therefore, Mrs. Tan paid \textdollar$\frac{6}{1}$ or \textdollar6 for electricity last month.

\section*{Question 17}
\textbf{Metadata}

\begin{itemize}
  \item Question ID: P4-DcMul2d1d\_P4-DcAdd2nd\_sonnet4\_Household Finance\_01
  \item Primary KC: DECIMALS | Multiplication | multiplying decimals (up to 2 decimal places) by a 1-digit whole number
  \item Secondary KC: DECIMALS | Addition | adding decimals
  \item Topic: Household finance such as income, utility bills, money, interest, savings, instalment, mortgage, financial planning etc.
  \item Grade: Primary 4
\end{itemize}

\textbf{Solution}

Step 1: Find the total cost for electricity bills over 3 months.
Electricity bill per month = \textdollar23.45
Electricity bills for 3 months = $23.45 \times 3 = \textdollar70.35$

Step 2: Find the total cost for water bills over 3 months.
Water bill per month = \textdollar18.60
Water bills for 3 months = $18.60 \times 3 = \textdollar55.80$

Step 3: Find the total cost for gas bills over 3 months.
Gas bill per month = \textdollar12.75
Gas bills for 3 months = $12.75 \times 3 = \textdollar38.25$

Step 4: Add all the costs together to find the total amount.
Total amount = $70.35 + 55.80 + 38.25 = \textdollar164.40$

Therefore, Mrs. Tan needs to pay \textdollar164.40 for all three types of utility bills over the 3 months.

\section*{Question 18}
\textbf{Metadata}

\begin{itemize}
  \item Question ID: P4-DcDiv2d1d\_P4-DcCmp3d\_sonnet4\_Household Finance\_01
  \item Primary KC: DECIMALS | Division | dividing decimals (up to 2 decimal places) by a 1-digit whole number
  \item Secondary KC: DECIMALS | Comparison and ordering | comparing and ordering decimals up to 3 decimal places
  \item Topic: Household finance such as income, utility bills, money, interest, savings, instalment, mortgage, financial planning etc.
  \item Grade: Primary 4
\end{itemize}

\textbf{Solution}

Step 1: Find each person's share for the current month by dividing the total bill amount by 3.

Each person's share = \textdollar84.60 \div 3

To divide \textdollar84.60 by 3:
$84.60 \div 3 = 28.20$

So each person's share for the current month is \textdollar28.20.

Step 2: Compare the current month's share with the previous month's share.

Current month's share per person: \textdollar28.20 = \textdollar28.200
Previous month's share per person: \textdollar27.850

Comparing the decimal numbers:
$28.200$ and $27.850$

Since $28.200 > 27.850$, the current month's share per person is greater than the previous month's share.

Therefore, each person needs to pay \textdollar28.20 this month, which is more than last month's payment of \textdollar27.85.

\section*{Question 19}
\textbf{Metadata}

\begin{itemize}
  \item Question ID: P4-DcDiv2d1d\_P4-DcAdd2nd\_sonnet4\_Household Finance\_01
  \item Primary KC: DECIMALS | Division | dividing decimals (up to 2 decimal places) by a 1-digit whole number
  \item Secondary KC: DECIMALS | Addition | adding decimals
  \item Topic: Household finance such as income, utility bills, money, interest, savings, instalment, mortgage, financial planning etc.
  \item Grade: Primary 4
\end{itemize}

\textbf{Solution}

Step 1: Find the total amount of all utility bills by adding the decimals.
Electricity bill + Water bill + Gas bill
$\textdollar68.45 + \textdollar32.80 + \textdollar41.75$

Align the decimal points:
$\begin{array}{r} 68.45 \\ 32.80 \\ + 41.75 \\ \hline 143.00 \end{array}$

Total amount = \textdollar143.00

Step 2: Divide the total amount by 3 to find each instalment.
$\textdollar143.00 \div 3$

$\begin{array}{r} 47.66\frac{2}{3} \\ 3 \enclose{longdiv}{143.00} \\ \underline{12} \\ 23 \\ \underline{21} \\ 20 \\ \underline{18} \\ 20 \\ \underline{18} \\ 2 \end{array}$

Since we get a remainder of 2 cents, we express this as:
$143.00 \div 3 = 47.66\frac{2}{3}$

Rounding to the nearest cent: \textdollar47.67

Therefore, each instalment will be \textdollar47.67.

\section*{Question 20}
\textbf{Metadata}

\begin{itemize}
  \item Question ID: P5-FrAddMix\_P2-FrCmp\_sonnet4\_Household Finance\_01
  \item Primary KC: FRACTIONS | Addition | adding mixed numbers
  \item Secondary KC: FRACTIONS | Comparison and ordering | comparing and ordering fractions
  \item Topic: Household finance such as income, utility bills, money, interest, savings, instalment, mortgage, financial planning etc.
  \item Grade: Primary 5
\end{itemize}

\textbf{Solution}

Step 1: Find Mrs. Tan's monthly savings goal.
If Mrs. Tan's income is $2\frac{1}{4}$ times her savings goal:
$2\frac{1}{4} \times \text{savings goal} = \textdollar2400$
$\frac{9}{4} \times \text{savings goal} = \textdollar2400$
Savings goal $= \textdollar2400 \div \frac{9}{4} = \textdollar2400 \times \frac{4}{9} = \textdollar\frac{9600}{9} = \textdollar\frac{3200}{3} = \textdollar1066\frac{2}{3}$

Step 2: Find Mr. Tan's monthly income.
Mr. Tan's income $= 1\frac{2}{3} \times \textdollar2400 = \frac{5}{3} \times \textdollar2400 = \textdollar\frac{12000}{3} = \textdollar4000$

Step 3: Find their total monthly income.
Total income $= \textdollar2400 + \textdollar4000 = \textdollar6400$

Step 4: Find their total monthly expenses.
Total expenses $= \textdollar180 + \textdollar520 + \textdollar150 = \textdollar850$

Step 5: Find the total amount needed (expenses + savings goal).
Total needed $= \textdollar850 + \textdollar1066\frac{2}{3} = \textdollar1916\frac{2}{3}$

Step 6: Compare total income with total needed amount.
$\textdollar6400$ compared to $\textdollar1916\frac{2}{3}$
Since $6400 > 1916\frac{2}{3}$, their combined monthly income is enough to cover their expenses and meet the savings goal.

Therefore, their total monthly income is \textdollar6400, and yes, it exceeds their expenses and savings goal by $\textdollar6400 - \textdollar1916\frac{2}{3} = \textdollar4483\frac{1}{3}$.

\section*{Question 21}
\textbf{Metadata}

\begin{itemize}
  \item Question ID: P5-FrSubMix\_P2-FrCmp\_sonnet4\_Household Finance\_01
  \item Primary KC: FRACTIONS | Subtraction | subtracting mixed numbers
  \item Secondary KC: FRACTIONS | Comparison and ordering | comparing and ordering fractions
  \item Topic: Household finance such as income, utility bills, money, interest, savings, instalment, mortgage, financial planning etc.
  \item Grade: Primary 5
\end{itemize}

\textbf{Solution}

First, I need to find the total amount spent by adding the two expenses.

To add $2\frac{3}{4}$ and $1\frac{5}{6}$, I need to find a common denominator.
The LCD of 4 and 6 is 12.

$2\frac{3}{4} = 2\frac{9}{12}$
$1\frac{5}{6} = 1\frac{10}{12}$

Total spent = $2\frac{9}{12} + 1\frac{10}{12} = 3\frac{19}{12} = 4\frac{7}{12}$ thousand dollars

Next, I need to subtract the total spent from the budget.
Budget = \textdollar4200 = $4\frac{2}{10} = 4\frac{1}{5}$ thousand dollars

To subtract $4\frac{7}{12}$ from $4\frac{1}{5}$, I need a common denominator.
The LCD of 5 and 12 is 60.

$4\frac{1}{5} = 4\frac{12}{60}$
$4\frac{7}{12} = 4\frac{35}{60}$

Since $\frac{12}{60} < \frac{35}{60}$, I need to borrow from the whole number:
$4\frac{12}{60} = 3\frac{72}{60}$

Remaining amount = $3\frac{72}{60} - 4\frac{35}{60} = -\frac{23}{60}$

Wait, this gives a negative result. Let me recalculate.

Actually, $4\frac{7}{12} = 4 + \frac{35}{60} = \frac{275}{60}$
$4\frac{1}{5} = 4 + \frac{12}{60} = \frac{252}{60}$

Since the expenses exceed the budget, they are over budget by $\frac{23}{60}$ thousand dollars.

To compare with $\frac{1}{2}$ thousand dollars:
$\frac{1}{2} = \frac{30}{60}$

Since $\frac{23}{60} < \frac{30}{60}$, they are over budget by less than $\frac{1}{2}$ thousand dollars.

Therefore, the Lim family exceeded their budget by $\frac{23}{60}$ thousand dollars, which is approximately \textdollar383, and this amount is less than $\frac{1}{2}$ thousand dollars.

\section*{Question 22}
\textbf{Metadata}

\begin{itemize}
  \item Question ID: P5-FrMulImN\_P2-FrCmp\_sonnet4\_Household Finance\_01
  \item Primary KC: FRACTIONS | Multiplication | multiplying a proper/improper fraction and a whole number
  \item Secondary KC: FRACTIONS | Comparison and ordering | comparing and ordering fractions
  \item Topic: Household finance such as income, utility bills, money, interest, savings, instalment, mortgage, financial planning etc.
  \item Grade: Primary 5
\end{itemize}

\textbf{Solution}

(a) Amount spent on groceries = $\frac{2}{9} \times 3600$

$= \frac{2 \times 3600}{9}$

$= \frac{7200}{9}$

$= 800$

The family spends \textdollar800 on groceries each month.

(b) Amount spent on utilities = $\frac{1}{6} \times 3600$

$= \frac{1 \times 3600}{6}$

$= \frac{3600}{6}$

$= 600$

The family spends \textdollar600 on utilities each month.

(c) To compare the expenses, we need to compare $\frac{2}{9}$ and $\frac{1}{6}$.

To compare these fractions, we find a common denominator.

LCM of 9 and 6 = 18

$\frac{2}{9} = \frac{2 \times 2}{9 \times 2} = \frac{4}{18}$

$\frac{1}{6} = \frac{1 \times 3}{6 \times 3} = \frac{3}{18}$

Since $\frac{4}{18} > \frac{3}{18}$, we have $\frac{2}{9} > \frac{1}{6}$.

Therefore, the family spends more money on groceries than on utilities.

Difference = \textdollar800 - \textdollar600 = \textdollar200

The family spends \textdollar200 more on groceries than on utilities.

\section*{Question 23}
\textbf{Metadata}

\begin{itemize}
  \item Question ID: P5-FrMulPIm\_P2-FrCmp\_sonnet4\_Household Finance\_01
  \item Primary KC: FRACTIONS | Multiplication | multiplying a proper fraction and a proper/improper fractions
  \item Secondary KC: FRACTIONS | Comparison and ordering | comparing and ordering fractions
  \item Topic: Household finance such as income, utility bills, money, interest, savings, instalment, mortgage, financial planning etc.
  \item Grade: Primary 5
\end{itemize}

\textbf{Solution}

Let Mrs. Lim's monthly income be represented as 1 whole unit.

Step 1: Find the fraction of monthly income spent on groceries and utilities.
Money allocated for household expenses = $\frac{2}{5}$ of monthly income
Money spent on groceries and utilities = $\frac{3}{4}$ of allocated household expenses
Money spent on groceries and utilities = $\frac{3}{4} \times \frac{2}{5} = \frac{3 \times 2}{4 \times 5} = \frac{6}{20} = \frac{3}{10}$ of monthly income

Step 2: Find the fraction of monthly income saved in the first week.
Money allocated for savings = $\frac{3}{7}$ of monthly income
Money saved in first week = $\frac{5}{6}$ of allocated savings
Money saved in first week = $\frac{5}{6} \times \frac{3}{7} = \frac{5 \times 3}{6 \times 7} = \frac{15}{42} = \frac{5}{14}$ of monthly income

Step 3: Compare $\frac{3}{10}$ and $\frac{5}{14}$.
To compare these fractions, we find a common denominator.
LCM of 10 and 14 = 70
$\frac{3}{10} = \frac{3 \times 7}{10 \times 7} = \frac{21}{70}$
$\frac{5}{14} = \frac{5 \times 5}{14 \times 5} = \frac{25}{70}$

Since $\frac{25}{70} > \frac{21}{70}$, we have $\frac{5}{14} > \frac{3}{10}$.

Therefore, the money Mrs. Lim saved in the first week ($\frac{5}{14}$ of monthly income) is greater than the money she spent on groceries and utilities ($\frac{3}{10}$ of monthly income).

\section*{Question 24}
\textbf{Metadata}

\begin{itemize}
  \item Question ID: P5-FrMulPIm\_P2-FrAdd2nd\_sonnet4\_Household Finance\_01
  \item Primary KC: FRACTIONS | Multiplication | multiplying a proper fraction and a proper/improper fractions
  \item Secondary KC: FRACTIONS | Addition | adding fractions
  \item Topic: Household finance such as income, utility bills, money, interest, savings, instalment, mortgage, financial planning etc.
  \item Grade: Primary 5
\end{itemize}

\textbf{Solution}

Step 1: Calculate how much Mrs. Tan saves each month.
Amount saved = $\frac{1}{4} \times \textdollar3600 = \textdollar900$

Step 2: Calculate how much Mrs. Tan spends on household expenses each month.
Amount spent on household expenses = $\frac{2}{5} \times \textdollar3600 = \textdollar1440$

Step 3: Find the total amount used for savings and household expenses.
To add the fractions: $\frac{1}{4} + \frac{2}{5}$
Find the common denominator: $\frac{1}{4} = \frac{5}{20}$ and $\frac{2}{5} = \frac{8}{20}$
$\frac{5}{20} + \frac{8}{20} = \frac{13}{20}$

Step 4: Calculate the fraction of income used for other expenses.
Fraction for other expenses = $1 - \frac{13}{20} = \frac{20}{20} - \frac{13}{20} = \frac{7}{20}$

Step 5: Calculate the amount used for other expenses.
Amount for other expenses = $\frac{7}{20} \times \textdollar3600 = \textdollar1260$

Therefore, Mrs. Tan uses \textdollar1260 for other expenses each month.

\section*{Question 25}
\textbf{Metadata}

\begin{itemize}
  \item Question ID: P5-FrMulPIm\_P3-FrSmp\_sonnet4\_Household Finance\_01
  \item Primary KC: FRACTIONS | Multiplication | multiplying a proper fraction and a proper/improper fractions
  \item Secondary KC: FRACTIONS | Simplifying | expressing a fraction in its simplest form
  \item Topic: Household finance such as income, utility bills, money, interest, savings, instalment, mortgage, financial planning etc.
  \item Grade: Primary 5
\end{itemize}

\textbf{Solution}

Step 1: Find how much Mrs. Tan saves from her salary.
Mrs. Tan saves $\frac{2}{5}$ of \textdollar3000.
Amount saved = $\frac{2}{5} \times 3000 = \frac{2 \times 3000}{5} = \frac{6000}{5} = 1200$
So Mrs. Tan saves \textdollar1200.

Step 2: Find how much she spends on utility bills.
She spends $\frac{3}{4}$ of her savings on utility bills.
We need to calculate: $\frac{3}{4} \times \frac{2}{5}$ of her salary
$\frac{3}{4} \times \frac{2}{5} = \frac{3 \times 2}{4 \times 5} = \frac{6}{20}$

Step 3: Simplify the fraction.
To simplify $\frac{6}{20}$, we find the highest common factor of 6 and 20.
Factors of 6: 1, 2, 3, 6
Factors of 20: 1, 2, 4, 5, 10, 20
Highest common factor = 2
$\frac{6}{20} = \frac{6 \div 2}{20 \div 2} = \frac{3}{10}$

Step 4: Convert to dollars.
Amount spent on utility bills = $\frac{3}{10} \times 3000 = \frac{3 \times 3000}{10} = \frac{9000}{10} = 900$

Therefore, Mrs. Tan spends $\frac{3}{10}$ of her salary, which is \textdollar900, on utility bills.

\section*{Question 26}
\textbf{Metadata}

\begin{itemize}
  \item Question ID: P5-FrMulImIm\_P2-FrAdd2nd\_sonnet4\_Household Finance\_01
  \item Primary KC: FRACTIONS | Multiplication | multiplying two improper fractions
  \item Secondary KC: FRACTIONS | Addition | adding fractions
  \item Topic: Household finance such as income, utility bills, money, interest, savings, instalment, mortgage, financial planning etc.
  \item Grade: Primary 5
\end{itemize}

\textbf{Solution}

Step 1: Find Mrs. Tan's monthly savings.
Monthly savings = $\frac{8}{15} \times \textdollar2400$
$= \frac{8 \times 2400}{15}$
$= \frac{19200}{15}$
$= \textdollar1280$

Step 2: Find the amount spent on household bills.
Amount spent on household bills = $\frac{5}{4} \times \textdollar1280$
$= \frac{5 \times 1280}{4}$
$= \frac{6400}{4}$
$= \textdollar1600$

Step 3: Find the amount spent on groceries.
Amount spent on groceries = $\frac{7}{3} \times \textdollar1280$
$= \frac{7 \times 1280}{3}$
$= \frac{8960}{3}$
$= \textdollar2986\frac{2}{3}$

Step 4: Find the total amount spent on household bills and groceries.
Total amount = \textdollar1600 + \textdollar2986$\frac{2}{3}$
$= \textdollar1600 + \textdollar\frac{8960}{3}$
$= \textdollar\frac{4800}{3} + \textdollar\frac{8960}{3}$
$= \textdollar\frac{4800 + 8960}{3}$
$= \textdollar\frac{13760}{3}$
$= \textdollar4586\frac{2}{3}$

Therefore, Mrs. Tan spends \textdollar4586$\frac{2}{3}$ on household bills and groceries altogether each month.

\section*{Question 27}
\textbf{Metadata}

\begin{itemize}
  \item Question ID: P5-FrMulImIm\_P2-FrSub2nd\_sonnet4\_Household Finance\_01
  \item Primary KC: FRACTIONS | Multiplication | multiplying two improper fractions
  \item Secondary KC: FRACTIONS | Subtraction | subtracting fractions
  \item Topic: Household finance such as income, utility bills, money, interest, savings, instalment, mortgage, financial planning etc.
  \item Grade: Primary 5
\end{itemize}

\textbf{Solution}

Step 1: Find how much of the monthly budget was allocated for utilities and groceries.
Amount allocated = $\frac{7}{4} \times \frac{8}{5}$
$= \frac{7 \times 8}{4 \times 5}$
$= \frac{56}{20}$
$= \frac{14}{5}$ of their monthly budget

Step 2: Find how much they saved.
Amount saved = $\frac{1}{6} \times \frac{14}{5}$
$= \frac{1 \times 14}{6 \times 5}$
$= \frac{14}{30}$
$= \frac{7}{15}$ of their monthly budget

Step 3: Find how much they actually spent on utilities and groceries.
Actual amount spent = Amount allocated - Amount saved
$= \frac{14}{5} - \frac{7}{15}$

To subtract, we need a common denominator:
$\frac{14}{5} = \frac{14 \times 3}{5 \times 3} = \frac{42}{15}$

Actual amount spent = $\frac{42}{15} - \frac{7}{15}$
$= \frac{42 - 7}{15}$
$= \frac{35}{15}$
$= \frac{7}{3}$ of their monthly budget

Therefore, the Tan family actually spent $\frac{7}{3}$ of their monthly budget on utilities and groceries after the savings.

\section*{Question 28}
\textbf{Metadata}

\begin{itemize}
  \item Question ID: P5-FrMulImIm\_P3-FrSmp\_sonnet4\_Household Finance\_01
  \item Primary KC: FRACTIONS | Multiplication | multiplying two improper fractions
  \item Secondary KC: FRACTIONS | Simplifying | expressing a fraction in its simplest form
  \item Topic: Household finance such as income, utility bills, money, interest, savings, instalment, mortgage, financial planning etc.
  \item Grade: Primary 5
\end{itemize}

\textbf{Solution}

To find the new daily savings amount, I need to multiply the original daily savings by the increase factor.

Original daily savings = \textdollar$\frac{13}{4}$

Increase factor = $\frac{7}{3}$ times

New daily savings = $\frac{13}{4} \times \frac{7}{3}$

To multiply fractions, I multiply the numerators together and the denominators together:

$\frac{13}{4} \times \frac{7}{3} = \frac{13 \times 7}{4 \times 3} = \frac{91}{12}$

Now I need to express $\frac{91}{12}$ in its simplest form by finding the greatest common factor (GCF) of 91 and 12.

Factors of 91: 1, 7, 13, 91
Factors of 12: 1, 2, 3, 4, 6, 12

The GCF of 91 and 12 is 1, so $\frac{91}{12}$ is already in its simplest form.

Therefore, Sarah's family will save \textdollar$\frac{91}{12}$ per day after the increase.

\section*{Question 29}
\textbf{Metadata}

\begin{itemize}
  \item Question ID: P5-FrMulMixN\_P2-FrSub2nd\_sonnet4\_Household Finance\_01
  \item Primary KC: FRACTIONS | Multiplication | multiplying a mixed number and a whole number
  \item Secondary KC: FRACTIONS | Subtraction | subtracting fractions
  \item Topic: Household finance such as income, utility bills, money, interest, savings, instalment, mortgage, financial planning etc.
  \item Grade: Primary 5
\end{itemize}

\textbf{Solution}

Step 1: Find the increased electricity bill amount.
Usual bill = \textdollar72
This month she used $2\frac{1}{4}$ times the usual amount.
Increased bill = $72 \times 2\frac{1}{4}$

First, convert the mixed number to an improper fraction:
$2\frac{1}{4} = \frac{9}{4}$

$72 \times \frac{9}{4} = \frac{72 \times 9}{4} = \frac{648}{4} = 162$

So the increased bill = \textdollar162

Step 2: Find the rebate amount.
Rebate = $\frac{3}{8}$ of the increased amount
Rebate = $\frac{3}{8} \times 162 = \frac{3 \times 162}{8} = \frac{486}{8} = 60\frac{6}{8} = 60\frac{3}{4}$

So the rebate = \textdollar$60\frac{3}{4}$

Step 3: Find the final bill after rebate.
Final bill = Increased bill - Rebate
Final bill = $162 - 60\frac{3}{4}$

Convert 162 to a mixed number with denominator 4:
$162 = 161\frac{4}{4}$

$161\frac{4}{4} - 60\frac{3}{4} = (161-60) + (\frac{4}{4} - \frac{3}{4}) = 101 + \frac{1}{4} = 101\frac{1}{4}$

Therefore, Mrs. Tan's final electricity bill for this month is \textdollar$101\frac{1}{4}$.

\section*{Question 30}
\textbf{Metadata}

\begin{itemize}
  \item Question ID: P5-DcMul3dK\_P4-DcCnv2Fr\_sonnet4\_Household Finance\_01
  \item Primary KC: DECIMALS | Multiplication | multiplying decimals (up to 3 decimal places) by 10, 100, 1000 and their multiples
  \item Secondary KC: DECIMALS | Conversion from decimals to fraction | expressing decimals as fractions
  \item Topic: Household finance such as income, utility bills, money, interest, savings, instalment, mortgage, financial planning etc.
  \item Grade: Primary 5
\end{itemize}

\textbf{Solution}

**Solution:**

**(a) Converting decimals to fractions:**

For $0.125$:
$0.125 = \frac{125}{1000} = \frac{125 \div 125}{1000 \div 125} = \frac{1}{8}$

For $0.08$:
$0.08 = \frac{8}{100} = \frac{8 \div 4}{100 \div 4} = \frac{2}{25}$

**(b) Money spent on groceries:**

To find $0.125 \times 2400$:
$0.125 \times 2400$

Since $0.125 = \frac{125}{1000}$, we can rewrite this as:
$\frac{125}{1000} \times 2400 = 125 \times \frac{2400}{1000} = 125 \times 2.4$

Using multiplication of decimals by multiples of 10:
$125 \times 2.4 = 125 \times 24 \div 10 = 3000 \div 10 = 300$

Therefore, Mrs. Tan spends \textdollar300 on groceries.

**(c) Money spent on utilities:**

To find $0.08 \times 2400$:
$0.08 \times 2400$

Since $0.08 = \frac{8}{100}$, we can rewrite this as:
$\frac{8}{100} \times 2400 = 8 \times \frac{2400}{100} = 8 \times 24 = 192$

Therefore, Mrs. Tan spends \textdollar192 on utilities.

**(d) Total amount spent on groceries and utilities:**

\textdollar300 + \textdollar192 = \textdollar492

Mrs. Tan spends a total of \textdollar492 on groceries and utilities combined.

\section*{Question 31}
\textbf{Metadata}

\begin{itemize}
  \item Question ID: P5-DcMul3dK\_P4-DcSub2nd\_sonnet4\_Household Finance\_01
  \item Primary KC: DECIMALS | Multiplication | multiplying decimals (up to 3 decimal places) by 10, 100, 1000 and their multiples
  \item Secondary KC: DECIMALS | Subtraction | subtracting decimals
  \item Topic: Household finance such as income, utility bills, money, interest, savings, instalment, mortgage, financial planning etc.
  \item Grade: Primary 5
\end{itemize}

\textbf{Solution}

Step 1: Calculate the monthly electricity bill.
Electricity cost per unit = \textdollar0.125
Number of units used = 800
Electricity bill = $0.125 \times 800$
To multiply $0.125$ by $800$, I can multiply $0.125$ by $8 \times 100$:
$0.125 \times 8 = 1.000$
$1.000 \times 100 = 100.000$
Electricity bill = \textdollar100.00

Step 2: Calculate the monthly water bill.
Water cost per unit = \textdollar0.078
Number of units used = 1200
Water bill = $0.078 \times 1200$
To multiply $0.078$ by $1200$, I can multiply $0.078$ by $12 \times 100$:
$0.078 \times 12 = 0.936$
$0.936 \times 100 = 93.600$
Water bill = \textdollar93.60

Step 3: Calculate total utility bills.
Total utility bills = Electricity bill + Water bill
Total utility bills = \textdollar100.00 + \textdollar93.60 = \textdollar193.60

Step 4: Calculate monthly savings.
Monthly income = \textdollar4500.00
Monthly savings = Monthly income - Total utility bills
Monthly savings = \textdollar4500.00 - \textdollar193.60 = \textdollar4306.40

Therefore, Mrs. Tan can save \textdollar4306.40 each month after paying her utility bills.

\section*{Question 32}
\textbf{Metadata}

\begin{itemize}
  \item Question ID: P5-DcDiv3dK\_P4-DcRnd3d\_sonnet4\_Household Finance\_01
  \item Primary KC: DECIMALS | Division | dividing decimals (up to 3 decimal places) by 10, 100, 1000 and their multiples
  \item Secondary KC: DECIMALS | Rounding | rounding decimals up to 3 decimal places to the nearest whole number, 1 decimal place and 2 decimal places 
  \item Topic: Household finance such as income, utility bills, money, interest, savings, instalment, mortgage, financial planning etc.
  \item Grade: Primary 5
\end{itemize}

\textbf{Solution}

Step 1: Find how much each child needs to pay.
Each child's share = \textdollar348.750 ÷ 3 = \textdollar116.250

Step 2: Find the weekly payment amount for each child.
Weekly payment = \textdollar116.250 ÷ 20 = \textdollar5.8125

Step 3: Round the weekly payment to the nearest cent (2 decimal places).
\textdollar5.8125 rounded to the nearest cent = \textdollar5.81

Therefore, each child needs to make a weekly payment of \textdollar5.81.

\section*{Question 33}
\textbf{Metadata}

\begin{itemize}
  \item Question ID: P5-DcDiv3dK\_P4-DcAdd2nd\_sonnet4\_Household Finance\_01
  \item Primary KC: DECIMALS | Division | dividing decimals (up to 3 decimal places) by 10, 100, 1000 and their multiples
  \item Secondary KC: DECIMALS | Addition | adding decimals
  \item Topic: Household finance such as income, utility bills, money, interest, savings, instalment, mortgage, financial planning etc.
  \item Grade: Primary 5
\end{itemize}

\textbf{Solution}

Step 1: Find the daily cost for each utility bill by dividing by 30.

Electricity daily cost: $186.400 \div 30 = 186.400 \div (3 \times 10) = 186.400 \div 3 \div 10 = 62.133 \div 10 = \textdollar6.213$

Water daily cost: $92.350 \div 30 = 92.350 \div (3 \times 10) = 92.350 \div 3 \div 10 = 30.783 \div 10 = \textdollar3.078$

Gas daily cost: $74.280 \div 30 = 74.280 \div (3 \times 10) = 74.280 \div 3 \div 10 = 24.760 \div 10 = \textdollar2.476$

Step 2: Add the daily costs to find the total average daily utility expense.

Total daily expense = $6.213 + 3.078 + 2.476 = \textdollar11.767$

Therefore, the total average daily utility expense is \textdollar11.767.

\section*{Question 34}
\textbf{Metadata}

\begin{itemize}
  \item Question ID: P5-PcRepWh\_P1-WNMul2nd\_sonnet4\_Household Finance\_01
  \item Primary KC: PERCENTAGE | Representation and concept | expressing a part of a whole as a percentage
  \item Secondary KC: WHOLE NUMBERS | Multiplication | multiplying whole numbers
  \item Topic: Household finance such as income, utility bills, money, interest, savings, instalment, mortgage, financial planning etc.
  \item Grade: Primary 5
\end{itemize}

\textbf{Solution}

Step 1: Find the total amount spent on groceries and utilities.
25\% of \textdollar4800 = $\frac{25}{100} \times 4800 = \textdollar1200$

Step 2: Set up the relationship between groceries and utilities costs.
Let the cost of utilities = $x$
Then the cost of groceries = $3x$ (since groceries cost 3 times as much)

Step 3: Write an equation and solve.
Cost of utilities + Cost of groceries = Total amount spent
$x + 3x = 1200$
$4x = 1200$
$x = 1200 \div 4 = 300$

Therefore, Mrs. Lim spends \textdollar300 on utilities each month.

\section*{Question 35}
\textbf{Metadata}

\begin{itemize}
  \item Question ID: P5-PcRepWh\_P1-WNDiv2nd\_sonnet4\_Household Finance\_01
  \item Primary KC: PERCENTAGE | Representation and concept | expressing a part of a whole as a percentage
  \item Secondary KC: WHOLE NUMBERS | Division | dividing whole numbers
  \item Topic: Household finance such as income, utility bills, money, interest, savings, instalment, mortgage, financial planning etc.
  \item Grade: Primary 5
\end{itemize}

\textbf{Solution}

To find what percentage of her income Mrs. Tan spends on groceries and utilities:

Step 1: Identify the given information.
- Monthly income = \textdollar4800
- Amount spent on groceries and utilities = \textdollar1200

Step 2: Find what fraction of her income is spent on groceries and utilities.
Fraction = $\frac{\text{Amount spent}}{\text{Total income}} = \frac{1200}{4800}$

Step 3: Simplify the fraction by dividing both numerator and denominator.
$1200 \div 4800 = \frac{1200}{4800} = \frac{1}{4}$

Step 4: Convert the fraction to a percentage.
$\frac{1}{4} = \frac{1 \times 25}{4 \times 25} = \frac{25}{100} = 25\%$

Therefore, Mrs. Tan spends $25\%$ of her monthly income on groceries and utilities.

\section*{Question 36}
\textbf{Metadata}

\begin{itemize}
  \item Question ID: P5-RtFndU\_P2-DcCnvN2D\_sonnet4\_Household Finance\_01
  \item Primary KC: RATE | Finding number of unit | finding number of units given rate and total amount
  \item Secondary KC: DECIMALS | Conversion to larger units | converting a measurement from a smaller unit to a larger unit in decimal form
  \item Topic: Household finance such as income, utility bills, money, interest, savings, instalment, mortgage, financial planning etc.
  \item Grade: Primary 5
\end{itemize}

\textbf{Solution}

Step 1: Find the number of kilowatt-hours used.

Using the formula: Number of units = Total amount ÷ Rate per unit

Number of kWh = \textdollar67.20 ÷ \textdollar0.28 per kWh
Number of kWh = $67.20 ÷ 0.28$
Number of kWh = $240$ kWh

Step 2: Convert kilowatt-hours to megawatt-hours.

Since 1 megawatt-hour = 1000 kilowatt-hours, we need to divide by 1000 to convert from the smaller unit (kWh) to the larger unit (MWh).

Megawatt-hours = $240 ÷ 1000 = 0.24$ MWh

Therefore, Mrs. Tan used 240 kilowatt-hours of electricity, which is equal to 0.24 megawatt-hours.

\section*{Question 37}
\textbf{Metadata}

\begin{itemize}
  \item Question ID: P6-FrDivPN\_P2-FrAdd2nd\_sonnet4\_Household Finance\_01
  \item Primary KC: FRACTIONS | Division | dividing a proper fraction by a whole number
  \item Secondary KC: FRACTIONS | Addition | adding fractions
  \item Topic: Household finance such as income, utility bills, money, interest, savings, instalment, mortgage, financial planning etc.
  \item Grade: Primary 6
\end{itemize}

\textbf{Solution}

Step 1: Find the amount spent on groceries and utilities.
Amount spent = $\frac{2}{3} \times 900 = \frac{2 \times 900}{3} = \frac{1800}{3} = 600$
So Mrs. Tan spends \textdollar600 on groceries and utilities.

Step 2: Find the remaining money.
Remaining money = $900 - 600 = 300$
So \textdollar300 remains for the children's pocket money.

Step 3: Express the remaining money as a fraction of the original allowance.
Remaining fraction = $\frac{300}{900} = \frac{1}{3}$

Step 4: Divide this fraction by 4 to find each child's pocket money.
Each child receives = $\frac{1}{3} \div 4 = \frac{1}{3} \times \frac{1}{4} = \frac{1}{12}$ of the original allowance

Step 5: Convert this fraction to dollar amount.
Each child's pocket money = $\frac{1}{12} \times 900 = \frac{900}{12} = 75$

Therefore, each child receives \textdollar75 per week as pocket money, which is $\frac{1}{12}$ of Mrs. Tan's monthly allowance.

\section*{Question 38}
\textbf{Metadata}

\begin{itemize}
  \item Question ID: P6-FrDivPN\_P5-FrCnv2Dc\_sonnet4\_Household Finance\_01
  \item Primary KC: FRACTIONS | Division | dividing a proper fraction by a whole number
  \item Secondary KC: FRACTIONS | Conversion to decimals | expressing fractions as decimals
  \item Topic: Household finance such as income, utility bills, money, interest, savings, instalment, mortgage, financial planning etc.
  \item Grade: Primary 6
\end{itemize}

\textbf{Solution}

Step 1: Find the total reduction in electricity bill.
Total reduction = $\frac{2}{5} \times \textdollar90 = \frac{2 \times 90}{5} = \frac{180}{5} = \textdollar36$

Step 2: Find the monthly reduction by dividing the total reduction by 6 months.
Monthly reduction = $\frac{36}{6} = \textdollar6$

Alternatively, we can solve this by dividing the fraction by the whole number directly:
Monthly reduction = $\frac{2}{5} \times \textdollar90 \div 6$
= $\frac{2}{5} \div 6 \times \textdollar90$
= $\frac{2}{5} \times \frac{1}{6} \times \textdollar90$
= $\frac{2}{30} \times \textdollar90$
= $\frac{1}{15} \times \textdollar90$
= $\frac{90}{15} = \textdollar6$

Step 3: Express the answer as a decimal.
$\textdollar6 = \textdollar6.00$

Therefore, the monthly reduction in Mrs. Tan's electricity bill is \textdollar6.00.

\section*{Question 39}
\textbf{Metadata}

\begin{itemize}
  \item Question ID: P6-FrDivPP\_P2-FrSub2nd\_sonnet4\_Household Finance\_01
  \item Primary KC: FRACTIONS | Division | dividing a whole number/proper fraction by a proper fraction
  \item Secondary KC: FRACTIONS | Subtraction | subtracting fractions
  \item Topic: Household finance such as income, utility bills, money, interest, savings, instalment, mortgage, financial planning etc.
  \item Grade: Primary 6
\end{itemize}

\textbf{Solution}

Let me find the Tan family's total monthly income step by step.

First, I need to find what fraction of their income was spent on utilities.

The family spends $\frac{3}{8}$ of their income on household expenses, and after paying utilities, they have $\frac{1}{4}$ left.

This means: Fraction spent on utilities = $\frac{3}{8} - \frac{1}{4}$

To subtract these fractions, I need a common denominator:
$\frac{3}{8} - \frac{1}{4} = \frac{3}{8} - \frac{2}{8} = \frac{1}{8}$

So the family spent $\frac{1}{8}$ of their income on utilities.

Now I know that $\frac{1}{4}$ of their monthly income equals \textdollar900.

To find their total monthly income, I need to divide \textdollar900 by $\frac{1}{4}$:

Total income = $900 \div \frac{1}{4}$

To divide by a fraction, I multiply by its reciprocal:
$900 \div \frac{1}{4} = 900 \times \frac{4}{1} = 900 \times 4 = 3600$

Therefore, the Tan family's total monthly income is \textdollar3600.

Let me verify: 
- Household expenses: $\frac{3}{8} \times 3600 = 1350$
- Utilities: $\frac{1}{8} \times 3600 = 450$
- Left for savings: $\frac{1}{4} \times 3600 = 900$ ✓

\section*{Question 40}
\textbf{Metadata}

\begin{itemize}
  \item Question ID: P6-FrDivPP\_P5-FrCnv2Dc\_sonnet4\_Household Finance\_01
  \item Primary KC: FRACTIONS | Division | dividing a whole number/proper fraction by a proper fraction
  \item Secondary KC: FRACTIONS | Conversion to decimals | expressing fractions as decimals
  \item Topic: Household finance such as income, utility bills, money, interest, savings, instalment, mortgage, financial planning etc.
  \item Grade: Primary 6
\end{itemize}

\textbf{Solution}

Step 1: Find the amount Mrs. Tan withdrew from her savings.
Amount withdrawn = $\frac{3}{4} \times \textdollar240$
$= \frac{3 \times 240}{4}$
$= \frac{720}{4}$
$= \textdollar180$

Step 2: Find the amount used to pay the electricity bill.
Amount for electricity bill = $\frac{2}{5}$ of \textdollar180
This means we need to divide \textdollar180 by $\frac{5}{2}$ (which is the same as multiplying by $\frac{2}{5}$)
$\textdollar180 \div \frac{5}{2} = \textdollar180 \times \frac{2}{5}$
$= \frac{180 \times 2}{5}$
$= \frac{360}{5}$
$= \textdollar72$

Step 3: Convert the answer to decimal form.
\textdollar72 = \textdollar72.00

Therefore, Mrs. Tan used \textdollar72.00 to pay the electricity bill.

\section*{Question 41}
\textbf{Metadata}

\begin{itemize}
  \item Question ID: P6-PcFndWN\_P1-WNAdd2nd\_sonnet4\_Household Finance\_01
  \item Primary KC: PERCENTAGE | Finding the whole | finding the whole given a part and the percentage
  \item Secondary KC: WHOLE NUMBERS | Addition | adding whole numbers
  \item Topic: Household finance such as income, utility bills, money, interest, savings, instalment, mortgage, financial planning etc.
  \item Grade: Primary 6
\end{itemize}

\textbf{Solution}

Step 1: Find the total monthly utility expenses.
Since the electricity bill of \textdollar180 represents 15% of the total utility expenses:
15% of total utility expenses = \textdollar180
$\frac{15}{100} \times \text{total utility expenses} = 180$
Total utility expenses = $180 \div \frac{15}{100} = 180 \times \frac{100}{15} = 180 \times \frac{20}{3} = 1200$
So the total monthly utility expenses = \textdollar1200

Step 2: Find the water bill amount.
Water bill = Electricity bill + \textdollar45
Water bill = \textdollar180 + \textdollar45 = \textdollar225

Step 3: Find the total amount spent on electricity and water bills combined.
Total amount = Electricity bill + Water bill
Total amount = \textdollar180 + \textdollar225 = \textdollar405

Therefore, Mrs. Tan spends \textdollar405 on electricity and water bills combined.

\section*{Question 42}
\textbf{Metadata}

\begin{itemize}
  \item Question ID: P6-PcFndChg\_P1-WNDiv2nd\_sonnet4\_Household Finance\_01
  \item Primary KC: PERCENTAGE | Finding change | finding percentage increase/decrease
  \item Secondary KC: WHOLE NUMBERS | Division | dividing whole numbers
  \item Topic: Household finance such as income, utility bills, money, interest, savings, instalment, mortgage, financial planning etc.
  \item Grade: Primary 6
\end{itemize}

\textbf{Solution}

**Step 1: Find the increase in the electricity bill**

Increase in bill = February bill - January bill
Increase in bill = \textdollar288 - \textdollar240 = \textdollar48

**Step 2: Calculate the percentage increase**

Percentage increase = $\frac{\text{Increase}}{\text{Original amount}} \times 100\%$

Percentage increase = $\frac{48}{240} \times 100\%$

Percentage increase = $\frac{1}{5} \times 100\% = 20\%$

**Step 3: Calculate the weekly budget needed in February**

Weekly budget = February bill $\div$ Number of weeks

Weekly budget = \textdollar288 $\div$ 4 = \textdollar72

**Answer:**
The electricity bill increased by 20% from January to February.
The family needs to budget \textdollar72 per week for electricity in February.

\section*{Question 43}
\textbf{Metadata}

\begin{itemize}
  \item Question ID: P6-RoFndRoWN\_P1-WNAdd2nd\_sonnet4\_Household Finance\_01
  \item Primary KC: RATIO | Finding ratio | finding the ratio of two or three given whole numbers
  \item Secondary KC: WHOLE NUMBERS | Addition | adding whole numbers
  \item Topic: Household finance such as income, utility bills, money, interest, savings, instalment, mortgage, financial planning etc.
  \item Grade: Primary 6
\end{itemize}

\textbf{Solution}

First, I need to find the February expenses for each category by adding the increases to the January amounts.

February grocery expenses:
$\textdollar450 + \textdollar50 = \textdollar500$

February utility expenses:
$\textdollar300 + \textdollar25 = \textdollar325$

February transportation expenses:
$\textdollar150 + \textdollar25 = \textdollar175$

Now I need to find the ratio of grocery : utility : transportation expenses in February.
The amounts are $\textdollar500 : \textdollar325 : \textdollar175$

To simplify this ratio, I need to find the highest common factor of $500$, $325$, and $175$.

$500 = 4 \times 125 = 4 \times 5^3$
$325 = 13 \times 25 = 13 \times 5^2$
$175 = 7 \times 25 = 7 \times 5^2$

The highest common factor is $25$.

$500 \div 25 = 20$
$325 \div 25 = 13$
$175 \div 25 = 7$

Therefore, the ratio of grocery expenses to utility expenses to transportation expenses in February is $20 : 13 : 7$.

\section*{Question 44}
\textbf{Metadata}

\begin{itemize}
  \item Question ID: P6-AgRepLrEx\_P6-AgSmpLrEx\_sonnet4\_Household Finance\_01
  \item Primary KC: ALGEBRA | Representation and concept | translation of real-world situations into linear algebraic expressions
  \item Secondary KC: ALGEBRA | Simplifying | simplifying linear expressions
  \item Topic: Household finance such as income, utility bills, money, interest, savings, instalment, mortgage, financial planning etc.
  \item Grade: Primary 6
\end{itemize}

\textbf{Solution}

Let me solve this step by step.

**Step 1: Identify the given information**
- Monthly salary = \textdollar3200
- Phone bill = \textdollar80
- Utility bills = \textdollar150 more than phone bill
- Groceries = \textdollar600
- Transport = \textdollar400
- Monthly savings = \textdollar200 more than twice the phone bill

**Step 2: Translate into algebraic expressions**

Let $p = 80$ (phone bill in dollars)

- Utility bills = $p + 150$
- Monthly savings = $2p + 200$

**Step 3: Write expression for total monthly expenses**

Total monthly expenses = Phone bill + Utility bills + Groceries + Transport
Total monthly expenses = $p + (p + 150) + 600 + 400$

**Step 4: Simplify the expenses expression**

Total monthly expenses = $p + p + 150 + 600 + 400$
Total monthly expenses = $2p + 1150$

Substituting $p = 80$:
Total monthly expenses = $2(80) + 1150 = 160 + 1150 = 1310$

So the expression is $2p + 1150$ or \textdollar1310

**Step 5: Write expression for total monthly savings after expenses**

Total monthly savings = Salary - Total expenses
Total monthly savings = $3200 - (2p + 1150)$
Total monthly savings = $3200 - 2p - 1150$
Total monthly savings = $2050 - 2p$

Substituting $p = 80$:
Total monthly savings = $2050 - 2(80) = 2050 - 160 = 1890$

So the expression is $2050 - 2p$ or \textdollar1890

**Final Answer:**
- Total monthly expenses: $2p + 1150 = $ \textdollar1310
- Total monthly savings after expenses: $2050 - 2p = $ \textdollar1890

\section*{Question 45}
\textbf{Metadata}

\begin{itemize}
  \item Question ID: O1-PcFndRslt\_P1-WNSub2nd\_sonnet4\_Household Finance\_01
  \item Primary KC: PERCENTAGE | Finding result after change | increasing/decreasing a quantity by a given percentage
  \item Secondary KC: WHOLE NUMBERS | Subtraction | subtracting whole numbers
  \item Topic: Household finance such as income, utility bills, money, interest, savings, instalment, mortgage, financial planning etc.
  \item Grade: Secondary O-level 1
\end{itemize}

\textbf{Solution}

Step 1: Calculate the February electricity bill after a 15\% decrease from January.
January bill = \textdollar180
Decrease amount = $15\% \times \textdollar180 = 0.15 \times 180 = \textdollar27$
February bill = $\textdollar180 - \textdollar27 = \textdollar153$

Step 2: Calculate the March electricity bill after a 20\% increase from February.
Increase amount = $20\% \times \textdollar153 = 0.20 \times 153 = \textdollar30.60$
March bill = $\textdollar153 + \textdollar30.60 = \textdollar183.60$

Step 3: Compare the actual March bill with the budgeted amount.
Budgeted amount = \textdollar200
Actual March bill = \textdollar183.60
Amount left over = $\textdollar200 - \textdollar183.60 = \textdollar16.40$

Therefore, the Tan family had \textdollar16.40 left over from their March electricity budget.

\section*{Question 46}
\textbf{Metadata}

\begin{itemize}
  \item Question ID: O1-PcRepRvs\_O1-PcCnv2Dc\_sonnet4\_Household Finance\_01
  \item Primary KC: PERCENTAGE | Representation and concept | reverse percentages
  \item Secondary KC: PERCENTAGE | Conversion to decimals | expressing percentage as a decimal
  \item Topic: Household finance such as income, utility bills, money, interest, savings, instalment, mortgage, financial planning etc.
  \item Grade: Secondary O-level 1
\end{itemize}

\textbf{Solution}

Step 1: Find Sarah's original monthly salary using reverse percentages.

Let the original salary be $x$ dollars.
After a $15\%$ increase, her new salary is $115\%$ of her original salary.

$115\% \times x = \textdollar4600$

Step 2: Convert the percentage to decimal form.
$115\% = \frac{115}{100} = 1.15$

Step 3: Set up and solve the equation.
$1.15x = 4600$
$x = \frac{4600}{1.15} = 4000$

Therefore, Sarah's original monthly salary was \textdollar4000.

Step 4: Calculate her monthly savings.
Sarah saves $20\%$ of her original salary.
$20\% = \frac{20}{100} = 0.2$

Monthly savings $= 0.2 \times 4000 = \textdollar800$

Therefore, Sarah will save \textdollar800 each month for her vacation.

\section*{Question 47}
\textbf{Metadata}

\begin{itemize}
  \item Question ID: O2-RoRepDP\_P1-WNMul2nd\_sonnet4\_Household Finance\_01
  \item Primary KC: RATIO | Representation and concept | direct proportion
  \item Secondary KC: WHOLE NUMBERS | Multiplication | multiplying whole numbers
  \item Topic: Household finance such as income, utility bills, money, interest, savings, instalment, mortgage, financial planning etc.
  \item Grade: Secondary O-level 2
\end{itemize}

\textbf{Solution}

**Part (a):**

Given ratio for groceries : utilities : entertainment = $3:2:1$

Let the common factor be $k$.
Then groceries = $3k$, utilities = $2k$, entertainment = $k$

Since groceries = \textdollar240:
$3k = 240$
$k = 240 \div 3 = 80$

Therefore:
- Utilities = $2k = 2 \times 80 = \textdollar160$
- Entertainment = $k = 80 = \textdollar80$

**Part (b):**

Original total monthly expenses = $240 + 160 + 80 = \textdollar480$

New total monthly expenses = $480 \times 4 = \textdollar1920$

Since the ratio remains the same $(3:2:1)$, the new expenses will be:
- New groceries = $240 \times 4 = \textdollar960$
- New utilities = $160 \times 4 = \textdollar640$
- New entertainment = $80 \times 4 = \textdollar320$

Verification: $960 + 640 + 320 = \textdollar1920$ ✓

\section*{Question 48}
\textbf{Metadata}

\begin{itemize}
  \item Question ID: O3-MXMul\_O3-MXAdd\_sonnet4\_Household Finance\_01
  \item Primary KC: MATRICES | Multiplication | multiplication of matrices
  \item Secondary KC: MATRICES | Addition | addition of matrices
  \item Topic: Household finance such as income, utility bills, money, interest, savings, instalment, mortgage, financial planning etc.
  \item Grade: Secondary O-level 3/4
\end{itemize}

\textbf{Solution}

**Step 1: Calculate matrix multiplication $AB$**

To multiply matrices $A$ and $B$, we need to ensure they are compatible for multiplication. However, $A$ is $3 \times 3$ and $B$ is also $3 \times 3$, so we can multiply them.

For matrix multiplication $AB$, each element is calculated as:
$(AB)_{ij} = \sum_{k=1}^{3} A_{ik} \cdot B_{kj}$

Calculating each element:

Row 1: 
- $(AB)_{11} = 150(0.05) + 400(0.06) + 200(0.04) = 7.5 + 24 + 8 = 39.5$
- $(AB)_{12} = 150(0.08) + 400(0.07) + 200(0.09) = 12 + 28 + 18 = 58$
- $(AB)_{13} = 150(0.10) + 400(0.12) + 200(0.08) = 15 + 48 + 16 = 79$

Row 2:
- $(AB)_{21} = 180(0.05) + 450(0.06) + 180(0.04) = 9 + 27 + 7.2 = 43.2$
- $(AB)_{22} = 180(0.08) + 450(0.07) + 180(0.09) = 14.4 + 31.5 + 16.2 = 62.1$
- $(AB)_{23} = 180(0.10) + 450(0.12) + 180(0.08) = 18 + 54 + 14.4 = 86.4$

Row 3:
- $(AB)_{31} = 160(0.05) + 420(0.06) + 220(0.04) = 8 + 25.2 + 8.8 = 42$
- $(AB)_{32} = 160(0.08) + 420(0.07) + 220(0.09) = 12.8 + 29.4 + 19.8 = 62$
- $(AB)_{33} = 160(0.10) + 420(0.12) + 220(0.08) = 16 + 50.4 + 17.6 = 84$

Therefore: $AB = \begin{pmatrix} 39.5 & 58 & 79 \\ 43.2 & 62.1 & 86.4 \\ 42 & 62 & 84 \end{pmatrix}$

**Step 2: Add matrices $A$ and $AB$**

To find the second quarter expenses, we add the original expenses to the increase amounts:
$A + AB = \begin{pmatrix} 150 & 400 & 200 \\ 180 & 450 & 180 \\ 160 & 420 & 220 \end{pmatrix} + \begin{pmatrix} 39.5 & 58 & 79 \\ 43.2 & 62.1 & 86.4 \\ 42 & 62 & 84 \end{pmatrix}$

$= \begin{pmatrix} 189.5 & 458 & 279 \\ 223.2 & 512.1 & 266.4 \\ 202 & 482 & 304 \end{pmatrix}$

**Answer:** The family's expenses for the second quarter are \textdollar189.50, \textdollar458, and \textdollar279 for April; \textdollar223.20, \textdollar512.10, and \textdollar266.40 for May; and \textdollar202, \textdollar482, and \textdollar304 for June (utilities, groceries, and entertainment respectively).

\section*{Question 49}
\textbf{Metadata}

\begin{itemize}
  \item Question ID: O3-SPFndstd\_O2-SPFndmean\_sonnet4\_Household Finance\_01
  \item Primary KC: STATISTICS AND PROBABILITY | Finding standard deviation | calculation of the standard deviation for a set of data
  \item Secondary KC: STATISTICS AND PROBABILITY | Finding mean deviation | calculation of the mean for a set of data
  \item Topic: Household finance such as income, utility bills, money, interest, savings, instalment, mortgage, financial planning etc.
  \item Grade: Secondary O-level 3/4
\end{itemize}

\textbf{Solution}

To find the mean and standard deviation of the Tan family's monthly electricity bills.

Given data: \textdollar120, \textdollar135, \textdollar110, \textdollar145, \textdollar125, \textdollar115

Step 1: Calculate the mean.
Mean = $\frac{120 + 135 + 110 + 145 + 125 + 115}{6}$
Mean = $\frac{750}{6} = 125$

The mean monthly electricity bill is \textdollar125.

Step 2: Calculate the standard deviation.
First, find the deviations from the mean:
- $120 - 125 = -5$
- $135 - 125 = 10$
- $110 - 125 = -15$
- $145 - 125 = 20$
- $125 - 125 = 0$
- $115 - 125 = -10$

Next, square each deviation:
- $(-5)^2 = 25$
- $(10)^2 = 100$
- $(-15)^2 = 225$
- $(20)^2 = 400$
- $(0)^2 = 0$
- $(-10)^2 = 100$

Sum of squared deviations = $25 + 100 + 225 + 400 + 0 + 100 = 850$

Variance = $\frac{850}{6} = 141.67$ (rounded to 2 decimal places)

Standard deviation = $\sqrt{141.67} = 11.90$ (rounded to 2 decimal places)

Therefore, the mean monthly electricity bill is \textdollar125 and the standard deviation is \textdollar11.90.

\end{document}
