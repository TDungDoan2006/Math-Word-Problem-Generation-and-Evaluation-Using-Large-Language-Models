\documentclass{article}
\usepackage[utf8]{inputenc}
\usepackage{amsmath}
\usepackage{amsfonts}
\usepackage{amssymb}
\usepackage{graphicx}
\usepackage{hyperref}
\title{'Minh Questions household finance v3 CLAUDE '}
\author{Tien Dung Doan}
\begin{document}
\maketitle
\section*{Question 1}
\textbf{Metadata}

\begin{itemize}
  \item Question ID: P3-WNSub4d\_P1-WNCmp\_sonnet4\_Household Finance\_01
  \item Primary KC: WHOLE NUMBERS | Subtraction | subtracting whole numbers up to 4 digits
  \item Secondary KC: WHOLE NUMBERS | Comparison and ordering | comparing and ordering whole numbers
  \item Topic: Household finance such as income, utility bills, money, interest, savings, instalment, mortgage, financial planning etc.
  \item Grade: Primary 3
\end{itemize}

\textbf{Question}

The Tan family is planning their monthly budget. Mr. Tan earns \textdollar3250 per month and Mrs. Tan earns \textdollar2890 per month. Their monthly expenses include rent of \textdollar1450, groceries costing \textdollar680, and utility bills totaling \textdollar285. After paying all their expenses, how much money do they have left for savings? Compare this savings amount with their rent expense. Which amount is greater and by how much?

\section*{Question 2}
\textbf{Metadata}

\begin{itemize}
  \item Question ID: P3-WNDivRmd3d\_P1-WNAdd2nd\_sonnet4\_Household Finance\_01
  \item Primary KC: WHOLE NUMBERS | Division | dividing whole numbers up to 3 digits by 1 digit with remainder 
  \item Secondary KC: WHOLE NUMBERS | Addition | adding whole numbers
  \item Topic: Household finance such as income, utility bills, money, interest, savings, instalment, mortgage, financial planning etc.
  \item Grade: Primary 3
\end{itemize}

\textbf{Question}

Mrs. Tan receives \textdollar684 as her monthly salary. She wants to save the same amount of money each week for 4 weeks. However, after dividing her salary equally among the 4 weeks, she finds that there is some money left over that cannot be divided equally. She decides to add this leftover money to her savings for the first week. How much money will Mrs. Tan save in the first week?

\section*{Question 3}
\textbf{Metadata}

\begin{itemize}
  \item Question ID: P3-WNDivRmd3d\_P1-WNMul2nd\_sonnet4\_Household Finance\_01
  \item Primary KC: WHOLE NUMBERS | Division | dividing whole numbers up to 3 digits by 1 digit with remainder 
  \item Secondary KC: WHOLE NUMBERS | Multiplication | multiplying whole numbers
  \item Topic: Household finance such as income, utility bills, money, interest, savings, instalment, mortgage, financial planning etc.
  \item Grade: Primary 3
\end{itemize}

\textbf{Question}

Mrs. Tan earns \textdollar756 every month from her part-time job. She wants to save the same amount of money each week for her family's vacation fund. If there are 4 weeks in a month, how much money can she save each week? How much money will she have left over that cannot be divided equally among the weeks?

\section*{Question 4}
\textbf{Metadata}

\begin{itemize}
  \item Question ID: P3-WNDiv3d1d\_P1-WNAdd2nd\_sonnet4\_Household Finance\_01
  \item Primary KC: WHOLE NUMBERS | Division | dividing whole numbers up to 3 digits by 1 digit
  \item Secondary KC: WHOLE NUMBERS | Addition | adding whole numbers
  \item Topic: Household finance such as income, utility bills, money, interest, savings, instalment, mortgage, financial planning etc.
  \item Grade: Primary 3
\end{itemize}

\textbf{Question}

Mrs. Tan wants to save money for her family's vacation. She plans to save \textdollar648 over the next 8 months by setting aside the same amount each month. She also receives \textdollar25 every month from recycling bottles and cans, which she adds to her vacation savings. How much money will Mrs. Tan save each month in total for her vacation?

\section*{Question 5}
\textbf{Metadata}

\begin{itemize}
  \item Question ID: P3-WNDiv3d1d\_P1-WNSub2nd\_sonnet4\_Household Finance\_01
  \item Primary KC: WHOLE NUMBERS | Division | dividing whole numbers up to 3 digits by 1 digit
  \item Secondary KC: WHOLE NUMBERS | Subtraction | subtracting whole numbers
  \item Topic: Household finance such as income, utility bills, money, interest, savings, instalment, mortgage, financial planning etc.
  \item Grade: Primary 3
\end{itemize}

\textbf{Question}

Mrs. Tan's family has a monthly electricity bill of \textdollar348. She decides to pay this bill in equal amounts over 3 months. Her current savings account has \textdollar500. How much money will she have left in her savings account after making the first payment?

\section*{Question 6}
\textbf{Metadata}

\begin{itemize}
  \item Question ID: P3-WNDiv3d1d\_P1-WNMul2nd\_sonnet4\_Household Finance\_01
  \item Primary KC: WHOLE NUMBERS | Division | dividing whole numbers up to 3 digits by 1 digit
  \item Secondary KC: WHOLE NUMBERS | Multiplication | multiplying whole numbers
  \item Topic: Household finance such as income, utility bills, money, interest, savings, instalment, mortgage, financial planning etc.
  \item Grade: Primary 3
\end{itemize}

\textbf{Question}

Mrs. Tan pays \textdollar348 every month for her electricity bill. She wants to know how much she pays per week. There are 4 weeks in a month. After finding her weekly electricity cost, Mrs. Tan realizes she can save \textdollar12 per week by using energy-saving appliances. How much money will she save in 6 weeks?

\section*{Question 7}
\textbf{Metadata}

\begin{itemize}
  \item Question ID: P3-FrAddRl12\_P2-FrCmp\_sonnet4\_Household Finance\_01
  \item Primary KC: FRACTIONS | Addition | adding two related fractions within one whole with denominators of given fractions not exceeding 12
  \item Secondary KC: FRACTIONS | Comparison and ordering | comparing and ordering fractions
  \item Topic: Household finance such as income, utility bills, money, interest, savings, instalment, mortgage, financial planning etc.
  \item Grade: Primary 3
\end{itemize}

\textbf{Question}

Mrs. Tan is planning her monthly household budget. She spends $\frac{2}{5}$ of her monthly income on groceries and $\frac{1}{10}$ of her monthly income on utility bills. 

(a) What fraction of her monthly income does Mrs. Tan spend on groceries and utility bills altogether?

(b) Does Mrs. Tan spend more on groceries or utility bills? Show how you know.

\section*{Question 8}
\textbf{Metadata}

\begin{itemize}
  \item Question ID: P3-FrSubRl12\_P2-FrAdd2nd\_sonnet4\_Household Finance\_01
  \item Primary KC: FRACTIONS | Subtraction | subtracting two related fractions within one whole with denominators of given fractions not exceeding 12
  \item Secondary KC: FRACTIONS | Addition | adding fractions
  \item Topic: Household finance such as income, utility bills, money, interest, savings, instalment, mortgage, financial planning etc.
  \item Grade: Primary 3
\end{itemize}

\textbf{Question}

Mrs. Tan's family budget shows that $\frac{5}{8}$ of their monthly income goes to household expenses. Out of this amount, $\frac{1}{4}$ is spent on groceries and $\frac{1}{8}$ is spent on utility bills. What fraction of their monthly income is left from the household expenses after paying for groceries and utility bills?

\section*{Question 9}
\textbf{Metadata}

\begin{itemize}
  \item Question ID: P4-WNMul4d1d\_P1-WNAdd2nd\_sonnet4\_Household Finance\_01
  \item Primary KC: WHOLE NUMBERS | Multiplication | multiplying whole numbers up to 4 digits by 1 digit or up to 3 digits by 2 digits
  \item Secondary KC: WHOLE NUMBERS | Addition | adding whole numbers
  \item Topic: Household finance such as income, utility bills, money, interest, savings, instalment, mortgage, financial planning etc.
  \item Grade: Primary 4
\end{itemize}

\textbf{Question}

The Lim family is planning their monthly budget. Mr. Lim earns \textdollar3,250 each month from his job. Mrs. Lim works part-time and earns \textdollar28 per day. She works 18 days in a month. Their monthly expenses include rent of \textdollar1,800, utilities of \textdollar145, groceries of \textdollar680, and transportation costs of \textdollar320. How much money does the Lim family have left for savings each month?

\section*{Question 10}
\textbf{Metadata}

\begin{itemize}
  \item Question ID: P4-WNDiv4d1d\_P1-WNCmp\_sonnet4\_Household Finance\_01
  \item Primary KC: WHOLE NUMBERS | Division | dividing whole numbers up to 4 digits by 1 digit
  \item Secondary KC: WHOLE NUMBERS | Comparison and ordering | comparing and ordering whole numbers
  \item Topic: Household finance such as income, utility bills, money, interest, savings, instalment, mortgage, financial planning etc.
  \item Grade: Primary 4
\end{itemize}

\textbf{Question}

Mrs. Tan is comparing electricity bills from three different months to help with her family's financial planning. In January, her electricity bill was \textdollar2,184. In February, her bill was \textdollar1,456. In March, her bill was \textdollar1,872. Mrs. Tan wants to find the average monthly electricity cost over these three months by dividing the total amount by 3. She also wants to arrange the three monthly bills from the lowest to the highest amount to see which month had the most expensive bill. What is the average monthly electricity cost? Arrange the three monthly bills in order from lowest to highest.

\section*{Question 11}
\textbf{Metadata}

\begin{itemize}
  \item Question ID: P4-WNDiv4d1d\_P1-WNMul2nd\_sonnet4\_Household Finance\_01
  \item Primary KC: WHOLE NUMBERS | Division | dividing whole numbers up to 4 digits by 1 digit
  \item Secondary KC: WHOLE NUMBERS | Multiplication | multiplying whole numbers
  \item Topic: Household finance such as income, utility bills, money, interest, savings, instalment, mortgage, financial planning etc.
  \item Grade: Primary 4
\end{itemize}

\textbf{Question}

The Lim family pays \textdollar2856 for their electricity bill every 3 months. They want to save money by setting aside the same amount each month to pay for this quarterly bill. Additionally, they plan to save an extra \textdollar24 each month for 6 months to buy a new energy-efficient refrigerator. How much money should the Lim family set aside each month in total for both the electricity bill and the refrigerator savings?

\section*{Question 12}
\textbf{Metadata}

\begin{itemize}
  \item Question ID: P4-FrAddU12\_P3-FrSmp\_sonnet4\_Household Finance\_01
  \item Primary KC: FRACTIONS | Addition | adding unlike fractions with two different denominators not exceeding 12
  \item Secondary KC: FRACTIONS | Simplifying | expressing a fraction in its simplest form
  \item Topic: Household finance such as income, utility bills, money, interest, savings, instalment, mortgage, financial planning etc.
  \item Grade: Primary 4
\end{itemize}

\textbf{Question}

Sarah is helping her mother plan the family budget. They need to set aside money for two important bills this month. For the electricity bill, they need to save $\frac{1}{3}$ of their monthly allowance. For the water bill, they need to save $\frac{1}{4}$ of their monthly allowance. What fraction of their monthly allowance do they need to set aside for both bills combined? Express your answer in its simplest form.

\section*{Question 13}
\textbf{Metadata}

\begin{itemize}
  \item Question ID: P4-FrSubU12\_P2-FrCmp\_sonnet4\_Household Finance\_01
  \item Primary KC: FRACTIONS | Subtraction | subtracting unlike fractions with two different denominators not exceeding 12
  \item Secondary KC: FRACTIONS | Comparison and ordering | comparing and ordering fractions
  \item Topic: Household finance such as income, utility bills, money, interest, savings, instalment, mortgage, financial planning etc.
  \item Grade: Primary 4
\end{itemize}

\textbf{Question}

Mrs. Lim is tracking her family's monthly expenses. She found that $\frac{5}{8}$ of their monthly income goes to household bills, while $\frac{1}{3}$ of their monthly income goes to groceries. How much more of their monthly income is spent on household bills compared to groceries? Express your answer as a fraction in its simplest form. Is the amount spent on household bills greater than the amount spent on groceries?

\section*{Question 14}
\textbf{Metadata}

\begin{itemize}
  \item Question ID: P4-DcAdd2d\_P4-DcCnv2Fr\_sonnet4\_Household Finance\_01
  \item Primary KC: DECIMALS | Addition | adding decimals (up to 2 decimal places)
  \item Secondary KC: DECIMALS | Conversion from decimals to fraction | expressing decimals as fractions
  \item Topic: Household finance such as income, utility bills, money, interest, savings, instalment, mortgage, financial planning etc.
  \item Grade: Primary 4
\end{itemize}

\textbf{Question}

Mrs. Tan is planning her monthly household budget. She spent \textdollar12.50 on electricity, \textdollar8.75 on water, and \textdollar15.25 on gas for the month of January. She wants to know her total utility expenses and also express this total as a fraction in its simplest form.

\section*{Question 15}
\textbf{Metadata}

\begin{itemize}
  \item Question ID: P4-DcAdd2d\_P4-DcRnd3d\_sonnet4\_Household Finance\_01
  \item Primary KC: DECIMALS | Addition | adding decimals (up to 2 decimal places)
  \item Secondary KC: DECIMALS | Rounding | rounding decimals up to 3 decimal places to the nearest whole number, 1 decimal place and 2 decimal places 
  \item Topic: Household finance such as income, utility bills, money, interest, savings, instalment, mortgage, financial planning etc.
  \item Grade: Primary 4
\end{itemize}

\textbf{Question}

Mrs. Tan is calculating her monthly household expenses. Her electricity bill is \textdollar48.735, her water bill is \textdollar23.687, and her gas bill is \textdollar31.248. She wants to round each bill to the nearest cent (2 decimal places) before adding them up to find her total utility expenses. What is the total amount Mrs. Tan needs to pay for her utilities after rounding?

\section*{Question 16}
\textbf{Metadata}

\begin{itemize}
  \item Question ID: P4-DcSub2d\_P4-DcAdd2nd\_sonnet4\_Household Finance\_01
  \item Primary KC: DECIMALS | Subtraction | subtracting decimals (up to 2 decimal places)
  \item Secondary KC: DECIMALS | Addition | adding decimals
  \item Topic: Household finance such as income, utility bills, money, interest, savings, instalment, mortgage, financial planning etc.
  \item Grade: Primary 4
\end{itemize}

\textbf{Question}

Mrs. Tan received her monthly electricity and water bills. Her electricity bill was \textdollar45.80 and her water bill was \textdollar23.65. She had set aside \textdollar75.00 from her monthly budget to pay for these utility bills. How much money will she have left after paying both bills?

\section*{Question 17}
\textbf{Metadata}

\begin{itemize}
  \item Question ID: P4-DcDiv2d1d\_P4-DcCnv2Fr\_sonnet4\_Household Finance\_01
  \item Primary KC: DECIMALS | Division | dividing decimals (up to 2 decimal places) by a 1-digit whole number
  \item Secondary KC: DECIMALS | Conversion from decimals to fraction | expressing decimals as fractions
  \item Topic: Household finance such as income, utility bills, money, interest, savings, instalment, mortgage, financial planning etc.
  \item Grade: Primary 4
\end{itemize}

\textbf{Question}

Mrs. Tan received her electricity bill for \textdollar12.48 for the past 4 months. She wants to find out her average monthly electricity cost and express this amount as a fraction in its simplest form. What is Mrs. Tan's average monthly electricity cost, and how can this amount be expressed as a fraction?

\section*{Question 18}
\textbf{Metadata}

\begin{itemize}
  \item Question ID: P4-DcDiv2d1d\_P4-DcSub2nd\_sonnet4\_Household Finance\_01
  \item Primary KC: DECIMALS | Division | dividing decimals (up to 2 decimal places) by a 1-digit whole number
  \item Secondary KC: DECIMALS | Subtraction | subtracting decimals
  \item Topic: Household finance such as income, utility bills, money, interest, savings, instalment, mortgage, financial planning etc.
  \item Grade: Primary 4
\end{itemize}

\textbf{Question}

Mrs. Tan received her monthly electricity bill of \textdollar68.40. She also received her water bill of \textdollar25.80. She decided to pay both bills in equal instalments over 4 months. If she has \textdollar200.00 in her utility budget, how much money will she have left after setting aside the money needed for the first month's instalment?

\section*{Question 19}
\textbf{Metadata}

\begin{itemize}
  \item Question ID: P5-FrAddMix\_P3-FrSmp\_sonnet4\_Household Finance\_01
  \item Primary KC: FRACTIONS | Addition | adding mixed numbers
  \item Secondary KC: FRACTIONS | Simplifying | expressing a fraction in its simplest form
  \item Topic: Household finance such as income, utility bills, money, interest, savings, instalment, mortgage, financial planning etc.
  \item Grade: Primary 5
\end{itemize}

\textbf{Question}

Mrs. Tan is planning her family's monthly budget. She receives her salary in three installments during the month. In the first week, she receives $2\frac{3}{8}$ thousand. In the second week, she receives $1\frac{5}{6}$ thousand. In the third week, she receives $3\frac{1}{4}$ thousand. What is her total monthly salary in its simplest form?

\section*{Question 20}
\textbf{Metadata}

\begin{itemize}
  \item Question ID: P5-FrSubMix\_P2-FrAdd2nd\_sonnet4\_Household Finance\_01
  \item Primary KC: FRACTIONS | Subtraction | subtracting mixed numbers
  \item Secondary KC: FRACTIONS | Addition | adding fractions
  \item Topic: Household finance such as income, utility bills, money, interest, savings, instalment, mortgage, financial planning etc.
  \item Grade: Primary 5
\end{itemize}

\textbf{Question}

The Tan family is planning their monthly budget. They allocate $4\frac{3}{4}$ of their income for household expenses. Out of this amount, they spend $1\frac{1}{6}$ on groceries and $\frac{5}{12}$ on utility bills. How much of their income is left from the household expense allocation after paying for groceries and utility bills?

\section*{Question 21}
\textbf{Metadata}

\begin{itemize}
  \item Question ID: P5-FrMulImN\_P2-FrSub2nd\_sonnet4\_Household Finance\_01
  \item Primary KC: FRACTIONS | Multiplication | multiplying a proper/improper fraction and a whole number
  \item Secondary KC: FRACTIONS | Subtraction | subtracting fractions
  \item Topic: Household finance such as income, utility bills, money, interest, savings, instalment, mortgage, financial planning etc.
  \item Grade: Primary 5
\end{itemize}

\textbf{Question}

The Tan family's monthly electricity bill is \textdollar240. In January, they used $\frac{3}{4}$ of their usual electricity consumption due to cooler weather. In February, they used $\frac{5}{6}$ of their usual electricity consumption. How much less did they pay for electricity in January compared to February?

\section*{Question 22}
\textbf{Metadata}

\begin{itemize}
  \item Question ID: P5-FrMulPIm\_P2-FrSub2nd\_sonnet4\_Household Finance\_01
  \item Primary KC: FRACTIONS | Multiplication | multiplying a proper fraction and a proper/improper fractions
  \item Secondary KC: FRACTIONS | Subtraction | subtracting fractions
  \item Topic: Household finance such as income, utility bills, money, interest, savings, instalment, mortgage, financial planning etc.
  \item Grade: Primary 5
\end{itemize}

\textbf{Question}

Sarah's family spends $\frac{3}{5}$ of their monthly income on household expenses. Of this amount spent on household expenses, $\frac{2}{7}$ goes to utility bills and $\frac{1}{4}$ goes to groceries. If their monthly income is \textdollar4200, find:

(a) How much money is spent on utility bills?

(b) How much more money is spent on groceries than on utility bills?

\section*{Question 23}
\textbf{Metadata}

\begin{itemize}
  \item Question ID: P5-FrMulPIm\_P5-FrCnv2Dc\_sonnet4\_Household Finance\_01
  \item Primary KC: FRACTIONS | Multiplication | multiplying a proper fraction and a proper/improper fractions
  \item Secondary KC: FRACTIONS | Conversion to decimals | expressing fractions as decimals
  \item Topic: Household finance such as income, utility bills, money, interest, savings, instalment, mortgage, financial planning etc.
  \item Grade: Primary 5
\end{itemize}

\textbf{Question}

Mrs. Lim's monthly household budget is \textdollar2400. She spends $\frac{3}{8}$ of her budget on groceries and $\frac{2}{5}$ of the amount spent on groceries on household cleaning supplies. Find the amount spent on household cleaning supplies and express your answer as a decimal.

\section*{Question 24}
\textbf{Metadata}

\begin{itemize}
  \item Question ID: P5-FrMulMixN\_P2-FrCmp\_sonnet4\_Household Finance\_01
  \item Primary KC: FRACTIONS | Multiplication | multiplying a mixed number and a whole number
  \item Secondary KC: FRACTIONS | Comparison and ordering | comparing and ordering fractions
  \item Topic: Household finance such as income, utility bills, money, interest, savings, instalment, mortgage, financial planning etc.
  \item Grade: Primary 5
\end{itemize}

\textbf{Question}

Mrs. Tan is planning her family's monthly budget. She earns \textdollar3600 per month. She spends $2\frac{1}{4}$ times her utility bill amount on groceries each month. Her utility bill is \textdollar160 per month. Mrs. Tan also spends $\frac{3}{8}$ of her monthly income on rent and $\frac{1}{6}$ of her monthly income on transportation. Compare the amounts she spends on groceries and rent. Which expense is greater and by how much?

\section*{Question 25}
\textbf{Metadata}

\begin{itemize}
  \item Question ID: P5-FrMulMixN\_P3-FrSmp\_sonnet4\_Household Finance\_01
  \item Primary KC: FRACTIONS | Multiplication | multiplying a mixed number and a whole number
  \item Secondary KC: FRACTIONS | Simplifying | expressing a fraction in its simplest form
  \item Topic: Household finance such as income, utility bills, money, interest, savings, instalment, mortgage, financial planning etc.
  \item Grade: Primary 5
\end{itemize}

\textbf{Question}

The Tan family's monthly electricity bill is \textdollar84. Due to increased usage of air conditioning, their bill increased by $2\frac{1}{4}$ times the original amount. What is their new monthly electricity bill? Express your answer as a mixed number in its simplest form.

\section*{Question 26}
\textbf{Metadata}

\begin{itemize}
  \item Question ID: P5-FrMulMixN\_P5-FrCnv2Dc\_sonnet4\_Household Finance\_01
  \item Primary KC: FRACTIONS | Multiplication | multiplying a mixed number and a whole number
  \item Secondary KC: FRACTIONS | Conversion to decimals | expressing fractions as decimals
  \item Topic: Household finance such as income, utility bills, money, interest, savings, instalment, mortgage, financial planning etc.
  \item Grade: Primary 5
\end{itemize}

\textbf{Question}

Mrs. Tan pays a monthly electricity bill of \textdollar48. Due to increased usage during the school holidays, her electricity bill for December increased by $2\frac{1}{4}$ times the usual amount. If Mrs. Tan wants to track her expenses digitally, she needs to enter the December bill amount as a decimal. What is Mrs. Tan's December electricity bill expressed as a decimal?

\section*{Question 27}
\textbf{Metadata}

\begin{itemize}
  \item Question ID: P5-DcMul3dK\_P4-DcRnd3d\_sonnet4\_Household Finance\_01
  \item Primary KC: DECIMALS | Multiplication | multiplying decimals (up to 3 decimal places) by 10, 100, 1000 and their multiples
  \item Secondary KC: DECIMALS | Rounding | rounding decimals up to 3 decimal places to the nearest whole number, 1 decimal place and 2 decimal places 
  \item Topic: Household finance such as income, utility bills, money, interest, savings, instalment, mortgage, financial planning etc.
  \item Grade: Primary 5
\end{itemize}

\textbf{Question}

Mrs. Lim is calculating her monthly utility expenses. Her electricity bill shows that she used $142.75$ units of electricity last month. The electricity company charges \textdollar0.246 per unit. To estimate her total electricity cost, Mrs. Lim first multiplies the cost per unit by $1000$ to get the cost for $1000$ units, then calculates her actual bill. She wants to round her final answer to the nearest cent (2 decimal places) for her household budget planning. What is Mrs. Lim's electricity bill rounded to the nearest cent?

\section*{Question 28}
\textbf{Metadata}

\begin{itemize}
  \item Question ID: P5-DcDiv3dK\_P4-DcCnv2Fr\_sonnet4\_Household Finance\_01
  \item Primary KC: DECIMALS | Division | dividing decimals (up to 3 decimal places) by 10, 100, 1000 and their multiples
  \item Secondary KC: DECIMALS | Conversion from decimals to fraction | expressing decimals as fractions
  \item Topic: Household finance such as income, utility bills, money, interest, savings, instalment, mortgage, financial planning etc.
  \item Grade: Primary 5
\end{itemize}

\textbf{Question}

Mrs. Tan received her monthly electricity bill of \textdollar126.50. She wants to set aside money each day to pay for this bill over the next 50 days. How much should she save each day? Express your answer as a fraction in its simplest form.

\section*{Question 29}
\textbf{Metadata}

\begin{itemize}
  \item Question ID: P5-RtFndR\_P2-DcCnvN2D\_sonnet4\_Household Finance\_01
  \item Primary KC: RATE | Finding rate | finding rate given total amount and number of units
  \item Secondary KC: DECIMALS | Conversion to larger units | converting a measurement from a smaller unit to a larger unit in decimal form
  \item Topic: Household finance such as income, utility bills, money, interest, savings, instalment, mortgage, financial planning etc.
  \item Grade: Primary 5
\end{itemize}

\textbf{Question}

Mrs. Tan paid \textdollar43.50 for her electricity bill last month. She used 2900 units of electricity. What was the rate charged per unit of electricity in dollars? Express your answer as a decimal.

\section*{Question 30}
\textbf{Metadata}

\begin{itemize}
  \item Question ID: P5-RtFndR\_P2-DcCnvD2N\_sonnet4\_Household Finance\_01
  \item Primary KC: RATE | Finding rate | finding rate given total amount and number of units
  \item Secondary KC: DECIMALS | Conversion to smaller units | converting a measurement from a larger unit in decimal form to a smaller unit
  \item Topic: Household finance such as income, utility bills, money, interest, savings, instalment, mortgage, financial planning etc.
  \item Grade: Primary 5
\end{itemize}

\textbf{Question}

Mrs. Tan receives her monthly electricity bill which shows that her family used $127.5$ kilowatt-hours (kWh) of electricity. The total cost for the electricity consumption is \textdollar76.50. Find the rate charged per kilowatt-hour in dollars. Express your answer as the cost per watt-hour in cents.

\section*{Question 31}
\textbf{Metadata}

\begin{itemize}
  \item Question ID: P5-RtFndT\_P2-DcCnvN2D\_sonnet4\_Household Finance\_01
  \item Primary KC: RATE | Finding total amount | finding total amount, given rate and number of units
  \item Secondary KC: DECIMALS | Conversion to larger units | converting a measurement from a smaller unit to a larger unit in decimal form
  \item Topic: Household finance such as income, utility bills, money, interest, savings, instalment, mortgage, financial planning etc.
  \item Grade: Primary 5
\end{itemize}

\textbf{Question}

Mrs. Tan pays \textdollar0.25 per kilowatt-hour (kWh) for electricity. Last month, her family used 2840 watts of power for their air conditioner that ran for 150 hours. How much did Mrs. Tan pay for the electricity used by the air conditioner? Express your answer in dollars.

\section*{Question 32}
\textbf{Metadata}

\begin{itemize}
  \item Question ID: P5-RtFndU\_P2-DcCnvD2N\_sonnet4\_Household Finance\_01
  \item Primary KC: RATE | Finding number of unit | finding number of units given rate and total amount
  \item Secondary KC: DECIMALS | Conversion to smaller units | converting a measurement from a larger unit in decimal form to a smaller unit
  \item Topic: Household finance such as income, utility bills, money, interest, savings, instalment, mortgage, financial planning etc.
  \item Grade: Primary 5
\end{itemize}

\textbf{Question}

Mrs. Chen pays her electricity bill every month. The electricity company charges \textdollar0.25 per kilowatt-hour (kWh) of electricity used. Last month, Mrs. Chen's electricity bill was \textdollar48.75. She wants to track her electricity usage more carefully and needs to convert this to watt-hours for her home energy monitoring system. How many kilowatt-hours of electricity did Mrs. Chen use last month? Express your answer in watt-hours.

\section*{Question 33}
\textbf{Metadata}

\begin{itemize}
  \item Question ID: P6-FrDivPN\_P3-FrSmp\_sonnet4\_Household Finance\_01
  \item Primary KC: FRACTIONS | Division | dividing a proper fraction by a whole number
  \item Secondary KC: FRACTIONS | Simplifying | expressing a fraction in its simplest form
  \item Topic: Household finance such as income, utility bills, money, interest, savings, instalment, mortgage, financial planning etc.
  \item Grade: Primary 6
\end{itemize}

\textbf{Question}

Sarah's family saves \textdollar600 every month for their emergency fund. They decide to divide $\frac{2}{3}$ of their monthly savings equally among 4 different savings accounts. How much money will be deposited into each savings account? Express your answer as a fraction in its simplest form.

\section*{Question 34}
\textbf{Metadata}

\begin{itemize}
  \item Question ID: P6-FrDivPP\_P2-FrCmp\_sonnet4\_Household Finance\_01
  \item Primary KC: FRACTIONS | Division | dividing a whole number/proper fraction by a proper fraction
  \item Secondary KC: FRACTIONS | Comparison and ordering | comparing and ordering fractions
  \item Topic: Household finance such as income, utility bills, money, interest, savings, instalment, mortgage, financial planning etc.
  \item Grade: Primary 6
\end{itemize}

\textbf{Question}

The Tan family allocates $\frac{3}{8}$ of their monthly income for household expenses. Out of this allocated amount, they spend $\frac{2}{5}$ on utility bills and the remaining amount on groceries. If the Tan family spent \textdollar420 on utility bills this month, find:
(a) How much money did the Tan family allocate for household expenses this month?
(b) How much money is the Tan family's monthly income?
(c) The family is considering two savings plans. Plan A allows them to save $\frac{1}{6}$ of their monthly income, while Plan B allows them to save $\frac{3}{16}$ of their monthly income. Which plan allows them to save more money?

\section*{Question 35}
\textbf{Metadata}

\begin{itemize}
  \item Question ID: P6-FrDivPP\_P3-FrSmp\_sonnet4\_Household Finance\_01
  \item Primary KC: FRACTIONS | Division | dividing a whole number/proper fraction by a proper fraction
  \item Secondary KC: FRACTIONS | Simplifying | expressing a fraction in its simplest form
  \item Topic: Household finance such as income, utility bills, money, interest, savings, instalment, mortgage, financial planning etc.
  \item Grade: Primary 6
\end{itemize}

\textbf{Question}

Mrs. Tan saves \textdollar$\frac{3}{4}$ of her monthly salary for her family's expenses. She decides to allocate $\frac{2}{5}$ of her savings equally among her 3 children for their school expenses. If Mrs. Tan's total savings amount to \textdollar360, what is the amount each child receives? Express your answer as a fraction in its simplest form and then as a dollar amount.

\section*{Question 36}
\textbf{Metadata}

\begin{itemize}
  \item Question ID: P6-PcFndWN\_P1-WNSub2nd\_sonnet4\_Household Finance\_01
  \item Primary KC: PERCENTAGE | Finding the whole | finding the whole given a part and the percentage
  \item Secondary KC: WHOLE NUMBERS | Subtraction | subtracting whole numbers
  \item Topic: Household finance such as income, utility bills, money, interest, savings, instalment, mortgage, financial planning etc.
  \item Grade: Primary 6
\end{itemize}

\textbf{Question}

Mrs. Tan received her monthly electricity bill and noticed that she had paid \textdollar84 more than usual. She found out that this \textdollar84 represented 40\% of her total electricity bill for the month. After reviewing her usage, she realized she could reduce her next month's bill by \textdollar25. What will be her electricity bill next month?

\section*{Question 37}
\textbf{Metadata}

\begin{itemize}
  \item Question ID: P6-PcFndChg\_P1-WNSub2nd\_sonnet4\_Household Finance\_01
  \item Primary KC: PERCENTAGE | Finding change | finding percentage increase/decrease
  \item Secondary KC: WHOLE NUMBERS | Subtraction | subtracting whole numbers
  \item Topic: Household finance such as income, utility bills, money, interest, savings, instalment, mortgage, financial planning etc.
  \item Grade: Primary 6
\end{itemize}

\textbf{Question}

The Tan family's monthly electricity bill was \textdollar180 in January. In February, their electricity bill increased to \textdollar207. What is the percentage increase in their electricity bill from January to February?

\section*{Question 38}
\textbf{Metadata}

\begin{itemize}
  \item Question ID: P6-PcFndChg\_P1-WNMul2nd\_sonnet4\_Household Finance\_01
  \item Primary KC: PERCENTAGE | Finding change | finding percentage increase/decrease
  \item Secondary KC: WHOLE NUMBERS | Multiplication | multiplying whole numbers
  \item Topic: Household finance such as income, utility bills, money, interest, savings, instalment, mortgage, financial planning etc.
  \item Grade: Primary 6
\end{itemize}

\textbf{Question}

Mrs. Tan's monthly electricity bill was \textdollar120 in January. Due to increased usage of air conditioning, her electricity bill increased by 25\% in February. In March, she decided to be more energy-conscious and reduced her electricity usage. Her March bill was 15\% less than her February bill. If Mrs. Tan pays her electricity bills in 4 equal monthly instalments, how much does she pay each month for her March electricity bill?

\section*{Question 39}
\textbf{Metadata}

\begin{itemize}
  \item Question ID: P6-RoFndDvqWN\_P1-WNAdd2nd\_sonnet4\_Household Finance\_01
  \item Primary KC: RATIO | Finding divided quantities | dividing a given quantity in a given ratio
  \item Secondary KC: WHOLE NUMBERS | Addition | adding whole numbers
  \item Topic: Household finance such as income, utility bills, money, interest, savings, instalment, mortgage, financial planning etc.
  \item Grade: Primary 6
\end{itemize}

\textbf{Question}

Mrs. Tan received a bonus of \textdollar1800 from her company. She decided to divide this money among three categories in the ratio $2:3:4$ for savings, household expenses, and her children's education fund respectively. After allocating the money according to this ratio, she received an additional \textdollar200 from selling some old furniture. She added this \textdollar200 to the amount originally allocated for household expenses. How much money does Mrs. Tan now have for household expenses?

\section*{Question 40}
\textbf{Metadata}

\begin{itemize}
  \item Question ID: P6-RoFndRoWN\_P1-WNSub2nd\_sonnet4\_Household Finance\_01
  \item Primary KC: RATIO | Finding ratio | finding the ratio of two or three given whole numbers
  \item Secondary KC: WHOLE NUMBERS | Subtraction | subtracting whole numbers
  \item Topic: Household finance such as income, utility bills, money, interest, savings, instalment, mortgage, financial planning etc.
  \item Grade: Primary 6
\end{itemize}

\textbf{Question}

The Tan family's monthly household expenses are \textdollar2400. They spend \textdollar800 on groceries, \textdollar600 on utilities, and the rest on other expenses. After reviewing their budget, they decided to reduce their grocery spending by \textdollar200 and their utility spending by \textdollar150. Find the ratio of their new grocery spending to their new utility spending to their new spending on other expenses.

\section*{Question 41}
\textbf{Metadata}

\begin{itemize}
  \item Question ID: P6-RoFndRoWN\_P1-WNDiv2nd\_sonnet4\_Household Finance\_01
  \item Primary KC: RATIO | Finding ratio | finding the ratio of two or three given whole numbers
  \item Secondary KC: WHOLE NUMBERS | Division | dividing whole numbers
  \item Topic: Household finance such as income, utility bills, money, interest, savings, instalment, mortgage, financial planning etc.
  \item Grade: Primary 6
\end{itemize}

\textbf{Question}

The Tan family's monthly household expenses are divided into three categories: groceries, utilities, and entertainment. In January, they spent \textdollar1800 on groceries, \textdollar900 on utilities, and \textdollar450 on entertainment. To better manage their budget, Mr. Tan wants to find the ratio of their spending on groceries to utilities to entertainment. Before calculating the ratio, he needs to first divide the total amount spent in each category by 150 to convert the amounts into smaller, more manageable units for his budget planning spreadsheet. What is the ratio of groceries to utilities to entertainment after this conversion?

\section*{Question 42}
\textbf{Metadata}

\begin{itemize}
  \item Question ID: P6-RoFndTmWN\_P1-WNAdd2nd\_sonnet4\_Household Finance\_01
  \item Primary KC: RATIO | Finding a missing term | finding the missing term in a pair of equivalent ratios
  \item Secondary KC: WHOLE NUMBERS | Addition | adding whole numbers
  \item Topic: Household finance such as income, utility bills, money, interest, savings, instalment, mortgage, financial planning etc.
  \item Grade: Primary 6
\end{itemize}

\textbf{Question}

Mrs. Tan manages her household budget by allocating money for groceries and utilities in the ratio $3:2$. In January, she spent \textdollar180 on groceries. In February, she increased her total spending on groceries and utilities by \textdollar60 compared to January, while maintaining the same ratio. How much did she spend on utilities in February?

\section*{Question 43}
\textbf{Metadata}

\begin{itemize}
  \item Question ID: P6-AgRepLrEx\_P6-AgEvlLrEx\_sonnet4\_Household Finance\_01
  \item Primary KC: ALGEBRA | Representation and concept | translation of real-world situations into linear algebraic expressions
  \item Secondary KC: ALGEBRA | Evaluation | evaluating linear expressions by substitution
  \item Topic: Household finance such as income, utility bills, money, interest, savings, instalment, mortgage, financial planning etc.
  \item Grade: Primary 6
\end{itemize}

\textbf{Question}

Mrs. Chen is planning her monthly household budget. She earns a fixed salary of \textdollar3200 per month. Additionally, she receives \textdollar15 for each hour of overtime work she does. Her monthly expenses include rent of \textdollar800, groceries costing \textdollar12 per person in her family, and utility bills totaling \textdollar180. Mrs. Chen's family has 4 people. 

(a) Write an algebraic expression for Mrs. Chen's total monthly income if she works $h$ hours of overtime.

(b) Write an algebraic expression for her total monthly expenses.

(c) If Mrs. Chen works 8 hours of overtime this month, calculate her total monthly income and determine how much money she will have left after paying all her expenses.

\section*{Question 44}
\textbf{Metadata}

\begin{itemize}
  \item Question ID: P6-AgSlvLrN\_P6-AgRepLrEx\_sonnet4\_Household Finance\_01
  \item Primary KC: ALGEBRA | Solving simple linear equations | solving linear equations involving whole number coefficient and one variable only
  \item Secondary KC: ALGEBRA | Representation and concept | translation of real-world situations into linear algebraic expressions
  \item Topic: Household finance such as income, utility bills, money, interest, savings, instalment, mortgage, financial planning etc.
  \item Grade: Primary 6
\end{itemize}

\textbf{Question}

Sarah's family pays a monthly electricity bill that consists of a fixed service charge plus a variable charge based on their electricity usage. Last month, they used $320$ units of electricity and paid a total bill of \textdollar$68$. This month, they used $280$ units of electricity and paid a total bill of \textdollar$60$. What is the fixed service charge that Sarah's family pays each month?

\section*{Question 45}
\textbf{Metadata}

\begin{itemize}
  \item Question ID: O1-RoRepFr\_P2-FrSub2nd\_sonnet4\_Household Finance\_01
  \item Primary KC: RATIO | Representation and concept | ratios involving fractions
  \item Secondary KC: FRACTIONS | Subtraction | subtracting fractions
  \item Topic: Household finance such as income, utility bills, money, interest, savings, instalment, mortgage, financial planning etc.
  \item Grade: Secondary O-level 1
\end{itemize}

\textbf{Question}

The Tan family's monthly household expenses are divided among three categories: utilities, groceries, and savings. The ratio of their spending on utilities to groceries to savings is $\frac{2}{3} : \frac{5}{6} : \frac{1}{2}$. If their total monthly budget for these three categories is \textdollar1800, find:

(a) The amount spent on each category.

(b) If the family decides to reduce their grocery spending by $\frac{1}{8}$ of the original grocery amount and their utility spending by $\frac{1}{12}$ of the original utility amount, what is the new ratio of utilities to groceries to savings?

\section*{Question 46}
\textbf{Metadata}

\begin{itemize}
  \item Question ID: O1-RoRepFr\_P6-FrDiv2nd\_sonnet4\_Household Finance\_01
  \item Primary KC: RATIO | Representation and concept | ratios involving fractions
  \item Secondary KC: FRACTIONS | Division | fraction division
  \item Topic: Household finance such as income, utility bills, money, interest, savings, instalment, mortgage, financial planning etc.
  \item Grade: Secondary O-level 1
\end{itemize}

\textbf{Question}

The Tan family allocates their monthly household budget in the ratio $2\frac{1}{4} : 1\frac{1}{2} : \frac{3}{4}$ for housing expenses, food expenses, and savings respectively. If their total monthly budget is \textdollar4200, find:

(a) How much money is allocated for each category?

(b) The family decides to increase their savings by taking $\frac{1}{5}$ of their food budget and adding it to their savings. What is the new ratio of housing expenses to food expenses to savings?

\section*{Question 47}
\textbf{Metadata}

\begin{itemize}
  \item Question ID: O1-RoRepDc\_P4-DcSub2nd\_sonnet4\_Household Finance\_01
  \item Primary KC: RATIO | Representation and concept | ratios involving decimals
  \item Secondary KC: DECIMALS | Subtraction | subtracting decimals
  \item Topic: Household finance such as income, utility bills, money, interest, savings, instalment, mortgage, financial planning etc.
  \item Grade: Secondary O-level 1
\end{itemize}

\textbf{Question}

The Tan family's monthly household expenses are divided into three categories: utilities, groceries, and entertainment. The ratio of their spending on utilities to groceries to entertainment is $2.5 : 4.2 : 1.8$. In January, they spent a total of \textdollar850 on these three categories. However, in February, they managed to reduce their utilities bill by \textdollar15.50 and their groceries spending by \textdollar22.75, while keeping their entertainment spending the same. What was the Tan family's total spending on these three categories in February?

\section*{Question 48}
\textbf{Metadata}

\begin{itemize}
  \item Question ID: O1-PcRep2q\_O1-PcCnv2Dc\_sonnet4\_Household Finance\_01
  \item Primary KC: PERCENTAGE | Representation and concept | comparing two quantities by percentage
  \item Secondary KC: PERCENTAGE | Conversion to decimals | expressing percentage as a decimal
  \item Topic: Household finance such as income, utility bills, money, interest, savings, instalment, mortgage, financial planning etc.
  \item Grade: Secondary O-level 1
\end{itemize}

\textbf{Question}

The Tan family's monthly household income is \textdollar4800. They spend \textdollar1440 on rent and \textdollar960 on groceries each month. Express the amount spent on rent as a percentage of their total income, then convert this percentage to a decimal. Similarly, express the amount spent on groceries as a percentage of their total income and convert it to a decimal. Compare these two percentages to determine which expense takes up a larger portion of their income and by how much.

\section*{Question 49}
\textbf{Metadata}

\begin{itemize}
  \item Question ID: O1-PcFndRslt\_P1-WNDiv2nd\_sonnet4\_Household Finance\_01
  \item Primary KC: PERCENTAGE | Finding result after change | increasing/decreasing a quantity by a given percentage
  \item Secondary KC: WHOLE NUMBERS | Division | dividing whole numbers
  \item Topic: Household finance such as income, utility bills, money, interest, savings, instalment, mortgage, financial planning etc.
  \item Grade: Secondary O-level 1
\end{itemize}

\textbf{Question}

The Tan family's monthly electricity bill is \textdollar240. Due to energy conservation efforts, they managed to reduce their electricity consumption by $15\%$ this month. If they want to split the reduced bill equally among 4 family members for budgeting purposes, how much should each family member contribute?

\section*{Question 50}
\textbf{Metadata}

\begin{itemize}
  \item Question ID: O1-AgSlvFrLr\_O1-AgRepEq\_sonnet4\_Household Finance\_01
  \item Primary KC: ALGEBRA | Solving | solving simple fractional equations that can be reduced to linear equations
  \item Secondary KC: ALGEBRA | Representation and concept | translation of simple real-world situations to equations
  \item Topic: Household finance such as income, utility bills, money, interest, savings, instalment, mortgage, financial planning etc.
  \item Grade: Secondary O-level 1
\end{itemize}

\textbf{Question}

Sarah works part-time and earns a monthly salary. She spends $\frac{2}{5}$ of her salary on household expenses and saves \textdollar240 each month. If her remaining money after expenses and savings is \textdollar60, find her monthly salary.

\section*{Question 51}
\textbf{Metadata}

\begin{itemize}
  \item Question ID: O2-RoRepIvP\_P1-WNMul2nd\_sonnet4\_Household Finance\_01
  \item Primary KC: RATIO | Representation and concept | inverse proportion
  \item Secondary KC: WHOLE NUMBERS | Multiplication | multiplying whole numbers
  \item Topic: Household finance such as income, utility bills, money, interest, savings, instalment, mortgage, financial planning etc.
  \item Grade: Secondary O-level 2
\end{itemize}

\textbf{Question}

The Johnson family's monthly electricity bill is inversely proportional to the number of energy-saving LED bulbs they use in their home. When they use 8 LED bulbs, their monthly electricity bill is \textdollar120. The family decides to purchase additional LED bulbs to reduce their electricity costs. If they want to reduce their monthly bill to \textdollar80, how many LED bulbs do they need to use? Each LED bulb costs \textdollar15. How much will the family need to spend on purchasing the additional LED bulbs?

\section*{Question 52}
\textbf{Metadata}

\begin{itemize}
  \item Question ID: O2-RoRepIvP\_P1-WNDiv2nd\_sonnet4\_Household Finance\_01
  \item Primary KC: RATIO | Representation and concept | inverse proportion
  \item Secondary KC: WHOLE NUMBERS | Division | dividing whole numbers
  \item Topic: Household finance such as income, utility bills, money, interest, savings, instalment, mortgage, financial planning etc.
  \item Grade: Secondary O-level 2
\end{itemize}

\textbf{Question}

The Tan family's monthly electricity bill is inversely proportional to the number of energy-saving devices they use in their home. When they use 4 energy-saving devices, their monthly electricity bill is \textdollar240. Due to budget constraints, they want to reduce their electricity bill to \textdollar160 per month. How many energy-saving devices do they need to use? If each energy-saving device costs \textdollar45, what is the total additional cost they need to spend on buying the extra devices?

\section*{Question 53}
\textbf{Metadata}

\begin{itemize}
  \item Question ID: O2-AgSlvLr2v\_O1-AgRepEq\_sonnet4\_Household Finance\_01
  \item Primary KC: ALGEBRA | Solving | solving linear equations in two variables
  \item Secondary KC: ALGEBRA | Representation and concept | translation of simple real-world situations to equations
  \item Topic: Household finance such as income, utility bills, money, interest, savings, instalment, mortgage, financial planning etc.
  \item Grade: Secondary O-level 2
\end{itemize}

\textbf{Question}

The Tan family is planning their monthly budget. They noticed that their combined electricity and water bills total \textdollar180. The electricity bill is \textdollar30 more than twice the water bill. Find the cost of each utility bill.

\section*{Question 54}
\textbf{Metadata}

\begin{itemize}
  \item Question ID: O2-SPFndmdn\_O2-SPFndmode\_sonnet4\_Household Finance\_01
  \item Primary KC: STATISTICS AND PROBABILITY | Finding median | Finding median for a set of data
  \item Secondary KC: STATISTICS AND PROBABILITY | Finding mode | Finding mode for a set of data
  \item Topic: Household finance such as income, utility bills, money, interest, savings, instalment, mortgage, financial planning etc.
  \item Grade: Secondary O-level 2
\end{itemize}

\textbf{Question}

Mrs. Tan is reviewing her family's monthly electricity bills for the past year to better plan their household budget. The monthly electricity bills (in dollars) for the 12 months were: \textdollar85, \textdollar92, \textdollar78, \textdollar85, \textdollar95, \textdollar88, \textdollar85, \textdollar90, \textdollar82, \textdollar94, \textdollar87, \textdollar89.

(a) Find the median monthly electricity bill.

(b) Find the mode of the monthly electricity bills.

(c) Mrs. Tan wants to set aside money each month for electricity bills. Should she use the median or the mode as a guide for her monthly budget planning? Explain your reasoning based on the values you found.

\section*{Question 55}
\textbf{Metadata}

\begin{itemize}
  \item Question ID: O2-SPFndmdn\_O3-SPFndPctl\_sonnet4\_Household Finance\_01
  \item Primary KC: STATISTICS AND PROBABILITY | Finding median | Finding median for a set of data
  \item Secondary KC: STATISTICS AND PROBABILITY | Finding percentiles | finding percentiles for a set of data
  \item Topic: Household finance such as income, utility bills, money, interest, savings, instalment, mortgage, financial planning etc.
  \item Grade: Secondary O-level 2
\end{itemize}

\textbf{Question}

Mrs. Tan is analyzing her family's monthly electricity bills for the past 12 months to better plan their household budget. The monthly bills (in dollars) are: \textdollar45, \textdollar52, \textdollar38, \textdollar61, \textdollar47, \textdollar55, \textdollar42, \textdollar58, \textdollar49, \textdollar53, \textdollar44, \textdollar56. (a) Find the median monthly electricity bill. (b) Mrs. Tan wants to set a monthly budget such that her electricity bill will not exceed this budget in 75% of the months. Find the 75th percentile of the electricity bills to help her determine this budget amount.

\section*{Question 56}
\textbf{Metadata}

\begin{itemize}
  \item Question ID: O3-BPOpr\_O3-BPRepPosI\_sonnet4\_Household Finance\_01
  \item Primary KC: BASE AND POWER | Operations | laws of indices
  \item Secondary KC: BASE AND POWER | Representation and concept  | positive indices that is not 1
  \item Topic: Household finance such as income, utility bills, money, interest, savings, instalment, mortgage, financial planning etc.
  \item Grade: Secondary O-level 3/4
\end{itemize}

\textbf{Question}

Sarah is planning her family's savings strategy. She invests \textdollar2000 in a compound interest account that doubles her money every 3 years. At the same time, her monthly utility bills follow a pattern where the electricity bill is \textdollar40, the water bill is \textdollar40$^2$, and the gas bill is \textdollar40$^3$. 

(a) Express the total value of Sarah's investment after 9 years using indices, then calculate the actual amount.

(b) Calculate the total monthly utility bill and express your final answer using the laws of indices to simplify $40^2 \times 40^3$.

\section*{Question 57}
\textbf{Metadata}

\begin{itemize}
  \item Question ID: O3-BPOpr\_O3-BPRepFrI\_sonnet4\_Household Finance\_01
  \item Primary KC: BASE AND POWER | Operations | laws of indices
  \item Secondary KC: BASE AND POWER | Representation and concept  | fractional indices
  \item Topic: Household finance such as income, utility bills, money, interest, savings, instalment, mortgage, financial planning etc.
  \item Grade: Secondary O-level 3/4
\end{itemize}

\textbf{Question}

Sarah is planning her retirement savings strategy. She invests \textdollar8000 in a special savings account that compounds interest according to the formula $A = P \times 2^{t/4}$, where $A$ is the final amount, $P$ is the principal amount, and $t$ is the time in years.

(a) After how many years will her investment be worth \textdollar32000?

(b) Sarah wants to know the value of her investment after 6 years. Calculate the amount she will have, giving your answer to the nearest dollar.

\section*{Question 58}
\textbf{Metadata}

\begin{itemize}
  \item Question ID: O3-STOprUn\_O3-STOprIns\_sonnet4\_Household Finance\_01
  \item Primary KC: SET | Set operations | union of two sets
  \item Secondary KC: SET | Set operations | intersection of two sets
  \item Topic: Household finance such as income, utility bills, money, interest, savings, instalment, mortgage, financial planning etc.
  \item Grade: Secondary O-level 3/4
\end{itemize}

\textbf{Question}

The Chen family is reviewing their household expenses for financial planning. They categorize their monthly bills into two groups based on payment methods. Set $A$ represents bills they can pay using their credit card: $A = \{\text{electricity, water, internet, phone, insurance}\}$. Set $B$ represents bills they can pay through online banking: $B = \{\text{water, internet, mortgage, car loan, insurance}\}$. 

To optimize their payment strategy and earn reward points, they want to identify: 
(a) All bills that can be paid using either credit card or online banking (or both methods) 
(b) Bills that can be paid using both credit card and online banking 
(c) If the total number of distinct bills they pay monthly is 7, and they know that 2 bills can only be paid by methods other than credit card or online banking, verify whether their sets $A$ and $B$ are correctly identified.

\section*{Question 59}
\textbf{Metadata}

\begin{itemize}
  \item Question ID: O3-MXMulSM\_O3-MXSub\_sonnet4\_Household Finance\_01
  \item Primary KC: MATRICES | Multiplication | product of a scalar quantity and a matrix
  \item Secondary KC: MATRICES | Subtraction | subtraction of matrices
  \item Topic: Household finance such as income, utility bills, money, interest, savings, instalment, mortgage, financial planning etc.
  \item Grade: Secondary O-level 3/4
\end{itemize}

\textbf{Question}

The Tan family tracks their monthly expenses using matrices. In January, their expense matrix $A$ represents the costs in dollars for different categories: utilities, groceries, and entertainment for two family members (parents and children).

$A = \begin{pmatrix} 120 & 80 \\ 400 & 300 \\ 150 & 100 \end{pmatrix}$

where the first column represents parents' expenses and the second column represents children's expenses for utilities (first row), groceries (second row), and entertainment (third row).

In February, the family decides to reduce their expenses. Their February expense matrix $B$ is:

$B = \begin{pmatrix} 110 & 75 \\ 380 & 280 \\ 120 & 90 \end{pmatrix}$

The family financial planner suggests that they should aim to save an additional amount equal to $0.15$ times their January expenses while maintaining their February spending levels.

(a) Calculate the additional savings target matrix $S$ that represents $0.15$ times the January expenses.

(b) Find the net financial impact matrix $N$ by subtracting the February expenses from the additional savings target. What does this matrix represent in the context of the family's financial planning?

\section*{Question 60}
\textbf{Metadata}

\begin{itemize}
  \item Question ID: O3-MXMulSM\_O3-MXMul\_sonnet4\_Household Finance\_01
  \item Primary KC: MATRICES | Multiplication | product of a scalar quantity and a matrix
  \item Secondary KC: MATRICES | Multiplication | multiplication of matrices
  \item Topic: Household finance such as income, utility bills, money, interest, savings, instalment, mortgage, financial planning etc.
  \item Grade: Secondary O-level 3/4
\end{itemize}

\textbf{Question}

The Chen family tracks their monthly household expenses using matrices. Their expenses for utilities, groceries, and transportation over the past 3 months are represented by the matrix $E = \begin{pmatrix} 180 & 420 & 150 \\ 195 & 385 & 165 \\ 210 & 450 & 140 \end{pmatrix}$, where each row represents a month (January, February, March) and each column represents the expense category (utilities, groceries, transportation) in dollars.

Due to inflation, the family expects all their expenses to increase by 15% next quarter. Additionally, they want to calculate their total quarterly expenses by multiplying the expense matrix by the column vector $V = \begin{pmatrix} 1 \\ 1 \\ 1 \end{pmatrix}$.

(a) Find the matrix representing the family's expected expenses for the next quarter after the 15% increase.

(b) Calculate the total monthly expenses for each of the original 3 months by finding $E \times V$.

(c) What will be the total quarterly expense after the 15% increase?

\section*{Question 61}
\textbf{Metadata}

\begin{itemize}
  \item Question ID: O3-MXMul\_O3-MXSub\_sonnet4\_Household Finance\_01
  \item Primary KC: MATRICES | Multiplication | multiplication of matrices
  \item Secondary KC: MATRICES | Subtraction | subtraction of matrices
  \item Topic: Household finance such as income, utility bills, money, interest, savings, instalment, mortgage, financial planning etc.
  \item Grade: Secondary O-level 3/4
\end{itemize}

\textbf{Question}

The Chen family tracks their monthly household expenses using matrices. They categorize their expenses into three types: utilities (electricity, water, gas), groceries, and transportation. Over the past three months, their expenses were recorded as follows:

January expenses: $\begin{pmatrix} 120 & 450 & 200 \end{pmatrix}$
February expenses: $\begin{pmatrix} 135 & 480 & 180 \end{pmatrix}$
March expenses: $\begin{pmatrix} 110 & 520 & 220 \end{pmatrix}$

The family wants to calculate their total quarterly expenses and compare them with their budget. Their quarterly budget matrix is $\begin{pmatrix} 400 & 1500 & 650 \end{pmatrix}$.

To help with future planning, they also use a projection matrix $P = \begin{pmatrix} 1.05 \\ 1.08 \\ 1.03 \end{pmatrix}$ which represents the expected percentage increase for each expense category for the next quarter.

(a) Find the total expenses for the quarter by adding all three monthly expense matrices.

(b) Calculate the difference between their actual quarterly expenses and their budget.

(c) Use matrix multiplication to project their next quarter's expenses by multiplying the total quarterly expenses matrix with the projection matrix $P$.

\section*{Question 62}
\textbf{Metadata}

\begin{itemize}
  \item Question ID: O3-SPFndQtl\_O3-SPFndIQR\_sonnet4\_Household Finance\_01
  \item Primary KC: STATISTICS AND PROBABILITY | Finding quartiles | finding quartiles for a set of data
  \item Secondary KC: STATISTICS AND PROBABILITY | Finding range | finding interquartile range as measures of spread for a set of data 
  \item Topic: Household finance such as income, utility bills, money, interest, savings, instalment, mortgage, financial planning etc.
  \item Grade: Secondary O-level 3/4
\end{itemize}

\textbf{Question}

Mrs. Chen is analyzing her family's monthly electricity bills over the past 12 months to better plan their household budget. The monthly electricity bills (in dollars) were recorded as follows: \textdollar85, \textdollar92, \textdollar78, \textdollar105, \textdollar88, \textdollar96, \textdollar82, \textdollar91, \textdollar87, \textdollar99, \textdollar83, \textdollar94. (a) Find the first quartile ($Q_1$), second quartile ($Q_2$), and third quartile ($Q_3$) of the electricity bills. (b) Calculate the interquartile range of the electricity bills. (c) Mrs. Chen wants to budget for electricity costs and decides to set aside an amount that covers the middle 50% of her monthly bills. Based on the interquartile range, what range of monthly amounts should she expect for typical electricity bills?

\section*{Question 63}
\textbf{Metadata}

\begin{itemize}
  \item Question ID: O3-SPMulProb\_O2-SPRepPrSE\_sonnet4\_Household Finance\_01
  \item Primary KC: STATISTICS AND PROBABILITY | Multiplication | multiplication of probabilities
  \item Secondary KC: STATISTICS AND PROBABILITY | Representation and concept | probability of single events
  \item Topic: Household finance such as income, utility bills, money, interest, savings, instalment, mortgage, financial planning etc.
  \item Grade: Secondary O-level 3/4
\end{itemize}

\textbf{Question}

The Lee family is reviewing their monthly household expenses to create a financial plan. They have identified that their electricity bill payment is sometimes delayed due to various factors. Based on past records, the probability that they pay their electricity bill on time in any given month is $\frac{3}{4}$. The probability that they pay their water bill on time in any given month is $\frac{4}{5}$. The two bill payments are independent of each other. 

(a) What is the probability that they pay their electricity bill on time in a specific month?

(b) What is the probability that they pay both their electricity and water bills on time in the same month?

(c) If they want to maintain this payment pattern, what is the probability that they will pay both bills on time for two consecutive months?

\end{document}
