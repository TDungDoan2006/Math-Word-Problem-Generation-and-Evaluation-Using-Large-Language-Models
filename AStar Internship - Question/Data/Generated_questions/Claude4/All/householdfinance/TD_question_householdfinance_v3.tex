\documentclass{article}
\usepackage[utf8]{inputenc}
\usepackage{amsmath}
\usepackage{amsfonts}
\usepackage{amssymb}
\usepackage{graphicx}
\usepackage{hyperref}
\title{'TD Questions household finance v3 CLAUDE '}
\author{Tien Dung Doan}
\begin{document}
\maketitle
\section*{Question 1}
\textbf{Metadata}

\begin{itemize}
  \item Question ID: P3-WNAdd4d\_P1-WNCmp\_sonnet4\_Household Finance\_01
  \item Primary KC: WHOLE NUMBERS | Addition | adding whole numbers up to 4 digits
  \item Secondary KC: WHOLE NUMBERS | Comparison and ordering | comparing and ordering whole numbers
  \item Topic: Household finance such as income, utility bills, money, interest, savings, instalment, mortgage, financial planning etc.
  \item Grade: Primary 3
\end{itemize}

\textbf{Question}

The Tan family is planning their monthly budget. Mr. Tan earns \textdollar2845 per month and Mrs. Tan earns \textdollar1967 per month. Their monthly expenses include rent of \textdollar1250, utilities of \textdollar185, groceries of \textdollar720, and transportation of \textdollar395. 

(a) What is the Tan family's total monthly income?

(b) What is their total monthly expenses?

(c) Compare their total monthly income and total monthly expenses. Do they have enough income to cover all their expenses?

\section*{Question 2}
\textbf{Metadata}

\begin{itemize}
  \item Question ID: P3-WNSub4d\_P1-WNAdd2nd\_sonnet4\_Household Finance\_01
  \item Primary KC: WHOLE NUMBERS | Subtraction | subtracting whole numbers up to 4 digits
  \item Secondary KC: WHOLE NUMBERS | Addition | adding whole numbers
  \item Topic: Household finance such as income, utility bills, money, interest, savings, instalment, mortgage, financial planning etc.
  \item Grade: Primary 3
\end{itemize}

\textbf{Question}

The Tan family is planning their monthly household budget. In January, Mr. Tan earned \textdollar3250 from his job and Mrs. Tan earned \textdollar2180 from her part-time work. Their total monthly expenses include \textdollar1820 for rent, \textdollar650 for groceries, \textdollar380 for utilities, and \textdollar290 for transport. How much money does the Tan family have left for savings after paying all their expenses?

\section*{Question 3}
\textbf{Metadata}

\begin{itemize}
  \item Question ID: P3-WNDivRmd3d\_P1-WNSub2nd\_sonnet4\_Household Finance\_01
  \item Primary KC: WHOLE NUMBERS | Division | dividing whole numbers up to 3 digits by 1 digit with remainder 
  \item Secondary KC: WHOLE NUMBERS | Subtraction | subtracting whole numbers
  \item Topic: Household finance such as income, utility bills, money, interest, savings, instalment, mortgage, financial planning etc.
  \item Grade: Primary 3
\end{itemize}

\textbf{Question}

Mrs. Lim has \textdollar348 in her savings account. She wants to divide this money equally among her 7 grandchildren for their Chinese New Year ang bao. However, she also needs to keep \textdollar25 for buying oranges for the celebration. How much money will each grandchild receive?

\section*{Question 4}
\textbf{Metadata}

\begin{itemize}
  \item Question ID: P3-WNMul3d1d\_P1-WNCmp\_sonnet4\_Household Finance\_01
  \item Primary KC: WHOLE NUMBERS | Multiplication | multiplying whole numbers up to 3 digits by 1 digit
  \item Secondary KC: WHOLE NUMBERS | Comparison and ordering | comparing and ordering whole numbers
  \item Topic: Household finance such as income, utility bills, money, interest, savings, instalment, mortgage, financial planning etc.
  \item Grade: Primary 3
\end{itemize}

\textbf{Question}

Mrs. Tan is comparing the monthly utility bills for three different apartments. Apartment A has an electricity bill of \textdollar138 per month. Apartment B has an electricity bill that is 3 times as much as Apartment A. Apartment C has an electricity bill of \textdollar425 per month. Which apartment has the highest electricity bill and which has the lowest? What is the electricity bill for Apartment B?

\section*{Question 5}
\textbf{Metadata}

\begin{itemize}
  \item Question ID: P3-WNDiv3d1d\_P1-WNCmp\_sonnet4\_Household Finance\_01
  \item Primary KC: WHOLE NUMBERS | Division | dividing whole numbers up to 3 digits by 1 digit
  \item Secondary KC: WHOLE NUMBERS | Comparison and ordering | comparing and ordering whole numbers
  \item Topic: Household finance such as income, utility bills, money, interest, savings, instalment, mortgage, financial planning etc.
  \item Grade: Primary 3
\end{itemize}

\textbf{Question}

Mrs. Chen needs to plan her monthly household budget. She has \textdollar648 saved for utilities and wants to divide this amount equally among 3 different bills: electricity, water, and gas. At the same time, she has \textdollar756 set aside for groceries that she wants to divide equally over 4 weeks. How much money will Mrs. Chen allocate for each utility bill and how much for groceries each week? Which weekly expense is higher - utilities or groceries?

\section*{Question 6}
\textbf{Metadata}

\begin{itemize}
  \item Question ID: P3-FrSubRl12\_P2-FrCmp\_sonnet4\_Household Finance\_01
  \item Primary KC: FRACTIONS | Subtraction | subtracting two related fractions within one whole with denominators of given fractions not exceeding 12
  \item Secondary KC: FRACTIONS | Comparison and ordering | comparing and ordering fractions
  \item Topic: Household finance such as income, utility bills, money, interest, savings, instalment, mortgage, financial planning etc.
  \item Grade: Primary 3
\end{itemize}

\textbf{Question}

Mrs. Tan is planning her monthly household budget. She allocates $\frac{5}{8}$ of her monthly income for essential expenses like food and utilities. Out of this amount, she spends $\frac{3}{8}$ of her income on groceries. How much of her monthly income does she have left from the essential expenses allocation after buying groceries? Compare this remaining amount with the grocery expenses. Which amount is larger?

\section*{Question 7}
\textbf{Metadata}

\begin{itemize}
  \item Question ID: P4-WNMul4d1d\_P1-WNCmp\_sonnet4\_Household Finance\_01
  \item Primary KC: WHOLE NUMBERS | Multiplication | multiplying whole numbers up to 4 digits by 1 digit or up to 3 digits by 2 digits
  \item Secondary KC: WHOLE NUMBERS | Comparison and ordering | comparing and ordering whole numbers
  \item Topic: Household finance such as income, utility bills, money, interest, savings, instalment, mortgage, financial planning etc.
  \item Grade: Primary 4
\end{itemize}

\textbf{Question}

The Tan family is comparing electricity bills from three different months to plan their budget. In January, their daily electricity usage was 24 units, and there are 31 days in January. In February, their daily electricity usage was 28 units, and there are 28 days in February. In March, their daily electricity usage was 26 units, and there are 31 days in March. Each unit of electricity costs \textdollar3. (a) Calculate the total electricity cost for each month. (b) Arrange the three months in order from the highest electricity cost to the lowest electricity cost.

\section*{Question 8}
\textbf{Metadata}

\begin{itemize}
  \item Question ID: P4-WNMul4d1d\_P4-WNRnd5d\_sonnet4\_Household Finance\_01
  \item Primary KC: WHOLE NUMBERS | Multiplication | multiplying whole numbers up to 4 digits by 1 digit or up to 3 digits by 2 digits
  \item Secondary KC: WHOLE NUMBERS | Rounding | rounding whole numbers up to 100000 to the nearest 10, 100 or 1000 
  \item Topic: Household finance such as income, utility bills, money, interest, savings, instalment, mortgage, financial planning etc.
  \item Grade: Primary 4
\end{itemize}

\textbf{Question}

Mrs. Lim is planning her family's monthly budget. She calculates that their electricity bill is \textdollar127 per month and their water bill is \textdollar89 per month. To plan for the whole year, she wants to estimate their total utility costs for $12$ months. She decides to round each monthly bill to the nearest \textdollar10 before calculating the yearly total. What is Mrs. Lim's estimated total utility cost for the year?

\section*{Question 9}
\textbf{Metadata}

\begin{itemize}
  \item Question ID: P4-WNDiv4d1d\_P1-WNSub2nd\_sonnet4\_Household Finance\_01
  \item Primary KC: WHOLE NUMBERS | Division | dividing whole numbers up to 4 digits by 1 digit
  \item Secondary KC: WHOLE NUMBERS | Subtraction | subtracting whole numbers
  \item Topic: Household finance such as income, utility bills, money, interest, savings, instalment, mortgage, financial planning etc.
  \item Grade: Primary 4
\end{itemize}

\textbf{Question}

The Tan family needs to save money for their daughter's education. They have \textdollar8736 in their savings account. They plan to divide this amount equally among 6 different investment funds. However, they also need to keep \textdollar528 for emergency expenses. How much money will they invest in each fund?

\section*{Question 10}
\textbf{Metadata}

\begin{itemize}
  \item Question ID: P4-WNDiv4d1d\_P4-WNRnd5d\_sonnet4\_Household Finance\_01
  \item Primary KC: WHOLE NUMBERS | Division | dividing whole numbers up to 4 digits by 1 digit
  \item Secondary KC: WHOLE NUMBERS | Rounding | rounding whole numbers up to 100000 to the nearest 10, 100 or 1000 
  \item Topic: Household finance such as income, utility bills, money, interest, savings, instalment, mortgage, financial planning etc.
  \item Grade: Primary 4
\end{itemize}

\textbf{Question}

The Wong family wants to save money for a vacation. They plan to save \textdollar2436 over the next 6 months by setting aside the same amount each month. Mrs. Wong also wants to round their monthly savings amount to the nearest \textdollar10 to make it easier to remember. How much should the Wong family save each month before rounding? What is the rounded amount they should actually save each month?

\section*{Question 11}
\textbf{Metadata}

\begin{itemize}
  \item Question ID: P4-FrAddU12\_P2-FrCmp\_sonnet4\_Household Finance\_01
  \item Primary KC: FRACTIONS | Addition | adding unlike fractions with two different denominators not exceeding 12
  \item Secondary KC: FRACTIONS | Comparison and ordering | comparing and ordering fractions
  \item Topic: Household finance such as income, utility bills, money, interest, savings, instalment, mortgage, financial planning etc.
  \item Grade: Primary 4
\end{itemize}

\textbf{Question}

Mrs. Tan is planning her monthly household budget. She spends $\frac{1}{3}$ of her monthly income on groceries and $\frac{1}{4}$ of her monthly income on utility bills. She wants to know what fraction of her total monthly income she spends on groceries and utility bills combined. After calculating this, she also wants to determine whether she spends more on groceries and utility bills combined, or on her mortgage payment, which takes up $\frac{7}{12}$ of her monthly income. What fraction of her monthly income does Mrs. Tan spend on groceries and utility bills combined? Does she spend more on groceries and utility bills combined, or on her mortgage payment?

\section*{Question 12}
\textbf{Metadata}

\begin{itemize}
  \item Question ID: P4-FrSubU12\_P3-FrSmp\_sonnet4\_Household Finance\_01
  \item Primary KC: FRACTIONS | Subtraction | subtracting unlike fractions with two different denominators not exceeding 12
  \item Secondary KC: FRACTIONS | Simplifying | expressing a fraction in its simplest form
  \item Topic: Household finance such as income, utility bills, money, interest, savings, instalment, mortgage, financial planning etc.
  \item Grade: Primary 4
\end{itemize}

\textbf{Question}

Mrs. Tan is planning her monthly household budget. She allocates $\frac{5}{8}$ of her monthly income for essential expenses like groceries and utilities. She also sets aside $\frac{1}{6}$ of her monthly income for her children's education fund. What fraction of her monthly income does she have left for other expenses and savings?

\section*{Question 13}
\textbf{Metadata}

\begin{itemize}
  \item Question ID: P4-DcAdd2d\_P4-DcCmp3d\_sonnet4\_Household Finance\_01
  \item Primary KC: DECIMALS | Addition | adding decimals (up to 2 decimal places)
  \item Secondary KC: DECIMALS | Comparison and ordering | comparing and ordering decimals up to 3 decimal places
  \item Topic: Household finance such as income, utility bills, money, interest, savings, instalment, mortgage, financial planning etc.
  \item Grade: Primary 4
\end{itemize}

\textbf{Question}

The Tan family is comparing their monthly utility bills for the past three months to plan their budget better. In January, their electricity bill was \textdollar45.60 and their water bill was \textdollar23.40. In February, their electricity bill was \textdollar48.75 and their water bill was \textdollar21.85. In March, their electricity bill was \textdollar44.20 and their water bill was \textdollar25.30. 

(a) What was the total utility bill for each month?

(b) Arrange the three months in order from the lowest total utility bill to the highest total utility bill.

\section*{Question 14}
\textbf{Metadata}

\begin{itemize}
  \item Question ID: P4-DcSub2d\_P4-DcCnv2Fr\_sonnet4\_Household Finance\_01
  \item Primary KC: DECIMALS | Subtraction | subtracting decimals (up to 2 decimal places)
  \item Secondary KC: DECIMALS | Conversion from decimals to fraction | expressing decimals as fractions
  \item Topic: Household finance such as income, utility bills, money, interest, savings, instalment, mortgage, financial planning etc.
  \item Grade: Primary 4
\end{itemize}

\textbf{Question}

Mrs. Tan is managing her household expenses for the month. She budgeted \textdollar45.80 for electricity bills but only spent \textdollar32.65. She wants to save the money she didn't spend and add it to her savings account. How much money did Mrs. Tan save from her electricity budget? Express your answer as a fraction in its simplest form.

\section*{Question 15}
\textbf{Metadata}

\begin{itemize}
  \item Question ID: P4-DcSub2d\_P4-DcRnd3d\_sonnet4\_Household Finance\_01
  \item Primary KC: DECIMALS | Subtraction | subtracting decimals (up to 2 decimal places)
  \item Secondary KC: DECIMALS | Rounding | rounding decimals up to 3 decimal places to the nearest whole number, 1 decimal place and 2 decimal places 
  \item Topic: Household finance such as income, utility bills, money, interest, savings, instalment, mortgage, financial planning etc.
  \item Grade: Primary 4
\end{itemize}

\textbf{Question}

Mrs. Tan is reviewing her monthly household expenses. Her electricity bill is \textdollar68.75, her water bill is \textdollar29.38, and her gas bill is \textdollar45.92. She had budgeted \textdollar150.00 for all three utility bills this month. How much money does she have left from her utility budget? Round your answer to the nearest whole number.

\section*{Question 16}
\textbf{Metadata}

\begin{itemize}
  \item Question ID: P4-DcMul2d1d\_P4-DcCnv2Fr\_sonnet4\_Household Finance\_01
  \item Primary KC: DECIMALS | Multiplication | multiplying decimals (up to 2 decimal places) by a 1-digit whole number
  \item Secondary KC: DECIMALS | Conversion from decimals to fraction | expressing decimals as fractions
  \item Topic: Household finance such as income, utility bills, money, interest, savings, instalment, mortgage, financial planning etc.
  \item Grade: Primary 4
\end{itemize}

\textbf{Question}

Mrs. Tan pays \textdollar0.75 for each unit of electricity she uses at home. Last month, her family used 8 units of electricity. How much did she pay for electricity in total? Express your answer as a fraction in its simplest form.

\section*{Question 17}
\textbf{Metadata}

\begin{itemize}
  \item Question ID: P4-DcMul2d1d\_P4-DcAdd2nd\_sonnet4\_Household Finance\_01
  \item Primary KC: DECIMALS | Multiplication | multiplying decimals (up to 2 decimal places) by a 1-digit whole number
  \item Secondary KC: DECIMALS | Addition | adding decimals
  \item Topic: Household finance such as income, utility bills, money, interest, savings, instalment, mortgage, financial planning etc.
  \item Grade: Primary 4
\end{itemize}

\textbf{Question}

Mrs. Tan receives her monthly utility bills. Her electricity bill is \textdollar23.45, her water bill is \textdollar18.60, and her gas bill is \textdollar12.75. She needs to pay these bills for 3 months. What is the total amount she needs to pay for all three types of utility bills over the 3 months?

\section*{Question 18}
\textbf{Metadata}

\begin{itemize}
  \item Question ID: P4-DcDiv2d1d\_P4-DcCmp3d\_sonnet4\_Household Finance\_01
  \item Primary KC: DECIMALS | Division | dividing decimals (up to 2 decimal places) by a 1-digit whole number
  \item Secondary KC: DECIMALS | Comparison and ordering | comparing and ordering decimals up to 3 decimal places
  \item Topic: Household finance such as income, utility bills, money, interest, savings, instalment, mortgage, financial planning etc.
  \item Grade: Primary 4
\end{itemize}

\textbf{Question}

Mrs. Tan receives her monthly utility bills for electricity, water, and gas. The total amount for all three bills is \textdollar84.60. She needs to divide this amount equally among her 3 adult children who live with her. After calculating each person's share, Mrs. Tan wants to compare this amount with her previous month's share per person of \textdollar27.850. Is the current month's share per person greater than, less than, or equal to the previous month's share?

\section*{Question 19}
\textbf{Metadata}

\begin{itemize}
  \item Question ID: P4-DcDiv2d1d\_P4-DcAdd2nd\_sonnet4\_Household Finance\_01
  \item Primary KC: DECIMALS | Division | dividing decimals (up to 2 decimal places) by a 1-digit whole number
  \item Secondary KC: DECIMALS | Addition | adding decimals
  \item Topic: Household finance such as income, utility bills, money, interest, savings, instalment, mortgage, financial planning etc.
  \item Grade: Primary 4
\end{itemize}

\textbf{Question}

Mrs. Lee received her monthly utility bills. Her electricity bill was \textdollar68.45, her water bill was \textdollar32.80, and her gas bill was \textdollar41.75. She wants to pay the total amount in 3 equal instalments over the next 3 months. How much will each instalment be?

\section*{Question 20}
\textbf{Metadata}

\begin{itemize}
  \item Question ID: P5-FrAddMix\_P2-FrCmp\_sonnet4\_Household Finance\_01
  \item Primary KC: FRACTIONS | Addition | adding mixed numbers
  \item Secondary KC: FRACTIONS | Comparison and ordering | comparing and ordering fractions
  \item Topic: Household finance such as income, utility bills, money, interest, savings, instalment, mortgage, financial planning etc.
  \item Grade: Primary 5
\end{itemize}

\textbf{Question}

The Tan family is planning their monthly budget. Mrs. Tan earns \textdollar2400 from her part-time job, which is $2\frac{1}{4}$ times her monthly savings goal. Mr. Tan's monthly income is $1\frac{2}{3}$ times Mrs. Tan's income. Their monthly expenses include utilities costing \textdollar180, groceries costing \textdollar520, and transport expenses of \textdollar150. After paying all expenses, they want to save the remaining money. Will their combined monthly income be enough to cover their expenses and still meet Mrs. Tan's savings goal? Find their total monthly income and determine if it exceeds the sum of their expenses and savings goal.

\section*{Question 21}
\textbf{Metadata}

\begin{itemize}
  \item Question ID: P5-FrSubMix\_P2-FrCmp\_sonnet4\_Household Finance\_01
  \item Primary KC: FRACTIONS | Subtraction | subtracting mixed numbers
  \item Secondary KC: FRACTIONS | Comparison and ordering | comparing and ordering fractions
  \item Topic: Household finance such as income, utility bills, money, interest, savings, instalment, mortgage, financial planning etc.
  \item Grade: Primary 5
\end{itemize}

\textbf{Question}

The Lim family has a monthly budget of \textdollar4200. In January, they spent $2\frac{3}{4}$ thousand dollars on household expenses and $1\frac{5}{6}$ thousand dollars on other necessities. How much money do they have left from their budget? Is the remaining amount more than or less than $\frac{1}{2}$ thousand dollars?

\section*{Question 22}
\textbf{Metadata}

\begin{itemize}
  \item Question ID: P5-FrMulImN\_P2-FrCmp\_sonnet4\_Household Finance\_01
  \item Primary KC: FRACTIONS | Multiplication | multiplying a proper/improper fraction and a whole number
  \item Secondary KC: FRACTIONS | Comparison and ordering | comparing and ordering fractions
  \item Topic: Household finance such as income, utility bills, money, interest, savings, instalment, mortgage, financial planning etc.
  \item Grade: Primary 5
\end{itemize}

\textbf{Question}

The Tan family has a monthly household budget of \textdollar3600. They spend $\frac{2}{9}$ of their budget on groceries and $\frac{1}{6}$ of their budget on utilities. 

(a) How much money does the family spend on groceries each month?

(b) How much money does the family spend on utilities each month?

(c) On which expense does the family spend more money, groceries or utilities? How much more?

\section*{Question 23}
\textbf{Metadata}

\begin{itemize}
  \item Question ID: P5-FrMulPIm\_P2-FrCmp\_sonnet4\_Household Finance\_01
  \item Primary KC: FRACTIONS | Multiplication | multiplying a proper fraction and a proper/improper fractions
  \item Secondary KC: FRACTIONS | Comparison and ordering | comparing and ordering fractions
  \item Topic: Household finance such as income, utility bills, money, interest, savings, instalment, mortgage, financial planning etc.
  \item Grade: Primary 5
\end{itemize}

\textbf{Question}

Mrs. Lim is planning her monthly household budget. She allocates $\frac{2}{5}$ of her monthly income for household expenses and $\frac{3}{7}$ of her monthly income for savings. During the first week, she spent $\frac{3}{4}$ of her allocated household expense money on groceries and utilities. In the same week, she managed to save $\frac{5}{6}$ of her allocated savings money. Which amount is greater - the money she spent on groceries and utilities, or the money she saved in the first week? Express both amounts as fractions of her monthly income.

\section*{Question 24}
\textbf{Metadata}

\begin{itemize}
  \item Question ID: P5-FrMulPIm\_P2-FrAdd2nd\_sonnet4\_Household Finance\_01
  \item Primary KC: FRACTIONS | Multiplication | multiplying a proper fraction and a proper/improper fractions
  \item Secondary KC: FRACTIONS | Addition | adding fractions
  \item Topic: Household finance such as income, utility bills, money, interest, savings, instalment, mortgage, financial planning etc.
  \item Grade: Primary 5
\end{itemize}

\textbf{Question}

Mrs. Tan earns \textdollar3600 each month. She saves $\frac{1}{4}$ of her monthly income and spends $\frac{2}{5}$ of her monthly income on household expenses. The remaining amount is used for other expenses. How much money does Mrs. Tan use for other expenses each month?

\section*{Question 25}
\textbf{Metadata}

\begin{itemize}
  \item Question ID: P5-FrMulPIm\_P3-FrSmp\_sonnet4\_Household Finance\_01
  \item Primary KC: FRACTIONS | Multiplication | multiplying a proper fraction and a proper/improper fractions
  \item Secondary KC: FRACTIONS | Simplifying | expressing a fraction in its simplest form
  \item Topic: Household finance such as income, utility bills, money, interest, savings, instalment, mortgage, financial planning etc.
  \item Grade: Primary 5
\end{itemize}

\textbf{Question}

Mrs. Tan saves $\frac{2}{5}$ of her monthly salary for her family's expenses. She spends $\frac{3}{4}$ of her savings on utility bills. If Mrs. Tan's monthly salary is \textdollar3000, how much money does she spend on utility bills? Express your answer as a fraction in its simplest form, then convert it to dollars.

\section*{Question 26}
\textbf{Metadata}

\begin{itemize}
  \item Question ID: P5-FrMulImIm\_P2-FrAdd2nd\_sonnet4\_Household Finance\_01
  \item Primary KC: FRACTIONS | Multiplication | multiplying two improper fractions
  \item Secondary KC: FRACTIONS | Addition | adding fractions
  \item Topic: Household finance such as income, utility bills, money, interest, savings, instalment, mortgage, financial planning etc.
  \item Grade: Primary 5
\end{itemize}

\textbf{Question}

Mrs. Tan receives a monthly salary of \textdollar2400. She spends $\frac{5}{4}$ of her monthly savings on household bills and $\frac{7}{3}$ of her monthly savings on groceries. If her monthly savings is $\frac{8}{15}$ of her salary, how much money does she spend on household bills and groceries altogether each month?

\section*{Question 27}
\textbf{Metadata}

\begin{itemize}
  \item Question ID: P5-FrMulImIm\_P2-FrSub2nd\_sonnet4\_Household Finance\_01
  \item Primary KC: FRACTIONS | Multiplication | multiplying two improper fractions
  \item Secondary KC: FRACTIONS | Subtraction | subtracting fractions
  \item Topic: Household finance such as income, utility bills, money, interest, savings, instalment, mortgage, financial planning etc.
  \item Grade: Primary 5
\end{itemize}

\textbf{Question}

The Tan family spends $\frac{7}{4}$ of their monthly budget on household expenses. Of this amount, $\frac{8}{5}$ is used for utilities and groceries. However, they managed to save $\frac{1}{6}$ of the money allocated for utilities and groceries by using energy-saving methods and buying items on sale. How much of their monthly budget did they actually spend on utilities and groceries after the savings?

\section*{Question 28}
\textbf{Metadata}

\begin{itemize}
  \item Question ID: P5-FrMulImIm\_P3-FrSmp\_sonnet4\_Household Finance\_01
  \item Primary KC: FRACTIONS | Multiplication | multiplying two improper fractions
  \item Secondary KC: FRACTIONS | Simplifying | expressing a fraction in its simplest form
  \item Topic: Household finance such as income, utility bills, money, interest, savings, instalment, mortgage, financial planning etc.
  \item Grade: Primary 5
\end{itemize}

\textbf{Question}

Sarah's family saves \textdollar$\frac{13}{4}$ every day for their utility bills. They decided to increase their daily savings to $\frac{7}{3}$ times the original amount to prepare for higher electricity costs next month. How much money will Sarah's family save per day after the increase? Express your answer as a fraction in its simplest form.

\section*{Question 29}
\textbf{Metadata}

\begin{itemize}
  \item Question ID: P5-FrMulMixN\_P2-FrSub2nd\_sonnet4\_Household Finance\_01
  \item Primary KC: FRACTIONS | Multiplication | multiplying a mixed number and a whole number
  \item Secondary KC: FRACTIONS | Subtraction | subtracting fractions
  \item Topic: Household finance such as income, utility bills, money, interest, savings, instalment, mortgage, financial planning etc.
  \item Grade: Primary 5
\end{itemize}

\textbf{Question}

Mrs. Tan's monthly electricity bill is usually \textdollar72. This month, she used $2\frac{1}{4}$ times her usual amount of electricity due to the hot weather. However, she received a government rebate that reduced her bill by $\frac{3}{8}$ of the increased amount. What is her final electricity bill for this month?

\section*{Question 30}
\textbf{Metadata}

\begin{itemize}
  \item Question ID: P5-DcMul3dK\_P4-DcCnv2Fr\_sonnet4\_Household Finance\_01
  \item Primary KC: DECIMALS | Multiplication | multiplying decimals (up to 3 decimal places) by 10, 100, 1000 and their multiples
  \item Secondary KC: DECIMALS | Conversion from decimals to fraction | expressing decimals as fractions
  \item Topic: Household finance such as income, utility bills, money, interest, savings, instalment, mortgage, financial planning etc.
  \item Grade: Primary 5
\end{itemize}

\textbf{Question}

Mrs. Tan is planning her monthly budget. She earns a monthly salary of \textdollar2400. She spends $0.125$ of her salary on groceries and $0.08$ of her salary on utilities. To better track her expenses, she wants to convert these decimal portions to fractions and calculate the exact dollar amounts she spends on each category.

(a) Express $0.125$ and $0.08$ as fractions in their simplest form.

(b) Calculate how much money Mrs. Tan spends on groceries each month.

(c) Calculate how much money Mrs. Tan spends on utilities each month.

(d) What is the total amount she spends on groceries and utilities combined?

\section*{Question 31}
\textbf{Metadata}

\begin{itemize}
  \item Question ID: P5-DcMul3dK\_P4-DcSub2nd\_sonnet4\_Household Finance\_01
  \item Primary KC: DECIMALS | Multiplication | multiplying decimals (up to 3 decimal places) by 10, 100, 1000 and their multiples
  \item Secondary KC: DECIMALS | Subtraction | subtracting decimals
  \item Topic: Household finance such as income, utility bills, money, interest, savings, instalment, mortgage, financial planning etc.
  \item Grade: Primary 5
\end{itemize}

\textbf{Question}

Mrs. Tan is planning her monthly household budget. She needs to calculate her total monthly expenses and determine how much she can save. Her electricity bill is \textdollar0.125 per unit, and she uses 800 units per month. Her water bill is \textdollar0.078 per unit, and she uses 1200 units per month. Her total monthly income is \textdollar4500. How much money can Mrs. Tan save each month after paying these utility bills?

\section*{Question 32}
\textbf{Metadata}

\begin{itemize}
  \item Question ID: P5-DcDiv3dK\_P4-DcRnd3d\_sonnet4\_Household Finance\_01
  \item Primary KC: DECIMALS | Division | dividing decimals (up to 3 decimal places) by 10, 100, 1000 and their multiples
  \item Secondary KC: DECIMALS | Rounding | rounding decimals up to 3 decimal places to the nearest whole number, 1 decimal place and 2 decimal places 
  \item Topic: Household finance such as income, utility bills, money, interest, savings, instalment, mortgage, financial planning etc.
  \item Grade: Primary 5
\end{itemize}

\textbf{Question}

Mrs. Tan received her monthly electricity bill of \textdollar348.750. She decided to split this amount equally among her 3 children as part of their financial responsibility education. Each child will pay their share over 20 weeks by making equal weekly payments. Calculate the weekly payment amount each child needs to make, rounded to the nearest cent.

\section*{Question 33}
\textbf{Metadata}

\begin{itemize}
  \item Question ID: P5-DcDiv3dK\_P4-DcAdd2nd\_sonnet4\_Household Finance\_01
  \item Primary KC: DECIMALS | Division | dividing decimals (up to 3 decimal places) by 10, 100, 1000 and their multiples
  \item Secondary KC: DECIMALS | Addition | adding decimals
  \item Topic: Household finance such as income, utility bills, money, interest, savings, instalment, mortgage, financial planning etc.
  \item Grade: Primary 5
\end{itemize}

\textbf{Question}

Mrs. Tan is calculating her family's monthly utility expenses. She receives three utility bills: electricity bill for \textdollar186.400, water bill for \textdollar92.350, and gas bill for \textdollar74.280. To better manage her budget, she wants to find the average daily cost for each utility by dividing the monthly amounts by 30 days. After finding the daily costs, she plans to calculate the total average daily utility expense for her family. What is the total average daily utility expense?

\section*{Question 34}
\textbf{Metadata}

\begin{itemize}
  \item Question ID: P5-PcRepWh\_P1-WNMul2nd\_sonnet4\_Household Finance\_01
  \item Primary KC: PERCENTAGE | Representation and concept | expressing a part of a whole as a percentage
  \item Secondary KC: WHOLE NUMBERS | Multiplication | multiplying whole numbers
  \item Topic: Household finance such as income, utility bills, money, interest, savings, instalment, mortgage, financial planning etc.
  \item Grade: Primary 5
\end{itemize}

\textbf{Question}

Mrs. Lim's monthly household income is \textdollar4800. She spends 25\% of her income on groceries and utilities. If groceries cost 3 times as much as utilities, how much does she spend on utilities each month?

\section*{Question 35}
\textbf{Metadata}

\begin{itemize}
  \item Question ID: P5-PcRepWh\_P1-WNDiv2nd\_sonnet4\_Household Finance\_01
  \item Primary KC: PERCENTAGE | Representation and concept | expressing a part of a whole as a percentage
  \item Secondary KC: WHOLE NUMBERS | Division | dividing whole numbers
  \item Topic: Household finance such as income, utility bills, money, interest, savings, instalment, mortgage, financial planning etc.
  \item Grade: Primary 5
\end{itemize}

\textbf{Question}

Mrs. Tan's monthly household income is \textdollar4800. She spends \textdollar1200 on groceries and utilities each month. What percentage of her monthly income does Mrs. Tan spend on groceries and utilities?

\section*{Question 36}
\textbf{Metadata}

\begin{itemize}
  \item Question ID: P5-RtFndU\_P2-DcCnvN2D\_sonnet4\_Household Finance\_01
  \item Primary KC: RATE | Finding number of unit | finding number of units given rate and total amount
  \item Secondary KC: DECIMALS | Conversion to larger units | converting a measurement from a smaller unit to a larger unit in decimal form
  \item Topic: Household finance such as income, utility bills, money, interest, savings, instalment, mortgage, financial planning etc.
  \item Grade: Primary 5
\end{itemize}

\textbf{Question}

Mrs. Tan pays \textdollar0.28 per kilowatt-hour (kWh) for electricity. Last month, her electricity bill was \textdollar67.20. She wants to know how many kilowatt-hours of electricity she used. After finding this, help her convert the total electricity usage from kilowatt-hours to megawatt-hours. (Note: 1 megawatt-hour = 1000 kilowatt-hours)

\section*{Question 37}
\textbf{Metadata}

\begin{itemize}
  \item Question ID: P6-FrDivPN\_P2-FrAdd2nd\_sonnet4\_Household Finance\_01
  \item Primary KC: FRACTIONS | Division | dividing a proper fraction by a whole number
  \item Secondary KC: FRACTIONS | Addition | adding fractions
  \item Topic: Household finance such as income, utility bills, money, interest, savings, instalment, mortgage, financial planning etc.
  \item Grade: Primary 6
\end{itemize}

\textbf{Question}

Mrs. Tan receives \textdollar900 as her monthly household allowance. She spends $\frac{2}{3}$ of this allowance on groceries and utilities. The remaining money is divided equally among her 4 children as their weekly pocket money for the month. How much pocket money does each child receive per week? Express your answer as a fraction in its simplest form.

\section*{Question 38}
\textbf{Metadata}

\begin{itemize}
  \item Question ID: P6-FrDivPN\_P5-FrCnv2Dc\_sonnet4\_Household Finance\_01
  \item Primary KC: FRACTIONS | Division | dividing a proper fraction by a whole number
  \item Secondary KC: FRACTIONS | Conversion to decimals | expressing fractions as decimals
  \item Topic: Household finance such as income, utility bills, money, interest, savings, instalment, mortgage, financial planning etc.
  \item Grade: Primary 6
\end{itemize}

\textbf{Question}

Mrs. Tan's monthly electricity bill is \textdollar90. She decides to reduce her electricity consumption by $\frac{2}{5}$ of the original amount over the next 6 months by using energy-saving appliances. If the reduction is spread equally over the 6 months, what is the monthly reduction in her electricity bill? Express your answer as a decimal.

\section*{Question 39}
\textbf{Metadata}

\begin{itemize}
  \item Question ID: P6-FrDivPP\_P2-FrSub2nd\_sonnet4\_Household Finance\_01
  \item Primary KC: FRACTIONS | Division | dividing a whole number/proper fraction by a proper fraction
  \item Secondary KC: FRACTIONS | Subtraction | subtracting fractions
  \item Topic: Household finance such as income, utility bills, money, interest, savings, instalment, mortgage, financial planning etc.
  \item Grade: Primary 6
\end{itemize}

\textbf{Question}

The Tan family spends $\frac{3}{8}$ of their monthly income on household expenses. After paying for utilities, they have $\frac{1}{4}$ of their monthly income left for savings and other expenses. If the amount left for savings and other expenses is \textdollar900, what was the Tan family's total monthly income?

\section*{Question 40}
\textbf{Metadata}

\begin{itemize}
  \item Question ID: P6-FrDivPP\_P5-FrCnv2Dc\_sonnet4\_Household Finance\_01
  \item Primary KC: FRACTIONS | Division | dividing a whole number/proper fraction by a proper fraction
  \item Secondary KC: FRACTIONS | Conversion to decimals | expressing fractions as decimals
  \item Topic: Household finance such as income, utility bills, money, interest, savings, instalment, mortgage, financial planning etc.
  \item Grade: Primary 6
\end{itemize}

\textbf{Question}

Mrs. Tan has \textdollar240 in her savings account. She decides to withdraw $\frac{3}{4}$ of her savings to pay for household expenses. From the amount she withdrew, she uses $\frac{2}{5}$ of it to pay the electricity bill. How much money did she use to pay the electricity bill? Express your answer as a decimal.

\section*{Question 41}
\textbf{Metadata}

\begin{itemize}
  \item Question ID: P6-PcFndWN\_P1-WNAdd2nd\_sonnet4\_Household Finance\_01
  \item Primary KC: PERCENTAGE | Finding the whole | finding the whole given a part and the percentage
  \item Secondary KC: WHOLE NUMBERS | Addition | adding whole numbers
  \item Topic: Household finance such as income, utility bills, money, interest, savings, instalment, mortgage, financial planning etc.
  \item Grade: Primary 6
\end{itemize}

\textbf{Question}

Mrs. Tan is planning her monthly household budget. She knows that her electricity bill of \textdollar180 represents 15% of her total monthly utility expenses. Her water bill is \textdollar45 more than her electricity bill. What is the total amount Mrs. Tan spends on electricity and water bills combined?

\section*{Question 42}
\textbf{Metadata}

\begin{itemize}
  \item Question ID: P6-PcFndChg\_P1-WNDiv2nd\_sonnet4\_Household Finance\_01
  \item Primary KC: PERCENTAGE | Finding change | finding percentage increase/decrease
  \item Secondary KC: WHOLE NUMBERS | Division | dividing whole numbers
  \item Topic: Household finance such as income, utility bills, money, interest, savings, instalment, mortgage, financial planning etc.
  \item Grade: Primary 6
\end{itemize}

\textbf{Question}

The Tan family's monthly electricity bill was \textdollar240 in January. In February, their bill increased to \textdollar288. Mrs. Tan wants to understand how much their electricity usage has changed. She also noticed that if they divide their February bill equally among the 4 weeks of the month, they would need to budget a certain amount per week. Find the percentage increase in their electricity bill from January to February, and calculate how much they need to budget per week in February.

\section*{Question 43}
\textbf{Metadata}

\begin{itemize}
  \item Question ID: P6-RoFndRoWN\_P1-WNAdd2nd\_sonnet4\_Household Finance\_01
  \item Primary KC: RATIO | Finding ratio | finding the ratio of two or three given whole numbers
  \item Secondary KC: WHOLE NUMBERS | Addition | adding whole numbers
  \item Topic: Household finance such as income, utility bills, money, interest, savings, instalment, mortgage, financial planning etc.
  \item Grade: Primary 6
\end{itemize}

\textbf{Question}

The Tan family tracks their monthly household expenses. In January, they spent \textdollar450 on groceries, \textdollar300 on utilities, and \textdollar150 on transportation. In February, their grocery expenses increased by \textdollar50, utility expenses increased by \textdollar25, and transportation expenses increased by \textdollar25. Find the ratio of their grocery expenses to utility expenses to transportation expenses in February.

\section*{Question 44}
\textbf{Metadata}

\begin{itemize}
  \item Question ID: P6-AgRepLrEx\_P6-AgSmpLrEx\_sonnet4\_Household Finance\_01
  \item Primary KC: ALGEBRA | Representation and concept | translation of real-world situations into linear algebraic expressions
  \item Secondary KC: ALGEBRA | Simplifying | simplifying linear expressions
  \item Topic: Household finance such as income, utility bills, money, interest, savings, instalment, mortgage, financial planning etc.
  \item Grade: Primary 6
\end{itemize}

\textbf{Question}

Mrs. Tan is planning her family's monthly budget. She earns a fixed salary of \textdollar3200 per month. Her monthly expenses include utility bills that cost \textdollar150 more than her phone bill, and her phone bill costs \textdollar80. She also spends \textdollar600 on groceries and \textdollar400 on transport. Additionally, she saves \textdollar200 more than twice her phone bill amount each month. Write an algebraic expression for her total monthly expenses (utility bills, phone bill, groceries, and transport). Then write an algebraic expression for her total monthly savings after all expenses. Simplify both expressions.

\section*{Question 45}
\textbf{Metadata}

\begin{itemize}
  \item Question ID: O1-PcFndRslt\_P1-WNSub2nd\_sonnet4\_Household Finance\_01
  \item Primary KC: PERCENTAGE | Finding result after change | increasing/decreasing a quantity by a given percentage
  \item Secondary KC: WHOLE NUMBERS | Subtraction | subtracting whole numbers
  \item Topic: Household finance such as income, utility bills, money, interest, savings, instalment, mortgage, financial planning etc.
  \item Grade: Secondary O-level 1
\end{itemize}

\textbf{Question}

The Tan family's monthly electricity bill was \textdollar180 in January. Due to energy conservation efforts, their February bill decreased by 15\%. However, they forgot to turn off their air conditioning for a few days in March, causing their bill to increase by 20\% compared to February. If the Tan family had budgeted \textdollar200 for their March electricity bill, how much money did they have left over or go over their budget?

\section*{Question 46}
\textbf{Metadata}

\begin{itemize}
  \item Question ID: O1-PcRepRvs\_O1-PcCnv2Dc\_sonnet4\_Household Finance\_01
  \item Primary KC: PERCENTAGE | Representation and concept | reverse percentages
  \item Secondary KC: PERCENTAGE | Conversion to decimals | expressing percentage as a decimal
  \item Topic: Household finance such as income, utility bills, money, interest, savings, instalment, mortgage, financial planning etc.
  \item Grade: Secondary O-level 1
\end{itemize}

\textbf{Question}

Sarah's monthly salary was increased by $15\%$ at the beginning of the year. Her new monthly salary is now \textdollar4600. She plans to save $20\%$ of her original salary each month for a family vacation. How much money will Sarah save each month for her vacation?

\section*{Question 47}
\textbf{Metadata}

\begin{itemize}
  \item Question ID: O2-RoRepDP\_P1-WNMul2nd\_sonnet4\_Household Finance\_01
  \item Primary KC: RATIO | Representation and concept | direct proportion
  \item Secondary KC: WHOLE NUMBERS | Multiplication | multiplying whole numbers
  \item Topic: Household finance such as income, utility bills, money, interest, savings, instalment, mortgage, financial planning etc.
  \item Grade: Secondary O-level 2
\end{itemize}

\textbf{Question}

The Chen family's monthly household expenses are divided into three categories in the ratio $3:2:1$ for groceries, utilities, and entertainment respectively. If they spend \textdollar240 on groceries each month, calculate:

(a) How much do they spend on utilities and entertainment each month?

(b) If the family decides to increase their total monthly expenses by 4 times to accommodate a larger household, what will be their new monthly spending on each category?

\section*{Question 48}
\textbf{Metadata}

\begin{itemize}
  \item Question ID: O3-MXMul\_O3-MXAdd\_sonnet4\_Household Finance\_01
  \item Primary KC: MATRICES | Multiplication | multiplication of matrices
  \item Secondary KC: MATRICES | Addition | addition of matrices
  \item Topic: Household finance such as income, utility bills, money, interest, savings, instalment, mortgage, financial planning etc.
  \item Grade: Secondary O-level 3/4
\end{itemize}

\textbf{Question}

The Tan family tracks their monthly expenses using matrices. They have expenses in three categories: utilities, groceries, and entertainment. Their expenses for the first quarter of the year are recorded in matrix $A$, where each row represents a month (January, February, March) and each column represents a category (utilities, groceries, entertainment):

$A = \begin{pmatrix} 150 & 400 & 200 \\ 180 & 450 & 180 \\ 160 & 420 & 220 \end{pmatrix}$

Due to inflation, all their expenses increased in the second quarter. The percentage increases are represented in matrix $B$:

$B = \begin{pmatrix} 0.05 & 0.08 & 0.10 \\ 0.06 & 0.07 & 0.12 \\ 0.04 & 0.09 & 0.08 \end{pmatrix}$

The conversion factors to calculate the actual increase amounts are given in matrix $C$:

$C = \begin{pmatrix} 1 \\ 1 \\ 1 \end{pmatrix}$

First, calculate matrix $AB$ to find the total increase amounts. Then, add this result to the original expenses matrix $A$ to find the family's expenses for the second quarter.

\section*{Question 49}
\textbf{Metadata}

\begin{itemize}
  \item Question ID: O3-SPFndstd\_O2-SPFndmean\_sonnet4\_Household Finance\_01
  \item Primary KC: STATISTICS AND PROBABILITY | Finding standard deviation | calculation of the standard deviation for a set of data
  \item Secondary KC: STATISTICS AND PROBABILITY | Finding mean deviation | calculation of the mean for a set of data
  \item Topic: Household finance such as income, utility bills, money, interest, savings, instalment, mortgage, financial planning etc.
  \item Grade: Secondary O-level 3/4
\end{itemize}

\textbf{Question}

The Tan family tracks their monthly electricity bills over the past 6 months to better plan their household budget. Their monthly electricity bills (in dollars) were: \textdollar120, \textdollar135, \textdollar110, \textdollar145, \textdollar125, and \textdollar115. To understand the variability in their electricity expenses, they want to calculate both the mean and standard deviation of their monthly bills. Find the mean monthly electricity bill and the standard deviation of their electricity bills.

\end{document}
