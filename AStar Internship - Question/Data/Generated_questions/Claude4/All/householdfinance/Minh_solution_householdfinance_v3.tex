\documentclass{article}
\usepackage[utf8]{inputenc}
\usepackage{amsmath}
\usepackage{amsfonts}
\usepackage{amssymb}
\usepackage{graphicx}
\usepackage{hyperref}
\title{'Minh Solutions household finance v3 CLAUDE '}
\author{Tien Dung Doan}
\begin{document}
\maketitle
\section*{Question 1}
\textbf{Metadata}

\begin{itemize}
  \item Question ID: P3-WNSub4d\_P1-WNCmp\_sonnet4\_Household Finance\_01
  \item Primary KC: WHOLE NUMBERS | Subtraction | subtracting whole numbers up to 4 digits
  \item Secondary KC: WHOLE NUMBERS | Comparison and ordering | comparing and ordering whole numbers
  \item Topic: Household finance such as income, utility bills, money, interest, savings, instalment, mortgage, financial planning etc.
  \item Grade: Primary 3
\end{itemize}

\textbf{Solution}

Step 1: Calculate the family's total monthly income.
Total income = Mr. Tan's income + Mrs. Tan's income
Total income = \textdollar3250 + \textdollar2890 = \textdollar6140

Step 2: Calculate their total monthly expenses.
Total expenses = Rent + Groceries + Utility bills
Total expenses = \textdollar1450 + \textdollar680 + \textdollar285 = \textdollar2415

Step 3: Calculate how much money they have left for savings.
Savings = Total income - Total expenses
Savings = \textdollar6140 - \textdollar2415 = \textdollar3725

Step 4: Compare the savings amount with the rent expense.
Savings = \textdollar3725
Rent = \textdollar1450

Since $3725 > 1450$, the savings amount is greater than the rent expense.

Step 5: Find the difference between savings and rent.
Difference = \textdollar3725 - \textdollar1450 = \textdollar2275

Therefore, the Tan family has \textdollar3725 left for savings each month. The savings amount is greater than their rent expense by \textdollar2275.

\section*{Question 2}
\textbf{Metadata}

\begin{itemize}
  \item Question ID: P3-WNDivRmd3d\_P1-WNAdd2nd\_sonnet4\_Household Finance\_01
  \item Primary KC: WHOLE NUMBERS | Division | dividing whole numbers up to 3 digits by 1 digit with remainder 
  \item Secondary KC: WHOLE NUMBERS | Addition | adding whole numbers
  \item Topic: Household finance such as income, utility bills, money, interest, savings, instalment, mortgage, financial planning etc.
  \item Grade: Primary 3
\end{itemize}

\textbf{Solution}

Step 1: Find how much Mrs. Tan can save equally each week and the remainder.

Divide \textdollar684 by 4 weeks:
$684 \div 4 = 171$ remainder $0$

Let me recalculate: $4 \times 171 = 684$

Since $684 \div 4 = 171$ with no remainder, let me use a different amount.

Let's say Mrs. Tan receives \textdollar687 as her monthly salary.

$687 \div 4 = 171$ remainder $3$

Check: $4 \times 171 = 684$ and $687 - 684 = 3$ ✓

So Mrs. Tan can save \textdollar171 each week with \textdollar3 left over.

Step 2: Find how much she will save in the first week.

Amount saved in first week = Equal weekly amount + Leftover money
$= 171 + 3 = 174$

Therefore, Mrs. Tan will save \textdollar174 in the first week.

\section*{Question 3}
\textbf{Metadata}

\begin{itemize}
  \item Question ID: P3-WNDivRmd3d\_P1-WNMul2nd\_sonnet4\_Household Finance\_01
  \item Primary KC: WHOLE NUMBERS | Division | dividing whole numbers up to 3 digits by 1 digit with remainder 
  \item Secondary KC: WHOLE NUMBERS | Multiplication | multiplying whole numbers
  \item Topic: Household finance such as income, utility bills, money, interest, savings, instalment, mortgage, financial planning etc.
  \item Grade: Primary 3
\end{itemize}

\textbf{Solution}

To find how much Mrs. Tan can save each week, I need to divide her monthly earnings by the number of weeks in a month.

Mrs. Tan's monthly earnings = \textdollar756
Number of weeks in a month = 4

Dividing: $756 \div 4$

$756 \div 4 = 189$ remainder $0$

Let me check: $189 \times 4 = 756$

Therefore, Mrs. Tan can save \textdollar189 each week, and she will have \textdollar0 left over.

\section*{Question 4}
\textbf{Metadata}

\begin{itemize}
  \item Question ID: P3-WNDiv3d1d\_P1-WNAdd2nd\_sonnet4\_Household Finance\_01
  \item Primary KC: WHOLE NUMBERS | Division | dividing whole numbers up to 3 digits by 1 digit
  \item Secondary KC: WHOLE NUMBERS | Addition | adding whole numbers
  \item Topic: Household finance such as income, utility bills, money, interest, savings, instalment, mortgage, financial planning etc.
  \item Grade: Primary 3
\end{itemize}

\textbf{Solution}

Step 1: Find how much Mrs. Tan needs to save from her regular income each month.

Amount to save from regular income each month = \textdollar648 ÷ 8 = \textdollar81

Step 2: Find the total amount Mrs. Tan will save each month.

Total monthly savings = Amount from regular income + Amount from recycling
Total monthly savings = \textdollar81 + \textdollar25 = \textdollar106

Therefore, Mrs. Tan will save \textdollar106 each month in total for her vacation.

\section*{Question 5}
\textbf{Metadata}

\begin{itemize}
  \item Question ID: P3-WNDiv3d1d\_P1-WNSub2nd\_sonnet4\_Household Finance\_01
  \item Primary KC: WHOLE NUMBERS | Division | dividing whole numbers up to 3 digits by 1 digit
  \item Secondary KC: WHOLE NUMBERS | Subtraction | subtracting whole numbers
  \item Topic: Household finance such as income, utility bills, money, interest, savings, instalment, mortgage, financial planning etc.
  \item Grade: Primary 3
\end{itemize}

\textbf{Solution}

Step 1: Find how much Mrs. Tan needs to pay each month.
Monthly payment = \textdollar348 $\div$ 3 = \textdollar116

Step 2: Find how much money she will have left after the first payment.
Money left = \textdollar500 $-$ \textdollar116 = \textdollar384

Therefore, Mrs. Tan will have \textdollar384 left in her savings account after making the first payment.

\section*{Question 6}
\textbf{Metadata}

\begin{itemize}
  \item Question ID: P3-WNDiv3d1d\_P1-WNMul2nd\_sonnet4\_Household Finance\_01
  \item Primary KC: WHOLE NUMBERS | Division | dividing whole numbers up to 3 digits by 1 digit
  \item Secondary KC: WHOLE NUMBERS | Multiplication | multiplying whole numbers
  \item Topic: Household finance such as income, utility bills, money, interest, savings, instalment, mortgage, financial planning etc.
  \item Grade: Primary 3
\end{itemize}

\textbf{Solution}

Step 1: Find how much Mrs. Tan pays per week for electricity.
Monthly electricity bill = \textdollar348
Number of weeks in a month = 4
Weekly electricity cost = \textdollar348 ÷ 4 = \textdollar87

Step 2: Find how much money she will save in 6 weeks.
Savings per week = \textdollar12
Number of weeks = 6
Total savings in 6 weeks = \textdollar12 × 6 = \textdollar72

Therefore, Mrs. Tan pays \textdollar87 per week for electricity and will save \textdollar72 in 6 weeks by using energy-saving appliances.

\section*{Question 7}
\textbf{Metadata}

\begin{itemize}
  \item Question ID: P3-FrAddRl12\_P2-FrCmp\_sonnet4\_Household Finance\_01
  \item Primary KC: FRACTIONS | Addition | adding two related fractions within one whole with denominators of given fractions not exceeding 12
  \item Secondary KC: FRACTIONS | Comparison and ordering | comparing and ordering fractions
  \item Topic: Household finance such as income, utility bills, money, interest, savings, instalment, mortgage, financial planning etc.
  \item Grade: Primary 3
\end{itemize}

\textbf{Solution}

(a) To find the total fraction spent on groceries and utility bills, I need to add $\frac{2}{5}$ and $\frac{1}{10}$.

First, I need to make the denominators the same.
$\frac{2}{5} = \frac{2 \times 2}{5 \times 2} = \frac{4}{10}$

Now I can add: $\frac{4}{10} + \frac{1}{10} = \frac{5}{10} = \frac{1}{2}$

Mrs. Tan spends $\frac{1}{2}$ of her monthly income on groceries and utility bills altogether.

(b) To compare $\frac{2}{5}$ and $\frac{1}{10}$, I need to make the denominators the same.

From part (a), I know that $\frac{2}{5} = \frac{4}{10}$

Comparing $\frac{4}{10}$ and $\frac{1}{10}$:
Since $4 > 1$, we have $\frac{4}{10} > \frac{1}{10}$

Therefore, $\frac{2}{5} > \frac{1}{10}$

Mrs. Tan spends more on groceries than on utility bills.

\section*{Question 8}
\textbf{Metadata}

\begin{itemize}
  \item Question ID: P3-FrSubRl12\_P2-FrAdd2nd\_sonnet4\_Household Finance\_01
  \item Primary KC: FRACTIONS | Subtraction | subtracting two related fractions within one whole with denominators of given fractions not exceeding 12
  \item Secondary KC: FRACTIONS | Addition | adding fractions
  \item Topic: Household finance such as income, utility bills, money, interest, savings, instalment, mortgage, financial planning etc.
  \item Grade: Primary 3
\end{itemize}

\textbf{Solution}

First, I need to find the total fraction spent on groceries and utility bills.

Fraction spent on groceries = $\frac{1}{4}$
Fraction spent on utility bills = $\frac{1}{8}$

To add these fractions, I need a common denominator. The LCD of 4 and 8 is 8.
$\frac{1}{4} = \frac{2}{8}$

Total fraction spent on groceries and utility bills:
$\frac{2}{8} + \frac{1}{8} = \frac{3}{8}$

Now I need to find what fraction is left from the household expenses.

Total household expenses = $\frac{5}{8}$ of monthly income
Amount spent on groceries and utility bills = $\frac{3}{8}$ of monthly income

Fraction left from household expenses:
$\frac{5}{8} - \frac{3}{8} = \frac{2}{8} = \frac{1}{4}$

Therefore, $\frac{1}{4}$ of their monthly income is left from the household expenses after paying for groceries and utility bills.

\section*{Question 9}
\textbf{Metadata}

\begin{itemize}
  \item Question ID: P4-WNMul4d1d\_P1-WNAdd2nd\_sonnet4\_Household Finance\_01
  \item Primary KC: WHOLE NUMBERS | Multiplication | multiplying whole numbers up to 4 digits by 1 digit or up to 3 digits by 2 digits
  \item Secondary KC: WHOLE NUMBERS | Addition | adding whole numbers
  \item Topic: Household finance such as income, utility bills, money, interest, savings, instalment, mortgage, financial planning etc.
  \item Grade: Primary 4
\end{itemize}

\textbf{Solution}

Step 1: Calculate Mrs. Lim's monthly income.
Mrs. Lim earns \textdollar28 per day and works 18 days.
Mrs. Lim's monthly income $= 28 \times 18 = 504$
So Mrs. Lim earns \textdollar504 each month.

Step 2: Calculate the family's total monthly income.
Total monthly income $= 3250 + 504 = 3754$
The family's total monthly income is \textdollar3,754.

Step 3: Calculate the family's total monthly expenses.
Total monthly expenses $= 1800 + 145 + 680 + 320 = 2945$
The family's total monthly expenses are \textdollar2,945.

Step 4: Calculate how much money is left for savings.
Money left for savings $= 3754 - 2945 = 809$

Therefore, the Lim family has \textdollar809 left for savings each month.

\section*{Question 10}
\textbf{Metadata}

\begin{itemize}
  \item Question ID: P4-WNDiv4d1d\_P1-WNCmp\_sonnet4\_Household Finance\_01
  \item Primary KC: WHOLE NUMBERS | Division | dividing whole numbers up to 4 digits by 1 digit
  \item Secondary KC: WHOLE NUMBERS | Comparison and ordering | comparing and ordering whole numbers
  \item Topic: Household finance such as income, utility bills, money, interest, savings, instalment, mortgage, financial planning etc.
  \item Grade: Primary 4
\end{itemize}

\textbf{Solution}

Step 1: Find the total electricity cost for the three months.
Total cost = \textdollar2,184 + \textdollar1,456 + \textdollar1,872 = \textdollar5,512

Step 2: Calculate the average monthly electricity cost by dividing the total by 3.
Average monthly cost = \textdollar5,512 ÷ 3 = \textdollar1,837 remainder \textdollar1
Since we are dealing with money, the average monthly cost is \textdollar1,837.

Step 3: Compare and order the three monthly bills from lowest to highest.
January: \textdollar2,184
February: \textdollar1,456
March: \textdollar1,872

Comparing the amounts:
\textdollar1,456 < \textdollar1,872 < \textdollar2,184

Therefore:
- The average monthly electricity cost is \textdollar1,837
- The bills arranged from lowest to highest are: February (\textdollar1,456), March (\textdollar1,872), January (\textdollar2,184)

\section*{Question 11}
\textbf{Metadata}

\begin{itemize}
  \item Question ID: P4-WNDiv4d1d\_P1-WNMul2nd\_sonnet4\_Household Finance\_01
  \item Primary KC: WHOLE NUMBERS | Division | dividing whole numbers up to 4 digits by 1 digit
  \item Secondary KC: WHOLE NUMBERS | Multiplication | multiplying whole numbers
  \item Topic: Household finance such as income, utility bills, money, interest, savings, instalment, mortgage, financial planning etc.
  \item Grade: Primary 4
\end{itemize}

\textbf{Solution}

Step 1: Find the monthly amount needed for the electricity bill.
The electricity bill is \textdollar2856 every 3 months.
Monthly amount for electricity = \textdollar2856 \div 3 = \textdollar952

Step 2: Find the monthly amount needed for refrigerator savings.
They want to save \textdollar24 each month for the refrigerator.
Monthly amount for refrigerator = \textdollar24

Step 3: Find the total monthly amount to set aside.
Total monthly amount = Monthly electricity amount + Monthly refrigerator amount
Total monthly amount = \textdollar952 + \textdollar24 = \textdollar976

To verify: Over 6 months, they would save \textdollar24 \times 6 = \textdollar144 for the refrigerator.

Therefore, the Lim family should set aside \textdollar976 each month in total.

\section*{Question 12}
\textbf{Metadata}

\begin{itemize}
  \item Question ID: P4-FrAddU12\_P3-FrSmp\_sonnet4\_Household Finance\_01
  \item Primary KC: FRACTIONS | Addition | adding unlike fractions with two different denominators not exceeding 12
  \item Secondary KC: FRACTIONS | Simplifying | expressing a fraction in its simplest form
  \item Topic: Household finance such as income, utility bills, money, interest, savings, instalment, mortgage, financial planning etc.
  \item Grade: Primary 4
\end{itemize}

\textbf{Solution}

To find the total fraction needed for both bills, I need to add the fractions for the electricity bill and water bill.

Electricity bill: $\frac{1}{3}$ of monthly allowance
Water bill: $\frac{1}{4}$ of monthly allowance

To add $\frac{1}{3} + \frac{1}{4}$, I need to find a common denominator.

The denominators are 3 and 4.
The least common multiple of 3 and 4 is 12.

Converting to equivalent fractions with denominator 12:
$\frac{1}{3} = \frac{1 \times 4}{3 \times 4} = \frac{4}{12}$
$\frac{1}{4} = \frac{1 \times 3}{4 \times 3} = \frac{3}{12}$

Now I can add the fractions:
$\frac{4}{12} + \frac{3}{12} = \frac{4 + 3}{12} = \frac{7}{12}$

To check if $\frac{7}{12}$ is in its simplest form, I need to find the highest common factor of 7 and 12.
Factors of 7: 1, 7
Factors of 12: 1, 2, 3, 4, 6, 12
The highest common factor is 1.

Since the highest common factor is 1, $\frac{7}{12}$ is already in its simplest form.

Therefore, Sarah and her mother need to set aside $\frac{7}{12}$ of their monthly allowance for both bills combined.

\section*{Question 13}
\textbf{Metadata}

\begin{itemize}
  \item Question ID: P4-FrSubU12\_P2-FrCmp\_sonnet4\_Household Finance\_01
  \item Primary KC: FRACTIONS | Subtraction | subtracting unlike fractions with two different denominators not exceeding 12
  \item Secondary KC: FRACTIONS | Comparison and ordering | comparing and ordering fractions
  \item Topic: Household finance such as income, utility bills, money, interest, savings, instalment, mortgage, financial planning etc.
  \item Grade: Primary 4
\end{itemize}

\textbf{Solution}

To find how much more of their monthly income is spent on household bills compared to groceries, I need to subtract the fraction spent on groceries from the fraction spent on household bills.

Fraction spent on household bills = $\frac{5}{8}$
Fraction spent on groceries = $\frac{1}{3}$

I need to calculate: $\frac{5}{8} - \frac{1}{3}$

To subtract fractions with different denominators, I need to find a common denominator.
The denominators are 8 and 3.
The least common multiple of 8 and 3 is 24.

Converting to equivalent fractions:
$\frac{5}{8} = \frac{5 \times 3}{8 \times 3} = \frac{15}{24}$

$\frac{1}{3} = \frac{1 \times 8}{3 \times 8} = \frac{8}{24}$

Now I can subtract:
$\frac{15}{24} - \frac{8}{24} = \frac{15-8}{24} = \frac{7}{24}$

To check if this is in simplest form, I need to find the greatest common factor of 7 and 24.
Since 7 is prime and does not divide 24, the greatest common factor is 1.
Therefore, $\frac{7}{24}$ is already in simplest form.

To compare which expense is greater:
$\frac{5}{8} = \frac{15}{24}$ and $\frac{1}{3} = \frac{8}{24}$

Since $\frac{15}{24} > \frac{8}{24}$, we have $\frac{5}{8} > \frac{1}{3}$.

Therefore, $\frac{7}{24}$ more of their monthly income is spent on household bills compared to groceries, and yes, the amount spent on household bills is greater than the amount spent on groceries.

\section*{Question 14}
\textbf{Metadata}

\begin{itemize}
  \item Question ID: P4-DcAdd2d\_P4-DcCnv2Fr\_sonnet4\_Household Finance\_01
  \item Primary KC: DECIMALS | Addition | adding decimals (up to 2 decimal places)
  \item Secondary KC: DECIMALS | Conversion from decimals to fraction | expressing decimals as fractions
  \item Topic: Household finance such as income, utility bills, money, interest, savings, instalment, mortgage, financial planning etc.
  \item Grade: Primary 4
\end{itemize}

\textbf{Solution}

Step 1: Find the total utility expenses by adding the decimals.

Electricity: \textdollar12.50
Water: \textdollar8.75
Gas: \textdollar15.25

Total = \textdollar12.50 + \textdollar8.75 + \textdollar15.25

Adding the dollars: $12 + 8 + 15 = 35$
Adding the cents: $50 + 75 + 25 = 150$ cents = \textdollar1.50

Total utility expenses = \textdollar35 + \textdollar1.50 = \textdollar36.50

Step 2: Express \textdollar36.50 as a fraction in its simplest form.

\textdollar36.50 = \textdollar$\frac{3650}{100}$

To simplify $\frac{3650}{100}$:
Both 3650 and 100 can be divided by 50.

$\frac{3650}{100} = \frac{3650 \div 50}{100 \div 50} = \frac{73}{2}$

Therefore, Mrs. Tan's total utility expenses are \textdollar36.50, which can be expressed as \textdollar$\frac{73}{2}$.

\section*{Question 15}
\textbf{Metadata}

\begin{itemize}
  \item Question ID: P4-DcAdd2d\_P4-DcRnd3d\_sonnet4\_Household Finance\_01
  \item Primary KC: DECIMALS | Addition | adding decimals (up to 2 decimal places)
  \item Secondary KC: DECIMALS | Rounding | rounding decimals up to 3 decimal places to the nearest whole number, 1 decimal place and 2 decimal places 
  \item Topic: Household finance such as income, utility bills, money, interest, savings, instalment, mortgage, financial planning etc.
  \item Grade: Primary 4
\end{itemize}

\textbf{Solution}

Step 1: Round each bill to the nearest cent (2 decimal places).

Electricity bill: \textdollar48.735
Look at the third decimal place: 5
Since 5 ≥ 5, round up: \textdollar48.74

Water bill: \textdollar23.687
Look at the third decimal place: 7
Since 7 ≥ 5, round up: \textdollar23.69

Gas bill: \textdollar31.248
Look at the third decimal place: 8
Since 8 ≥ 5, round up: \textdollar31.25

Step 2: Add the rounded amounts.
$48.74 + $23.69 + $31.25$

$\begin{align}
  48.74 \\
  23.69 \\
+ 31.25 \\
\hline
 103.68
\end{align}$

Therefore, Mrs. Tan needs to pay \textdollar103.68 for her total utility expenses.

\section*{Question 16}
\textbf{Metadata}

\begin{itemize}
  \item Question ID: P4-DcSub2d\_P4-DcAdd2nd\_sonnet4\_Household Finance\_01
  \item Primary KC: DECIMALS | Subtraction | subtracting decimals (up to 2 decimal places)
  \item Secondary KC: DECIMALS | Addition | adding decimals
  \item Topic: Household finance such as income, utility bills, money, interest, savings, instalment, mortgage, financial planning etc.
  \item Grade: Primary 4
\end{itemize}

\textbf{Solution}

Step 1: Find the total amount Mrs. Tan needs to pay for both bills.
Electricity bill + Water bill = \textdollar45.80 + \textdollar23.65 = \textdollar69.45

Step 2: Find how much money she will have left after paying both bills.
Money left = Amount budgeted - Total bill amount
Money left = \textdollar75.00 - \textdollar69.45 = \textdollar5.55

Therefore, Mrs. Tan will have \textdollar5.55 left after paying both utility bills.

\section*{Question 17}
\textbf{Metadata}

\begin{itemize}
  \item Question ID: P4-DcDiv2d1d\_P4-DcCnv2Fr\_sonnet4\_Household Finance\_01
  \item Primary KC: DECIMALS | Division | dividing decimals (up to 2 decimal places) by a 1-digit whole number
  \item Secondary KC: DECIMALS | Conversion from decimals to fraction | expressing decimals as fractions
  \item Topic: Household finance such as income, utility bills, money, interest, savings, instalment, mortgage, financial planning etc.
  \item Grade: Primary 4
\end{itemize}

\textbf{Solution}

Step 1: Find the average monthly electricity cost by dividing the total bill by the number of months.

Average monthly cost = \textdollar12.48 ÷ 4

To divide $12.48$ by $4$:
$12.48 ÷ 4 = 3.12$

So the average monthly electricity cost is \textdollar3.12.

Step 2: Express \textdollar3.12 as a fraction.

$3.12 = 3\frac{12}{100}$

Simplify the fractional part:
$\frac{12}{100} = \frac{12 ÷ 4}{100 ÷ 4} = \frac{3}{25}$

Therefore: $3.12 = 3\frac{3}{25}$

As an improper fraction:
$3\frac{3}{25} = \frac{3 × 25 + 3}{25} = \frac{75 + 3}{25} = \frac{78}{25}$

Answer: Mrs. Tan's average monthly electricity cost is \textdollar3.12, which can be expressed as the fraction $\frac{78}{25}$ or $3\frac{3}{25}$.

\section*{Question 18}
\textbf{Metadata}

\begin{itemize}
  \item Question ID: P4-DcDiv2d1d\_P4-DcSub2nd\_sonnet4\_Household Finance\_01
  \item Primary KC: DECIMALS | Division | dividing decimals (up to 2 decimal places) by a 1-digit whole number
  \item Secondary KC: DECIMALS | Subtraction | subtracting decimals
  \item Topic: Household finance such as income, utility bills, money, interest, savings, instalment, mortgage, financial planning etc.
  \item Grade: Primary 4
\end{itemize}

\textbf{Solution}

Step 1: Find the total amount of both bills.
Total bill amount = \textdollar68.40 + \textdollar25.80 = \textdollar94.20

Step 2: Find the monthly instalment amount by dividing the total by 4.
Monthly instalment = \textdollar94.20 ÷ 4 = \textdollar23.55

Step 3: Find how much money Mrs. Tan will have left after setting aside money for the first month's instalment.
Money left = \textdollar200.00 - \textdollar23.55 = \textdollar176.45

Therefore, Mrs. Tan will have \textdollar176.45 left in her utility budget.

\section*{Question 19}
\textbf{Metadata}

\begin{itemize}
  \item Question ID: P5-FrAddMix\_P3-FrSmp\_sonnet4\_Household Finance\_01
  \item Primary KC: FRACTIONS | Addition | adding mixed numbers
  \item Secondary KC: FRACTIONS | Simplifying | expressing a fraction in its simplest form
  \item Topic: Household finance such as income, utility bills, money, interest, savings, instalment, mortgage, financial planning etc.
  \item Grade: Primary 5
\end{itemize}

\textbf{Solution}

To find Mrs. Tan's total monthly salary, I need to add the three amounts: $2\frac{3}{8} + 1\frac{5}{6} + 3\frac{1}{4}$ thousand.

First, I'll convert all mixed numbers to improper fractions:
$2\frac{3}{8} = \frac{16 + 3}{8} = \frac{19}{8}$
$1\frac{5}{6} = \frac{6 + 5}{6} = \frac{11}{6}$
$3\frac{1}{4} = \frac{12 + 1}{4} = \frac{13}{4}$

Next, I need to find a common denominator. The LCM of 8, 6, and 4 is 24.

Converting to equivalent fractions with denominator 24:
$\frac{19}{8} = \frac{19 \times 3}{8 \times 3} = \frac{57}{24}$
$\frac{11}{6} = \frac{11 \times 4}{6 \times 4} = \frac{44}{24}$
$\frac{13}{4} = \frac{13 \times 6}{4 \times 6} = \frac{78}{24}$

Adding the fractions:
$\frac{57}{24} + \frac{44}{24} + \frac{78}{24} = \frac{57 + 44 + 78}{24} = \frac{179}{24}$

To check if $\frac{179}{24}$ is in its simplest form, I need to find the GCD of 179 and 24.
Since 179 is a prime number and does not divide 24, the GCD is 1.

Therefore, $\frac{179}{24}$ is already in its simplest form.

Converting back to a mixed number: $\frac{179}{24} = 7\frac{11}{24}$

Mrs. Tan's total monthly salary is $7\frac{11}{24}$ thousand or \textdollar7458.33.

\section*{Question 20}
\textbf{Metadata}

\begin{itemize}
  \item Question ID: P5-FrSubMix\_P2-FrAdd2nd\_sonnet4\_Household Finance\_01
  \item Primary KC: FRACTIONS | Subtraction | subtracting mixed numbers
  \item Secondary KC: FRACTIONS | Addition | adding fractions
  \item Topic: Household finance such as income, utility bills, money, interest, savings, instalment, mortgage, financial planning etc.
  \item Grade: Primary 5
\end{itemize}

\textbf{Solution}

Step 1: Find the total amount spent on groceries and utility bills.
Groceries: $1\frac{1}{6}$
Utility bills: $\frac{5}{12}$

To add these fractions, I need to convert $1\frac{1}{6}$ to an improper fraction and find a common denominator.
$1\frac{1}{6} = \frac{7}{6}$

Common denominator of 6 and 12 is 12.
$\frac{7}{6} = \frac{14}{12}$

Total spent = $\frac{14}{12} + \frac{5}{12} = \frac{19}{12} = 1\frac{7}{12}$

Step 2: Subtract the total spent from the household expense allocation.
Household expense allocation: $4\frac{3}{4}$
Total spent: $1\frac{7}{12}$

To subtract, I need a common denominator. The common denominator of 4 and 12 is 12.
$4\frac{3}{4} = 4\frac{9}{12}$

Amount left = $4\frac{9}{12} - 1\frac{7}{12} = 3\frac{2}{12} = 3\frac{1}{6}$

Therefore, $3\frac{1}{6}$ of their income is left from the household expense allocation after paying for groceries and utility bills.

\section*{Question 21}
\textbf{Metadata}

\begin{itemize}
  \item Question ID: P5-FrMulImN\_P2-FrSub2nd\_sonnet4\_Household Finance\_01
  \item Primary KC: FRACTIONS | Multiplication | multiplying a proper/improper fraction and a whole number
  \item Secondary KC: FRACTIONS | Subtraction | subtracting fractions
  \item Topic: Household finance such as income, utility bills, money, interest, savings, instalment, mortgage, financial planning etc.
  \item Grade: Primary 5
\end{itemize}

\textbf{Solution}

Step 1: Find the electricity bill for January.
January bill = $\frac{3}{4} \times 240 = \frac{3 \times 240}{4} = \frac{720}{4} = 180$
So the January bill is \textdollar180.

Step 2: Find the electricity bill for February.
February bill = $\frac{5}{6} \times 240 = \frac{5 \times 240}{6} = \frac{1200}{6} = 200$
So the February bill is \textdollar200.

Step 3: Find the difference between February and January bills.
Difference = \textdollar200 - \textdollar180 = \textdollar20

Alternatively, we can express both amounts as fractions and subtract:
January bill = $\frac{3}{4} \times 240 = \frac{180}{1}$
February bill = $\frac{5}{6} \times 240 = \frac{200}{1}$
Difference = $\frac{200}{1} - \frac{180}{1} = \frac{20}{1} = 20$

Therefore, the Tan family paid \textdollar20 less for electricity in January compared to February.

\section*{Question 22}
\textbf{Metadata}

\begin{itemize}
  \item Question ID: P5-FrMulPIm\_P2-FrSub2nd\_sonnet4\_Household Finance\_01
  \item Primary KC: FRACTIONS | Multiplication | multiplying a proper fraction and a proper/improper fractions
  \item Secondary KC: FRACTIONS | Subtraction | subtracting fractions
  \item Topic: Household finance such as income, utility bills, money, interest, savings, instalment, mortgage, financial planning etc.
  \item Grade: Primary 5
\end{itemize}

\textbf{Solution}

(a) First, find the total amount spent on household expenses:
Amount spent on household expenses = $\frac{3}{5} \times \textdollar4200 = \textdollar2520$

Next, find the amount spent on utility bills:
Amount spent on utility bills = $\frac{2}{7} \times \textdollar2520 = \frac{2 \times 2520}{7} = \frac{5040}{7} = \textdollar720$

(b) First, find the amount spent on groceries:
Amount spent on groceries = $\frac{1}{4} \times \textdollar2520 = \frac{2520}{4} = \textdollar630$

To find how much more is spent on groceries than utility bills, we need to subtract:
Difference = Amount spent on groceries - Amount spent on utility bills
Difference = $\textdollar630 - \textdollar720 = -\textdollar90$

Since the result is negative, utility bills cost more than groceries.
Therefore, groceries cost \textdollar90 less than utility bills, or utility bills cost \textdollar90 more than groceries.

\section*{Question 23}
\textbf{Metadata}

\begin{itemize}
  \item Question ID: P5-FrMulPIm\_P5-FrCnv2Dc\_sonnet4\_Household Finance\_01
  \item Primary KC: FRACTIONS | Multiplication | multiplying a proper fraction and a proper/improper fractions
  \item Secondary KC: FRACTIONS | Conversion to decimals | expressing fractions as decimals
  \item Topic: Household finance such as income, utility bills, money, interest, savings, instalment, mortgage, financial planning etc.
  \item Grade: Primary 5
\end{itemize}

\textbf{Solution}

Step 1: Find the amount spent on groceries.
Amount spent on groceries = $\frac{3}{8} \times \textdollar2400$
$= \frac{3 \times 2400}{8}$
$= \frac{7200}{8}$
$= \textdollar900$

Step 2: Find the amount spent on household cleaning supplies.
Amount spent on cleaning supplies = $\frac{2}{5} \times \textdollar900$
$= \frac{2 \times 900}{5}$
$= \frac{1800}{5}$
$= \textdollar360$

Step 3: Express the answer as a decimal.
$\textdollar360 = \textdollar360.00$

Therefore, Mrs. Lim spends \textdollar360.00 on household cleaning supplies.

\section*{Question 24}
\textbf{Metadata}

\begin{itemize}
  \item Question ID: P5-FrMulMixN\_P2-FrCmp\_sonnet4\_Household Finance\_01
  \item Primary KC: FRACTIONS | Multiplication | multiplying a mixed number and a whole number
  \item Secondary KC: FRACTIONS | Comparison and ordering | comparing and ordering fractions
  \item Topic: Household finance such as income, utility bills, money, interest, savings, instalment, mortgage, financial planning etc.
  \item Grade: Primary 5
\end{itemize}

\textbf{Solution}

Step 1: Calculate the amount spent on groceries.
Groceries = $2\frac{1}{4} \times \textdollar160$
First, convert the mixed number to an improper fraction: $2\frac{1}{4} = \frac{9}{4}$
Groceries = $\frac{9}{4} \times 160 = \frac{9 \times 160}{4} = \frac{1440}{4} = \textdollar360$

Step 2: Calculate the amount spent on rent.
Rent = $\frac{3}{8} \times \textdollar3600 = \frac{3 \times 3600}{8} = \frac{10800}{8} = \textdollar1350$

Step 3: Compare the fractions to determine which expense is greater.
To compare groceries and rent as fractions of income:
Groceries as fraction of income = $\frac{\textdollar360}{\textdollar3600} = \frac{360}{3600} = \frac{1}{10}$
Rent as fraction of income = $\frac{3}{8}$

To compare $\frac{1}{10}$ and $\frac{3}{8}$, find a common denominator:
$\frac{1}{10} = \frac{4}{40}$ and $\frac{3}{8} = \frac{15}{40}$

Since $\frac{15}{40} > \frac{4}{40}$, rent is greater than groceries.

Step 4: Find the difference.
Difference = \textdollar1350 - \textdollar360 = \textdollar990

Therefore, Mrs. Tan spends more on rent than on groceries. The rent expense is greater by \textdollar990.

\section*{Question 25}
\textbf{Metadata}

\begin{itemize}
  \item Question ID: P5-FrMulMixN\_P3-FrSmp\_sonnet4\_Household Finance\_01
  \item Primary KC: FRACTIONS | Multiplication | multiplying a mixed number and a whole number
  \item Secondary KC: FRACTIONS | Simplifying | expressing a fraction in its simplest form
  \item Topic: Household finance such as income, utility bills, money, interest, savings, instalment, mortgage, financial planning etc.
  \item Grade: Primary 5
\end{itemize}

\textbf{Solution}

To find the new electricity bill, I need to multiply the original bill by $2\frac{1}{4}$.

Original bill = \textdollar84
Increase factor = $2\frac{1}{4}$

First, I'll convert the mixed number to an improper fraction:
$2\frac{1}{4} = \frac{2 \times 4 + 1}{4} = \frac{9}{4}$

Now I'll multiply:
New bill = $84 \times \frac{9}{4}$

New bill = $\frac{84 \times 9}{4} = \frac{756}{4}$

To convert this improper fraction to a mixed number:
$756 \div 4 = 189$ with remainder $0$

So $\frac{756}{4} = 189$

Since $\frac{756}{4}$ can be simplified:
$\frac{756}{4} = \frac{756 \div 4}{4 \div 4} = \frac{189}{1} = 189$

Therefore, the Tan family's new monthly electricity bill is \textdollar189.

\section*{Question 26}
\textbf{Metadata}

\begin{itemize}
  \item Question ID: P5-FrMulMixN\_P5-FrCnv2Dc\_sonnet4\_Household Finance\_01
  \item Primary KC: FRACTIONS | Multiplication | multiplying a mixed number and a whole number
  \item Secondary KC: FRACTIONS | Conversion to decimals | expressing fractions as decimals
  \item Topic: Household finance such as income, utility bills, money, interest, savings, instalment, mortgage, financial planning etc.
  \item Grade: Primary 5
\end{itemize}

\textbf{Solution}

Step 1: Find the December electricity bill amount.

The December bill is $2\frac{1}{4}$ times the usual monthly bill.

December bill = $2\frac{1}{4} \times \textdollar48$

Step 2: Convert the mixed number to an improper fraction.

$2\frac{1}{4} = \frac{2 \times 4 + 1}{4} = \frac{9}{4}$

Step 3: Multiply the fraction by the whole number.

$\frac{9}{4} \times 48 = \frac{9 \times 48}{4} = \frac{432}{4} = 108$

So the December bill is \textdollar108.

Step 4: Express the amount as a decimal.

Since \textdollar108 is already a whole number, when expressed as a decimal it is \textdollar108.00.

Therefore, Mrs. Tan's December electricity bill expressed as a decimal is \textdollar108.00.

\section*{Question 27}
\textbf{Metadata}

\begin{itemize}
  \item Question ID: P5-DcMul3dK\_P4-DcRnd3d\_sonnet4\_Household Finance\_01
  \item Primary KC: DECIMALS | Multiplication | multiplying decimals (up to 3 decimal places) by 10, 100, 1000 and their multiples
  \item Secondary KC: DECIMALS | Rounding | rounding decimals up to 3 decimal places to the nearest whole number, 1 decimal place and 2 decimal places 
  \item Topic: Household finance such as income, utility bills, money, interest, savings, instalment, mortgage, financial planning etc.
  \item Grade: Primary 5
\end{itemize}

\textbf{Solution}

Step 1: Find the cost for $1000$ units of electricity.

Cost per unit = \textdollar0.246

Cost for $1000$ units = $0.246 \times 1000 = 246$

So $1000$ units cost \textdollar246.

Step 2: Calculate the actual electricity bill.

Units used = $142.75$

Total cost = $142.75 \times 0.246$

To calculate this, we can use: $142.75 \times 0.246 = 142.75 \times \frac{246}{1000} = \frac{142.75 \times 246}{1000}$

$142.75 \times 246 = 35116.5$

So total cost = $\frac{35116.5}{1000} = 35.1165$

Therefore, the exact bill is \textdollar35.1165.

Step 3: Round to the nearest cent (2 decimal places).

\textdollar35.1165 rounded to 2 decimal places:

Look at the third decimal place: $6$

Since $6 \geq 5$, we round up.

\textdollar35.1165 rounded to the nearest cent = \textdollar35.12

Therefore, Mrs. Lim's electricity bill rounded to the nearest cent is \textdollar35.12.

\section*{Question 28}
\textbf{Metadata}

\begin{itemize}
  \item Question ID: P5-DcDiv3dK\_P4-DcCnv2Fr\_sonnet4\_Household Finance\_01
  \item Primary KC: DECIMALS | Division | dividing decimals (up to 3 decimal places) by 10, 100, 1000 and their multiples
  \item Secondary KC: DECIMALS | Conversion from decimals to fraction | expressing decimals as fractions
  \item Topic: Household finance such as income, utility bills, money, interest, savings, instalment, mortgage, financial planning etc.
  \item Grade: Primary 5
\end{itemize}

\textbf{Solution}

To find how much Mrs. Tan should save each day, I need to divide the total bill by the number of days.

Total electricity bill = \textdollar126.50
Number of days = 50

Daily savings needed = \textdollar126.50 ÷ 50

To divide by 50, I can think of it as dividing by 100 first, then multiplying by 2:
\textdollar126.50 ÷ 100 = \textdollar1.265
\textdollar1.265 × 2 = \textdollar2.53

So Mrs. Tan needs to save \textdollar2.53 each day.

Now I need to express \textdollar2.53 as a fraction.

\textdollar2.53 = 2.53 = $\frac{253}{100}$

To check if this fraction can be simplified, I need to find the highest common factor of 253 and 100.
Factors of 253: 1, 11, 23, 253
Factors of 100: 1, 2, 4, 5, 10, 20, 25, 50, 100

The highest common factor is 1, so the fraction is already in its simplest form.

Therefore, Mrs. Tan should save $\frac{253}{100}$ dollars or \textdollar2.53 each day.

\section*{Question 29}
\textbf{Metadata}

\begin{itemize}
  \item Question ID: P5-RtFndR\_P2-DcCnvN2D\_sonnet4\_Household Finance\_01
  \item Primary KC: RATE | Finding rate | finding rate given total amount and number of units
  \item Secondary KC: DECIMALS | Conversion to larger units | converting a measurement from a smaller unit to a larger unit in decimal form
  \item Topic: Household finance such as income, utility bills, money, interest, savings, instalment, mortgage, financial planning etc.
  \item Grade: Primary 5
\end{itemize}

\textbf{Solution}

To find the rate per unit, I need to divide the total amount paid by the number of units used.

Rate per unit = Total amount ÷ Number of units
Rate per unit = \textdollar43.50 ÷ 2900
Rate per unit = \textdollar0.015

Therefore, the rate charged per unit of electricity is \textdollar0.015.

\section*{Question 30}
\textbf{Metadata}

\begin{itemize}
  \item Question ID: P5-RtFndR\_P2-DcCnvD2N\_sonnet4\_Household Finance\_01
  \item Primary KC: RATE | Finding rate | finding rate given total amount and number of units
  \item Secondary KC: DECIMALS | Conversion to smaller units | converting a measurement from a larger unit in decimal form to a smaller unit
  \item Topic: Household finance such as income, utility bills, money, interest, savings, instalment, mortgage, financial planning etc.
  \item Grade: Primary 5
\end{itemize}

\textbf{Solution}

**Step 1: Find the rate per kilowatt-hour**

Total electricity used = $127.5$ kWh

Total cost = \textdollar76.50

Rate per kWh = Total cost $\div$ Total electricity used

Rate per kWh = \textdollar76.50 $\div$ 127.5

Rate per kWh = \textdollar0.60

**Step 2: Convert from rate per kilowatt-hour to rate per watt-hour**

Since $1$ kilowatt-hour = $1000$ watt-hours

Rate per watt-hour = Rate per kWh $\div$ 1000

Rate per watt-hour = \textdollar0.60 $\div$ 1000 = \textdollar0.0006

**Step 3: Convert dollars to cents**

Since \textdollar1 = 100 cents

Rate per watt-hour = \textdollar0.0006 $\times$ 100 = 0.06 cents

Therefore, the rate charged per watt-hour is $0.06$ cents.

\section*{Question 31}
\textbf{Metadata}

\begin{itemize}
  \item Question ID: P5-RtFndT\_P2-DcCnvN2D\_sonnet4\_Household Finance\_01
  \item Primary KC: RATE | Finding total amount | finding total amount, given rate and number of units
  \item Secondary KC: DECIMALS | Conversion to larger units | converting a measurement from a smaller unit to a larger unit in decimal form
  \item Topic: Household finance such as income, utility bills, money, interest, savings, instalment, mortgage, financial planning etc.
  \item Grade: Primary 5
\end{itemize}

\textbf{Solution}

Step 1: Find the total energy consumed by the air conditioner.
Power = 2840 watts
Time = 150 hours
Energy = Power × Time = 2840 × 150 = 426000 watt-hours

Step 2: Convert watt-hours to kilowatt-hours.
1 kilowatt-hour = 1000 watt-hours
426000 watt-hours = 426000 ÷ 1000 = 426 kWh

Step 3: Calculate the total cost.
Rate = \textdollar0.25 per kWh
Number of units = 426 kWh
Total cost = Rate × Number of units = \textdollar0.25 × 426 = \textdollar106.50

Therefore, Mrs. Tan paid \textdollar106.50 for the electricity used by the air conditioner.

\section*{Question 32}
\textbf{Metadata}

\begin{itemize}
  \item Question ID: P5-RtFndU\_P2-DcCnvD2N\_sonnet4\_Household Finance\_01
  \item Primary KC: RATE | Finding number of unit | finding number of units given rate and total amount
  \item Secondary KC: DECIMALS | Conversion to smaller units | converting a measurement from a larger unit in decimal form to a smaller unit
  \item Topic: Household finance such as income, utility bills, money, interest, savings, instalment, mortgage, financial planning etc.
  \item Grade: Primary 5
\end{itemize}

\textbf{Solution}

Step 1: Find the number of kilowatt-hours used.
Given information:
- Rate: \textdollar0.25 per kWh
- Total amount paid: \textdollar48.75

To find the number of units (kWh), I divide the total amount by the rate:
Number of kWh = Total amount $\div$ Rate per kWh
Number of kWh = \textdollar48.75 $\div$ \textdollar0.25
Number of kWh = $48.75 \div 0.25 = 195$ kWh

Step 2: Convert kilowatt-hours to watt-hours.
Since 1 kilowatt = 1000 watts,
1 kilowatt-hour = 1000 watt-hours

To convert 195 kWh to watt-hours:
195 kWh = $195 \times 1000$ watt-hours = 195,000 watt-hours

Therefore, Mrs. Chen used 195 kilowatt-hours of electricity last month, which is equal to 195,000 watt-hours.

\section*{Question 33}
\textbf{Metadata}

\begin{itemize}
  \item Question ID: P6-FrDivPN\_P3-FrSmp\_sonnet4\_Household Finance\_01
  \item Primary KC: FRACTIONS | Division | dividing a proper fraction by a whole number
  \item Secondary KC: FRACTIONS | Simplifying | expressing a fraction in its simplest form
  \item Topic: Household finance such as income, utility bills, money, interest, savings, instalment, mortgage, financial planning etc.
  \item Grade: Primary 6
\end{itemize}

\textbf{Solution}

Step 1: Find $\frac{2}{3}$ of the monthly savings.
$\frac{2}{3} \times \textdollar600 = \frac{2 \times 600}{3} = \frac{1200}{3} = \textdollar400$

Step 2: Divide this amount equally among 4 savings accounts.
Amount per account = $\frac{\textdollar400}{4}$

Since we need to express this as a fraction, we write:
$\textdollar400 = \frac{400}{1}$

So we need to calculate: $\frac{400}{1} \div 4 = \frac{400}{1} \times \frac{1}{4} = \frac{400}{4} = \frac{100}{1} = 100$

Alternatively, we can work with fractions throughout:
The amount to be divided is $\frac{2}{3}$ of \textdollar600.
Amount per account = $\frac{2}{3} \times \textdollar600 \div 4 = \frac{2}{3} \times \frac{\textdollar600}{4} = \frac{2}{3} \times \textdollar150 = \frac{2 \times 150}{3} = \frac{300}{3} = \textdollar100$

We can also solve this by dividing the fraction by the whole number:
$\frac{2}{3} \div 4 = \frac{2}{3} \times \frac{1}{4} = \frac{2}{12} = \frac{1}{6}$ (simplified form)

Then: $\frac{1}{6} \times \textdollar600 = \frac{600}{6} = \textdollar100$

Therefore, \textdollar100 will be deposited into each savings account.

\section*{Question 34}
\textbf{Metadata}

\begin{itemize}
  \item Question ID: P6-FrDivPP\_P2-FrCmp\_sonnet4\_Household Finance\_01
  \item Primary KC: FRACTIONS | Division | dividing a whole number/proper fraction by a proper fraction
  \item Secondary KC: FRACTIONS | Comparison and ordering | comparing and ordering fractions
  \item Topic: Household finance such as income, utility bills, money, interest, savings, instalment, mortgage, financial planning etc.
  \item Grade: Primary 6
\end{itemize}

\textbf{Solution}

(a) Finding the money allocated for household expenses:
Since $\frac{2}{5}$ of the household expenses allocation equals \textdollar420 (utility bills),
We need to find: \textdollar420 $\div \frac{2}{5}$
\textdollar420 $\div \frac{2}{5} = $ \textdollar420 $\times \frac{5}{2} = $ \textdollar420 $\times 2.5 = $ \textdollar1050
Therefore, the household expenses allocation is \textdollar1050.

(b) Finding the monthly income:
Since $\frac{3}{8}$ of the monthly income equals \textdollar1050 (household expenses allocation),
We need to find: \textdollar1050 $\div \frac{3}{8}$
\textdollar1050 $\div \frac{3}{8} = $ \textdollar1050 $\times \frac{8}{3} = $ \textdollar1050 $\times \frac{8}{3} = $ \textdollar2800
Therefore, the monthly income is \textdollar2800.

(c) Comparing the savings plans:
Plan A: $\frac{1}{6}$ of \textdollar2800 $= \frac{1}{6} \times $ \textdollar2800 $= $ \textdollar$\frac{2800}{6} = $ \textdollar$466.67$
Plan B: $\frac{3}{16}$ of \textdollar2800 $= \frac{3}{16} \times $ \textdollar2800 $= $ \textdollar$\frac{8400}{16} = $ \textdollar$525$

To compare $\frac{1}{6}$ and $\frac{3}{16}$:
Finding common denominator: LCM of 6 and 16 is 48
$\frac{1}{6} = \frac{8}{48}$ and $\frac{3}{16} = \frac{9}{48}$
Since $\frac{9}{48} > \frac{8}{48}$, we have $\frac{3}{16} > \frac{1}{6}$

Therefore, Plan B allows the family to save more money (\textdollar525 compared to \textdollar466.67).

\section*{Question 35}
\textbf{Metadata}

\begin{itemize}
  \item Question ID: P6-FrDivPP\_P3-FrSmp\_sonnet4\_Household Finance\_01
  \item Primary KC: FRACTIONS | Division | dividing a whole number/proper fraction by a proper fraction
  \item Secondary KC: FRACTIONS | Simplifying | expressing a fraction in its simplest form
  \item Topic: Household finance such as income, utility bills, money, interest, savings, instalment, mortgage, financial planning etc.
  \item Grade: Primary 6
\end{itemize}

\textbf{Solution}

First, I need to find how much money Mrs. Tan allocates for her children's school expenses.

Mrs. Tan allocates $\frac{2}{5}$ of her \textdollar360 savings:
$\frac{2}{5} \times 360 = \frac{2 \times 360}{5} = \frac{720}{5} = 144$

So \textdollar144 is allocated for the children's school expenses.

Next, I need to find how much each child receives by dividing \textdollar144 by 3.

Since there are 3 children, each child receives:
$144 \div 3 = \frac{144}{3} = 48$

To express this as a fraction division:
$144 \div 3 = 144 \times \frac{1}{3} = \frac{144}{3}$

Simplifying $\frac{144}{3}$:
$144 \div 3 = 48$, so $\frac{144}{3} = \frac{48}{1} = 48$

Since 48 and 1 have no common factors other than 1, $\frac{48}{1}$ is already in its simplest form.

Therefore, each child receives \textdollar48.

\section*{Question 36}
\textbf{Metadata}

\begin{itemize}
  \item Question ID: P6-PcFndWN\_P1-WNSub2nd\_sonnet4\_Household Finance\_01
  \item Primary KC: PERCENTAGE | Finding the whole | finding the whole given a part and the percentage
  \item Secondary KC: WHOLE NUMBERS | Subtraction | subtracting whole numbers
  \item Topic: Household finance such as income, utility bills, money, interest, savings, instalment, mortgage, financial planning etc.
  \item Grade: Primary 6
\end{itemize}

\textbf{Solution}

Step 1: Find the total electricity bill for this month.

Given that \textdollar84 represents 40\% of the total bill:
40\% of total bill = \textdollar84
$\frac{40}{100} \times \text{total bill} = 84$
$\text{total bill} = 84 \div \frac{40}{100}$
$\text{total bill} = 84 \times \frac{100}{40}$
$\text{total bill} = 84 \times 2.5 = 210$

So the total electricity bill for this month is \textdollar210.

Step 2: Find next month's electricity bill.

Next month's bill = This month's bill - Reduction
Next month's bill = \textdollar210 - \textdollar25 = \textdollar185

Therefore, Mrs. Tan's electricity bill next month will be \textdollar185.

\section*{Question 37}
\textbf{Metadata}

\begin{itemize}
  \item Question ID: P6-PcFndChg\_P1-WNSub2nd\_sonnet4\_Household Finance\_01
  \item Primary KC: PERCENTAGE | Finding change | finding percentage increase/decrease
  \item Secondary KC: WHOLE NUMBERS | Subtraction | subtracting whole numbers
  \item Topic: Household finance such as income, utility bills, money, interest, savings, instalment, mortgage, financial planning etc.
  \item Grade: Primary 6
\end{itemize}

\textbf{Solution}

To find the percentage increase, I need to:

1) Find the increase in the electricity bill
Increase = February bill - January bill
Increase = \textdollar207 - \textdollar180 = \textdollar27

2) Calculate the percentage increase
Percentage increase = $\frac{\text{Increase}}{\text{Original amount}} \times 100\%$

Percentage increase = $\frac{27}{180} \times 100\%$

Percentage increase = $0.15 \times 100\% = 15\%$

Therefore, the electricity bill increased by 15% from January to February.

\section*{Question 38}
\textbf{Metadata}

\begin{itemize}
  \item Question ID: P6-PcFndChg\_P1-WNMul2nd\_sonnet4\_Household Finance\_01
  \item Primary KC: PERCENTAGE | Finding change | finding percentage increase/decrease
  \item Secondary KC: WHOLE NUMBERS | Multiplication | multiplying whole numbers
  \item Topic: Household finance such as income, utility bills, money, interest, savings, instalment, mortgage, financial planning etc.
  \item Grade: Primary 6
\end{itemize}

\textbf{Solution}

Step 1: Find Mrs. Tan's February electricity bill.
January bill = \textdollar120
Increase in February = 25\% of \textdollar120 = $\frac{25}{100} \times 120 = \textdollar30$
February bill = \textdollar120 + \textdollar30 = \textdollar150

Step 2: Find Mrs. Tan's March electricity bill.
Decrease in March = 15\% of \textdollar150 = $\frac{15}{100} \times 150 = \textdollar22.50$
March bill = \textdollar150 - \textdollar22.50 = \textdollar127.50

Step 3: Find the monthly instalment amount for March bill.
March bill is paid in 4 equal monthly instalments.
Monthly instalment = \textdollar127.50 $\div$ 4 = \textdollar31.875 = \textdollar31.88

Therefore, Mrs. Tan pays \textdollar31.88 each month for her March electricity bill.

\section*{Question 39}
\textbf{Metadata}

\begin{itemize}
  \item Question ID: P6-RoFndDvqWN\_P1-WNAdd2nd\_sonnet4\_Household Finance\_01
  \item Primary KC: RATIO | Finding divided quantities | dividing a given quantity in a given ratio
  \item Secondary KC: WHOLE NUMBERS | Addition | adding whole numbers
  \item Topic: Household finance such as income, utility bills, money, interest, savings, instalment, mortgage, financial planning etc.
  \item Grade: Primary 6
\end{itemize}

\textbf{Solution}

First, I need to find how much money Mrs. Tan allocated to each category from her \textdollar1800 bonus.

The ratio for savings : household expenses : education fund is $2:3:4$.

Total ratio parts = $2 + 3 + 4 = 9$ parts

Value of 1 part = \textdollar$1800 \div 9 = $\textdollar$200$

Amount allocated to each category:
- Savings = $2 \times $\textdollar$200 = $\textdollar$400$
- Household expenses = $3 \times $\textdollar$200 = $\textdollar$600$
- Education fund = $4 \times $\textdollar$200 = $\textdollar$800$

Next, I need to add the additional \textdollar200 from selling furniture to the household expenses.

Final amount for household expenses = \textdollar$600 + $\textdollar$200 = $\textdollar$800$

Therefore, Mrs. Tan now has \textdollar800 for household expenses.

\section*{Question 40}
\textbf{Metadata}

\begin{itemize}
  \item Question ID: P6-RoFndRoWN\_P1-WNSub2nd\_sonnet4\_Household Finance\_01
  \item Primary KC: RATIO | Finding ratio | finding the ratio of two or three given whole numbers
  \item Secondary KC: WHOLE NUMBERS | Subtraction | subtracting whole numbers
  \item Topic: Household finance such as income, utility bills, money, interest, savings, instalment, mortgage, financial planning etc.
  \item Grade: Primary 6
\end{itemize}

\textbf{Solution}

First, I need to find the original spending on other expenses.

Original other expenses = Total expenses - Groceries - Utilities
Original other expenses = \textdollar2400 - \textdollar800 - \textdollar600 = \textdollar1000

Next, I'll calculate the new spending amounts after the reductions.

New grocery spending = \textdollar800 - \textdollar200 = \textdollar600
New utility spending = \textdollar600 - \textdollar150 = \textdollar450
New other expenses = \textdollar1000 (unchanged)

Now I can find the ratio of new grocery spending : new utility spending : new other expenses.

The ratio is \textdollar600 : \textdollar450 : \textdollar1000

To simplify this ratio, I need to find the highest common factor of 600, 450, and 1000.
$600 = 2^3 \times 3 \times 5^2$
$450 = 2 \times 3^2 \times 5^2$
$1000 = 2^3 \times 5^3$

The highest common factor is $2 \times 5^2 = 50$

Dividing each amount by 50:
$600 \div 50 = 12$
$450 \div 50 = 9$
$1000 \div 50 = 20$

Therefore, the ratio of new grocery spending to new utility spending to new other expenses is $12:9:20$.

\section*{Question 41}
\textbf{Metadata}

\begin{itemize}
  \item Question ID: P6-RoFndRoWN\_P1-WNDiv2nd\_sonnet4\_Household Finance\_01
  \item Primary KC: RATIO | Finding ratio | finding the ratio of two or three given whole numbers
  \item Secondary KC: WHOLE NUMBERS | Division | dividing whole numbers
  \item Topic: Household finance such as income, utility bills, money, interest, savings, instalment, mortgage, financial planning etc.
  \item Grade: Primary 6
\end{itemize}

\textbf{Solution}

Step 1: Find the total amount spent in each category.
Groceries: \textdollar1800
Utilities: \textdollar900
Entertainment: \textdollar450

Step 2: Divide each amount by 150 to convert into smaller units.
Groceries: $1800 \div 150 = 12$ units
Utilities: $900 \div 150 = 6$ units
Entertainment: $450 \div 150 = 3$ units

Step 3: Find the ratio of groceries to utilities to entertainment.
The ratio is $12 : 6 : 3$

Step 4: Simplify the ratio by finding the highest common factor.
The highest common factor of 12, 6, and 3 is 3.
$12 \div 3 = 4$
$6 \div 3 = 2$
$3 \div 3 = 1$

Therefore, the ratio of groceries to utilities to entertainment is $4 : 2 : 1$.

\section*{Question 42}
\textbf{Metadata}

\begin{itemize}
  \item Question ID: P6-RoFndTmWN\_P1-WNAdd2nd\_sonnet4\_Household Finance\_01
  \item Primary KC: RATIO | Finding a missing term | finding the missing term in a pair of equivalent ratios
  \item Secondary KC: WHOLE NUMBERS | Addition | adding whole numbers
  \item Topic: Household finance such as income, utility bills, money, interest, savings, instalment, mortgage, financial planning etc.
  \item Grade: Primary 6
\end{itemize}

\textbf{Solution}

Step 1: Find the total spending in January.
Since groceries and utilities are in the ratio $3:2$, and groceries cost \textdollar180:
If groceries represent 3 parts, then 1 part = \textdollar180 ÷ 3 = \textdollar60
Utilities represent 2 parts = 2 × \textdollar60 = \textdollar120
Total spending in January = \textdollar180 + \textdollar120 = \textdollar300

Step 2: Find the total spending in February.
Total spending in February = \textdollar300 + \textdollar60 = \textdollar360

Step 3: Find the spending on utilities in February.
The ratio remains $3:2$, so total parts = 3 + 2 = 5 parts
1 part in February = \textdollar360 ÷ 5 = \textdollar72
Utilities in February = 2 parts = 2 × \textdollar72 = \textdollar144

Therefore, Mrs. Tan spent \textdollar144 on utilities in February.

\section*{Question 43}
\textbf{Metadata}

\begin{itemize}
  \item Question ID: P6-AgRepLrEx\_P6-AgEvlLrEx\_sonnet4\_Household Finance\_01
  \item Primary KC: ALGEBRA | Representation and concept | translation of real-world situations into linear algebraic expressions
  \item Secondary KC: ALGEBRA | Evaluation | evaluating linear expressions by substitution
  \item Topic: Household finance such as income, utility bills, money, interest, savings, instalment, mortgage, financial planning etc.
  \item Grade: Primary 6
\end{itemize}

\textbf{Solution}

(a) Mrs. Chen's total monthly income consists of her fixed salary plus overtime pay.

Fixed salary = \textdollar3200
Overtime pay = \textdollar15 per hour × $h$ hours = \textdollar$15h$

Total monthly income = \textdollar$(3200 + 15h)$

(b) Mrs. Chen's total monthly expenses include rent, groceries, and utilities.

Rent = \textdollar800
Groceries = \textdollar12 per person × 4 people = \textdollar48
Utilities = \textdollar180

Total monthly expenses = \textdollar$(800 + 48 + 180) = \textdollar1028$

(c) When Mrs. Chen works 8 hours of overtime ($h = 8$):

Total monthly income = \textdollar$(3200 + 15 \times 8)$
                     = \textdollar$(3200 + 120)$
                     = \textdollar3320

Money left after expenses = Total income - Total expenses
                          = \textdollar3320 - \textdollar1028
                          = \textdollar2292

Therefore, Mrs. Chen will have \textdollar2292 left after paying all her expenses.

\section*{Question 44}
\textbf{Metadata}

\begin{itemize}
  \item Question ID: P6-AgSlvLrN\_P6-AgRepLrEx\_sonnet4\_Household Finance\_01
  \item Primary KC: ALGEBRA | Solving simple linear equations | solving linear equations involving whole number coefficient and one variable only
  \item Secondary KC: ALGEBRA | Representation and concept | translation of real-world situations into linear algebraic expressions
  \item Topic: Household finance such as income, utility bills, money, interest, savings, instalment, mortgage, financial planning etc.
  \item Grade: Primary 6
\end{itemize}

\textbf{Solution}

Let $x$ be the fixed service charge in dollars and $y$ be the cost per unit of electricity in dollars.

From the given information, I can write two equations:
Last month: $x + 320y = 68$ ... (1)
This month: $x + 280y = 60$ ... (2)

To find the fixed service charge $x$, I will eliminate $y$ by subtracting equation (2) from equation (1):
$(x + 320y) - (x + 280y) = 68 - 60$
$x + 320y - x - 280y = 8$
$40y = 8$
$y = \frac{8}{40} = 0.2$

Now I can substitute $y = 0.2$ into equation (2) to find $x$:
$x + 280(0.2) = 60$
$x + 56 = 60$
$x = 60 - 56$
$x = 4$

Therefore, the fixed service charge that Sarah's family pays each month is \textdollar$4$.

To verify: Last month: $4 + 320(0.2) = 4 + 64 = 68$ ✓
This month: $4 + 280(0.2) = 4 + 56 = 60$ ✓

\section*{Question 45}
\textbf{Metadata}

\begin{itemize}
  \item Question ID: O1-RoRepFr\_P2-FrSub2nd\_sonnet4\_Household Finance\_01
  \item Primary KC: RATIO | Representation and concept | ratios involving fractions
  \item Secondary KC: FRACTIONS | Subtraction | subtracting fractions
  \item Topic: Household finance such as income, utility bills, money, interest, savings, instalment, mortgage, financial planning etc.
  \item Grade: Secondary O-level 1
\end{itemize}

\textbf{Solution}

**Part (a): Finding the amount spent on each category**

Given ratio: $\frac{2}{3} : \frac{5}{6} : \frac{1}{2}$

To work with this ratio, I need to convert it to whole numbers by finding the LCM of the denominators 3, 6, and 2.

LCM of 3, 6, and 2 = 6

Multiplying each fraction by 6:
- Utilities: $\frac{2}{3} \times 6 = 4$
- Groceries: $\frac{5}{6} \times 6 = 5$  
- Savings: $\frac{1}{2} \times 6 = 3$

So the ratio becomes $4 : 5 : 3$

Total parts = $4 + 5 + 3 = 12$ parts

Total budget = \textdollar1800

Value of 1 part = $\frac{\textdollar1800}{12} = \textdollar150$

Therefore:
- Utilities: $4 \times \textdollar150 = \textdollar600$
- Groceries: $5 \times \textdollar150 = \textdollar750$
- Savings: $3 \times \textdollar150 = \textdollar450$

**Part (b): Finding the new ratio after reductions**

Reduction in grocery spending: $\frac{1}{8} \times \textdollar750 = \textdollar93.75$

Reduction in utility spending: $\frac{1}{12} \times \textdollar600 = \textdollar50$

New amounts:
- New utilities: $\textdollar600 - \textdollar50 = \textdollar550$
- New groceries: $\textdollar750 - \textdollar93.75 = \textdollar656.25$
- Savings: $\textdollar450$ (unchanged)

To find the ratio, I'll divide each amount by the GCD.

Converting to fractions:
- Utilities: $\textdollar550 = \frac{550}{1}$
- Groceries: $\textdollar656.25 = \frac{2625}{4}$
- Savings: $\textdollar450 = \frac{450}{1}$

To express as a ratio with common denominators:
$\frac{2200}{4} : \frac{2625}{4} : \frac{1800}{4}$

Simplifying by dividing by 25:
$\frac{88}{4} : \frac{105}{4} : \frac{72}{4}$

Therefore, the new ratio is $22 : 26.25 : 18$ or $\frac{22}{1} : \frac{21}{0.8} : \frac{18}{1}$

Expressed in fractional form: $22 : \frac{105}{4} : 18$

\section*{Question 46}
\textbf{Metadata}

\begin{itemize}
  \item Question ID: O1-RoRepFr\_P6-FrDiv2nd\_sonnet4\_Household Finance\_01
  \item Primary KC: RATIO | Representation and concept | ratios involving fractions
  \item Secondary KC: FRACTIONS | Division | fraction division
  \item Topic: Household finance such as income, utility bills, money, interest, savings, instalment, mortgage, financial planning etc.
  \item Grade: Secondary O-level 1
\end{itemize}

\textbf{Solution}

**Solution:**

**(a) Finding the allocation for each category**

Given ratio: $2\frac{1}{4} : 1\frac{1}{2} : \frac{3}{4}$

First, convert mixed numbers to improper fractions:
$2\frac{1}{4} = \frac{9}{4}$
$1\frac{1}{2} = \frac{3}{2}$
$\frac{3}{4} = \frac{3}{4}$

So the ratio is $\frac{9}{4} : \frac{3}{2} : \frac{3}{4}$

To work with whole numbers, find the LCM of denominators 4, 2, and 4, which is 4.

Multiply each ratio term by 4:
$\frac{9}{4} \times 4 = 9$
$\frac{3}{2} \times 4 = 6$
$\frac{3}{4} \times 4 = 3$

The simplified ratio is $9 : 6 : 3$

Total ratio parts = $9 + 6 + 3 = 18$

Total budget = \textdollar4200

Value of each part = $\frac{\textdollar4200}{18} = \textdollar\frac{4200}{18} = \textdollar\frac{700}{3}$

Housing expenses = $9 \times \textdollar\frac{700}{3} = \textdollar\frac{6300}{3} = \textdollar2100$

Food expenses = $6 \times \textdollar\frac{700}{3} = \textdollar\frac{4200}{3} = \textdollar1400$

Savings = $3 \times \textdollar\frac{700}{3} = \textdollar\frac{2100}{3} = \textdollar700$

**(b) Finding the new ratio after transferring money**

Amount transferred from food to savings = $\frac{1}{5} \times \textdollar1400$

To calculate: $\frac{1}{5} \times 1400 = \frac{1400}{5} = 280$

So \textdollar280 is transferred from food to savings.

New allocations:
- Housing expenses: \textdollar2100 (unchanged)
- Food expenses: $\textdollar1400 - \textdollar280 = \textdollar1120$
- Savings: $\textdollar700 + \textdollar280 = \textdollar980$

New ratio = $2100 : 1120 : 980$

To simplify, find GCD of 2100, 1120, and 980:
$2100 = 2^2 \times 3 \times 5^2 \times 7$
$1120 = 2^5 \times 5 \times 7$
$980 = 2^2 \times 5 \times 7^2$

GCD = $2^2 \times 5 \times 7 = 140$

Dividing by 140:
$\frac{2100}{140} : \frac{1120}{140} : \frac{980}{140} = 15 : 8 : 7$

Therefore, the new ratio is $15 : 8 : 7$.

\section*{Question 47}
\textbf{Metadata}

\begin{itemize}
  \item Question ID: O1-RoRepDc\_P4-DcSub2nd\_sonnet4\_Household Finance\_01
  \item Primary KC: RATIO | Representation and concept | ratios involving decimals
  \item Secondary KC: DECIMALS | Subtraction | subtracting decimals
  \item Topic: Household finance such as income, utility bills, money, interest, savings, instalment, mortgage, financial planning etc.
  \item Grade: Secondary O-level 1
\end{itemize}

\textbf{Solution}

Step 1: Find the spending on each category in January.

The ratio is $2.5 : 4.2 : 1.8$
Total ratio parts = $2.5 + 4.2 + 1.8 = 8.5$

Total spending in January = \textdollar850

Utilities spending = $\frac{2.5}{8.5} \times 850 = \textdollar250$

Groceries spending = $\frac{4.2}{8.5} \times 850 = \textdollar420$

Entertainment spending = $\frac{1.8}{8.5} \times 850 = \textdollar180$

Step 2: Calculate the spending for each category in February.

Utilities in February = $250 - 15.50 = \textdollar234.50$

Groceries in February = $420 - 22.75 = \textdollar397.25$

Entertainment in February = \textdollar180 (unchanged)

Step 3: Find the total spending in February.

Total spending in February = $234.50 + 397.25 + 180 = \textdollar811.75$

Therefore, the Tan family's total spending on these three categories in February was \textdollar811.75.

\section*{Question 48}
\textbf{Metadata}

\begin{itemize}
  \item Question ID: O1-PcRep2q\_O1-PcCnv2Dc\_sonnet4\_Household Finance\_01
  \item Primary KC: PERCENTAGE | Representation and concept | comparing two quantities by percentage
  \item Secondary KC: PERCENTAGE | Conversion to decimals | expressing percentage as a decimal
  \item Topic: Household finance such as income, utility bills, money, interest, savings, instalment, mortgage, financial planning etc.
  \item Grade: Secondary O-level 1
\end{itemize}

\textbf{Solution}

Given information:
- Monthly income = \textdollar4800
- Rent = \textdollar1440
- Groceries = \textdollar960

Step 1: Express rent as a percentage of total income.
Percentage spent on rent = $\frac{\text{Rent}}{\text{Total Income}} \times 100\%$
Percentage spent on rent = $\frac{1440}{4800} \times 100\%$
Percentage spent on rent = $0.3 \times 100\% = 30\%$

Step 2: Convert rent percentage to decimal.
$30\% = \frac{30}{100} = 0.3$

Step 3: Express groceries as a percentage of total income.
Percentage spent on groceries = $\frac{\text{Groceries}}{\text{Total Income}} \times 100\%$
Percentage spent on groceries = $\frac{960}{4800} \times 100\%$
Percentage spent on groceries = $0.2 \times 100\% = 20\%$

Step 4: Convert groceries percentage to decimal.
$20\% = \frac{20}{100} = 0.2$

Step 5: Compare the two percentages.
Rent percentage: $30\%$ (decimal: $0.3$)
Groceries percentage: $20\%$ (decimal: $0.2$)

Since $30\% > 20\%$, rent takes up a larger portion of their income.
Difference = $30\% - 20\% = 10\%$

Therefore, the Tan family spends $30\%$ (or $0.3$ as a decimal) of their income on rent and $20\%$ (or $0.2$ as a decimal) on groceries. Rent takes up a larger portion of their income by $10\%$.

\section*{Question 49}
\textbf{Metadata}

\begin{itemize}
  \item Question ID: O1-PcFndRslt\_P1-WNDiv2nd\_sonnet4\_Household Finance\_01
  \item Primary KC: PERCENTAGE | Finding result after change | increasing/decreasing a quantity by a given percentage
  \item Secondary KC: WHOLE NUMBERS | Division | dividing whole numbers
  \item Topic: Household finance such as income, utility bills, money, interest, savings, instalment, mortgage, financial planning etc.
  \item Grade: Secondary O-level 1
\end{itemize}

\textbf{Solution}

Step 1: Calculate the reduction in the electricity bill.
Reduction = $15\% \times \textdollar240 = \frac{15}{100} \times \textdollar240 = \textdollar36$

Step 2: Find the new electricity bill after the reduction.
New bill = Original bill - Reduction
New bill = $\textdollar240 - \textdollar36 = \textdollar204$

Step 3: Divide the reduced bill equally among 4 family members.
Amount per person = $\textdollar204 \div 4 = \textdollar51$

Therefore, each family member should contribute \textdollar51.

\section*{Question 50}
\textbf{Metadata}

\begin{itemize}
  \item Question ID: O1-AgSlvFrLr\_O1-AgRepEq\_sonnet4\_Household Finance\_01
  \item Primary KC: ALGEBRA | Solving | solving simple fractional equations that can be reduced to linear equations
  \item Secondary KC: ALGEBRA | Representation and concept | translation of simple real-world situations to equations
  \item Topic: Household finance such as income, utility bills, money, interest, savings, instalment, mortgage, financial planning etc.
  \item Grade: Secondary O-level 1
\end{itemize}

\textbf{Solution}

Let Sarah's monthly salary be $x$ dollars.

From the given information:
- Amount spent on household expenses = $\frac{2}{5}x$ dollars
- Amount saved = \textdollar240
- Remaining money = \textdollar60

Since her total salary equals the sum of expenses, savings, and remaining money, we can write the equation:
$$x = \frac{2}{5}x + 240 + 60$$

Simplifying the right side:
$$x = \frac{2}{5}x + 300$$

Subtracting $\frac{2}{5}x$ from both sides:
$$x - \frac{2}{5}x = 300$$

$$\frac{5}{5}x - \frac{2}{5}x = 300$$

$$\frac{3}{5}x = 300$$

Multiplying both sides by $\frac{5}{3}$:
$$x = 300 \times \frac{5}{3}$$

$$x = \frac{1500}{3}$$

$$x = 500$$

Therefore, Sarah's monthly salary is \textdollar500.

Verification:
- Household expenses: $\frac{2}{5} \times 500 = \textdollar200$
- Savings: \textdollar240
- Remaining: \textdollar60
- Total: $200 + 240 + 60 = \textdollar500$ ✓

\section*{Question 51}
\textbf{Metadata}

\begin{itemize}
  \item Question ID: O2-RoRepIvP\_P1-WNMul2nd\_sonnet4\_Household Finance\_01
  \item Primary KC: RATIO | Representation and concept | inverse proportion
  \item Secondary KC: WHOLE NUMBERS | Multiplication | multiplying whole numbers
  \item Topic: Household finance such as income, utility bills, money, interest, savings, instalment, mortgage, financial planning etc.
  \item Grade: Secondary O-level 2
\end{itemize}

\textbf{Solution}

Let $b$ represent the number of LED bulbs and $C$ represent the monthly electricity bill in dollars.

Since the bill is inversely proportional to the number of bulbs:
$C = \frac{k}{b}$ where $k$ is a constant.

First, I'll find the constant $k$ using the given information.
When $b = 8$, $C = 120$:
$120 = \frac{k}{8}$
$k = 120 \times 8 = 960$

So the relationship is: $C = \frac{960}{b}$

To find the number of bulbs needed for a \textdollar80 bill:
$80 = \frac{960}{b}$
$b = \frac{960}{80} = 12$

Therefore, they need 12 LED bulbs to achieve a \textdollar80 monthly bill.

Since they currently have 8 bulbs, they need to purchase:
$12 - 8 = 4$ additional bulbs

The cost of purchasing 4 additional bulbs:
$4 \times 15 = 60$

Therefore, the family needs to use 12 LED bulbs and will spend \textdollar60 on purchasing the additional bulbs.

\section*{Question 52}
\textbf{Metadata}

\begin{itemize}
  \item Question ID: O2-RoRepIvP\_P1-WNDiv2nd\_sonnet4\_Household Finance\_01
  \item Primary KC: RATIO | Representation and concept | inverse proportion
  \item Secondary KC: WHOLE NUMBERS | Division | dividing whole numbers
  \item Topic: Household finance such as income, utility bills, money, interest, savings, instalment, mortgage, financial planning etc.
  \item Grade: Secondary O-level 2
\end{itemize}

\textbf{Solution}

Let $n$ be the number of energy-saving devices and $B$ be the monthly electricity bill in dollars.

Since the electricity bill is inversely proportional to the number of devices:
$B = \frac{k}{n}$ where $k$ is a constant.

Step 1: Find the constant $k$.
When $n = 4$, $B = 240$:
$240 = \frac{k}{4}$
$k = 240 \times 4 = 960$

Therefore, $B = \frac{960}{n}$

Step 2: Find the number of devices needed for a \textdollar160 bill.
When $B = 160$:
$160 = \frac{960}{n}$
$n = \frac{960}{160} = 6$

Step 3: Calculate the additional cost.
Extra devices needed = $6 - 4 = 2$ devices
Cost per device = \textdollar45
Total additional cost = $2 \times 45 = 90$

Therefore, the Tan family needs to use 6 energy-saving devices, and the total additional cost is \textdollar90.

\section*{Question 53}
\textbf{Metadata}

\begin{itemize}
  \item Question ID: O2-AgSlvLr2v\_O1-AgRepEq\_sonnet4\_Household Finance\_01
  \item Primary KC: ALGEBRA | Solving | solving linear equations in two variables
  \item Secondary KC: ALGEBRA | Representation and concept | translation of simple real-world situations to equations
  \item Topic: Household finance such as income, utility bills, money, interest, savings, instalment, mortgage, financial planning etc.
  \item Grade: Secondary O-level 2
\end{itemize}

\textbf{Solution}

Let me define variables for this problem:
Let $w$ = cost of water bill (in dollars)
Let $e$ = cost of electricity bill (in dollars)

From the given information, I can write two equations:

Equation 1: The total of both bills is \textdollar180
$e + w = 180$

Equation 2: The electricity bill is \textdollar30 more than twice the water bill
$e = 2w + 30$

Now I'll solve this system of linear equations using substitution.

Substituting equation 2 into equation 1:
$(2w + 30) + w = 180$
$3w + 30 = 180$
$3w = 180 - 30$
$3w = 150$
$w = 50$

Now I can find the electricity bill using equation 2:
$e = 2(50) + 30$
$e = 100 + 30$
$e = 130$

Let me verify this solution:
- Total bills: $130 + 50 = 180$ ✓
- Electricity bill is \textdollar30 more than twice the water bill: $130 = 2(50) + 30 = 130$ ✓

Therefore, the water bill costs \textdollar50 and the electricity bill costs \textdollar130.

\section*{Question 54}
\textbf{Metadata}

\begin{itemize}
  \item Question ID: O2-SPFndmdn\_O2-SPFndmode\_sonnet4\_Household Finance\_01
  \item Primary KC: STATISTICS AND PROBABILITY | Finding median | Finding median for a set of data
  \item Secondary KC: STATISTICS AND PROBABILITY | Finding mode | Finding mode for a set of data
  \item Topic: Household finance such as income, utility bills, money, interest, savings, instalment, mortgage, financial planning etc.
  \item Grade: Secondary O-level 2
\end{itemize}

\textbf{Solution}

(a) To find the median, we first arrange the data in ascending order:
\textdollar78, \textdollar82, \textdollar85, \textdollar85, \textdollar85, \textdollar87, \textdollar88, \textdollar89, \textdollar90, \textdollar92, \textdollar94, \textdollar95

Since there are 12 values (even number), the median is the average of the 6th and 7th values.
The 6th value is \textdollar87 and the 7th value is \textdollar88.
Median $= \frac{87 + 88}{2} = \frac{175}{2} = 87.5$

Therefore, the median monthly electricity bill is \textdollar87.50.

(b) To find the mode, we identify the value that appears most frequently:
\textdollar78 appears 1 time
\textdollar82 appears 1 time
\textdollar85 appears 3 times
\textdollar87 appears 1 time
\textdollar88 appears 1 time
\textdollar89 appears 1 time
\textdollar90 appears 1 time
\textdollar92 appears 1 time
\textdollar94 appears 1 time
\textdollar95 appears 1 time

The mode is \textdollar85 as it appears most frequently (3 times).

(c) Mrs. Tan should use the median (\textdollar87.50) rather than the mode (\textdollar85) for her monthly budget planning. The median represents the middle value and gives a better indication of the typical monthly bill amount. Since the median is higher than the mode, using the median ensures she sets aside enough money to cover most of her monthly electricity bills, providing a more reliable budget cushion.

\section*{Question 55}
\textbf{Metadata}

\begin{itemize}
  \item Question ID: O2-SPFndmdn\_O3-SPFndPctl\_sonnet4\_Household Finance\_01
  \item Primary KC: STATISTICS AND PROBABILITY | Finding median | Finding median for a set of data
  \item Secondary KC: STATISTICS AND PROBABILITY | Finding percentiles | finding percentiles for a set of data
  \item Topic: Household finance such as income, utility bills, money, interest, savings, instalment, mortgage, financial planning etc.
  \item Grade: Secondary O-level 2
\end{itemize}

\textbf{Solution}

(a) To find the median, first arrange the 12 monthly bills in ascending order:
\textdollar38, \textdollar42, \textdollar44, \textdollar45, \textdollar47, \textdollar49, \textdollar52, \textdollar53, \textdollar55, \textdollar56, \textdollar58, \textdollar61

Since there are 12 values (even number), the median is the average of the 6th and 7th values.
6th value = \textdollar49
7th value = \textdollar52
Median = $\frac{49 + 52}{2} = \frac{101}{2} = 50.5$

Therefore, the median monthly electricity bill is \textdollar50.50.

(b) To find the 75th percentile:
Position of 75th percentile = $\frac{75}{100} \times 12 = 9$

Since the position is exactly 9, the 75th percentile is the 9th value in the ordered list.
From the ordered data: \textdollar38, \textdollar42, \textdollar44, \textdollar45, \textdollar47, \textdollar49, \textdollar52, \textdollar53, \textdollar55, \textdollar56, \textdollar58, \textdollar61

The 9th value is \textdollar55.

Therefore, Mrs. Tan should set her monthly electricity budget at \textdollar55 to ensure that her bill will not exceed this amount in 75% of the months.

\section*{Question 56}
\textbf{Metadata}

\begin{itemize}
  \item Question ID: O3-BPOpr\_O3-BPRepPosI\_sonnet4\_Household Finance\_01
  \item Primary KC: BASE AND POWER | Operations | laws of indices
  \item Secondary KC: BASE AND POWER | Representation and concept  | positive indices that is not 1
  \item Topic: Household finance such as income, utility bills, money, interest, savings, instalment, mortgage, financial planning etc.
  \item Grade: Secondary O-level 3/4
\end{itemize}

\textbf{Solution}

(a) Finding the investment value after 9 years:

Since the investment doubles every 3 years, after 9 years it will have doubled $9 \div 3 = 3$ times.

The investment value can be expressed as: \textdollar2000 $\times 2^3$

Calculating: $2^3 = 8$

Therefore, the investment value after 9 years = \textdollar2000 $\times 8$ = \textdollar16000

(b) Calculating the total monthly utility bill:

Electricity bill = \textdollar40 = \textdollar$40^1$
Water bill = \textdollar$40^2$ = \textdollar1600
Gas bill = \textdollar$40^3$ = \textdollar64000

Total monthly utility bill = \textdollar40 + \textdollar1600 + \textdollar64000 = \textdollar65640

To express $40^2 \times 40^3$ using laws of indices:
Using the law $a^m \times a^n = a^{m+n}$:
$40^2 \times 40^3 = 40^{2+3} = 40^5$

Calculating: $40^5 = 102400000$

\section*{Question 57}
\textbf{Metadata}

\begin{itemize}
  \item Question ID: O3-BPOpr\_O3-BPRepFrI\_sonnet4\_Household Finance\_01
  \item Primary KC: BASE AND POWER | Operations | laws of indices
  \item Secondary KC: BASE AND POWER | Representation and concept  | fractional indices
  \item Topic: Household finance such as income, utility bills, money, interest, savings, instalment, mortgage, financial planning etc.
  \item Grade: Secondary O-level 3/4
\end{itemize}

\textbf{Solution}

(a) We need to find $t$ when $A = \textdollar32000$ and $P = \textdollar8000$.

Substituting into the formula:
$32000 = 8000 \times 2^{t/4}$

Dividing both sides by 8000:
$4 = 2^{t/4}$

Since $4 = 2^2$, we have:
$2^2 = 2^{t/4}$

Using the law of indices (if $a^m = a^n$, then $m = n$):
$2 = \frac{t}{4}$

Therefore: $t = 8$ years

(b) We need to find $A$ when $t = 6$ years.

Substituting into the formula:
$A = 8000 \times 2^{6/4}$
$A = 8000 \times 2^{3/2}$

Using fractional indices: $2^{3/2} = 2^3 \times 2^{1/2} = 8 \times \sqrt{2}$

Since $\sqrt{2} \approx 1.414$:
$2^{3/2} = 8 \times 1.414 = 11.312$

Therefore:
$A = 8000 \times 11.312 = 90496$

To the nearest dollar, Sarah will have \textdollar90496 after 6 years.

\section*{Question 58}
\textbf{Metadata}

\begin{itemize}
  \item Question ID: O3-STOprUn\_O3-STOprIns\_sonnet4\_Household Finance\_01
  \item Primary KC: SET | Set operations | union of two sets
  \item Secondary KC: SET | Set operations | intersection of two sets
  \item Topic: Household finance such as income, utility bills, money, interest, savings, instalment, mortgage, financial planning etc.
  \item Grade: Secondary O-level 3/4
\end{itemize}

\textbf{Solution}

Given: 
$A = \{\text{electricity, water, internet, phone, insurance}\}$ 
$B = \{\text{water, internet, mortgage, car loan, insurance}\}$ 

(a) Bills that can be paid using either credit card or online banking represent the union $A \cup B$. 

$A \cup B = \{\text{electricity, water, internet, phone, insurance, mortgage, car loan}\}$ 

Therefore, 7 bills can be paid using either method. 

(b) Bills that can be paid using both methods represent the intersection $A \cap B$. 

$A \cap B = \{\text{water, internet, insurance}\}$ 

Therefore, 3 bills can be paid using both credit card and online banking. 

(c) Verification: 
Total distinct bills = 7 
Bills payable by credit card or online banking = $|A \cup B| = 7$ 
Bills payable only by other methods = 2 

Total bills accounted for = $7 + 2 = 9$ bills 

However, we're told the total number of distinct bills is 7. This means: 
$|A \cup B| + \text{bills paid by other methods only} = 7$ 
$7 + 2 = 9 \neq 7$ 

This indicates an inconsistency. The correct interpretation should be: 
If there are 7 total distinct bills and 2 can only be paid by other methods, then $|A \cup B| = 7 - 2 = 5$. 

Since our calculation shows $|A \cup B| = 7$, the sets $A$ and $B$ contain more bills than can actually be paid by these methods, suggesting the family needs to re-examine their categorization.

\section*{Question 59}
\textbf{Metadata}

\begin{itemize}
  \item Question ID: O3-MXMulSM\_O3-MXSub\_sonnet4\_Household Finance\_01
  \item Primary KC: MATRICES | Multiplication | product of a scalar quantity and a matrix
  \item Secondary KC: MATRICES | Subtraction | subtraction of matrices
  \item Topic: Household finance such as income, utility bills, money, interest, savings, instalment, mortgage, financial planning etc.
  \item Grade: Secondary O-level 3/4
\end{itemize}

\textbf{Solution}

(a) To find the additional savings target matrix $S$, we multiply the January expense matrix $A$ by the scalar $0.15$:

$S = 0.15 \times A = 0.15 \times \begin{pmatrix} 120 & 80 \\ 400 & 300 \\ 150 & 100 \end{pmatrix}$

$S = \begin{pmatrix} 0.15 \times 120 & 0.15 \times 80 \\ 0.15 \times 400 & 0.15 \times 300 \\ 0.15 \times 150 & 0.15 \times 100 \end{pmatrix}$

$S = \begin{pmatrix} 18 & 12 \\ 60 & 45 \\ 22.5 & 15 \end{pmatrix}$

(b) To find the net financial impact matrix $N$, we subtract the February expenses matrix $B$ from the additional savings target matrix $S$:

$N = S - B = \begin{pmatrix} 18 & 12 \\ 60 & 45 \\ 22.5 & 15 \end{pmatrix} - \begin{pmatrix} 110 & 75 \\ 380 & 280 \\ 120 & 90 \end{pmatrix}$

$N = \begin{pmatrix} 18-110 & 12-75 \\ 60-380 & 45-280 \\ 22.5-120 & 15-90 \end{pmatrix}$

$N = \begin{pmatrix} -92 & -63 \\ -320 & -235 \\ -97.5 & -75 \end{pmatrix}$

The matrix $N$ represents the financial shortfall in each category. The negative values indicate that the family's February expenses exceed their additional savings target by these amounts. This means the family needs to either reduce their February expenses further or increase their savings target to meet their financial planning goals.

\section*{Question 60}
\textbf{Metadata}

\begin{itemize}
  \item Question ID: O3-MXMulSM\_O3-MXMul\_sonnet4\_Household Finance\_01
  \item Primary KC: MATRICES | Multiplication | product of a scalar quantity and a matrix
  \item Secondary KC: MATRICES | Multiplication | multiplication of matrices
  \item Topic: Household finance such as income, utility bills, money, interest, savings, instalment, mortgage, financial planning etc.
  \item Grade: Secondary O-level 3/4
\end{itemize}

\textbf{Solution}

(a) To find the expected expenses after a 15% increase, we multiply the matrix $E$ by the scalar $1.15$:

$1.15E = 1.15 \times \begin{pmatrix} 180 & 420 & 150 \\ 195 & 385 & 165 \\ 210 & 450 & 140 \end{pmatrix}$

$= \begin{pmatrix} 1.15 \times 180 & 1.15 \times 420 & 1.15 \times 150 \\ 1.15 \times 195 & 1.15 \times 385 & 1.15 \times 165 \\ 1.15 \times 210 & 1.15 \times 450 & 1.15 \times 140 \end{pmatrix}$

$= \begin{pmatrix} 207 & 483 & 172.5 \\ 224.25 & 442.75 & 189.75 \\ 241.5 & 517.5 & 161 \end{pmatrix}$

(b) To find the total monthly expenses, we calculate $E \times V$:

$E \times V = \begin{pmatrix} 180 & 420 & 150 \\ 195 & 385 & 165 \\ 210 & 450 & 140 \end{pmatrix} \times \begin{pmatrix} 1 \\ 1 \\ 1 \end{pmatrix}$

$= \begin{pmatrix} 180 + 420 + 150 \\ 195 + 385 + 165 \\ 210 + 450 + 140 \end{pmatrix}$

$= \begin{pmatrix} 750 \\ 745 \\ 800 \end{pmatrix}$

The total expenses were \textdollar750 in January, \textdollar745 in February, and \textdollar800 in March.

(c) To find the total quarterly expense after the 15% increase, we multiply the increased expense matrix by $V$:

$(1.15E) \times V = \begin{pmatrix} 207 & 483 & 172.5 \\ 224.25 & 442.75 & 189.75 \\ 241.5 & 517.5 & 161 \end{pmatrix} \times \begin{pmatrix} 1 \\ 1 \\ 1 \end{pmatrix}$

$= \begin{pmatrix} 862.5 \\ 856.75 \\ 920 \end{pmatrix}$

Total quarterly expense after 15% increase = $862.5 + 856.75 + 920 = \textdollar2639.25$

Alternatively, we could calculate this as $1.15 \times (750 + 745 + 800) = 1.15 \times 2295 = \textdollar2639.25$.

\section*{Question 61}
\textbf{Metadata}

\begin{itemize}
  \item Question ID: O3-MXMul\_O3-MXSub\_sonnet4\_Household Finance\_01
  \item Primary KC: MATRICES | Multiplication | multiplication of matrices
  \item Secondary KC: MATRICES | Subtraction | subtraction of matrices
  \item Topic: Household finance such as income, utility bills, money, interest, savings, instalment, mortgage, financial planning etc.
  \item Grade: Secondary O-level 3/4
\end{itemize}

\textbf{Solution}

(a) To find the total quarterly expenses, we add the three monthly expense matrices:

Total quarterly expenses = January + February + March
$= \begin{pmatrix} 120 & 450 & 200 \end{pmatrix} + \begin{pmatrix} 135 & 480 & 180 \end{pmatrix} + \begin{pmatrix} 110 & 520 & 220 \end{pmatrix}$
$= \begin{pmatrix} 120+135+110 & 450+480+520 & 200+180+220 \end{pmatrix}$
$= \begin{pmatrix} 365 & 1450 & 600 \end{pmatrix}$

(b) To find the difference between actual expenses and budget, we subtract:

Difference = Actual expenses - Budget
$= \begin{pmatrix} 365 & 1450 & 600 \end{pmatrix} - \begin{pmatrix} 400 & 1500 & 650 \end{pmatrix}$
$= \begin{pmatrix} 365-400 & 1450-1500 & 600-650 \end{pmatrix}$
$= \begin{pmatrix} -35 & -50 & -50 \end{pmatrix}$

This means they spent \textdollar35 less on utilities, \textdollar50 less on groceries, and \textdollar50 less on transportation than budgeted.

(c) To project next quarter's expenses, we multiply the total quarterly expenses by the projection matrix:

Next quarter projection = $\begin{pmatrix} 365 & 1450 & 600 \end{pmatrix} \times \begin{pmatrix} 1.05 \\ 1.08 \\ 1.03 \end{pmatrix}$
$= (365 \times 1.05) + (1450 \times 1.08) + (600 \times 1.03)$
$= 383.25 + 1566 + 618$
$= 2567.25$

Therefore, the projected total expenses for next quarter is \textdollar2567.25.

\section*{Question 62}
\textbf{Metadata}

\begin{itemize}
  \item Question ID: O3-SPFndQtl\_O3-SPFndIQR\_sonnet4\_Household Finance\_01
  \item Primary KC: STATISTICS AND PROBABILITY | Finding quartiles | finding quartiles for a set of data
  \item Secondary KC: STATISTICS AND PROBABILITY | Finding range | finding interquartile range as measures of spread for a set of data 
  \item Topic: Household finance such as income, utility bills, money, interest, savings, instalment, mortgage, financial planning etc.
  \item Grade: Secondary O-level 3/4
\end{itemize}

\textbf{Solution}

**Step 1: Arrange the data in ascending order**

The 12 monthly electricity bills in ascending order are:
\textdollar78, \textdollar82, \textdollar83, \textdollar85, \textdollar87, \textdollar88, \textdollar91, \textdollar92, \textdollar94, \textdollar96, \textdollar99, \textdollar105

**Step 2: Find the quartiles**

For $n = 12$ data points:
- Position of $Q_1 = \frac{1}{4}(n+1) = \frac{1}{4}(13) = 3.25$
- Position of $Q_2 = \frac{2}{4}(n+1) = \frac{2}{4}(13) = 6.5$
- Position of $Q_3 = \frac{3}{4}(n+1) = \frac{3}{4}(13) = 9.75$

**Finding $Q_1$:**
$Q_1$ is at position 3.25, so we interpolate between the 3rd and 4th values:
$Q_1 = 83 + 0.25(85 - 83) = 83 + 0.25(2) = 83.5$

**Finding $Q_2$:**
$Q_2$ is at position 6.5, so we interpolate between the 6th and 7th values:
$Q_2 = 88 + 0.5(91 - 88) = 88 + 0.5(3) = 89.5$

**Finding $Q_3$:**
$Q_3$ is at position 9.75, so we interpolate between the 9th and 10th values:
$Q_3 = 94 + 0.75(96 - 94) = 94 + 0.75(2) = 95.5$

**(a) Answer:** $Q_1 = \textdollar83.50$, $Q_2 = \textdollar89.50$, $Q_3 = \textdollar95.50$

**Step 3: Calculate the interquartile range**

Interquartile Range (IQR) $= Q_3 - Q_1 = 95.5 - 83.5 = 12$

**(b) Answer:** The interquartile range is \textdollar12.

**(c) Answer:** The middle 50% of electricity bills fall between \textdollar83.50 and \textdollar95.50. Therefore, Mrs. Chen should expect typical monthly electricity bills to range from \textdollar83.50 to \textdollar95.50.

\section*{Question 63}
\textbf{Metadata}

\begin{itemize}
  \item Question ID: O3-SPMulProb\_O2-SPRepPrSE\_sonnet4\_Household Finance\_01
  \item Primary KC: STATISTICS AND PROBABILITY | Multiplication | multiplication of probabilities
  \item Secondary KC: STATISTICS AND PROBABILITY | Representation and concept | probability of single events
  \item Topic: Household finance such as income, utility bills, money, interest, savings, instalment, mortgage, financial planning etc.
  \item Grade: Secondary O-level 3/4
\end{itemize}

\textbf{Solution}

(a) The probability that they pay their electricity bill on time in a specific month is given directly in the problem.

P(electricity bill paid on time) = $\frac{3}{4}$

(b) Since the two bill payments are independent events, we multiply their individual probabilities to find the probability that both events occur.

P(both bills paid on time in same month) = P(electricity on time) × P(water on time)
= $\frac{3}{4} \times \frac{4}{5}$
= $\frac{12}{20}$
= $\frac{3}{5}$

(c) The probability of paying both bills on time in any single month is $\frac{3}{5}$ (from part b). For two consecutive months, since the monthly payments are independent:

P(both bills on time for two consecutive months) = P(both bills on time in month 1) × P(both bills on time in month 2)
= $\frac{3}{5} \times \frac{3}{5}$
= $\frac{9}{25}$

Therefore:
(a) $\frac{3}{4}$
(b) $\frac{3}{5}$
(c) $\frac{9}{25}$

\end{document}
