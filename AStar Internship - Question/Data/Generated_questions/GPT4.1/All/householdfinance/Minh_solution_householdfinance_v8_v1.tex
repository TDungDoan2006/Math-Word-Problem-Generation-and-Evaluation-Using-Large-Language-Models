\documentclass{article}
\usepackage[utf8]{inputenc}
\usepackage{amsmath}
\usepackage{amsfonts}
\usepackage{amssymb}
\usepackage{graphicx}
\usepackage{hyperref}
\title{'Minh Solutions household finance v8 v1'}
\author{Tien Dung Doan}
\begin{document}
\maketitle
\section*{Question 1}
\textbf{Metadata}

\begin{itemize}
  \item Question ID: P3-WNSub4d\_P1-WNCmp\_GPT4.1\_Household Finance\_03
  \item Primary KC: WHOLE NUMBERS | Subtraction | subtracting whole numbers up to 4 digits
  \item Secondary KC: WHOLE NUMBERS | Comparison and ordering | comparing and ordering whole numbers
  \item Topic: Household finance such as income, utility bills, money, interest, savings, instalment, mortgage, financial planning etc.
  \item Grade: Primary 3
\end{itemize}

\textbf{Solution}

First, find out how much Mrs. Lim paid in total for her bills:

\[
\text{Total amount paid} = 1325 + 760 = 2085
\]

Next, subtract the total bills from Mrs. Lim's salary to find out how much she has left:

\[
\text{Mrs. Lim's money left} = 4250 - 2085 = 2165
\]

Now, compare the amount Mrs. Lim has left to the amount Mr. Wong has left (\textdollar2300):

\[
2300 - 2165 = 135
\]

Mr. Wong has \textdollar135 more left than Mrs. Lim.

\textbf{Answer:} Mrs. Lim has \textdollar2165 left. Mr. Wong has more money left than Mrs. Lim by \textdollar135.

\section*{Question 2}
\textbf{Metadata}

\begin{itemize}
  \item Question ID: P3-WNDivRmd3d\_P1-WNAdd2nd\_GPT4.1\_Household Finance\_02
  \item Primary KC: WHOLE NUMBERS | Division | dividing whole numbers up to 3 digits by 1 digit with remainder 
  \item Secondary KC: WHOLE NUMBERS | Addition | adding whole numbers
  \item Topic: Household finance such as income, utility bills, money, interest, savings, instalment, mortgage, financial planning etc.
  \item Grade: Primary 3
\end{itemize}

\textbf{Solution}

First, add Mr. Tan's total amount of money: 

\[
342 + 128 = 470
\]

Next, divide \textdollar470 equally among 5 children:

\[
470 \div 5 = 94 \text{ R } 0
\]

So, each child gets \textdollar94, and there is no money left over. 

\textbf{Answer:} Each child gets \textdollar94, and Mr. Tan has \textdollar0 left over.

\section*{Question 3}
\textbf{Metadata}

\begin{itemize}
  \item Question ID: P3-WNDivRmd3d\_P1-WNMul2nd\_GPT4.1\_Household Finance\_02
  \item Primary KC: WHOLE NUMBERS | Division | dividing whole numbers up to 3 digits by 1 digit with remainder 
  \item Secondary KC: WHOLE NUMBERS | Multiplication | multiplying whole numbers
  \item Topic: Household finance such as income, utility bills, money, interest, savings, instalment, mortgage, financial planning etc.
  \item Grade: Primary 3
\end{itemize}

\textbf{Solution}

First, find how many packs of apples Mr Lim can buy with \textdollar156:

Each pack costs \textdollar7, so
\[
156 \div 7 = 22\ \text{packs, remainder}\ 2
\]

So, Mr Lim can buy 22 packs, and he will have \textdollar2 left.

Next, he gives 2 packs to each of his 8 neighbours:
\[
8 \times 2 = 16\ \text{packs given away}
\]

Packs remaining:
\[
22 - 16 = 6\ \text{packs left}
\]

\boxed{6}

Therefore, Mr Lim has 6 packs of apples left after giving to his neighbours.

\section*{Question 4}
\textbf{Metadata}

\begin{itemize}
  \item Question ID: P3-WNDiv3d1d\_P1-WNAdd2nd\_GPT4.1\_Household Finance\_02
  \item Primary KC: WHOLE NUMBERS | Division | dividing whole numbers up to 3 digits by 1 digit
  \item Secondary KC: WHOLE NUMBERS | Addition | adding whole numbers
  \item Topic: Household finance such as income, utility bills, money, interest, savings, instalment, mortgage, financial planning etc.
  \item Grade: Primary 3
\end{itemize}

\textbf{Solution}

First, we divide \textdollar285 equally over 3 months:

\[
\frac{285}{3} = 95
\]

So, Mr. Tan saves \textdollar95 each month. After 3 months, the total amount saved is:

\[
95 \times 3 = 285
\]

He adds \textdollar50 to his savings:

\[
285 + 50 = 335
\]

\textbf{Answer}: Mr. Tan has \textdollar335 altogether after 3 months.

\section*{Question 5}
\textbf{Metadata}

\begin{itemize}
  \item Question ID: P3-WNDiv3d1d\_P1-WNSub2nd\_GPT4.1\_Household Finance\_02
  \item Primary KC: WHOLE NUMBERS | Division | dividing whole numbers up to 3 digits by 1 digit
  \item Secondary KC: WHOLE NUMBERS | Subtraction | subtracting whole numbers
  \item Topic: Household finance such as income, utility bills, money, interest, savings, instalment, mortgage, financial planning etc.
  \item Grade: Primary 3
\end{itemize}

\textbf{Solution}

First, find out how much money she uses to pay the utility bills in total by subtracting the leftover amount from the original amount: 

\[
168 - 20 = 148
\]

So, Aunt Mei pays \textdollar148 in total for her utility bills in 4 weeks. 

Next, divide the total amount paid by the number of weeks to find the amount paid each week:

\[
148 \div 4 = 37
\]

Therefore, Aunt Mei pays \textdollar37 each week for her utility bills.

\section*{Question 6}
\textbf{Metadata}

\begin{itemize}
  \item Question ID: P3-WNDiv3d1d\_P1-WNMul2nd\_GPT4.1\_Household Finance\_02
  \item Primary KC: WHOLE NUMBERS | Division | dividing whole numbers up to 3 digits by 1 digit
  \item Secondary KC: WHOLE NUMBERS | Multiplication | multiplying whole numbers
  \item Topic: Household finance such as income, utility bills, money, interest, savings, instalment, mortgage, financial planning etc.
  \item Grade: Primary 3
\end{itemize}

\textbf{Solution}

First, find the cost of one chair by dividing the total amount by the number of chairs:

\[
\text{Cost of one chair} = \frac{624}{8} = 78
\]

So, each chair costs \textdollar78.

Next, find the total cost for 6 chairs by multiplying the cost of one chair by 6:

\[
\text{Total cost for 6 chairs} = 78 \times 6 = 468
\]

Therefore, the family needs to pay \textdollar468 for 6 more chairs.

\section*{Question 7}
\textbf{Metadata}

\begin{itemize}
  \item Question ID: P3-FrAddRl12\_P2-FrCmp\_GPT4.1\_Household Finance\_02
  \item Primary KC: FRACTIONS | Addition | adding two related fractions within one whole with denominators of given fractions not exceeding 12
  \item Secondary KC: FRACTIONS | Comparison and ordering | comparing and ordering fractions
  \item Topic: Household finance such as income, utility bills, money, interest, savings, instalment, mortgage, financial planning etc.
  \item Grade: Primary 3
\end{itemize}

\textbf{Solution}

(a) The total fraction Maya spent on electricity and water bills is:

\[
\frac{1}{4} + \frac{2}{4} = \frac{3}{4}
\]

So, Maya spent $\frac{3}{4}$ of her savings in total on these two bills.

(b) We need to compare $\frac{3}{4}$ to half ($\frac{1}{2}$) of her savings.

$\frac{3}{4}$ is greater than $\frac{1}{2}$ because:

\[
\frac{3}{4} > \frac{1}{2}
\]

So, Maya spent more than half of her savings on these two bills.

\section*{Question 8}
\textbf{Metadata}

\begin{itemize}
  \item Question ID: P3-FrSubRl12\_P2-FrAdd2nd\_GPT4.1\_Household Finance\_02
  \item Primary KC: FRACTIONS | Subtraction | subtracting two related fractions within one whole with denominators of given fractions not exceeding 12
  \item Secondary KC: FRACTIONS | Addition | adding fractions
  \item Topic: Household finance such as income, utility bills, money, interest, savings, instalment, mortgage, financial planning etc.
  \item Grade: Primary 3
\end{itemize}

\textbf{Solution}

First, let's add the fractions to find the total fraction of the pocket money Sarah spent.

$\frac{5}{12} + \frac{1}{4}$

We need a common denominator. The lowest common denominator of $12$ and $4$ is $12$.

$\frac{1}{4} = \frac{1 \times 3}{4 \times 3} = \frac{3}{12}$

So,

$\frac{5}{12} + \frac{3}{12} = \frac{8}{12} = \frac{2}{3}$

Sarah spent $\frac{2}{3}$ of her pocket money in total.

Next, let's find how much more she spent on snacks than on the storybook:

$\frac{5}{12} - \frac{3}{12} = \frac{2}{12} = \frac{1}{6}$

Sarah spent $\frac{1}{6}$ more of her pocket money on snacks than on the storybook.

\section*{Question 9}
\textbf{Metadata}

\begin{itemize}
  \item Question ID: P4-WNMul4d1d\_P1-WNAdd2nd\_GPT4.1\_Household Finance\_02
  \item Primary KC: WHOLE NUMBERS | Multiplication | multiplying whole numbers up to 4 digits by 1 digit or up to 3 digits by 2 digits
  \item Secondary KC: WHOLE NUMBERS | Addition | adding whole numbers
  \item Topic: Household finance such as income, utility bills, money, interest, savings, instalment, mortgage, financial planning etc.
  \item Grade: Primary 4
\end{itemize}

\textbf{Solution}

First, we find the number of bars in the 3 boxes:

$125 \times 3 = 375$

Next, we find the number of bars in 2 bags:

$82 \times 2 = 164$

Now, we add both results to find the total number of bars:

$375 + 164 = 539$

\textbf{Answer:} Mrs Lim has a total of 539 bars of soap.

\section*{Question 10}
\textbf{Metadata}

\begin{itemize}
  \item Question ID: P4-WNDiv4d1d\_P1-WNCmp\_GPT4.1\_Household Finance\_02
  \item Primary KC: WHOLE NUMBERS | Division | dividing whole numbers up to 4 digits by 1 digit
  \item Secondary KC: WHOLE NUMBERS | Comparison and ordering | comparing and ordering whole numbers
  \item Topic: Household finance such as income, utility bills, money, interest, savings, instalment, mortgage, financial planning etc.
  \item Grade: Primary 4
\end{itemize}

\textbf{Solution}

(a) To find out how much Mrs Tan saves each week, divide her monthly salary by 4:\\

$\displaystyle \frac{3276}{4} = 819$\\

So, Mrs Tan saves $\textdollar819$ each week.\\

(b) After saving for 4 weeks, Mrs Tan's total savings are:\\

$819 \times 4 = 3276$\\

Her daughter's total savings are $\textdollar800$.\\

Comparing the two savings, Mrs Tan has saved:\\

$3276 - 800 = 2476$\\

Mrs Tan has saved more money than her daughter by $\textdollar2476$.

\section*{Question 11}
\textbf{Metadata}

\begin{itemize}
  \item Question ID: P4-WNDiv4d1d\_P1-WNMul2nd\_GPT4.1\_Household Finance\_02
  \item Primary KC: WHOLE NUMBERS | Division | dividing whole numbers up to 4 digits by 1 digit
  \item Secondary KC: WHOLE NUMBERS | Multiplication | multiplying whole numbers
  \item Topic: Household finance such as income, utility bills, money, interest, savings, instalment, mortgage, financial planning etc.
  \item Grade: Primary 4
\end{itemize}

\textbf{Solution}

First, find the total cost of the 4 packs of rice: 

$4 \times 12 = 48$.

Mrs Lim spends \textdollar48 on rice in total.

Next, divide this equally among herself and 3 neighbours (4 people): 

$48 \div 4 = 12$.

Each neighbour needs to pay \textdollar12.

\section*{Question 12}
\textbf{Metadata}

\begin{itemize}
  \item Question ID: P4-FrAddU12\_P3-FrSmp\_GPT4.1\_Household Finance\_02
  \item Primary KC: FRACTIONS | Addition | adding unlike fractions with two different denominators not exceeding 12
  \item Secondary KC: FRACTIONS | Simplifying | expressing a fraction in its simplest form
  \item Topic: Household finance such as income, utility bills, money, interest, savings, instalment, mortgage, financial planning etc.
  \item Grade: Primary 4
\end{itemize}

\textbf{Solution}

To find the total fraction of her expenses spent on both groceries and utility bills, we add the two fractions:

\[
\frac{5}{12} + \frac{1}{6}
\]

First, we need a common denominator. The denominators are $12$ and $6$. The least common denominator is $12$.

Express $\frac{1}{6}$ as a fraction with denominator $12$:
\[
\frac{1}{6} = \frac{1\times2}{6\times2} = \frac{2}{12}
\]

Now add:
\[
\frac{5}{12} + \frac{2}{12} = \frac{7}{12}
\]

$\boxed{\frac{7}{12}}$ of Mrs Lim's expenses were spent on groceries and utility bills together.

\section*{Question 13}
\textbf{Metadata}

\begin{itemize}
  \item Question ID: P4-FrSubU12\_P2-FrCmp\_GPT4.1\_Household Finance\_02
  \item Primary KC: FRACTIONS | Subtraction | subtracting unlike fractions with two different denominators not exceeding 12
  \item Secondary KC: FRACTIONS | Comparison and ordering | comparing and ordering fractions
  \item Topic: Household finance such as income, utility bills, money, interest, savings, instalment, mortgage, financial planning etc.
  \item Grade: Primary 4
\end{itemize}

\textbf{Solution}

(a) Fraction of salary left after paying bills:

$\displaystyle \frac{2}{3} - \frac{5}{12}$

Find a common denominator. The lowest common denominator for $3$ and $12$ is $12$.

$\frac{2}{3} = \frac{8}{12}$

So,

$\frac{8}{12} - \frac{5}{12} = \frac{3}{12} = \frac{1}{4}$

Fraction of her salary left for other expenses is $\boxed{\frac{1}{4}}$.

(b) Compare $\frac{2}{3}$ (amount saved for expenses) and $\frac{5}{12}$ (amount for bills):

Express both with denominator $12$:

$\frac{2}{3} = \frac{8}{12}$

$\frac{8}{12}$ (saved for expenses) is greater than $\frac{5}{12}$ (spent on bills).

To find the difference:

$\frac{8}{12} - \frac{5}{12} = \frac{3}{12} = \frac{1}{4}$

So, Mrs Tan saved a greater amount for household expenses by $\boxed{\frac{1}{4}}$ of her monthly salary compared to the amount spent on bills.

\section*{Question 14}
\textbf{Metadata}

\begin{itemize}
  \item Question ID: P4-DcAdd2d\_P4-DcCnv2Fr\_GPT4.1\_Household Finance\_02
  \item Primary KC: DECIMALS | Addition | adding decimals (up to 2 decimal places)
  \item Secondary KC: DECIMALS | Conversion from decimals to fraction | expressing decimals as fractions
  \item Topic: Household finance such as income, utility bills, money, interest, savings, instalment, mortgage, financial planning etc.
  \item Grade: Primary 4
\end{itemize}

\textbf{Solution}

(a) Total amount spent:

$ \textdollar3.40 + \textdollar1.25 + \textdollar0.80 = \textdollar5.45 $

So Mrs Tan spent \textdollar5.45 in total.

(b) The amount spent on the banana is $0.80$. To write this as a fraction of a dollar:

$0.80 = \frac{80}{100} = \frac{4}{5}$

So Mrs Tan spent $\frac{4}{5}$ of a dollar on the banana.

\section*{Question 15}
\textbf{Metadata}

\begin{itemize}
  \item Question ID: P4-DcAdd2d\_P4-DcRnd3d\_GPT4.1\_Household Finance\_02
  \item Primary KC: DECIMALS | Addition | adding decimals (up to 2 decimal places)
  \item Secondary KC: DECIMALS | Rounding | rounding decimals up to 3 decimal places to the nearest whole number, 1 decimal place and 2 decimal places 
  \item Topic: Household finance such as income, utility bills, money, interest, savings, instalment, mortgage, financial planning etc.
  \item Grade: Primary 4
\end{itemize}

\textbf{Solution}

First, add the expenses together:

\begin{align*}
\text{Electricity} &= \textdollar73.45 \\
\text{Water} &= \textdollar42.38 \\
\text{Groceries} &= \textdollar128.67 \\
\text{Total} &= 73.45 + 42.38 + 128.67 = 244.50
\end{align*}

So, the total household expense for June is \textdollar244.50.

Now, round \textdollar244.50 to:

\begin{itemize}
    \item The nearest dollar: Since the decimal part is 0.50, we round up to \textdollar245.
    \item 1 decimal place: \textdollar244.5
    \item 2 decimal places: \textdollar244.50
\end{itemize}

\textbf{Final Answers:}
\begin{itemize}
\item Nearest dollar: \textdollar245
\item 1 decimal place: \textdollar244.5
\item 2 decimal places: \textdollar244.50
\end{itemize}

\section*{Question 16}
\textbf{Metadata}

\begin{itemize}
  \item Question ID: P4-DcSub2d\_P4-DcAdd2nd\_GPT4.1\_Household Finance\_02
  \item Primary KC: DECIMALS | Subtraction | subtracting decimals (up to 2 decimal places)
  \item Secondary KC: DECIMALS | Addition | adding decimals
  \item Topic: Household finance such as income, utility bills, money, interest, savings, instalment, mortgage, financial planning etc.
  \item Grade: Primary 4
\end{itemize}

\textbf{Solution}

First, find the new balance after Mrs. Tan deposits \textdollar35.75:

\[ \textdollar120.40 + \textdollar35.75 = \textdollar156.15 \]

Next, subtract the amount she used to pay her electricity bill:

\[ \textdollar156.15 - \textdollar28.90 = \textdollar127.25 \]

Therefore, Mrs. Tan has \textdollar127.25 in her savings account now.

\section*{Question 17}
\textbf{Metadata}

\begin{itemize}
  \item Question ID: P4-DcDiv2d1d\_P4-DcCnv2Fr\_GPT4.1\_Household Finance\_01
  \item Primary KC: DECIMALS | Division | dividing decimals (up to 2 decimal places) by a 1-digit whole number
  \item Secondary KC: DECIMALS | Conversion from decimals to fraction | expressing decimals as fractions
  \item Topic: Household finance such as income, utility bills, money, interest, savings, instalment, mortgage, financial planning etc.
  \item Grade: Primary 4
\end{itemize}

\textbf{Solution}

First, divide the total bill by 4 to find out how much each person needs to pay:

\[
15.36 \div 4 = 3.84
\]

So, each person pays $\textdollar3.84$.

Next, express $3.84$ as a fraction:

\[
3.84 = \frac{384}{100}
\]

Simplify $\frac{384}{100}$ by dividing both the numerator and denominator by 4:

\[
\frac{384 \div 4}{100 \div 4} = \frac{96}{25}
\]

Final answers:

Each person pays $\textdollar3.84$, which is $\frac{96}{25}$ in simplest fraction form.

\section*{Question 18}
\textbf{Metadata}

\begin{itemize}
  \item Question ID: P4-DcDiv2d1d\_P4-DcSub2nd\_GPT4.1\_Household Finance\_01
  \item Primary KC: DECIMALS | Division | dividing decimals (up to 2 decimal places) by a 1-digit whole number
  \item Secondary KC: DECIMALS | Subtraction | subtracting decimals
  \item Topic: Household finance such as income, utility bills, money, interest, savings, instalment, mortgage, financial planning etc.
  \item Grade: Primary 4
\end{itemize}

\textbf{Solution}

First, divide \textdollar36.45 equally among 5 family members: 

\[
36.45 \div 5 = 7.29
\]

So, each family member receives \textdollar7.29.

Sarah spends \textdollar0.75 on a drink from her share:

\[
7.29 - 0.75 = 6.54
\]

Sarah has \textdollar6.54 left from her share after buying the drink.

\section*{Question 19}
\textbf{Metadata}

\begin{itemize}
  \item Question ID: P5-FrAddMix\_P3-FrSmp\_GPT4.1\_Household Finance\_01
  \item Primary KC: FRACTIONS | Addition | adding mixed numbers
  \item Secondary KC: FRACTIONS | Simplifying | expressing a fraction in its simplest form
  \item Topic: Household finance such as income, utility bills, money, interest, savings, instalment, mortgage, financial planning etc.
  \item Grade: Primary 5
\end{itemize}

\textbf{Solution}

To find the total amount of water used in January and February, add the two mixed numbers:

$2\dfrac{1}{4} + 3\dfrac{2}{3}$

First, convert both to improper fractions:

$2\dfrac{1}{4} = 2 + \dfrac{1}{4} = \dfrac{8}{4} + \dfrac{1}{4} = \dfrac{9}{4}$
$3\dfrac{2}{3} = 3 + \dfrac{2}{3} = \dfrac{9}{3} + \dfrac{2}{3} = \dfrac{11}{3}$

Find a common denominator for $4$ and $3$, which is $12$:

$\dfrac{9}{4} = \dfrac{9 \times 3}{4 \times 3} = \dfrac{27}{12}$
$\dfrac{11}{3} = \dfrac{11 \times 4}{3 \times 4} = \dfrac{44}{12}$

Now add them:

$\dfrac{27}{12} + \dfrac{44}{12} = \dfrac{71}{12}$

Convert $\dfrac{71}{12}$ back to a mixed number:

$71 \div 12 = 5$ remainder $11$, so $\dfrac{71}{12} = 5\dfrac{11}{12}$

The answer is already in simplest form.

\textbf{Final Answer:} Lina’s family used $5\dfrac{11}{12}$ cubic metres of water in total in January and February.

\section*{Question 20}
\textbf{Metadata}

\begin{itemize}
  \item Question ID: P5-FrSubMix\_P2-FrAdd2nd\_GPT4.1\_Household Finance\_01
  \item Primary KC: FRACTIONS | Subtraction | subtracting mixed numbers
  \item Secondary KC: FRACTIONS | Addition | adding fractions
  \item Topic: Household finance such as income, utility bills, money, interest, savings, instalment, mortgage, financial planning etc.
  \item Grade: Primary 5
\end{itemize}

\textbf{Solution}

First, subtract the discount from the electricity bill amount:

$4\dfrac{2}{5} - \dfrac{2}{5} = 4 + \dfrac{2}{5} - \dfrac{2}{5} = 4 + 0 = 4$

So, after the discount, the electricity bill is $4$ dollars.

Now, add the water bill and the discounted electricity bill:

$2\dfrac{1}{3} + 4 = 2\dfrac{1}{3} + 4 = 6\dfrac{1}{3}$

Therefore, Amy spent $\textdollar6\dfrac{1}{3}$ in total on her water and electricity bills after the discount.

\section*{Question 21}
\textbf{Metadata}

\begin{itemize}
  \item Question ID: P5-FrMulImN\_P2-FrSub2nd\_GPT4.1\_Household Finance\_01
  \item Primary KC: FRACTIONS | Multiplication | multiplying a proper/improper fraction and a whole number
  \item Secondary KC: FRACTIONS | Subtraction | subtracting fractions
  \item Topic: Household finance such as income, utility bills, money, interest, savings, instalment, mortgage, financial planning etc.
  \item Grade: Primary 5
\end{itemize}

\textbf{Solution}

Step 1: Calculate how much Mrs Lim earns in one week from selling cakes. 

She earns $\frac{2}{3}$ of \textdollar60:  
$60 \times \frac{2}{3} = 40$.  
So, she earns \textdollar40 per week.

Step 2: Find her total earnings after 4 weeks. 
$40 \times 4 = 160$
So, in 4 weeks, she earns \textdollar160.

Step 3: Calculate the amount used to pay the electricity bill.
She pays $\frac{1}{4}$ of \textdollar160:
$160 \times \frac{1}{4} = 40$
So, she pays \textdollar40 for the electricity bill.

Step 4: Subtract the amount paid for the electricity bill from her total earnings after 4 weeks:
$160 - 40 = 120$

\textbf{Answer:} Mrs Lim has \textdollar120 left after paying the bill.

\section*{Question 22}
\textbf{Metadata}

\begin{itemize}
  \item Question ID: P5-FrMulPIm\_P2-FrSub2nd\_GPT4.1\_Household Finance\_01
  \item Primary KC: FRACTIONS | Multiplication | multiplying a proper fraction and a proper/improper fractions
  \item Secondary KC: FRACTIONS | Subtraction | subtracting fractions
  \item Topic: Household finance such as income, utility bills, money, interest, savings, instalment, mortgage, financial planning etc.
  \item Grade: Primary 5
\end{itemize}

\textbf{Solution}

First, find out how much Sarah spent on household items:

\[
\text{Amount spent on household} = \frac{3}{5} \times 120 = \frac{3 \times 120}{5} = \frac{360}{5} = 72
\]

Next, find out how much was spent on the electricity bill:

\[
\text{Electricity bill} = \frac{2}{3} \times 72 = \frac{2 \times 72}{3} = \frac{144}{3} = 48
\]

Amount spent on groceries:

\[
\text{Groceries} = 72 - 48 = 24
\]

Difference between the electricity bill and groceries:

\[
48 - 24 = 24
\]

\textbf{Therefore, Sarah spent \textdollar24 more on the electricity bill than on groceries.}

\section*{Question 23}
\textbf{Metadata}

\begin{itemize}
  \item Question ID: P5-FrMulPIm\_P5-FrCnv2Dc\_GPT4.1\_Household Finance\_01
  \item Primary KC: FRACTIONS | Multiplication | multiplying a proper fraction and a proper/improper fractions
  \item Secondary KC: FRACTIONS | Conversion to decimals | expressing fractions as decimals
  \item Topic: Household finance such as income, utility bills, money, interest, savings, instalment, mortgage, financial planning etc.
  \item Grade: Primary 5
\end{itemize}

\textbf{Solution}

Let the total savings Jenny had be $\textdollar80$.

(a) To find how much Jenny paid for the electricity bill:

She spent $\frac{3}{5}$ of $\textdollar80$, so
$$\text{Amount paid for the electricity bill} = \frac{3}{5} \times 80 = 48$$
Jenny paid $\textdollar48$ for the electricity bill.

(b) To find the amount Jenny paid using cash:

She used $\frac{2}{3}$ of $\textdollar48$:
$$\text{Amount paid using cash} = \frac{2}{3} \times 48 = 32$$

To express $32$ as a decimal: $32.00$.

\textbf{Answer:}

(a) Jenny paid $\textdollar48$ for the electricity bill.

(b) The amount she paid using cash, as a decimal, is $32.00$.

\section*{Question 24}
\textbf{Metadata}

\begin{itemize}
  \item Question ID: P5-FrMulMixN\_P2-FrCmp\_GPT4.1\_Household Finance\_01
  \item Primary KC: FRACTIONS | Multiplication | multiplying a mixed number and a whole number
  \item Secondary KC: FRACTIONS | Comparison and ordering | comparing and ordering fractions
  \item Topic: Household finance such as income, utility bills, money, interest, savings, instalment, mortgage, financial planning etc.
  \item Grade: Primary 5
\end{itemize}

\textbf{Solution}

(a) Jasmine needs $2\dfrac{1}{2}$ metres of cloth.\
\
First, convert the mixed number to an improper fraction: $2\dfrac{1}{2} = \frac{5}{2}$.\
\
Total cost $= \frac{5}{2} \times 12 = 5 \times 6 = \textdollar30$.\
\
(b) Option A: Pay the full amount upfront: \textdollar30.\
Option B: Pay three equal monthly instalments of \textdollar10 each.\
Total for Option B $= 3 \times 10 = \textdollar30$.\
\
Both options result in a total payment of \textdollar30.\
\
Ordering from least to greatest:\
Option A = Option B ($\textdollar30 = \textdollar30$).\
\
Both options cost the same, so Jasmine can choose either one without paying more.

\section*{Question 25}
\textbf{Metadata}

\begin{itemize}
  \item Question ID: P5-FrMulMixN\_P3-FrSmp\_GPT4.1\_Household Finance\_01
  \item Primary KC: FRACTIONS | Multiplication | multiplying a mixed number and a whole number
  \item Secondary KC: FRACTIONS | Simplifying | expressing a fraction in its simplest form
  \item Topic: Household finance such as income, utility bills, money, interest, savings, instalment, mortgage, financial planning etc.
  \item Grade: Primary 5
\end{itemize}

\textbf{Solution}

First, we find how many loaves Mrs Tan sold on Saturday.

Number of loaves sold on Saturday $= 2\frac{1}{4} \times 8$

Convert $2\frac{1}{4}$ to an improper fraction:

$2\frac{1}{4} = \frac{9}{4}$

Now, multiply:

$\frac{9}{4} \times 8 = \frac{9 \times 8}{4} = \frac{72}{4} = 18$

So, Mrs Tan sold 18 loaves on Saturday.

Total number of loaves sold over both days $= 8 + 18 = 26$

Each loaf sells for $\textdollar3$, so total money collected:

$26 \times 3 = 78$

Therefore, Mrs Tan collected $\textdollar78$ from selling all the loaves over both days.

\section*{Question 26}
\textbf{Metadata}

\begin{itemize}
  \item Question ID: P5-FrMulMixN\_P5-FrCnv2Dc\_GPT4.1\_Household Finance\_01
  \item Primary KC: FRACTIONS | Multiplication | multiplying a mixed number and a whole number
  \item Secondary KC: FRACTIONS | Conversion to decimals | expressing fractions as decimals
  \item Topic: Household finance such as income, utility bills, money, interest, savings, instalment, mortgage, financial planning etc.
  \item Grade: Primary 5
\end{itemize}

\textbf{Solution}

(a) To find the total weight, multiply the mixed number by the whole number:

$2\dfrac{1}{2} \times 3 = \left(2 + \dfrac{1}{2}\right) \times 3 = 2 \times 3 + \dfrac{1}{2} \times 3 = 6 + \dfrac{3}{2} = 6 + 1\dfrac{1}{2} = 7\dfrac{1}{2}$ kg.

So, Mr Tan buys $7\dfrac{1}{2}$ kg of rice in a month.

(b) Converting $\dfrac{1}{2}$ to a decimal:

$\dfrac{1}{2} = 0.5$

Therefore, $7\dfrac{1}{2} = 7.5$

So, the total weight is $7.5$ kg.

\section*{Question 27}
\textbf{Metadata}

\begin{itemize}
  \item Question ID: P5-DcMul3dK\_P4-DcRnd3d\_GPT4.1\_Household Finance\_01
  \item Primary KC: DECIMALS | Multiplication | multiplying decimals (up to 3 decimal places) by 10, 100, 1000 and their multiples
  \item Secondary KC: DECIMALS | Rounding | rounding decimals up to 3 decimal places to the nearest whole number, 1 decimal place and 2 decimal places 
  \item Topic: Household finance such as income, utility bills, money, interest, savings, instalment, mortgage, financial planning etc.
  \item Grade: Primary 5
\end{itemize}

\textbf{Solution}

(a) To find out how much Kelly has to pay, multiply the number of kilowatt-hours by the price per kWh:

\[
132.768 \times 0.183 = 24.290544
\]

So, before rounding, Kelly has to pay \textdollar24.290544.

(b) Round \$24.290544 as required:

- To the nearest cent (2 decimal places):
  \[
  \textdollar24.29
  \]
- To the nearest dollar:
  \[
  \textdollar24
  \]

\textbf{Answers:}

Kelly has to pay \textdollar24.29 (rounded to the nearest cent) or \textdollar24 (rounded to the nearest dollar) for her electricity that month.

\section*{Question 28}
\textbf{Metadata}

\begin{itemize}
  \item Question ID: P5-DcDiv3dK\_P4-DcCnv2Fr\_GPT4.1\_Household Finance\_01
  \item Primary KC: DECIMALS | Division | dividing decimals (up to 3 decimal places) by 10, 100, 1000 and their multiples
  \item Secondary KC: DECIMALS | Conversion from decimals to fraction | expressing decimals as fractions
  \item Topic: Household finance such as income, utility bills, money, interest, savings, instalment, mortgage, financial planning etc.
  \item Grade: Primary 5
\end{itemize}

\textbf{Solution}

(a) To find out how much money Mrs. Tan will put in each envelope, we divide the total amount by $100$:

\[
\frac{187.250}{100} = 1.8725
\]

So, Mrs. Tan will put $\textdollar1.8725$ in each envelope.

(b) To express $1.8725$ as a fraction:

1.8725 can be written as $\frac{18725}{10000}$ (since there are 4 decimal places).

Now, simplify $\frac{18725}{10000}$:

Find the greatest common divisor (GCD) of $18725$ and $10000$, which is $25$.

\[
\frac{18725 \div 25}{10000 \div 25} = \frac{749}{400}
\]

Therefore, the amount of money in each envelope as a fraction in its simplest form is $\frac{749}{400}$.

\section*{Question 29}
\textbf{Metadata}

\begin{itemize}
  \item Question ID: P5-RtFndR\_P2-DcCnvN2D\_GPT4.1\_Household Finance\_01
  \item Primary KC: RATE | Finding rate | finding rate given total amount and number of units
  \item Secondary KC: DECIMALS | Conversion to larger units | converting a measurement from a smaller unit to a larger unit in decimal form
  \item Topic: Household finance such as income, utility bills, money, interest, savings, instalment, mortgage, financial planning etc.
  \item Grade: Primary 5
\end{itemize}

\textbf{Solution}

First, find the average water used per day in millilitres:

\[
\text{Average per day (mL)} = \frac{7250}{5} = 1450 \; \text{millilitres}
\]

Next, convert 1450 millilitres to litres:

\[
\text{Average per day (L)} = \frac{1450}{1000} = 1.45 \; \text{litres}
\]

\textbf{Answer:} Mr. Lim used 1.45 litres of water per day on average.

\section*{Question 30}
\textbf{Metadata}

\begin{itemize}
  \item Question ID: P5-RtFndR\_P2-DcCnvD2N\_GPT4.1\_Household Finance\_01
  \item Primary KC: RATE | Finding rate | finding rate given total amount and number of units
  \item Secondary KC: DECIMALS | Conversion to smaller units | converting a measurement from a larger unit in decimal form to a smaller unit
  \item Topic: Household finance such as income, utility bills, money, interest, savings, instalment, mortgage, financial planning etc.
  \item Grade: Primary 5
\end{itemize}

\textbf{Solution}

First, find the cost per kilowatt-hour: 

\[
\text{Cost per kWh} = \frac{\text{Total amount}}{\text{Total kWh used}} = \frac{\textdollar104.60}{52.3} = \textdollar2.00 \text{ per kWh}
\]

Next, convert $1$ kilowatt-hour to watt-hours:

\[1\,\text{kWh} = 1000\,\text{Wh}\]

Find the cost per watt-hour:

\[
\text{Cost per Wh} = \frac{\textdollar2.00}{1000} = \textdollar0.002 \text{ per Wh}
\]

Now, convert \textdollar0.002 to cents:

\[
\textdollar0.002 = 0.2 \text{ cents}
\]

\textbf{Answer:} The cost per watt-hour of electricity for this month is $0.2$ cents.

\section*{Question 31}
\textbf{Metadata}

\begin{itemize}
  \item Question ID: P5-RtFndT\_P2-DcCnvN2D\_GPT4.1\_Household Finance\_01
  \item Primary KC: RATE | Finding total amount | finding total amount, given rate and number of units
  \item Secondary KC: DECIMALS | Conversion to larger units | converting a measurement from a smaller unit to a larger unit in decimal form
  \item Topic: Household finance such as income, utility bills, money, interest, savings, instalment, mortgage, financial planning etc.
  \item Grade: Primary 5
\end{itemize}

\textbf{Solution}

First, convert $32500$ watt-hours to kilowatt-hours (kWh):

$32500 \div 1000 = 32.5$ kWh

Next, multiply the number of kilowatt-hours by the rate per kWh to find the total amount:

$32.5 \times 0.22 = 7.15$

Therefore, Mrs. Tan had to pay $\textdollar 7.15$ for electricity last month.

\section*{Question 32}
\textbf{Metadata}

\begin{itemize}
  \item Question ID: P5-RtFndU\_P2-DcCnvD2N\_GPT4.1\_Household Finance\_01
  \item Primary KC: RATE | Finding number of unit | finding number of units given rate and total amount
  \item Secondary KC: DECIMALS | Conversion to smaller units | converting a measurement from a larger unit in decimal form to a smaller unit
  \item Topic: Household finance such as income, utility bills, money, interest, savings, instalment, mortgage, financial planning etc.
  \item Grade: Primary 5
\end{itemize}

\textbf{Solution}

First, we need to find the total cost in dollars.\begin{align*}
\text{Total cost in dollars} &= \text{Rate} \times \text{Number of units} \\
&= 0.25 \times 36.5 \\
&= 9.125
\end{align*}
Mr. Tan spent \textdollar9.125. Now, to convert dollars to cents, multiply by $100$: \begin{align*}
\text{Total cost in cents} &= 9.125 \times 100 \\
&= 912.5
\end{align*}
Therefore, Mr. Tan spent $913$ cents (rounded to the nearest cent) on electricity in June.

\section*{Question 33}
\textbf{Metadata}

\begin{itemize}
  \item Question ID: P6-FrDivPN\_P3-FrSmp\_GPT4.1\_Household Finance\_01
  \item Primary KC: FRACTIONS | Division | dividing a proper fraction by a whole number
  \item Secondary KC: FRACTIONS | Simplifying | expressing a fraction in its simplest form
  \item Topic: Household finance such as income, utility bills, money, interest, savings, instalment, mortgage, financial planning etc.
  \item Grade: Primary 6
\end{itemize}

\textbf{Solution}

To find out how much cake each child received, we need to divide $\frac{3}{4}$ of a cake by $3$.

Let the amount each child received be $x$.

$\displaystyle x = \frac{3}{4} \div 3$

Dividing a fraction by a whole number is the same as multiplying by its reciprocal:

$\displaystyle x = \frac{3}{4} \times \frac{1}{3}$

$ = \frac{3 \times 1}{4 \times 3}$

$ = \frac{3}{12}$

Now, simplify $\frac{3}{12}$ to its simplest form:

Both numerator and denominator can be divided by $3$:

$\frac{3 \div 3}{12 \div 3} = \frac{1}{4}$

Thus, each child received $\frac{1}{4}$ of a cake.

\section*{Question 34}
\textbf{Metadata}

\begin{itemize}
  \item Question ID: P6-FrDivPP\_P2-FrCmp\_GPT4.1\_Household Finance\_01
  \item Primary KC: FRACTIONS | Division | dividing a whole number/proper fraction by a proper fraction
  \item Secondary KC: FRACTIONS | Comparison and ordering | comparing and ordering fractions
  \item Topic: Household finance such as income, utility bills, money, interest, savings, instalment, mortgage, financial planning etc.
  \item Grade: Primary 6
\end{itemize}

\textbf{Solution}

(a) To find the number of bottles Mrs Tan can fill when each bottle has $\frac{2}{3}$ of a dollar's worth of detergent:

Number of bottles $= 8 \div \frac{2}{3} = 8 \times \frac{3}{2} = 12$ bottles.

(b) In the second situation, each bottle contains $\frac{3}{4}$ of a dollar's worth of detergent.

Number of bottles $= 8 \div \frac{3}{4} = 8 \times \frac{4}{3} = \frac{32}{3} = 10\frac{2}{3}$ bottles.

Now, compare $\frac{2}{3}$ and $\frac{3}{4}$.

To compare, convert both to a common denominator:

$\frac{2}{3} = \frac{8}{12}$
$\frac{3}{4} = \frac{9}{12}$

Since $\frac{8}{12} < \frac{9}{12}$, $\frac{2}{3} < \frac{3}{4}$.

This means each bottle in the first situation contains less detergent, so she can fill more bottles in the first situation.

Therefore, Mrs Tan can fill more bottles when she uses bottles with $\frac{2}{3}$ of a dollar's worth of detergent (12 bottles), compared to $\frac{3}{4}$ of a dollar's worth of detergent per bottle ($10\frac{2}{3}$ bottles).

\section*{Question 35}
\textbf{Metadata}

\begin{itemize}
  \item Question ID: P6-FrDivPP\_P3-FrSmp\_GPT4.1\_Household Finance\_01
  \item Primary KC: FRACTIONS | Division | dividing a whole number/proper fraction by a proper fraction
  \item Secondary KC: FRACTIONS | Simplifying | expressing a fraction in its simplest form
  \item Topic: Household finance such as income, utility bills, money, interest, savings, instalment, mortgage, financial planning etc.
  \item Grade: Primary 6
\end{itemize}

\textbf{Solution}

(a) To find how many $\dfrac{3}{4}$-litre packets Mrs Lim can make with $6$ litres of detergent, divide $6$ by $\dfrac{3}{4}$:

\[
6 \div \dfrac{3}{4} = 6 \times \dfrac{4}{3}
\]

Multiplying gives:
\[
6 \times \dfrac{4}{3} = \dfrac{6 \times 4}{3} = \dfrac{24}{3} = 8
\]

So, Mrs Lim can make $8$ packets.

(b) Since $8$ is a whole number, as a fraction in simplest form it is $\boxed{8}$.

\section*{Question 36}
\textbf{Metadata}

\begin{itemize}
  \item Question ID: P6-PcFndWN\_P1-WNSub2nd\_GPT4.1\_Household Finance\_01
  \item Primary KC: PERCENTAGE | Finding the whole | finding the whole given a part and the percentage
  \item Secondary KC: WHOLE NUMBERS | Subtraction | subtracting whole numbers
  \item Topic: Household finance such as income, utility bills, money, interest, savings, instalment, mortgage, financial planning etc.
  \item Grade: Primary 6
\end{itemize}

\textbf{Solution}

Let the total utility bill for the month be $x$.\newline
Given that $\textdollar 108$ is $45\%$ less than $x$, that means $\textdollar 108$ is $55\%$ of $x$.\newline
$55\%$ of $x = \textdollar 108$\newline
$0.55x = 108$\newline
$x = \frac{108}{0.55}$\newline
$x = 196.36$\newline
\textbf{Since the amounts must be whole dollars, round to the nearest whole number:}\newline
$x \approx \textdollar 196$\newline
\textbf{So, Mr. Lim spent about $\textdollar 196$ in total on utility bills last month.}

\section*{Question 37}
\textbf{Metadata}

\begin{itemize}
  \item Question ID: P6-PcFndChg\_P1-WNSub2nd\_GPT4.1\_Household Finance\_01
  \item Primary KC: PERCENTAGE | Finding change | finding percentage increase/decrease
  \item Secondary KC: WHOLE NUMBERS | Subtraction | subtracting whole numbers
  \item Topic: Household finance such as income, utility bills, money, interest, savings, instalment, mortgage, financial planning etc.
  \item Grade: Primary 6
\end{itemize}

\textbf{Solution}

First, we find the difference in the bill amount:

\[
\text{Decrease} = \textdollar200 - \textdollar170 = \textdollar30
\]

Next, we find the percentage decrease:

\[
\text{Percentage decrease} = \frac{\text{Decrease}}{\text{Original amount}} \times 100\% = \frac{30}{200} \times 100\%
\]

\[
= 0.15 \times 100\% = 15\%
\]

The percentage decrease in Mrs Tan's electricity bill from May to June is $15\%$.

\section*{Question 38}
\textbf{Metadata}

\begin{itemize}
  \item Question ID: P6-PcFndChg\_P1-WNMul2nd\_GPT4.1\_Household Finance\_01
  \item Primary KC: PERCENTAGE | Finding change | finding percentage increase/decrease
  \item Secondary KC: WHOLE NUMBERS | Multiplication | multiplying whole numbers
  \item Topic: Household finance such as income, utility bills, money, interest, savings, instalment, mortgage, financial planning etc.
  \item Grade: Primary 6
\end{itemize}

\textbf{Solution}

First, find the percentage increase in Mrs Lee's electricity bill for this month.  

The increase is $15\%$ of \textdollar80:

$\text{Increase} = 15\% \times 80 = \frac{15}{100} \times 80 = 12$  

So, the electricity bill for this month:

$80 + 12 = 92$  

Next, find the cost of water:

Cost of water $= 4 \times 92 = 368$  

Total amount for both:

$92 + 368 = 460$  

\boxed{\textdollar460}$

Mrs Lee needs to pay \textdollar460 in total for electricity and water this month.

\section*{Question 39}
\textbf{Metadata}

\begin{itemize}
  \item Question ID: P6-RoFndDvqWN\_P1-WNAdd2nd\_GPT4.1\_Household Finance\_01
  \item Primary KC: RATIO | Finding divided quantities | dividing a given quantity in a given ratio
  \item Secondary KC: WHOLE NUMBERS | Addition | adding whole numbers
  \item Topic: Household finance such as income, utility bills, money, interest, savings, instalment, mortgage, financial planning etc.
  \item Grade: Primary 6
\end{itemize}

\textbf{Solution}

Let the three shares be $2x$, $3x$, and $5x$.

$2x + 3x + 5x = 10x = 480$

So $x = \frac{480}{10} = 48$.

Amounts:
- First child: $2x = 2 \times 48 = 96$
- Second child: $3x = 3 \times 48 = 144$
- Third child: $5x = 5 \times 48 = 240$

Therefore, the first child receives \textdollar96, the second child receives \textdollar144, and the third child receives \textdollar240.

(b) The total amount after Mrs Tan adds \textdollar120:
\[
480 + 120 = 600
\]

So, after saving \textdollar120 more, the new total amount distributed is \textdollar600.

\section*{Question 40}
\textbf{Metadata}

\begin{itemize}
  \item Question ID: P6-RoFndRoWN\_P1-WNSub2nd\_GPT4.1\_Household Finance\_01
  \item Primary KC: RATIO | Finding ratio | finding the ratio of two or three given whole numbers
  \item Secondary KC: WHOLE NUMBERS | Subtraction | subtracting whole numbers
  \item Topic: Household finance such as income, utility bills, money, interest, savings, instalment, mortgage, financial planning etc.
  \item Grade: Primary 6
\end{itemize}

\textbf{Solution}

First, find the amount left after paying for utility bills: 

\begin{align*}
\text{Amount left} &= \textdollar210 - \textdollar80 \\
&= \textdollar130
\end{align*}

Let the amount kept as emergency savings be $x$. 
Then, the amount spent on groceries is $2x$.

So, 
\begin{align*}
x + 2x &= 130 \\
3x &= 130 \\
x &= \frac{130}{3}
\end{align*}

Amount spent on groceries: $2x = 2 \times \frac{130}{3} = \frac{260}{3}$

Now, list all three amounts:

- Amount spent on utility bills: \textdollar80
- Amount spent on groceries: $\frac{260}{3}$
- Amount kept as emergency savings: $\frac{130}{3}$

To write the ratio in whole numbers, multiply all terms by 3:

\begin{align*}
\text{Ratio} &= 80 : \frac{260}{3} : \frac{130}{3} \\
&= 80 \times 3 : \frac{260}{3} \times 3 : \frac{130}{3} \times 3 \\
&= 240 : 260 : 130
\end{align*}

Simplify the ratio by dividing each term by 10:

$240 : 260 : 130 = 24 : 26 : 13$

\textbf{Answer:}

The ratio of the amount spent on utility bills, the amount spent on groceries, and the amount kept as emergency savings is $24:26:13$. 

\section*{Question 41}
\textbf{Metadata}

\begin{itemize}
  \item Question ID: P6-RoFndRoWN\_P1-WNDiv2nd\_GPT4.1\_Household Finance\_01
  \item Primary KC: RATIO | Finding ratio | finding the ratio of two or three given whole numbers
  \item Secondary KC: WHOLE NUMBERS | Division | dividing whole numbers
  \item Topic: Household finance such as income, utility bills, money, interest, savings, instalment, mortgage, financial planning etc.
  \item Grade: Primary 6
\end{itemize}

\textbf{Solution}

First, let us find the sum of the parts in the ratio: \[2 + 3 + 5 = 10\text{ parts}\]

Each part is therefore worth: \[\frac{\textdollar120}{10} = \textdollar12\]

Amount received by Amy (2 parts): \[2 \times \textdollar12 = \textdollar24\]

Amount received by Ben (3 parts): \[3 \times \textdollar12 = \textdollar36\]

Amount received by Carl (5 parts): \[5 \times \textdollar12 = \textdollar60\]

Thus, Amy receives \textdollar24, Ben receives \textdollar36, and Carl receives \textdollar60.

\section*{Question 42}
\textbf{Metadata}

\begin{itemize}
  \item Question ID: P6-RoFndTmWN\_P1-WNAdd2nd\_GPT4.1\_Household Finance\_01
  \item Primary KC: RATIO | Finding a missing term | finding the missing term in a pair of equivalent ratios
  \item Secondary KC: WHOLE NUMBERS | Addition | adding whole numbers
  \item Topic: Household finance such as income, utility bills, money, interest, savings, instalment, mortgage, financial planning etc.
  \item Grade: Primary 6
\end{itemize}

\textbf{Solution}

Let the original amount Mr Lim spends on rent and groceries be $5x$ and $2x$ respectively. 

Original rent: $5x = 1200$.

$\Rightarrow x = \frac{1200}{5} = 240$

So, original groceries: 
$2x = 2 \times 240 = 480$

But actual groceries spent: \textdollar400 (but for the ratio, we will follow the calculated $2x$ part, as this is in proportion with rent).

If rent increases by \textdollar200: 
New rent $= 1200 + 200 = 1400$

Let new groceries be $y$ such that the ratio stays $5 : 2$. 

So, $\frac{1400}{y} = \frac{5}{2}$

$\Rightarrow 5y = 2 \times 1400$

$\Rightarrow 5y = 2800$

$\Rightarrow y = \frac{2800}{5} = 560$

So, the new amount to spend on groceries is \textdollar560.

The increase in the groceries spending: $560 - 480 = 80$

\textbf{Answer:} Mr Lim would have to spend \textdollar80 more on groceries if his rent increased by \textdollar200 next month, keeping the ratio the same.

\section*{Question 43}
\textbf{Metadata}

\begin{itemize}
  \item Question ID: P6-AgRepLrEx\_P6-AgEvlLrEx\_GPT4.1\_Household Finance\_01
  \item Primary KC: ALGEBRA | Representation and concept | translation of real-world situations into linear algebraic expressions
  \item Secondary KC: ALGEBRA | Evaluation | evaluating linear expressions by substitution
  \item Topic: Household finance such as income, utility bills, money, interest, savings, instalment, mortgage, financial planning etc.
  \item Grade: Primary 6
\end{itemize}

\textbf{Solution}

(a) Sarah's total allowance can be represented by the expression:
\[
\text{Total allowance} = 120 + 3x
\]
where $x$ is the number of chores completed.

(b) When $x = 12$,
\[
\text{Total allowance} = 120 + 3 \times 12 = 120 + 36 = 156
\]
So, Sarah's total allowance in March is $\textdollar156$.

\section*{Question 44}
\textbf{Metadata}

\begin{itemize}
  \item Question ID: P6-AgSlvLrN\_P6-AgRepLrEx\_GPT4.1\_Household Finance\_01
  \item Primary KC: ALGEBRA | Solving simple linear equations | solving linear equations involving whole number coefficient and one variable only
  \item Secondary KC: ALGEBRA | Representation and concept | translation of real-world situations into linear algebraic expressions
  \item Topic: Household finance such as income, utility bills, money, interest, savings, instalment, mortgage, financial planning etc.
  \item Grade: Primary 6
\end{itemize}

\textbf{Solution}

Let $x$ be the internet subscription fee in dollars.

The electricity bill = $x + 30$.

Total monthly bills: $x + (x + 30) = 120$

Simplify:
$2x + 30 = 120$

Subtract 30 from both sides:
$2x = 120 - 30$
$2x = 90$

Divide both sides by 2:
$x = \frac{90}{2}$
$x = 45$

Mr Tan pays $\textdollar45$ for his internet subscription each month.

\section*{Question 45}
\textbf{Metadata}

\begin{itemize}
  \item Question ID: O1-RoRepFr\_P2-FrSub2nd\_GPT4.1\_Household Finance\_01
  \item Primary KC: RATIO | Representation and concept | ratios involving fractions
  \item Secondary KC: FRACTIONS | Subtraction | subtracting fractions
  \item Topic: Household finance such as income, utility bills, money, interest, savings, instalment, mortgage, financial planning etc.
  \item Grade: Secondary O-level 1
\end{itemize}

\textbf{Solution}

Let Aisha's allowance be $a$ and her brother's allowance be $b$. 

Aisha saved $\frac{3}{5}a$, and her brother saved $\frac{1}{2}b$.

The ratio given is $\frac{3}{5}a : \frac{1}{2}b = \frac{5}{6} : 1$.

Express as a linear equation:

$\frac{3}{5}a = \frac{5}{6} \times \frac{1}{2}b$

$\frac{3}{5}a = \frac{5}{12}b$

Now solve for $a$ in terms of $b$:

$a = \frac{5}{3} \times \frac{5}{12}b = \frac{25}{36}b$

Now, Aisha spends $\frac{1}{4}$ of her savings:

Amount spent on book: $\frac{1}{4} \times \frac{3}{5}a = \frac{3}{20}a$

Amount Aisha contributes to the bill:
$\frac{3}{5}a - \frac{3}{20}a = (\frac{12}{20}a - \frac{3}{20}a) = \frac{9}{20}a$

Fraction of her original allowance:

$\frac{9}{20}a \div a = \frac{9}{20}$

\textbf{Answer:} Aisha contributes $\boxed{\frac{9}{20}}$ of her original allowance to the utility bill.

\section*{Question 46}
\textbf{Metadata}

\begin{itemize}
  \item Question ID: O1-RoRepFr\_P6-FrDiv2nd\_GPT4.1\_Household Finance\_01
  \item Primary KC: RATIO | Representation and concept | ratios involving fractions
  \item Secondary KC: FRACTIONS | Division | fraction division
  \item Topic: Household finance such as income, utility bills, money, interest, savings, instalment, mortgage, financial planning etc.
  \item Grade: Secondary O-level 1
\end{itemize}

\textbf{Solution}

(a) Let the fraction used by Charles be $c$.\\
$\frac{1}{4} + \frac{1}{3} + c = 1$\\
$\frac{3}{12} + \frac{4}{12} + c = 1$\\
$\frac{7}{12} + c = 1$\\
$c = 1 - \frac{7}{12} = \frac{5}{12}$\\
\\
(b) The ratio of their usage is $\frac{1}{4} : \frac{1}{3} : \frac{5}{12}$.\\
Convert all to twelfths: $\frac{3}{12} : \frac{4}{12} : \frac{5}{12}$\\
So the ratio is $3:4:5$.\\
\\
(c) Charles's share of the bill is $\frac{5}{12} \div 1 \times 180 = \frac{5}{12} \times 180 = 5 \times 15 = 75$\\
So, Charles must pay \textdollar75.

\section*{Question 47}
\textbf{Metadata}

\begin{itemize}
  \item Question ID: O1-RoRepDc\_P4-DcSub2nd\_GPT4.1\_Household Finance\_01
  \item Primary KC: RATIO | Representation and concept | ratios involving decimals
  \item Secondary KC: DECIMALS | Subtraction | subtracting decimals
  \item Topic: Household finance such as income, utility bills, money, interest, savings, instalment, mortgage, financial planning etc.
  \item Grade: Secondary O-level 1
\end{itemize}

\textbf{Solution}

Let the amount spent on utilities be $x$.

Given the ratio of utilities to savings is $0.6 : 1$ and the savings amount is \textdollar800,

$\dfrac{x}{800} = 0.6$

$\Rightarrow x = 0.6 \times 800 = 480$

So, the amount spent on utilities is \textdollar480.

Total spent on utilities and savings $= 480 + 800 = 1280$

Amount left after paying for utilities and savings:

$1850 - 1280 = 570$

Therefore:

- Amount spent on utilities is \textdollar480.
- Amount left after paying for utilities and savings is \textdollar570.

\section*{Question 48}
\textbf{Metadata}

\begin{itemize}
  \item Question ID: O1-PcRep2q\_O1-PcCnv2Dc\_GPT4.1\_Household Finance\_02
  \item Primary KC: PERCENTAGE | Representation and concept | comparing two quantities by percentage
  \item Secondary KC: PERCENTAGE | Conversion to decimals | expressing percentage as a decimal
  \item Topic: Household finance such as income, utility bills, money, interest, savings, instalment, mortgage, financial planning etc.
  \item Grade: Secondary O-level 1
\end{itemize}

\textbf{Solution}

(a) The difference between the water and electricity bill is $400 - 320 = 80$.

To find the percentage increase from the electricity bill:

\[
\text{Percentage increase} = \frac{\text{Difference}}{\text{Electricity Bill}} \times 100\% = \frac{80}{320} \times 100\% = 25.0\%
\]

So, the water bill is $25.0\%$ higher than the electricity bill.

(b) To express $25.0\%$ as a decimal:

\[
25.0\% = \frac{25.0}{100} = 0.250
\]

So, the decimal is $0.250$, correct to 3 decimal places.

\section*{Question 49}
\textbf{Metadata}

\begin{itemize}
  \item Question ID: O1-PcFndRslt\_P1-WNDiv2nd\_GPT4.1\_Household Finance\_01
  \item Primary KC: PERCENTAGE | Finding result after change | increasing/decreasing a quantity by a given percentage
  \item Secondary KC: WHOLE NUMBERS | Division | dividing whole numbers
  \item Topic: Household finance such as income, utility bills, money, interest, savings, instalment, mortgage, financial planning etc.
  \item Grade: Secondary O-level 1
\end{itemize}

\textbf{Solution}

First, we need to find the new electricity bill after a $15\%$ increase.

Amount of increase $= 15\%$ of $180 = \frac{15}{100} \times 180 = 27$

New bill $= 180 + 27 = \textdollar207$

Next, divide the new bill by 4 to find the amount each family member has to pay:

Amount per person $= \frac{207}{4} = \textdollar51.75$

Each person has to pay \textdollar51.75 this month.

\section*{Question 50}
\textbf{Metadata}

\begin{itemize}
  \item Question ID: O1-AgSlvFrLr\_O1-AgRepEq\_GPT4.1\_Household Finance\_02
  \item Primary KC: ALGEBRA | Solving | solving simple fractional equations that can be reduced to linear equations
  \item Secondary KC: ALGEBRA | Representation and concept | translation of simple real-world situations to equations
  \item Topic: Household finance such as income, utility bills, money, interest, savings, instalment, mortgage, financial planning etc.
  \item Grade: Secondary O-level 1
\end{itemize}

\textbf{Solution}

(a) Let the water bill be $x$.

The electricity bill is $\frac{3}{4}x$.

The total paid is $x + \frac{3}{4}x = \textdollar140$.

(b)
\begin{align*}
x + \frac{3}{4}x &= 140 \\
\frac{7}{4}x &= 140 \\
x &= 140 \times \frac{4}{7} \\
x &= 80
\end{align*}

So, James paid $\textdollar80$ for his water bill.

The electricity bill is $\frac{3}{4} \times 80 = 60$.

James paid $\textdollar60$ for his electricity bill.

\section*{Question 51}
\textbf{Metadata}

\begin{itemize}
  \item Question ID: O2-RoRepIvP\_P1-WNMul2nd\_GPT4.1\_Household Finance\_01
  \item Primary KC: RATIO | Representation and concept | inverse proportion
  \item Secondary KC: WHOLE NUMBERS | Multiplication | multiplying whole numbers
  \item Topic: Household finance such as income, utility bills, money, interest, savings, instalment, mortgage, financial planning etc.
  \item Grade: Secondary O-level 2
\end{itemize}

\textbf{Solution}

Let $x$ represent the total cost of the bulbs. When 4 members share the cost: 

\[
\frac{x}{4} = 18
\]

Multiplying both sides by 4:
\[
x = 4 \times 18 = 72
\]

So, the total cost of the bulbs is \textdollar72.

If only 3 members are sharing the total cost:
\[
\text{Amount each pays} = \frac{72}{3} = 24
\]

Each member must pay \textdollar24.

\textbf{Answers:} 
1. Each member pays \textdollar24 if 3 share the cost. 
2. The total cost of the bulbs is \textdollar72.

\section*{Question 52}
\textbf{Metadata}

\begin{itemize}
  \item Question ID: O2-RoRepIvP\_P1-WNDiv2nd\_GPT4.1\_Household Finance\_01
  \item Primary KC: RATIO | Representation and concept | inverse proportion
  \item Secondary KC: WHOLE NUMBERS | Division | dividing whole numbers
  \item Topic: Household finance such as income, utility bills, money, interest, savings, instalment, mortgage, financial planning etc.
  \item Grade: Secondary O-level 2
\end{itemize}

\textbf{Solution}

(a) Let $d$ be the number of days and $n$ be the number of family members. Since $d$ is inversely proportional to $n$, we have $d \propto \frac{1}{n}$ or $d = \frac{k}{n}$ for some constant $k$.

When $n=4$, $d=12$:
$$
12 = \frac{k}{4}
$$
$$
k = 12 \times 4 = 48
$$

When $n=6$:
$$
d = \frac{48}{6} = 8
$$
So, the groceries will last for $8$ days if 6 family members share them.

(b) The total cost of groceries is \textdollar1200. When divided among 6 members:
$$
\text{Cost per person} = \frac{\textdollar1200}{6} = \textdollar200
$$

Each of the 6 family members will have to pay \textdollar200.

\section*{Question 53}
\textbf{Metadata}

\begin{itemize}
  \item Question ID: O2-AgSlvLr2v\_O1-AgRepEq\_GPT4.1\_Household Finance\_01
  \item Primary KC: ALGEBRA | Solving | solving linear equations in two variables
  \item Secondary KC: ALGEBRA | Representation and concept | translation of simple real-world situations to equations
  \item Topic: Household finance such as income, utility bills, money, interest, savings, instalment, mortgage, financial planning etc.
  \item Grade: Secondary O-level 2
\end{itemize}

\textbf{Solution}

(a) From the problem, we have:

$\begin{align*}
x + y &= 180\\
x &= 2y + 20
\end{align*}$

(b) Substitute $x$ from the second equation into the first:

$2y + 20 + y = 180$

$3y + 20 = 180$

$3y = 180 - 20$

$3y = 160$

$y = \frac{160}{3} = 53.33$

Now substitute $y$ into the second equation to find $x$:

$x = 2 \times 53.33 + 20 = 106.66 + 20 = 126.66$

Therefore, the electricity bill was \textdollar126.66 and the water bill was \textdollar53.33 (rounded to the nearest cent).

\section*{Question 54}
\textbf{Metadata}

\begin{itemize}
  \item Question ID: O2-SPFndmdn\_O2-SPFndmode\_GPT4.1\_Household Finance\_01
  \item Primary KC: STATISTICS AND PROBABILITY | Finding median | Finding median for a set of data
  \item Secondary KC: STATISTICS AND PROBABILITY | Finding mode | Finding mode for a set of data
  \item Topic: Household finance such as income, utility bills, money, interest, savings, instalment, mortgage, financial planning etc.
  \item Grade: Secondary O-level 2
\end{itemize}

\textbf{Solution}

(a) To find the median, first arrange the amounts in ascending order:

\[\textdollar60, \textdollar60, \textdollar60, \textdollar75, \textdollar75, \textdollar80, \textdollar90\]

There are 7 data values, so the median is the 4th value.

The median amount is $\textdollar75$.

(b) The mode is the value that appears most frequently. $\textdollar60$ occurs 3 times, more than any other value.

The mode of the amounts spent is $\textdollar60$.

\section*{Question 55}
\textbf{Metadata}

\begin{itemize}
  \item Question ID: O2-SPFndmdn\_O3-SPFndPctl\_GPT4.1\_Household Finance\_01
  \item Primary KC: STATISTICS AND PROBABILITY | Finding median | Finding median for a set of data
  \item Secondary KC: STATISTICS AND PROBABILITY | Finding percentiles | finding percentiles for a set of data
  \item Topic: Household finance such as income, utility bills, money, interest, savings, instalment, mortgage, financial planning etc.
  \item Grade: Secondary O-level 2
\end{itemize}

\textbf{Solution}

(a) To find the median, first arrange the bills in ascending order: 

$85, 86, 87, 88, 89, 90, 91, 92, 93, 95$

There are 10 values (even number), so the median is the average of the 5th and 6th values:

Median $= \frac{89+90}{2} = \frac{179}{2} = 89.5$

So, the median monthly electricity bill is \textdollar89.5.

(b) To find the 70th percentile:

Use the formula for the rank position: $P = \frac{70}{100} \times (n + 1) = 0.7 \times (10 + 1) = 0.7 \times 11 = 7.7$

So, the 70th percentile is at the 7.7th value between the 7th and 8th data points in the ordered list.

The 7th value is 91, the 8th value is 92.

70th percentile $= 91 + 0.7 \times (92 - 91) = 91 + 0.7 = 91.7$

Thus, the 70th percentile of the monthly electricity bills is \textdollar91.7.

\section*{Question 56}
\textbf{Metadata}

\begin{itemize}
  \item Question ID: O3-BPOpr\_O3-BPRepPosI\_GPT4.1\_Household Finance\_01
  \item Primary KC: BASE AND POWER | Operations | laws of indices
  \item Secondary KC: BASE AND POWER | Representation and concept  | positive indices that is not 1
  \item Topic: Household finance such as income, utility bills, money, interest, savings, instalment, mortgage, financial planning etc.
  \item Grade: Secondary O-level 3/4
\end{itemize}

\textbf{Solution}

(a) The monthly interest rate is $2^2\% = 4\%$.

(b) Amount to be paid through instalments: $\textdollar1200 - \textdollar300 = \textdollar900$.

Let each instalment be $x$.

First month's payment: $\dfrac{900}{3} = \textdollar300$.
Second month's payment with one month's interest: $300 \times (1 + \frac{4}{100}) = 300 \times 1.04 = \textdollar312$.
Third month's payment with two months of interest:
Original sum for third instalment: $\textdollar300$.
After 2 months' interest: $300 \times (1.04)^2 = 300 \times 1.0816 = \textdollar324.48$

Total paid = Down payment + All instalments with interest:
$= \textdollar300 + \textdollar300 + \textdollar312 + \textdollar324.48$

$= \textdollar300 + \textdollar300 + \textdollar312 + \textdollar324.48$

$= \textdollar1236.48$

So in terms of indices, third payment is $300 \times (1.04)^2$.

(c) Option B is $\textdollar1236.48 - \textdollar1200 = \textdollar36.48$ more expensive than Option A.

\section*{Question 57}
\textbf{Metadata}

\begin{itemize}
  \item Question ID: O3-BPOpr\_O3-BPRepFrI\_GPT4.1\_Household Finance\_01
  \item Primary KC: BASE AND POWER | Operations | laws of indices
  \item Secondary KC: BASE AND POWER | Representation and concept  | fractional indices
  \item Topic: Household finance such as income, utility bills, money, interest, savings, instalment, mortgage, financial planning etc.
  \item Grade: Secondary O-level 3/4
\end{itemize}

\textbf{Solution}

(a) Let's first simplify $A$:\\
\\
$A = 5000 \times 2^{\frac{3}{2}}$\\
Recall that $a^{\frac{m}{n}} = (a^{m})^{\frac{1}{n}} = (a^{\frac{1}{n}})^m$.\\
So $2^{\frac{3}{2}} = (2^3)^{1/2} = 8^{1/2} = \sqrt{8}$.\\
$\sqrt{8} = \sqrt{4 \times 2} = 2 \sqrt{2}$.\\
Thus, $A = 5000 \times 2 \sqrt{2} = 10000 \sqrt{2}$.\\
\\
Now, let us simplify $B$:\\
\\
$B = 5000 \times (8)^{\frac{1}{3}}$\\
$8^{\frac{1}{3}}$ is the cube root of $8$, which is $2$.\\
Thus, $B = 5000 \times 2 = 10000$.\\
\\
(b) Comparing $A$ and $B$:\\
$A = 10000 \sqrt{2}$ and $B = 10000$.\\
$\sqrt{2} \approx 1.414$.\\
So, $A \approx 10000 \times 1.414 = 14140$.\\
$B = 10000$.\\
\\
Therefore, Plan A will give Mr. Tan a greater amount in his emergency fund after this period.

\section*{Question 58}
\textbf{Metadata}

\begin{itemize}
  \item Question ID: O3-STOprUn\_O3-STOprIns\_GPT4.1\_Household Finance\_01
  \item Primary KC: SET | Set operations | union of two sets
  \item Secondary KC: SET | Set operations | intersection of two sets
  \item Topic: Household finance such as income, utility bills, money, interest, savings, instalment, mortgage, financial planning etc.
  \item Grade: Secondary O-level 3/4
\end{itemize}

\textbf{Solution}

(a) Let $E$ be the set of households that paid for electricity bills using online banking, and $W$ be the set of households that paid for water bills using online banking.

Given:
$|E| = 32$
$|W| = 20$
$|E \cap W| = 12$

The number of households that paid for either electricity or water bills (i.e. union) is:
\[
|E \cup W| = |E| + |W| - |E \cap W| = 32 + 20 - 12 = 40
\]

(b) The total number of households is 50. The number of households that paid for neither is:
\[
50 - |E \cup W| = 50 - 40 = 10
\]

\textbf{Answers:}

(a) $40$ households paid for either electricity or water bills using online banking.

(b) $10$ households paid for neither electricity nor water bills using online banking.

\section*{Question 59}
\textbf{Metadata}

\begin{itemize}
  \item Question ID: O3-MXMulSM\_O3-MXSub\_GPT4.1\_Household Finance\_01
  \item Primary KC: MATRICES | Multiplication | product of a scalar quantity and a matrix
  \item Secondary KC: MATRICES | Subtraction | subtraction of matrices
  \item Topic: Household finance such as income, utility bills, money, interest, savings, instalment, mortgage, financial planning etc.
  \item Grade: Secondary O-level 3/4
\end{itemize}

\textbf{Solution}

(a) February expenses increased by 10\%, so multiply January's matrix by $1.10$: \
\begin{align*}
A &= \begin{bmatrix} 1200 \\ 200 \\ 400 \end{bmatrix} \\
\text{February (before discount)} &= 1.10 \times \begin{bmatrix} 1200 \\ 200 \\ 400 \end{bmatrix} = \begin{bmatrix} 1.10 \times 1200 \\ 1.10 \times 200 \\ 1.10 \times 400 \end{bmatrix} = \begin{bmatrix} 1320 \\ 220 \\ 440 \end{bmatrix}
\end{align*}

(b) Applying the discounts to utilities and groceries:
\begin{align*}
B &= \begin{bmatrix}
1320 \\ 220 - 30 \\ 440 - 50
\end{bmatrix} = \begin{bmatrix} 1320 \\ 190 \\ 390 \end{bmatrix}
\end{align*}

(c) The change in expenses from January to discounted February: 
\begin{align*}
B - A &= \begin{bmatrix} 1320 \\ 190 \\ 390 \end{bmatrix} - \begin{bmatrix} 1200 \\ 200 \\ 400 \end{bmatrix} = \begin{bmatrix} 1320-1200 \\ 190-200 \\ 390-400 \end{bmatrix} = \begin{bmatrix} 120 \\ -10 \\ -10 \end{bmatrix}
\end{align*}

This means, compared to January, Jenny paid \textdollar120 more for rent, \textdollar10 less for utilities, and \textdollar10 less for groceries in February after applying the discounts.

\section*{Question 60}
\textbf{Metadata}

\begin{itemize}
  \item Question ID: O3-MXMulSM\_O3-MXMul\_GPT4.1\_Household Finance\_01
  \item Primary KC: MATRICES | Multiplication | product of a scalar quantity and a matrix
  \item Secondary KC: MATRICES | Multiplication | multiplication of matrices
  \item Topic: Household finance such as income, utility bills, money, interest, savings, instalment, mortgage, financial planning etc.
  \item Grade: Secondary O-level 3/4
\end{itemize}

\textbf{Solution}

(a) The new rates per unit matrix, after applying a surcharge of $1.5$ to the electricity rate, is:
\[
\begin{aligned}
\text{Original rate for electricity} &= 0.20 \\
\text{New rate for electricity} &= 1.5 \times 0.20 = 0.30
\end{aligned}
\]
So, the new rates matrix $B' = \begin{bmatrix} 0.30 & 0.30 & 0.50 \end{bmatrix}$.

(b) The total monthly cost $C$ can be found by multiplying the new rates matrix $B'$ by the usage matrix $A$:
\[
C = B' \times A = \begin{bmatrix} 0.30 & 0.30 & 0.50 \end{bmatrix} \begin{bmatrix} 120 \\ 45 \\ 30 \end{bmatrix}
\]
Calculate:
\[
C = (0.30 \times 120) + (0.30 \times 45) + (0.50 \times 30) \\
= 36 + 13.5 + 15 \\
= 64.5
\]
Therefore, the total monthly cost for the three utilities during the summer month is \textdollar64.5.

\section*{Question 61}
\textbf{Metadata}

\begin{itemize}
  \item Question ID: O3-MXMul\_O3-MXSub\_GPT4.1\_Household Finance\_01
  \item Primary KC: MATRICES | Multiplication | multiplication of matrices
  \item Secondary KC: MATRICES | Subtraction | subtraction of matrices
  \item Topic: Household finance such as income, utility bills, money, interest, savings, instalment, mortgage, financial planning etc.
  \item Grade: Secondary O-level 3/4
\end{itemize}

\textbf{Solution}

First, perform the matrix subtraction to find $C = B - A$: 

$C = \begin{bmatrix} 370 - 350 & 150 - 120 \\ 110 - 100 & 90 - 80 \\ 140 - 130 & 70 - 60 \end{bmatrix} = \begin{bmatrix} 20 & 30 \\ 10 & 10 \\ 10 & 10 \end{bmatrix}$

$C$ shows the change in expenses from January to February for both father (first column) and mother (second column) in each category.

To find the combined change in expenses for each category, add the changes for the father and mother in each row (category):

Let row vector $v = \begin{bmatrix} 1 & 1 \end{bmatrix}$. The change in combined expenses for each category is $C \times v^T$:

$\begin{bmatrix} 20 & 30 \\ 10 & 10 \\ 10 & 10 \end{bmatrix} \times \begin{bmatrix} 1 \\ 1 \end{bmatrix} = \begin{bmatrix} 20 \times 1 + 30 \times 1 \\ 10 \times 1 + 10 \times 1 \\ 10 \times 1 + 10 \times 1 \end{bmatrix} = \begin{bmatrix} 50 \\ 20 \\ 20 \end{bmatrix}$

So, the resulting row vector showing the change in combined expenses for:
- Food: \textdollar50
- Utilities: \textdollar20
- Transportation: \textdollar20

Thus, matrix $C$ is $\begin{bmatrix} 20 & 30 \\ 10 & 10 \\ 10 & 10 \end{bmatrix}$, and the change in combined expenses for each category is $\begin{bmatrix} 50 & 20 & 20 \end{bmatrix}^T$ (or for clarity, Food: \textdollar50, Utilities: \textdollar20, Transportation: \textdollar20).

\section*{Question 62}
\textbf{Metadata}

\begin{itemize}
  \item Question ID: O3-SPFndQtl\_O3-SPFndIQR\_GPT4.1\_Household Finance\_01
  \item Primary KC: STATISTICS AND PROBABILITY | Finding quartiles | finding quartiles for a set of data
  \item Secondary KC: STATISTICS AND PROBABILITY | Finding range | finding interquartile range as measures of spread for a set of data 
  \item Topic: Household finance such as income, utility bills, money, interest, savings, instalment, mortgage, financial planning etc.
  \item Grade: Secondary O-level 3/4
\end{itemize}

\textbf{Solution}

(a) First, order the bills in ascending order: $67,\ 74,\ 76,\ 78,\ 82,\ 86,\ 89,\ 92,\ 95,\ 100$.\
\
There are $10$ values.\
\
Median ($Q_2$):\
Positions 5 and 6: $82$ and $86$; $Q_2 = \dfrac{82 + 86}{2} = 84$.\
\
Lower quartile ($Q_1$): The median of the lower half (1st to 5th value): $67, 74, 76, 78, 82$\
The median here is $76$. So $Q_1 = 76$.\
\
Upper quartile ($Q_3$): The median of the upper half (6th to 10th value): $86, 89, 92, 95, 100$\
The median here is $92$. So $Q_3 = 92$.\
\
(b) The interquartile range (IQR) is: \
$IQR = Q_3 - Q_1 = 92 - 76 = 16$.\
\
\textbf{Final answers:}\\
$Q_1 = 76$\textdollar,\ $Q_2 = 84$\textdollar,\ $Q_3 = 92$\textdollar.\\
$IQR = 16$\textdollar.

\section*{Question 63}
\textbf{Metadata}

\begin{itemize}
  \item Question ID: O3-SPMulProb\_O2-SPRepPrSE\_GPT4.1\_Household Finance\_01
  \item Primary KC: STATISTICS AND PROBABILITY | Multiplication | multiplication of probabilities
  \item Secondary KC: STATISTICS AND PROBABILITY | Representation and concept | probability of single events
  \item Topic: Household finance such as income, utility bills, money, interest, savings, instalment, mortgage, financial planning etc.
  \item Grade: Secondary O-level 3/4
\end{itemize}

\textbf{Solution}

(a) Let $A$ be the event that Maya receives an allowance, and $S$ be the event that she saves part of it. 

We are given:
$$P(A) = 0.7$$
$$P(S\,|\,A) = 0.8$$

The probability that Maya both receives an allowance and saves part of it is:
$$P(A \cap S) = P(A) \times P(S | A) = 0.7 \times 0.8 = 0.56$$

So, the answer is $0.56$.

(b) If Maya does not receive an allowance, she does not have any money to save. Therefore, the probability that she saves part of it is $0$.

So, the answer is $0$.

\end{document}
