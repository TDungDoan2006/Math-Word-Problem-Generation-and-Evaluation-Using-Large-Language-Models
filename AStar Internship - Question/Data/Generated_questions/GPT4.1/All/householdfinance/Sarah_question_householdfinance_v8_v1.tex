\documentclass{article}
\usepackage[utf8]{inputenc}
\usepackage{amsmath}
\usepackage{amsfonts}
\usepackage{amssymb}
\usepackage{graphicx}
\usepackage{hyperref}
\title{'Sarah Questions household finance v8 v1'}
\author{Tien Dung Doan}
\begin{document}
\maketitle
\section*{Question 1}
\textbf{Metadata}

\begin{itemize}
  \item Question ID: P3-WNDivRmd3d\_P1-WNCmp\_GPT4.1\_Household Finance\_02
  \item Primary KC: WHOLE NUMBERS | Division | dividing whole numbers up to 3 digits by 1 digit with remainder 
  \item Secondary KC: WHOLE NUMBERS | Comparison and ordering | comparing and ordering whole numbers
  \item Topic: Household finance such as income, utility bills, money, interest, savings, instalment, mortgage, financial planning etc.
  \item Grade: Primary 3
\end{itemize}

\textbf{Question}

Mrs. Tan has $275 saved from her grocery shopping budget. She wants to divide the money equally among her 4 family members to give them each the same amount for their own shopping. 

(a) How much money will each family member receive? Will there be any money left over?

(b) If Mrs. Tan also considered dividing $189 among her 4 family members, which amount would give each person more money?

Arrange both amounts from the greatest to the least based on how much each family member would receive.

\section*{Question 2}
\textbf{Metadata}

\begin{itemize}
  \item Question ID: P3-WNMul3d1d\_P1-WNAdd2nd\_GPT4.1\_Household Finance\_02
  \item Primary KC: WHOLE NUMBERS | Multiplication | multiplying whole numbers up to 3 digits by 1 digit
  \item Secondary KC: WHOLE NUMBERS | Addition | adding whole numbers
  \item Topic: Household finance such as income, utility bills, money, interest, savings, instalment, mortgage, financial planning etc.
  \item Grade: Primary 3
\end{itemize}

\textbf{Question}

Mrs. Lim buys 4 boxes of light bulbs for her home. Each box contains 125 light bulbs. She already has 37 light bulbs at home. How many light bulbs does Mrs. Lim have altogether after buying the new boxes?

\section*{Question 3}
\textbf{Metadata}

\begin{itemize}
  \item Question ID: P3-WNMul3d1d\_P1-WNSub2nd\_GPT4.1\_Household Finance\_02
  \item Primary KC: WHOLE NUMBERS | Multiplication | multiplying whole numbers up to 3 digits by 1 digit
  \item Secondary KC: WHOLE NUMBERS | Subtraction | subtracting whole numbers
  \item Topic: Household finance such as income, utility bills, money, interest, savings, instalment, mortgage, financial planning etc.
  \item Grade: Primary 3
\end{itemize}

\textbf{Question}

Sarah helps her mother by collecting the family's laundry every week. She receives \textdollar8 each week as a reward. After 5 weeks, Sarah spends \textdollar12 on a new pencil case. 

How much money does Sarah have left after 5 weeks?

\section*{Question 4}
\textbf{Metadata}

\begin{itemize}
  \item Question ID: P3-FrAddRl12\_P3-FrSmp\_GPT4.1\_Household Finance\_02
  \item Primary KC: FRACTIONS | Addition | adding two related fractions within one whole with denominators of given fractions not exceeding 12
  \item Secondary KC: FRACTIONS | Simplifying | expressing a fraction in its simplest form
  \item Topic: Household finance such as income, utility bills, money, interest, savings, instalment, mortgage, financial planning etc.
  \item Grade: Primary 3
\end{itemize}

\textbf{Question}

Lina is helping her mother keep track of the family's monthly expenses. Last month, they spent $\frac{3}{12}$ of their total income on electricity bills and $\frac{5}{12}$ of their income on water bills. 

What fraction of their total income was spent on both electricity and water bills together? Give your answer in its simplest form.

\section*{Question 5}
\textbf{Metadata}

\begin{itemize}
  \item Question ID: P3-FrSubRl12\_P3-FrSmp\_GPT4.1\_Household Finance\_02
  \item Primary KC: FRACTIONS | Subtraction | subtracting two related fractions within one whole with denominators of given fractions not exceeding 12
  \item Secondary KC: FRACTIONS | Simplifying | expressing a fraction in its simplest form
  \item Topic: Household finance such as income, utility bills, money, interest, savings, instalment, mortgage, financial planning etc.
  \item Grade: Primary 3
\end{itemize}

\textbf{Question}

Jenny spent $\frac{7}{12}$ of her pocket money on buying a storybook. She then used $\frac{3}{12}$ of her pocket money to buy a new pencil. What fraction of her pocket money had she not spent? Give your answer in its simplest form.

\section*{Question 6}
\textbf{Metadata}

\begin{itemize}
  \item Question ID: P4-WNMul4d1d\_P1-WNSub2nd\_GPT4.1\_Household Finance\_02
  \item Primary KC: WHOLE NUMBERS | Multiplication | multiplying whole numbers up to 4 digits by 1 digit or up to 3 digits by 2 digits
  \item Secondary KC: WHOLE NUMBERS | Subtraction | subtracting whole numbers
  \item Topic: Household finance such as income, utility bills, money, interest, savings, instalment, mortgage, financial planning etc.
  \item Grade: Primary 4
\end{itemize}

\textbf{Question}

Mrs Tan wants to buy 5 new chairs for her dining room. Each chair costs \textdollar245. She has saved \textdollar1200 for this purchase. 

(a) How much does Mrs Tan need to pay for 5 chairs altogether?

(b) After buying the chairs, how much money does Mrs Tan have left?

\section*{Question 7}
\textbf{Metadata}

\begin{itemize}
  \item Question ID: P4-WNDiv4d1d\_P1-WNAdd2nd\_GPT4.1\_Household Finance\_02
  \item Primary KC: WHOLE NUMBERS | Division | dividing whole numbers up to 4 digits by 1 digit
  \item Secondary KC: WHOLE NUMBERS | Addition | adding whole numbers
  \item Topic: Household finance such as income, utility bills, money, interest, savings, instalment, mortgage, financial planning etc.
  \item Grade: Primary 4
\end{itemize}

\textbf{Question}

Peter earned \textdollar2892 from his part-time job over 4 months. He decided to divide the earnings equally among himself and his 3 siblings. After sharing, Peter and his siblings combined all their shares and added an extra \textdollar108 from savings to buy a new washing machine. What is the total amount of money they have after adding the savings?

\section*{Question 8}
\textbf{Metadata}

\begin{itemize}
  \item Question ID: P4-FrRepSet\_P3-FrCnvEq\_GPT4.1\_Household Finance\_02
  \item Primary KC: FRACTIONS | Representation and concept | expressing a part of a set as a fraction
  \item Secondary KC: FRACTIONS | Conversion to equivalent fractions | Conversion to equivalent fractions (given either the denominator or the numerator)
  \item Topic: Household finance such as income, utility bills, money, interest, savings, instalment, mortgage, financial planning etc.
  \item Grade: Primary 4
\end{itemize}

\textbf{Question}

Maya and her sister shared a jar of coins equally. Maya then decided to give $\frac{3}{8}$ of her share to charity. 

(a) What fraction of the whole jar of coins did Maya give to charity?

(b) Express your answer in (a) as an equivalent fraction with denominator $16$.

\section*{Question 9}
\textbf{Metadata}

\begin{itemize}
  \item Question ID: P4-FrSubU12\_P2-FrAdd2nd\_GPT4.1\_Household Finance\_02
  \item Primary KC: FRACTIONS | Subtraction | subtracting unlike fractions with two different denominators not exceeding 12
  \item Secondary KC: FRACTIONS | Addition | adding fractions
  \item Topic: Household finance such as income, utility bills, money, interest, savings, instalment, mortgage, financial planning etc.
  \item Grade: Primary 4
\end{itemize}

\textbf{Question}

Sarah spent $\frac{3}{4}$ of her weekly pocket money on groceries and $\frac{2}{3}$ of her weekly pocket money on utility bills in the same week. She decided to add both expenses to find out the total fraction of her pocket money spent on these two items. Then, she wanted to know what fraction of her pocket money she had left after paying for both groceries and utility bills if she started with one whole pocket money. How much of her pocket money did Sarah spend in total, and what fraction of her pocket money did she have left?

\section*{Question 10}
\textbf{Metadata}

\begin{itemize}
  \item Question ID: P4-DcSub2d\_P4-DcCmp3d\_GPT4.1\_Household Finance\_02
  \item Primary KC: DECIMALS | Subtraction | subtracting decimals (up to 2 decimal places)
  \item Secondary KC: DECIMALS | Comparison and ordering | comparing and ordering decimals up to 3 decimal places
  \item Topic: Household finance such as income, utility bills, money, interest, savings, instalment, mortgage, financial planning etc.
  \item Grade: Primary 4
\end{itemize}

\textbf{Question}

Mrs Tan checked three of her utility bills for the month. The amounts were:

$\textdollar54.28$, $\textdollar54.105$, and $\textdollar53.99$.

(a) Arrange the amounts in order from the smallest to the largest.

(b) Mrs Tan paid the highest bill last month. How much less did she pay for the lowest bill compared to the highest bill this month, rounded to 2 decimal places?

\section*{Question 11}
\textbf{Metadata}

\begin{itemize}
  \item Question ID: P4-DcMul2d1d\_P4-DcCmp3d\_GPT4.1\_Household Finance\_01
  \item Primary KC: DECIMALS | Multiplication | multiplying decimals (up to 2 decimal places) by a 1-digit whole number
  \item Secondary KC: DECIMALS | Comparison and ordering | comparing and ordering decimals up to 3 decimal places
  \item Topic: Household finance such as income, utility bills, money, interest, savings, instalment, mortgage, financial planning etc.
  \item Grade: Primary 4
\end{itemize}

\textbf{Question}

Sarah is buying groceries for her family. She buys 3 packets of rice, each costing $4.25, and 2 cartons of milk, each costing $2.49. She also considers buying a box of cookies that costs $7.350, but she only has $20. Which item, between a packet of rice, a carton of milk, and the box of cookies, is the most expensive and which is the least expensive? Can Sarah afford to buy all the groceries including the box of cookies with her $20? Show your calculations.

\section*{Question 12}
\textbf{Metadata}

\begin{itemize}
  \item Question ID: P4-DcMul2d1d\_P4-DcRnd3d\_GPT4.1\_Household Finance\_01
  \item Primary KC: DECIMALS | Multiplication | multiplying decimals (up to 2 decimal places) by a 1-digit whole number
  \item Secondary KC: DECIMALS | Rounding | rounding decimals up to 3 decimal places to the nearest whole number, 1 decimal place and 2 decimal places 
  \item Topic: Household finance such as income, utility bills, money, interest, savings, instalment, mortgage, financial planning etc.
  \item Grade: Primary 4
\end{itemize}

\textbf{Question}

Mrs Tan pays \textdollar1.36 per litre for petrol for her car. Last month, she bought 7 litres of petrol each week for 4 weeks. 

(a) How much did she spend on petrol altogether last month?

(b) After calculating the total amount spent, round the amount to the nearest whole number, to 1 decimal place, and to 2 decimal places.

\section*{Question 13}
\textbf{Metadata}

\begin{itemize}
  \item Question ID: P4-DcMul2d1d\_P4-DcSub2nd\_GPT4.1\_Household Finance\_01
  \item Primary KC: DECIMALS | Multiplication | multiplying decimals (up to 2 decimal places) by a 1-digit whole number
  \item Secondary KC: DECIMALS | Subtraction | subtracting decimals
  \item Topic: Household finance such as income, utility bills, money, interest, savings, instalment, mortgage, financial planning etc.
  \item Grade: Primary 4
\end{itemize}

\textbf{Question}

Maya buys 3 packs of apples at the supermarket. Each pack costs \textdollar2.75. She uses a discount voucher and saves \textdollar1.20 on her total bill. How much does she pay after using the voucher?

\section*{Question 14}
\textbf{Metadata}

\begin{itemize}
  \item Question ID: P4-DcDiv2d1d\_P4-DcRnd3d\_GPT4.1\_Household Finance\_01
  \item Primary KC: DECIMALS | Division | dividing decimals (up to 2 decimal places) by a 1-digit whole number
  \item Secondary KC: DECIMALS | Rounding | rounding decimals up to 3 decimal places to the nearest whole number, 1 decimal place and 2 decimal places 
  \item Topic: Household finance such as income, utility bills, money, interest, savings, instalment, mortgage, financial planning etc.
  \item Grade: Primary 4
\end{itemize}

\textbf{Question}

Jasmine received her monthly electricity bill of \textdollar83.76. She decided to share the bill equally among herself and her 3 housemates. 

(a) How much does each person have to pay? Give your answer in dollars and cents, rounded to 2 decimal places.

(b) Round each person's share from part (a) to the nearest whole number and to 1 decimal place. Write both answers clearly.

\section*{Question 15}
\textbf{Metadata}

\begin{itemize}
  \item Question ID: P5-FrAddMix\_P5-FrCnv2Dc\_GPT4.1\_Household Finance\_01
  \item Primary KC: FRACTIONS | Addition | adding mixed numbers
  \item Secondary KC: FRACTIONS | Conversion to decimals | expressing fractions as decimals
  \item Topic: Household finance such as income, utility bills, money, interest, savings, instalment, mortgage, financial planning etc.
  \item Grade: Primary 5
\end{itemize}

\textbf{Question}

Mr. Tan spent $2\frac{3}{4}$ hours on Monday and $1\frac{2}{5}$ hours on Tuesday checking and paying his household utility bills online. 

(a) How many hours did Mr. Tan spend in total on both days? Give your answer as a mixed number, simplest form.  

(b) Express the total time spent as a decimal, correct to 2 decimal places.

\section*{Question 16}
\textbf{Metadata}

\begin{itemize}
  \item Question ID: P5-FrSubMix\_P3-FrSmp\_GPT4.1\_Household Finance\_01
  \item Primary KC: FRACTIONS | Subtraction | subtracting mixed numbers
  \item Secondary KC: FRACTIONS | Simplifying | expressing a fraction in its simplest form
  \item Topic: Household finance such as income, utility bills, money, interest, savings, instalment, mortgage, financial planning etc.
  \item Grade: Primary 5
\end{itemize}

\textbf{Question}

Amanda paid for her electricity and water bills this month. Her total bill for electricity was $3\dfrac{2}{5}$\textdollar, and her total bill for water was $1\dfrac{3}{10}$\textdollar. How much more did she spend on electricity than on water? Express your answer in its simplest form.

\section*{Question 17}
\textbf{Metadata}

\begin{itemize}
  \item Question ID: P5-FrSubMix\_P5-FrCnv2Dc\_GPT4.1\_Household Finance\_01
  \item Primary KC: FRACTIONS | Subtraction | subtracting mixed numbers
  \item Secondary KC: FRACTIONS | Conversion to decimals | expressing fractions as decimals
  \item Topic: Household finance such as income, utility bills, money, interest, savings, instalment, mortgage, financial planning etc.
  \item Grade: Primary 5
\end{itemize}

\textbf{Question}

Mrs Tan had $3\frac{1}{4}$ litres of cooking oil at the start of the month. By the end of the month, she had $1\frac{2}{5}$ litres left. 

(a) How many litres of cooking oil did Mrs Tan use during the month? 

(b) Express your answer in (a) as a decimal, correct to 2 decimal places.

\section*{Question 18}
\textbf{Metadata}

\begin{itemize}
  \item Question ID: P5-FrMulImN\_P2-FrAdd2nd\_GPT4.1\_Household Finance\_01
  \item Primary KC: FRACTIONS | Multiplication | multiplying a proper/improper fraction and a whole number
  \item Secondary KC: FRACTIONS | Addition | adding fractions
  \item Topic: Household finance such as income, utility bills, money, interest, savings, instalment, mortgage, financial planning etc.
  \item Grade: Primary 5
\end{itemize}

\textbf{Question}

Mrs Lee wants to save money each month for her family's groceries. She decides to use $\frac{2}{5}$ of her income every month for groceries. If her monthly income is \textdollar2500, how much does she spend on groceries in 4 months?

After 4 months, her brother gives her an extra $\frac{3}{10}$ of what she has already saved for groceries. How much money will Mrs Lee have for groceries in total after receiving her brother's extra contribution?

\section*{Question 19}
\textbf{Metadata}

\begin{itemize}
  \item Question ID: P5-FrMulImN\_P3-FrSmp\_GPT4.1\_Household Finance\_01
  \item Primary KC: FRACTIONS | Multiplication | multiplying a proper/improper fraction and a whole number
  \item Secondary KC: FRACTIONS | Simplifying | expressing a fraction in its simplest form
  \item Topic: Household finance such as income, utility bills, money, interest, savings, instalment, mortgage, financial planning etc.
  \item Grade: Primary 5
\end{itemize}

\textbf{Question}

Mr Lim pays $\textdollar120$ each month for his water bill. In the month of June, he used only $\frac{2}{5}$ of his normal water usage. 

(a) How much, in dollars, did Mr Lim pay for water in June, before simplifying the answer?

(b) Express the amount Mr Lim paid in June, in its simplest form.

\section*{Question 20}
\textbf{Metadata}

\begin{itemize}
  \item Question ID: P5-FrMulImN\_P5-FrCnv2Dc\_GPT4.1\_Household Finance\_01
  \item Primary KC: FRACTIONS | Multiplication | multiplying a proper/improper fraction and a whole number
  \item Secondary KC: FRACTIONS | Conversion to decimals | expressing fractions as decimals
  \item Topic: Household finance such as income, utility bills, money, interest, savings, instalment, mortgage, financial planning etc.
  \item Grade: Primary 5
\end{itemize}

\textbf{Question}

Mrs Tan uses $\frac{4}{5}$ of a packet of rice each week. If she buys 7 packets of rice, how many packets of rice will she have used after 7 weeks? Express your answer as a fraction first, then convert it to a decimal.

\section*{Question 21}
\textbf{Metadata}

\begin{itemize}
  \item Question ID: P5-FrMulImIm\_P2-FrCmp\_GPT4.1\_Household Finance\_01
  \item Primary KC: FRACTIONS | Multiplication | multiplying two improper fractions
  \item Secondary KC: FRACTIONS | Comparison and ordering | comparing and ordering fractions
  \item Topic: Household finance such as income, utility bills, money, interest, savings, instalment, mortgage, financial planning etc.
  \item Grade: Primary 5
\end{itemize}

\textbf{Question}

Mrs Tan uses $\frac{9}{4}$ units of electricity every week. Each unit of electricity costs $\frac{5}{3}$ times the cost per unit compared to last year. If the cost per unit last year was \textdollar2, how much does Mrs Tan pay for electricity in a week now? Compare this weekly cost with a household that uses $\frac{7}{3}$ units per week at the same current rate, and state which household pays more in a week and by how much.

\section*{Question 22}
\textbf{Metadata}

\begin{itemize}
  \item Question ID: P5-FrMulImIm\_P5-FrCnv2Dc\_GPT4.1\_Household Finance\_01
  \item Primary KC: FRACTIONS | Multiplication | multiplying two improper fractions
  \item Secondary KC: FRACTIONS | Conversion to decimals | expressing fractions as decimals
  \item Topic: Household finance such as income, utility bills, money, interest, savings, instalment, mortgage, financial planning etc.
  \item Grade: Primary 5
\end{itemize}

\textbf{Question}

Samantha is tracking her monthly electricity usage at home. Last month, her total usage was $\frac{9}{5}$ times the usual monthly amount, and her usual monthly usage is $\frac{7}{4}$ times what her neighbour, Julie, uses. 

If Julie uses 100 kWh of electricity each month, how many kWh did Samantha use last month? Express your answer as a decimal.

\section*{Question 23}
\textbf{Metadata}

\begin{itemize}
  \item Question ID: P5-FrMulMixN\_P2-FrAdd2nd\_GPT4.1\_Household Finance\_01
  \item Primary KC: FRACTIONS | Multiplication | multiplying a mixed number and a whole number
  \item Secondary KC: FRACTIONS | Addition | adding fractions
  \item Topic: Household finance such as income, utility bills, money, interest, savings, instalment, mortgage, financial planning etc.
  \item Grade: Primary 5
\end{itemize}

\textbf{Question}

Mrs Tan saved $2\frac{1}{2}$ each day for 5 days. On Saturday, she received an extra $\frac{3}{4}$ from her father. How much money did Mrs Tan have in total after 5 days and the extra amount from her father?

\section*{Question 24}
\textbf{Metadata}

\begin{itemize}
  \item Question ID: P5-DcMul3dK\_P4-DcCmp3d\_GPT4.1\_Household Finance\_01
  \item Primary KC: DECIMALS | Multiplication | multiplying decimals (up to 3 decimal places) by 10, 100, 1000 and their multiples
  \item Secondary KC: DECIMALS | Comparison and ordering | comparing and ordering decimals up to 3 decimal places
  \item Topic: Household finance such as income, utility bills, money, interest, savings, instalment, mortgage, financial planning etc.
  \item Grade: Primary 5
\end{itemize}

\textbf{Question}

Mrs Tan received her monthly utility bill, which showed that her electricity usage for May was $3.285$ kilowatt-hours (kWh) each day. The electricity company charges \textdollar0.19 per kWh. 

(a) Calculate Mrs Tan's total electricity usage for May (31 days) in kWh.

(b) Find the total amount Mrs Tan needs to pay for electricity used in May.

(c) Mrs Tan compared her May usage of $3.285$ kWh per day with her friend Mr Lee, who used $3.295$ kWh per day. Who used more electricity per day and by how much?


\section*{Question 25}
\textbf{Metadata}

\begin{itemize}
  \item Question ID: P5-DcMul3dK\_P4-DcAdd2nd\_GPT4.1\_Household Finance\_01
  \item Primary KC: DECIMALS | Multiplication | multiplying decimals (up to 3 decimal places) by 10, 100, 1000 and their multiples
  \item Secondary KC: DECIMALS | Addition | adding decimals
  \item Topic: Household finance such as income, utility bills, money, interest, savings, instalment, mortgage, financial planning etc.
  \item Grade: Primary 5
\end{itemize}

\textbf{Question}

Mrs. Tan is planning her monthly expenses. She spends $2.37 on electricity every day. 

(a) How much does she spend on electricity in 100 days?

She also pays $125.40 each month for water and $45.70 for gas. 

(b) What is her total utility bill for water, gas, and electricity for 100 days (assuming 100 days is about 3 months)?

\section*{Question 26}
\textbf{Metadata}

\begin{itemize}
  \item Question ID: P5-DcDiv3dK\_P4-DcCmp3d\_GPT4.1\_Household Finance\_01
  \item Primary KC: DECIMALS | Division | dividing decimals (up to 3 decimal places) by 10, 100, 1000 and their multiples
  \item Secondary KC: DECIMALS | Comparison and ordering | comparing and ordering decimals up to 3 decimal places
  \item Topic: Household finance such as income, utility bills, money, interest, savings, instalment, mortgage, financial planning etc.
  \item Grade: Primary 5
\end{itemize}

\textbf{Question}

Mei Ling received her monthly utility bill for water amounting to $\textdollar124.500$. She decides to split this amount equally with her two housemates, so each person needs to pay the same amount.\
\
(a) How much does each person pay?\
\
(b) Mei Ling checks her bank statement and finds she only has $\textdollar42.500$, $\textdollar41.580$, and $\textdollar41.670$ left in her three separate accounts. Arrange the amounts in ascending order and state if Mei Ling can pay her share using the account with the highest balance.

\section*{Question 27}
\textbf{Metadata}

\begin{itemize}
  \item Question ID: P5-DcDiv3dK\_P4-DcSub2nd\_GPT4.1\_Household Finance\_01
  \item Primary KC: DECIMALS | Division | dividing decimals (up to 3 decimal places) by 10, 100, 1000 and their multiples
  \item Secondary KC: DECIMALS | Subtraction | subtracting decimals
  \item Topic: Household finance such as income, utility bills, money, interest, savings, instalment, mortgage, financial planning etc.
  \item Grade: Primary 5
\end{itemize}

\textbf{Question}

Mrs Lim checked her electricity bill for two consecutive months. In the first month, she used $245.370$ kWh of electricity. In the second month, she used $\frac{245.370}{10}$ kWh of electricity less than the first month. How much less electricity, in kWh, did Mrs Lim use in the second month than in the first month?

\section*{Question 28}
\textbf{Metadata}

\begin{itemize}
  \item Question ID: P5-PcRepWh\_P1-WNAdd2nd\_GPT4.1\_Household Finance\_01
  \item Primary KC: PERCENTAGE | Representation and concept | expressing a part of a whole as a percentage
  \item Secondary KC: WHOLE NUMBERS | Addition | adding whole numbers
  \item Topic: Household finance such as income, utility bills, money, interest, savings, instalment, mortgage, financial planning etc.
  \item Grade: Primary 5
\end{itemize}

\textbf{Question}

Amy saved \textdollar80 in January and \textdollar120 in February. What percentage of her total savings in these two months did she save in January?

\section*{Question 29}
\textbf{Metadata}

\begin{itemize}
  \item Question ID: P5-PcRepWh\_P1-WNSub2nd\_GPT4.1\_Household Finance\_01
  \item Primary KC: PERCENTAGE | Representation and concept | expressing a part of a whole as a percentage
  \item Secondary KC: WHOLE NUMBERS | Subtraction | subtracting whole numbers
  \item Topic: Household finance such as income, utility bills, money, interest, savings, instalment, mortgage, financial planning etc.
  \item Grade: Primary 5
\end{itemize}

\textbf{Question}

Mrs Lee received her monthly electricity bill. The total bill was \textdollar120. She has already paid \textdollar45. What percentage of the total bill does she still need to pay?

\section*{Question 30}
\textbf{Metadata}

\begin{itemize}
  \item Question ID: P5-RtFndT\_P2-DcCnvD2N\_GPT4.1\_Household Finance\_01
  \item Primary KC: RATE | Finding total amount | finding total amount, given rate and number of units
  \item Secondary KC: DECIMALS | Conversion to smaller units | converting a measurement from a larger unit in decimal form to a smaller unit
  \item Topic: Household finance such as income, utility bills, money, interest, savings, instalment, mortgage, financial planning etc.
  \item Grade: Primary 5
\end{itemize}

\textbf{Question}

Mrs Lim uses $2.75$ kilowatt-hours of electricity each day for her household appliances. The cost of electricity is \textdollar0.19 per kilowatt-hour. How much does Mrs Lim pay for her electricity usage in $5$ days? Give your answer in cents.

\section*{Question 31}
\textbf{Metadata}

\begin{itemize}
  \item Question ID: P6-FrDivPN\_P2-FrCmp\_GPT4.1\_Household Finance\_01
  \item Primary KC: FRACTIONS | Division | dividing a proper fraction by a whole number
  \item Secondary KC: FRACTIONS | Comparison and ordering | comparing and ordering fractions
  \item Topic: Household finance such as income, utility bills, money, interest, savings, instalment, mortgage, financial planning etc.
  \item Grade: Primary 6
\end{itemize}

\textbf{Question}

Mrs. Tan has \textdollar180 saved up for paying her monthly utility bills. Each month, she only spends $\frac{3}{4}$ of what she saved on the bills. If she shares the monthly utility bill equally with her brother, how much does each person pay? Then, who pays more: Mrs. Tan after sharing the bill with her brother, or her friend Mr. Lee who pays $\frac{2}{5}$ of \textdollar180 by himself each month? Compare the amounts paid by Mrs. Tan and Mr. Lee.

\section*{Question 32}
\textbf{Metadata}

\begin{itemize}
  \item Question ID: P6-FrDivPN\_P2-FrSub2nd\_GPT4.1\_Household Finance\_01
  \item Primary KC: FRACTIONS | Division | dividing a proper fraction by a whole number
  \item Secondary KC: FRACTIONS | Subtraction | subtracting fractions
  \item Topic: Household finance such as income, utility bills, money, interest, savings, instalment, mortgage, financial planning etc.
  \item Grade: Primary 6
\end{itemize}

\textbf{Question}

Mrs Tan has $\textdollar 1$ and she wants to share it equally among her 4 children. She decides to keep $\frac{1}{8}$ of the money for herself first before dividing the rest equally. 

(a) What fraction of $\textdollar 1$ does Mrs Tan distribute among her children after keeping $\frac{1}{8}$ for herself?

(b) What fraction of $\textdollar 1$ does each child receive?

\section*{Question 33}
\textbf{Metadata}

\begin{itemize}
  \item Question ID: P6-FrDivPN\_P5-FrMul2nd\_GPT4.1\_Household Finance\_01
  \item Primary KC: FRACTIONS | Division | dividing a proper fraction by a whole number
  \item Secondary KC: FRACTIONS | Multiplication | fraction multiplication
  \item Topic: Household finance such as income, utility bills, money, interest, savings, instalment, mortgage, financial planning etc.
  \item Grade: Primary 6
\end{itemize}

\textbf{Question}

Mrs Lim saved $\textdollar18$ from her monthly allowance. She wants to divide $\frac{3}{4}$ of her savings equally among her 3 children as pocket money. How much money does each child receive?

\section*{Question 34}
\textbf{Metadata}

\begin{itemize}
  \item Question ID: P6-FrDivPP\_P2-FrAdd2nd\_GPT4.1\_Household Finance\_01
  \item Primary KC: FRACTIONS | Division | dividing a whole number/proper fraction by a proper fraction
  \item Secondary KC: FRACTIONS | Addition | adding fractions
  \item Topic: Household finance such as income, utility bills, money, interest, savings, instalment, mortgage, financial planning etc.
  \item Grade: Primary 6
\end{itemize}

\textbf{Question}

Mrs Tan baked $\dfrac{3}{4}$ of a cake for a family gathering. She decided to cut equal slices from the cake, where each slice is $\dfrac{1}{8}$ of the cake. 

(a) How many slices did she cut from $\dfrac{3}{4}$ of the cake?

After dinner, $\dfrac{1}{3}$ of the slices were left. Mrs Tan added another $\dfrac{1}{4}$ of a whole cake that she had kept in the fridge to the leftovers.

(b) What fraction of a whole cake did Mrs Tan now have in total after adding the leftover slices and the extra cake from the fridge?

\section*{Question 35}
\textbf{Metadata}

\begin{itemize}
  \item Question ID: P6-FrDivPP\_P5-FrMul2nd\_GPT4.1\_Household Finance\_01
  \item Primary KC: FRACTIONS | Division | dividing a whole number/proper fraction by a proper fraction
  \item Secondary KC: FRACTIONS | Multiplication | fraction multiplication
  \item Topic: Household finance such as income, utility bills, money, interest, savings, instalment, mortgage, financial planning etc.
  \item Grade: Primary 6
\end{itemize}

\textbf{Question}

Mrs Lee saved $\textdollar48$ to pay for her electricity bills over several months. If each month's bill is $\frac{2}{3}$ as much as her water bill, and her water bill each month is $\frac{3}{4}$ of $\textdollar12$, for how many months can Mrs Lee pay her electricity bills with her savings?

\section*{Question 36}
\textbf{Metadata}

\begin{itemize}
  \item Question ID: P6-PcFndWN\_P1-WNMul2nd\_GPT4.1\_Household Finance\_01
  \item Primary KC: PERCENTAGE | Finding the whole | finding the whole given a part and the percentage
  \item Secondary KC: WHOLE NUMBERS | Multiplication | multiplying whole numbers
  \item Topic: Household finance such as income, utility bills, money, interest, savings, instalment, mortgage, financial planning etc.
  \item Grade: Primary 6
\end{itemize}

\textbf{Question}

Mrs Tan spent 25\% of her monthly salary on utility bills. If the amount she spent on utility bills was \textdollar400, what was Mrs Tan's total monthly salary? If Mrs Tan received her salary for 3 months, what was the total amount she received in those 3 months?

\section*{Question 37}
\textbf{Metadata}

\begin{itemize}
  \item Question ID: P6-PcFndWN\_P1-WNDiv2nd\_GPT4.1\_Household Finance\_01
  \item Primary KC: PERCENTAGE | Finding the whole | finding the whole given a part and the percentage
  \item Secondary KC: WHOLE NUMBERS | Division | dividing whole numbers
  \item Topic: Household finance such as income, utility bills, money, interest, savings, instalment, mortgage, financial planning etc.
  \item Grade: Primary 6
\end{itemize}

\textbf{Question}

Mrs Lim spent $\textdollar180$ on her monthly electricity bill. This amount was $15\%$ of her total monthly income. How much was Mrs Lim's total monthly income? After finding her total income, if she decides to divide it equally among her 3 savings accounts, how much money will go into each account?

\section*{Question 38}
\textbf{Metadata}

\begin{itemize}
  \item Question ID: P6-PcFndChg\_P1-WNAdd2nd\_GPT4.1\_Household Finance\_01
  \item Primary KC: PERCENTAGE | Finding change | finding percentage increase/decrease
  \item Secondary KC: WHOLE NUMBERS | Addition | adding whole numbers
  \item Topic: Household finance such as income, utility bills, money, interest, savings, instalment, mortgage, financial planning etc.
  \item Grade: Primary 6
\end{itemize}

\textbf{Question}

Mr Tan's monthly electricity bill was \textdollar120 in January. In February, his bill increased by $15$\%. Mr Tan also paid his monthly water bill, which was \textdollar30 in both months. 

What was the total amount Mr Tan paid for his electricity and water bills in February?

\section*{Question 39}
\textbf{Metadata}

\begin{itemize}
  \item Question ID: P6-RoFndDvqWN\_P1-WNSub2nd\_GPT4.1\_Household Finance\_01
  \item Primary KC: RATIO | Finding divided quantities | dividing a given quantity in a given ratio
  \item Secondary KC: WHOLE NUMBERS | Subtraction | subtracting whole numbers
  \item Topic: Household finance such as income, utility bills, money, interest, savings, instalment, mortgage, financial planning etc.
  \item Grade: Primary 6
\end{itemize}

\textbf{Question}

Mrs. Lee has \textdollar900 to spend on her household's monthly expenses after paying her utility bills. She decides to divide this remaining money between groceries and savings in the ratio $5 : 4$.

(a) How much money does Mrs. Lee put aside for groceries and for savings, respectively?

(b) If Mrs. Lee needs to pay an unexpected repair bill of \textdollar130 from her savings, how much will she have left in her savings after the repair?

\section*{Question 40}
\textbf{Metadata}

\begin{itemize}
  \item Question ID: P6-RoFndRoWN\_P1-WNMul2nd\_GPT4.1\_Household Finance\_01
  \item Primary KC: RATIO | Finding ratio | finding the ratio of two or three given whole numbers
  \item Secondary KC: WHOLE NUMBERS | Multiplication | multiplying whole numbers
  \item Topic: Household finance such as income, utility bills, money, interest, savings, instalment, mortgage, financial planning etc.
  \item Grade: Primary 6
\end{itemize}

\textbf{Question}

Mr. Lim spends his monthly income on three main household expenses: groceries, utilities, and transport. Last month, he spent \textdollar540 on groceries, \textdollar300 on utilities, and \textdollar180 on transport.<br><br>(a) Find the ratio of the amount he spent on groceries to utilities to transport.<br><br>(b) If Mr. Lim decides to spend 2 times as much on utilities next month, what will be the new ratio of the amount spent on groceries to utilities to transport?

\section*{Question 41}
\textbf{Metadata}

\begin{itemize}
  \item Question ID: P6-RoFndRoWN\_P6-RoSmpWN\_GPT4.1\_Household Finance\_01
  \item Primary KC: RATIO | Finding ratio | finding the ratio of two or three given whole numbers
  \item Secondary KC: RATIO | Simplifying | expressing a ratio in its simplest form
  \item Topic: Household finance such as income, utility bills, money, interest, savings, instalment, mortgage, financial planning etc.
  \item Grade: Primary 6
\end{itemize}

\textbf{Question}

A family spends \textdollar600 on rent, \textdollar200 on groceries, and \textdollar100 on utility bills every month. 

(a) Find the ratio of the amount spent on rent to the amount spent on groceries to the amount spent on utility bills.

(b) Express this ratio in its simplest form.

\section*{Question 42}
\textbf{Metadata}

\begin{itemize}
  \item Question ID: P6-RoFndTmWN\_P1-WNSub2nd\_GPT4.1\_Household Finance\_01
  \item Primary KC: RATIO | Finding a missing term | finding the missing term in a pair of equivalent ratios
  \item Secondary KC: WHOLE NUMBERS | Subtraction | subtracting whole numbers
  \item Topic: Household finance such as income, utility bills, money, interest, savings, instalment, mortgage, financial planning etc.
  \item Grade: Primary 6
\end{itemize}

\textbf{Question}

Amy and her sister save money every month in the ratio $5:3$. Last month, Amy saved \textdollar80, which is the same as the month before. This month, Amy decides to spend \textdollar15 from her savings before her sister saves any money, and her sister still saves the same amount as last month. How much does Amy have left after her spending if the ratio of Amy’s and her sister’s savings this month remains $5:3$?

\section*{Question 43}
\textbf{Metadata}

\begin{itemize}
  \item Question ID: O1-RoRepFr\_P2-FrAdd2nd\_GPT4.1\_Household Finance\_01
  \item Primary KC: RATIO | Representation and concept | ratios involving fractions
  \item Secondary KC: FRACTIONS | Addition | adding fractions
  \item Topic: Household finance such as income, utility bills, money, interest, savings, instalment, mortgage, financial planning etc.
  \item Grade: Secondary O-level 1
\end{itemize}

\textbf{Question}

Mr. Lim decided to allocate his monthly household income to two expenses: electricity bills and grocery shopping. He allocated the money in the ratio $\frac{2}{3} : \frac{5}{6}$ (electricity : groceries).

If his total monthly expense on both electricity and groceries is \textdollar240, how much did Mr. Lim spend on each category?

*Hint: You may need to add and compare fractions as part of this problem.*

\section*{Question 44}
\textbf{Metadata}

\begin{itemize}
  \item Question ID: O1-RoRepFr\_P5-FrMul2nd\_GPT4.1\_Household Finance\_01
  \item Primary KC: RATIO | Representation and concept | ratios involving fractions
  \item Secondary KC: FRACTIONS | Multiplication | fraction multiplication
  \item Topic: Household finance such as income, utility bills, money, interest, savings, instalment, mortgage, financial planning etc.
  \item Grade: Secondary O-level 1
\end{itemize}

\textbf{Question}

Siti sets aside a portion of her monthly income for household bills. She spends $\frac{2}{5}$ of her income on utility bills and $\frac{1}{4}$ of her income on groceries.\
\
(a) Express the ratio of the amount Siti spends on utility bills to the amount she spends on groceries in its simplest form.\
\
(b) If Siti's monthly income is \textdollar2000, how much does she spend on utility bills and groceries altogether?

\section*{Question 45}
\textbf{Metadata}

\begin{itemize}
  \item Question ID: O1-RoRepFr\_O1-RoSmpFr\_GPT4.1\_Household Finance\_02
  \item Primary KC: RATIO | Representation and concept | ratios involving fractions
  \item Secondary KC: RATIO | Simplifying | converting a ratio involving fractions to its simplest form
  \item Topic: Household finance such as income, utility bills, money, interest, savings, instalment, mortgage, financial planning etc.
  \item Grade: Secondary O-level 1
\end{itemize}

\textbf{Question}

Mrs Tan sets aside a portion of her monthly household income for utilities and groceries. Last month, the amount she spent on utilities was $\frac{2}{5}$ of her monthly income, while the amount spent on groceries was $\frac{3}{10}$ of her monthly income.

(a) Express the ratio of the amount spent on utilities to the amount spent on groceries in its simplest form.

\section*{Question 46}
\textbf{Metadata}

\begin{itemize}
  \item Question ID: O1-RoRepDc\_P4-DcAdd2nd\_GPT4.1\_Household Finance\_01
  \item Primary KC: RATIO | Representation and concept | ratios involving decimals
  \item Secondary KC: DECIMALS | Addition | adding decimals
  \item Topic: Household finance such as income, utility bills, money, interest, savings, instalment, mortgage, financial planning etc.
  \item Grade: Secondary O-level 1
\end{itemize}

\textbf{Question}

Siti and her brother share the monthly cost of their home's electricity bill. Last month, the ratio of the amount Siti paid to the amount her brother paid was $1.2 : 2.4$. If the total electricity bill was \textdollar72.50, how much did Siti pay and how much did her brother pay?

\section*{Question 47}
\textbf{Metadata}

\begin{itemize}
  \item Question ID: O1-RoRepDc\_O1-RoSmpDc\_GPT4.1\_Household Finance\_02
  \item Primary KC: RATIO | Representation and concept | ratios involving decimals
  \item Secondary KC: RATIO | Simplifying | converting a ratio involving decimals to its simplest form
  \item Topic: Household finance such as income, utility bills, money, interest, savings, instalment, mortgage, financial planning etc.
  \item Grade: Secondary O-level 1
\end{itemize}

\textbf{Question}

Mr Tan spends his monthly income in three categories: savings, utility bills, and groceries. The ratio of the amount he spends on savings to utility bills to groceries is $2.5:1.5:4$. 

(a) Write this ratio in its simplest whole-number form.

(b) If Mr Tan's total monthly income is $\textdollar1600$, how much does he spend on each category?

\section*{Question 48}
\textbf{Metadata}

\begin{itemize}
  \item Question ID: O1-PcRep2q\_O1-PcCnv2Fr\_GPT4.1\_Household Finance\_02
  \item Primary KC: PERCENTAGE | Representation and concept | comparing two quantities by percentage
  \item Secondary KC: PERCENTAGE | Conversion to fraction | expressing percentage as a fraction
  \item Topic: Household finance such as income, utility bills, money, interest, savings, instalment, mortgage, financial planning etc.
  \item Grade: Secondary O-level 1
\end{itemize}

\textbf{Question}

Mrs Tan's monthly electricity bill in March was \textdollar120. In April, her bill increased to \textdollar150. 

(a) By what percentage did her electricity bill increase from March to April?

(b) Express this percentage increase as a fraction in its simplest form.

\section*{Question 49}
\textbf{Metadata}

\begin{itemize}
  \item Question ID: O1-PcFndRslt\_P1-WNAdd2nd\_GPT4.1\_Household Finance\_01
  \item Primary KC: PERCENTAGE | Finding result after change | increasing/decreasing a quantity by a given percentage
  \item Secondary KC: WHOLE NUMBERS | Addition | adding whole numbers
  \item Topic: Household finance such as income, utility bills, money, interest, savings, instalment, mortgage, financial planning etc.
  \item Grade: Secondary O-level 1
\end{itemize}

\textbf{Question}

A family used to pay \textdollar200 each month for electricity. In June, the electricity company increased their rates by $12\%$. Additionally, the family bought a new fan, which uses an extra \textdollar18 worth of electricity each month. 

What is the new total monthly electricity bill after the rate increase and including the new fan's electricity usage?

\section*{Question 50}
\textbf{Metadata}

\begin{itemize}
  \item Question ID: O1-PcFndRslt\_P1-WNMul2nd\_GPT4.1\_Household Finance\_01
  \item Primary KC: PERCENTAGE | Finding result after change | increasing/decreasing a quantity by a given percentage
  \item Secondary KC: WHOLE NUMBERS | Multiplication | multiplying whole numbers
  \item Topic: Household finance such as income, utility bills, money, interest, savings, instalment, mortgage, financial planning etc.
  \item Grade: Secondary O-level 1
\end{itemize}

\textbf{Question}

Mrs Lim pays a monthly electricity bill of \textdollar120. This month, due to an increase in electricity rates, her bill will increase by $7\%$. If Mrs Lim has to pay the new bill for 4 months in a row, how much does she have to pay in total for these 4 months?

\section*{Question 51}
\textbf{Metadata}

\begin{itemize}
  \item Question ID: O1-PcRepRvs\_O1-PcCnv2Fr\_GPT4.1\_Household Finance\_02
  \item Primary KC: PERCENTAGE | Representation and concept | reverse percentages
  \item Secondary KC: PERCENTAGE | Conversion to fraction | expressing percentage as a fraction
  \item Topic: Household finance such as income, utility bills, money, interest, savings, instalment, mortgage, financial planning etc.
  \item Grade: Secondary O-level 1
\end{itemize}

\textbf{Question}

Mrs Tan found out that after receiving a 20\% discount on her electricity bill, she paid \textdollar96. 

(a) What fraction of the original bill did she pay, expressing your answer in its simplest form?

(b) What was the original amount of the electricity bill before the discount?

\section*{Question 52}
\textbf{Metadata}

\begin{itemize}
  \item Question ID: O1-AgRepExSq\_O1-AgEvlEx\_GPT4.1\_Household Finance\_02
  \item Primary KC: ALGEBRA | Representation and concept | translation of simple real-world situations into quadratic algebraic expressions
  \item Secondary KC: ALGEBRA | Evaluation | evaluation of algebraic expressions and formulae
  \item Topic: Household finance such as income, utility bills, money, interest, savings, instalment, mortgage, financial planning etc.
  \item Grade: Secondary O-level 1
\end{itemize}

\textbf{Question}

Mary wants to renovate her living room. She plans to buy a carpet and the cost to carpet the floor depends on the area of the rectangular room. The length of the room is $(x+2)$ metres and the width is $(x-1)$ metres. The cost per square metre of carpet is \textdollar25.\
\
(a) Write a quadratic algebraic expression, in terms of $x$, for the total cost of carpeting the room.\
\
(b) If $x = 4$, calculate the actual total cost of the carpet.

\section*{Question 53}
\textbf{Metadata}

\begin{itemize}
  \item Question ID: O1-AgRepnth\_O1-AgEvlEx\_GPT4.1\_Household Finance\_02
  \item Primary KC: ALGEBRA | Representation and concept | recognising and representing patterns/relationships by finding an algebraic expression for the nth term
  \item Secondary KC: ALGEBRA | Evaluation | evaluation of algebraic expressions and formulae
  \item Topic: Household finance such as income, utility bills, money, interest, savings, instalment, mortgage, financial planning etc.
  \item Grade: Secondary O-level 1
\end{itemize}

\textbf{Question}

Jia Li decides to save money every month by increasing her savings in a pattern. In her first month, she saves \textdollar10. Each subsequent month, she saves \textdollar5 more than the previous month. 

(a) Let $n$ be the number of the month. Find an algebraic expression for the amount Jia Li saves in the $n$th month.

(b) Using your expression from (a), find how much Jia Li saves in the 8th month.

\section*{Question 54}
\textbf{Metadata}

\begin{itemize}
  \item Question ID: O2-RoRepDP\_P1-WNDiv2nd\_GPT4.1\_Household Finance\_01
  \item Primary KC: RATIO | Representation and concept | direct proportion
  \item Secondary KC: WHOLE NUMBERS | Division | dividing whole numbers
  \item Topic: Household finance such as income, utility bills, money, interest, savings, instalment, mortgage, financial planning etc.
  \item Grade: Secondary O-level 2
\end{itemize}

\textbf{Question}

Lisa and her sister Zoe share the cost of their monthly household electricity bill in the ratio $5:3$. This month, the total bill is \textdollar 184. \newline
(a) How much does Zoe need to pay? \newline
(b) If Zoe decides to divide her share equally over 4 weeks, how much must she set aside each week?

\section*{Question 55}
\textbf{Metadata}

\begin{itemize}
  \item Question ID: O2-AgSlvIneq\_O2-AgRepIneq\_GPT4.1\_Household Finance\_01
  \item Primary KC: ALGEBRA | Solving | solving simple linear inequalities with one variable
  \item Secondary KC: ALGEBRA | Representation and concept | translation of simple real-world situations to simple linear inequalities with one variable
  \item Topic: Household finance such as income, utility bills, money, interest, savings, instalment, mortgage, financial planning etc.
  \item Grade: Secondary O-level 2
\end{itemize}

\textbf{Question}

Ming wants to save at least \textdollar120 every month after paying for his monthly expenses. His monthly salary is \textdollar500, and his monthly utility bills cost \textdollarx. If the rest of his expenses (excluding utility bills) total \textdollar350 each month, write and solve a linear inequality to find the maximum amount Ming can spend on utility bills and still achieve his savings goal.

\section*{Question 56}
\textbf{Metadata}

\begin{itemize}
  \item Question ID: O2-AgSlvSq1v\_O1-AgRepEq\_GPT4.1\_Household Finance\_01
  \item Primary KC: ALGEBRA | Solving | solving quadratic equations in one variable
  \item Secondary KC: ALGEBRA | Representation and concept | translation of simple real-world situations to equations
  \item Topic: Household finance such as income, utility bills, money, interest, savings, instalment, mortgage, financial planning etc.
  \item Grade: Secondary O-level 2
\end{itemize}

\textbf{Question}

Wei Leng wants to renovate her living room and intends to buy a new television. To afford this, she decides to set aside an amount of money each month from her salary as savings. After 4 months, the total amount she has saved is \textdollar120, and after 8 months, she has saved a total of \textdollar192.

She realises the amount she saves each month increases by a fixed amount every month. Let the amount she saves in the first month be $x$ dollars and the constant increase each month be $y$ dollars. 

Wei Leng calculates that if she continues saving in this way, in $n$ months, she will have saved exactly \textdollar360.

(i) Form an equation to express the total amount saved after $n$ months, based on the given situation.

(ii) Find the value of $n$.

\section*{Question 57}
\textbf{Metadata}

\begin{itemize}
  \item Question ID: O2-SPFndmdn\_O3-SPFndrng\_GPT4.1\_Household Finance\_01
  \item Primary KC: STATISTICS AND PROBABILITY | Finding median | Finding median for a set of data
  \item Secondary KC: STATISTICS AND PROBABILITY | Finding range | finding range as measures of spread for a set of data 
  \item Topic: Household finance such as income, utility bills, money, interest, savings, instalment, mortgage, financial planning etc.
  \item Grade: Secondary O-level 2
\end{itemize}

\textbf{Question}

A family tracks their monthly electricity bills (in dollars) for 7 consecutive months. The amounts paid were: \textdollar72, \textdollar85, \textdollar78, \textdollar90, \textdollar68, \textdollar95, and \textdollar80.\
(a) What is the median monthly electricity bill for these 7 months?\
(b) What is the range of their monthly electricity bills for this period?

\section*{Question 58}
\textbf{Metadata}

\begin{itemize}
  \item Question ID: O2-SPFndmean\_O3-BPRepSN\_GPT4.1\_Household Finance\_01
  \item Primary KC: STATISTICS AND PROBABILITY | Finding mean deviation | calculation of the mean for a set of data
  \item Secondary KC: BASE AND POWER | Representation and concept  | use of standard form Ax10^n , where n is an integer, and 1<= A<= 10
  \item Topic: Household finance such as income, utility bills, money, interest, savings, instalment, mortgage, financial planning etc.
  \item Grade: Secondary O-level 2
\end{itemize}

\textbf{Question}

Mr Tan records the monthly electricity bills (in dollars) for his household over 4 months: \textdollar2.5 \times 10^2, \textdollar3.1 \times 10^2, \textdollar2.8 \times 10^2, and \textdollar2.6 \times 10^2.

(a) Calculate the mean monthly electricity bill in standard form $A \times 10^n$ where $1 \leq A < 10$ and $n$ is an integer.

(b) Find the mean deviation of the monthly bills from the mean, giving your answer in standard form as well.

\section*{Question 59}
\textbf{Metadata}

\begin{itemize}
  \item Question ID: O3-BPOpr\_O3-BPRepNegI\_GPT4.1\_Household Finance\_01
  \item Primary KC: BASE AND POWER | Operations | laws of indices
  \item Secondary KC: BASE AND POWER | Representation and concept  | negative indices
  \item Topic: Household finance such as income, utility bills, money, interest, savings, instalment, mortgage, financial planning etc.
  \item Grade: Secondary O-level 3/4
\end{itemize}

\textbf{Question}

Jasmine noticed that her monthly electricity bill has been increasing exponentially due to the use of new appliances. In January, her bill was \textdollar100. The electricity provider stated that the bill grows each month by a factor of $2^1.5$. 

(a) Express the bill for March, in terms of January, as a single power of $2$ using the laws of indices.

(b) Jasmine decided to compare the bill in March with that in January by finding the value of $2^{-2}$ and interpreting what it means in the context of negative indices, if the bill had decreased by that factor instead of increasing.

\section*{Question 60}
\textbf{Metadata}

\begin{itemize}
  \item Question ID: O3-MXMulSM\_O3-MXAdd\_GPT4.1\_Household Finance\_01
  \item Primary KC: MATRICES | Multiplication | product of a scalar quantity and a matrix
  \item Secondary KC: MATRICES | Addition | addition of matrices
  \item Topic: Household finance such as income, utility bills, money, interest, savings, instalment, mortgage, financial planning etc.
  \item Grade: Secondary O-level 3/4
\end{itemize}

\textbf{Question}

A family is planning their monthly finances. Let matrix $A$ represent their regular monthly utility bills in dollars for three months, and matrix $B$ represent unexpected additional expenses for the same months:
\[
A = \begin{bmatrix} 120 \\ 130 \\ 140 \end{bmatrix}, \quad B = \begin{bmatrix} 20 \\ 30 \\ 10 \end{bmatrix}
\]
In the next three months, they expect their utility bills (represented by $A$) to increase by 5% each month due to higher usage. 

(a) Represent the new utility bills for each month as a matrix, by multiplying $A$ by the appropriate scalar. 

(b) Find the total amount for each month, combining the new utility bills and the unexpected expenses.

\section*{Question 61}
\textbf{Metadata}

\begin{itemize}
  \item Question ID: O3-MXSub\_O3-MXAdd\_GPT4.1\_Household Finance\_01
  \item Primary KC: MATRICES | Subtraction | subtraction of matrices
  \item Secondary KC: MATRICES | Addition | addition of matrices
  \item Topic: Household finance such as income, utility bills, money, interest, savings, instalment, mortgage, financial planning etc.
  \item Grade: Secondary O-level 3/4
\end{itemize}

\textbf{Question}

The monthly household expenses for the Tan family in January and February, broken down into three categories (utilities, groceries, and transport), are represented by the following matrices:

January: $\begin{pmatrix} 120 & 350 & 80 \\ \end{pmatrix}$

February: $\begin{pmatrix} 110 & 370 & 90 \\ \end{pmatrix}$

(a) Find the change in their expenses for each category from January to February by subtracting the January expenses matrix from the February expenses matrix.

(b) In March, the family's expenses in each category increased by $20$, $15$, and $10$ respectively compared to February. Represent this increase by a matrix and, using matrix addition, find the Tan family’s total household expenses for March in each category.

\section*{Question 62}
\textbf{Metadata}

\begin{itemize}
  \item Question ID: O3-SPAddProb\_O2-SPRepPrSE\_GPT4.1\_Household Finance\_01
  \item Primary KC: STATISTICS AND PROBABILITY | Addition | addition of probabilities
  \item Secondary KC: STATISTICS AND PROBABILITY | Representation and concept | probability of single events
  \item Topic: Household finance such as income, utility bills, money, interest, savings, instalment, mortgage, financial planning etc.
  \item Grade: Secondary O-level 3/4
\end{itemize}

\textbf{Question}

A family keeps track of the days they receive different utility bills by post. For any given month, the probability that the electricity bill arrives on a Monday is $0.2$, and the probability that the water bill arrives on a Monday is $0.3$. The two events are mutually exclusive, meaning both bills cannot arrive on the same Monday.

What is the probability that, in any given month, at least one utility bill (electricity or water) arrives on a Monday?

\section*{Question 63}
\textbf{Metadata}

\begin{itemize}
  \item Question ID: O3-SPAddProb\_O3-SPFndPrCE\_GPT4.1\_Household Finance\_01
  \item Primary KC: STATISTICS AND PROBABILITY | Addition | addition of probabilities
  \item Secondary KC: STATISTICS AND PROBABILITY | Finding probability | probability of simple combined events
  \item Topic: Household finance such as income, utility bills, money, interest, savings, instalment, mortgage, financial planning etc.
  \item Grade: Secondary O-level 3/4
\end{itemize}

\textbf{Question}

Jia Wei keeps track of his monthly household expenses. Last month, he found that the probability his family spent more than \textdollar200 on electricity was $0.3$, and the probability they spent more than \textdollar120 on water was $0.2$. The probability that in a month at least one of these two events happens (either electricity is more than \textdollar200 or water is more than \textdollar120) is $0.4$.

What is the probability that in a month, Jia Wei's family will spend more than \textdollar200 on electricity and more than \textdollar120 on water at the same time?

Express your answer as a decimal.

\section*{Question 64}
\textbf{Metadata}

\begin{itemize}
  \item Question ID: O3-SPMulProb\_O3-SPFndPrCE\_GPT4.1\_Household Finance\_01
  \item Primary KC: STATISTICS AND PROBABILITY | Multiplication | multiplication of probabilities
  \item Secondary KC: STATISTICS AND PROBABILITY | Finding probability | probability of simple combined events
  \item Topic: Household finance such as income, utility bills, money, interest, savings, instalment, mortgage, financial planning etc.
  \item Grade: Secondary O-level 3/4
\end{itemize}

\textbf{Question}

Jiawei is planning to pay his electricity and water bills next month. The probability that his electricity bill is below \textdollar90 is $0.6$, and the probability that his water bill is below \textdollar30 is $0.7$. If Jiawei believes that the amounts of his electricity and water bills are independent of each other, what is the probability that both his electricity bill is below \textdollar90 \\textbf{and} his water bill is below \textdollar30 next month?

\end{document}
