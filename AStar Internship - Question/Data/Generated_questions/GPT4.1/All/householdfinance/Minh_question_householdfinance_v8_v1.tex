\documentclass{article}
\usepackage[utf8]{inputenc}
\usepackage{amsmath}
\usepackage{amsfonts}
\usepackage{amssymb}
\usepackage{graphicx}
\usepackage{hyperref}
\title{'Minh Questions household finance v8 v1'}
\author{Tien Dung Doan}
\begin{document}
\maketitle
\section*{Question 1}
\textbf{Metadata}

\begin{itemize}
  \item Question ID: P3-WNSub4d\_P1-WNCmp\_GPT4.1\_Household Finance\_03
  \item Primary KC: WHOLE NUMBERS | Subtraction | subtracting whole numbers up to 4 digits
  \item Secondary KC: WHOLE NUMBERS | Comparison and ordering | comparing and ordering whole numbers
  \item Topic: Household finance such as income, utility bills, money, interest, savings, instalment, mortgage, financial planning etc.
  \item Grade: Primary 3
\end{itemize}

\textbf{Question}

Mrs. Lim received her monthly salary of \textdollar4250. She needed to pay the following bills this month: \textdollar1325 for the house mortgage and \textdollar760 for utility bills. After paying these two bills, how much money does Mrs. Lim have left from her salary? If her friend, Mr. Wong, had \textdollar2300 left after paying his bills, who has more money left, and by how much?

\section*{Question 2}
\textbf{Metadata}

\begin{itemize}
  \item Question ID: P3-WNDivRmd3d\_P1-WNAdd2nd\_GPT4.1\_Household Finance\_02
  \item Primary KC: WHOLE NUMBERS | Division | dividing whole numbers up to 3 digits by 1 digit with remainder 
  \item Secondary KC: WHOLE NUMBERS | Addition | adding whole numbers
  \item Topic: Household finance such as income, utility bills, money, interest, savings, instalment, mortgage, financial planning etc.
  \item Grade: Primary 3
\end{itemize}

\textbf{Question}

Mr. Tan received \textdollar342 from his salary this week and \textdollar128 from tutoring. He wants to divide all of this money equally among his 5 children for their weekly allowance. After giving each child an equal amount, how much money does each child get and how much money will Mr. Tan have left over?

\section*{Question 3}
\textbf{Metadata}

\begin{itemize}
  \item Question ID: P3-WNDivRmd3d\_P1-WNMul2nd\_GPT4.1\_Household Finance\_02
  \item Primary KC: WHOLE NUMBERS | Division | dividing whole numbers up to 3 digits by 1 digit with remainder 
  \item Secondary KC: WHOLE NUMBERS | Multiplication | multiplying whole numbers
  \item Topic: Household finance such as income, utility bills, money, interest, savings, instalment, mortgage, financial planning etc.
  \item Grade: Primary 3
\end{itemize}

\textbf{Question}

Mr Lim has \textdollar156 and wants to buy packs of apples for his children. Each pack costs \textdollar7. After buying as many packs as possible with his money, he gives 2 packs to each of his 8 neighbours. 

How many packs of apples does Mr Lim have left after giving to his neighbours?

\section*{Question 4}
\textbf{Metadata}

\begin{itemize}
  \item Question ID: P3-WNDiv3d1d\_P1-WNAdd2nd\_GPT4.1\_Household Finance\_02
  \item Primary KC: WHOLE NUMBERS | Division | dividing whole numbers up to 3 digits by 1 digit
  \item Secondary KC: WHOLE NUMBERS | Addition | adding whole numbers
  \item Topic: Household finance such as income, utility bills, money, interest, savings, instalment, mortgage, financial planning etc.
  \item Grade: Primary 3
\end{itemize}

\textbf{Question}

Mr. Tan received \textdollar285 from his part-time job for cleaning houses. He wants to save this money equally over 3 months. After saving for 3 months, he decides to add \textdollar50 to his total savings. 

How much money does Mr. Tan have altogether after 3 months?

\section*{Question 5}
\textbf{Metadata}

\begin{itemize}
  \item Question ID: P3-WNDiv3d1d\_P1-WNSub2nd\_GPT4.1\_Household Finance\_02
  \item Primary KC: WHOLE NUMBERS | Division | dividing whole numbers up to 3 digits by 1 digit
  \item Secondary KC: WHOLE NUMBERS | Subtraction | subtracting whole numbers
  \item Topic: Household finance such as income, utility bills, money, interest, savings, instalment, mortgage, financial planning etc.
  \item Grade: Primary 3
\end{itemize}

\textbf{Question}

Aunt Mei has \textdollar168 to pay for her monthly utility bills. She pays the same amount each week for 4 weeks. After paying for the 4 weeks, she has \textdollar20 left. 

How much does Aunt Mei pay for her utility bills each week?

\section*{Question 6}
\textbf{Metadata}

\begin{itemize}
  \item Question ID: P3-WNDiv3d1d\_P1-WNMul2nd\_GPT4.1\_Household Finance\_02
  \item Primary KC: WHOLE NUMBERS | Division | dividing whole numbers up to 3 digits by 1 digit
  \item Secondary KC: WHOLE NUMBERS | Multiplication | multiplying whole numbers
  \item Topic: Household finance such as income, utility bills, money, interest, savings, instalment, mortgage, financial planning etc.
  \item Grade: Primary 3
\end{itemize}

\textbf{Question}

A family paid \textdollar624 to buy 8 identical chairs for their dining room. If they want to buy 6 more of the same chairs for their living room, how much money will they need to pay in total for the 6 chairs?

\section*{Question 7}
\textbf{Metadata}

\begin{itemize}
  \item Question ID: P3-FrAddRl12\_P2-FrCmp\_GPT4.1\_Household Finance\_02
  \item Primary KC: FRACTIONS | Addition | adding two related fractions within one whole with denominators of given fractions not exceeding 12
  \item Secondary KC: FRACTIONS | Comparison and ordering | comparing and ordering fractions
  \item Topic: Household finance such as income, utility bills, money, interest, savings, instalment, mortgage, financial planning etc.
  \item Grade: Primary 3
\end{itemize}

\textbf{Question}

Maya spent $\frac{1}{4}$ of her savings on the electricity bill and $\frac{2}{4}$ of her savings on the water bill this month. 

(a) What fraction of her savings did she spend in total on these two bills?

(b) Did she spend more than, less than, or exactly half of her savings on these two bills?

\section*{Question 8}
\textbf{Metadata}

\begin{itemize}
  \item Question ID: P3-FrSubRl12\_P2-FrAdd2nd\_GPT4.1\_Household Finance\_02
  \item Primary KC: FRACTIONS | Subtraction | subtracting two related fractions within one whole with denominators of given fractions not exceeding 12
  \item Secondary KC: FRACTIONS | Addition | adding fractions
  \item Topic: Household finance such as income, utility bills, money, interest, savings, instalment, mortgage, financial planning etc.
  \item Grade: Primary 3
\end{itemize}

\textbf{Question}

Sarah spent $\frac{5}{12}$ of her pocket money on buying snacks and $\frac{1}{4}$ on buying a storybook. What fraction of her pocket money did she spend in total? How much more did she spend on snacks than on the storybook?

\section*{Question 9}
\textbf{Metadata}

\begin{itemize}
  \item Question ID: P4-WNMul4d1d\_P1-WNAdd2nd\_GPT4.1\_Household Finance\_02
  \item Primary KC: WHOLE NUMBERS | Multiplication | multiplying whole numbers up to 4 digits by 1 digit or up to 3 digits by 2 digits
  \item Secondary KC: WHOLE NUMBERS | Addition | adding whole numbers
  \item Topic: Household finance such as income, utility bills, money, interest, savings, instalment, mortgage, financial planning etc.
  \item Grade: Primary 4
\end{itemize}

\textbf{Question}

Mrs Lim buys 3 boxes of soap for her family. Each box contains 125 bars of soap. She also decides to buy another set of soap bars that is packed in 2 bags, with each bag containing 82 bars. 

How many bars of soap does Mrs Lim have altogether after all her purchases?

\section*{Question 10}
\textbf{Metadata}

\begin{itemize}
  \item Question ID: P4-WNDiv4d1d\_P1-WNCmp\_GPT4.1\_Household Finance\_02
  \item Primary KC: WHOLE NUMBERS | Division | dividing whole numbers up to 4 digits by 1 digit
  \item Secondary KC: WHOLE NUMBERS | Comparison and ordering | comparing and ordering whole numbers
  \item Topic: Household finance such as income, utility bills, money, interest, savings, instalment, mortgage, financial planning etc.
  \item Grade: Primary 4
\end{itemize}

\textbf{Question}

Mrs Tan receives her monthly salary of $\textdollar3276$. She wants to save an equal amount each week for 4 weeks in a month.\
\
(a) How much money does Mrs Tan save each week?\
\
(b) After saving for 4 weeks, Mrs Tan compares her savings to her daughter's total savings of $\textdollar800$. Who has saved more money and by how much?

\section*{Question 11}
\textbf{Metadata}

\begin{itemize}
  \item Question ID: P4-WNDiv4d1d\_P1-WNMul2nd\_GPT4.1\_Household Finance\_02
  \item Primary KC: WHOLE NUMBERS | Division | dividing whole numbers up to 4 digits by 1 digit
  \item Secondary KC: WHOLE NUMBERS | Multiplication | multiplying whole numbers
  \item Topic: Household finance such as income, utility bills, money, interest, savings, instalment, mortgage, financial planning etc.
  \item Grade: Primary 4
\end{itemize}

\textbf{Question}

Mrs Lim is budgeting her monthly groceries. She bought 4 packs of rice, and each pack costs \textdollar12. She then wants to share the total cost of the rice equally with 3 neighbours. How much does each neighbour need to pay?

\section*{Question 12}
\textbf{Metadata}

\begin{itemize}
  \item Question ID: P4-FrAddU12\_P3-FrSmp\_GPT4.1\_Household Finance\_02
  \item Primary KC: FRACTIONS | Addition | adding unlike fractions with two different denominators not exceeding 12
  \item Secondary KC: FRACTIONS | Simplifying | expressing a fraction in its simplest form
  \item Topic: Household finance such as income, utility bills, money, interest, savings, instalment, mortgage, financial planning etc.
  \item Grade: Primary 4
\end{itemize}

\textbf{Question}

Mrs Lim spent $\textdollar120$ of her monthly expenses on groceries and $\textdollar80$ on utility bills. She knew that $\frac{5}{12}$ of her expenses went to groceries and $\frac{1}{6}$ of her expenses went to utility bills. What fraction of her expenses did Mrs Lim spend on groceries and utility bills together? Give your answer in its simplest form.

\section*{Question 13}
\textbf{Metadata}

\begin{itemize}
  \item Question ID: P4-FrSubU12\_P2-FrCmp\_GPT4.1\_Household Finance\_02
  \item Primary KC: FRACTIONS | Subtraction | subtracting unlike fractions with two different denominators not exceeding 12
  \item Secondary KC: FRACTIONS | Comparison and ordering | comparing and ordering fractions
  \item Topic: Household finance such as income, utility bills, money, interest, savings, instalment, mortgage, financial planning etc.
  \item Grade: Primary 4
\end{itemize}

\textbf{Question}

Mrs Tan saved $\textdollar2/3$ of her monthly salary to pay for her household expenses. Later, she found out that her electricity and water bills together cost $\textdollar5/12$ of her salary. 

(a) What fraction of her salary is left for other expenses after paying the electricity and water bills? 

(b) Mrs Tan also compared how much she saved for household expenses with how much was spent on bills. Which is the greater amount, and by how much?

\section*{Question 14}
\textbf{Metadata}

\begin{itemize}
  \item Question ID: P4-DcAdd2d\_P4-DcCnv2Fr\_GPT4.1\_Household Finance\_02
  \item Primary KC: DECIMALS | Addition | adding decimals (up to 2 decimal places)
  \item Secondary KC: DECIMALS | Conversion from decimals to fraction | expressing decimals as fractions
  \item Topic: Household finance such as income, utility bills, money, interest, savings, instalment, mortgage, financial planning etc.
  \item Grade: Primary 4
\end{itemize}

\textbf{Question}

Mrs Tan wants to buy some fruits at the supermarket. She buys a packet of grapes for $ \textdollar3.40 $, an orange for $ \textdollar1.25 $ and a banana for $ \textdollar0.80 $. 

(a) What is the total amount Mrs Tan spent on fruits?

(b) Express the amount Mrs Tan spent on the banana as a fraction of a dollar.

\section*{Question 15}
\textbf{Metadata}

\begin{itemize}
  \item Question ID: P4-DcAdd2d\_P4-DcRnd3d\_GPT4.1\_Household Finance\_02
  \item Primary KC: DECIMALS | Addition | adding decimals (up to 2 decimal places)
  \item Secondary KC: DECIMALS | Rounding | rounding decimals up to 3 decimal places to the nearest whole number, 1 decimal place and 2 decimal places 
  \item Topic: Household finance such as income, utility bills, money, interest, savings, instalment, mortgage, financial planning etc.
  \item Grade: Primary 4
\end{itemize}

\textbf{Question}

Aisha helps her family keep track of their monthly household expenses. In June, they spent \textdollar73.45 on electricity, \textdollar42.38 on water, and \textdollar128.67 on groceries. What was their total household expense for June? Round your answer to the nearest dollar, to 1 decimal place, and to 2 decimal places.

\section*{Question 16}
\textbf{Metadata}

\begin{itemize}
  \item Question ID: P4-DcSub2d\_P4-DcAdd2nd\_GPT4.1\_Household Finance\_02
  \item Primary KC: DECIMALS | Subtraction | subtracting decimals (up to 2 decimal places)
  \item Secondary KC: DECIMALS | Addition | adding decimals
  \item Topic: Household finance such as income, utility bills, money, interest, savings, instalment, mortgage, financial planning etc.
  \item Grade: Primary 4
\end{itemize}

\textbf{Question}

Mrs. Tan had \textdollar120.40 in her savings account. She deposited \textdollar35.75 more into her account. Later, she used \textdollar28.90 from her account to pay her electricity bill. How much money does Mrs. Tan have in her savings account now?

\section*{Question 17}
\textbf{Metadata}

\begin{itemize}
  \item Question ID: P4-DcDiv2d1d\_P4-DcCnv2Fr\_GPT4.1\_Household Finance\_01
  \item Primary KC: DECIMALS | Division | dividing decimals (up to 2 decimal places) by a 1-digit whole number
  \item Secondary KC: DECIMALS | Conversion from decimals to fraction | expressing decimals as fractions
  \item Topic: Household finance such as income, utility bills, money, interest, savings, instalment, mortgage, financial planning etc.
  \item Grade: Primary 4
\end{itemize}

\textbf{Question}

Rachel paid her internet bill of $\textdollar15.36$ for April. She plans to split the cost equally with her 3 housemates. How much does each person need to pay? Express the amount each person pays as a decimal and also as a fraction in the simplest form.

\section*{Question 18}
\textbf{Metadata}

\begin{itemize}
  \item Question ID: P4-DcDiv2d1d\_P4-DcSub2nd\_GPT4.1\_Household Finance\_01
  \item Primary KC: DECIMALS | Division | dividing decimals (up to 2 decimal places) by a 1-digit whole number
  \item Secondary KC: DECIMALS | Subtraction | subtracting decimals
  \item Topic: Household finance such as income, utility bills, money, interest, savings, instalment, mortgage, financial planning etc.
  \item Grade: Primary 4
\end{itemize}

\textbf{Question}

Sarah saves \textdollar36.45 from her monthly allowance. She decides to divide this amount equally among her 5 family members as gifts. After giving out the money, she spends \textdollar0.75 from her own share on a drink. How much money does Sarah have left from her share after buying the drink?

\section*{Question 19}
\textbf{Metadata}

\begin{itemize}
  \item Question ID: P5-FrAddMix\_P3-FrSmp\_GPT4.1\_Household Finance\_01
  \item Primary KC: FRACTIONS | Addition | adding mixed numbers
  \item Secondary KC: FRACTIONS | Simplifying | expressing a fraction in its simplest form
  \item Topic: Household finance such as income, utility bills, money, interest, savings, instalment, mortgage, financial planning etc.
  \item Grade: Primary 5
\end{itemize}

\textbf{Question}

Lina is helping her mother check the water bills for two months. The bill for January shows that Lina’s family used $2\dfrac{1}{4}$ cubic metres of water. In February, they used $3\dfrac{2}{3}$ cubic metres of water. 

What is the total amount of water, in simplest form, that Lina’s family used in January and February?

\section*{Question 20}
\textbf{Metadata}

\begin{itemize}
  \item Question ID: P5-FrSubMix\_P2-FrAdd2nd\_GPT4.1\_Household Finance\_01
  \item Primary KC: FRACTIONS | Subtraction | subtracting mixed numbers
  \item Secondary KC: FRACTIONS | Addition | adding fractions
  \item Topic: Household finance such as income, utility bills, money, interest, savings, instalment, mortgage, financial planning etc.
  \item Grade: Primary 5
\end{itemize}

\textbf{Question}

Amy is checking her expenses for the month. She spent $2\dfrac{1}{3}$ dollars on water bills and $4\dfrac{2}{5}$ dollars on electricity bills. Later, she realises that she received a discount of $\dfrac{2}{5}$ dollar on her electricity bill. 

How much did she spend on the water and electricity bills in total after the discount? Give your answer as a mixed number in its simplest form.

\section*{Question 21}
\textbf{Metadata}

\begin{itemize}
  \item Question ID: P5-FrMulImN\_P2-FrSub2nd\_GPT4.1\_Household Finance\_01
  \item Primary KC: FRACTIONS | Multiplication | multiplying a proper/improper fraction and a whole number
  \item Secondary KC: FRACTIONS | Subtraction | subtracting fractions
  \item Topic: Household finance such as income, utility bills, money, interest, savings, instalment, mortgage, financial planning etc.
  \item Grade: Primary 5
\end{itemize}

\textbf{Question}

Mrs Lim bakes cakes to earn extra income at home. Each week, she spends \textdollar60 on baking ingredients. She earns $\frac{2}{3}$ of this amount selling cakes every week. After 4 weeks, Mrs Lim uses $\frac{1}{4}$ of her total earnings to pay her electricity bill. How much money does Mrs Lim have left after paying the bill?

\section*{Question 22}
\textbf{Metadata}

\begin{itemize}
  \item Question ID: P5-FrMulPIm\_P2-FrSub2nd\_GPT4.1\_Household Finance\_01
  \item Primary KC: FRACTIONS | Multiplication | multiplying a proper fraction and a proper/improper fractions
  \item Secondary KC: FRACTIONS | Subtraction | subtracting fractions
  \item Topic: Household finance such as income, utility bills, money, interest, savings, instalment, mortgage, financial planning etc.
  \item Grade: Primary 5
\end{itemize}

\textbf{Question}

Sarah spent $\frac{3}{5}$ of her monthly allowance on household items. Out of the amount spent, $\frac{2}{3}$ was used to pay the electricity bill and the rest was spent on groceries. If her monthly allowance is \textdollar120, how much more money did she spend on the electricity bill than on groceries?

\section*{Question 23}
\textbf{Metadata}

\begin{itemize}
  \item Question ID: P5-FrMulPIm\_P5-FrCnv2Dc\_GPT4.1\_Household Finance\_01
  \item Primary KC: FRACTIONS | Multiplication | multiplying a proper fraction and a proper/improper fractions
  \item Secondary KC: FRACTIONS | Conversion to decimals | expressing fractions as decimals
  \item Topic: Household finance such as income, utility bills, money, interest, savings, instalment, mortgage, financial planning etc.
  \item Grade: Primary 5
\end{itemize}

\textbf{Question}

Jenny saved $\textdollar80$ from her monthly allowance. She decided to spend $\frac{3}{5}$ of her savings to pay for the electricity bill. Out of the amount she used for the electricity bill, $\frac{2}{3}$ was paid using cash and the rest using her debit card. 

(a) How much did Jenny pay for the electricity bill?

(b) Express the amount she paid using cash as a decimal.

\section*{Question 24}
\textbf{Metadata}

\begin{itemize}
  \item Question ID: P5-FrMulMixN\_P2-FrCmp\_GPT4.1\_Household Finance\_01
  \item Primary KC: FRACTIONS | Multiplication | multiplying a mixed number and a whole number
  \item Secondary KC: FRACTIONS | Comparison and ordering | comparing and ordering fractions
  \item Topic: Household finance such as income, utility bills, money, interest, savings, instalment, mortgage, financial planning etc.
  \item Grade: Primary 5
\end{itemize}

\textbf{Question}

Jasmine wants to buy a set of new curtains for her living room. Each metre of curtain cloth costs \textdollar12. She needs $2\dfrac{1}{2}$ metres of cloth for her living room windows.\
\
(a) How much will Jasmine need to pay for the curtain cloth?\
\
Later, Jasmine looks at two payment options:\
- Option A: Pay the full amount upfront.\
- Option B: Pay three equal monthly instalments of \textdollar10 each month.\
\
(b) Arrange the two payment options in order from the least to greatest total amount paid. Which is the better option for Jasmine?

\section*{Question 25}
\textbf{Metadata}

\begin{itemize}
  \item Question ID: P5-FrMulMixN\_P3-FrSmp\_GPT4.1\_Household Finance\_01
  \item Primary KC: FRACTIONS | Multiplication | multiplying a mixed number and a whole number
  \item Secondary KC: FRACTIONS | Simplifying | expressing a fraction in its simplest form
  \item Topic: Household finance such as income, utility bills, money, interest, savings, instalment, mortgage, financial planning etc.
  \item Grade: Primary 5
\end{itemize}

\textbf{Question}

Mrs Tan bakes some loaves of bread to sell at the market. Each loaf sells for $\textdollar3. She sold $2\frac{1}{4}$ times as many loaves on Saturday than on Friday. If she sold 8 loaves on Friday, how much money did she collect from selling all the loaves over both days? Give your answer in its simplest form where necessary.

\section*{Question 26}
\textbf{Metadata}

\begin{itemize}
  \item Question ID: P5-FrMulMixN\_P5-FrCnv2Dc\_GPT4.1\_Household Finance\_01
  \item Primary KC: FRACTIONS | Multiplication | multiplying a mixed number and a whole number
  \item Secondary KC: FRACTIONS | Conversion to decimals | expressing fractions as decimals
  \item Topic: Household finance such as income, utility bills, money, interest, savings, instalment, mortgage, financial planning etc.
  \item Grade: Primary 5
\end{itemize}

\textbf{Question}

Mr Tan buys 3 bags of rice for his family every month. Each bag weighs $2 \dfrac{1}{2}$ kg. 

(a) What is the total weight of rice in kilograms Mr Tan buys in a month?

(b) Express the total weight as a decimal.

\section*{Question 27}
\textbf{Metadata}

\begin{itemize}
  \item Question ID: P5-DcMul3dK\_P4-DcRnd3d\_GPT4.1\_Household Finance\_01
  \item Primary KC: DECIMALS | Multiplication | multiplying decimals (up to 3 decimal places) by 10, 100, 1000 and their multiples
  \item Secondary KC: DECIMALS | Rounding | rounding decimals up to 3 decimal places to the nearest whole number, 1 decimal place and 2 decimal places 
  \item Topic: Household finance such as income, utility bills, money, interest, savings, instalment, mortgage, financial planning etc.
  \item Grade: Primary 5
\end{itemize}

\textbf{Question}

Kelly checks her electricity bill and sees that she used $132.768$ kilowatt-hours (kWh) of electricity in one month. Each kWh is charged at $\textdollar0.183$.

(a) How much did Kelly have to pay for electricity that month, before rounding? (Multiply the amount of electricity used by the price per kWh.)

(b) Round the calculated amount in part (a) to the nearest cent (2 decimal places), and also to the nearest dollar (whole number). Write both rounded amounts.

\section*{Question 28}
\textbf{Metadata}

\begin{itemize}
  \item Question ID: P5-DcDiv3dK\_P4-DcCnv2Fr\_GPT4.1\_Household Finance\_01
  \item Primary KC: DECIMALS | Division | dividing decimals (up to 3 decimal places) by 10, 100, 1000 and their multiples
  \item Secondary KC: DECIMALS | Conversion from decimals to fraction | expressing decimals as fractions
  \item Topic: Household finance such as income, utility bills, money, interest, savings, instalment, mortgage, financial planning etc.
  \item Grade: Primary 5
\end{itemize}

\textbf{Question}

Mrs. Tan saved $187.250 in her savings account. She decided to divide her savings equally into 100 envelopes to give as gifts for the upcoming festival. 

(a) How much money will Mrs. Tan put in each envelope? 

(b) Express the amount of money in each envelope as a fraction in its simplest form.

\section*{Question 29}
\textbf{Metadata}

\begin{itemize}
  \item Question ID: P5-RtFndR\_P2-DcCnvN2D\_GPT4.1\_Household Finance\_01
  \item Primary KC: RATE | Finding rate | finding rate given total amount and number of units
  \item Secondary KC: DECIMALS | Conversion to larger units | converting a measurement from a smaller unit to a larger unit in decimal form
  \item Topic: Household finance such as income, utility bills, money, interest, savings, instalment, mortgage, financial planning etc.
  \item Grade: Primary 5
\end{itemize}

\textbf{Question}

Mr. Lim received his water bill for the month, showing that he used a total of 7250 millilitres of water over 5 days. 

What was the average amount of water, in litres, that Mr. Lim used each day? (1 litre = 1000 millilitres)

\section*{Question 30}
\textbf{Metadata}

\begin{itemize}
  \item Question ID: P5-RtFndR\_P2-DcCnvD2N\_GPT4.1\_Household Finance\_01
  \item Primary KC: RATE | Finding rate | finding rate given total amount and number of units
  \item Secondary KC: DECIMALS | Conversion to smaller units | converting a measurement from a larger unit in decimal form to a smaller unit
  \item Topic: Household finance such as income, utility bills, money, interest, savings, instalment, mortgage, financial planning etc.
  \item Grade: Primary 5
\end{itemize}

\textbf{Question}

Mr Tan received his monthly electricity bill, which showed that his total electricity usage for the month was $52.3$ kilowatt-hours (kWh). The total amount he has to pay is \textdollar104.60. What is the cost per watt-hour (Wh), in cents, of electricity for this month? (Note: $1$ kilowatt-hour $= 1000$ watt-hours)

\section*{Question 31}
\textbf{Metadata}

\begin{itemize}
  \item Question ID: P5-RtFndT\_P2-DcCnvN2D\_GPT4.1\_Household Finance\_01
  \item Primary KC: RATE | Finding total amount | finding total amount, given rate and number of units
  \item Secondary KC: DECIMALS | Conversion to larger units | converting a measurement from a smaller unit to a larger unit in decimal form
  \item Topic: Household finance such as income, utility bills, money, interest, savings, instalment, mortgage, financial planning etc.
  \item Grade: Primary 5
\end{itemize}

\textbf{Question}

Mrs. Tan receives her electricity bill every month. The cost of electricity is $\textdollar 0.22$ per kilowatt-hour (kWh). Last month, her household used $32500$ watt-hours of electricity. What was the total amount Mrs. Tan had to pay for electricity last month? (Note: $1000$ watt-hours $= 1$ kWh)

\section*{Question 32}
\textbf{Metadata}

\begin{itemize}
  \item Question ID: P5-RtFndU\_P2-DcCnvD2N\_GPT4.1\_Household Finance\_01
  \item Primary KC: RATE | Finding number of unit | finding number of units given rate and total amount
  \item Secondary KC: DECIMALS | Conversion to smaller units | converting a measurement from a larger unit in decimal form to a smaller unit
  \item Topic: Household finance such as income, utility bills, money, interest, savings, instalment, mortgage, financial planning etc.
  \item Grade: Primary 5
\end{itemize}

\textbf{Question}

Mr. Tan checked his electricity bill and saw that he used $36.5$ kilowatt-hours (kWh) of electricity in June. Each kilowatt-hour costs \textdollar0.25. How much did Mr. Tan spend on electricity in June? Give your answer in cents.

\section*{Question 33}
\textbf{Metadata}

\begin{itemize}
  \item Question ID: P6-FrDivPN\_P3-FrSmp\_GPT4.1\_Household Finance\_01
  \item Primary KC: FRACTIONS | Division | dividing a proper fraction by a whole number
  \item Secondary KC: FRACTIONS | Simplifying | expressing a fraction in its simplest form
  \item Topic: Household finance such as income, utility bills, money, interest, savings, instalment, mortgage, financial planning etc.
  \item Grade: Primary 6
\end{itemize}

\textbf{Question}

Mrs Tan baked $6$ chocolate cakes for a family gathering. She wanted to share $\frac{3}{4}$ of a cake equally among her $3$ children after the gathering. How much cake did each child receive? Express your answer in its simplest form.

\section*{Question 34}
\textbf{Metadata}

\begin{itemize}
  \item Question ID: P6-FrDivPP\_P2-FrCmp\_GPT4.1\_Household Finance\_01
  \item Primary KC: FRACTIONS | Division | dividing a whole number/proper fraction by a proper fraction
  \item Secondary KC: FRACTIONS | Comparison and ordering | comparing and ordering fractions
  \item Topic: Household finance such as income, utility bills, money, interest, savings, instalment, mortgage, financial planning etc.
  \item Grade: Primary 6
\end{itemize}

\textbf{Question}

Mrs Tan has $\textdollar8$ worth of detergent and wants to share it equally among her friends. She pours detergent into bottles, with each bottle containing $\frac{2}{3}$ of a dollar's worth of detergent. 

(a) How many bottles can she fill?

(b) If she wants to compare this amount to another situation where each bottle contains $\frac{3}{4}$ of a dollar's worth of detergent, would she be able to fill more bottles in the first situation or the second? Justify your answer using a comparison of the fractions.

\section*{Question 35}
\textbf{Metadata}

\begin{itemize}
  \item Question ID: P6-FrDivPP\_P3-FrSmp\_GPT4.1\_Household Finance\_01
  \item Primary KC: FRACTIONS | Division | dividing a whole number/proper fraction by a proper fraction
  \item Secondary KC: FRACTIONS | Simplifying | expressing a fraction in its simplest form
  \item Topic: Household finance such as income, utility bills, money, interest, savings, instalment, mortgage, financial planning etc.
  \item Grade: Primary 6
\end{itemize}

\textbf{Question}

Mrs Lim bought a box of laundry detergent that contains $\dfrac{3}{4}$ litres in each packet. She has $6$ litres of detergent at home and wants to repack all the detergent into packets, using the same amount as in one packet from the box she bought.

(a) How many packets can Mrs Lim make with her $6$ litres of detergent?

(b) Express your answer in (a) as a fraction in its simplest form, if necessary.

\section*{Question 36}
\textbf{Metadata}

\begin{itemize}
  \item Question ID: P6-PcFndWN\_P1-WNSub2nd\_GPT4.1\_Household Finance\_01
  \item Primary KC: PERCENTAGE | Finding the whole | finding the whole given a part and the percentage
  \item Secondary KC: WHOLE NUMBERS | Subtraction | subtracting whole numbers
  \item Topic: Household finance such as income, utility bills, money, interest, savings, instalment, mortgage, financial planning etc.
  \item Grade: Primary 6
\end{itemize}

\textbf{Question}

Mr. Lim spent $\textdollar 108$ on his electricity bill last month. This amount was $45\%$ less than his total utility bill for the month, because he had already paid the rest for water and gas earlier. How much did Mr. Lim spend in total on utility bills last month?

\section*{Question 37}
\textbf{Metadata}

\begin{itemize}
  \item Question ID: P6-PcFndChg\_P1-WNSub2nd\_GPT4.1\_Household Finance\_01
  \item Primary KC: PERCENTAGE | Finding change | finding percentage increase/decrease
  \item Secondary KC: WHOLE NUMBERS | Subtraction | subtracting whole numbers
  \item Topic: Household finance such as income, utility bills, money, interest, savings, instalment, mortgage, financial planning etc.
  \item Grade: Primary 6
\end{itemize}

\textbf{Question}

Mrs Tan paid \textdollar200 for her monthly electricity bill in May. In June, her bill decreased to \textdollar170. What was the percentage decrease in Mrs Tan's electricity bill from May to June?

\section*{Question 38}
\textbf{Metadata}

\begin{itemize}
  \item Question ID: P6-PcFndChg\_P1-WNMul2nd\_GPT4.1\_Household Finance\_01
  \item Primary KC: PERCENTAGE | Finding change | finding percentage increase/decrease
  \item Secondary KC: WHOLE NUMBERS | Multiplication | multiplying whole numbers
  \item Topic: Household finance such as income, utility bills, money, interest, savings, instalment, mortgage, financial planning etc.
  \item Grade: Primary 6
\end{itemize}

\textbf{Question}

Mrs Lee's monthly electricity bill was \textdollar80 last month. This month, her electricity use increased, so her bill went up by 15\% compared to last month. If Mrs Lee also pays for water, and the cost for water for one month is 4 times her electricity bill for this month, what is the total amount she needs to pay for both electricity and water this month?

\section*{Question 39}
\textbf{Metadata}

\begin{itemize}
  \item Question ID: P6-RoFndDvqWN\_P1-WNAdd2nd\_GPT4.1\_Household Finance\_01
  \item Primary KC: RATIO | Finding divided quantities | dividing a given quantity in a given ratio
  \item Secondary KC: WHOLE NUMBERS | Addition | adding whole numbers
  \item Topic: Household finance such as income, utility bills, money, interest, savings, instalment, mortgage, financial planning etc.
  \item Grade: Primary 6
\end{itemize}

\textbf{Question}

Mrs Tan wants to split \textdollar480 equally among her three children for their monthly allowance. However, she decides to give them their allowances in the ratio $2:3:5$. 

(a) How much does each child receive?

(b) If Mrs Tan saves an additional \textdollar120 this month to add to the total amount given as allowances, what is the new total amount distributed to her children?

\section*{Question 40}
\textbf{Metadata}

\begin{itemize}
  \item Question ID: P6-RoFndRoWN\_P1-WNSub2nd\_GPT4.1\_Household Finance\_01
  \item Primary KC: RATIO | Finding ratio | finding the ratio of two or three given whole numbers
  \item Secondary KC: WHOLE NUMBERS | Subtraction | subtracting whole numbers
  \item Topic: Household finance such as income, utility bills, money, interest, savings, instalment, mortgage, financial planning etc.
  \item Grade: Primary 6
\end{itemize}

\textbf{Question}

Sarah has \textdollar210 saved from her monthly allowances. She decides to spend \textdollar80 on utility bills this month. The rest of her savings will be divided into two parts: one part for buying groceries and the other part for keeping as emergency savings. If the amount spent on groceries is twice the amount kept as emergency savings, find the ratio of the amount spent on utility bills, the amount spent on groceries, and the amount kept as emergency savings.

\section*{Question 41}
\textbf{Metadata}

\begin{itemize}
  \item Question ID: P6-RoFndRoWN\_P1-WNDiv2nd\_GPT4.1\_Household Finance\_01
  \item Primary KC: RATIO | Finding ratio | finding the ratio of two or three given whole numbers
  \item Secondary KC: WHOLE NUMBERS | Division | dividing whole numbers
  \item Topic: Household finance such as income, utility bills, money, interest, savings, instalment, mortgage, financial planning etc.
  \item Grade: Primary 6
\end{itemize}

\textbf{Question}

Mrs. Lee wants to divide \textdollar120 among her three children, Amy, Ben, and Carl, in the ratio $2:3:5$ according to their ages. How much does each child receive? Find the amount each child receives by first finding the correct ratio and then dividing the total money accordingly.

\section*{Question 42}
\textbf{Metadata}

\begin{itemize}
  \item Question ID: P6-RoFndTmWN\_P1-WNAdd2nd\_GPT4.1\_Household Finance\_01
  \item Primary KC: RATIO | Finding a missing term | finding the missing term in a pair of equivalent ratios
  \item Secondary KC: WHOLE NUMBERS | Addition | adding whole numbers
  \item Topic: Household finance such as income, utility bills, money, interest, savings, instalment, mortgage, financial planning etc.
  \item Grade: Primary 6
\end{itemize}

\textbf{Question}

Mr Lim spends his monthly income on rent and groceries in the ratio $5 : 2$. This month, after paying \textdollar1200 for rent and \textdollar400 for groceries, he also spends \textdollar300 on utility bills. If the ratio of his spending on rent to groceries remains the same as $5 : 2$, how much more would Mr Lim have to spend on groceries if his rent increased by \textdollar200 next month, keeping the ratio the same?

\section*{Question 43}
\textbf{Metadata}

\begin{itemize}
  \item Question ID: P6-AgRepLrEx\_P6-AgEvlLrEx\_GPT4.1\_Household Finance\_01
  \item Primary KC: ALGEBRA | Representation and concept | translation of real-world situations into linear algebraic expressions
  \item Secondary KC: ALGEBRA | Evaluation | evaluating linear expressions by substitution
  \item Topic: Household finance such as income, utility bills, money, interest, savings, instalment, mortgage, financial planning etc.
  \item Grade: Primary 6
\end{itemize}

\textbf{Question}

Sarah earns a basic monthly allowance of $\textdollar120$ from her parents. In addition, she receives $\textdollar3$ for every chore she completes in a month. Let $x$ represent the number of chores Sarah completes in a month.

(a) Write a linear algebraic expression in terms of $x$ to represent Sarah’s total allowance for a month.

(b) If Sarah completes $12$ chores in March, find her total allowance for the month.

\section*{Question 44}
\textbf{Metadata}

\begin{itemize}
  \item Question ID: P6-AgSlvLrN\_P6-AgRepLrEx\_GPT4.1\_Household Finance\_01
  \item Primary KC: ALGEBRA | Solving simple linear equations | solving linear equations involving whole number coefficient and one variable only
  \item Secondary KC: ALGEBRA | Representation and concept | translation of real-world situations into linear algebraic expressions
  \item Topic: Household finance such as income, utility bills, money, interest, savings, instalment, mortgage, financial planning etc.
  \item Grade: Primary 6
\end{itemize}

\textbf{Question}

Mr Tan pays a total of $\textdollar120$ every month for his household utility bills. This total includes a fixed internet subscription fee and an electricity bill. The electricity bill is $\textdollar30$ more than the internet subscription fee. Let $x$ represent the internet subscription fee in dollars. 

Form a linear equation and find the amount Mr Tan pays for his internet subscription each month.

\section*{Question 45}
\textbf{Metadata}

\begin{itemize}
  \item Question ID: O1-RoRepFr\_P2-FrSub2nd\_GPT4.1\_Household Finance\_01
  \item Primary KC: RATIO | Representation and concept | ratios involving fractions
  \item Secondary KC: FRACTIONS | Subtraction | subtracting fractions
  \item Topic: Household finance such as income, utility bills, money, interest, savings, instalment, mortgage, financial planning etc.
  \item Grade: Secondary O-level 1
\end{itemize}

\textbf{Question}

Aisha and her brother save their monthly allowance to help pay the household's utility bill. Last month, Aisha saved $\frac{3}{5}$ of her allowance and her brother saved $\frac{1}{2}$ of his allowance. The ratio of the amount Aisha saved to the amount her brother saved is $\frac{5}{6} : 1$. If Aisha decided to spend $\frac{1}{4}$ of her saved amount on a new book before contributing to the utility bill, what fraction of her original allowance does she finally contribute to the bill? Express your answer as a simplified fraction.

\section*{Question 46}
\textbf{Metadata}

\begin{itemize}
  \item Question ID: O1-RoRepFr\_P6-FrDiv2nd\_GPT4.1\_Household Finance\_01
  \item Primary KC: RATIO | Representation and concept | ratios involving fractions
  \item Secondary KC: FRACTIONS | Division | fraction division
  \item Topic: Household finance such as income, utility bills, money, interest, savings, instalment, mortgage, financial planning etc.
  \item Grade: Secondary O-level 1
\end{itemize}

\textbf{Question}

A mother wants to split her monthly electricity bill among her three children based on how much electricity each child uses. Andy uses $\frac{1}{4}$ of the total electricity, Bella uses $\frac{1}{3}$, and Charles uses the rest. The ratio of the amount each child should pay is equal to the ratio of the fractions of electricity each uses. The total electricity bill is \textdollar180.\
\text{(a) Find the fraction of electricity used by Charles.} \\
\text{(b) Express the ratio of Andy's, Bella's and Charles's usage in its simplest form.} \\
\text{(c) How much must Charles pay for the bill?}  

\section*{Question 47}
\textbf{Metadata}

\begin{itemize}
  \item Question ID: O1-RoRepDc\_P4-DcSub2nd\_GPT4.1\_Household Finance\_01
  \item Primary KC: RATIO | Representation and concept | ratios involving decimals
  \item Secondary KC: DECIMALS | Subtraction | subtracting decimals
  \item Topic: Household finance such as income, utility bills, money, interest, savings, instalment, mortgage, financial planning etc.
  \item Grade: Secondary O-level 1
\end{itemize}

\textbf{Question}

A family spends their monthly income on utilities and savings. The ratio of the amount spent on utilities to the amount saved is $0.6 : 1$. If the family saved \textdollar800 this month, how much did they spend on utilities? If their total income for the month was \textdollar1850, how much money was left after paying for utilities and savings?

\section*{Question 48}
\textbf{Metadata}

\begin{itemize}
  \item Question ID: O1-PcRep2q\_O1-PcCnv2Dc\_GPT4.1\_Household Finance\_02
  \item Primary KC: PERCENTAGE | Representation and concept | comparing two quantities by percentage
  \item Secondary KC: PERCENTAGE | Conversion to decimals | expressing percentage as a decimal
  \item Topic: Household finance such as income, utility bills, money, interest, savings, instalment, mortgage, financial planning etc.
  \item Grade: Secondary O-level 1
\end{itemize}

\textbf{Question}

Mrs Tan reviewed her family's monthly expenses. In June, she spent \textdollar 320 on electricity and \textdollar 400 on water. 

(a) By what percentage is the water bill higher than the electricity bill? Express your answer to 1 decimal place.

(b) Express your answer in (a) as a decimal, correct to 3 decimal places.

\section*{Question 49}
\textbf{Metadata}

\begin{itemize}
  \item Question ID: O1-PcFndRslt\_P1-WNDiv2nd\_GPT4.1\_Household Finance\_01
  \item Primary KC: PERCENTAGE | Finding result after change | increasing/decreasing a quantity by a given percentage
  \item Secondary KC: WHOLE NUMBERS | Division | dividing whole numbers
  \item Topic: Household finance such as income, utility bills, money, interest, savings, instalment, mortgage, financial planning etc.
  \item Grade: Secondary O-level 1
\end{itemize}

\textbf{Question}

Mr. Lim's monthly electricity bill was \textdollar180 last month. This month, the electricity company increased the rate by $15\%$. Mr. Lim decided to share the new bill equally among his 4 family members. How much does each person have to pay this month?

\section*{Question 50}
\textbf{Metadata}

\begin{itemize}
  \item Question ID: O1-AgSlvFrLr\_O1-AgRepEq\_GPT4.1\_Household Finance\_02
  \item Primary KC: ALGEBRA | Solving | solving simple fractional equations that can be reduced to linear equations
  \item Secondary KC: ALGEBRA | Representation and concept | translation of simple real-world situations to equations
  \item Topic: Household finance such as income, utility bills, money, interest, savings, instalment, mortgage, financial planning etc.
  \item Grade: Secondary O-level 1
\end{itemize}

\textbf{Question}

James notices that his monthly electricity bill is always three-fourths of his water bill. Last month, he paid a total of $\textdollar 140$ for both his electricity and water bills. Let $x$ represent the amount (in dollars) he paid for his water bill last month.  

(a) Write an equation to represent the total amount James paid using $x$.

(b) Solve the equation to find how much James paid for his water bill and how much he paid for his electricity bill.

\section*{Question 51}
\textbf{Metadata}

\begin{itemize}
  \item Question ID: O2-RoRepIvP\_P1-WNMul2nd\_GPT4.1\_Household Finance\_01
  \item Primary KC: RATIO | Representation and concept | inverse proportion
  \item Secondary KC: WHOLE NUMBERS | Multiplication | multiplying whole numbers
  \item Topic: Household finance such as income, utility bills, money, interest, savings, instalment, mortgage, financial planning etc.
  \item Grade: Secondary O-level 2
\end{itemize}

\textbf{Question}

A family is planning to buy energy-saving LED bulbs for their house. They want to share the total cost evenly among all family members. If 4 family members share the total cost, each pays \textdollar18. If the family buys the same number of bulbs but has only 3 members sharing the total cost, how much must each member pay? What is the total cost of the bulbs?

\section*{Question 52}
\textbf{Metadata}

\begin{itemize}
  \item Question ID: O2-RoRepIvP\_P1-WNDiv2nd\_GPT4.1\_Household Finance\_01
  \item Primary KC: RATIO | Representation and concept | inverse proportion
  \item Secondary KC: WHOLE NUMBERS | Division | dividing whole numbers
  \item Topic: Household finance such as income, utility bills, money, interest, savings, instalment, mortgage, financial planning etc.
  \item Grade: Secondary O-level 2
\end{itemize}

\textbf{Question}

A family has a fixed monthly budget of \textdollar1200 for their household expenses. The time it takes for them to finish their groceries depends on how many family members share the groceries, and is inversely proportional to the number of family members. If 4 family members share the groceries, the groceries last for 12 days. 

(a) How many days will the groceries last if 6 family members share them, assuming everyone eats the same amount?

(b) If the total cost of the groceries is divided equally among the 6 family members, how much does each person have to pay?

\section*{Question 53}
\textbf{Metadata}

\begin{itemize}
  \item Question ID: O2-AgSlvLr2v\_O1-AgRepEq\_GPT4.1\_Household Finance\_01
  \item Primary KC: ALGEBRA | Solving | solving linear equations in two variables
  \item Secondary KC: ALGEBRA | Representation and concept | translation of simple real-world situations to equations
  \item Topic: Household finance such as income, utility bills, money, interest, savings, instalment, mortgage, financial planning etc.
  \item Grade: Secondary O-level 2
\end{itemize}

\textbf{Question}

Aman and Bella share the monthly payment for their family's utility bills. The electricity bill and the water bill together amount to \textdollar180 each month. Last month, the electricity bill was \textdollar20 more than twice the amount of the water bill. 

Let $x$ be the electricity bill and $y$ be the water bill in dollars.

(a) Write down two equations based on the information given.

(b) Calculate the amount paid for the electricity bill and the water bill last month.

\section*{Question 54}
\textbf{Metadata}

\begin{itemize}
  \item Question ID: O2-SPFndmdn\_O2-SPFndmode\_GPT4.1\_Household Finance\_01
  \item Primary KC: STATISTICS AND PROBABILITY | Finding median | Finding median for a set of data
  \item Secondary KC: STATISTICS AND PROBABILITY | Finding mode | Finding mode for a set of data
  \item Topic: Household finance such as income, utility bills, money, interest, savings, instalment, mortgage, financial planning etc.
  \item Grade: Secondary O-level 2
\end{itemize}

\textbf{Question}

Ms. Tan recorded the total amount of money, in \textdollar, that her family spent on groceries each week over 7 weeks. The amounts are as follows:

\[\textdollar75, \textdollar60, \textdollar80, \textdollar60, \textdollar90, \textdollar75, \textdollar60\]

(a) What is the median amount spent by Ms. Tan's family on groceries per week?

(b) What is the mode of the amounts spent on groceries?

\section*{Question 55}
\textbf{Metadata}

\begin{itemize}
  \item Question ID: O2-SPFndmdn\_O3-SPFndPctl\_GPT4.1\_Household Finance\_01
  \item Primary KC: STATISTICS AND PROBABILITY | Finding median | Finding median for a set of data
  \item Secondary KC: STATISTICS AND PROBABILITY | Finding percentiles | finding percentiles for a set of data
  \item Topic: Household finance such as income, utility bills, money, interest, savings, instalment, mortgage, financial planning etc.
  \item Grade: Secondary O-level 2
\end{itemize}

\textbf{Question}

The monthly electricity bills (in \textdollar) for a family over ten months are as follows: 85, 90, 88, 93, 87, 91, 89, 95, 92, 86. 

(a) Find the median monthly electricity bill amount.

(b) Find the 70th percentile of the monthly electricity bills.

\section*{Question 56}
\textbf{Metadata}

\begin{itemize}
  \item Question ID: O3-BPOpr\_O3-BPRepPosI\_GPT4.1\_Household Finance\_01
  \item Primary KC: BASE AND POWER | Operations | laws of indices
  \item Secondary KC: BASE AND POWER | Representation and concept  | positive indices that is not 1
  \item Topic: Household finance such as income, utility bills, money, interest, savings, instalment, mortgage, financial planning etc.
  \item Grade: Secondary O-level 3/4
\end{itemize}

\textbf{Question}

Mrs Tan is planning to upgrade appliances in her house to save on electricity bills. She considers two payment options for a new air-conditioner which costs \textdollar1200.

Option A: Pay the full amount at once.

Option B: Pay \textdollar300 as a down payment and the remaining amount in 3 equal monthly instalments. The shop charges an interest rate of $2^2\%$ per month (where $2^2$ means the power of 2 with index 2).

(a) What is the monthly interest rate in percentage?

(b) Calculate the total amount paid if Mrs Tan chooses Option B. (Give your final answer in terms of indices first, then calculate the value.)

(c) By how much is Option B more expensive than Option A?

\section*{Question 57}
\textbf{Metadata}

\begin{itemize}
  \item Question ID: O3-BPOpr\_O3-BPRepFrI\_GPT4.1\_Household Finance\_01
  \item Primary KC: BASE AND POWER | Operations | laws of indices
  \item Secondary KC: BASE AND POWER | Representation and concept  | fractional indices
  \item Topic: Household finance such as income, utility bills, money, interest, savings, instalment, mortgage, financial planning etc.
  \item Grade: Secondary O-level 3/4
\end{itemize}

\textbf{Question}

Mr. Tan is considering two savings plans offered by a local bank to grow his household emergency fund. Plan A offers an interest rate that causes his savings to grow according to the formula $A = 5000 \times 2^{\frac{3}{2}}$ after a certain period, where $5000$ is the initial amount. Plan B offers a different interest rate and results in his savings growing to $B = 5000 \times (8)^{\frac{1}{3}}$ after the same period.\
\\
(a) Without using a calculator, simplify both expressions for $A$ and $B$ using the laws of indices and knowledge of fractional indices.\
\\
(b) Which plan will give Mr. Tan a greater amount in his emergency fund after this period?\\


\section*{Question 58}
\textbf{Metadata}

\begin{itemize}
  \item Question ID: O3-STOprUn\_O3-STOprIns\_GPT4.1\_Household Finance\_01
  \item Primary KC: SET | Set operations | union of two sets
  \item Secondary KC: SET | Set operations | intersection of two sets
  \item Topic: Household finance such as income, utility bills, money, interest, savings, instalment, mortgage, financial planning etc.
  \item Grade: Secondary O-level 3/4
\end{itemize}

\textbf{Question}

In a study of 50 households in a neighbourhood, it was found that 32 households paid for electricity bills using online banking, and 20 households paid for water bills using online banking. There are 12 households that paid for both electricity and water bills using online banking. 

(a) How many households paid for either electricity or water bills using online banking? 

(b) How many households paid for neither electricity nor water bills using online banking?

\section*{Question 59}
\textbf{Metadata}

\begin{itemize}
  \item Question ID: O3-MXMulSM\_O3-MXSub\_GPT4.1\_Household Finance\_01
  \item Primary KC: MATRICES | Multiplication | product of a scalar quantity and a matrix
  \item Secondary KC: MATRICES | Subtraction | subtraction of matrices
  \item Topic: Household finance such as income, utility bills, money, interest, savings, instalment, mortgage, financial planning etc.
  \item Grade: Secondary O-level 3/4
\end{itemize}

\textbf{Question}

Jenny tracks her household expenses in two separate months using matrices. In January, her expenses on rent, utilities, and groceries can be represented by the matrix $A = \begin{bmatrix} 1200 \\ 200 \\ 400 \end{bmatrix}$, where each entry is in \textdollar. In February, each of these expenses increased by 10\%.\
\
(a) Write a matrix that represents her February expenses by multiplying January's matrix $A$ by a suitable scalar.\\
(b) If Jenny managed to use discount coupons in February and reduced her utilities and groceries expenses by \textdollar30 and \textdollar50 respectively before paying, write a matrix $B$ that shows her actual payment for February after the discounts.\\
(c) Calculate the matrix representing the change in expenses from January to February after applying the discounts, by subtracting January’s expense matrix $A$ from the discounted February payment matrix $B$.

\section*{Question 60}
\textbf{Metadata}

\begin{itemize}
  \item Question ID: O3-MXMulSM\_O3-MXMul\_GPT4.1\_Household Finance\_01
  \item Primary KC: MATRICES | Multiplication | product of a scalar quantity and a matrix
  \item Secondary KC: MATRICES | Multiplication | multiplication of matrices
  \item Topic: Household finance such as income, utility bills, money, interest, savings, instalment, mortgage, financial planning etc.
  \item Grade: Secondary O-level 3/4
\end{itemize}

\textbf{Question}

A family tracks the monthly costs (in \textdollar) of three household utilities: electricity, water, and gas. The total monthly usage is represented by the column matrix $A = \begin{bmatrix} 120 \\ 45 \\ 30 \end{bmatrix}$, where the entries represent the amounts used for electricity, water, and gas respectively, in units. The respective rates per unit (in \textdollar) for electricity, water, and gas are given in the row matrix $B = \begin{bmatrix} 0.20 & 0.30 & 0.50 \end{bmatrix}$. During the summer months, the electricity company imposes a surcharge of $1.5$ times the usual rate for electricity, while the rates for water and gas remain unchanged. 

(a) Represent the new rates per unit in matrix form after the surcharge is applied as a scalar multiplication of the electricity rate in $B$.

(b) Compute the total monthly cost for the three utilities in the summer month by matrix multiplication.

\section*{Question 61}
\textbf{Metadata}

\begin{itemize}
  \item Question ID: O3-MXMul\_O3-MXSub\_GPT4.1\_Household Finance\_01
  \item Primary KC: MATRICES | Multiplication | multiplication of matrices
  \item Secondary KC: MATRICES | Subtraction | subtraction of matrices
  \item Topic: Household finance such as income, utility bills, money, interest, savings, instalment, mortgage, financial planning etc.
  \item Grade: Secondary O-level 3/4
\end{itemize}

\textbf{Question}

A family is planning their household expenses for two consecutive months. The breakdown of their expenses for food, utilities, and transportation (in \textdollar) for January is represented by matrix $A$, and for February by matrix $B$:  

$A = \begin{bmatrix} 350 & 120 \\ 100 & 80 \\ 130 & 60 \end{bmatrix}$ and $B = \begin{bmatrix} 370 & 150 \\ 110 & 90 \\ 140 & 70 \end{bmatrix}$, where the first column in each matrix represents expenses for the father and the second column represents the mother.  

The family decides to calculate the month-on-month change in their expenses for each person by subtracting January's expenses ($A$) from February's expenses ($B$) to get matrix $C$ ($C = B - A$).  

They then want to find the total change in expenses for each expense category (food, utilities, transportation) combined for both parents. Represent the combined expenses for January and February as row vectors, then find the change in combined expenses (February minus January) for each category using matrix multiplication. 

What is the matrix $C$, and what is the resulting row vector showing the change in combined expenses for each category?

\section*{Question 62}
\textbf{Metadata}

\begin{itemize}
  \item Question ID: O3-SPFndQtl\_O3-SPFndIQR\_GPT4.1\_Household Finance\_01
  \item Primary KC: STATISTICS AND PROBABILITY | Finding quartiles | finding quartiles for a set of data
  \item Secondary KC: STATISTICS AND PROBABILITY | Finding range | finding interquartile range as measures of spread for a set of data 
  \item Topic: Household finance such as income, utility bills, money, interest, savings, instalment, mortgage, financial planning etc.
  \item Grade: Secondary O-level 3/4
\end{itemize}

\textbf{Question}

The monthly electricity bills (in \textdollar) for 10 households in an HDB block over the past month were as follows: $67,\ 82,\ 74,\ 89,\ 100,\ 76,\ 92,\ 86,\ 95,\ 78$.\
\
(a) Find the lower quartile ($Q_1$), median ($Q_2$), and upper quartile ($Q_3$) for the set of electricity bills.\
\
(b) Calculate the interquartile range (IQR) for this set of data.

\section*{Question 63}
\textbf{Metadata}

\begin{itemize}
  \item Question ID: O3-SPMulProb\_O2-SPRepPrSE\_GPT4.1\_Household Finance\_01
  \item Primary KC: STATISTICS AND PROBABILITY | Multiplication | multiplication of probabilities
  \item Secondary KC: STATISTICS AND PROBABILITY | Representation and concept | probability of single events
  \item Topic: Household finance such as income, utility bills, money, interest, savings, instalment, mortgage, financial planning etc.
  \item Grade: Secondary O-level 3/4
\end{itemize}

\textbf{Question}

Maya plans to save some money for her upcoming school camp. Each week, she decides whether to save some money based on whether she receives any allowance from her parents. The probability that Maya receives an allowance in a particular week is $0.7$. If she receives an allowance in a week, the probability that she saves part of it is $0.8$. 

(a) What is the probability that in a given week, Maya both receives an allowance and saves part of it?

(b) If Maya does not receive an allowance in a given week, what is the probability she saves part of it?

\end{document}
