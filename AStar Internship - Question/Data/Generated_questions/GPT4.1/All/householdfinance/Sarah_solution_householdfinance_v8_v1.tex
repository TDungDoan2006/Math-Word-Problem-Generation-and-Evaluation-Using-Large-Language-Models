\documentclass{article}
\usepackage[utf8]{inputenc}
\usepackage{amsmath}
\usepackage{amsfonts}
\usepackage{amssymb}
\usepackage{graphicx}
\usepackage{hyperref}
\title{'Sarah Solutions household finance v8 v1'}
\author{Tien Dung Doan}
\begin{document}
\maketitle
\section*{Question 1}
\textbf{Metadata}

\begin{itemize}
  \item Question ID: P3-WNDivRmd3d\_P1-WNCmp\_GPT4.1\_Household Finance\_02
  \item Primary KC: WHOLE NUMBERS | Division | dividing whole numbers up to 3 digits by 1 digit with remainder 
  \item Secondary KC: WHOLE NUMBERS | Comparison and ordering | comparing and ordering whole numbers
  \item Topic: Household finance such as income, utility bills, money, interest, savings, instalment, mortgage, financial planning etc.
  \item Grade: Primary 3
\end{itemize}

\textbf{Solution}

(a) \newline
To find out how much each family member will receive from $275, we divide $275 by 4: 

\[
275 \div 4 = 68 \text{ remainder } 3
\]

Each family member will receive \textdollar68, and there will be \textdollar3 left over.

(b) \newline
Now, dividing $189 among 4 family members:
\[
189 \div 4 = 47 \text{ remainder } 1
\]

Each family member would receive \textdollar47, with \textdollar1 left over.

Comparison:
Each person gets \textdollar68 from $275, or \textdollar47 from $189. \textdollar68$ is greater than \textdollar47.

Arranging the amounts from greatest to least based on what each family member would receive:
\[
68 > 47
\]
So, $275 divided among 4 people gives each person more money than $189 divided among 4 people.

\section*{Question 2}
\textbf{Metadata}

\begin{itemize}
  \item Question ID: P3-WNMul3d1d\_P1-WNAdd2nd\_GPT4.1\_Household Finance\_02
  \item Primary KC: WHOLE NUMBERS | Multiplication | multiplying whole numbers up to 3 digits by 1 digit
  \item Secondary KC: WHOLE NUMBERS | Addition | adding whole numbers
  \item Topic: Household finance such as income, utility bills, money, interest, savings, instalment, mortgage, financial planning etc.
  \item Grade: Primary 3
\end{itemize}

\textbf{Solution}

First, find out how many light bulbs are in the 4 boxes:

$125 \times 4 = 500$

Next, add the 37 light bulbs she already has:

$500 + 37 = 537$

So, Mrs. Lim has $537$ light bulbs altogether.

\section*{Question 3}
\textbf{Metadata}

\begin{itemize}
  \item Question ID: P3-WNMul3d1d\_P1-WNSub2nd\_GPT4.1\_Household Finance\_02
  \item Primary KC: WHOLE NUMBERS | Multiplication | multiplying whole numbers up to 3 digits by 1 digit
  \item Secondary KC: WHOLE NUMBERS | Subtraction | subtracting whole numbers
  \item Topic: Household finance such as income, utility bills, money, interest, savings, instalment, mortgage, financial planning etc.
  \item Grade: Primary 3
\end{itemize}

\textbf{Solution}

First, find the total amount Sarah receives after 5 weeks: 

$8 \times 5 = 40$

Sarah receives \textdollar40 after 5 weeks.

Next, subtract the amount she spends on the pencil case:

$40 - 12 = 28$

Sarah has \textdollar28 left after 5 weeks.

\section*{Question 4}
\textbf{Metadata}

\begin{itemize}
  \item Question ID: P3-FrAddRl12\_P3-FrSmp\_GPT4.1\_Household Finance\_02
  \item Primary KC: FRACTIONS | Addition | adding two related fractions within one whole with denominators of given fractions not exceeding 12
  \item Secondary KC: FRACTIONS | Simplifying | expressing a fraction in its simplest form
  \item Topic: Household finance such as income, utility bills, money, interest, savings, instalment, mortgage, financial planning etc.
  \item Grade: Primary 3
\end{itemize}

\textbf{Solution}

First, add the fractions:

$\frac{3}{12}$ (electricity) $+$ $\frac{5}{12}$ (water) $=$ $\frac{3+5}{12} = \frac{8}{12}$.

Next, simplify $\frac{8}{12}$ by dividing both numerator and denominator by the greatest common factor, which is 4.

$\frac{8 \div 4}{12 \div 4} = \frac{2}{3}$.

So, Lina's family spent $\frac{2}{3}$ of their total income on both electricity and water bills together.

\section*{Question 5}
\textbf{Metadata}

\begin{itemize}
  \item Question ID: P3-FrSubRl12\_P3-FrSmp\_GPT4.1\_Household Finance\_02
  \item Primary KC: FRACTIONS | Subtraction | subtracting two related fractions within one whole with denominators of given fractions not exceeding 12
  \item Secondary KC: FRACTIONS | Simplifying | expressing a fraction in its simplest form
  \item Topic: Household finance such as income, utility bills, money, interest, savings, instalment, mortgage, financial planning etc.
  \item Grade: Primary 3
\end{itemize}

\textbf{Solution}

Amount of pocket money Jenny spent in total: $\frac{7}{12} + \frac{3}{12} = \frac{10}{12}$. 

Fraction of pocket money not spent: $1 - \frac{10}{12} = \frac{12}{12} - \frac{10}{12} = \frac{2}{12}$.

Simplifying $\frac{2}{12}$, we get $\frac{2 \div 2}{12 \div 2} = \frac{1}{6}$.

Therefore, Jenny had not spent $\frac{1}{6}$ of her pocket money.

\section*{Question 6}
\textbf{Metadata}

\begin{itemize}
  \item Question ID: P4-WNMul4d1d\_P1-WNSub2nd\_GPT4.1\_Household Finance\_02
  \item Primary KC: WHOLE NUMBERS | Multiplication | multiplying whole numbers up to 4 digits by 1 digit or up to 3 digits by 2 digits
  \item Secondary KC: WHOLE NUMBERS | Subtraction | subtracting whole numbers
  \item Topic: Household finance such as income, utility bills, money, interest, savings, instalment, mortgage, financial planning etc.
  \item Grade: Primary 4
\end{itemize}

\textbf{Solution}

(a) To find the total cost of 5 chairs, multiply the cost of one chair by 5:

\[
\text{Total cost} = 245 \times 5 = 1225
\]

Mrs Tan needs to pay \textdollar1225 for 5 chairs.

(b) To find the amount of money Mrs Tan has left, subtract the total cost from her savings:

\[
\text{Money left} = 1200 - 1225 = -25
\]

Since the answer is negative, Mrs Tan does not have enough money. She needs \textdollar25 more to buy all 5 chairs.

\section*{Question 7}
\textbf{Metadata}

\begin{itemize}
  \item Question ID: P4-WNDiv4d1d\_P1-WNAdd2nd\_GPT4.1\_Household Finance\_02
  \item Primary KC: WHOLE NUMBERS | Division | dividing whole numbers up to 4 digits by 1 digit
  \item Secondary KC: WHOLE NUMBERS | Addition | adding whole numbers
  \item Topic: Household finance such as income, utility bills, money, interest, savings, instalment, mortgage, financial planning etc.
  \item Grade: Primary 4
\end{itemize}

\textbf{Solution}

First, find how much each person received. There are 4 people in total (Peter and his 3 siblings). Divide the earnings:

\[\frac{2892}{4} = 723\]

So, each person received \textdollar723.

Combined, they have:

\[723 \times 4 = 2892\]

Add the extra savings:

\[2892 + 108 = 3000\]

So, they have a total of \textdollar3000 to buy the washing machine.

\section*{Question 8}
\textbf{Metadata}

\begin{itemize}
  \item Question ID: P4-FrRepSet\_P3-FrCnvEq\_GPT4.1\_Household Finance\_02
  \item Primary KC: FRACTIONS | Representation and concept | expressing a part of a set as a fraction
  \item Secondary KC: FRACTIONS | Conversion to equivalent fractions | Conversion to equivalent fractions (given either the denominator or the numerator)
  \item Topic: Household finance such as income, utility bills, money, interest, savings, instalment, mortgage, financial planning etc.
  \item Grade: Primary 4
\end{itemize}

\textbf{Solution}

(a) Since Maya and her sister shared the jar equally, Maya had $\frac{1}{2}$ of the coins in the jar.

She gave $\frac{3}{8}$ of her share, so she gave:
$\frac{3}{8} \times \frac{1}{2} = \frac{3}{16}$

So, Maya gave $\frac{3}{16}$ of the whole jar to charity.

(b) With a denominator of $16$, the answer from (a) is already $\frac{3}{16}$, which is equivalent.

\textbf{Final Answers:}
(a) $\frac{3}{16}$

(b) $\frac{3}{16}$

\section*{Question 9}
\textbf{Metadata}

\begin{itemize}
  \item Question ID: P4-FrSubU12\_P2-FrAdd2nd\_GPT4.1\_Household Finance\_02
  \item Primary KC: FRACTIONS | Subtraction | subtracting unlike fractions with two different denominators not exceeding 12
  \item Secondary KC: FRACTIONS | Addition | adding fractions
  \item Topic: Household finance such as income, utility bills, money, interest, savings, instalment, mortgage, financial planning etc.
  \item Grade: Primary 4
\end{itemize}

\textbf{Solution}

First, add the fractions spent on groceries and utility bills:

\[
\frac{3}{4} + \frac{2}{3}
\]

Find a common denominator for 4 and 3, which is 12:

\[
\frac{3}{4} = \frac{9}{12}, \quad \frac{2}{3} = \frac{8}{12}
\]

Add them:

\[
\frac{9}{12} + \frac{8}{12} = \frac{17}{12}
\]

Sarah spent $\frac{17}{12}$ of her pocket money, which means she actually spent more than her pocket money (probably used savings or borrowed).

Next, subtract this amount from 1 (her total pocket money) to find out what she has left:

\[
1 - \frac{17}{12}
\]

Rewrite 1 as $\frac{12}{12}$:

\[
\frac{12}{12} - \frac{17}{12} = -\frac{5}{12}
\]

Sarah has $-\frac{5}{12}$ of her pocket money left, which means she used all her pocket money and also spent $\frac{5}{12}$ more, possibly from savings or borrowing.

**Final Answers:**

- Total fraction spent: $\frac{17}{12}$
- Fraction left: $-\frac{5}{12}$ (which means she overspent by $\frac{5}{12}$ of her weekly pocket money)

\section*{Question 10}
\textbf{Metadata}

\begin{itemize}
  \item Question ID: P4-DcSub2d\_P4-DcCmp3d\_GPT4.1\_Household Finance\_02
  \item Primary KC: DECIMALS | Subtraction | subtracting decimals (up to 2 decimal places)
  \item Secondary KC: DECIMALS | Comparison and ordering | comparing and ordering decimals up to 3 decimal places
  \item Topic: Household finance such as income, utility bills, money, interest, savings, instalment, mortgage, financial planning etc.
  \item Grade: Primary 4
\end{itemize}

\textbf{Solution}

(a) Let's compare the amounts:
\begin{align*}
\$54.28 &= 54.280 \text{ (up to 3 decimal places)} \\
\$54.105 &= 54.105 \\
\$53.99 &= 53.990
\end{align*}

Order from smallest to largest:
\[
\textdollar53.99,\ \textdollar54.105,\ \textdollar54.28
\]

(b) The highest bill is $\textdollar54.28$, and the lowest bill is $\textdollar53.99$.

Amount paid less for the lowest bill compared to the highest bill:
\[
\textdollar54.28 - \textdollar53.99 = \textdollar0.29
\]

Therefore, Mrs Tan paid \textdollar0.29 less for the lowest bill compared to the highest bill this month.

\section*{Question 11}
\textbf{Metadata}

\begin{itemize}
  \item Question ID: P4-DcMul2d1d\_P4-DcCmp3d\_GPT4.1\_Household Finance\_01
  \item Primary KC: DECIMALS | Multiplication | multiplying decimals (up to 2 decimal places) by a 1-digit whole number
  \item Secondary KC: DECIMALS | Comparison and ordering | comparing and ordering decimals up to 3 decimal places
  \item Topic: Household finance such as income, utility bills, money, interest, savings, instalment, mortgage, financial planning etc.
  \item Grade: Primary 4
\end{itemize}

\textbf{Solution}

First, calculate the total cost of each type of item:

For rice:
$3 \times 4.25 = 12.75$

For milk:
$2 \times 2.49 = 4.98$

Cost of the box of cookies is $7.350$.

List the costs to compare:
Rice (one packet): $4.25$
Milk (one carton): $2.49$
Cookies (one box): $7.350$

Comparing the prices:
$2.49 < 4.25 < 7.350$
So, the carton of milk is the least expensive, and the box of cookies is the most expensive.

Now, find the total amount Sarah needs if she buys all the items:
Total = $12.75$ (rice) $+ 4.98$ (milk) $+ 7.350$ (cookies) $= 25.08$

Since $25.08 > 20$, Sarah does not have enough money to buy all the items.

\textbf{Final Answers:}
\begin{itemize}
    \item The box of cookies is the most expensive, the carton of milk is the least expensive.
    \item No, Sarah cannot afford to buy all the groceries, including the box of cookies, with her \textdollar20.
\end{itemize}

\section*{Question 12}
\textbf{Metadata}

\begin{itemize}
  \item Question ID: P4-DcMul2d1d\_P4-DcRnd3d\_GPT4.1\_Household Finance\_01
  \item Primary KC: DECIMALS | Multiplication | multiplying decimals (up to 2 decimal places) by a 1-digit whole number
  \item Secondary KC: DECIMALS | Rounding | rounding decimals up to 3 decimal places to the nearest whole number, 1 decimal place and 2 decimal places 
  \item Topic: Household finance such as income, utility bills, money, interest, savings, instalment, mortgage, financial planning etc.
  \item Grade: Primary 4
\end{itemize}

\textbf{Solution}

(a) Mrs Tan bought $7$ litres of petrol each week for $4$ weeks, so in total, she bought:

$7 \times 4 = 28$ litres.

The cost per litre is \textdollar1.36, so the total amount spent is:

$28 \times 1.36 = 38.08$

Therefore, Mrs Tan spent \textdollar38.08 on petrol last month.

(b) Rounding $38.08$:

To the nearest whole number: $38.08$ rounds to $38$.

To 1 decimal place: $38.08$ rounds to $38.1$.

To 2 decimal places: $38.08$ is already $38.08$.

\textbf{Final answers:}

- Nearest whole number: \textdollar38
- 1 decimal place: \textdollar38.1
- 2 decimal places: \textdollar38.08

\section*{Question 13}
\textbf{Metadata}

\begin{itemize}
  \item Question ID: P4-DcMul2d1d\_P4-DcSub2nd\_GPT4.1\_Household Finance\_01
  \item Primary KC: DECIMALS | Multiplication | multiplying decimals (up to 2 decimal places) by a 1-digit whole number
  \item Secondary KC: DECIMALS | Subtraction | subtracting decimals
  \item Topic: Household finance such as income, utility bills, money, interest, savings, instalment, mortgage, financial planning etc.
  \item Grade: Primary 4
\end{itemize}

\textbf{Solution}

First, we find the total cost of 3 packs of apples without the discount:

$2.75 \times 3 = 8.25$

Next, Maya saves \textdollar1.20 with her voucher, so we subtract the savings from the total cost:

$8.25 - 1.20 = 7.05$

Thus, Maya pays \textdollar7.05 after using the voucher.

\section*{Question 14}
\textbf{Metadata}

\begin{itemize}
  \item Question ID: P4-DcDiv2d1d\_P4-DcRnd3d\_GPT4.1\_Household Finance\_01
  \item Primary KC: DECIMALS | Division | dividing decimals (up to 2 decimal places) by a 1-digit whole number
  \item Secondary KC: DECIMALS | Rounding | rounding decimals up to 3 decimal places to the nearest whole number, 1 decimal place and 2 decimal places 
  \item Topic: Household finance such as income, utility bills, money, interest, savings, instalment, mortgage, financial planning etc.
  \item Grade: Primary 4
\end{itemize}

\textbf{Solution}

First, divide the total bill among the 4 people:

\[
\frac{83.76}{4} = 20.94
\]

So, each person pays \textdollar20.94 (rounded to 2 decimal places).

(a) Answer: \textdollar20.94

(b)
To the nearest whole number:
\[
20.94 \approx 21
\]
To the nearest 1 decimal place:
\[
20.94 \approx 20.9
\]

Final answers:
- Nearest whole number: \textdollar21
- Nearest 1 decimal place: \textdollar20.9

\section*{Question 15}
\textbf{Metadata}

\begin{itemize}
  \item Question ID: P5-FrAddMix\_P5-FrCnv2Dc\_GPT4.1\_Household Finance\_01
  \item Primary KC: FRACTIONS | Addition | adding mixed numbers
  \item Secondary KC: FRACTIONS | Conversion to decimals | expressing fractions as decimals
  \item Topic: Household finance such as income, utility bills, money, interest, savings, instalment, mortgage, financial planning etc.
  \item Grade: Primary 5
\end{itemize}

\textbf{Solution}

(a) 
Let us add the two mixed numbers:

$2\frac{3}{4} + 1\frac{2}{5}$

First, convert the mixed numbers to improper fractions:

$2\frac{3}{4} = \frac{2 \times 4 + 3}{4} = \frac{8 + 3}{4} = \frac{11}{4}$

$1\frac{2}{5} = \frac{1 \times 5 + 2}{5} = \frac{5 + 2}{5} = \frac{7}{5}$

To add $\frac{11}{4}$ and $\frac{7}{5}$, find common denominator (LCM of 4 and 5 is 20):

$\frac{11}{4} = \frac{11 \times 5}{4 \times 5} = \frac{55}{20}$
$\frac{7}{5} = \frac{7 \times 4}{5 \times 4} = \frac{28}{20}$

Add:
$\frac{55}{20} + \frac{28}{20} = \frac{83}{20}$

Convert $\frac{83}{20}$ to a mixed number:

$20$ goes into $83$ four times ($4 \times 20 = 80$) with a remainder of $3$. 
So $\frac{83}{20} = 4\frac{3}{20}$.

\textbf{Answer to (a):} 
Mr. Tan spent $4\frac{3}{20}$ hours in total.

(b) 
Express $4\frac{3}{20}$ as a decimal correct to 2 decimal places.

$\frac{3}{20} = 0.15$

Therefore, $4\frac{3}{20} = 4.15$

\textbf{Answer to (b):} 
Mr. Tan spent $4.15$ hours in total.

\section*{Question 16}
\textbf{Metadata}

\begin{itemize}
  \item Question ID: P5-FrSubMix\_P3-FrSmp\_GPT4.1\_Household Finance\_01
  \item Primary KC: FRACTIONS | Subtraction | subtracting mixed numbers
  \item Secondary KC: FRACTIONS | Simplifying | expressing a fraction in its simplest form
  \item Topic: Household finance such as income, utility bills, money, interest, savings, instalment, mortgage, financial planning etc.
  \item Grade: Primary 5
\end{itemize}

\textbf{Solution}

To find out how much more Amanda spent on electricity than water, we subtract the water bill from the electricity bill:

$3\dfrac{2}{5} - 1\dfrac{3}{10}$

First, write the mixed numbers as improper fractions:
$3\dfrac{2}{5} = \dfrac{3 \times 5 + 2}{5} = \dfrac{17}{5}$

$1\dfrac{3}{10} = \dfrac{1 \times 10 + 3}{10} = \dfrac{13}{10}$

Now subtract:
$\dfrac{17}{5} - \dfrac{13}{10}$
Find a common denominator:
LCM of $5$ and $10$ is $10$.

$\dfrac{17}{5} = \dfrac{17 \times 2}{5 \times 2} = \dfrac{34}{10}$

$\dfrac{34}{10} - \dfrac{13}{10} = \dfrac{21}{10}$

Express $\dfrac{21}{10}$ as a mixed number:
$\dfrac{21}{10} = 2\dfrac{1}{10}$

Thus, Amanda spent $2\dfrac{1}{10}$\textdollar~more on electricity than on water.

\section*{Question 17}
\textbf{Metadata}

\begin{itemize}
  \item Question ID: P5-FrSubMix\_P5-FrCnv2Dc\_GPT4.1\_Household Finance\_01
  \item Primary KC: FRACTIONS | Subtraction | subtracting mixed numbers
  \item Secondary KC: FRACTIONS | Conversion to decimals | expressing fractions as decimals
  \item Topic: Household finance such as income, utility bills, money, interest, savings, instalment, mortgage, financial planning etc.
  \item Grade: Primary 5
\end{itemize}

\textbf{Solution}

(a) To find how much cooking oil Mrs Tan used, we subtract the amount left from the starting amount:

$3\frac{1}{4} - 1\frac{2}{5}$

First, convert the mixed numbers to improper fractions:

$3\frac{1}{4} = \frac{13}{4}$

$1\frac{2}{5} = \frac{7}{5}$

Next, make the denominators the same. The lowest common multiple of 4 and 5 is 20.

$\frac{13}{4} = \frac{65}{20}$

$\frac{7}{5} = \frac{28}{20}$

Now subtract:

$\frac{65}{20} - \frac{28}{20} = \frac{37}{20}$

$\frac{37}{20}$ litres

(b) Express $\frac{37}{20}$ as a decimal:

$\frac{37}{20} = 1.85$

So, Mrs Tan used \textdollar1.85 litres of cooking oil during the month (to 2 decimal places).

\section*{Question 18}
\textbf{Metadata}

\begin{itemize}
  \item Question ID: P5-FrMulImN\_P2-FrAdd2nd\_GPT4.1\_Household Finance\_01
  \item Primary KC: FRACTIONS | Multiplication | multiplying a proper/improper fraction and a whole number
  \item Secondary KC: FRACTIONS | Addition | adding fractions
  \item Topic: Household finance such as income, utility bills, money, interest, savings, instalment, mortgage, financial planning etc.
  \item Grade: Primary 5
\end{itemize}

\textbf{Solution}

First, we find out how much Mrs Lee saves for groceries each month:

She spends $\frac{2}{5}$ of \textdollar2500 each month:

$\frac{2}{5} \times 2500 = \frac{2 \times 2500}{5} = \frac{5000}{5} = 1000$

So, she spends \textdollar1000 on groceries each month.

In 4 months, she spends:

$\textdollar1000 \times 4 = \textdollar4000$

Next, her brother gives her $\frac{3}{10}$ of what she has already saved for groceries (
\textdollar4000):

$\frac{3}{10} \times 4000 = \frac{3 \times 4000}{10} = \frac{12000}{10} = 1200$

Her brother gives her \textdollar1200.

Total money for groceries:

$\textdollar4000 + \textdollar1200 = \textdollar5200$

\textbf{Answer:} Mrs Lee will have \textdollar5200 for groceries in total after receiving her brother's extra contribution.

\section*{Question 19}
\textbf{Metadata}

\begin{itemize}
  \item Question ID: P5-FrMulImN\_P3-FrSmp\_GPT4.1\_Household Finance\_01
  \item Primary KC: FRACTIONS | Multiplication | multiplying a proper/improper fraction and a whole number
  \item Secondary KC: FRACTIONS | Simplifying | expressing a fraction in its simplest form
  \item Topic: Household finance such as income, utility bills, money, interest, savings, instalment, mortgage, financial planning etc.
  \item Grade: Primary 5
\end{itemize}

\textbf{Solution}

(a) The amount Mr Lim paid in June:

$ \textdollar120 \times \frac{2}{5} = \frac{120 \times 2}{5} = \frac{240}{5} $ dollars.

(b) Simplifying $ \frac{240}{5} $:

$ \frac{240}{5} = 48 $

So, Mr Lim paid $\textdollar48$ for his water in June.

\section*{Question 20}
\textbf{Metadata}

\begin{itemize}
  \item Question ID: P5-FrMulImN\_P5-FrCnv2Dc\_GPT4.1\_Household Finance\_01
  \item Primary KC: FRACTIONS | Multiplication | multiplying a proper/improper fraction and a whole number
  \item Secondary KC: FRACTIONS | Conversion to decimals | expressing fractions as decimals
  \item Topic: Household finance such as income, utility bills, money, interest, savings, instalment, mortgage, financial planning etc.
  \item Grade: Primary 5
\end{itemize}

\textbf{Solution}

First, we multiply the amount of rice Mrs Tan uses each week by the number of weeks:\\
\\
$\frac{4}{5} \times 7 = \frac{4 \times 7}{5} = \frac{28}{5}$\\
\\
So, Mrs Tan will have used $\frac{28}{5}$ packets of rice after 7 weeks.\\
\\
To express $\frac{28}{5}$ as a decimal: \\ $28 \div 5 = 5.6$\\
\\
Therefore, Mrs Tan will have used $5.6$ packets of rice after 7 weeks.

\section*{Question 21}
\textbf{Metadata}

\begin{itemize}
  \item Question ID: P5-FrMulImIm\_P2-FrCmp\_GPT4.1\_Household Finance\_01
  \item Primary KC: FRACTIONS | Multiplication | multiplying two improper fractions
  \item Secondary KC: FRACTIONS | Comparison and ordering | comparing and ordering fractions
  \item Topic: Household finance such as income, utility bills, money, interest, savings, instalment, mortgage, financial planning etc.
  \item Grade: Primary 5
\end{itemize}

\textbf{Solution}

First, calculate the current cost per unit: 

Last year's cost per unit: $\textdollar2$

Increase: $\frac{5}{3}$ times last year's cost per unit

Current cost per unit $= 2 \times \frac{5}{3} = \frac{10}{3}$

Now, calculate Mrs Tan's weekly bill:
She uses $\frac{9}{4}$ units per week.

Amount paid per week $= \frac{9}{4} \times \frac{10}{3} = \frac{90}{12} = \frac{15}{2}$

So Mrs Tan pays $\frac{15}{2}$ per week, in dollars.

Calculate the other household's bill:
Units used: $\frac{7}{3}$

Amount paid per week $= \frac{7}{3} \times \frac{10}{3} = \frac{70}{9}$

Compare $\frac{15}{2}$ and $\frac{70}{9}$:

Cross-multiply to compare:
$15 \times 9 = 135$

$2 \times 70 = 140$

Since $140 > 135$, $\frac{70}{9} > \frac{15}{2}$.

So the second household pays more.

Find the difference:

$\text{Difference} = \frac{70}{9} - \frac{15}{2}$

Convert to common denominator:

$\frac{70}{9} - \frac{15}{2} = \frac{140}{18} - \frac{135}{18} = \frac{5}{18}$

So the other household pays $\frac{5}{18}$ dollars more per week. 

**Final Answers:**

Mrs Tan's weekly cost: $\frac{15}{2}$ dollars

Second household's weekly cost: $\frac{70}{9}$ dollars

The second household pays more by $\frac{5}{18}$ dollars per week.

\section*{Question 22}
\textbf{Metadata}

\begin{itemize}
  \item Question ID: P5-FrMulImIm\_P5-FrCnv2Dc\_GPT4.1\_Household Finance\_01
  \item Primary KC: FRACTIONS | Multiplication | multiplying two improper fractions
  \item Secondary KC: FRACTIONS | Conversion to decimals | expressing fractions as decimals
  \item Topic: Household finance such as income, utility bills, money, interest, savings, instalment, mortgage, financial planning etc.
  \item Grade: Primary 5
\end{itemize}

\textbf{Solution}

First, we find Samantha's usual usage:

Samantha's usual monthly usage $= \frac{7}{4} \times 100 = 175$ kWh.

Her usage last month:
$= \frac{9}{5} \times 175$ kWh.

To multiply:
$\frac{9}{5} \times 175 = \frac{9}{5} \times \frac{175}{1} = \frac{9 \times 175}{5 \times 1} = \frac{1575}{5} = 315$ kWh.

Expressing as a decimal: $315$ (already a whole number, which is $315.0$ as a decimal).

\textbf{Answer:} Samantha used $315.0$ kWh of electricity last month.

\section*{Question 23}
\textbf{Metadata}

\begin{itemize}
  \item Question ID: P5-FrMulMixN\_P2-FrAdd2nd\_GPT4.1\_Household Finance\_01
  \item Primary KC: FRACTIONS | Multiplication | multiplying a mixed number and a whole number
  \item Secondary KC: FRACTIONS | Addition | adding fractions
  \item Topic: Household finance such as income, utility bills, money, interest, savings, instalment, mortgage, financial planning etc.
  \item Grade: Primary 5
\end{itemize}

\textbf{Solution}

First, we calculate the total amount Mrs Tan saved in 5 days:\\

$2\frac{1}{2} \times 5 = \frac{5}{2} \times 5 = \frac{25}{2} = 12\frac{1}{2}$\\

Next, we add the extra money she received from her father:\\

$12\frac{1}{2} + \frac{3}{4} = \frac{25}{2} + \frac{3}{4}$\\

Convert $\frac{25}{2}$ to quarters to add the fractions: \\ $\frac{25}{2} = \frac{50}{4}$\\

So, $\frac{50}{4} + \frac{3}{4} = \frac{53}{4} = 13\frac{1}{4}$\\

Therefore, Mrs Tan had $\boxed{\textdollar13\frac{1}{4}}$ in total.

\section*{Question 24}
\textbf{Metadata}

\begin{itemize}
  \item Question ID: P5-DcMul3dK\_P4-DcCmp3d\_GPT4.1\_Household Finance\_01
  \item Primary KC: DECIMALS | Multiplication | multiplying decimals (up to 3 decimal places) by 10, 100, 1000 and their multiples
  \item Secondary KC: DECIMALS | Comparison and ordering | comparing and ordering decimals up to 3 decimal places
  \item Topic: Household finance such as income, utility bills, money, interest, savings, instalment, mortgage, financial planning etc.
  \item Grade: Primary 5
\end{itemize}

\textbf{Solution}

(a) Mrs Tan's total electricity usage for May:

$3.285 \times 31 = 101.835$ kWh

(b) The total amount to pay is:
$101.835 \times 0.19 = 19.34865$

Rounding to the nearest cent, Mrs Tan needs to pay \textdollar19.35 for May.

(c) Comparing daily usage:
Mr Lee: $3.295$ kWh 
Mrs Tan: $3.285$ kWh

$3.295 - 3.285 = 0.010$ kWh

Mr Lee used more electricity per day than Mrs Tan by $0.010$ kWh.

\section*{Question 25}
\textbf{Metadata}

\begin{itemize}
  \item Question ID: P5-DcMul3dK\_P4-DcAdd2nd\_GPT4.1\_Household Finance\_01
  \item Primary KC: DECIMALS | Multiplication | multiplying decimals (up to 3 decimal places) by 10, 100, 1000 and their multiples
  \item Secondary KC: DECIMALS | Addition | adding decimals
  \item Topic: Household finance such as income, utility bills, money, interest, savings, instalment, mortgage, financial planning etc.
  \item Grade: Primary 5
\end{itemize}

\textbf{Solution}

(a) Amount spent on electricity in one day: $2.37$

Amount spent in 100 days: $2.37 \times 100 = 237.00$

So, Mrs. Tan spends \textdollar237.00 on electricity in 100 days.

(b) Water cost for 3 months: $125.40 \times 3 = 376.20$
Gas cost for 3 months: $45.70 \times 3 = 137.10$
Electricity for 100 days (from part a): $237.00$

Total utility bill for 100 days: $376.20 + 137.10 + 237.00 = 750.30$

Therefore, Mrs. Tan's total utility bill for water, gas, and electricity for 100 days is \textdollar750.30.

\section*{Question 26}
\textbf{Metadata}

\begin{itemize}
  \item Question ID: P5-DcDiv3dK\_P4-DcCmp3d\_GPT4.1\_Household Finance\_01
  \item Primary KC: DECIMALS | Division | dividing decimals (up to 3 decimal places) by 10, 100, 1000 and their multiples
  \item Secondary KC: DECIMALS | Comparison and ordering | comparing and ordering decimals up to 3 decimal places
  \item Topic: Household finance such as income, utility bills, money, interest, savings, instalment, mortgage, financial planning etc.
  \item Grade: Primary 5
\end{itemize}

\textbf{Solution}

(a) To find out how much each person pays, divide $124.500$ by $3$:\\

$124.500 \div 3 = 41.500$\\

So, each person pays $\textdollar41.500$.\\

(b) The three account balances are:\\
$42.500$, $41.580$, $41.670$\\

Arranging in ascending order:\\

$41.580 < 41.670 < 42.500$\\

The account with the highest balance is $\textdollar42.500$.\\

Since $42.500 > 41.500$, Mei Ling can pay her share using the account with the highest balance.

\section*{Question 27}
\textbf{Metadata}

\begin{itemize}
  \item Question ID: P5-DcDiv3dK\_P4-DcSub2nd\_GPT4.1\_Household Finance\_01
  \item Primary KC: DECIMALS | Division | dividing decimals (up to 3 decimal places) by 10, 100, 1000 and their multiples
  \item Secondary KC: DECIMALS | Subtraction | subtracting decimals
  \item Topic: Household finance such as income, utility bills, money, interest, savings, instalment, mortgage, financial planning etc.
  \item Grade: Primary 5
\end{itemize}

\textbf{Solution}

First, we need to find out how much less electricity was used in the second month:

Amount less used $= \frac{245.370}{10}$

$245.370 \div 10 = 24.537$

So, Mrs Lim used $24.537$ kWh less in the second month.

Next, to check the final answer:

\boxed{24.537}\ \text{ kWh less electricity was used in the second month than in the first month.}

\section*{Question 28}
\textbf{Metadata}

\begin{itemize}
  \item Question ID: P5-PcRepWh\_P1-WNAdd2nd\_GPT4.1\_Household Finance\_01
  \item Primary KC: PERCENTAGE | Representation and concept | expressing a part of a whole as a percentage
  \item Secondary KC: WHOLE NUMBERS | Addition | adding whole numbers
  \item Topic: Household finance such as income, utility bills, money, interest, savings, instalment, mortgage, financial planning etc.
  \item Grade: Primary 5
\end{itemize}

\textbf{Solution}

First, find Amy's total savings in January and February: 

$80 + 120 = 200$

Next, calculate the percentage of the total savings that she saved in January:

$\frac{80}{200} \times 100\% = 40\%$

So, Amy saved $40\%$ of her total savings in January.

\section*{Question 29}
\textbf{Metadata}

\begin{itemize}
  \item Question ID: P5-PcRepWh\_P1-WNSub2nd\_GPT4.1\_Household Finance\_01
  \item Primary KC: PERCENTAGE | Representation and concept | expressing a part of a whole as a percentage
  \item Secondary KC: WHOLE NUMBERS | Subtraction | subtracting whole numbers
  \item Topic: Household finance such as income, utility bills, money, interest, savings, instalment, mortgage, financial planning etc.
  \item Grade: Primary 5
\end{itemize}

\textbf{Solution}

First, calculate how much Mrs Lee still needs to pay:

\[
\text{Amount left to pay} = \textdollar120 - \textdollar45 = \textdollar75
\]

Next, express the amount left to pay as a percentage of the total bill:

\[
\text{Percentage still to pay} = \frac{75}{120} \times 100\%
\]

\[
= \frac{75 \times 100}{120}\%
\]

\[
= \frac{7500}{120}\% = 62.5\%
\]

Mrs Lee still needs to pay \boxed{62.5\%} of her electricity bill.

\section*{Question 30}
\textbf{Metadata}

\begin{itemize}
  \item Question ID: P5-RtFndT\_P2-DcCnvD2N\_GPT4.1\_Household Finance\_01
  \item Primary KC: RATE | Finding total amount | finding total amount, given rate and number of units
  \item Secondary KC: DECIMALS | Conversion to smaller units | converting a measurement from a larger unit in decimal form to a smaller unit
  \item Topic: Household finance such as income, utility bills, money, interest, savings, instalment, mortgage, financial planning etc.
  \item Grade: Primary 5
\end{itemize}

\textbf{Solution}

First, find the total amount of electricity used in $5$ days:

\[
2.75 \times 5 = 13.75 \text{ kilowatt-hours}
\]

Next, find the total cost in dollars:

\[
13.75 \times 0.19 = 2.6125 \text{ dollars}
\]

To convert dollars to cents, multiply by $100$:

\[
2.6125 \times 100 = 261.25 \text{ cents}
\]

Thus, Mrs Lim pays 261.25 cents for her electricity usage in $5$ days.

\section*{Question 31}
\textbf{Metadata}

\begin{itemize}
  \item Question ID: P6-FrDivPN\_P2-FrCmp\_GPT4.1\_Household Finance\_01
  \item Primary KC: FRACTIONS | Division | dividing a proper fraction by a whole number
  \item Secondary KC: FRACTIONS | Comparison and ordering | comparing and ordering fractions
  \item Topic: Household finance such as income, utility bills, money, interest, savings, instalment, mortgage, financial planning etc.
  \item Grade: Primary 6
\end{itemize}

\textbf{Solution}

First, determine how much Mrs. Tan spends on her utility bill in a month:

$\frac{3}{4} \times \textdollar180 = \textdollar135$

She shares this utility bill equally with her brother:

$\frac{135}{2} = \textdollar67.50$

Now, calculate how much Mr. Lee pays per month:

$\frac{2}{5} \times \textdollar180 = \textdollar72$

Now compare the amounts:

- Mrs. Tan pays $\textdollar67.50$ per month.
- Mr. Lee pays $\textdollar72$ per month.

Since $\textdollar67.50 < \textdollar72$, Mrs. Tan pays less than Mr. Lee.

Comparing the fractions:

- Mrs. Tan: $\frac{1}{2} \times \frac{3}{4} = \frac{3}{8}$ of \textdollar180 per person.
- Mr. Lee: $\frac{2}{5}$ of \textdollar180.

To compare $\frac{3}{8}$ and $\frac{2}{5}$:

Find a common denominator:

$\frac{3}{8} = \frac{15}{40}$

$\frac{2}{5} = \frac{16}{40}$

So, $\frac{3}{8} < \frac{2}{5}$.

Therefore, each month, Mrs. Tan pays less than Mr. Lee after dividing her bill with her brother.

\section*{Question 32}
\textbf{Metadata}

\begin{itemize}
  \item Question ID: P6-FrDivPN\_P2-FrSub2nd\_GPT4.1\_Household Finance\_01
  \item Primary KC: FRACTIONS | Division | dividing a proper fraction by a whole number
  \item Secondary KC: FRACTIONS | Subtraction | subtracting fractions
  \item Topic: Household finance such as income, utility bills, money, interest, savings, instalment, mortgage, financial planning etc.
  \item Grade: Primary 6
\end{itemize}

\textbf{Solution}

(a) Mrs Tan keeps $\frac{1}{8}$ of $\textdollar 1$ for herself. So, the amount left to distribute = $1 - \frac{1}{8} = \frac{8}{8} - \frac{1}{8} = \frac{7}{8}$ of $\textdollar 1$.

(b) She shares $\frac{7}{8}$ of $\textdollar 1$ equally among 4 children.
Each child receives:
$$
\frac{7}{8} \div 4 = \frac{7}{8} \times \frac{1}{4} = \frac{7}{32}
$$
So, each child receives $\frac{7}{32}$ of $\textdollar 1$.

\section*{Question 33}
\textbf{Metadata}

\begin{itemize}
  \item Question ID: P6-FrDivPN\_P5-FrMul2nd\_GPT4.1\_Household Finance\_01
  \item Primary KC: FRACTIONS | Division | dividing a proper fraction by a whole number
  \item Secondary KC: FRACTIONS | Multiplication | fraction multiplication
  \item Topic: Household finance such as income, utility bills, money, interest, savings, instalment, mortgage, financial planning etc.
  \item Grade: Primary 6
\end{itemize}

\textbf{Solution}

First, we find $\frac{3}{4}$ of Mrs Lim's total savings:

$\frac{3}{4} \times 18 = 13.5$

So, $\textdollar13.50$ is the amount she will give to her children in total.

This $\textdollar13.50$ will be divided equally among her 3 children:

$\frac{13.5}{3} = 4.5$

So, each child receives $\textdollar4.50$.

Alternatively, use fraction division:

$\frac{3}{4} \times 18 = 13.5$

Then, $\frac{13.5}{3} = 4.5$

Or, using fraction division:

$\frac{3}{4} \div 3 = \frac{3}{4} \times \frac{1}{3} = \frac{1}{4}$

Then, $\frac{1}{4} \times 18 = 4.5$

Thus, each child receives $\textdollar4.50$.

\section*{Question 34}
\textbf{Metadata}

\begin{itemize}
  \item Question ID: P6-FrDivPP\_P2-FrAdd2nd\_GPT4.1\_Household Finance\_01
  \item Primary KC: FRACTIONS | Division | dividing a whole number/proper fraction by a proper fraction
  \item Secondary KC: FRACTIONS | Addition | adding fractions
  \item Topic: Household finance such as income, utility bills, money, interest, savings, instalment, mortgage, financial planning etc.
  \item Grade: Primary 6
\end{itemize}

\textbf{Solution}

First, let us find the number of slices Mrs Tan cut from $\dfrac{3}{4}$ of a cake, where each slice is $\dfrac{1}{8}$ of the cake.

Number of slices $= \dfrac{3}{4} \div \dfrac{1}{8} = \dfrac{3}{4} \times \dfrac{8}{1} = \dfrac{24}{4} = 6$

So, Mrs Tan cut $6$ slices from $\dfrac{3}{4}$ of the cake.

(a) The answer is $6$ slices.

$\ \ $

Next, $\dfrac{1}{3}$ of the slices were left. The number of slices left $= \dfrac{1}{3} \times 6 = 2$

Since each slice is $\dfrac{1}{8}$ of the cake, total leftover from the original cake $= 2 \times \dfrac{1}{8} = \dfrac{2}{8} = \dfrac{1}{4}$ (of a whole cake).

Then Mrs Tan added $\dfrac{1}{4}$ of a whole cake from the fridge.

Fraction of a whole cake she now has $= \dfrac{1}{4} + \dfrac{1}{4} = \dfrac{2}{4} = \dfrac{1}{2}$

(b) The answer is $\dfrac{1}{2}$ of a whole cake.

\section*{Question 35}
\textbf{Metadata}

\begin{itemize}
  \item Question ID: P6-FrDivPP\_P5-FrMul2nd\_GPT4.1\_Household Finance\_01
  \item Primary KC: FRACTIONS | Division | dividing a whole number/proper fraction by a proper fraction
  \item Secondary KC: FRACTIONS | Multiplication | fraction multiplication
  \item Topic: Household finance such as income, utility bills, money, interest, savings, instalment, mortgage, financial planning etc.
  \item Grade: Primary 6
\end{itemize}

\textbf{Solution}

First, we find the amount of the water bill for one month:

\[
\text{Monthly water bill} = \frac{3}{4} \times 12 = 9
\]

Next, the monthly electricity bill is:

\[
\text{Monthly electricity bill} = \frac{2}{3} \times 9 = 6
\]

Mrs Lee has $\textdollar48$ saved. To find out how many months she can pay the electricity bill, we divide her total savings by one month's electricity bill:

\[
\text{Number of months} = \frac{48}{6} = 8
\]

\textbf{Answer:} Mrs Lee can pay her electricity bills for 8 months with her savings.

\section*{Question 36}
\textbf{Metadata}

\begin{itemize}
  \item Question ID: P6-PcFndWN\_P1-WNMul2nd\_GPT4.1\_Household Finance\_01
  \item Primary KC: PERCENTAGE | Finding the whole | finding the whole given a part and the percentage
  \item Secondary KC: WHOLE NUMBERS | Multiplication | multiplying whole numbers
  \item Topic: Household finance such as income, utility bills, money, interest, savings, instalment, mortgage, financial planning etc.
  \item Grade: Primary 6
\end{itemize}

\textbf{Solution}

Let $x$ be Mrs Tan's total monthly salary.\\

We are told that 25\% of her salary is \textdollar400.\\

25\% of $x = \textdollar400$\\

$\Rightarrow \frac{25}{100} \times x = 400$\\

$\Rightarrow \frac{1}{4} x = 400$\\

$\Rightarrow x = 400 \times 4$\\

$\Rightarrow x = 1600$\\

Mrs Tan's total monthly salary was \textdollar1600.\\

If she received this amount for 3 months, the total amount in 3 months is:\\

$1600 \times 3 = 4800$\\

So, Mrs Tan received \textdollar4800 in 3 months.

\section*{Question 37}
\textbf{Metadata}

\begin{itemize}
  \item Question ID: P6-PcFndWN\_P1-WNDiv2nd\_GPT4.1\_Household Finance\_01
  \item Primary KC: PERCENTAGE | Finding the whole | finding the whole given a part and the percentage
  \item Secondary KC: WHOLE NUMBERS | Division | dividing whole numbers
  \item Topic: Household finance such as income, utility bills, money, interest, savings, instalment, mortgage, financial planning etc.
  \item Grade: Primary 6
\end{itemize}

\textbf{Solution}

Let the total monthly income be $x$.

Given $15\%$ of her total monthly income is $\textdollar180$,
\[
15\% \times x = 180
\]
\[
0.15x = 180
\]
\[
x = \frac{180}{0.15}
\]
\[
x = 1200
\]
So, Mrs Lim's total monthly income was $\textdollar1200$.

Next, if she divides $\textdollar1200$ equally among her 3 savings accounts:
\[
\text{Amount for each account} = \frac{1200}{3} = 400
\]

Therefore, Mrs Lim will put $\textdollar400$ into each account.

\section*{Question 38}
\textbf{Metadata}

\begin{itemize}
  \item Question ID: P6-PcFndChg\_P1-WNAdd2nd\_GPT4.1\_Household Finance\_01
  \item Primary KC: PERCENTAGE | Finding change | finding percentage increase/decrease
  \item Secondary KC: WHOLE NUMBERS | Addition | adding whole numbers
  \item Topic: Household finance such as income, utility bills, money, interest, savings, instalment, mortgage, financial planning etc.
  \item Grade: Primary 6
\end{itemize}

\textbf{Solution}

First, find the amount of the electricity bill increase:

Percentage increase $= 15\%$ of \textdollar120

$= \frac{15}{100} \times 120 = 18$

New electricity bill in February $= 120 + 18 = \textdollar138$

The water bill remains at \textdollar30.

Total amount Mr Tan paid in February:

$138 + 30 = \textdollar168$

\textbf{Final answer:} Mr Tan paid \textdollar168 for his electricity and water bills in February.

\section*{Question 39}
\textbf{Metadata}

\begin{itemize}
  \item Question ID: P6-RoFndDvqWN\_P1-WNSub2nd\_GPT4.1\_Household Finance\_01
  \item Primary KC: RATIO | Finding divided quantities | dividing a given quantity in a given ratio
  \item Secondary KC: WHOLE NUMBERS | Subtraction | subtracting whole numbers
  \item Topic: Household finance such as income, utility bills, money, interest, savings, instalment, mortgage, financial planning etc.
  \item Grade: Primary 6
\end{itemize}

\textbf{Solution}

First, we know Mrs. Lee has \textdollar900 after paying the utility bills.

She divides this between groceries and savings in the ratio $5 : 4$.

Let the total number of equal parts be $5 + 4 = 9$ parts.

Amount for 1 part:
$900 \div 9 = 100$

So,
Amount set aside for groceries:
$5 \times 100 = \textdollar500$

Amount set aside for savings:
$4 \times 100 = \textdollar400$

(a) Mrs. Lee puts aside \textdollar500 for groceries and \textdollar400 for savings.

(b) After paying a repair bill of \textdollar130 from her savings:

Amount left in savings:
$400 - 130 = \textdollar270$

So, after the repair bill, Mrs. Lee will have \textdollar270 left in her savings.

\section*{Question 40}
\textbf{Metadata}

\begin{itemize}
  \item Question ID: P6-RoFndRoWN\_P1-WNMul2nd\_GPT4.1\_Household Finance\_01
  \item Primary KC: RATIO | Finding ratio | finding the ratio of two or three given whole numbers
  \item Secondary KC: WHOLE NUMBERS | Multiplication | multiplying whole numbers
  \item Topic: Household finance such as income, utility bills, money, interest, savings, instalment, mortgage, financial planning etc.
  \item Grade: Primary 6
\end{itemize}

\textbf{Solution}

(a) The ratio of the amount spent on groceries to utilities to transport is $540 : 300 : 180$.<br><br>First, find the highest common factor of $540$, $300$, and $180$.<br><br>Let's factorise each number:<br>$540 = 2 \times 2 \times 3 \times 3 \times 3 \times 5$<br>$300 = 2 \times 2 \times 3 \times 5 \times 5$<br>$180 = 2 \times 2 \times 3 \times 3 \times 5$<br><br>The highest common factor is $60$.<br><br>So, divide each number by $60$: <br>$540 \div 60 = 9$<br>$300 \div 60 = 5$<br>$180 \div 60 = 3$<br><br>Therefore, the simplified ratio is $9:5:3$.<br><br>(b) New amount for utilities is $2 \times 300 = 600$.<br><br>The new ratio is $540 : 600 : 180$.<br><br>Find the highest common factor of $540$, $600$, and $180$.<br>$540 \div 60 = 9$<br>$600 \div 60 = 10$<br>$180 \div 60 = 3$<br><br>The new simplified ratio is $9:10:3$.

\section*{Question 41}
\textbf{Metadata}

\begin{itemize}
  \item Question ID: P6-RoFndRoWN\_P6-RoSmpWN\_GPT4.1\_Household Finance\_01
  \item Primary KC: RATIO | Finding ratio | finding the ratio of two or three given whole numbers
  \item Secondary KC: RATIO | Simplifying | expressing a ratio in its simplest form
  \item Topic: Household finance such as income, utility bills, money, interest, savings, instalment, mortgage, financial planning etc.
  \item Grade: Primary 6
\end{itemize}

\textbf{Solution}

(a) The amounts spent are:
\begin{itemize}
\item Rent: \textdollar600
\item Groceries: \textdollar200
\item Utility bills: \textdollar100
\end{itemize}

So, the ratio is $600:200:100$.

(b) To express the ratio in its simplest form, find the highest common factor of $600$, $200$, and $100$, which is $100$.

Divide each amount by $100$:
\begin{align*}
600 \div 100 &= 6 \\
200 \div 100 &= 2 \\
100 \div 100 &= 1
\end{align*}

Therefore, the simplest form of the ratio is $6:2:1$.

\section*{Question 42}
\textbf{Metadata}

\begin{itemize}
  \item Question ID: P6-RoFndTmWN\_P1-WNSub2nd\_GPT4.1\_Household Finance\_01
  \item Primary KC: RATIO | Finding a missing term | finding the missing term in a pair of equivalent ratios
  \item Secondary KC: WHOLE NUMBERS | Subtraction | subtracting whole numbers
  \item Topic: Household finance such as income, utility bills, money, interest, savings, instalment, mortgage, financial planning etc.
  \item Grade: Primary 6
\end{itemize}

\textbf{Solution}

Let the amount Amy’s sister saves each month be $x$. Last month, Amy saved \textdollar80 and the ratio was $5:3$, so we have:  

\[
\frac{80}{x} = \frac{5}{3}
\]
\[
80 \times 3 = 5x
\]
\[
240 = 5x
\]
\[
x = 48
\]

So Amy’s sister saves \textdollar48 each month. 

This month, Amy spends \textdollar15 from her savings, so Amy’s leftover savings is:
\[
80 - 15 = 65
\]

Now, check if the new ratio of Amy’s and her sister’s savings is $5:3$:
\[
\frac{65}{48}
\]

If $65:48$ is equivalent to $5:3$, then:
\[
\frac{65}{48} = \frac{5}{3}
\]
\[
65 \times 3 = 195,\quad 48 \times 5 = 240
\]

They are not equal, but the question only asks how much Amy has left after spending, so the answer is \textdollar65.

\section*{Question 43}
\textbf{Metadata}

\begin{itemize}
  \item Question ID: O1-RoRepFr\_P2-FrAdd2nd\_GPT4.1\_Household Finance\_01
  \item Primary KC: RATIO | Representation and concept | ratios involving fractions
  \item Secondary KC: FRACTIONS | Addition | adding fractions
  \item Topic: Household finance such as income, utility bills, money, interest, savings, instalment, mortgage, financial planning etc.
  \item Grade: Secondary O-level 1
\end{itemize}

\textbf{Solution}

Let the amount spent on electricity be $E$ and groceries be $G$.

The ratio is given as $E : G = \frac{2}{3} : \frac{5}{6}$.

First, convert the ratios to a common denominator:
$$\frac{2}{3} = \frac{4}{6}$$
So, the ratio is $\frac{4}{6} : \frac{5}{6}$.

This means for every $\frac{4}{6}$ parts spent on electricity, $\frac{5}{6}$ parts are spent on groceries.

Total number of parts $= \frac{4}{6} + \frac{5}{6} = \frac{9}{6}$.

Let the value for each "part" be $k$.

So,
\begin{align*}
E &= \frac{4}{6}k \\
G &= \frac{5}{6}k
\end{align*}

The total spending is given by:
$$E + G = \textdollar240$$
$$\frac{4}{6}k + \frac{5}{6}k = \textdollar240$$
$$\frac{9}{6}k = \textdollar240$$
$$k = \textdollar240 \times \frac{6}{9} = \textdollar160$$

Now,
\begin{align*}
E &= \frac{4}{6} \times 160 = \frac{2}{3} \times 160 = 2 \times \frac{160}{3} = 2 \times 53.33 = \textdollar106.67 \\
G &= \frac{5}{6} \times 160 = \frac{5 \times 160}{6} = 800 \div 6 = \textdollar133.33
\end{align*}

**Mr. Lim spent $\textdollar106.67$ on electricity and $\textdollar133.33$ on groceries.**

\section*{Question 44}
\textbf{Metadata}

\begin{itemize}
  \item Question ID: O1-RoRepFr\_P5-FrMul2nd\_GPT4.1\_Household Finance\_01
  \item Primary KC: RATIO | Representation and concept | ratios involving fractions
  \item Secondary KC: FRACTIONS | Multiplication | fraction multiplication
  \item Topic: Household finance such as income, utility bills, money, interest, savings, instalment, mortgage, financial planning etc.
  \item Grade: Secondary O-level 1
\end{itemize}

\textbf{Solution}

(a) Let Siti's monthly income be $I$.\
\
Amount spent on utility bills $= \frac{2}{5}I$\
\nAmount spent on groceries $= \frac{1}{4}I$\
\nThe ratio is $\frac{2}{5}I : \frac{1}{4}I$.\
\nDividing both parts by $I$ (since $I > 0$), the ratio simplifies to $\frac{2}{5} : \frac{1}{4}$.\
\nTo write this as a ratio of whole numbers, multiply both terms by 20 (the least common multiple of 5 and 4):\
\n$\left(\frac{2}{5} \times 20\right) : \left(\frac{1}{4} \times 20\right) = 8 : 5$\
\nSo, the simplest ratio is $8 : 5$.\
\n(b) If $I = \textdollar2000$,\
\nAmount spent on utility bills $= \frac{2}{5} \times 2000 = \frac{4000}{5} = \textdollar800$\
\nAmount spent on groceries $= \frac{1}{4} \times 2000 = \frac{2000}{4} = \textdollar500$\
\nTotal amount spent $= \textdollar800 + \textdollar500 = \textdollar1300$\
\n\textbf{Answer:}\
\begin{enumerate}
  \item The ratio of amount spent on utility bills to groceries is $8 : 5$.
  \item Siti spends \textdollar800 on utility bills and \textdollar500 on groceries, for a total of \textdollar1300.
\end{enumerate}

\section*{Question 45}
\textbf{Metadata}

\begin{itemize}
  \item Question ID: O1-RoRepFr\_O1-RoSmpFr\_GPT4.1\_Household Finance\_02
  \item Primary KC: RATIO | Representation and concept | ratios involving fractions
  \item Secondary KC: RATIO | Simplifying | converting a ratio involving fractions to its simplest form
  \item Topic: Household finance such as income, utility bills, money, interest, savings, instalment, mortgage, financial planning etc.
  \item Grade: Secondary O-level 1
\end{itemize}

\textbf{Solution}

The amount spent on utilities: $\frac{2}{5}$ of income.
The amount spent on groceries: $\frac{3}{10}$ of income.

The ratio of utilities to groceries is:

$\frac{2}{5} : \frac{3}{10}$

To express this ratio in whole numbers, divide both sides by $\frac{1}{10}$ (the lowest common denominator):

$= \frac{2}{5} \div \frac{1}{10} : \frac{3}{10} \div \frac{1}{10}$

$= (\frac{2}{5} \times 10) : (\frac{3}{10} \times 10)$

$= 4 : 3$

Therefore, the ratio of the amount spent on utilities to the amount spent on groceries in its simplest form is $4 : 3$.

\section*{Question 46}
\textbf{Metadata}

\begin{itemize}
  \item Question ID: O1-RoRepDc\_P4-DcAdd2nd\_GPT4.1\_Household Finance\_01
  \item Primary KC: RATIO | Representation and concept | ratios involving decimals
  \item Secondary KC: DECIMALS | Addition | adding decimals
  \item Topic: Household finance such as income, utility bills, money, interest, savings, instalment, mortgage, financial planning etc.
  \item Grade: Secondary O-level 1
\end{itemize}

\textbf{Solution}

Let the amount Siti paid be $1.2x$ and her brother paid $2.4x$. The total cost is $1.2x + 2.4x = 3.6x$.

Set $3.6x = 72.50$. Solve for $x$:
\\[
x = \frac{72.50}{3.6} = 20.1388\ldots
\\]
Now, calculate the amounts:

Siti: $1.2x = 1.2 \times 20.1388 \approx 24.17$

Her brother: $2.4x = 2.4 \times 20.1388 \approx 48.33$

Check: $24.17 + 48.33 = 72.50$

Therefore, Siti paid \textdollar24.17 and her brother paid \textdollar48.33.

\section*{Question 47}
\textbf{Metadata}

\begin{itemize}
  \item Question ID: O1-RoRepDc\_O1-RoSmpDc\_GPT4.1\_Household Finance\_02
  \item Primary KC: RATIO | Representation and concept | ratios involving decimals
  \item Secondary KC: RATIO | Simplifying | converting a ratio involving decimals to its simplest form
  \item Topic: Household finance such as income, utility bills, money, interest, savings, instalment, mortgage, financial planning etc.
  \item Grade: Secondary O-level 1
\end{itemize}

\textbf{Solution}

(a) To simplify the ratio $2.5:1.5:4$, find the smallest number to multiply each term to eliminate decimals. Multiply each term by $2$:

$2.5 \times 2 : 1.5 \times 2 : 4 \times 2 = 5 : 3 : 8$

So the simplest whole-number form is $5:3:8$.

(b) Let the total number of parts be $5 + 3 + 8 = 16$ parts.

Each part is worth $\frac{\textdollar1600}{16} = \textdollar100$.

So:

- Savings: $5 \times \textdollar100 = \textdollar500$
- Utility bills: $3 \times \textdollar100 = \textdollar300$
- Groceries: $8 \times \textdollar100 = \textdollar800$

Therefore, Mr Tan spends $\textdollar500$ on savings, $\textdollar300$ on utility bills, and $\textdollar800$ on groceries each month.

\section*{Question 48}
\textbf{Metadata}

\begin{itemize}
  \item Question ID: O1-PcRep2q\_O1-PcCnv2Fr\_GPT4.1\_Household Finance\_02
  \item Primary KC: PERCENTAGE | Representation and concept | comparing two quantities by percentage
  \item Secondary KC: PERCENTAGE | Conversion to fraction | expressing percentage as a fraction
  \item Topic: Household finance such as income, utility bills, money, interest, savings, instalment, mortgage, financial planning etc.
  \item Grade: Secondary O-level 1
\end{itemize}

\textbf{Solution}

Let the increase in the bill be $\textdollar150 - \textdollar120 = \textdollar30$.

(a) To find the percentage increase:

\[
\text{Percentage increase} = \frac{\text{Increase}}{\text{Original amount}} \times 100\% = \frac{30}{120} \times 100\% = 25\%
\]

(b) To express 25\% as a fraction:

\[
25\% = \frac{25}{100} = \frac{1}{4}
\]

\textbf{Answers:}

(a) The electricity bill increased by $25\%$.

(b) This percentage increase is $\frac{1}{4}$ in its simplest form.

\section*{Question 49}
\textbf{Metadata}

\begin{itemize}
  \item Question ID: O1-PcFndRslt\_P1-WNAdd2nd\_GPT4.1\_Household Finance\_01
  \item Primary KC: PERCENTAGE | Finding result after change | increasing/decreasing a quantity by a given percentage
  \item Secondary KC: WHOLE NUMBERS | Addition | adding whole numbers
  \item Topic: Household finance such as income, utility bills, money, interest, savings, instalment, mortgage, financial planning etc.
  \item Grade: Secondary O-level 1
\end{itemize}

\textbf{Solution}

First, calculate the increased electricity cost after the $12\%$ rate increase:

Original bill $= \textdollar200$

Increase $= 12\%$ of $\textdollar200 = 0.12 \times 200 = \textdollar24$

New bill after percentage increase $= \textdollar200 + \textdollar24 = \textdollar224$

Now, add the extra cost from the new fan:

Total monthly bill $= \textdollar224 + \textdollar18 = \textdollar242$

\textbf{Answer:} The new total monthly electricity bill is \textdollar242.

\section*{Question 50}
\textbf{Metadata}

\begin{itemize}
  \item Question ID: O1-PcFndRslt\_P1-WNMul2nd\_GPT4.1\_Household Finance\_01
  \item Primary KC: PERCENTAGE | Finding result after change | increasing/decreasing a quantity by a given percentage
  \item Secondary KC: WHOLE NUMBERS | Multiplication | multiplying whole numbers
  \item Topic: Household finance such as income, utility bills, money, interest, savings, instalment, mortgage, financial planning etc.
  \item Grade: Secondary O-level 1
\end{itemize}

\textbf{Solution}

First, calculate the percentage increase: $7\%$ of \textdollar120 is $0.07 \times 120 = 8.40$. 

Add this to her original bill: \textdollar120 + \textdollar8.40 = \textdollar128.40$.

Now, multiply by 4 months: $4 \times 128.40 = 513.60$.

So, Mrs Lim has to pay \textdollar513.60 in total for these 4 months.

\section*{Question 51}
\textbf{Metadata}

\begin{itemize}
  \item Question ID: O1-PcRepRvs\_O1-PcCnv2Fr\_GPT4.1\_Household Finance\_02
  \item Primary KC: PERCENTAGE | Representation and concept | reverse percentages
  \item Secondary KC: PERCENTAGE | Conversion to fraction | expressing percentage as a fraction
  \item Topic: Household finance such as income, utility bills, money, interest, savings, instalment, mortgage, financial planning etc.
  \item Grade: Secondary O-level 1
\end{itemize}

\textbf{Solution}

(a) She paid 100\% - 20\% = 80\% of the original bill.

Expressing 80\% as a fraction:

$80\% = \frac{80}{100} = \frac{4}{5}$

So, Mrs Tan paid $\frac{4}{5}$ of the original bill.

(b) Let the original bill be $x$.

She paid 80\% of $x$ and the amount paid was \textdollar96:

$0.8x = 96$

$x = \frac{96}{0.8} = 120$

Therefore, the original amount of the electricity bill was \textdollar120.

\section*{Question 52}
\textbf{Metadata}

\begin{itemize}
  \item Question ID: O1-AgRepExSq\_O1-AgEvlEx\_GPT4.1\_Household Finance\_02
  \item Primary KC: ALGEBRA | Representation and concept | translation of simple real-world situations into quadratic algebraic expressions
  \item Secondary KC: ALGEBRA | Evaluation | evaluation of algebraic expressions and formulae
  \item Topic: Household finance such as income, utility bills, money, interest, savings, instalment, mortgage, financial planning etc.
  \item Grade: Secondary O-level 1
\end{itemize}

\textbf{Solution}

(a)\
\\
The area of the room $= (x+2)(x-1) = x^2 + 2x - x - 2 = x^2 + x - 2$ (in square metres).\
\\
Total cost $= 25 \times$ (area) $= 25(x^2 + x - 2)$.\
\\
The required quadratic algebraic expression is $25(x^2 + x - 2)$.\
\\
(b)\
\\
Substitute $x = 4$:\
\\
$25(x^2 + x - 2) = 25(4^2 + 4 - 2) = 25(16 + 4 - 2) = 25(18) = \textdollar450$.\
\\
Therefore, the actual total cost of the carpet is \textdollar450.

\section*{Question 53}
\textbf{Metadata}

\begin{itemize}
  \item Question ID: O1-AgRepnth\_O1-AgEvlEx\_GPT4.1\_Household Finance\_02
  \item Primary KC: ALGEBRA | Representation and concept | recognising and representing patterns/relationships by finding an algebraic expression for the nth term
  \item Secondary KC: ALGEBRA | Evaluation | evaluation of algebraic expressions and formulae
  \item Topic: Household finance such as income, utility bills, money, interest, savings, instalment, mortgage, financial planning etc.
  \item Grade: Secondary O-level 1
\end{itemize}

\textbf{Solution}

(a) The savings form an arithmetic sequence: the first term is \textdollar10 and the common difference is \textdollar5.

The $n$th term of an arithmetic sequence is given by: $a_n = a_1 + (n-1)d$.

Here, $a_1 = 10$, $d = 5$.

So, $a_n = 10 + (n-1) \times 5$

$\quad\;\, = 10 + 5n - 5$

$\quad\;\, = 5n + 5$

Therefore, the algebraic expression for the amount saved in the $n$th month is $5n + 5$.

(b) Substitute $n = 8$:

Amount saved in the 8th month $= 5 \times 8 + 5 = 40 + 5 = 45$

Therefore, Jia Li saves \textdollar45 in the 8th month.

\section*{Question 54}
\textbf{Metadata}

\begin{itemize}
  \item Question ID: O2-RoRepDP\_P1-WNDiv2nd\_GPT4.1\_Household Finance\_01
  \item Primary KC: RATIO | Representation and concept | direct proportion
  \item Secondary KC: WHOLE NUMBERS | Division | dividing whole numbers
  \item Topic: Household finance such as income, utility bills, money, interest, savings, instalment, mortgage, financial planning etc.
  \item Grade: Secondary O-level 2
\end{itemize}

\textbf{Solution}

(a) Let the total number of parts = $5 + 3 = 8$.\newline
Zoe's share = $3$ parts.\newline
Amount Zoe needs to pay = $\dfrac{3}{8} \times 184 = 69$. \newline
So, Zoe needs to pay \textdollar69.\newline
(b) To divide her share equally over 4 weeks: $\dfrac{69}{4} = 17.25$.\newline
Zoe must set aside \textdollar17.25 each week.

\section*{Question 55}
\textbf{Metadata}

\begin{itemize}
  \item Question ID: O2-AgSlvIneq\_O2-AgRepIneq\_GPT4.1\_Household Finance\_01
  \item Primary KC: ALGEBRA | Solving | solving simple linear inequalities with one variable
  \item Secondary KC: ALGEBRA | Representation and concept | translation of simple real-world situations to simple linear inequalities with one variable
  \item Topic: Household finance such as income, utility bills, money, interest, savings, instalment, mortgage, financial planning etc.
  \item Grade: Secondary O-level 2
\end{itemize}

\textbf{Solution}

Let $x$ be the amount Ming spends on utility bills each month.

Ming's total monthly expenses are $x + 350$.

The amount Ming saves each month is: $500 - (x + 350) = 150 - x$

He wants to save at least \textdollar120, so:

$150 - x \geq 120$

Solving the inequality:
$150 - x \geq 120$

Subtract $150$ from both sides:
$-x \geq 120 - 150$
$-x \geq -30$

Multiply both sides by $-1$ (remember to reverse the inequality sign):
$x \leq 30$

Therefore, Ming can spend at most \textdollar30 on his utility bills each month to save at least \textdollar120.

\section*{Question 56}
\textbf{Metadata}

\begin{itemize}
  \item Question ID: O2-AgSlvSq1v\_O1-AgRepEq\_GPT4.1\_Household Finance\_01
  \item Primary KC: ALGEBRA | Solving | solving quadratic equations in one variable
  \item Secondary KC: ALGEBRA | Representation and concept | translation of simple real-world situations to equations
  \item Topic: Household finance such as income, utility bills, money, interest, savings, instalment, mortgage, financial planning etc.
  \item Grade: Secondary O-level 2
\end{itemize}

\textbf{Solution}

(i) The pattern of savings forms an arithmetic sequence: the amount saved each month is $x$, $(x+y)$, $(x+2y)$, ..., $(x+(n-1)y)$ for $n$ months. 

Total savings after $n$ months: 
\[ S = x + (x+y) + (x+2y) + ... + (x+(n-1)y) \]
There are $n$ terms. The sum of this arithmetic sequence is:
\[ S = \frac{n}{2} [2x + (n-1)y] \]
Given $S=360$, we get:
\[ \frac{n}{2} [2x + (n-1)y] = 360 \]

(ii) First, use the other data to find $x$ and $y$:
- After 4 months: \[ S_4 = \frac{4}{2} [2x + 3y] = 2(2x+3y) = 4x + 6y = 120 \]
- After 8 months: \[ S_8 = \frac{8}{2} [2x + 7y] = 4(2x+7y) = 8x + 28y = 192 \]

Now, solve these simultaneous equations:
\[
\begin{cases}
4x + 6y = 120 \\
8x + 28y = 192
\end{cases}
\]
Divide the second equation by 2: 
\[
4x + 14y = 96
\]
Subtract the first equation from this:
\[
(4x + 14y) - (4x + 6y) = 96 - 120 \\
8y = -24 \\
y = -3
\]
Substitute $y = -3$ into $4x + 6(-3) = 120$:
\[
4x - 18 = 120 \\
4x = 138 \\
x = 34.5
\]
Now, substitute $x$ and $y$ into the formula:
\[ \frac{n}{2}[2x + (n-1)y] = 360 \]
\[
\frac{n}{2}[2 \times 34.5 + (n-1)(-3)] = 360 \\
\frac{n}{2}[69 - 3(n-1)] = 360 \\
\frac{n}{2}[69 - 3n + 3] = 360 \\
\frac{n}{2}[72 - 3n] = 360 \\
n(72 - 3n) = 720 \\
72n - 3n^2 = 720 \\
3n^2 - 72n + 720 = 0 \\
n^2 - 24n + 240 = 0
\]
This is a quadratic equation. Solve for $n$:
\[
n^2 - 24n + 240 = 0 \\
n = \frac{24 \pm \sqrt{24^2 - 4 \times 1 \times 240}}{2} \\
n = \frac{24 \pm \sqrt{576 - 960}}{2} \\
n = \frac{24 \pm \sqrt{-384}}{2} 
\]
Since the discriminant is negative, this means the values for $x$ and $y$ cause the total savings never to reach \textdollar360. This suggests a review of the real-world logic: negative $y$ means her savings decrease each month, so she cannot reach \textdollar360. 

But assuming she 'saves less each month' is the intended scenario:
- Number of months Wei Leng can save until her total is \textdollar360 is actually **never reached** (the sequence turns negative before then), thus no positive integer $n$ solution exists for $S_n = \textdollar360$ with these $x$ and $y$.

Therefore, the final answer is: Wei Leng is unable to reach a total of \textdollar360 with her current savings plan as her monthly savings decrease over time.

\section*{Question 57}
\textbf{Metadata}

\begin{itemize}
  \item Question ID: O2-SPFndmdn\_O3-SPFndrng\_GPT4.1\_Household Finance\_01
  \item Primary KC: STATISTICS AND PROBABILITY | Finding median | Finding median for a set of data
  \item Secondary KC: STATISTICS AND PROBABILITY | Finding range | finding range as measures of spread for a set of data 
  \item Topic: Household finance such as income, utility bills, money, interest, savings, instalment, mortgage, financial planning etc.
  \item Grade: Secondary O-level 2
\end{itemize}

\textbf{Solution}

(a) First, list the bills in ascending order: \ 
\[ 68,\ 72,\ 78,\ 80,\ 85,\ 90,\ 95 \] \
Since there are 7 values (an odd number), the median is the 4th value.\
Therefore, the median monthly electricity bill is \textdollar80.\

(b) The range is found by subtracting the lowest bill from the highest bill:\ 
\[ \text{Range} = 95 - 68 = 27 \] \
So, the range of the electricity bills is \textdollar27.

\section*{Question 58}
\textbf{Metadata}

\begin{itemize}
  \item Question ID: O2-SPFndmean\_O3-BPRepSN\_GPT4.1\_Household Finance\_01
  \item Primary KC: STATISTICS AND PROBABILITY | Finding mean deviation | calculation of the mean for a set of data
  \item Secondary KC: BASE AND POWER | Representation and concept  | use of standard form Ax10^n , where n is an integer, and 1<= A<= 10
  \item Topic: Household finance such as income, utility bills, money, interest, savings, instalment, mortgage, financial planning etc.
  \item Grade: Secondary O-level 2
\end{itemize}

\textbf{Solution}

(a) Sum of the bills:
$$
(2.5 \times 10^2) + (3.1 \times 10^2) + (2.8 \times 10^2) + (2.6 \times 10^2) = (2.5 + 3.1 + 2.8 + 2.6) \times 10^2 = 11.0 \times 10^2
$$

Mean monthly bill:
$$
\text{Mean} = \frac{11.0 \times 10^2}{4} = 2.75 \times 10^2
$$
To express in standard form ($1 \leq A < 10$):
$$
2.75 \times 10^2 \quad (\text{already in standard form})
$$

(b) Find the deviations from the mean and their absolute values:
\begin{align*}
|2.5 \times 10^2 - 2.75 \times 10^2| &= | -0.25 \times 10^2 | = 0.25 \times 10^2 \\
|3.1 \times 10^2 - 2.75 \times 10^2| &= |0.35 \times 10^2| = 0.35 \times 10^2 \\
|2.8 \times 10^2 - 2.75 \times 10^2| &= |0.05 \times 10^2| = 0.05 \times 10^2 \\
|2.6 \times 10^2 - 2.75 \times 10^2| &= |-0.15 \times 10^2| = 0.15 \times 10^2
\end{align*}

Sum of absolute deviations:
$$
0.25 \times 10^2 + 0.35 \times 10^2 + 0.05 \times 10^2 + 0.15 \times 10^2 = 0.8 \times 10^2
$$

Mean deviation:
$$
\text{Mean deviation} = \frac{0.8 \times 10^2}{4} = 0.2 \times 10^2 = 2.0 \times 10^1
$$

\textbf{Final Answers:}
(a) Mean monthly bill: $2.75 \times 10^2$

(b) Mean deviation: $2.0 \times 10^1$

\section*{Question 59}
\textbf{Metadata}

\begin{itemize}
  \item Question ID: O3-BPOpr\_O3-BPRepNegI\_GPT4.1\_Household Finance\_01
  \item Primary KC: BASE AND POWER | Operations | laws of indices
  \item Secondary KC: BASE AND POWER | Representation and concept  | negative indices
  \item Topic: Household finance such as income, utility bills, money, interest, savings, instalment, mortgage, financial planning etc.
  \item Grade: Secondary O-level 3/4
\end{itemize}

\textbf{Solution}

(a) From January to February, the bill is multiplied by $2^{1.5}$, and from February to March, it is again multiplied by $2^{1.5}$. Therefore, the amount in March is:
\[
100 \times (2^{1.5}) \times (2^{1.5}) = 100 \times 2^{1.5 + 1.5} = 100 \times 2^{3}.
\]
So, the bill for March is \textdollar$100 \times 2^{3}$, or \textdollar$800.

(b) If Jasmine's bill had decreased by a factor of $2^{-2}$, then the new bill would be:
\[
100 \times 2^{-2} = 100 \times \frac{1}{2^2} = 100 \times \frac{1}{4} = 25.
\]
This means a negative index, $2^{-2}$, represents dividing by $2^2$ or 4. If her bill decreased by this factor, it would become one-fourth the original amount, or \textdollar$25 in March.

\section*{Question 60}
\textbf{Metadata}

\begin{itemize}
  \item Question ID: O3-MXMulSM\_O3-MXAdd\_GPT4.1\_Household Finance\_01
  \item Primary KC: MATRICES | Multiplication | product of a scalar quantity and a matrix
  \item Secondary KC: MATRICES | Addition | addition of matrices
  \item Topic: Household finance such as income, utility bills, money, interest, savings, instalment, mortgage, financial planning etc.
  \item Grade: Secondary O-level 3/4
\end{itemize}

\textbf{Solution}

(a) An increase of 5% means multiplying each element of matrix $A$ by $1.05$:
\[
\text{New utility bills} = 1.05 \times A = 1.05 \times \begin{bmatrix} 120 \\ 130 \\ 140 \end{bmatrix} = \begin{bmatrix} 1.05 \times 120 \\ 1.05 \times 130 \\ 1.05 \times 140 \end{bmatrix} = \begin{bmatrix} 126 \\ 136.5 \\ 147 \end{bmatrix}
\]

(b) To find the total for each month, add the unexpected expenses ($B$) to the new utility bills:
\[
\text{Total for each month} = \begin{bmatrix} 126 \\ 136.5 \\ 147 \end{bmatrix} + \begin{bmatrix} 20 \\ 30 \\ 10 \end{bmatrix} = \begin{bmatrix} 126 + 20 \\ 136.5 + 30 \\ 147 + 10 \end{bmatrix} = \begin{bmatrix} 146 \\ 166.5 \\ 157 \end{bmatrix}
\]

So, the total monthly expenses (utility + unexpected) for the next three months will be \textdollar146, \textdollar166.5, and \textdollar157 respectively.

\section*{Question 61}
\textbf{Metadata}

\begin{itemize}
  \item Question ID: O3-MXSub\_O3-MXAdd\_GPT4.1\_Household Finance\_01
  \item Primary KC: MATRICES | Subtraction | subtraction of matrices
  \item Secondary KC: MATRICES | Addition | addition of matrices
  \item Topic: Household finance such as income, utility bills, money, interest, savings, instalment, mortgage, financial planning etc.
  \item Grade: Secondary O-level 3/4
\end{itemize}

\textbf{Solution}

(a) To find the change in each category from January to February, subtract the January expenses from the February expenses:

\[
\begin{pmatrix}
110 & 370 & 90
\end{pmatrix}
-
\begin{pmatrix}
120 & 350 & 80
\end{pmatrix}
=
\begin{pmatrix}
110 - 120 & 370 - 350 & 90 - 80
\end{pmatrix}
=
\begin{pmatrix}
-10 & 20 & 10
\end{pmatrix}
\]

So, the utilities decreased by \textdollar10, groceries increased by \textdollar20, and transport increased by \textdollar10.

(b) The increase from February to March is represented by the matrix:

\[
\begin{pmatrix} 20 & 15 & 10 \end{pmatrix}
\]

Adding this to the February expenses:

\[
\begin{pmatrix}
110 & 370 & 90
\end{pmatrix}
+
\begin{pmatrix}
20 & 15 & 10
\end{pmatrix}
=
\begin{pmatrix}
110+20 & 370+15 & 90+10
\end{pmatrix}
=
\begin{pmatrix}
130 & 385 & 100
\end{pmatrix}
\]

Therefore, in March, the Tan family spent \textdollar130 on utilities, \textdollar385 on groceries, and \textdollar100 on transport.

\section*{Question 62}
\textbf{Metadata}

\begin{itemize}
  \item Question ID: O3-SPAddProb\_O2-SPRepPrSE\_GPT4.1\_Household Finance\_01
  \item Primary KC: STATISTICS AND PROBABILITY | Addition | addition of probabilities
  \item Secondary KC: STATISTICS AND PROBABILITY | Representation and concept | probability of single events
  \item Topic: Household finance such as income, utility bills, money, interest, savings, instalment, mortgage, financial planning etc.
  \item Grade: Secondary O-level 3/4
\end{itemize}

\textbf{Solution}

Let $P(E)$ be the probability that the electricity bill arrives on a Monday, and $P(W)$ be the probability that the water bill arrives on a Monday. We are told $P(E) = 0.2$, $P(W) = 0.3$. Since the two events are mutually exclusive, the probability that at least one utility bill arrives on a Monday is:

\[
P(E \text{ or } W) = P(E) + P(W) = 0.2 + 0.3 = 0.5
\]

Thus, the probability that at least one utility bill arrives on a Monday is $0.5$.

\section*{Question 63}
\textbf{Metadata}

\begin{itemize}
  \item Question ID: O3-SPAddProb\_O3-SPFndPrCE\_GPT4.1\_Household Finance\_01
  \item Primary KC: STATISTICS AND PROBABILITY | Addition | addition of probabilities
  \item Secondary KC: STATISTICS AND PROBABILITY | Finding probability | probability of simple combined events
  \item Topic: Household finance such as income, utility bills, money, interest, savings, instalment, mortgage, financial planning etc.
  \item Grade: Secondary O-level 3/4
\end{itemize}

\textbf{Solution}

Let $A$ be the event that the family spends more than \textdollar200 on electricity ($P(A) = 0.3$) and $B$ be the event that the family spends more than \textdollar120 on water ($P(B) = 0.2$). We are given that $P(A \cup B) = 0.4$.

Recall that:
$$P(A \cup B) = P(A) + P(B) - P(A \cap B)$$
Substitute the given probabilities:
$$0.4 = 0.3 + 0.2 - P(A \cap B)$$
$$0.4 = 0.5 - P(A \cap B)$$
Therefore:
$$P(A \cap B) = 0.5 - 0.4 = 0.1$$

So, the probability that Jia Wei's family spends more than \textdollar200 on electricity and more than \textdollar120 on water at the same time is $0.1$.

\section*{Question 64}
\textbf{Metadata}

\begin{itemize}
  \item Question ID: O3-SPMulProb\_O3-SPFndPrCE\_GPT4.1\_Household Finance\_01
  \item Primary KC: STATISTICS AND PROBABILITY | Multiplication | multiplication of probabilities
  \item Secondary KC: STATISTICS AND PROBABILITY | Finding probability | probability of simple combined events
  \item Topic: Household finance such as income, utility bills, money, interest, savings, instalment, mortgage, financial planning etc.
  \item Grade: Secondary O-level 3/4
\end{itemize}

\textbf{Solution}

Let $E$ be the event that Jiawei's electricity bill is below \textdollar90, and let $W$ be the event that his water bill is below \textdollar30. We are given:
\begin{align*}
P(E) &= 0.6 \\
P(W) &= 0.7
\end{align*}
Since the bills are independent, the probability that both events occur is:
\begin{align*}
P(E \text{ and } W) = P(E) \times P(W) = 0.6 \times 0.7 = 0.42
\end{align*}
\textbf{Thus, the probability that both bills are below their respective amounts next month is $0.42$.}

\end{document}
