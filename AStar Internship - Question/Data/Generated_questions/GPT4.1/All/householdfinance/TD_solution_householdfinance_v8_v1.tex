\documentclass{article}
\usepackage[utf8]{inputenc}
\usepackage{amsmath}
\usepackage{amsfonts}
\usepackage{amssymb}
\usepackage{graphicx}
\usepackage{hyperref}
\title{'TD Solutions household finance v8 v1'}
\author{Tien Dung Doan}
\begin{document}
\maketitle
\section*{Question 1}
\textbf{Metadata}

\begin{itemize}
  \item Question ID: P3-WNAdd4d\_P1-WNCmp\_GPT4.1\_Household Finance\_03
  \item Primary KC: WHOLE NUMBERS | Addition | adding whole numbers up to 4 digits
  \item Secondary KC: WHOLE NUMBERS | Comparison and ordering | comparing and ordering whole numbers
  \item Topic: Household finance such as income, utility bills, money, interest, savings, instalment, mortgage, financial planning etc.
  \item Grade: Primary 3
\end{itemize}

\textbf{Solution}

To find the total amount Mr Lim needs to pay for utilities, we add the three bills together:

$
1589 + 623 + 812 = 2222 + 812 = 2634
$

So, the total utilities bill is \textdollar2634.

Now, compare this to \textdollar3000:

$
2634 < 3000
$

Therefore, the total bill is less than \textdollar3000.

\section*{Question 2}
\textbf{Metadata}

\begin{itemize}
  \item Question ID: P3-WNSub4d\_P1-WNAdd2nd\_GPT4.1\_Household Finance\_02
  \item Primary KC: WHOLE NUMBERS | Subtraction | subtracting whole numbers up to 4 digits
  \item Secondary KC: WHOLE NUMBERS | Addition | adding whole numbers
  \item Topic: Household finance such as income, utility bills, money, interest, savings, instalment, mortgage, financial planning etc.
  \item Grade: Primary 3
\end{itemize}

\textbf{Solution}

First, we find out how much Sarah spent before her father gave her extra money: 

She started with \textdollar450 and had \textdollar215 left, so she spent:

$
450 - 215 = 235
$

Sarah's father then gave her \textdollar80 more, but this does not affect the amount she spent on groceries before receiving the extra money. So, Sarah spent \textdollar235 on groceries before her father gave her extra money.

\section*{Question 3}
\textbf{Metadata}

\begin{itemize}
  \item Question ID: P3-WNDivRmd3d\_P1-WNSub2nd\_GPT4.1\_Household Finance\_02
  \item Primary KC: WHOLE NUMBERS | Division | dividing whole numbers up to 3 digits by 1 digit with remainder 
  \item Secondary KC: WHOLE NUMBERS | Subtraction | subtracting whole numbers
  \item Topic: Household finance such as income, utility bills, money, interest, savings, instalment, mortgage, financial planning etc.
  \item Grade: Primary 3
\end{itemize}

\textbf{Solution}

(a) To find the number of candies each child receives, we divide 248 by 5:

$ 248 \div 5 = 49 \text{ remainder } 3 $

So, each child receives $49$ candies, and $3$ candies are left.

(b) Mrs Tan has $3$ leftover candies. If she eats $2$ fewer candies than the leftover, she eats:

$ 3 - 2 = 1 $

So, Mrs Tan eats $1$ candy.

\section*{Question 4}
\textbf{Metadata}

\begin{itemize}
  \item Question ID: P3-WNMul3d1d\_P1-WNCmp\_GPT4.1\_Household Finance\_02
  \item Primary KC: WHOLE NUMBERS | Multiplication | multiplying whole numbers up to 3 digits by 1 digit
  \item Secondary KC: WHOLE NUMBERS | Comparison and ordering | comparing and ordering whole numbers
  \item Topic: Household finance such as income, utility bills, money, interest, savings, instalment, mortgage, financial planning etc.
  \item Grade: Primary 3
\end{itemize}

\textbf{Solution}

(a) Ben saves \textdollar7 each week for 8 weeks.\newline
Amount saved by Ben $= 7 \times 8 = 56$\newline
So, Ben has \textdollar56 after 8 weeks.

(b) Jane saves \textdollar9 each week for 6 weeks.\newline
Amount saved by Jane $= 9 \times 6 = 54$\newline
So, Jane has \textdollar54 after 6 weeks.

(c) Comparing their savings:\newline
Ben: \textdollar56\newline
Jane: \textdollar54\newline
$56 > 54$, so Ben has more money.\newline
Difference $= 56 - 54 = 2$\newline
So, Ben has \textdollar2 more than Jane.

\section*{Question 5}
\textbf{Metadata}

\begin{itemize}
  \item Question ID: P3-WNDiv3d1d\_P1-WNCmp\_GPT4.1\_Household Finance\_02
  \item Primary KC: WHOLE NUMBERS | Division | dividing whole numbers up to 3 digits by 1 digit
  \item Secondary KC: WHOLE NUMBERS | Comparison and ordering | comparing and ordering whole numbers
  \item Topic: Household finance such as income, utility bills, money, interest, savings, instalment, mortgage, financial planning etc.
  \item Grade: Primary 3
\end{itemize}

\textbf{Solution}

(a) To find out how much money each child receives, divide Mdm Tan's savings by 3:

$276 \div 3 = 92$

So, each child receives $\textdollar92$.

(b) After giving out the pocket money, Mdm Tan has:

$276 - 3 \times 92 = 276 - 276 = 0$

So, Mdm Tan has $\textdollar0$ left in her savings.

Comparing with Mrs Lim who has $\textdollar92$ left:

$92 - 0 = 92$

Mrs Lim has more money left in her savings than Mdm Tan, by $\textdollar92$.

\section*{Question 6}
\textbf{Metadata}

\begin{itemize}
  \item Question ID: P3-FrSubRl12\_P2-FrCmp\_GPT4.1\_Household Finance\_02
  \item Primary KC: FRACTIONS | Subtraction | subtracting two related fractions within one whole with denominators of given fractions not exceeding 12
  \item Secondary KC: FRACTIONS | Comparison and ordering | comparing and ordering fractions
  \item Topic: Household finance such as income, utility bills, money, interest, savings, instalment, mortgage, financial planning etc.
  \item Grade: Primary 3
\end{itemize}

\textbf{Solution}

(a) To find how much more Siti spent on groceries than on electricity bills, subtract the fraction for electricity bills from groceries:
$
\frac{5}{12} - \frac{1}{4}
$
First, convert $\frac{1}{4}$ to twelfths:
$
\frac{1}{4} = \frac{1 \times 3}{4 \times 3} = \frac{3}{12}
$
So,
$
\frac{5}{12} - \frac{3}{12} = \frac{2}{12} = \frac{1}{6}
$
Siti spent $\frac{1}{6}$ more of her allowance on groceries than on electricity bills.

(b) Comparing $\frac{5}{12}$ and $\frac{1}{4}$:
Convert $\frac{1}{4}$ to $\frac{3}{12}$.
So, $\frac{1}{4}$ ($\frac{3}{12}$) is less than $\frac{5}{12}$.

Order from least to greatest: $\frac{1}{4},\ \frac{5}{12}$.

\section*{Question 7}
\textbf{Metadata}

\begin{itemize}
  \item Question ID: P4-WNMul4d1d\_P1-WNCmp\_GPT4.1\_Household Finance\_02
  \item Primary KC: WHOLE NUMBERS | Multiplication | multiplying whole numbers up to 4 digits by 1 digit or up to 3 digits by 2 digits
  \item Secondary KC: WHOLE NUMBERS | Comparison and ordering | comparing and ordering whole numbers
  \item Topic: Household finance such as income, utility bills, money, interest, savings, instalment, mortgage, financial planning etc.
  \item Grade: Primary 4
\end{itemize}

\textbf{Solution}

First, calculate the total number of light bulbs each family bought.\
\
For the first family:\\
$6 \times 125 = 750$\
\
For the second family:\\
$9 \times 90 = 810$\
\
Now, compare the numbers $750$ and $810$.\
\
$810 > 750$\
\
So, the second family has more light bulbs.\
\
Find how many more light bulbs they have:\\
$810 - 750 = 60$\
\
\textbf{Answer:} The second family has 60 more light bulbs than the first family.

\section*{Question 8}
\textbf{Metadata}

\begin{itemize}
  \item Question ID: P4-WNMul4d1d\_P4-WNRnd5d\_GPT4.1\_Household Finance\_02
  \item Primary KC: WHOLE NUMBERS | Multiplication | multiplying whole numbers up to 4 digits by 1 digit or up to 3 digits by 2 digits
  \item Secondary KC: WHOLE NUMBERS | Rounding | rounding whole numbers up to 100000 to the nearest 10, 100 or 1000 
  \item Topic: Household finance such as income, utility bills, money, interest, savings, instalment, mortgage, financial planning etc.
  \item Grade: Primary 4
\end{itemize}

\textbf{Solution}

(a) There are 12 months in 1 year.

Total amount spent in 1 year $= 57 \times 12 = 684$

So, the family spends \textdollar684 in total on electricity in 1 year.

(b) We need to round 684 to the nearest hundred dollars.

Since the tens digit of 684 is 8 (which is 5 or more), we round up.

684 rounded to the nearest hundred dollars is \textdollar700.

\section*{Question 9}
\textbf{Metadata}

\begin{itemize}
  \item Question ID: P4-WNDiv4d1d\_P1-WNSub2nd\_GPT4.1\_Household Finance\_02
  \item Primary KC: WHOLE NUMBERS | Division | dividing whole numbers up to 4 digits by 1 digit
  \item Secondary KC: WHOLE NUMBERS | Subtraction | subtracting whole numbers
  \item Topic: Household finance such as income, utility bills, money, interest, savings, instalment, mortgage, financial planning etc.
  \item Grade: Primary 4
\end{itemize}

\textbf{Solution}

First, subtract the amount Mr. Lim kept for household expenses from his total salary:

$
3872 - 320 = 3552
$

Mr. Lim divided \textdollar3552 equally among his 4 children. To find out how much each child received, divide:

$
3552 \div 4 = 888
$

Therefore, each child received \textdollar888.

\section*{Question 10}
\textbf{Metadata}

\begin{itemize}
  \item Question ID: P4-WNDiv4d1d\_P4-WNRnd5d\_GPT4.1\_Household Finance\_02
  \item Primary KC: WHOLE NUMBERS | Division | dividing whole numbers up to 4 digits by 1 digit
  \item Secondary KC: WHOLE NUMBERS | Rounding | rounding whole numbers up to 100000 to the nearest 10, 100 or 1000 
  \item Topic: Household finance such as income, utility bills, money, interest, savings, instalment, mortgage, financial planning etc.
  \item Grade: Primary 4
\end{itemize}

\textbf{Solution}

(a) The family's total electricity bill is \textdollar3287. 

To round \$3287 to the nearest ten:
- The digit in the tens place is 8 and the digit to its right (ones place) is 7.
- Since 7 is 5 or more, we round up: \$3287 rounded to the nearest ten is \textdollar3290.

So, the rounded amount is \boxed{\textdollar3290}

(b) To find how much the family pays each month, divide the rounded amount by 4:

$3290 \div 4 = 822.5$

So, each month, they need to pay \boxed{\textdollar822.50}$.

\section*{Question 11}
\textbf{Metadata}

\begin{itemize}
  \item Question ID: P4-FrAddU12\_P2-FrCmp\_GPT4.1\_Household Finance\_02
  \item Primary KC: FRACTIONS | Addition | adding unlike fractions with two different denominators not exceeding 12
  \item Secondary KC: FRACTIONS | Comparison and ordering | comparing and ordering fractions
  \item Topic: Household finance such as income, utility bills, money, interest, savings, instalment, mortgage, financial planning etc.
  \item Grade: Primary 4
\end{itemize}

\textbf{Solution}

(a) To find the total fraction saved:

The denominators are $8$ and $3$. The least common denominator (LCD) is $24$.

Convert $\frac{3}{8}$ to a denominator of $24$:
$\frac{3}{8} = \frac{3 \times 3}{8 \times 3} = \frac{9}{24}$

Convert $\frac{2}{3}$ to a denominator of $24$:
$\frac{2}{3} = \frac{2 \times 8}{3 \times 8} = \frac{16}{24}$

Add the two fractions:
$\frac{9}{24} + \frac{16}{24} = \frac{25}{24}$

Grace saved $\frac{25}{24}$ of her allowance in total.

(b) Since $\frac{25}{24} > 1$, Grace did save more than 1 whole allowance over the two weeks. $\frac{25}{24} = 1 \frac{1}{24}$, which is more than $1$.

\section*{Question 12}
\textbf{Metadata}

\begin{itemize}
  \item Question ID: P4-FrSubU12\_P3-FrSmp\_GPT4.1\_Household Finance\_02
  \item Primary KC: FRACTIONS | Subtraction | subtracting unlike fractions with two different denominators not exceeding 12
  \item Secondary KC: FRACTIONS | Simplifying | expressing a fraction in its simplest form
  \item Topic: Household finance such as income, utility bills, money, interest, savings, instalment, mortgage, financial planning etc.
  \item Grade: Primary 4
\end{itemize}

\textbf{Solution}

First, find out how much Amy spent by writing both money amounts as improper fractions of a dollar:

$\frac{7\times12+5}{12} = \frac{89}{12}$ and $\frac{1\times3+1}{3} = \frac{4}{3}$

Now, subtract $\frac{4}{3}$ from $\frac{89}{12}$ with a common denominator of 12:
$
\frac{4}{3} = \frac{4 \times 4}{3 \times 4} = \frac{16}{12}
$
So,
$
\frac{89}{12} - \frac{16}{12} = \frac{73}{12}
$

Now, write $\frac{73}{12}$ as a mixed number:
- $73 \div 12 = 6$ remainder 1, so $\frac{73}{12} = 6\frac{1}{12}$

Therefore, Amy has $\textdollar6$ $\frac{1}{12}$ left.

\textbf{Answer:} Amy had $\textdollar6$ $\frac{1}{12}$ left.

\section*{Question 13}
\textbf{Metadata}

\begin{itemize}
  \item Question ID: P4-DcAdd2d\_P4-DcCmp3d\_GPT4.1\_Household Finance\_02
  \item Primary KC: DECIMALS | Addition | adding decimals (up to 2 decimal places)
  \item Secondary KC: DECIMALS | Comparison and ordering | comparing and ordering decimals up to 3 decimal places
  \item Topic: Household finance such as income, utility bills, money, interest, savings, instalment, mortgage, financial planning etc.
  \item Grade: Primary 4
\end{itemize}

\textbf{Solution}

(a) To find the total amount spent in June, we add the three bills together:

$45.60 + 23.75 + 12.85 = 82.20$

Sarah spent $\textdollar 82.20$ on utility bills in June.

(b) In July, the electricity bill is $45.605$ and in June it was $45.60$.

Since $45.605 > 45.60$ (because $45.605$ is $0.005$ more than $45.600$), the electricity bill was higher in July than in June.

\section*{Question 14}
\textbf{Metadata}

\begin{itemize}
  \item Question ID: P4-DcSub2d\_P4-DcCnv2Fr\_GPT4.1\_Household Finance\_02
  \item Primary KC: DECIMALS | Subtraction | subtracting decimals (up to 2 decimal places)
  \item Secondary KC: DECIMALS | Conversion from decimals to fraction | expressing decimals as fractions
  \item Topic: Household finance such as income, utility bills, money, interest, savings, instalment, mortgage, financial planning etc.
  \item Grade: Primary 4
\end{itemize}

\textbf{Solution}

(a) Amount of money Rachel had left:

$
12.60 - 4.75 = 7.85
$

Rachel had \textdollar7.85 left.

(b) Express \textdollar7.85 as a fraction:

$
7.85 = 7 + 0.85 = 7 + \frac{85}{100} = 7 + \frac{17}{20}
$

As an improper fraction:
$
7\frac{17}{20} = \frac{140}{20} + \frac{17}{20} = \frac{157}{20}
$

So, Rachel had \( \frac{157}{20} \) dollars left, in simplest form.

\section*{Question 15}
\textbf{Metadata}

\begin{itemize}
  \item Question ID: P4-DcSub2d\_P4-DcRnd3d\_GPT4.1\_Household Finance\_02
  \item Primary KC: DECIMALS | Subtraction | subtracting decimals (up to 2 decimal places)
  \item Secondary KC: DECIMALS | Rounding | rounding decimals up to 3 decimal places to the nearest whole number, 1 decimal place and 2 decimal places 
  \item Topic: Household finance such as income, utility bills, money, interest, savings, instalment, mortgage, financial planning etc.
  \item Grade: Primary 4
\end{itemize}

\textbf{Solution}

Part (a):

To find out how much less Mrs Tan paid in May than in April, subtract the May bill from the April bill:

$
123.45 - 97.88 = 25.57
$

Mrs Tan paid \textdollar25.57 less in May.

Part (b):

- To the nearest dollar: \( 25.57 \) rounds to \textdollar26 (since the tenths digit is 5 or above).
- To 1 decimal place: \( 25.57 \) rounds to \textdollar25.6 (since the hundredths digit is 7, so we round up the tenths digit).

$
\boxed{\textdollar26}\quad\text{(nearest dollar)},\quad \boxed{\textdollar25.6}\quad\text{(1 decimal place)}
$


\section*{Question 16}
\textbf{Metadata}

\begin{itemize}
  \item Question ID: P4-DcMul2d1d\_P4-DcCnv2Fr\_GPT4.1\_Household Finance\_01
  \item Primary KC: DECIMALS | Multiplication | multiplying decimals (up to 2 decimal places) by a 1-digit whole number
  \item Secondary KC: DECIMALS | Conversion from decimals to fraction | expressing decimals as fractions
  \item Topic: Household finance such as income, utility bills, money, interest, savings, instalment, mortgage, financial planning etc.
  \item Grade: Primary 4
\end{itemize}

\textbf{Solution}

(a) Amount spent in $5$ days:

$2.75 \times 5 = 13.75$

Mrs Lim spends \textdollar13.75 on lunch in $5$ days.

(b) Express $2.75$ as a fraction:

$2.75 = 2 + 0.75 = \frac{2}{1} + \frac{75}{100}$

Simplify $\frac{75}{100}$:

$\frac{75}{100} = \frac{3}{4}$

So, $2.75 = 2\frac{3}{4}$ or $\frac{11}{4}$

Mrs Lim spends $\frac{11}{4}$ dollars on lunch in one day.

\section*{Question 17}
\textbf{Metadata}

\begin{itemize}
  \item Question ID: P4-DcMul2d1d\_P4-DcAdd2nd\_GPT4.1\_Household Finance\_01
  \item Primary KC: DECIMALS | Multiplication | multiplying decimals (up to 2 decimal places) by a 1-digit whole number
  \item Secondary KC: DECIMALS | Addition | adding decimals
  \item Topic: Household finance such as income, utility bills, money, interest, savings, instalment, mortgage, financial planning etc.
  \item Grade: Primary 4
\end{itemize}

\textbf{Solution}

First, find the total amount spent on electricity in 4 weeks:

$
3.75 \times 4 = 15.00
$

So, they spent \textdollar15.00 on regular electricity usage. 

Next, add the additional cost for lighting:

$
15.00 + 5.20 = 20.20
$

Therefore, their total spending on electricity and lighting for the month is \textdollar20.20.

\section*{Question 18}
\textbf{Metadata}

\begin{itemize}
  \item Question ID: P4-DcDiv2d1d\_P4-DcCmp3d\_GPT4.1\_Household Finance\_01
  \item Primary KC: DECIMALS | Division | dividing decimals (up to 2 decimal places) by a 1-digit whole number
  \item Secondary KC: DECIMALS | Comparison and ordering | comparing and ordering decimals up to 3 decimal places
  \item Topic: Household finance such as income, utility bills, money, interest, savings, instalment, mortgage, financial planning etc.
  \item Grade: Primary 4
\end{itemize}

\textbf{Solution}

(a) To find out how much each child should pay:

$23.94 \div 3 = 7.98$

Each child should pay \textdollar7.98.

(b) The four amounts to compare are:

\textdollar7.980, \textdollar8.110, \textdollar7.950, \textdollar7.98

Since $\textdollar7.98$ and $\textdollar7.980$ are equal, we have:

$\textdollar7.950 < \textdollar7.98 = \textdollar7.980 < \textdollar8.110$

So, from least to greatest:

\textdollar7.950, \textdollar7.98, \textdollar7.980, \textdollar8.110

\section*{Question 19}
\textbf{Metadata}

\begin{itemize}
  \item Question ID: P4-DcDiv2d1d\_P4-DcAdd2nd\_GPT4.1\_Household Finance\_01
  \item Primary KC: DECIMALS | Division | dividing decimals (up to 2 decimal places) by a 1-digit whole number
  \item Secondary KC: DECIMALS | Addition | adding decimals
  \item Topic: Household finance such as income, utility bills, money, interest, savings, instalment, mortgage, financial planning etc.
  \item Grade: Primary 4
\end{itemize}

\textbf{Solution}

First, Maya divides \textdollar12.80 equally among 4 piggy banks:

$
\frac{12.80}{4} = 3.20
$

So, each piggy bank has \textdollar3.20 after the first division.

Next, Maya adds \textdollar3.20, dividing it equally among the 4 piggy banks:

$
\frac{3.20}{4} = 0.80
$

She adds \textdollar0.80 to each piggy bank.

So, the total in each piggy bank is:

$
3.20 + 0.80 = 4.00
$

\textbf{Each piggy bank now has \textdollar4.00.}

\section*{Question 20}
\textbf{Metadata}

\begin{itemize}
  \item Question ID: P5-FrAddMix\_P2-FrCmp\_GPT4.1\_Household Finance\_01
  \item Primary KC: FRACTIONS | Addition | adding mixed numbers
  \item Secondary KC: FRACTIONS | Comparison and ordering | comparing and ordering fractions
  \item Topic: Household finance such as income, utility bills, money, interest, savings, instalment, mortgage, financial planning etc.
  \item Grade: Primary 5
\end{itemize}

\textbf{Solution}

(a) To find the total amount Mrs Tan spent on groceries and utility bills, add the two mixed numbers:

$1\dfrac{1}{4} + 2\dfrac{2}{3} = \left(1 + 2\right) + \left(\dfrac{1}{4} + \dfrac{2}{3}\right)$

First, find a common denominator for the fractions.

The least common denominator (LCD) of 4 and 3 is 12.

$\dfrac{1}{4} = \dfrac{3}{12}$

$\dfrac{2}{3} = \dfrac{8}{12}$

So, $\dfrac{1}{4} + \dfrac{2}{3} = \dfrac{3}{12} + \dfrac{8}{12} = \dfrac{11}{12}$

Now add the whole numbers:

$1 + 2 = 3$

So the total amount spent is:

$3\dfrac{11}{12}$

(b) To compare $1\dfrac{1}{4}$ and $2\dfrac{2}{3}$, convert both to improper fractions:

$1\dfrac{1}{4} = \dfrac{5}{4}$

$2\dfrac{2}{3} = \dfrac{8}{3}$

Find the decimal values or a common denominator:

$\dfrac{5}{4} = 1.25$

$\dfrac{8}{3} \approx 2.67$

So, she spent less on groceries and more on utility bills.

Arranged in ascending order:

1. Groceries: $1\dfrac{1}{4}$
2. Utility bills: $2\dfrac{2}{3}$

\section*{Question 21}
\textbf{Metadata}

\begin{itemize}
  \item Question ID: P5-FrSubMix\_P2-FrCmp\_GPT4.1\_Household Finance\_01
  \item Primary KC: FRACTIONS | Subtraction | subtracting mixed numbers
  \item Secondary KC: FRACTIONS | Comparison and ordering | comparing and ordering fractions
  \item Topic: Household finance such as income, utility bills, money, interest, savings, instalment, mortgage, financial planning etc.
  \item Grade: Primary 5
\end{itemize}

\textbf{Solution}

(a) To find the amount of money Mrs Tan has left after paying the bill, subtract $4\dfrac{5}{8}$ from $6\dfrac{3}{4}$:\\
\\
$6\dfrac{3}{4} - 4\dfrac{5}{8}$\\
\\
First, write both mixed numbers as improper fractions:\\
$6\dfrac{3}{4} = \frac{27}{4}$\\
$4\dfrac{5}{8} = \frac{37}{8}$\\
\\
To subtract, convert $\frac{27}{4}$ to eighths:\\
$\frac{27}{4} = \frac{54}{8}$\\
\\
So, $\frac{54}{8} - \frac{37}{8} = \frac{17}{8}$\\
\\
Write $\frac{17}{8}$ as a mixed number:\\
$\frac{17}{8} = 2\dfrac{1}{8}$\\
\\
Mrs Tan has $2\dfrac{1}{8}$ left in her savings jar.\\
\\
(b) Compare $2\dfrac{1}{8}$ (Mrs Tan) and $2\dfrac{1}{2}$ (Mei).\\
Write $2\dfrac{1}{2}$ as an improper fraction:\\
$2\dfrac{1}{2} = \frac{5}{2}$\\
Convert $2\dfrac{1}{8}$ to an improper fraction for comparison:\\
$2\dfrac{1}{8} = \frac{17}{8}$\\
Now, convert $\frac{5}{2}$ to eighths: $\frac{5}{2} = \frac{20}{8}$\\
\\
So, $\frac{20}{8} > \frac{17}{8}$, which means Mei has more savings.\\
The difference is $\frac{20}{8} - \frac{17}{8} = \frac{3}{8}$.\\
\\
Answer: Mei has more savings by $\frac{3}{8}$ or $\textdollar\frac{3}{8}$ than Mrs Tan.

\section*{Question 22}
\textbf{Metadata}

\begin{itemize}
  \item Question ID: P5-FrMulImN\_P2-FrCmp\_GPT4.1\_Household Finance\_01
  \item Primary KC: FRACTIONS | Multiplication | multiplying a proper/improper fraction and a whole number
  \item Secondary KC: FRACTIONS | Comparison and ordering | comparing and ordering fractions
  \item Topic: Household finance such as income, utility bills, money, interest, savings, instalment, mortgage, financial planning etc.
  \item Grade: Primary 5
\end{itemize}

\textbf{Solution}

First, calculate the amount Mrs Lim saves in a month:

Amount saved $= \frac{3}{4} \times 120$

Amount saved $= \frac{3 \times 120}{4}$

Amount saved $= \frac{360}{4}$

Amount saved $= 90$

Now, compare $90$ to $90$.

$90 = 90$

Therefore, the amount Mrs Lim saves in a month is equal to \textdollar90.

\section*{Question 23}
\textbf{Metadata}

\begin{itemize}
  \item Question ID: P5-FrMulPIm\_P2-FrCmp\_GPT4.1\_Household Finance\_01
  \item Primary KC: FRACTIONS | Multiplication | multiplying a proper fraction and a proper/improper fractions
  \item Secondary KC: FRACTIONS | Comparison and ordering | comparing and ordering fractions
  \item Topic: Household finance such as income, utility bills, money, interest, savings, instalment, mortgage, financial planning etc.
  \item Grade: Primary 5
\end{itemize}

\textbf{Solution}

(a) Mr Tan's monthly spending on rent is $\frac{2}{5} \times 2400 = \frac{2 \times 2400}{5} = \frac{4800}{5} = 960$. So, he spends \textdollar960 on rent each month.\
\
(b) Amount spent on utilities is $\frac{3}{8}$ of rent = $\frac{3}{8} \times 960 = \frac{2880}{8} = 360$. So, he spends \textdollar360 on utilities each month.\
\
(c) Mr Tan's remaining income after paying for rent and utilities is $2400 - 960 - 360 = 1080$.\
\
Comparing the amounts: utilities (\textdollar360), rent (\textdollar960), remaining income (\textdollar1080).\
\
Arranged from least to greatest: \textdollar360 (utilities), \textdollar960 (rent), \textdollar1080 (remaining income).

\section*{Question 24}
\textbf{Metadata}

\begin{itemize}
  \item Question ID: P5-FrMulPIm\_P2-FrAdd2nd\_GPT4.1\_Household Finance\_01
  \item Primary KC: FRACTIONS | Multiplication | multiplying a proper fraction and a proper/improper fractions
  \item Secondary KC: FRACTIONS | Addition | adding fractions
  \item Topic: Household finance such as income, utility bills, money, interest, savings, instalment, mortgage, financial planning etc.
  \item Grade: Primary 5
\end{itemize}

\textbf{Solution}

First, calculate the amount Mrs Tan spends on groceries:

She spends $\frac{2}{3}$ of $\textdollar600$:

$\frac{2}{3} \times 600 = 400$

So, she spends $\textdollar400$ on groceries.

The amount left after groceries:

$600 - 400 = 200$

Next, Mrs Tan spends $\frac{3}{4}$ of the remaining $\textdollar200$ on utilities:

$\frac{3}{4} \times 200 = 150$

The amount left after paying for utilities:

$200 - 150 = 50$

Now, Mrs Tan decides to add $\frac{1}{6}$ of the remaining $\textdollar50$ to her savings:

$\frac{1}{6} \times 50 = 8\frac{1}{3}$

So, Mrs Tan puts $\textdollar8\frac{1}{3}$ into her savings.

\section*{Question 25}
\textbf{Metadata}

\begin{itemize}
  \item Question ID: P5-FrMulPIm\_P3-FrSmp\_GPT4.1\_Household Finance\_01
  \item Primary KC: FRACTIONS | Multiplication | multiplying a proper fraction and a proper/improper fractions
  \item Secondary KC: FRACTIONS | Simplifying | expressing a fraction in its simplest form
  \item Topic: Household finance such as income, utility bills, money, interest, savings, instalment, mortgage, financial planning etc.
  \item Grade: Primary 5
\end{itemize}

\textbf{Solution}

First, find the total amount spent on both bills:

She used $\frac{2}{5}$ of \textdollar500:

$\frac{2}{5} \times 500 = 200$

So, she spent \textdollar200 on both bills.

The electricity bill is $\frac{3}{4}$ of \textdollar200:

$\frac{3}{4} \times 200 = 150$

So, Mrs Lim paid \textdollar150 for her electricity bill.

Expressing as a fraction of her total savings and simplifying:

The electricity bill as a fraction of her savings:

$\frac{3}{4} \times \frac{2}{5} = \frac{3 \times 2}{4 \times 5} = \frac{6}{20}$

Simplify $\frac{6}{20}$:

Divide numerator and denominator by 2:

$\frac{6 \div 2}{20 \div 2} = \frac{3}{10}$

So, Mrs Lim spent $\frac{3}{10}$ of her savings on her electricity bill.

\section*{Question 26}
\textbf{Metadata}

\begin{itemize}
  \item Question ID: P5-FrMulImIm\_P2-FrAdd2nd\_GPT4.1\_Household Finance\_01
  \item Primary KC: FRACTIONS | Multiplication | multiplying two improper fractions
  \item Secondary KC: FRACTIONS | Addition | adding fractions
  \item Topic: Household finance such as income, utility bills, money, interest, savings, instalment, mortgage, financial planning etc.
  \item Grade: Primary 5
\end{itemize}

\textbf{Solution}

(a) Each carton contains $\frac{9}{4}$ litres. For 3 cartons:\\
$
\text{Total juice} = 3 \times \frac{9}{4} = \frac{27}{4}\text{ litres}
$

(b) Leftover after the gathering: $\frac{5}{3}$ litres from each carton, so total leftover is:\\
$
3 \times \frac{5}{3} = \frac{15}{3} = 5 \text{ litres}
$

(c) Amount consumed is initial total minus leftover:\\
$
\frac{27}{4}\text{ litres} - 5\text{ litres}
$
First, express 5 as a fraction with denominator 4: $5 = \frac{20}{4}$, so\\
$
\text{Amount consumed} = \frac{27}{4} - \frac{20}{4} = \frac{7}{4}\text{ litres}
$

\textbf{Answers:}
(a) $\frac{27}{4}$ litres
(b) $5$ litres
(c) $\frac{7}{4}$ litres

\section*{Question 27}
\textbf{Metadata}

\begin{itemize}
  \item Question ID: P5-FrMulImIm\_P2-FrSub2nd\_GPT4.1\_Household Finance\_01
  \item Primary KC: FRACTIONS | Multiplication | multiplying two improper fractions
  \item Secondary KC: FRACTIONS | Subtraction | subtracting fractions
  \item Topic: Household finance such as income, utility bills, money, interest, savings, instalment, mortgage, financial planning etc.
  \item Grade: Primary 5
\end{itemize}

\textbf{Solution}

First, find the fraction of the savings spent on utilities and groceries by multiplying the given fractions with the total amount saved. 

Amount spent on utilities:
$
\frac{7}{4} \times 240 = \frac{7 \times 240}{4} = \frac{1680}{4} = 420
$

Amount spent on groceries:
$
\frac{5}{3} \times 240 = \frac{5 \times 240}{3} = \frac{1200}{3} = 400
$

Next, subtract the amount spent on groceries from the amount spent on utilities:
$
420 - 400 = 20
$

Therefore, Mrs Tan spent \textdollar20 more on utilities than on groceries.

\section*{Question 28}
\textbf{Metadata}

\begin{itemize}
  \item Question ID: P5-FrMulImIm\_P3-FrSmp\_GPT4.1\_Household Finance\_01
  \item Primary KC: FRACTIONS | Multiplication | multiplying two improper fractions
  \item Secondary KC: FRACTIONS | Simplifying | expressing a fraction in its simplest form
  \item Topic: Household finance such as income, utility bills, money, interest, savings, instalment, mortgage, financial planning etc.
  \item Grade: Primary 5
\end{itemize}

\textbf{Solution}

Let the amount Mrs Tan spent on brushes be $x$. She spent $\frac{7}{4}x$ on paint. So, total spent is $x + \frac{7}{4}x = \frac{11}{4}x$. Given her budget was $\textdollar150$. Set up the equation: 

$\frac{11}{4}x = 150$

Solving for $x$:

$x = \frac{150 \times 4}{11} = \frac{600}{11}$

The amount of money actually used for brushes is $\frac{6}{5} \times x = \frac{6}{5} \times \frac{600}{11} = \frac{3600}{55}$

The fraction of the total budget used for brushes is:

$\frac{3600}{55} \div 150 = \frac{3600}{55} \times \frac{1}{150} = \frac{3600}{55 \times 150}$

$= \frac{3600}{8250}$

Simplifying:

$= \frac{72}{165}$ (Divide numerator and denominator by 50)

Since 72 and 165 have no common factor, the fraction in its simplest form is $\boxed{\frac{72}{165}}$.

So, $\frac{72}{165}$ of her total budget was used for brushes.

\section*{Question 29}
\textbf{Metadata}

\begin{itemize}
  \item Question ID: P5-FrMulMixN\_P2-FrSub2nd\_GPT4.1\_Household Finance\_01
  \item Primary KC: FRACTIONS | Multiplication | multiplying a mixed number and a whole number
  \item Secondary KC: FRACTIONS | Subtraction | subtracting fractions
  \item Topic: Household finance such as income, utility bills, money, interest, savings, instalment, mortgage, financial planning etc.
  \item Grade: Primary 5
\end{itemize}

\textbf{Solution}

(a) Number of batches Mrs Tan baked in 6 days:

$3\dfrac{1}{2} \times 6 = \left(\frac{7}{2}\right) \times 6 = \frac{7 \times 6}{2} = \frac{42}{2} = 21$

Mrs Tan baked 21 batches in 6 days.

(b) Total amount of chocolate used:

Each batch uses $\frac{3}{4}$ kilogram of chocolate.

Total chocolate used $= 21 \times \frac{3}{4} = \frac{63}{4} = 15\dfrac{3}{4}$ kilograms$

(c) Chocolate left at the end of the week:

She started with 20 kilograms.

Amount of chocolate left $= 20 - 15\dfrac{3}{4}$

Convert $15\dfrac{3}{4}$ to an improper fraction:
$15\dfrac{3}{4} = \frac{63}{4}$

$20 = \frac{80}{4}$

So, $\frac{80}{4} - \frac{63}{4} = \frac{17}{4} = 4\dfrac{1}{4}$

Mrs Tan has $4\dfrac{1}{4}$ kilograms of chocolate left at the end of the week.

\section*{Question 30}
\textbf{Metadata}

\begin{itemize}
  \item Question ID: P5-DcMul3dK\_P4-DcCnv2Fr\_GPT4.1\_Household Finance\_01
  \item Primary KC: DECIMALS | Multiplication | multiplying decimals (up to 3 decimal places) by 10, 100, 1000 and their multiples
  \item Secondary KC: DECIMALS | Conversion from decimals to fraction | expressing decimals as fractions
  \item Topic: Household finance such as income, utility bills, money, interest, savings, instalment, mortgage, financial planning etc.
  \item Grade: Primary 5
\end{itemize}

\textbf{Solution}

(a) The daily electricity usage is $3.75$ kWh.
Total usage for $100$ days $= 3.75 \times 100 = 375$ kWh.

Total cost $= 375 \times 0.174 = 65.25$.

Therefore, the cost of electricity used for $100$ days is \textdollar65.25.

(b) $3.75$ as a fraction:

$3.75 = \frac{375}{100}$. 
Simplify $\frac{375}{100}$ by dividing both numerator and denominator by $25$:

$\frac{375}{25} = 15$, $\frac{100}{25} = 4$.
So, $3.75 = \frac{15}{4}.$

Final answers:
(a) \textdollar65.25
(b) $\frac{15}{4}$

\section*{Question 31}
\textbf{Metadata}

\begin{itemize}
  \item Question ID: P5-DcMul3dK\_P4-DcSub2nd\_GPT4.1\_Household Finance\_01
  \item Primary KC: DECIMALS | Multiplication | multiplying decimals (up to 3 decimal places) by 10, 100, 1000 and their multiples
  \item Secondary KC: DECIMALS | Subtraction | subtracting decimals
  \item Topic: Household finance such as income, utility bills, money, interest, savings, instalment, mortgage, financial planning etc.
  \item Grade: Primary 5
\end{itemize}

\textbf{Solution}

First, calculate the electricity cost before including the service fee: 

$73.245 \times 0.15 = 10.98675$

So, the electricity cost is \textdollar10.99 (rounded to 2 decimal places).

Now, add the service fee:

$10.99 + 1.80 = 12.79$

Thus, the actual amount Mr Lim needs to pay is \textdollar12.79.

\section*{Question 32}
\textbf{Metadata}

\begin{itemize}
  \item Question ID: P5-DcDiv3dK\_P4-DcRnd3d\_GPT4.1\_Household Finance\_01
  \item Primary KC: DECIMALS | Division | dividing decimals (up to 3 decimal places) by 10, 100, 1000 and their multiples
  \item Secondary KC: DECIMALS | Rounding | rounding decimals up to 3 decimal places to the nearest whole number, 1 decimal place and 2 decimal places 
  \item Topic: Household finance such as income, utility bills, money, interest, savings, instalment, mortgage, financial planning etc.
  \item Grade: Primary 5
\end{itemize}

\textbf{Solution}

(a) To find the amount in each jar, divide \textdollar253.680 by 10:

$
\frac{253.680}{10} = 25.368
$

So, Aisha put \textdollar25.368 into each jar (to 3 decimal places).

(b) To the nearest whole number:

$
\textdollar25.368 \approx \textdollar25
$

So, there was about \textdollar25 in each jar if rounded to the nearest dollar.

(c) To the nearest 2 decimal places:

$
\textdollar25.368 \approx \textdollar25.37
$

So, Aisha would plan to spend \textdollar25.37 from each jar if she used only dollars and cents.

\section*{Question 33}
\textbf{Metadata}

\begin{itemize}
  \item Question ID: P5-DcDiv3dK\_P4-DcAdd2nd\_GPT4.1\_Household Finance\_01
  \item Primary KC: DECIMALS | Division | dividing decimals (up to 3 decimal places) by 10, 100, 1000 and their multiples
  \item Secondary KC: DECIMALS | Addition | adding decimals
  \item Topic: Household finance such as income, utility bills, money, interest, savings, instalment, mortgage, financial planning etc.
  \item Grade: Primary 5
\end{itemize}

\textbf{Solution}

First, divide the total amount of money in the jar by 10 to find the amount each grandchild (and Mrs Lim herself) receives:

Each share $= \frac{27.650}{10} = 2.765$

After distributing, Mrs Lim takes one share for herself: $2.765$.

She then adds the money found in the second jar:

$2.765 + 4.25 = 7.015$

Therefore, Mrs Lim has \textdollar7.015 after adding the extra \textdollar4.25 to her share.

\section*{Question 34}
\textbf{Metadata}

\begin{itemize}
  \item Question ID: P5-PcRepWh\_P1-WNMul2nd\_GPT4.1\_Household Finance\_01
  \item Primary KC: PERCENTAGE | Representation and concept | expressing a part of a whole as a percentage
  \item Secondary KC: WHOLE NUMBERS | Multiplication | multiplying whole numbers
  \item Topic: Household finance such as income, utility bills, money, interest, savings, instalment, mortgage, financial planning etc.
  \item Grade: Primary 5
\end{itemize}

\textbf{Solution}

First, we find $15\%$ of Marissa's savings.

$15\% = \frac{15}{100}$ 

Amount donated $= \frac{15}{100} \times 800 = 0.15 \times 800 = 120$

So, Marissa donated \textdollar120 to the charity.

Amount left $= 800 - 120 = 680$

Therefore, Marissa donated \textdollar120 to the charity and had \textdollar680 left after the donation.

\section*{Question 35}
\textbf{Metadata}

\begin{itemize}
  \item Question ID: P5-PcRepWh\_P1-WNDiv2nd\_GPT4.1\_Household Finance\_01
  \item Primary KC: PERCENTAGE | Representation and concept | expressing a part of a whole as a percentage
  \item Secondary KC: WHOLE NUMBERS | Division | dividing whole numbers
  \item Topic: Household finance such as income, utility bills, money, interest, savings, instalment, mortgage, financial planning etc.
  \item Grade: Primary 5
\end{itemize}

\textbf{Solution}

To find the percentage of Mrs Tan's salary spent on her electricity bill, we first need to divide the amount spent on the bill by the total salary and then express the answer as a percentage.

Let the electricity bill be $E = 240$, and the salary be $S = 1200$.

Step 1: Divide the electricity bill by the salary.
$
\frac{240}{1200} = 0.2
$

Step 2: Convert the decimal to a percentage by multiplying by $100$.
$
0.2 \times 100 = 20\%
$

\textbf{Final Answer:} Mrs Tan spent $20\%$ of her salary on her electricity bill.

\section*{Question 36}
\textbf{Metadata}

\begin{itemize}
  \item Question ID: P5-RtFndU\_P2-DcCnvN2D\_GPT4.1\_Household Finance\_01
  \item Primary KC: RATE | Finding number of unit | finding number of units given rate and total amount
  \item Secondary KC: DECIMALS | Conversion to larger units | converting a measurement from a smaller unit to a larger unit in decimal form
  \item Topic: Household finance such as income, utility bills, money, interest, savings, instalment, mortgage, financial planning etc.
  \item Grade: Primary 5
\end{itemize}

\textbf{Solution}

(a) To convert $82500$ $Wh$ to $kWh$:

$1\ kWh = 1000\ Wh$

So, $82500\ Wh = 82500 \div 1000 = 82.5\ kWh$

(b) To find the total amount paid:

Cost per $kWh = \textdollar0.22$

Total cost $= 82.5 \times \textdollar0.22 = \textdollar18.15$

Mrs Tan paid \textdollar18.15 for electricity last month.

\section*{Question 37}
\textbf{Metadata}

\begin{itemize}
  \item Question ID: P6-FrDivPN\_P2-FrAdd2nd\_GPT4.1\_Household Finance\_01
  \item Primary KC: FRACTIONS | Division | dividing a proper fraction by a whole number
  \item Secondary KC: FRACTIONS | Addition | adding fractions
  \item Topic: Household finance such as income, utility bills, money, interest, savings, instalment, mortgage, financial planning etc.
  \item Grade: Primary 6
\end{itemize}

\textbf{Solution}

(a) Amount spent on apples: $\frac{3}{4} \times \textdollar12 = \textdollar9$.

Amount spent on apples each week: $\frac{\textdollar9}{4} = \textdollar2.25$.

(b) Fraction spent on bananas: $\frac{1}{8}$ of $\textdollar12$.

Total fraction spent on apples and bananas:

$\frac{3}{4} + \frac{1}{8}$

First, make the denominators the same:

$\frac{3}{4} = \frac{6}{8}$

$\frac{6}{8} + \frac{1}{8} = \frac{7}{8}$

Therefore, Mrs Lee spent $\frac{7}{8}$ of her $\textdollar12$ on apples and bananas in total.

\section*{Question 38}
\textbf{Metadata}

\begin{itemize}
  \item Question ID: P6-FrDivPN\_P5-FrCnv2Dc\_GPT4.1\_Household Finance\_01
  \item Primary KC: FRACTIONS | Division | dividing a proper fraction by a whole number
  \item Secondary KC: FRACTIONS | Conversion to decimals | expressing fractions as decimals
  \item Topic: Household finance such as income, utility bills, money, interest, savings, instalment, mortgage, financial planning etc.
  \item Grade: Primary 6
\end{itemize}

\textbf{Solution}

(a) Mrs Lim baked $2\frac{1}{2}$ trays in total. 
Firstly, convert $2\frac{1}{2}$ to an improper fraction:

$2\frac{1}{2} = \frac{2 \times 2 + 1}{2} = \frac{5}{2}$

She divides $\frac{5}{2}$ trays equally among 5 boxes:

$\frac{5}{2} \div 5 = \frac{5}{2} \div \frac{5}{1} = \frac{5}{2} \times \frac{1}{5} = \frac{5 \times 1}{2 \times 5} = \frac{5}{10} = \frac{1}{2}$

Each donation box will receive $\frac{1}{2}$ tray of cookies.

(b) Express $\frac{1}{2}$ as a decimal:

$\frac{1}{2} = 0.5$

\textbf{Final answers:}

(a) $\frac{1}{2}$ tray

(b) $0.5$ trays

\section*{Question 39}
\textbf{Metadata}

\begin{itemize}
  \item Question ID: P6-FrDivPP\_P2-FrSub2nd\_GPT4.1\_Household Finance\_01
  \item Primary KC: FRACTIONS | Division | dividing a whole number/proper fraction by a proper fraction
  \item Secondary KC: FRACTIONS | Subtraction | subtracting fractions
  \item Topic: Household finance such as income, utility bills, money, interest, savings, instalment, mortgage, financial planning etc.
  \item Grade: Primary 6
\end{itemize}

\textbf{Solution}

First, let us find out how much 1 month's utility bill costs.\newline

$
\text{Cost of 1 month's bill} = \frac{3}{4} \times \textdollar120 = \textdollar90
$

Next, we want to know how many months Mrs Tan can pay with her savings. She divides her total savings by the cost of 1 month's bill:

$
\text{Number of months Mrs Tan can pay} = \frac{120}{90} = \frac{12}{9} = \frac{4}{3}\text{ months}
$

Now, after paying for 2 months, how much money will she have left?

$
\text{Amount paid for 2 months} = 2 \times \textdollar90 = \textdollar180
$

But Mrs Tan only has \textdollar120. So, paying for 2 months exceeds her savings. Let's check instead how much money she has left after paying for 1 month:

$
\text{Amount left after 1 month} = \textdollar120 - \textdollar90 = \textdollar30
$

Alternatively, if the intention is to compute the unused amount if she pays only for complete months (which is 1 month), then \textdollar30 is left. She cannot pay for 2 whole months as that would require \textdollar180, which is more than her savings.

\textbf{Answer:}

- Mrs Tan can pay for $\frac{4}{3}$ months (or 1 complete month with some savings left).
- After paying for 1 month, she has $\textdollar30$ left.

\section*{Question 40}
\textbf{Metadata}

\begin{itemize}
  \item Question ID: P6-FrDivPP\_P5-FrCnv2Dc\_GPT4.1\_Household Finance\_01
  \item Primary KC: FRACTIONS | Division | dividing a whole number/proper fraction by a proper fraction
  \item Secondary KC: FRACTIONS | Conversion to decimals | expressing fractions as decimals
  \item Topic: Household finance such as income, utility bills, money, interest, savings, instalment, mortgage, financial planning etc.
  \item Grade: Primary 6
\end{itemize}

\textbf{Solution}

(a) Mrs Lee was originally going to give each child: 

$\frac{\textdollar120}{3} = \textdollar40$

(b) Each child now receives:

$\text{Amount each child receives} = \textdollar40 \times \frac{3}{4} = \textdollar30$

Expressed as a decimal, $\textdollar30.00$

(c) Total given to 3 children now:

$\textdollar30 \times 3 = \textdollar90$

Mrs Lee saves:

$\textdollar120 - \textdollar90 = \textdollar30$

\section*{Question 41}
\textbf{Metadata}

\begin{itemize}
  \item Question ID: P6-PcFndWN\_P1-WNAdd2nd\_GPT4.1\_Household Finance\_01
  \item Primary KC: PERCENTAGE | Finding the whole | finding the whole given a part and the percentage
  \item Secondary KC: WHOLE NUMBERS | Addition | adding whole numbers
  \item Topic: Household finance such as income, utility bills, money, interest, savings, instalment, mortgage, financial planning etc.
  \item Grade: Primary 6
\end{itemize}

\textbf{Solution}

Let Alice's total monthly household expenses be $x$.

The electricity bill is $30\%$ of $x$, so:
$
30\% \times x = 210
$
$
\frac{30}{100} \times x = 210
$
$
0.3x = 210
$
$
x = \frac{210}{0.3} = 700
$

To check, sum up all known expenses:
$
\text{Total spent on groceries and transport} = 320 + 150 = 470
$
$
\text{Sum of all known parts: } 210 + 320 + 150 = 680
$

However, the rest of her expenses are not stated, and we already found the total monthly household expenses as $\textdollar700$.

\textbf{Final Answer:}

Alice's total monthly household expenses this month is $\boxed{\textdollar700}$.

\section*{Question 42}
\textbf{Metadata}

\begin{itemize}
  \item Question ID: P6-PcFndChg\_P1-WNDiv2nd\_GPT4.1\_Household Finance\_01
  \item Primary KC: PERCENTAGE | Finding change | finding percentage increase/decrease
  \item Secondary KC: WHOLE NUMBERS | Division | dividing whole numbers
  \item Topic: Household finance such as income, utility bills, money, interest, savings, instalment, mortgage, financial planning etc.
  \item Grade: Primary 6
\end{itemize}

\textbf{Solution}

(a) Amount decreased $= \textdollar120 - \textdollar90 = \textdollar30$

Percentage decrease $= \dfrac{\text{Decrease}}{\text{Original amount}} \times 100\% = \dfrac{30}{120} \times 100\% = 0.25 \times 100\% = 25\%$

(b) Amount each child receives $= \dfrac{\textdollar30}{3} = \textdollar10$

\section*{Question 43}
\textbf{Metadata}

\begin{itemize}
  \item Question ID: P6-RoFndRoWN\_P1-WNAdd2nd\_GPT4.1\_Household Finance\_01
  \item Primary KC: RATIO | Finding ratio | finding the ratio of two or three given whole numbers
  \item Secondary KC: WHOLE NUMBERS | Addition | adding whole numbers
  \item Topic: Household finance such as income, utility bills, money, interest, savings, instalment, mortgage, financial planning etc.
  \item Grade: Primary 6
\end{itemize}

\textbf{Solution}

(a) Ahmad contributes $60$, Bella contributes $90$, and Chloe contributes $120$.

Total contribution $= 60 + 90 + 120 = 270$

So, the total amount contributed is \textdollar270.

(b) The ratio of Ahmad's contribution to Bella's contribution to Chloe's contribution is:

$60 : 90 : 120$

To simplify, divide each amount by $30$:
$
\frac{60}{30} : \frac{90}{30} : \frac{120}{30} = 2 : 3 : 4
$

So, the simplest form of the ratio is $2 : 3 : 4$.

\section*{Question 44}
\textbf{Metadata}

\begin{itemize}
  \item Question ID: P6-AgRepLrEx\_P6-AgSmpLrEx\_GPT4.1\_Household Finance\_01
  \item Primary KC: ALGEBRA | Representation and concept | translation of real-world situations into linear algebraic expressions
  \item Secondary KC: ALGEBRA | Simplifying | simplifying linear expressions
  \item Topic: Household finance such as income, utility bills, money, interest, savings, instalment, mortgage, financial planning etc.
  \item Grade: Primary 6
\end{itemize}

\textbf{Solution}

(a) The fixed fee is $\textdollar45$. The extra fee for $h$ hours of extra data is $\textdollar2$ per hour, which amounts to $2h$ dollars. 

Total amount paid $= 45 + 2h$

(b) The expression $45 + 2h$ is already simplified as a linear expression. 

Final answer: $45 + 2h$

\section*{Question 45}
\textbf{Metadata}

\begin{itemize}
  \item Question ID: O1-PcFndRslt\_P1-WNSub2nd\_GPT4.1\_Household Finance\_01
  \item Primary KC: PERCENTAGE | Finding result after change | increasing/decreasing a quantity by a given percentage
  \item Secondary KC: WHOLE NUMBERS | Subtraction | subtracting whole numbers
  \item Topic: Household finance such as income, utility bills, money, interest, savings, instalment, mortgage, financial planning etc.
  \item Grade: Secondary O-level 1
\end{itemize}

\textbf{Solution}

First, we find $15\%$ of $120$. 

$15\% \times 120 = \frac{15}{100} \times 120 = 18$

So, the new bill is:

$120 - 18 = 102$

The amount saved each month is:

$120 - 102 = 18$

Therefore, the family saves $\textdollar18$ each month after the decrease.

\section*{Question 46}
\textbf{Metadata}

\begin{itemize}
  \item Question ID: O1-PcRepRvs\_O1-PcCnv2Dc\_GPT4.1\_Household Finance\_02
  \item Primary KC: PERCENTAGE | Representation and concept | reverse percentages
  \item Secondary KC: PERCENTAGE | Conversion to decimals | expressing percentage as a decimal
  \item Topic: Household finance such as income, utility bills, money, interest, savings, instalment, mortgage, financial planning etc.
  \item Grade: Secondary O-level 1
\end{itemize}

\textbf{Solution}

Let the original bill amount be $x$.

First, express 20\% as a decimal:

$20\% = \frac{20}{100} = 0.20$

This means Mrs Tan paid 80\% of the original bill:

$100\% - 20\% = 80\%$

Express 80\% as a decimal:

$80\% = \frac{80}{100} = 0.80$

So,

$0.80x = 80$

$x = \frac{80}{0.80} = 100$

Therefore, the original bill amount was \textdollar100.

\section*{Question 47}
\textbf{Metadata}

\begin{itemize}
  \item Question ID: O2-RoRepDP\_P1-WNMul2nd\_GPT4.1\_Household Finance\_01
  \item Primary KC: RATIO | Representation and concept | direct proportion
  \item Secondary KC: WHOLE NUMBERS | Multiplication | multiplying whole numbers
  \item Topic: Household finance such as income, utility bills, money, interest, savings, instalment, mortgage, financial planning etc.
  \item Grade: Secondary O-level 2
\end{itemize}

\textbf{Solution}

Let $x$ be the number of gas cylinders needed for 540 meals. Since the number of gas cylinders and the number of meals are in direct proportion, we have:

$
\frac{3}{180} = \frac{x}{540}
$

Cross-multiplying,

$
3 \times 540 = 180 \times x \\
1620 = 180x \\
x = \frac{1620}{180} = 9
$

So, Mrs. Tan will need $9$ gas cylinders.

Each cylinder costs $\textdollar28$, so the total cost is:

$
9 \times 28 = 252
$

Thus, Mrs. Tan will spend $\textdollar252$ to cook 540 meals.

\section*{Question 48}
\textbf{Metadata}

\begin{itemize}
  \item Question ID: O3-MXMul\_O3-MXAdd\_GPT4.1\_Household Finance\_01
  \item Primary KC: MATRICES | Multiplication | multiplication of matrices
  \item Secondary KC: MATRICES | Addition | addition of matrices
  \item Topic: Household finance such as income, utility bills, money, interest, savings, instalment, mortgage, financial planning etc.
  \item Grade: Secondary O-level 3/4
\end{itemize}

\textbf{Solution}

(a) To find the new March expenses, we multiply matrices $A$ and $B$ element-wise:
$
A \circ B = \begin{pmatrix}
120 \times 1.1 & 130 \times 1.1 \\
50 \times 1.1 & 55 \times 1.1 \\
40 \times 1.1 & 42 \times 1.1
\end{pmatrix}
= \begin{pmatrix}
132 & 143 \\
55 & 60.5 \\
44 & 46.2
\end{pmatrix}
$
(This gives the March expenses for each utility.)

(b) The family receives a refund matrix $C$ for March. To find the final expenses for March including refunds (only apply to the first column, as refunds are for March):
$
\text{Final March Expenses} = (A \circ B) - C = \begin{pmatrix}
132 - 10 & 143 - 0 \\
55 - 5 & 60.5 - 0 \\
44 - 3 & 46.2 - 0
\end{pmatrix} = \begin{pmatrix}
122 & 143 \\
50 & 60.5 \\
41 & 46.2
\end{pmatrix}
$

So, the family's final utility expenses for March are:
\begin{itemize}
  \item Electricity: \textdollar122, Water: \textdollar50, Gas: \textdollar41 (first column)
  \item (Second column represents recalculated February, so you answer March with the first column entries.)
\end{itemize}
Thus, after the price increase and refund, their March expenses are: \textdollar122 for electricity, \textdollar50 for water, and \textdollar41 for gas.

\section*{Question 49}
\textbf{Metadata}

\begin{itemize}
  \item Question ID: O3-SPFndstd\_O2-SPFndmean\_GPT4.1\_Household Finance\_01
  \item Primary KC: STATISTICS AND PROBABILITY | Finding standard deviation | calculation of the standard deviation for a set of data
  \item Secondary KC: STATISTICS AND PROBABILITY | Finding mean deviation | calculation of the mean for a set of data
  \item Topic: Household finance such as income, utility bills, money, interest, savings, instalment, mortgage, financial planning etc.
  \item Grade: Secondary O-level 3/4
\end{itemize}

\textbf{Solution}

(a) To find the mean: 
$
\text{Mean} = \frac{120 + 135 + 110 + 140 + 130 + 125}{6} = \frac{760}{6} \approx 126.67
$

(b) To find the standard deviation, first find each value's deviation from the mean, square these deviations, find their mean, then take the square root.

First, calculate each deviation from the mean:
$
\begin{align*}
120 - 126.67 & = -6.67 \\
135 - 126.67 & = 8.33 \\
110 - 126.67 & = -16.67 \\
140 - 126.67 & = 13.33 \\
130 - 126.67 & = 3.33 \\
125 - 126.67 & = -1.67 
\end{align*}
$

Now, square these deviations:
$
\begin{align*}
(-6.67)^2 & = 44.49 \\
(8.33)^2 & = 69.39 \\
(-16.67)^2 & = 278.89 \\
(13.33)^2 & = 177.69 \\
(3.33)^2 & = 11.09 \\
(-1.67)^2 & = 2.79 
\end{align*}
$

Sum these squared deviations:
$
44.49 + 69.39 + 278.89 + 177.69 + 11.09 + 2.79 = 584.34
$

The standard deviation is:
$
\sigma = \sqrt{\frac{584.34}{6}} = \sqrt{97.39} \approx 9.87
$

\textbf{Therefore,\ the mean monthly electricity bill is \textdollar126.67 (rounded to 2 decimal places) and the standard deviation is approximately \textdollar9.87.}

\end{document}
