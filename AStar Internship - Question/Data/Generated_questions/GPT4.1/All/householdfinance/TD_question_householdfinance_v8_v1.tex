\documentclass{article}
\usepackage[utf8]{inputenc}
\usepackage{amsmath}
\usepackage{amsfonts}
\usepackage{amssymb}
\usepackage{graphicx}
\usepackage{hyperref}
\title{'TD Questions household finance v8 v1'}
\author{Tien Dung Doan}
\begin{document}
\maketitle
\section*{Question 1}
\textbf{Metadata}

\begin{itemize}
  \item Question ID: P3-WNAdd4d\_P1-WNCmp\_GPT4.1\_Household Finance\_03
  \item Primary KC: WHOLE NUMBERS | Addition | adding whole numbers up to 4 digits
  \item Secondary KC: WHOLE NUMBERS | Comparison and ordering | comparing and ordering whole numbers
  \item Topic: Household finance such as income, utility bills, money, interest, savings, instalment, mortgage, financial planning etc.
  \item Grade: Primary 3
\end{itemize}

\textbf{Question}

Mr Lim received his monthly electricity bill, which is \textdollar1589. In the same month, his water bill is \textdollar623 and his gas bill is \textdollar812. What is the total amount Mr Lim needs to pay for his utilities this month? Compare the total utilities bill to \textdollar3000. Is the total bill more than or less than \textdollar3000?

\section*{Question 2}
\textbf{Metadata}

\begin{itemize}
  \item Question ID: P3-WNSub4d\_P1-WNAdd2nd\_GPT4.1\_Household Finance\_02
  \item Primary KC: WHOLE NUMBERS | Subtraction | subtracting whole numbers up to 4 digits
  \item Secondary KC: WHOLE NUMBERS | Addition | adding whole numbers
  \item Topic: Household finance such as income, utility bills, money, interest, savings, instalment, mortgage, financial planning etc.
  \item Grade: Primary 3
\end{itemize}

\textbf{Question}

Sarah's mother gave her \textdollar450 to buy groceries for the week. After buying fruits and vegetables, Sarah had \textdollar215 left. Later, her father gave her an additional \textdollar80 for any extra items she might need. How much money did Sarah spend on groceries before her father gave her extra money?

\section*{Question 3}
\textbf{Metadata}

\begin{itemize}
  \item Question ID: P3-WNDivRmd3d\_P1-WNSub2nd\_GPT4.1\_Household Finance\_02
  \item Primary KC: WHOLE NUMBERS | Division | dividing whole numbers up to 3 digits by 1 digit with remainder 
  \item Secondary KC: WHOLE NUMBERS | Subtraction | subtracting whole numbers
  \item Topic: Household finance such as income, utility bills, money, interest, savings, instalment, mortgage, financial planning etc.
  \item Grade: Primary 3
\end{itemize}

\textbf{Question}

Mrs Tan bought a packet of 248 candies to share equally among her 5 children. Each child will get the same number of candies. After giving the candies to her children, Mrs Tan keeps the leftover candies for herself.

(a) How many candies does each child receive?

(b) How many candies does Mrs Tan have left for herself if she decided to eat 2 fewer candies than the number of leftover candies?

\section*{Question 4}
\textbf{Metadata}

\begin{itemize}
  \item Question ID: P3-WNMul3d1d\_P1-WNCmp\_GPT4.1\_Household Finance\_02
  \item Primary KC: WHOLE NUMBERS | Multiplication | multiplying whole numbers up to 3 digits by 1 digit
  \item Secondary KC: WHOLE NUMBERS | Comparison and ordering | comparing and ordering whole numbers
  \item Topic: Household finance such as income, utility bills, money, interest, savings, instalment, mortgage, financial planning etc.
  \item Grade: Primary 3
\end{itemize}

\textbf{Question}

Mrs. Lim gives her son, Ben, a weekly allowance of \textdollar7. Ben decides to save his allowance for 8 weeks to buy a new toy. His sister, Jane, saves \textdollar9 each week for 6 weeks. 

(a) How much money does Ben have after 8 weeks?

(b) How much money does Jane have after 6 weeks?

(c) Who has more money after saving, and how much more do they have than the other?

\section*{Question 5}
\textbf{Metadata}

\begin{itemize}
  \item Question ID: P3-WNDiv3d1d\_P1-WNCmp\_GPT4.1\_Household Finance\_02
  \item Primary KC: WHOLE NUMBERS | Division | dividing whole numbers up to 3 digits by 1 digit
  \item Secondary KC: WHOLE NUMBERS | Comparison and ordering | comparing and ordering whole numbers
  \item Topic: Household finance such as income, utility bills, money, interest, savings, instalment, mortgage, financial planning etc.
  \item Grade: Primary 3
\end{itemize}

\textbf{Question}

Mdm Tan wants to give her three children an equal amount of pocket money from her savings of $276. 

(a) How much pocket money does each child receive?

After giving the money to her children, Mdm Tan compares how much money is left in her savings with her neighbour, Mrs Lim, who has $92 left in her savings. 

(b) Who has more money in their savings now, and by how much?

\section*{Question 6}
\textbf{Metadata}

\begin{itemize}
  \item Question ID: P3-FrSubRl12\_P2-FrCmp\_GPT4.1\_Household Finance\_02
  \item Primary KC: FRACTIONS | Subtraction | subtracting two related fractions within one whole with denominators of given fractions not exceeding 12
  \item Secondary KC: FRACTIONS | Comparison and ordering | comparing and ordering fractions
  \item Topic: Household finance such as income, utility bills, money, interest, savings, instalment, mortgage, financial planning etc.
  \item Grade: Primary 3
\end{itemize}

\textbf{Question}

Siti spent $\frac{5}{12}$ of her allowance on groceries and $\frac{1}{4}$ of her allowance on electricity bills this month. 

(a) How much more of her allowance did Siti spend on groceries than on electricity bills?

(b) Write the fractions representing groceries and electricity bills in order from least to greatest.

\section*{Question 7}
\textbf{Metadata}

\begin{itemize}
  \item Question ID: P4-WNMul4d1d\_P1-WNCmp\_GPT4.1\_Household Finance\_02
  \item Primary KC: WHOLE NUMBERS | Multiplication | multiplying whole numbers up to 4 digits by 1 digit or up to 3 digits by 2 digits
  \item Secondary KC: WHOLE NUMBERS | Comparison and ordering | comparing and ordering whole numbers
  \item Topic: Household finance such as income, utility bills, money, interest, savings, instalment, mortgage, financial planning etc.
  \item Grade: Primary 4
\end{itemize}

\textbf{Question}

A family buys 6 boxes of light bulbs for their new house. Each box contains 125 light bulbs. Another family buys 9 boxes of light bulbs. Each box contains 90 light bulbs.\

Which family has more light bulbs in total? How many more light bulbs do they have than the other family?

\section*{Question 8}
\textbf{Metadata}

\begin{itemize}
  \item Question ID: P4-WNMul4d1d\_P4-WNRnd5d\_GPT4.1\_Household Finance\_02
  \item Primary KC: WHOLE NUMBERS | Multiplication | multiplying whole numbers up to 4 digits by 1 digit or up to 3 digits by 2 digits
  \item Secondary KC: WHOLE NUMBERS | Rounding | rounding whole numbers up to 100000 to the nearest 10, 100 or 1000 
  \item Topic: Household finance such as income, utility bills, money, interest, savings, instalment, mortgage, financial planning etc.
  \item Grade: Primary 4
\end{itemize}

\textbf{Question}

A family spends \textdollar57 every month on electricity bills. 

(a) How much do they spend in total on electricity in 1 year?

(b) Round your answer in (a) to the nearest hundred dollars.

\section*{Question 9}
\textbf{Metadata}

\begin{itemize}
  \item Question ID: P4-WNDiv4d1d\_P1-WNSub2nd\_GPT4.1\_Household Finance\_02
  \item Primary KC: WHOLE NUMBERS | Division | dividing whole numbers up to 4 digits by 1 digit
  \item Secondary KC: WHOLE NUMBERS | Subtraction | subtracting whole numbers
  \item Topic: Household finance such as income, utility bills, money, interest, savings, instalment, mortgage, financial planning etc.
  \item Grade: Primary 4
\end{itemize}

\textbf{Question}

Mr. Lim received \textdollar3872 from his work as salary this month. He decided to divide this money equally among his 4 children to put into their savings accounts. After giving each child their share, Mr. Lim had \textdollar320 left for household expenses. 

How much did each child receive?

\section*{Question 10}
\textbf{Metadata}

\begin{itemize}
  \item Question ID: P4-WNDiv4d1d\_P4-WNRnd5d\_GPT4.1\_Household Finance\_02
  \item Primary KC: WHOLE NUMBERS | Division | dividing whole numbers up to 4 digits by 1 digit
  \item Secondary KC: WHOLE NUMBERS | Rounding | rounding whole numbers up to 100000 to the nearest 10, 100 or 1000 
  \item Topic: Household finance such as income, utility bills, money, interest, savings, instalment, mortgage, financial planning etc.
  \item Grade: Primary 4
\end{itemize}

\textbf{Question}

A family plans to divide their electricity bill equally among 4 months. The total bill for 4 months is \textdollar3287. Before starting the payments, the family decides to round the total bill to the nearest 10 dollars. 

(a) What is the rounded amount the family will be paying in total for the 4 months?

(b) How much does the family need to pay each month after they have rounded the amount?

\section*{Question 11}
\textbf{Metadata}

\begin{itemize}
  \item Question ID: P4-FrAddU12\_P2-FrCmp\_GPT4.1\_Household Finance\_02
  \item Primary KC: FRACTIONS | Addition | adding unlike fractions with two different denominators not exceeding 12
  \item Secondary KC: FRACTIONS | Comparison and ordering | comparing and ordering fractions
  \item Topic: Household finance such as income, utility bills, money, interest, savings, instalment, mortgage, financial planning etc.
  \item Grade: Primary 4
\end{itemize}

\textbf{Question}

Grace is saving her weekly allowance to buy a new school bag. In the first week, she saved $\frac{3}{8}$ of her allowance. In the second week, she saved $\frac{2}{3}$ of her allowance. 

(a) What fraction of her allowance did Grace save in total over the two weeks? 

(b) Did Grace save more than $1$ whole allowance over the two weeks? Explain your answer.

\section*{Question 12}
\textbf{Metadata}

\begin{itemize}
  \item Question ID: P4-FrSubU12\_P3-FrSmp\_GPT4.1\_Household Finance\_02
  \item Primary KC: FRACTIONS | Subtraction | subtracting unlike fractions with two different denominators not exceeding 12
  \item Secondary KC: FRACTIONS | Simplifying | expressing a fraction in its simplest form
  \item Topic: Household finance such as income, utility bills, money, interest, savings, instalment, mortgage, financial planning etc.
  \item Grade: Primary 4
\end{itemize}

\textbf{Question}

Amy saved $\textdollar7$ \( \frac{5}{12} \) from her weekly allowance and spent $\textdollar1$ \( \frac{1}{3} \) on buying a new storybook. How much money did Amy have left, in simplest fractional form?

\section*{Question 13}
\textbf{Metadata}

\begin{itemize}
  \item Question ID: P4-DcAdd2d\_P4-DcCmp3d\_GPT4.1\_Household Finance\_02
  \item Primary KC: DECIMALS | Addition | adding decimals (up to 2 decimal places)
  \item Secondary KC: DECIMALS | Comparison and ordering | comparing and ordering decimals up to 3 decimal places
  \item Topic: Household finance such as income, utility bills, money, interest, savings, instalment, mortgage, financial planning etc.
  \item Grade: Primary 4
\end{itemize}

\textbf{Question}

Sarah pays for three household utility bills each month: electricity, water, and gas. In June, the electricity bill was $\textdollar 45.60$, the water bill was $\textdollar 23.75$, and the gas bill was $\textdollar 12.85$. 

(a) What is the total amount Sarah spent on all three bills in June?

(b) The next month, her electricity bill increased to $\textdollar 45.605$. Compare her electricity bills for June and July and state which month had a higher electricity bill.

\section*{Question 14}
\textbf{Metadata}

\begin{itemize}
  \item Question ID: P4-DcSub2d\_P4-DcCnv2Fr\_GPT4.1\_Household Finance\_02
  \item Primary KC: DECIMALS | Subtraction | subtracting decimals (up to 2 decimal places)
  \item Secondary KC: DECIMALS | Conversion from decimals to fraction | expressing decimals as fractions
  \item Topic: Household finance such as income, utility bills, money, interest, savings, instalment, mortgage, financial planning etc.
  \item Grade: Primary 4
\end{itemize}

\textbf{Question}

Rachel had \textdollar12.60 in her savings jar. She spent \textdollar4.75 buying stationery. 

(a) How much money did Rachel have left? 

(b) Express the amount of money Rachel had left as a fraction in simplest form.

\section*{Question 15}
\textbf{Metadata}

\begin{itemize}
  \item Question ID: P4-DcSub2d\_P4-DcRnd3d\_GPT4.1\_Household Finance\_02
  \item Primary KC: DECIMALS | Subtraction | subtracting decimals (up to 2 decimal places)
  \item Secondary KC: DECIMALS | Rounding | rounding decimals up to 3 decimal places to the nearest whole number, 1 decimal place and 2 decimal places 
  \item Topic: Household finance such as income, utility bills, money, interest, savings, instalment, mortgage, financial planning etc.
  \item Grade: Primary 4
\end{itemize}

\textbf{Question}

Mrs Tan checked her electricity bill for two months. In April, she was charged \textdollar123.45. In May, her bill was \textdollar97.88. 

(a) How much less did Mrs Tan pay in May compared to April? 

(b) Round your answer from part (a) to the nearest dollar and to 1 decimal place.

\section*{Question 16}
\textbf{Metadata}

\begin{itemize}
  \item Question ID: P4-DcMul2d1d\_P4-DcCnv2Fr\_GPT4.1\_Household Finance\_01
  \item Primary KC: DECIMALS | Multiplication | multiplying decimals (up to 2 decimal places) by a 1-digit whole number
  \item Secondary KC: DECIMALS | Conversion from decimals to fraction | expressing decimals as fractions
  \item Topic: Household finance such as income, utility bills, money, interest, savings, instalment, mortgage, financial planning etc.
  \item Grade: Primary 4
\end{itemize}

\textbf{Question}

Mrs Lim pays $2.75$ dollars per day for her lunch at work. 

(a) How much does Mrs Lim spend on lunch in $5$ days? 

(b) Express the amount Mrs Lim spends in one day as a fraction in simplest form.

\section*{Question 17}
\textbf{Metadata}

\begin{itemize}
  \item Question ID: P4-DcMul2d1d\_P4-DcAdd2nd\_GPT4.1\_Household Finance\_01
  \item Primary KC: DECIMALS | Multiplication | multiplying decimals (up to 2 decimal places) by a 1-digit whole number
  \item Secondary KC: DECIMALS | Addition | adding decimals
  \item Topic: Household finance such as income, utility bills, money, interest, savings, instalment, mortgage, financial planning etc.
  \item Grade: Primary 4
\end{itemize}

\textbf{Question}

Mika helps her mum calculate the total cost of electricity for the month. Each week, their family spends \textdollar3.75 on electricity. There are 4 weeks in a month. At the end of the month, Mika realises they also spent \textdollar5.20 on extra lighting. What is their total spending on electricity and lighting for the month?

\section*{Question 18}
\textbf{Metadata}

\begin{itemize}
  \item Question ID: P4-DcDiv2d1d\_P4-DcCmp3d\_GPT4.1\_Household Finance\_01
  \item Primary KC: DECIMALS | Division | dividing decimals (up to 2 decimal places) by a 1-digit whole number
  \item Secondary KC: DECIMALS | Comparison and ordering | comparing and ordering decimals up to 3 decimal places
  \item Topic: Household finance such as income, utility bills, money, interest, savings, instalment, mortgage, financial planning etc.
  \item Grade: Primary 4
\end{itemize}

\textbf{Question}

Mrs Tan received her monthly water bill of \textdollar23.94. She wants to divide the total bill equally among her 3 children since they use about the same amount of water.

(a) How much should each child pay? Round your answer to 2 decimal places.

Later, Mrs Tan compared the amount each child pays to the bills she had for the last three months: \textdollar7.980, \textdollar8.110, and \textdollar7.950. 

(b) Arrange these four amounts (including this month’s amount each child pays) in order from least to greatest.

\section*{Question 19}
\textbf{Metadata}

\begin{itemize}
  \item Question ID: P4-DcDiv2d1d\_P4-DcAdd2nd\_GPT4.1\_Household Finance\_01
  \item Primary KC: DECIMALS | Division | dividing decimals (up to 2 decimal places) by a 1-digit whole number
  \item Secondary KC: DECIMALS | Addition | adding decimals
  \item Topic: Household finance such as income, utility bills, money, interest, savings, instalment, mortgage, financial planning etc.
  \item Grade: Primary 4
\end{itemize}

\textbf{Question}

Maya receives her monthly pocket money from her parents. This month, she got \textdollar12.80. She decided to divide this amount equally among her 4 piggy banks to save money for different purposes. Later, her grandmother gave her another \textdollar3.20, which she added equally to the same piggy banks. How much money does each piggy bank have now?

\section*{Question 20}
\textbf{Metadata}

\begin{itemize}
  \item Question ID: P5-FrAddMix\_P2-FrCmp\_GPT4.1\_Household Finance\_01
  \item Primary KC: FRACTIONS | Addition | adding mixed numbers
  \item Secondary KC: FRACTIONS | Comparison and ordering | comparing and ordering fractions
  \item Topic: Household finance such as income, utility bills, money, interest, savings, instalment, mortgage, financial planning etc.
  \item Grade: Primary 5
\end{itemize}

\textbf{Question}

Mrs Tan is reviewing her household expenses for the month. She spent $1\dfrac{1}{4}$ of her budget on groceries and $2\dfrac{2}{3}$ of her budget on utility bills. 

(a) How much of her budget did she spend in total on groceries and utility bills?

(b) Did Mrs Tan spend more on groceries or on utility bills? Arrange the expenses in ascending order.

\section*{Question 21}
\textbf{Metadata}

\begin{itemize}
  \item Question ID: P5-FrSubMix\_P2-FrCmp\_GPT4.1\_Household Finance\_01
  \item Primary KC: FRACTIONS | Subtraction | subtracting mixed numbers
  \item Secondary KC: FRACTIONS | Comparison and ordering | comparing and ordering fractions
  \item Topic: Household finance such as income, utility bills, money, interest, savings, instalment, mortgage, financial planning etc.
  \item Grade: Primary 5
\end{itemize}

\textbf{Question}

Mrs Tan has $6\dfrac{3}{4}$ in her savings jar. She wants to pay her electricity bill, which is $4\dfrac{5}{8}$. After paying the bill, Mrs Tan compares the amount left in her savings jar to her daughter Mei's savings, which is \textdollar2\dfrac{1}{2}$.\
\
(a) How much money does Mrs Tan have left in her savings jar after paying the bill?\
\
(b) Who has more savings left, Mrs Tan or Mei, and by how much?

\section*{Question 22}
\textbf{Metadata}

\begin{itemize}
  \item Question ID: P5-FrMulImN\_P2-FrCmp\_GPT4.1\_Household Finance\_01
  \item Primary KC: FRACTIONS | Multiplication | multiplying a proper/improper fraction and a whole number
  \item Secondary KC: FRACTIONS | Comparison and ordering | comparing and ordering fractions
  \item Topic: Household finance such as income, utility bills, money, interest, savings, instalment, mortgage, financial planning etc.
  \item Grade: Primary 5
\end{itemize}

\textbf{Question}

Mrs Lim saves $\frac{3}{4}$ of her monthly allowance. Her monthly allowance is \textdollar120. Compare the amount Mrs Lim saves in a month to \textdollar90. Which amount is greater?

\section*{Question 23}
\textbf{Metadata}

\begin{itemize}
  \item Question ID: P5-FrMulPIm\_P2-FrCmp\_GPT4.1\_Household Finance\_01
  \item Primary KC: FRACTIONS | Multiplication | multiplying a proper fraction and a proper/improper fractions
  \item Secondary KC: FRACTIONS | Comparison and ordering | comparing and ordering fractions
  \item Topic: Household finance such as income, utility bills, money, interest, savings, instalment, mortgage, financial planning etc.
  \item Grade: Primary 5
\end{itemize}

\textbf{Question}

Mr Tan earns \textdollar2400 per month. He spends $\frac{2}{5}$ of his income on rent and $\frac{3}{8}$ of what he spends on rent on utilities.\
\
(a) How much does Mr Tan spend on rent each month?\
\
(b) How much does he spend on utilities each month?\
\
(c) Arrange the following amounts in order from the least to the greatest: Mr Tan's spending on utilities, Mr Tan's spending on rent, Mr Tan's remaining income after paying for rent and utilities.

\section*{Question 24}
\textbf{Metadata}

\begin{itemize}
  \item Question ID: P5-FrMulPIm\_P2-FrAdd2nd\_GPT4.1\_Household Finance\_01
  \item Primary KC: FRACTIONS | Multiplication | multiplying a proper fraction and a proper/improper fractions
  \item Secondary KC: FRACTIONS | Addition | adding fractions
  \item Topic: Household finance such as income, utility bills, money, interest, savings, instalment, mortgage, financial planning etc.
  \item Grade: Primary 5
\end{itemize}

\textbf{Question}

Mrs Tan sets aside $\textdollar600$ of her monthly salary for household expenses. She spends $\frac{2}{3}$ of this amount on groceries and then uses $\frac{3}{4}$ of the remaining money to pay for utilities. After paying for groceries and utilities, she decides to add $\frac{1}{6}$ of what is left to her savings. How much money does Mrs Tan put into her savings?

\section*{Question 25}
\textbf{Metadata}

\begin{itemize}
  \item Question ID: P5-FrMulPIm\_P3-FrSmp\_GPT4.1\_Household Finance\_01
  \item Primary KC: FRACTIONS | Multiplication | multiplying a proper fraction and a proper/improper fractions
  \item Secondary KC: FRACTIONS | Simplifying | expressing a fraction in its simplest form
  \item Topic: Household finance such as income, utility bills, money, interest, savings, instalment, mortgage, financial planning etc.
  \item Grade: Primary 5
\end{itemize}

\textbf{Question}

Mrs Lim saved \textdollar500 this month. She decided to use $\frac{2}{5}$ of her savings to pay her electricity and water bills. The electricity bill is $\frac{3}{4}$ of the amount she spent on both bills. How much did Mrs Lim pay for her electricity bill? Give your answer in the simplest form.

\section*{Question 26}
\textbf{Metadata}

\begin{itemize}
  \item Question ID: P5-FrMulImIm\_P2-FrAdd2nd\_GPT4.1\_Household Finance\_01
  \item Primary KC: FRACTIONS | Multiplication | multiplying two improper fractions
  \item Secondary KC: FRACTIONS | Addition | adding fractions
  \item Topic: Household finance such as income, utility bills, money, interest, savings, instalment, mortgage, financial planning etc.
  \item Grade: Primary 5
\end{itemize}

\textbf{Question}

Mr Lim decided to buy 3 identical cartons of fruit juice for his family's gathering. Each carton contains $\frac{9}{4}$ litres of fruit juice. After the gathering, $\frac{5}{3}$ litres from each carton were left over.\
\
(a) What is the total amount of fruit juice in all 3 cartons before the gathering?\
\
(b) What is the total amount of fruit juice left after the gathering?\
\
(c) How much fruit juice was consumed during the gathering in total?

\section*{Question 27}
\textbf{Metadata}

\begin{itemize}
  \item Question ID: P5-FrMulImIm\_P2-FrSub2nd\_GPT4.1\_Household Finance\_01
  \item Primary KC: FRACTIONS | Multiplication | multiplying two improper fractions
  \item Secondary KC: FRACTIONS | Subtraction | subtracting fractions
  \item Topic: Household finance such as income, utility bills, money, interest, savings, instalment, mortgage, financial planning etc.
  \item Grade: Primary 5
\end{itemize}

\textbf{Question}

Mrs Tan divides her monthly savings into two parts. \(\frac{7}{4}\) of her savings is used to pay for utilities, and \(\frac{5}{3}\) of her savings is used to pay for groceries. At the end of the month, Mrs Tan subtracts the amount spent on groceries from the amount spent on utilities. If her total monthly savings is \textdollar240, how much more did she spend on utilities than on groceries?

\section*{Question 28}
\textbf{Metadata}

\begin{itemize}
  \item Question ID: P5-FrMulImIm\_P3-FrSmp\_GPT4.1\_Household Finance\_01
  \item Primary KC: FRACTIONS | Multiplication | multiplying two improper fractions
  \item Secondary KC: FRACTIONS | Simplifying | expressing a fraction in its simplest form
  \item Topic: Household finance such as income, utility bills, money, interest, savings, instalment, mortgage, financial planning etc.
  \item Grade: Primary 5
\end{itemize}

\textbf{Question}

Mrs Tan wanted to repaint her living room and set aside $\textdollar150$ for the paint. She spent $\frac{7}{4}$ times as much on paint as she did on brushes. If $\frac{6}{5}$ of the money she spent on brushes was actually used, what fraction of her total budget was used for brushes, in its simplest form?

\section*{Question 29}
\textbf{Metadata}

\begin{itemize}
  \item Question ID: P5-FrMulMixN\_P2-FrSub2nd\_GPT4.1\_Household Finance\_01
  \item Primary KC: FRACTIONS | Multiplication | multiplying a mixed number and a whole number
  \item Secondary KC: FRACTIONS | Subtraction | subtracting fractions
  \item Topic: Household finance such as income, utility bills, money, interest, savings, instalment, mortgage, financial planning etc.
  \item Grade: Primary 5
\end{itemize}

\textbf{Question}

Mrs Tan bakes cakes at home to sell. In one week, she baked $3\dfrac{1}{2}$ batches of brownies a day for 6 days. Each batch uses $\frac{3}{4}$ kilogram of chocolate. 

(a) How many batches of brownies did Mrs Tan bake in 6 days?

(b) What is the total amount of chocolate she used in 6 days?

(c) If Mrs Tan started the week with 20 kilograms of chocolate, how much chocolate does she have left at the end of the week?

\section*{Question 30}
\textbf{Metadata}

\begin{itemize}
  \item Question ID: P5-DcMul3dK\_P4-DcCnv2Fr\_GPT4.1\_Household Finance\_01
  \item Primary KC: DECIMALS | Multiplication | multiplying decimals (up to 3 decimal places) by 10, 100, 1000 and their multiples
  \item Secondary KC: DECIMALS | Conversion from decimals to fraction | expressing decimals as fractions
  \item Topic: Household finance such as income, utility bills, money, interest, savings, instalment, mortgage, financial planning etc.
  \item Grade: Primary 5
\end{itemize}

\textbf{Question}

Mrs Tan tracked her daily household electricity usage and found that, on average, her family uses $3.75$ kWh (kilowatt-hours) of electricity each day. The electricity company charges \textdollar0.174 for every 1 kWh used.

(a) Calculate the total cost of electricity Mrs Tan's family would have to pay for $100$ days, based on their daily usage.

(b) Express the average daily usage of $3.75$ kWh as a fraction in its simplest form.

\section*{Question 31}
\textbf{Metadata}

\begin{itemize}
  \item Question ID: P5-DcMul3dK\_P4-DcSub2nd\_GPT4.1\_Household Finance\_01
  \item Primary KC: DECIMALS | Multiplication | multiplying decimals (up to 3 decimal places) by 10, 100, 1000 and their multiples
  \item Secondary KC: DECIMALS | Subtraction | subtracting decimals
  \item Topic: Household finance such as income, utility bills, money, interest, savings, instalment, mortgage, financial planning etc.
  \item Grade: Primary 5
\end{itemize}

\textbf{Question}

Mr Lim received his monthly electricity bill showing a total energy consumption of $73.245$ kWh. Each kWh of electricity costs \textdollar0.15. Mr Lim estimated his total electricity cost by multiplying the total consumption by the cost per kWh. After checking for errors, he noticed that he had been charged an additional service fee of \textdollar1.80, which he needs to subtract from his cost calculation. What is the actual amount Mr Lim needs to pay, including the service fee?

\section*{Question 32}
\textbf{Metadata}

\begin{itemize}
  \item Question ID: P5-DcDiv3dK\_P4-DcRnd3d\_GPT4.1\_Household Finance\_01
  \item Primary KC: DECIMALS | Division | dividing decimals (up to 3 decimal places) by 10, 100, 1000 and their multiples
  \item Secondary KC: DECIMALS | Rounding | rounding decimals up to 3 decimal places to the nearest whole number, 1 decimal place and 2 decimal places 
  \item Topic: Household finance such as income, utility bills, money, interest, savings, instalment, mortgage, financial planning etc.
  \item Grade: Primary 5
\end{itemize}

\textbf{Question}

Aisha saved \textdollar253.680 from her pocket money over several months. She decided to divide her savings equally into 10 jars to help her manage her spending on different expenses each month. 

(a) How much money, in dollars, did she put into each jar? Give your answer to 3 decimal places.

(b) Round your answer from (a) to the nearest whole number, to find out about how much money there was in each jar if she only counted whole dollars.

(c) Round your answer from (a) to 2 decimal places, to find out how much money she would plan to spend from each jar if she only used dollars and cents.

\section*{Question 33}
\textbf{Metadata}

\begin{itemize}
  \item Question ID: P5-DcDiv3dK\_P4-DcAdd2nd\_GPT4.1\_Household Finance\_01
  \item Primary KC: DECIMALS | Division | dividing decimals (up to 3 decimal places) by 10, 100, 1000 and their multiples
  \item Secondary KC: DECIMALS | Addition | adding decimals
  \item Topic: Household finance such as income, utility bills, money, interest, savings, instalment, mortgage, financial planning etc.
  \item Grade: Primary 5
\end{itemize}

\textbf{Question}

Mrs Lim has \textdollar27.650 in a savings jar. She decides to split this amount equally among her 10 grandchildren. After giving each grandchild their share, she finds another jar with \textdollar4.25. She adds this amount to her own share from the original savings jar.

How much money does Mrs Lim have after she adds the extra \textdollar4.25 to her share?

\section*{Question 34}
\textbf{Metadata}

\begin{itemize}
  \item Question ID: P5-PcRepWh\_P1-WNMul2nd\_GPT4.1\_Household Finance\_01
  \item Primary KC: PERCENTAGE | Representation and concept | expressing a part of a whole as a percentage
  \item Secondary KC: WHOLE NUMBERS | Multiplication | multiplying whole numbers
  \item Topic: Household finance such as income, utility bills, money, interest, savings, instalment, mortgage, financial planning etc.
  \item Grade: Primary 5
\end{itemize}

\textbf{Question}

Marissa saved \textdollar800 from her monthly salary. She decided to donate $15\%$ of her savings to a local charity. How much money did Marissa donate to the charity? How much did she have left after the donation?

\section*{Question 35}
\textbf{Metadata}

\begin{itemize}
  \item Question ID: P5-PcRepWh\_P1-WNDiv2nd\_GPT4.1\_Household Finance\_01
  \item Primary KC: PERCENTAGE | Representation and concept | expressing a part of a whole as a percentage
  \item Secondary KC: WHOLE NUMBERS | Division | dividing whole numbers
  \item Topic: Household finance such as income, utility bills, money, interest, savings, instalment, mortgage, financial planning etc.
  \item Grade: Primary 5
\end{itemize}

\textbf{Question}

Mrs Tan set aside her monthly electricity bill to be paid from her salary. Last month, her electricity bill was \textdollar240. She wanted to find out what percentage of her \textdollar1200 salary was used to pay the electricity bill. What percentage of her salary did Mrs Tan spend on her electricity bill?

\section*{Question 36}
\textbf{Metadata}

\begin{itemize}
  \item Question ID: P5-RtFndU\_P2-DcCnvN2D\_GPT4.1\_Household Finance\_01
  \item Primary KC: RATE | Finding number of unit | finding number of units given rate and total amount
  \item Secondary KC: DECIMALS | Conversion to larger units | converting a measurement from a smaller unit to a larger unit in decimal form
  \item Topic: Household finance such as income, utility bills, money, interest, savings, instalment, mortgage, financial planning etc.
  \item Grade: Primary 5
\end{itemize}

\textbf{Question}

Mrs Tan pays for electricity at a rate of $\textdollar0.22$ per kilowatt-hour ($kWh$). Last month, her household used $82500$ watt-hours ($Wh$) of electricity. 

(a) Convert the amount of electricity used from $Wh$ to $kWh$ in decimal form. 

(b) How much did Mrs Tan pay for electricity last month?

\section*{Question 37}
\textbf{Metadata}

\begin{itemize}
  \item Question ID: P6-FrDivPN\_P2-FrAdd2nd\_GPT4.1\_Household Finance\_01
  \item Primary KC: FRACTIONS | Division | dividing a proper fraction by a whole number
  \item Secondary KC: FRACTIONS | Addition | adding fractions
  \item Topic: Household finance such as income, utility bills, money, interest, savings, instalment, mortgage, financial planning etc.
  \item Grade: Primary 6
\end{itemize}

\textbf{Question}

Mrs Lee has $\textdollar12$ to buy fruit for her family. She decides to use $\frac{3}{4}$ of the money to buy apples. She wants to split the amount she spends on apples equally over $4$ weeks. 

(a) How much money does Mrs Lee spend on apples each week?

After 4 weeks, she also buys bananas for $\frac{1}{8}$ of the $\textdollar12$ she had originally. 

(b) What is the total fraction of the $\textdollar12$ that Mrs Lee spent on apples and bananas?

\section*{Question 38}
\textbf{Metadata}

\begin{itemize}
  \item Question ID: P6-FrDivPN\_P5-FrCnv2Dc\_GPT4.1\_Household Finance\_01
  \item Primary KC: FRACTIONS | Division | dividing a proper fraction by a whole number
  \item Secondary KC: FRACTIONS | Conversion to decimals | expressing fractions as decimals
  \item Topic: Household finance such as income, utility bills, money, interest, savings, instalment, mortgage, financial planning etc.
  \item Grade: Primary 6
\end{itemize}

\textbf{Question}

Mrs Lim baked $2\frac{1}{2}$ trays of cookies for a school fundraiser. She planned to split all the cookies equally among 5 donation boxes. 

(a) What fraction of a tray of cookies will each donation box receive?

(b) Express your answer in (a) as a decimal.

\section*{Question 39}
\textbf{Metadata}

\begin{itemize}
  \item Question ID: P6-FrDivPP\_P2-FrSub2nd\_GPT4.1\_Household Finance\_01
  \item Primary KC: FRACTIONS | Division | dividing a whole number/proper fraction by a proper fraction
  \item Secondary KC: FRACTIONS | Subtraction | subtracting fractions
  \item Topic: Household finance such as income, utility bills, money, interest, savings, instalment, mortgage, financial planning etc.
  \item Grade: Primary 6
\end{itemize}

\textbf{Question}

Mrs Tan has saved $\textdollar120$ to pay for her monthly household utility bills. Each month's bill costs $\frac{3}{4}$ as much as what she saved. She decides to pay the bill for 2 months and then wants to find out how much money she will have left. How many months of utility bills can Mrs Tan pay with her $\textdollar120$? After paying for 2 months, how much money does she have left?

\section*{Question 40}
\textbf{Metadata}

\begin{itemize}
  \item Question ID: P6-FrDivPP\_P5-FrCnv2Dc\_GPT4.1\_Household Finance\_01
  \item Primary KC: FRACTIONS | Division | dividing a whole number/proper fraction by a proper fraction
  \item Secondary KC: FRACTIONS | Conversion to decimals | expressing fractions as decimals
  \item Topic: Household finance such as income, utility bills, money, interest, savings, instalment, mortgage, financial planning etc.
  \item Grade: Primary 6
\end{itemize}

\textbf{Question}

Mrs Lee wants to divide \textdollar120 equally among her 3 children for their weekly allowance. However, she decides to give each child only $\frac{3}{4}$ of the amount she originally planned per child, and saves the rest. 

(a) How much money was Mrs Lee going to give each child at first? 

(b) How much money does each child receive when this amount is divided by $\frac{3}{4}$? Express your answer as a decimal. 

(c) How much money does Mrs Lee save in total after making this change?

\section*{Question 41}
\textbf{Metadata}

\begin{itemize}
  \item Question ID: P6-PcFndWN\_P1-WNAdd2nd\_GPT4.1\_Household Finance\_01
  \item Primary KC: PERCENTAGE | Finding the whole | finding the whole given a part and the percentage
  \item Secondary KC: WHOLE NUMBERS | Addition | adding whole numbers
  \item Topic: Household finance such as income, utility bills, money, interest, savings, instalment, mortgage, financial planning etc.
  \item Grade: Primary 6
\end{itemize}

\textbf{Question}

Alice paid $\textdollar210$ this month for her electricity bill. This amount is $30\%$ of her total monthly household expenses. This month, she also spent $\textdollar320$ on groceries and $\textdollar150$ on transport. What is the total amount Alice spent on her household expenses this month?

\section*{Question 42}
\textbf{Metadata}

\begin{itemize}
  \item Question ID: P6-PcFndChg\_P1-WNDiv2nd\_GPT4.1\_Household Finance\_01
  \item Primary KC: PERCENTAGE | Finding change | finding percentage increase/decrease
  \item Secondary KC: WHOLE NUMBERS | Division | dividing whole numbers
  \item Topic: Household finance such as income, utility bills, money, interest, savings, instalment, mortgage, financial planning etc.
  \item Grade: Primary 6
\end{itemize}

\textbf{Question}

Mrs Tan noticed that her electricity bill last month was $\textdollar120$. This month, after making some changes to her household, her electricity bill decreased to $\textdollar90$.

(a) Find the percentage decrease in Mrs Tan's electricity bill.

(b) If Mrs Tan wants to divide her savings from the decrease equally among her 3 children, how much will each child receive?

\section*{Question 43}
\textbf{Metadata}

\begin{itemize}
  \item Question ID: P6-RoFndRoWN\_P1-WNAdd2nd\_GPT4.1\_Household Finance\_01
  \item Primary KC: RATIO | Finding ratio | finding the ratio of two or three given whole numbers
  \item Secondary KC: WHOLE NUMBERS | Addition | adding whole numbers
  \item Topic: Household finance such as income, utility bills, money, interest, savings, instalment, mortgage, financial planning etc.
  \item Grade: Primary 6
\end{itemize}

\textbf{Question}

Ahmad, Bella, and Chloe each contribute some money to pay for their family's monthly utility bills. Ahmad contributes \textdollar60, Bella contributes \textdollar90, and Chloe contributes \textdollar120. 

(a) What is the total amount of money they contribute together for the utility bills?

(b) Find the ratio of Ahmad's contribution to Bella's contribution to Chloe's contribution. Give your answer in the simplest form.

\section*{Question 44}
\textbf{Metadata}

\begin{itemize}
  \item Question ID: P6-AgRepLrEx\_P6-AgSmpLrEx\_GPT4.1\_Household Finance\_01
  \item Primary KC: ALGEBRA | Representation and concept | translation of real-world situations into linear algebraic expressions
  \item Secondary KC: ALGEBRA | Simplifying | simplifying linear expressions
  \item Topic: Household finance such as income, utility bills, money, interest, savings, instalment, mortgage, financial planning etc.
  \item Grade: Primary 6
\end{itemize}

\textbf{Question}

Mrs Tan pays a fixed monthly fee of $\textdollar45$ for her internet service plus $\textdollar2$ for every hour of extra data she uses. She used $h$ hours of extra data last month. 

(a) Write an algebraic expression to represent the total amount, in dollars, Mrs Tan paid last month for internet service.

(b) Simplify the expression you formed in part (a).

\section*{Question 45}
\textbf{Metadata}

\begin{itemize}
  \item Question ID: O1-PcFndRslt\_P1-WNSub2nd\_GPT4.1\_Household Finance\_01
  \item Primary KC: PERCENTAGE | Finding result after change | increasing/decreasing a quantity by a given percentage
  \item Secondary KC: WHOLE NUMBERS | Subtraction | subtracting whole numbers
  \item Topic: Household finance such as income, utility bills, money, interest, savings, instalment, mortgage, financial planning etc.
  \item Grade: Secondary O-level 1
\end{itemize}

\textbf{Question}

A family used to pay a monthly electricity bill of $\textdollar120$. After installing new energy-saving lights, their electricity bill decreased by $15\%$. How much do they save each month now compared to before?

\section*{Question 46}
\textbf{Metadata}

\begin{itemize}
  \item Question ID: O1-PcRepRvs\_O1-PcCnv2Dc\_GPT4.1\_Household Finance\_02
  \item Primary KC: PERCENTAGE | Representation and concept | reverse percentages
  \item Secondary KC: PERCENTAGE | Conversion to decimals | expressing percentage as a decimal
  \item Topic: Household finance such as income, utility bills, money, interest, savings, instalment, mortgage, financial planning etc.
  \item Grade: Secondary O-level 1
\end{itemize}

\textbf{Question}

Mrs Tan received her monthly electricity bill. After a 20\% discount was applied for energy-saving efforts, she paid \textdollar80. Express 20\% as a decimal and calculate what the original bill amount was before the discount.

\section*{Question 47}
\textbf{Metadata}

\begin{itemize}
  \item Question ID: O2-RoRepDP\_P1-WNMul2nd\_GPT4.1\_Household Finance\_01
  \item Primary KC: RATIO | Representation and concept | direct proportion
  \item Secondary KC: WHOLE NUMBERS | Multiplication | multiplying whole numbers
  \item Topic: Household finance such as income, utility bills, money, interest, savings, instalment, mortgage, financial planning etc.
  \item Grade: Secondary O-level 2
\end{itemize}

\textbf{Question}

Mrs. Tan uses gas cylinders for cooking at home. The amount of gas used is directly proportional to the number of meals cooked. If 3 gas cylinders are needed to cook 180 meals, how many gas cylinders will Mrs. Tan need to cook 540 meals? Each gas cylinder costs $\textdollar28$. How much will she spend in total to cook 540 meals?

\section*{Question 48}
\textbf{Metadata}

\begin{itemize}
  \item Question ID: O3-MXMul\_O3-MXAdd\_GPT4.1\_Household Finance\_01
  \item Primary KC: MATRICES | Multiplication | multiplication of matrices
  \item Secondary KC: MATRICES | Addition | addition of matrices
  \item Topic: Household finance such as income, utility bills, money, interest, savings, instalment, mortgage, financial planning etc.
  \item Grade: Secondary O-level 3/4
\end{itemize}

\textbf{Question}

A family records their monthly utility expenses in two matrices. Matrix $A$ represents the costs (in \textdollar) for electricity, water, and gas over January and February:
$
A = \begin{pmatrix}
120 & 130 \\
50 & 55 \\
40 & 42
\end{pmatrix}
$
The rows represent electricity, water, and gas respectively, and the columns represent January and February. The utility provider announces a 10\% increase for March, which can be represented by the following matrix $B$:
$
B = \begin{pmatrix}
1.1 & 1.1 \\
1.1 & 1.1 \\
1.1 & 1.1
\end{pmatrix}
$
(a) Find the new expenses for March by multiplying $A$ and $B$ element-wise (Hadamard product).

(b) If the family receives a one-time refund in March represented by the matrix $C = \begin{pmatrix} 10 & 0 \\ 5 & 0 \\ 3 & 0 \end{pmatrix}$, find the final utility expenses for March after applying the refund by adding the appropriate matrices.

\section*{Question 49}
\textbf{Metadata}

\begin{itemize}
  \item Question ID: O3-SPFndstd\_O2-SPFndmean\_GPT4.1\_Household Finance\_01
  \item Primary KC: STATISTICS AND PROBABILITY | Finding standard deviation | calculation of the standard deviation for a set of data
  \item Secondary KC: STATISTICS AND PROBABILITY | Finding mean deviation | calculation of the mean for a set of data
  \item Topic: Household finance such as income, utility bills, money, interest, savings, instalment, mortgage, financial planning etc.
  \item Grade: Secondary O-level 3/4
\end{itemize}

\textbf{Question}

A family recorded their monthly electricity bills (in \textdollar) over the first 6 months of the year as follows: $ 120, 135, 110, 140, 130, 125 $ 

(a) Calculate the mean monthly electricity bill. 

(b) Calculate the standard deviation of the monthly electricity bills to measure the variability in their expenses.

\end{document}
