\documentclass{article}
\usepackage[utf8]{inputenc}
\usepackage{amsmath}
\usepackage{amsfonts}
\usepackage{amssymb}
\usepackage{graphicx}
\usepackage{hyperref}
\title{'TD Solutions recreation v8 v1'}
\author{Tien Dung Doan}
\begin{document}
\maketitle
\section*{Question 1}
\textbf{Metadata}

\begin{itemize}
  \item Question ID: P3-WNSub4d\_P1-WNAdd2nd\_GPT4.1\_Recreation\_01
  \item Primary KC: WHOLE NUMBERS | Subtraction | subtracting whole numbers up to 4 digits
  \item Secondary KC: WHOLE NUMBERS | Addition | adding whole numbers
  \item Topic: Recreation such as sports, games, exercises, music, movie, dancing, painting, fishing and other recreation activities
  \item Grade: Primary 3
\end{itemize}

\textbf{Solution}

First, find the number of stickers Darren has after giving away the stickers: 

$
1562 - 738 = 824
$

Next, add the stickers he won:

$
824 + 247 = 1071
$

Darren has $1071$ stickers now.

\section*{Question 2}
\textbf{Metadata}

\begin{itemize}
  \item Question ID: P5-DcDiv3dK\_P4-DcRnd3d\_GPT4.1\_Recreation\_01
  \item Primary KC: DECIMALS | Division | dividing decimals (up to 3 decimal places) by 10, 100, 1000 and their multiples
  \item Secondary KC: DECIMALS | Rounding | rounding decimals up to 3 decimal places to the nearest whole number, 1 decimal place and 2 decimal places 
  \item Topic: Recreation such as sports, games, exercises, music, movie, dancing, painting, fishing and other recreation activities
  \item Grade: Primary 5
\end{itemize}

\textbf{Solution}

(a) To find Jia Wen's average time per $10$ metres:

Total time $=202.457$ seconds

Total distance $=500$ metres

Number of $10$ metre segments in $500$ metres $= 500 \div 10 = 50$

Average time per $10$ metres $= 202.457 \div 50$

$202.457 \div 50 = 4.04914$

So, to $3$ decimal places:

Average time per $10$ metres $= 4.049$ seconds

(b) Rounding $4.049$:

- To the nearest whole number: $4$
- To $1$ decimal place: $4.0$
- To $2$ decimal places: $4.05$

\textbf{Answers:}

(a) $4.049$ seconds

(b) To the nearest whole number: $4$; to $1$ decimal place: $4.0$; to $2$ decimal places: $4.05$

\section*{Question 3}
\textbf{Metadata}

\begin{itemize}
  \item Question ID: P5-DcDiv3dK\_P4-DcAdd2nd\_GPT4.1\_Recreation\_01
  \item Primary KC: DECIMALS | Division | dividing decimals (up to 3 decimal places) by 10, 100, 1000 and their multiples
  \item Secondary KC: DECIMALS | Addition | adding decimals
  \item Topic: Recreation such as sports, games, exercises, music, movie, dancing, painting, fishing and other recreation activities
  \item Grade: Primary 5
\end{itemize}

\textbf{Solution}

First, find the total distance Wei Lin swam over the two days:

$
8.250 + 3.750 = 12.000
$

Next, divide the total distance by $10$ to find the average distance per day:

$
12.000 \div 10 = 1.200
$

Therefore, Wei Lin should plan to swim \(1.200\) metres each day.

\section*{Question 4}
\textbf{Metadata}

\begin{itemize}
  \item Question ID: P4-DcMul2d1d\_P4-DcCnv2Fr\_GPT4.1\_Recreation\_01
  \item Primary KC: DECIMALS | Multiplication | multiplying decimals (up to 2 decimal places) by a 1-digit whole number
  \item Secondary KC: DECIMALS | Conversion from decimals to fraction | expressing decimals as fractions
  \item Topic: Recreation such as sports, games, exercises, music, movie, dancing, painting, fishing and other recreation activities
  \item Grade: Primary 4
\end{itemize}

\textbf{Solution}

First, let's find the total distance Jason skates by multiplying the distance for one round by the number of rounds.$ 1.25 \times 4 = 5 $So, Jason skates a total of $5$ km.\newline\newline Now, to express $5$ km as a fraction, notice that $5 = \frac{5}{1}$.\newline\newline Therefore, Jason skates $5$ km in total, which can be expressed as $\frac{5}{1}$ km.

\section*{Question 5}
\textbf{Metadata}

\begin{itemize}
  \item Question ID: P4-DcDiv2d1d\_P4-DcCmp3d\_GPT4.1\_Recreation\_01
  \item Primary KC: DECIMALS | Division | dividing decimals (up to 2 decimal places) by a 1-digit whole number
  \item Secondary KC: DECIMALS | Comparison and ordering | comparing and ordering decimals up to 3 decimal places
  \item Topic: Recreation such as sports, games, exercises, music, movie, dancing, painting, fishing and other recreation activities
  \item Grade: Primary 4
\end{itemize}

\textbf{Solution}

(a) For each activity, divide the total minutes by $3$.

- Singing: $7.68 \div 3 = 2.56$ minutes per day
- Piano: $3.276 \div 3 = 1.092$ minutes per day
- Drumming: $5.13 \div 3 = 1.71$ minutes per day

(b) To order the daily times:
- Piano: $1.092$ (shortest)
- Drumming: $1.71$
- Singing: $2.56$ (longest)

So, the order from shortest to longest is: Piano, Drumming, Singing.

\section*{Question 6}
\textbf{Metadata}

\begin{itemize}
  \item Question ID: P3-WNMul3d1d\_P1-WNCmp\_GPT4.1\_Recreation\_01
  \item Primary KC: WHOLE NUMBERS | Multiplication | multiplying whole numbers up to 3 digits by 1 digit
  \item Secondary KC: WHOLE NUMBERS | Comparison and ordering | comparing and ordering whole numbers
  \item Topic: Recreation such as sports, games, exercises, music, movie, dancing, painting, fishing and other recreation activities
  \item Grade: Primary 3
\end{itemize}

\textbf{Solution}

Sarah's total practice time in a week is $45 \times 4 = 180$ minutes.

David's total practice time in a week is $30 \times 5 = 150$ minutes.

Comparing $180$ minutes and $150$ minutes, Sarah spends more time practicing music.

The difference in practice time is $180 - 150 = 30$ minutes.

\textbf{Answer:} Sarah spends more time practicing music than David by $30$ minutes.

\section*{Question 7}
\textbf{Metadata}

\begin{itemize}
  \item Question ID: P6-FrDivPP\_P5-FrCnv2Dc\_GPT4.1\_Recreation\_01
  \item Primary KC: FRACTIONS | Division | dividing a whole number/proper fraction by a proper fraction
  \item Secondary KC: FRACTIONS | Conversion to decimals | expressing fractions as decimals
  \item Topic: Recreation such as sports, games, exercises, music, movie, dancing, painting, fishing and other recreation activities
  \item Grade: Primary 6
\end{itemize}

\textbf{Solution}

(a) To find how many solo dance routines Megan can perform in $2$ hours, we divide $2$ by $\frac{3}{4}$: 

$
\text{Number of routines} = \frac{2}{\frac{3}{4}}
$
$
= 2 \div \frac{3}{4}
$
$
= 2 \times \frac{4}{3}
$
$
= \frac{8}{3}
$

So, Megan can perform $\frac{8}{3}$ solo dance routines in $2$ hours.

(b) To express $\frac{8}{3}$ as a decimal:

$
\frac{8}{3} = 2.666\ldots
$

Rounded to 2 decimal places, $\frac{8}{3} \approx 2.67$.

\textbf{Final Answers:}  
(a) $\frac{8}{3}$ solo dance routines  
(b) $2.67$ solo dance routines (rounded to 2 decimal places) 

\section*{Question 8}
\textbf{Metadata}

\begin{itemize}
  \item Question ID: P4-DcSub2d\_P4-DcRnd3d\_GPT4.1\_Recreation\_01
  \item Primary KC: DECIMALS | Subtraction | subtracting decimals (up to 2 decimal places)
  \item Secondary KC: DECIMALS | Rounding | rounding decimals up to 3 decimal places to the nearest whole number, 1 decimal place and 2 decimal places 
  \item Topic: Recreation such as sports, games, exercises, music, movie, dancing, painting, fishing and other recreation activities
  \item Grade: Primary 4
\end{itemize}

\textbf{Solution}

(a) Total distance in the first two rounds:

$2.45 + 1.38 = 3.83$ km

(b) Difference between the first two rounds and the third round:

$3.83 - 0.755 = 3.075$ km

Rounded to $2$ decimal places: $3.08$ km

(c) Distance in the third round to the nearest whole number:

$0.755$ km rounds up to $1$ km

\section*{Question 9}
\textbf{Metadata}

\begin{itemize}
  \item Question ID: O3-MXMul\_O3-MXAdd\_GPT4.1\_Recreation\_01
  \item Primary KC: MATRICES | Multiplication | multiplication of matrices
  \item Secondary KC: MATRICES | Addition | addition of matrices
  \item Topic: Recreation such as sports, games, exercises, music, movie, dancing, painting, fishing and other recreation activities
  \item Grade: Secondary O-level 3/4
\end{itemize}

\textbf{Solution}

(a) To find the combined matrix $C$, we add $A$ and $B$:
$
C = A + B = \begin{pmatrix} 2 & 4 & 3 \\ 1 & 3 & 2 \end{pmatrix} + \begin{pmatrix} 3 & 2 & 4 \\ 2 & 1 & 5 \end{pmatrix} = \begin{pmatrix} 5 & 6 & 7 \\ 3 & 4 & 7 \end{pmatrix}
$
So, $C = \begin{pmatrix} 5 & 6 & 7 \\ 3 & 4 & 7 \end{pmatrix}$. Each row gives the total weekly hours spent by each member on the three activities.

(b) To find the matrix product $C \times M$:
$
\begin{pmatrix} 5 & 6 & 7 \\ 3 & 4 & 7 \end{pmatrix} \times \begin{pmatrix} 1 \\ 0.5 \\ 0.8 \end{pmatrix} = \begin{pmatrix} 5 \times 1 + 6 \times 0.5 + 7 \times 0.8 \\ 3 \times 1 + 4 \times 0.5 + 7 \times 0.8 \end{pmatrix}
$
Calculating each entry:

First row: $5 \times 1 = 5$; $6 \times 0.5 = 3$; $7 \times 0.8 = 5.6$. Sum: $5 + 3 + 5.6 = 13.6$

Second row: $3 \times 1 = 3$; $4 \times 0.5 = 2$; $7 \times 0.8 = 5.6$. Sum: $3 + 2 + 5.6 = 10.6$

So the result is:
$
C \times M = \begin{pmatrix} 13.6 \\ 10.6 \end{pmatrix}
$
This means Alice and Cindy together have a total weighted activity time of $13.6$ units, and Ben and Dan together have a total weighted activity time of $10.6$ units per week when considering the impact factors of each sport.

\section*{Question 10}
\textbf{Metadata}

\begin{itemize}
  \item Question ID: P6-PcFndWN\_P1-WNAdd2nd\_GPT4.1\_Recreation\_01
  \item Primary KC: PERCENTAGE | Finding the whole | finding the whole given a part and the percentage
  \item Secondary KC: WHOLE NUMBERS | Addition | adding whole numbers
  \item Topic: Recreation such as sports, games, exercises, music, movie, dancing, painting, fishing and other recreation activities
  \item Grade: Primary 6
\end{itemize}

\textbf{Solution}

Let the total number of students who took part in the carnival originally be $x$.\newline
It is given that $45$ students represent $30\%$ of all the students.\newline
So, $30\% \times x = 45$\newline
$ 0.3x = 45 $\newline
$ x = \frac{45}{0.3} = 150 $\newline
Thus, originally, there were $150$ students at the carnival.\newline
When $15$ more students joined for the relay race, the total number of students became:\newline
$ 150 + 15 = 165 $\newline
\textbf{Therefore, the total number of students who participated in the carnival after the relay race participants joined was $165$.}

\section*{Question 11}
\textbf{Metadata}

\begin{itemize}
  \item Question ID: P5-FrMulImIm\_P2-FrSub2nd\_GPT4.1\_Recreation\_01
  \item Primary KC: FRACTIONS | Multiplication | multiplying two improper fractions
  \item Secondary KC: FRACTIONS | Subtraction | subtracting fractions
  \item Topic: Recreation such as sports, games, exercises, music, movie, dancing, painting, fishing and other recreation activities
  \item Grade: Primary 5
\end{itemize}

\textbf{Solution}

(a) To find the number of hours Maya spent in music sessions on the first day, multiply:

$
\frac{7}{4} \times \frac{5}{3} = \frac{7 \times 5}{4 \times 3} = \frac{35}{12}\ \text{hours}
$

(b) To find how many more hours she spent in music sessions than practising:

$
\frac{35}{12} - \frac{11}{6}
$
First, convert $\frac{11}{6}$ to have a common denominator of 12:
$
\frac{11}{6} = \frac{11 \times 2}{6 \times 2} = \frac{22}{12}
$
Now subtract:
$
\frac{35}{12} - \frac{22}{12} = \frac{13}{12}\ \text{hours}
$

**Answer:**
(a) Maya spent $\frac{35}{12}$ hours in music sessions on the first day.

(b) She spent $\frac{13}{12}$ hours more in music sessions than practising by herself.

\section*{Question 12}
\textbf{Metadata}

\begin{itemize}
  \item Question ID: P6-AgRepLrEx\_P6-AgSmpLrEx\_GPT4.1\_Recreation\_01
  \item Primary KC: ALGEBRA | Representation and concept | translation of real-world situations into linear algebraic expressions
  \item Secondary KC: ALGEBRA | Simplifying | simplifying linear expressions
  \item Topic: Recreation such as sports, games, exercises, music, movie, dancing, painting, fishing and other recreation activities
  \item Grade: Primary 6
\end{itemize}

\textbf{Solution}

If Sarah buys $n$ tickets at the regular price, the total cost is $nx$.

If Sarah buys $(n + 3)$ tickets with a discount of $2$ per ticket, the cost per ticket becomes $(x - 2)$.

So, the total cost = $(n + 3)(x - 2)$

Let's expand and simplify this expression:

$\begin{align*}
(n + 3)(x - 2) &= n(x - 2) + 3(x - 2) \\
&= nx - 2n + 3x - 6 \\
&= nx + 3x - 2n - 6
\end{align*}$

Therefore, the total cost if Sarah buys $(n + 3)$ tickets with the discount is $nx + 3x - 2n - 6$.

\section*{Question 13}
\textbf{Metadata}

\begin{itemize}
  \item Question ID: P4-DcDiv2d1d\_P4-DcAdd2nd\_GPT4.1\_Recreation\_01
  \item Primary KC: DECIMALS | Division | dividing decimals (up to 2 decimal places) by a 1-digit whole number
  \item Secondary KC: DECIMALS | Addition | adding decimals
  \item Topic: Recreation such as sports, games, exercises, music, movie, dancing, painting, fishing and other recreation activities
  \item Grade: Primary 4
\end{itemize}

\textbf{Solution}

First, find the total length of ribbon Lisa has: 

$$1.25 + 2.40 + 1.65 = 5.30\text{ m}$$

Next, divide the total length by 4 to find out how much ribbon each person gets:

$$5.30 \div 4 = 1.325\text{ m}$$

So, each person will get $1.325 \text{ m}$ of ribbon.

\section*{Question 14}
\textbf{Metadata}

\begin{itemize}
  \item Question ID: O1-PcFndRslt\_P1-WNSub2nd\_GPT4.1\_Recreation\_01
  \item Primary KC: PERCENTAGE | Finding result after change | increasing/decreasing a quantity by a given percentage
  \item Secondary KC: WHOLE NUMBERS | Subtraction | subtracting whole numbers
  \item Topic: Recreation such as sports, games, exercises, music, movie, dancing, painting, fishing and other recreation activities
  \item Grade: Secondary O-level 1
\end{itemize}

\textbf{Solution}

(a) Number of members after increase in April:

Number increased by $15\%$:
$60 \times \frac{15}{100} = 9$

Total members after increase: $60 + 9 = 69$

(b) Members left in July: $10$

Number of members remaining: $69 - 10 = 59$

\textbf{Final answer:}

(a) $69$ members

(b) $59$ members

\section*{Question 15}
\textbf{Metadata}

\begin{itemize}
  \item Question ID: O3-SPFndstd\_O2-SPFndmean\_GPT4.1\_Recreation\_01
  \item Primary KC: STATISTICS AND PROBABILITY | Finding standard deviation | calculation of the standard deviation for a set of data
  \item Secondary KC: STATISTICS AND PROBABILITY | Finding mean deviation | calculation of the mean for a set of data
  \item Topic: Recreation such as sports, games, exercises, music, movie, dancing, painting, fishing and other recreation activities
  \item Grade: Secondary O-level 3/4
\end{itemize}

\textbf{Solution}

(a) To calculate the mean:

Let the data be: $x_1 = 2$, $x_2 = 5$, $x_3 = 4$, $x_4 = 3$, $x_5 = 6$.

Mean $\overline{x} = \dfrac{2 + 5 + 4 + 3 + 6}{5} = \dfrac{20}{5} = 4$ hours.

(b) To calculate the standard deviation:

First, subtract the mean from each data point and square the result:

$(2 - 4)^2 = 4$

$(5 - 4)^2 = 1$

$(4 - 4)^2 = 0$

$(3 - 4)^2 = 1$

$(6 - 4)^2 = 4$

Sum of squared deviations: $4 + 1 + 0 + 1 + 4 = 10$

Variance $= \dfrac{10}{5} = 2$

Standard deviation $= \sqrt{2} \approx 1.41$ (correct to 2 decimal places).

\textbf{Final Answers:}

(a) Mean $= 4$ hours.

(b) Standard deviation $= 1.41$ hours (correct to 2 decimal places).

\section*{Question 16}
\textbf{Metadata}

\begin{itemize}
  \item Question ID: P5-DcMul3dK\_P4-DcSub2nd\_GPT4.1\_Recreation\_01
  \item Primary KC: DECIMALS | Multiplication | multiplying decimals (up to 3 decimal places) by 10, 100, 1000 and their multiples
  \item Secondary KC: DECIMALS | Subtraction | subtracting decimals
  \item Topic: Recreation such as sports, games, exercises, music, movie, dancing, painting, fishing and other recreation activities
  \item Grade: Primary 5
\end{itemize}

\textbf{Solution}

First, calculate the actual points Siti earned:

$
\text{Points per note} = 0.175
$
$
\text{Number of notes} = 100
$
$
\text{Total points} = 0.175 \times 100 = 17.5
$

The system added $2.35$ points by mistake, so we need to subtract this excess:

$
\text{Corrected score} = 17.5 - 2.35 = 15.15
$

\textbf{Answer:} Siti should have $15.15$ points after subtracting the wrongly added points.

\section*{Question 17}
\textbf{Metadata}

\begin{itemize}
  \item Question ID: P5-FrMulPIm\_P3-FrSmp\_GPT4.1\_Recreation\_01
  \item Primary KC: FRACTIONS | Multiplication | multiplying a proper fraction and a proper/improper fractions
  \item Secondary KC: FRACTIONS | Simplifying | expressing a fraction in its simplest form
  \item Topic: Recreation such as sports, games, exercises, music, movie, dancing, painting, fishing and other recreation activities
  \item Grade: Primary 5
\end{itemize}

\textbf{Solution}

To find out how much red paint Sarah used this month, we need to multiply the number of sessions (in terms of last month) by the amount of paint used per session. 

She used $\frac{3}{4}$ of a tube per session and attended $\frac{5}{3}$ as many sessions. 

Total paint used $= \frac{3}{4} \times \frac{5}{3}$

$= \frac{3 \times 5}{4 \times 3}$

$= \frac{15}{12}$

Now, simplify $\frac{15}{12}$ by dividing both numerator and denominator by 3:

$\frac{15 \div 3}{12 \div 3} = \frac{5}{4}$

So, Sarah used $\frac{5}{4}$ tubes of red paint in total this month.

\section*{Question 18}
\textbf{Metadata}

\begin{itemize}
  \item Question ID: P5-PcRepWh\_P1-WNDiv2nd\_GPT4.1\_Recreation\_01
  \item Primary KC: PERCENTAGE | Representation and concept | expressing a part of a whole as a percentage
  \item Secondary KC: WHOLE NUMBERS | Division | dividing whole numbers
  \item Topic: Recreation such as sports, games, exercises, music, movie, dancing, painting, fishing and other recreation activities
  \item Grade: Primary 5
\end{itemize}

\textbf{Solution}

First, we need to find what fraction of the participants received certificates. 

$\frac{32}{80} = 0.4$

To express this as a percentage, multiply by $100$:

$0.4 \times 100 = 40$

So, $40\%$ of the participants received certificates of participation.

\section*{Question 19}
\textbf{Metadata}

\begin{itemize}
  \item Question ID: P6-FrDivPN\_P2-FrAdd2nd\_GPT4.1\_Recreation\_01
  \item Primary KC: FRACTIONS | Division | dividing a proper fraction by a whole number
  \item Secondary KC: FRACTIONS | Addition | adding fractions
  \item Topic: Recreation such as sports, games, exercises, music, movie, dancing, painting, fishing and other recreation activities
  \item Grade: Primary 6
\end{itemize}

\textbf{Solution}

First, find the amount of watermelon each person gets after dividing $\frac{3}{4}$ by 4:

$
\frac{3}{4} \div 4 = \frac{3}{4} \times \frac{1}{4} = \frac{3}{16}
$

So, each person gets $\frac{3}{16}$ of a watermelon.

When the fifth friend joins, Sarah adds $\frac{1}{8}$ of a watermelon to what each person already has:

Amount each person has = $\frac{3}{16} + \frac{1}{8}$

Convert $\frac{1}{8}$ to sixteenths:

$
\frac{1}{8} = \frac{2}{16}
$

Add the fractions:

$
\frac{3}{16} + \frac{2}{16} = \frac{5}{16}
$

Each person now has $\frac{5}{16}$ of a watermelon.

\section*{Question 20}
\textbf{Metadata}

\begin{itemize}
  \item Question ID: P5-FrMulPIm\_P2-FrCmp\_GPT4.1\_Recreation\_01
  \item Primary KC: FRACTIONS | Multiplication | multiplying a proper fraction and a proper/improper fractions
  \item Secondary KC: FRACTIONS | Comparison and ordering | comparing and ordering fractions
  \item Topic: Recreation such as sports, games, exercises, music, movie, dancing, painting, fishing and other recreation activities
  \item Grade: Primary 5
\end{itemize}

\textbf{Solution}

(a) The fraction of a painting completed this week is:

$\frac{3}{5} \times \frac{2}{3} = \frac{3\times2}{5\times3} = \frac{6}{15} = \frac{2}{5}$.

So, Lina completed $\frac{2}{5}$ of a painting this week.

(b) Last week: $\frac{3}{5}$

This week: $\frac{2}{5}$

To compare, $\frac{3}{5} > \frac{2}{5}$.

Lina completed more of her painting last week.

The difference is: $\frac{3}{5} - \frac{2}{5} = \frac{1}{5}$.

Therefore, Lina completed $\frac{1}{5}$ more of a painting last week than this week.

\section*{Question 21}
\textbf{Metadata}

\begin{itemize}
  \item Question ID: O1-PcRepRvs\_O1-PcCnv2Dc\_GPT4.1\_Recreation\_01
  \item Primary KC: PERCENTAGE | Representation and concept | reverse percentages
  \item Secondary KC: PERCENTAGE | Conversion to decimals | expressing percentage as a decimal
  \item Topic: Recreation such as sports, games, exercises, music, movie, dancing, painting, fishing and other recreation activities
  \item Grade: Secondary O-level 1
\end{itemize}

\textbf{Solution}

First, express 15\% as a decimal: 

15\% = \frac{15}{100} = 0.15.

Let the original price be $x$.

After a 15\% discount, Eileen paid 85\% of the original price:

$102 = 0.85x$

To find $x$, divide both sides by $0.85$:

$x = \frac{102}{0.85} = 120$

The original price of the registration fee was \textdollar120.

\section*{Question 22}
\textbf{Metadata}

\begin{itemize}
  \item Question ID: P4-DcAdd2d\_P4-DcCmp3d\_GPT4.1\_Recreation\_01
  \item Primary KC: DECIMALS | Addition | adding decimals (up to 2 decimal places)
  \item Secondary KC: DECIMALS | Comparison and ordering | comparing and ordering decimals up to 3 decimal places
  \item Topic: Recreation such as sports, games, exercises, music, movie, dancing, painting, fishing and other recreation activities
  \item Grade: Primary 4
\end{itemize}

\textbf{Solution}

(a) Mei's total score for all three rounds is: 

$8.75 + 7.90 + 8.125 = 16.65 + 8.125 = 24.775$

Mei scored $24.775$ points in total.

(b) Comparing final round scores:

Mei's final round score: $8.125$

Hui's final round score: $8.085$

Since $8.125 > 8.085$, Mei scored higher in the final round.

Arranging the final round scores from smallest to largest:

Hui: $8.085$, Mei: $8.125$.

\section*{Question 23}
\textbf{Metadata}

\begin{itemize}
  \item Question ID: P5-FrMulImIm\_P3-FrSmp\_GPT4.1\_Recreation\_01
  \item Primary KC: FRACTIONS | Multiplication | multiplying two improper fractions
  \item Secondary KC: FRACTIONS | Simplifying | expressing a fraction in its simplest form
  \item Topic: Recreation such as sports, games, exercises, music, movie, dancing, painting, fishing and other recreation activities
  \item Grade: Primary 5
\end{itemize}

\textbf{Solution}

Each student practises $\frac{7}{4}$ hours each day, and there are $\frac{9}{5}$ times as many days this month. To find the total hours each student spends:

Total hours $= \frac{7}{4} \times \frac{9}{5}$

First, multiply the numerators and denominators:

$\frac{7}{4} \times \frac{9}{5} = \frac{7 \times 9}{4 \times 5} = \frac{63}{20}$

Next, check if $\frac{63}{20}$ can be simplified. The greatest common divisor of 63 and 20 is 1, so it is already in its simplest form.

**Final Answer:** Each student spends $\frac{63}{20}$ hours practising in total this month.

\section*{Question 24}
\textbf{Metadata}

\begin{itemize}
  \item Question ID: O2-RoRepDP\_P1-WNMul2nd\_GPT4.1\_Recreation\_01
  \item Primary KC: RATIO | Representation and concept | direct proportion
  \item Secondary KC: WHOLE NUMBERS | Multiplication | multiplying whole numbers
  \item Topic: Recreation such as sports, games, exercises, music, movie, dancing, painting, fishing and other recreation activities
  \item Grade: Secondary O-level 2
\end{itemize}

\textbf{Solution}

Let the number of mats needed be $m$ and the number of participants be $p$. Since $m$ is directly proportional to $p$, we have $m = k \times p$, where $k$ is the constant of proportionality.\newline From the information, $6 = k \times 8$, so $k = \frac{6}{8} = \frac{3}{4}$.\newline For 30 participants: $m = \frac{3}{4} \times 30 = 22.5$.\newline Since the number of mats must be a whole number, the organiser should prepare 23 mats.\newline The total cost is $23 \times 25 = 575$.\newline \textbf{Answer:} 23 mats are needed, and the total cost is \textdollar575.

\section*{Question 25}
\textbf{Metadata}

\begin{itemize}
  \item Question ID: P4-WNDiv4d1d\_P4-WNRnd5d\_GPT4.1\_Recreation\_01
  \item Primary KC: WHOLE NUMBERS | Division | dividing whole numbers up to 4 digits by 1 digit
  \item Secondary KC: WHOLE NUMBERS | Rounding | rounding whole numbers up to 100000 to the nearest 10, 100 or 1000 
  \item Topic: Recreation such as sports, games, exercises, music, movie, dancing, painting, fishing and other recreation activities
  \item Grade: Primary 4
\end{itemize}

\textbf{Solution}

First, divide $4362$ by $7$:

$
4362 \div 7 = 623.142857... 
$

Each player gets $623$ points, with some left over.

Next, round $623$ to the nearest $10$:

The last digit is $3$ (less than $5$), so we round down to $620$.

\textbf{Answer:}
To the nearest $10$, each player will get $\textdollar620$ points.

\end{document}
