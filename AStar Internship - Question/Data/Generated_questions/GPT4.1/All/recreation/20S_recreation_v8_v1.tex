\documentclass{article}
\usepackage[utf8]{inputenc}
\usepackage{amsmath}
\usepackage{amsfonts}
\usepackage{amssymb}
\usepackage{graphicx}
\usepackage{hyperref}
\title{recreation\_v8\_v1}
\author{Tien Dung Doan}
\begin{document}
\maketitle
\section*{Question 1}
\textbf{Metadata}

\begin{itemize}
  \item Question ID: P5-FrAddMix\_P5-FrCnv2Dc\_GPT4.1\_Recreation\_01
  \item Primary KC: FRACTIONS | Addition | adding mixed numbers
  \item Secondary KC: FRACTIONS | Conversion to decimals | expressing fractions as decimals
  \item Topic: Recreation such as sports, games, exercises, music, movie, dancing, painting, fishing and other recreation activities
  \item Grade: Primary 5
\end{itemize}

\textbf{Solution}

(a) First, add the mixed numbers:

$1\dfrac{3}{4} + 2\dfrac{2}{5}$

Convert the mixed numbers to improper fractions:

$1\dfrac{3}{4} = \frac{7}{4}$

$2\dfrac{2}{5} = \frac{12}{5}$

Find a common denominator for $4$ and $5$, which is $20$:

$\frac{7}{4} = \frac{35}{20}$

$\frac{12}{5} = \frac{48}{20}$

Add the fractions:

$\frac{35}{20}+\frac{48}{20}=\frac{83}{20}$

Convert $\frac{83}{20}$ back to a mixed number:

$83\div20=4$ remainder $3$, so $\frac{83}{20} = 4\dfrac{3}{20}$

So, Kelly practised piano for $4\dfrac{3}{20}$ hours in total on Monday and Tuesday.

(b) Express $4\dfrac{3}{20}$ as a decimal:

$\dfrac{3}{20} = 0.15$

So, $4 + 0.15 = 4.15$

Kelly practised piano for $4.15$ hours in total on Monday and Tuesday.

\section*{Question 2}
\textbf{Metadata}

\begin{itemize}
  \item Question ID: P6-FrDivPP\_P5-FrMul2nd\_GPT4.1\_Recreation\_01
  \item Primary KC: FRACTIONS | Division | dividing a whole number/proper fraction by a proper fraction
  \item Secondary KC: FRACTIONS | Multiplication | fraction multiplication
  \item Topic: Recreation such as sports, games, exercises, music, movie, dancing, painting, fishing and other recreation activities
  \item Grade: Primary 6
\end{itemize}

\textbf{Solution}

(a) To find how many songs Alvin can practise:

Number of songs $= \frac{3}{4} \div \frac{1}{8} = \frac{3}{4} \times \frac{8}{1} = \frac{3 \times 8}{4 \times 1} = \frac{24}{4} = 6$

So Alvin can practise $6$ songs.

(b) Time spent on the first song $= \frac{2}{3} \times \frac{1}{8} = \frac{2 \times 1}{3 \times 8} = \frac{2}{24} = \frac{1}{12}$ hour.

So Alvin spends $\frac{1}{12}$ hour on the first song.

\section*{Question 3}
\textbf{Metadata}

\begin{itemize}
  \item Question ID: P3-WNDivRmd3d\_P1-WNSub2nd\_GPT4.1\_Recreation\_01
  \item Primary KC: WHOLE NUMBERS | Division | dividing whole numbers up to 3 digits by 1 digit with remainder 
  \item Secondary KC: WHOLE NUMBERS | Subtraction | subtracting whole numbers
  \item Topic: Recreation such as sports, games, exercises, music, movie, dancing, painting, fishing and other recreation activities
  \item Grade: Primary 3
\end{itemize}

\textbf{Solution}

To find out how many stickers each student gets and how many are left:

$154 \div 6 = 25$ remainder $4$

Each student gets $25$ stickers, and the teacher has $4$ stickers left.

If the teacher keeps $2$ stickers for herself:

$4 - 2 = 2$

So, after keeping $2$ stickers, the teacher has $2$ stickers remaining.

\section*{Question 4}
\textbf{Metadata}

\begin{itemize}
  \item Question ID: P4-WNDiv4d1d\_P1-WNCmp\_GPT4.1\_Recreation\_01
  \item Primary KC: WHOLE NUMBERS | Division | dividing whole numbers up to 4 digits by 1 digit
  \item Secondary KC: WHOLE NUMBERS | Comparison and ordering | comparing and ordering whole numbers
  \item Topic: Recreation such as sports, games, exercises, music, movie, dancing, painting, fishing and other recreation activities
  \item Grade: Primary 4
\end{itemize}

\textbf{Solution}

(a) To find the number of children in each team, divide $1,356$ by $4$.\\
$1,356 \div 4 = 339$ \\ Children in each team: $339$\\
Leftover children: $1,356 - (4 \times 339) = 1,356 - 1,356 = 0$\\
So, there are no leftover children.\\
\\
(b) Comparing the number of children in each team ($339$) and the number of leftover children ($0$), we see that $339 > 0$.\
\\
Therefore, the number of children in each team is greater than the number of leftover children.

\section*{Question 5}
\textbf{Metadata}

\begin{itemize}
  \item Question ID: P4-DcDiv2d1d\_P4-DcCnv2Fr\_GPT4.1\_Recreation\_01
  \item Primary KC: DECIMALS | Division | dividing decimals (up to 2 decimal places) by a 1-digit whole number
  \item Secondary KC: DECIMALS | Conversion from decimals to fraction | expressing decimals as fractions
  \item Topic: Recreation such as sports, games, exercises, music, movie, dancing, painting, fishing and other recreation activities
  \item Grade: Primary 4
\end{itemize}

\textbf{Solution}

(a) The total cost for 6 packs is \textdollar9.84. To find the price of each pack:

\[
\text{Price per pack} = \frac{9.84}{6} = 1.64
\]

So, Siti paid \textdollar1.64 for each pack of guitar strings.

(b) To express 1.64 as a fraction:

\[
1.64 = \frac{164}{100} = \frac{41}{25}
\]

So, the price of each pack as a fraction in its simplest form is \(\frac{41}{25}\).

\section*{Question 6}
\textbf{Metadata}

\begin{itemize}
  \item Question ID: P6-RoFndDvqWN\_P1-WNAdd2nd\_GPT4.1\_Recreation\_01
  \item Primary KC: RATIO | Finding divided quantities | dividing a given quantity in a given ratio
  \item Secondary KC: WHOLE NUMBERS | Addition | adding whole numbers
  \item Topic: Recreation such as sports, games, exercises, music, movie, dancing, painting, fishing and other recreation activities
  \item Grade: Primary 6
\end{itemize}

\textbf{Solution}

First, we divide \textdollar180 in the ratio $2:3$ between Aaron and Bella. 

Find the total number of parts:
$2 + 3 = 5$

Aaron's share:
$\frac{2}{5} \times 180 = 72$

Bella's share:
$\frac{3}{5} \times 180 = 108$

So, Aaron paid \textdollar72 and Bella paid \textdollar108 for the tickets.

To find the total amount spent by the group (tickets + snacks):
$180 + 40 = 220$

The group spent a total of \textdollar220, including Charlie's contribution for snacks.

\section*{Question 7}
\textbf{Metadata}

\begin{itemize}
  \item Question ID: P5-FrMulPIm\_P2-FrSub2nd\_GPT4.1\_Recreation\_01
  \item Primary KC: FRACTIONS | Multiplication | multiplying a proper fraction and a proper/improper fractions
  \item Secondary KC: FRACTIONS | Subtraction | subtracting fractions
  \item Topic: Recreation such as sports, games, exercises, music, movie, dancing, painting, fishing and other recreation activities
  \item Grade: Primary 5
\end{itemize}

\textbf{Solution}

(a) To find how many hours Ella spent playing scales:

Ella practised piano for $\dfrac{3}{5}$ \text{hour}.
She spent $\dfrac{4}{7}$ of this time on scales:

$\dfrac{4}{7} \times \dfrac{3}{5} = \dfrac{4 \times 3}{7 \times 5} = \dfrac{12}{35}$ \text{hour}.

Answer: Ella spent $\dfrac{12}{35}$ \text{hour} playing scales.

(b) Ella practised piano $\dfrac{1}{5}$ \text{hour} less than Ben:

Amount of time Ben spent $=$ Amount of time Ella spent $+$ $\dfrac{1}{5}$ \text{hour}

Ella: $\dfrac{3}{5}$ \text{hour}

So, Ben: $\dfrac{3}{5} + \dfrac{1}{5} = \dfrac{4}{5}$ \text{hour}

Answer: Ben spent $\dfrac{4}{5}$ \text{hour} practising piano.

\section*{Question 8}
\textbf{Metadata}

\begin{itemize}
  \item Question ID: P4-WNMul4d1d\_P1-WNCmp\_GPT4.1\_Recreation\_01
  \item Primary KC: WHOLE NUMBERS | Multiplication | multiplying whole numbers up to 4 digits by 1 digit or up to 3 digits by 2 digits
  \item Secondary KC: WHOLE NUMBERS | Comparison and ordering | comparing and ordering whole numbers
  \item Topic: Recreation such as sports, games, exercises, music, movie, dancing, painting, fishing and other recreation activities
  \item Grade: Primary 4
\end{itemize}

\textbf{Solution}

(a) The total points the Red Team collected is $135 \times 7 = 945$ points.\\
(b) The total points the Blue Team collected is $248 \times 6 = 1,488$ points.\\
(c) Comparing the two totals, the Blue Team collected more points.\\
Difference in points: $1,488 - 945 = 543$ points.\\
\\
\textbf{Answer:}\\
(a) 945 points\\
(b) 1,488 points\\
(c) The Blue Team collected more points by 543 points.

\section*{Question 9}
\textbf{Metadata}

\begin{itemize}
  \item Question ID: O2-SPFndmdn\_O3-SPFndPctl\_GPT4.1\_Recreation\_01
  \item Primary KC: STATISTICS AND PROBABILITY | Finding median | Finding median for a set of data
  \item Secondary KC: STATISTICS AND PROBABILITY | Finding percentiles | finding percentiles for a set of data
  \item Topic: Recreation such as sports, games, exercises, music, movie, dancing, painting, fishing and other recreation activities
  \item Grade: Secondary O-level 2
\end{itemize}

\textbf{Solution}

(a) To find the median, first check that the data is arranged in ascending order. There are 11 numbers (an odd number), so the median is the value at position $\frac{11+1}{2}=6$ in the data set.\\

The 6th value is $16$.\\

\textbf{Median = 16}$.$\\

(b) To find the 25th percentile (first quartile, $Q_1$), use the position formula: $\frac{1}{4}(n+1)$, where $n=11$.\\

So, $Q_1$ is at position $\frac{1}{4}\times(11+1) = \frac{1}{4}\times 12=3$; the 3rd value is $10$.\\

To find the 75th percentile (third quartile, $Q_3$), use the position formula: $\frac{3}{4}(n+1)$.\\

$Q_3$ is at position $\frac{3}{4}\times 12=9$; the 9th value is $20$.\\

\textbf{25th percentile (Q$_1$) = 10}$.$\\
\textbf{75th percentile (Q$_3$) = 20}$.$

\section*{Question 10}
\textbf{Metadata}

\begin{itemize}
  \item Question ID: P4-WNMul4d1d\_P1-WNAdd2nd\_GPT4.1\_Recreation\_01
  \item Primary KC: WHOLE NUMBERS | Multiplication | multiplying whole numbers up to 4 digits by 1 digit or up to 3 digits by 2 digits
  \item Secondary KC: WHOLE NUMBERS | Addition | adding whole numbers
  \item Topic: Recreation such as sports, games, exercises, music, movie, dancing, painting, fishing and other recreation activities
  \item Grade: Primary 4
\end{itemize}

\textbf{Solution}

First, find the total number of water bottles given to all teams:

\[
235 \times 7 = 1,645
\]

Next, add the 58 spare bottles:

\[
1,645 + 58 = 1,703
\]

So, there will be 1,703 water bottles in all after adding the spare bottles.

\section*{Question 11}
\textbf{Metadata}

\begin{itemize}
  \item Question ID: P5-FrSubMix\_P2-FrAdd2nd\_GPT4.1\_Recreation\_01
  \item Primary KC: FRACTIONS | Subtraction | subtracting mixed numbers
  \item Secondary KC: FRACTIONS | Addition | adding fractions
  \item Topic: Recreation such as sports, games, exercises, music, movie, dancing, painting, fishing and other recreation activities
  \item Grade: Primary 5
\end{itemize}

\textbf{Solution}

(a) Total distance Sarah cycled:

$4\frac{2}{3} + 1\frac{5}{6}$

First, convert mixed numbers to improper fractions:
$4\frac{2}{3} = \frac{14}{3}$
$1\frac{5}{6} = \frac{11}{6}$

Find a common denominator:
$
\frac{14}{3} = \frac{28}{6}$

Add the fractions:
$\frac{28}{6} + \frac{11}{6} = \frac{39}{6} = 6\frac{1}{2}$

Sarah cycled a total of $6\frac{1}{2}$ km.

(b) Distance left to reach her goal:

$8\frac{1}{2} - 6\frac{1}{2}$

Convert to improper fractions:
$8\frac{1}{2} = \frac{17}{2}$
$6\frac{1}{2} = \frac{13}{2}$

Subtract:
$\frac{17}{2} - \frac{13}{2} = \frac{4}{2} = 2$

Sarah needed to cycle 2 km more to reach her goal.

\section*{Question 12}
\textbf{Metadata}

\begin{itemize}
  \item Question ID: O3-MXSub\_O3-MXAdd\_GPT4.1\_Recreation\_01
  \item Primary KC: MATRICES | Subtraction | subtraction of matrices
  \item Secondary KC: MATRICES | Addition | addition of matrices
  \item Topic: Recreation such as sports, games, exercises, music, movie, dancing, painting, fishing and other recreation activities
  \item Grade: Secondary O-level 3/4
\end{itemize}

\textbf{Solution}

(a) To find the corrected medal count for each house after the first day, we subtract matrix $C$ from $A$:
\[
A - C = \begin{bmatrix} 5 & 3 \\ 2 & 4 \\ 6 & 1 \end{bmatrix} - \begin{bmatrix} 0 & 1 \\ 0 & 1 \\ 0 & 1 \end{bmatrix} = \begin{bmatrix} 5-0 & 3-1 \\ 2-0 & 4-1 \\ 6-0 & 1-1 \end{bmatrix} = \begin{bmatrix} 5 & 2 \\ 2 & 3 \\ 6 & 0 \end{bmatrix}.
\]

(b) To find the total medals after both days, we add matrix $B$ to the result from part (a):
\[
\begin{bmatrix} 5 & 2 \\ 2 & 3 \\ 6 & 0 \end{bmatrix} + \begin{bmatrix} 3 & 2 \\ 1 & 2 \\ 2 & 2 \end{bmatrix} = \begin{bmatrix} 5+3 & 2+2 \\ 2+1 & 3+2 \\ 6+2 & 0+2 \end{bmatrix} = \begin{bmatrix} 8 & 4 \\ 3 & 5 \\ 8 & 2 \end{bmatrix}.
\]

**Final Answer:**

The corrected medal counts after the first day for the Red, Blue, and Green houses are:
\[
\begin{bmatrix} 5 & 2 \\ 2 & 3 \\ 6 & 0 \end{bmatrix}
\]

The total medal counts after both days are:
\[
\begin{bmatrix} 8 & 4 \\ 3 & 5 \\ 8 & 2 \end{bmatrix}
\]
where rows represent the houses (Red, Blue, Green) and columns represent Track and Field and Swimming medals, respectively.

\section*{Question 13}
\textbf{Metadata}

\begin{itemize}
  \item Question ID: P5-PcRepWh\_P1-WNAdd2nd\_GPT4.1\_Recreation\_01
  \item Primary KC: PERCENTAGE | Representation and concept | expressing a part of a whole as a percentage
  \item Secondary KC: WHOLE NUMBERS | Addition | adding whole numbers
  \item Topic: Recreation such as sports, games, exercises, music, movie, dancing, painting, fishing and other recreation activities
  \item Grade: Primary 5
\end{itemize}

\textbf{Solution}

(a) First, we find the total number of students who participated:

$120 + 80 + 100 = 300$

So, there were $300$ students who participated in these three sports altogether.

(b) To find the percentage of participants in the swimming event:

\[
\text{Percentage} = \frac{80}{300} \times 100\%
\]
$\frac{80}{300} = 0.2666\ldots$

$0.2666\ldots \times 100\% = 26.666\ldots\%$

Correct to the nearest whole number, this is $27\%$.

\textbf{Answer:}

(a) $300$ students 

(b) $27\%$ of the participants joined the swimming event.

\section*{Question 14}
\textbf{Metadata}

\begin{itemize}
  \item Question ID: P5-PcRepWh\_P1-WNMul2nd\_GPT4.1\_Recreation\_01
  \item Primary KC: PERCENTAGE | Representation and concept | expressing a part of a whole as a percentage
  \item Secondary KC: WHOLE NUMBERS | Multiplication | multiplying whole numbers
  \item Topic: Recreation such as sports, games, exercises, music, movie, dancing, painting, fishing and other recreation activities
  \item Grade: Primary 5
\end{itemize}

\textbf{Solution}

The total number of students is $250$. The number of students who played basketball is $70$.

The percentage of students who played basketball is:
\[
\frac{70}{250} \times 100\% = \frac{70 \times 100}{250}\% = \frac{7000}{250}\% = 28\%
\]

So, $28\%$ of the students played basketball.

\section*{Question 15}
\textbf{Metadata}

\begin{itemize}
  \item Question ID: P5-DcMul3dK\_P4-DcRnd3d\_GPT4.1\_Recreation\_01
  \item Primary KC: DECIMALS | Multiplication | multiplying decimals (up to 3 decimal places) by 10, 100, 1000 and their multiples
  \item Secondary KC: DECIMALS | Rounding | rounding decimals up to 3 decimal places to the nearest whole number, 1 decimal place and 2 decimal places 
  \item Topic: Recreation such as sports, games, exercises, music, movie, dancing, painting, fishing and other recreation activities
  \item Grade: Primary 5
\end{itemize}

\textbf{Solution}

(a) To find the total minutes Amira spent, multiply the duration of one song by the number of songs:

$3.275 \times 20 = 65.5$

Therefore, Amira practised for $65.5$ minutes on Saturday.

(b) Rounding $65.5$ to the nearest whole number:

Since $0.5$ rounds up, $65.5$ rounds to $66$.

Amira spent $\textdollar66$ minutes practising, rounded to the nearest whole number.

(c) Rounding $65.5$ to $1$ decimal place:

There are no more digits after the $0.5$, so it stays as $65.5$.

Amira spent $65.5$ minutes practising, rounded to $1$ decimal place.

\section*{Question 16}
\textbf{Metadata}

\begin{itemize}
  \item Question ID: O2-AgSlvSq1v\_O1-AgRepEq\_GPT4.1\_Recreation\_01
  \item Primary KC: ALGEBRA | Solving | solving quadratic equations in one variable
  \item Secondary KC: ALGEBRA | Representation and concept | translation of simple real-world situations to equations
  \item Topic: Recreation such as sports, games, exercises, music, movie, dancing, painting, fishing and other recreation activities
  \item Grade: Secondary O-level 2
\end{itemize}

\textbf{Solution}

Let $h = 15$, so $15 = -5t^2 + 20t$. Rearranging, we get:

$$
-5t^2 + 20t - 15 = 0
$$
Divide both sides by $-5$:
$$
t^2 - 4t + 3 = 0
$$
This is a quadratic equation. We solve it by factoring:

$$(t - 1)(t - 3) = 0$$

Thus, $t = 1$ or $t = 3$.

So, the ball reaches the height of \textdollar15$ metres after $1$ second and $3$ seconds.

\boxed{1 \text{ s and } 3 \text{ s}}

\section*{Question 17}
\textbf{Metadata}

\begin{itemize}
  \item Question ID: P5-FrSubMix\_P2-FrCmp\_GPT4.1\_Recreation\_01
  \item Primary KC: FRACTIONS | Subtraction | subtracting mixed numbers
  \item Secondary KC: FRACTIONS | Comparison and ordering | comparing and ordering fractions
  \item Topic: Recreation such as sports, games, exercises, music, movie, dancing, painting, fishing and other recreation activities
  \item Grade: Primary 5
\end{itemize}

\textbf{Solution}

(a) Total distance = $6$ km.

Jayden cycled $4 \frac{2}{3}$ km before his break.

Distance Jayden had left:
$6 - 4 \frac{2}{3} = 6 - \frac{14}{3} = \frac{18}{3} - \frac{14}{3} = \frac{4}{3}$ km

Sarah cycled $3 \frac{5}{6}$ km before her break.

Distance Sarah had left:
$6 - 3 \frac{5}{6} = 6 - \frac{23}{6} = \frac{36}{6} - \frac{23}{6} = \frac{13}{6}$ km

Compare the remaining distances:
Jayden: $\frac{4}{3} = \frac{8}{6}$ km
Sarah: $\frac{13}{6}$ km

How many more kilometres did Jayden have left after his break than Sarah?
$\frac{8}{6} - \frac{13}{6} = -\frac{5}{6}$ km

This means Jayden actually had $\frac{5}{6}$ km less to cycle after his break than Sarah.

(b) Jayden had $\frac{8}{6} = 1 \frac{1}{3}$ km left, and Sarah had $2 \frac{1}{6}$ km left. 

Since $\frac{8}{6} < \frac{13}{6}$, Jayden had less distance left to cycle after their break.

\textbf{Final answers:}

(a) Jayden had $\frac{5}{6}$ km less left to cycle after his break than Sarah did after hers.

(b) Jayden had less distance left to cycle after his break.

\section*{Question 18}
\textbf{Metadata}

\begin{itemize}
  \item Question ID: P6-FrDivPP\_P2-FrSub2nd\_GPT4.1\_Recreation\_01
  \item Primary KC: FRACTIONS | Division | dividing a whole number/proper fraction by a proper fraction
  \item Secondary KC: FRACTIONS | Subtraction | subtracting fractions
  \item Topic: Recreation such as sports, games, exercises, music, movie, dancing, painting, fishing and other recreation activities
  \item Grade: Primary 6
\end{itemize}

\textbf{Solution}

(a) To find the number of bookmarks Lina can make, divide the total length of ribbon by the length used for each bookmark:
\[
\text{Number of bookmarks} = \frac{3}{\frac{1}{4}} = 3 \times 4 = 12
\]
Lina can make $12$ bookmarks.

(b) Lina gives $\frac{2}{5}$ of the bookmarks away:
\[
\text{Bookmarks given} = \frac{2}{5} \times 12 = \frac{24}{5} = 4.8
\]
Since Lina cannot give away part of a bookmark, she gives away $4$ whole bookmarks (assuming bookmarks must be whole), or, if parts are allowed, $4.8$ bookmarks.

Bookmarks Lina has left:
\[
12 - 4.8 = 7.2
\]
\textbf{Lina has $7.2$ bookmarks left after giving some to her friends.}

(If only whole bookmarks can be given, then she gives $4$ bookmarks and has $8$ left.)

\section*{Question 19}
\textbf{Metadata}

\begin{itemize}
  \item Question ID: P5-FrMulMixN\_P3-FrSmp\_GPT4.1\_Recreation\_01
  \item Primary KC: FRACTIONS | Multiplication | multiplying a mixed number and a whole number
  \item Secondary KC: FRACTIONS | Simplifying | expressing a fraction in its simplest form
  \item Topic: Recreation such as sports, games, exercises, music, movie, dancing, painting, fishing and other recreation activities
  \item Grade: Primary 5
\end{itemize}

\textbf{Solution}

Let the amount of fruit punch in each jug be $2\dfrac{1}{2}$ litres. 

First, write $2\dfrac{1}{2}$ as an improper fraction:

$2\dfrac{1}{2} = 2 + \dfrac{1}{2} = \dfrac{4}{2} + \dfrac{1}{2} = \dfrac{5}{2}$

(a) Total amount of fruit punch for 6 jugs = $6 \times \dfrac{5}{2}$

$6 \times \dfrac{5}{2} = \dfrac{6 \times 5}{2} = \dfrac{30}{2}$

(b) Simplify $\dfrac{30}{2}$:

$\dfrac{30}{2} = 15$

Melissa will prepare 15 litres of fruit punch in total.

\section*{Question 20}
\textbf{Metadata}

\begin{itemize}
  \item Question ID: P4-DcSub2d\_P4-DcCnv2Fr\_GPT4.1\_Recreation\_01
  \item Primary KC: DECIMALS | Subtraction | subtracting decimals (up to 2 decimal places)
  \item Secondary KC: DECIMALS | Conversion from decimals to fraction | expressing decimals as fractions
  \item Topic: Recreation such as sports, games, exercises, music, movie, dancing, painting, fishing and other recreation activities
  \item Grade: Primary 4
\end{itemize}

\textbf{Solution}

(a) To find how many more kilometres Mandy swam on Monday, subtract the distance on Tuesday from the distance on Monday:

$$2.75 - 1.30 = 1.45$$

Mandy swam $1.45$ km more on Monday than on Tuesday.

(b) Express $1.45$ as a fraction:

$1.45 = 1 + 0.45 = 1 + \frac{45}{100}$

Simplify $\frac{45}{100}$:

$\frac{45}{100} = \frac{9}{20}$

So, $1.45 = 1\frac{9}{20}$

Final answer:

Mandy swam $1.45$ km more on Monday than on Tuesday, or $1\frac{9}{20}$ km.

\end{document}
