\documentclass{article}
\usepackage[utf8]{inputenc}
\usepackage{amsmath}
\usepackage{amsfonts}
\usepackage{amssymb}
\usepackage{graphicx}
\usepackage{hyperref}
\title{'Minh Solutions recreation v8 v1'}
\author{Tien Dung Doan}
\begin{document}
\maketitle
\section*{Question 1}
\textbf{Metadata}

\begin{itemize}
  \item Question ID: P4-DcAdd2d\_P4-DcRnd3d\_GPT4.1\_Recreation\_01
  \item Primary KC: DECIMALS | Addition | adding decimals (up to 2 decimal places)
  \item Secondary KC: DECIMALS | Rounding | rounding decimals up to 3 decimal places to the nearest whole number, 1 decimal place and 2 decimal places 
  \item Topic: Recreation such as sports, games, exercises, music, movie, dancing, painting, fishing and other recreation activities
  \item Grade: Primary 4
\end{itemize}

\textbf{Solution}

Step 1: Add up the distances:

$1.28 + 2.53 + 1.797 = 1.280 + 2.530 + 1.797 = 5.607$ km

Step 2: Round $5.607$ to the nearest whole number:
- The digit after the decimal is $6$ (which is 5 or more), so round up:
- Nearest whole number: $6$ km

Step 3: Round $5.607$ to 1 decimal place:
- The second digit after the decimal is $0$ (which is less than 5), so round down:
- $5.607$ rounded to 1 decimal place is $5.6$ km

Step 4: Round $5.607$ to 2 decimal places:
- The third digit after the decimal is $7$ (which is 5 or more), so round up:
- $5.607$ rounded to 2 decimal places is $5.61$ km

Final Answers:

(a) Jia Wei swam a total of $5.607$ km.

(b)
- Nearest whole number: $6$ km
- 1 decimal place: $5.6$ km
- 2 decimal places: $5.61$ km

\section*{Question 2}
\textbf{Metadata}

\begin{itemize}
  \item Question ID: P6-FrDivPN\_P3-FrSmp\_GPT4.1\_Recreation\_01
  \item Primary KC: FRACTIONS | Division | dividing a proper fraction by a whole number
  \item Secondary KC: FRACTIONS | Simplifying | expressing a fraction in its simplest form
  \item Topic: Recreation such as sports, games, exercises, music, movie, dancing, painting, fishing and other recreation activities
  \item Grade: Primary 6
\end{itemize}

\textbf{Solution}

Each person gets $\dfrac{3}{4} \div 3$ of the pizza.\newline

$\dfrac{3}{4} \div 3 = \dfrac{3}{4} \times \dfrac{1}{3} = \dfrac{3 \times 1}{4 \times 3} = \dfrac{3}{12}$\newline

Now, we simplify $\dfrac{3}{12}$ by dividing both numerator and denominator by $3$:\newline

$\dfrac{3 \div 3}{12 \div 3} = \dfrac{1}{4}$\newline

\textbf{Each person gets $\dfrac{1}{4}$ of the pizza.}

\section*{Question 3}
\textbf{Metadata}

\begin{itemize}
  \item Question ID: P5-DcDiv3dK\_P4-DcCnv2Fr\_GPT4.1\_Recreation\_01
  \item Primary KC: DECIMALS | Division | dividing decimals (up to 3 decimal places) by 10, 100, 1000 and their multiples
  \item Secondary KC: DECIMALS | Conversion from decimals to fraction | expressing decimals as fractions
  \item Topic: Recreation such as sports, games, exercises, music, movie, dancing, painting, fishing and other recreation activities
  \item Grade: Primary 5
\end{itemize}

\textbf{Solution}

(a) To find how many hours each student gets to practise, divide $0.125$ hours by $1000$:

$$ 0.125 \div 1000 = 0.000125 $$

So, each student gets $0.000125$ hours to practise the song.

(b) To express $0.000125$ as a fraction:

$$ 0.000125 = \frac{125}{1000000} $$

Now, simplify $\frac{125}{1000000}$:
- Divide numerator and denominator by $125$:
$$ \frac{125 \div 125}{1000000 \div 125} = \frac{1}{8000} $$

So, each student practises $\frac{1}{8000}$ hours.

\section*{Question 4}
\textbf{Metadata}

\begin{itemize}
  \item Question ID: O3-MXMulSM\_O3-MXSub\_GPT4.1\_Recreation\_01
  \item Primary KC: MATRICES | Multiplication | product of a scalar quantity and a matrix
  \item Secondary KC: MATRICES | Subtraction | subtraction of matrices
  \item Topic: Recreation such as sports, games, exercises, music, movie, dancing, painting, fishing and other recreation activities
  \item Grade: Secondary O-level 3/4
\end{itemize}

\textbf{Solution}

First, calculate the product of a scalar and Team X's matrix:

\[
2M = 2 \times \begin{bmatrix} 2 & 4 \\ 3 & 5 \end{bmatrix} = \begin{bmatrix} 4 & 8 \\ 6 & 10 \end{bmatrix}
\]

Next, subtract Team Y's matrix:

\[
2M - N = \begin{bmatrix} 4 & 8 \\ 6 & 10 \end{bmatrix} - \begin{bmatrix} 1 & 2 \\ 2 & 3 \end{bmatrix}
= \begin{bmatrix} 4 - 1 & 8 - 2 \\ 6 - 2 & 10 - 3 \end{bmatrix}
= \begin{bmatrix} 3 & 6 \\ 4 & 7 \end{bmatrix}
\]

Therefore, the resulting matrix is:
\[
\boxed{\begin{bmatrix} 3 & 6 \\ 4 & 7 \end{bmatrix}}
\]
This matrix represents the final recorded hours for each instrument and day after the calculations.

\section*{Question 5}
\textbf{Metadata}

\begin{itemize}
  \item Question ID: P6-PcFndWN\_P1-WNSub2nd\_GPT4.1\_Recreation\_01
  \item Primary KC: PERCENTAGE | Finding the whole | finding the whole given a part and the percentage
  \item Secondary KC: WHOLE NUMBERS | Subtraction | subtracting whole numbers
  \item Topic: Recreation such as sports, games, exercises, music, movie, dancing, painting, fishing and other recreation activities
  \item Grade: Primary 6
\end{itemize}

\textbf{Solution}

Let the initial number of participants be $x$.

$20\%$ of participants played basketball, so the number of basketball players is $0.2x$.

After $12$ students left (none were basketball players), the remaining number of participants is $x - 12$.

Now, the basketball players make up $25\%$ of the remaining group:

$0.2x = 25\% \times (x - 12)$

$0.2x = 0.25(x - 12)$

$0.2x = 0.25x - 3$

$0.25x - 0.2x = 3$

$0.05x = 3$

$x = \frac{3}{0.05} = 60$

\textbf{Therefore, there were initially $60$ students at the sports event.}

\section*{Question 6}
\textbf{Metadata}

\begin{itemize}
  \item Question ID: P4-FrSubU12\_P2-FrCmp\_GPT4.1\_Recreation\_01
  \item Primary KC: FRACTIONS | Subtraction | subtracting unlike fractions with two different denominators not exceeding 12
  \item Secondary KC: FRACTIONS | Comparison and ordering | comparing and ordering fractions
  \item Topic: Recreation such as sports, games, exercises, music, movie, dancing, painting, fishing and other recreation activities
  \item Grade: Primary 4
\end{itemize}

\textbf{Solution}

(a) To find out how much longer Sarah spent practicing piano, subtract $\dfrac{5}{12}$ from $\dfrac{3}{4}$.

First, find a common denominator. The denominators are $4$ and $12$. The lowest common denominator is $12$.

$\dfrac{3}{4} = \dfrac{3 \times 3}{4 \times 3} = \dfrac{9}{12}$

Now subtract:
$\dfrac{9}{12} - \dfrac{5}{12} = \dfrac{9 - 5}{12} = \dfrac{4}{12} = \dfrac{1}{3}$

Sarah spent $\dfrac{1}{3}$ hour more practicing piano than painting.

(b) Comparing $\dfrac{3}{4}$ hour (practicing piano) and $\dfrac{5}{12}$ hour (painting):

Convert $\dfrac{3}{4}$ to twelfths:
$\dfrac{3}{4} = \dfrac{9}{12}$

Since $\dfrac{9}{12} > \dfrac{5}{12}$, Sarah spent more time practicing piano.

Order from shortest to longest:

Painting ($\dfrac{5}{12}$ hour), then Practicing Piano ($\dfrac{3}{4}$ hour).

\section*{Question 7}
\textbf{Metadata}

\begin{itemize}
  \item Question ID: P6-AgSlvLrN\_P6-AgRepLrEx\_GPT4.1\_Recreation\_01
  \item Primary KC: ALGEBRA | Solving simple linear equations | solving linear equations involving whole number coefficient and one variable only
  \item Secondary KC: ALGEBRA | Representation and concept | translation of real-world situations into linear algebraic expressions
  \item Topic: Recreation such as sports, games, exercises, music, movie, dancing, painting, fishing and other recreation activities
  \item Grade: Primary 6
\end{itemize}

\textbf{Solution}

Let $n$ be the number of tickets Jason buys. Each ticket costs $\textdollar12$, so the total cost is $12n$. Jason spends $\textdollar60$ on tickets. 

So, we write the equation:
$$
12n = 60
$$
Divide both sides by 12:
$$
n = \frac{60}{12}
$$
$$
n = 5
$$
Therefore, Jason buys $5$ tickets.

\section*{Question 8}
\textbf{Metadata}

\begin{itemize}
  \item Question ID: P5-RtFndR\_P2-DcCnvD2N\_GPT4.1\_Recreation\_01
  \item Primary KC: RATE | Finding rate | finding rate given total amount and number of units
  \item Secondary KC: DECIMALS | Conversion to smaller units | converting a measurement from a larger unit in decimal form to a smaller unit
  \item Topic: Recreation such as sports, games, exercises, music, movie, dancing, painting, fishing and other recreation activities
  \item Grade: Primary 5
\end{itemize}

\textbf{Solution}

(a) The total number of hours played is $6.4$ hours. The piano was played over $8$ days.

The number of hours played each day $= \dfrac{6.4}{8} = 0.8$ hours.

(b) $1$ hour $= 60$ minutes.

$0.8$ hours $= 0.8 \times 60 = 48$ minutes.

\textbf{Answer:}
(a) $0.8$ hours per day.
(b) $48$ minutes per day.

\section*{Question 9}
\textbf{Metadata}

\begin{itemize}
  \item Question ID: P5-FrMulPIm\_P5-FrCnv2Dc\_GPT4.1\_Recreation\_01
  \item Primary KC: FRACTIONS | Multiplication | multiplying a proper fraction and a proper/improper fractions
  \item Secondary KC: FRACTIONS | Conversion to decimals | expressing fractions as decimals
  \item Topic: Recreation such as sports, games, exercises, music, movie, dancing, painting, fishing and other recreation activities
  \item Grade: Primary 5
\end{itemize}

\textbf{Solution}

Let the amount of paint in the bottle be 1.

Sarah used $\frac{3}{4}$ of the bottle for the background, so paint remaining:

$1 - \frac{3}{4} = \frac{1}{4}$

Amount used for details:

$\frac{5}{6}$ of $\frac{1}{4} = \frac{5}{6} \times \frac{1}{4} = \frac{5}{24}$

(a) Sarah used $\frac{5}{24}$ of the whole bottle for the details.

(b) To express $\frac{5}{24}$ as a decimal:

$\frac{5}{24} = 0.2083$ (to 4 decimal places)

Correct to 2 decimal places: $0.21$

So, $\frac{5}{24} \approx 0.21$ (correct to 2 decimal places).

\section*{Question 10}
\textbf{Metadata}

\begin{itemize}
  \item Question ID: O3-SPFndQtl\_O3-SPFndIQR\_GPT4.1\_Recreation\_01
  \item Primary KC: STATISTICS AND PROBABILITY | Finding quartiles | finding quartiles for a set of data
  \item Secondary KC: STATISTICS AND PROBABILITY | Finding range | finding interquartile range as measures of spread for a set of data 
  \item Topic: Recreation such as sports, games, exercises, music, movie, dancing, painting, fishing and other recreation activities
  \item Grade: Secondary O-level 3/4
\end{itemize}

\textbf{Solution}

First, arrange the data in ascending order (already done): $30, 40, 45, 50, 55, 60, 70, 75, 80$.

There are $n = 9$ data values.

(a) **Finding Quartiles:**

- The median ($Q_2$) is the middle value, which is the $5^{\text{th}}$ value: $55$.
- To find $Q_1$, consider the lower half: $30, 40, 45, 50$ (first 4 values). $Q_1$ is the median of these 4 values: average of $40$ and $45$. 

\[
Q_1 = \frac{40 + 45}{2} = 42.5
\]

- To find $Q_3$, take the upper half: $60, 70, 75, 80$. $Q_3$ is the median of these 4 values: average of $70$ and $75$.

\[
Q_3 = \frac{70 + 75}{2} = 72.5
\]

Quartiles are:
- $Q_1 = 42.5$
- $Q_2 = 55$
- $Q_3 = 72.5$

(b) **Interquartile Range:**

\[
\text{Interquartile range} = Q_3 - Q_1 = 72.5 - 42.5 = 30
\]

**Answer:**
- The lower quartile is $42.5$ minutes, the median is $55$ minutes, and the upper quartile is $72.5$ minutes.
- The interquartile range is $30$ minutes.

\section*{Question 11}
\textbf{Metadata}

\begin{itemize}
  \item Question ID: O3-SPMulProb\_O2-SPRepPrSE\_GPT4.1\_Recreation\_01
  \item Primary KC: STATISTICS AND PROBABILITY | Multiplication | multiplication of probabilities
  \item Secondary KC: STATISTICS AND PROBABILITY | Representation and concept | probability of single events
  \item Topic: Recreation such as sports, games, exercises, music, movie, dancing, painting, fishing and other recreation activities
  \item Grade: Secondary O-level 3/4
\end{itemize}

\textbf{Solution}

(a) Since the two events are independent, the probability that both Sarah scores a basket and Amir wins his match is given by the multiplication of their individual probabilities:

\[
P(\text{Sarah scores and Amir wins}) = P(\text{Sarah scores}) \times P(\text{Amir wins}) = 0.8 \times 0.5 = 0.4
\]

So, the probability is $0.4$.

(b) The probability that only one of them succeeds means:

- Sarah scores and Amir does not win, OR
- Sarah does not score and Amir wins.

First, calculate the probability for each scenario:

Sarah scores and Amir does not win:
\[
P(\text{Sarah scores}) \times P(\text{Amir does not win}) = 0.8 \times (1-0.5) = 0.8 \times 0.5 = 0.4
\]

Sarah does not score and Amir wins:
\[
P(\text{Sarah does not score}) \times P(\text{Amir wins}) = (1-0.8) \times 0.5 = 0.2 \times 0.5 = 0.1
\]

Adding these probabilities together:
\[
P(\text{Only one succeeds}) = 0.4 + 0.1 = 0.5
\]

So, the probability that only one of them succeeds in their event is $0.5$.

\section*{Question 12}
\textbf{Metadata}

\begin{itemize}
  \item Question ID: O1-RoRepFr\_P6-FrDiv2nd\_GPT4.1\_Recreation\_01
  \item Primary KC: RATIO | Representation and concept | ratios involving fractions
  \item Secondary KC: FRACTIONS | Division | fraction division
  \item Topic: Recreation such as sports, games, exercises, music, movie, dancing, painting, fishing and other recreation activities
  \item Grade: Secondary O-level 1
\end{itemize}

\textbf{Solution}

Let the number of students who chose painting be $p$ and those who chose dancing be $d$. From the problem, the ratio of students who chose painting to dancing is $\frac{3}{4} : 2$. This means:

\[ \frac{p}{d} = \frac{\frac{3}{4}}{2} = \frac{3}{4} \times \frac{1}{2} = \frac{3}{8} \]

Alternatively, we are also told that there are $\frac{5}{6}$ as many students in painting as in dancing:
\[ p = \frac{5}{6}d \]
But we are told $d = 24$.

Step 1: Substitute $d = 24$ into $p = \frac{5}{6}d$
\[ p = \frac{5}{6} \times 24 = 5 \times 4 = 20 \]

Step 2: Check that the given ratio holds with the numbers:
\[ \frac{p}{d} = \frac{20}{24} = \frac{5}{6} \]
But the stated ratio at the start was $\frac{3}{8}$, so let's check:

Alternatively, since ratio $p$ to $d$ is $\frac{3}{4} : 2 = \frac{3}{4} \div 2 = \frac{3}{8}$.
So the number of students who chose painting is:
\[p = \frac{3}{8}d = \frac{3}{8} \times 24 = 9\]
But we also have $p = \frac{5}{6}d = 20$.

The two constraints cannot be simultaneously true unless the two ratios are the same. Let's rephrase with the correct logical link:

#### Corrected Problem Statement
During a holiday camp, a group of students participated in two activities: painting and dancing. The ratio of the number of students who chose painting to those who chose dancing was $\frac{3}{4} : 2$. If $24$ students participated in dancing, how many students participated in painting?

#### Solution
Let the number of students who chose painting be $p$.

Given ratio: $\dfrac{p}{24} = \dfrac{3}{4} \div 2 = \dfrac{3}{8}$.
\[p = \frac{3}{8} \times 24 = 9\]
So, \textbf{9 students participated in painting}.

#### If required to both use ratios involving fractions and fraction division (synergy):
Suppose the number of students who participated in painting is $\frac{3}{4}$ times the number who participated in dancing. If there were $24$ students who participated in dancing, how many participated in painting?

\[\text{Number in painting} = \frac{3}{4} \times 24 = 18\]

Therefore, $18$ students participated in painting.

\section*{Question 13}
\textbf{Metadata}

\begin{itemize}
  \item Question ID: O2-RoRepIvP\_P1-WNDiv2nd\_GPT4.1\_Recreation\_01
  \item Primary KC: RATIO | Representation and concept | inverse proportion
  \item Secondary KC: WHOLE NUMBERS | Division | dividing whole numbers
  \item Topic: Recreation such as sports, games, exercises, music, movie, dancing, painting, fishing and other recreation activities
  \item Grade: Secondary O-level 2
\end{itemize}

\textbf{Solution}

(a) Since the time taken to complete the game is inversely proportional to the number of players, we let $t$ be the time taken and $n$ be the number of players:

$t \propto \frac{1}{n}$

So, $t_1 n_1 = t_2 n_2$

Given $t_1 = 120$ min, $n_1 = 4$, $n_2 = 6$, so:

$120 \times 4 = t_2 \times 6$

$480 = 6 t_2$

$t_2 = \frac{480}{6} = 80$

So, it will take $80$ minutes for $6$ players to complete the game.

(b) The cost of renting the board game is \textdollar36, divided equally among $6$ players:

Amount each player pays $= \frac{\textdollar36}{6} = \textdollar6$

Each player needs to pay \textdollar6.

\section*{Question 14}
\textbf{Metadata}

\begin{itemize}
  \item Question ID: P3-WNDiv3d1d\_P1-WNAdd2nd\_GPT4.1\_Recreation\_01
  \item Primary KC: WHOLE NUMBERS | Division | dividing whole numbers up to 3 digits by 1 digit
  \item Secondary KC: WHOLE NUMBERS | Addition | adding whole numbers
  \item Topic: Recreation such as sports, games, exercises, music, movie, dancing, painting, fishing and other recreation activities
  \item Grade: Primary 3
\end{itemize}

\textbf{Solution}

First, divide the original 168 children by 4: 

$168 \div 4 = 42$

So, each group had 42 children at first.

When 15 more children join, the total number of children becomes:

$168 + 15 = 183$

Now, divide 183 children into 4 equal groups:

$183 \div 4 = 45$ remainder $3$

Each group will have 45 children, and there will be 3 children left over. So, if the organiser wants all groups to have the same number of children, each group will have 45 children, and 3 children will not be assigned to any group.

\section*{Question 15}
\textbf{Metadata}

\begin{itemize}
  \item Question ID: O2-AgSlvLr2v\_O1-AgRepEq\_GPT4.1\_Recreation\_01
  \item Primary KC: ALGEBRA | Solving | solving linear equations in two variables
  \item Secondary KC: ALGEBRA | Representation and concept | translation of simple real-world situations to equations
  \item Topic: Recreation such as sports, games, exercises, music, movie, dancing, painting, fishing and other recreation activities
  \item Grade: Secondary O-level 2
\end{itemize}

\textbf{Solution}

Let $x$ be the number of hours the court was rented, and $y$ be the number of rackets rented. 

The total amount spent is:
\[ \textdollar 4 \times x + \textdollar 2 \times y = \textdollar 28 \]

If $x = 3$ and $y = 5$,
\[ \textdollar 4 \times 3 + \textdollar 2 \times 5 = \textdollar 12 + \textdollar 10 = \textdollar 22 \]

So, if they played for 3 hours and rented 5 rackets, they spent \textdollar22 in total.

Equation representing the situation: 
\[ 4x + 2y = 28 \]
Substitute $x=3$, $y=5$:
\[ 4(3) + 2(5) = 12 + 10 = 22 \]

Hence, they spent \textdollar22.

\section*{Question 16}
\textbf{Metadata}

\begin{itemize}
  \item Question ID: P5-FrMulMixN\_P5-FrCnv2Dc\_GPT4.1\_Recreation\_01
  \item Primary KC: FRACTIONS | Multiplication | multiplying a mixed number and a whole number
  \item Secondary KC: FRACTIONS | Conversion to decimals | expressing fractions as decimals
  \item Topic: Recreation such as sports, games, exercises, music, movie, dancing, painting, fishing and other recreation activities
  \item Grade: Primary 5
\end{itemize}

\textbf{Solution}

(a) Alicia attends 4 sessions, each lasting $1\frac{1}{2}$ hours.

Let us convert the mixed number to an improper fraction:

$1\frac{1}{2} = \frac{3}{2}$

Now, multiply by 4:

$\frac{3}{2} \times 4 = \frac{3 \times 4}{2} = \frac{12}{2} = 6$

So, Alicia will spend $6$ hours in dance classes this month.

(b) Express $6$ hours as a decimal:

$6$ hours is already in decimal form.

\textbf{Final Answers:}

(a) $6$ hours

(b) $6$

\section*{Question 17}
\textbf{Metadata}

\begin{itemize}
  \item Question ID: P3-WNDiv3d1d\_P1-WNMul2nd\_GPT4.1\_Recreation\_01
  \item Primary KC: WHOLE NUMBERS | Division | dividing whole numbers up to 3 digits by 1 digit
  \item Secondary KC: WHOLE NUMBERS | Multiplication | multiplying whole numbers
  \item Topic: Recreation such as sports, games, exercises, music, movie, dancing, painting, fishing and other recreation activities
  \item Grade: Primary 3
\end{itemize}

\textbf{Solution}

First, divide $168$ stickers among $4$ students:

$$168 \div 4 = 42$$

Each student receives $42$ stickers.

Next, find out how many sheets are needed for $42$ stickers if each sheet holds $6$ stickers:

$$42 \div 6 = 7$$

So each student needs $7$ sheets to hold all their stickers.

\section*{Question 18}
\textbf{Metadata}

\begin{itemize}
  \item Question ID: P6-PcFndChg\_P1-WNSub2nd\_GPT4.1\_Recreation\_01
  \item Primary KC: PERCENTAGE | Finding change | finding percentage increase/decrease
  \item Secondary KC: WHOLE NUMBERS | Subtraction | subtracting whole numbers
  \item Topic: Recreation such as sports, games, exercises, music, movie, dancing, painting, fishing and other recreation activities
  \item Grade: Primary 6
\end{itemize}

\textbf{Solution}

(a) Number of hours played in the second week: $18$

Number of hours played in the first week: $12$

Difference in hours: $18 - 12 = 6$

Amir played $6$ more hours in the second week than the first week.

(b) Percentage increase $= \frac{\text{Increase}}{\text{Original amount}} \times 100\%$

Increase in hours $= 6$

Original amount (first week) $= 12$

Percentage increase $= \frac{6}{12} \times 100\% = 0.5 \times 100\% = 50\%$

The number of hours Amir played football increased by $50\%$ from the first week to the second week.

\section*{Question 19}
\textbf{Metadata}

\begin{itemize}
  \item Question ID: P3-FrAddRl12\_P2-FrCmp\_GPT4.1\_Recreation\_01
  \item Primary KC: FRACTIONS | Addition | adding two related fractions within one whole with denominators of given fractions not exceeding 12
  \item Secondary KC: FRACTIONS | Comparison and ordering | comparing and ordering fractions
  \item Topic: Recreation such as sports, games, exercises, music, movie, dancing, painting, fishing and other recreation activities
  \item Grade: Primary 3
\end{itemize}

\textbf{Solution}

(a) Amy spent $\dfrac{3}{8}$ hour on the piano and $\dfrac{1}{8}$ hour singing.

To find the total time she spent:
\\[
\dfrac{3}{8} + \dfrac{1}{8} = \dfrac{3+1}{8} = \dfrac{4}{8} = \dfrac{1}{2} \text{ hour.}
\\]

So, Amy spent $\dfrac{1}{2}$ hour practising piano and singing altogether.

(b) Ben practised for $\dfrac{1}{2}$ hour.

Comparing Amy’s and Ben’s total time:

Amy: $\dfrac{1}{2}$ hour  
Ben: $\dfrac{1}{2}$ hour

Both spent the same amount of time practising music.

\section*{Question 20}
\textbf{Metadata}

\begin{itemize}
  \item Question ID: P4-DcDiv2d1d\_P4-DcSub2nd\_GPT4.1\_Recreation\_01
  \item Primary KC: DECIMALS | Division | dividing decimals (up to 2 decimal places) by a 1-digit whole number
  \item Secondary KC: DECIMALS | Subtraction | subtracting decimals
  \item Topic: Recreation such as sports, games, exercises, music, movie, dancing, painting, fishing and other recreation activities
  \item Grade: Primary 4
\end{itemize}

\textbf{Solution}

First, let's find out how much Shawn pays each week by dividing the total fee by the number of weeks:

\[
\text{Fee per week} = \frac{36.84}{6} = 6.14
\]

After 4 weeks, Shawn has already paid:

\[
\text{Amount paid for 4 weeks} = 4 \times 6.14 = 24.56
\]

To find out how much more Shawn needs to pay, subtract what he has already paid from the total fee:

\[
\text{Amount left to pay} = 36.84 - 24.56 = 12.28
\]

So, Shawn has to pay \textdollar12.28 for the remaining 2 weeks.

\section*{Question 21}
\textbf{Metadata}

\begin{itemize}
  \item Question ID: P5-RtFndR\_P2-DcCnvN2D\_GPT4.1\_Recreation\_01
  \item Primary KC: RATE | Finding rate | finding rate given total amount and number of units
  \item Secondary KC: DECIMALS | Conversion to larger units | converting a measurement from a smaller unit to a larger unit in decimal form
  \item Topic: Recreation such as sports, games, exercises, music, movie, dancing, painting, fishing and other recreation activities
  \item Grade: Primary 5
\end{itemize}

\textbf{Solution}

(a) To find the average number of minutes Emily practised each day, divide the total time by the number of days:

\[
\text{Average minutes per day} = \frac{150}{5} = 30 \text{ minutes}
\]

(b) To convert $30$ minutes to hours:

There are $60$ minutes in $1$ hour.

\[
\text{Hours per day} = \frac{30}{60} = 0.5 \text{ hours}
\]

So, the average practice time per day is $0.5$ hours.

\section*{Question 22}
\textbf{Metadata}

\begin{itemize}
  \item Question ID: P6-PcFndChg\_P1-WNMul2nd\_GPT4.1\_Recreation\_01
  \item Primary KC: PERCENTAGE | Finding change | finding percentage increase/decrease
  \item Secondary KC: WHOLE NUMBERS | Multiplication | multiplying whole numbers
  \item Topic: Recreation such as sports, games, exercises, music, movie, dancing, painting, fishing and other recreation activities
  \item Grade: Primary 6
\end{itemize}

\textbf{Solution}

First, we find the percentage increase in the price of one tennis racket:

The amount of increase $= 20\% \times 55 = \frac{20}{100} \times 55 = 11$

New price of one racket $= 55 + 11 = \textdollar66$

Total amount for 8 rackets $= 66 \times 8 = \textdollar528$

Therefore, the school needs to pay \textdollar528 for 8 tennis rackets at the new price.

\section*{Question 23}
\textbf{Metadata}

\begin{itemize}
  \item Question ID: O3-STOprUn\_O3-STOprIns\_GPT4.1\_Recreation\_01
  \item Primary KC: SET | Set operations | union of two sets
  \item Secondary KC: SET | Set operations | intersection of two sets
  \item Topic: Recreation such as sports, games, exercises, music, movie, dancing, painting, fishing and other recreation activities
  \item Grade: Secondary O-level 3/4
\end{itemize}

\textbf{Solution}

(a) Let $S$ represent the set of students in the Science Club and $M$ the set of students in the Music Club. The information can be expressed as:
\[ |S| = 38,\quad |M| = 26,\quad |S \cap M| = 14. \]

(b) The number of students who joined either the Science Club or the Music Club or both is the number of students in the union of the two sets:
\[ |S \cup M| = |S| + |M| - |S \cap M|. \]
Substitute the given values:
\[ |S \cup M| = 38 + 26 - 14 = 50. \]
Therefore, $50$ students joined either the Science Club or the Music Club or both.

\section*{Question 24}
\textbf{Metadata}

\begin{itemize}
  \item Question ID: P3-WNDivRmd3d\_P1-WNAdd2nd\_GPT4.1\_Recreation\_01
  \item Primary KC: WHOLE NUMBERS | Division | dividing whole numbers up to 3 digits by 1 digit with remainder 
  \item Secondary KC: WHOLE NUMBERS | Addition | adding whole numbers
  \item Topic: Recreation such as sports, games, exercises, music, movie, dancing, painting, fishing and other recreation activities
  \item Grade: Primary 3
\end{itemize}

\textbf{Solution}

First, we divide the students among the bowling lanes:

Number of students: $125$
Number of students per lane: $6$

$125 \div 6 = 20$ lanes with a remainder of $5$ students.

So, $6 \times 20 = 120$ students are in the bowling lanes, and $125 - 120 = 5$ students do not have a lane and will play table tennis instead.

After that, $9$ more students join for table tennis:

Number of table tennis players = $5 + 9 = 14$

\textbf{Answer:} $14$ students are playing table tennis.

\section*{Question 25}
\textbf{Metadata}

\begin{itemize}
  \item Question ID: O2-SPFndmdn\_O2-SPFndmode\_GPT4.1\_Recreation\_01
  \item Primary KC: STATISTICS AND PROBABILITY | Finding median | Finding median for a set of data
  \item Secondary KC: STATISTICS AND PROBABILITY | Finding mode | Finding mode for a set of data
  \item Topic: Recreation such as sports, games, exercises, music, movie, dancing, painting, fishing and other recreation activities
  \item Grade: Secondary O-level 2
\end{itemize}

\textbf{Solution}

(a) To find the median, first arrange the data in order:

$2,\ 3,\ 3,\ 3,\ 4,\ 5,\ 5,\ 6,\ 7$

Since there are 9 data points (an odd number), the median is the value at position $\frac{9+1}{2} = 5$.

So, the median number of matches played is $4$.

(b) The mode is the number that occurs most frequently.

In the list $2,\ 3,\ 3,\ 3,\ 4,\ 5,\ 5,\ 6,\ 7$, the number $3$ appears $3$ times, which is more frequent than any other number.

Therefore, the mode of the number of matches played is $3$.

\end{document}
