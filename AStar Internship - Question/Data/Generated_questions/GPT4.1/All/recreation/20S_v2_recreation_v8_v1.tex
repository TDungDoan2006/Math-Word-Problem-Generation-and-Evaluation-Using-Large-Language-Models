\documentclass{article}
\usepackage[utf8]{inputenc}
\usepackage{amsmath}
\usepackage{amsfonts}
\usepackage{amssymb}
\usepackage{graphicx}
\usepackage{hyperref}
\title{recreation\_v8\_v1}
\author{Tien Dung Doan}
\begin{document}
\maketitle
\section*{Question 1}
\textbf{Metadata}

\begin{itemize}
  \item Question ID: P4-WNDiv4d1d\_P4-WNRnd5d\_GPT4.1\_Recreation\_01
  \item Primary KC: WHOLE NUMBERS | Division | dividing whole numbers up to 4 digits by 1 digit
  \item Secondary KC: WHOLE NUMBERS | Rounding | rounding whole numbers up to 100000 to the nearest 10, 100 or 1000 
  \item Topic: Recreation such as sports, games, exercises, music, movie, dancing, painting, fishing and other recreation activities
  \item Grade: Primary 4
\end{itemize}

\textbf{Solution}

First, divide $4362$ by $7$:

\[
4362 \div 7 = 623.142857... 
\]

Each player gets $623$ points, with some left over.

Next, round $623$ to the nearest $10$:

The last digit is $3$ (less than $5$), so we round down to $620$.

\textbf{Answer:}
To the nearest $10$, each player will get $\textdollar620$ points.

\section*{Question 2}
\textbf{Metadata}

\begin{itemize}
  \item Question ID: P6-RoFndRoWN\_P1-WNDiv2nd\_GPT4.1\_Recreation\_01
  \item Primary KC: RATIO | Finding ratio | finding the ratio of two or three given whole numbers
  \item Secondary KC: WHOLE NUMBERS | Division | dividing whole numbers
  \item Topic: Recreation such as sports, games, exercises, music, movie, dancing, painting, fishing and other recreation activities
  \item Grade: Primary 6
\end{itemize}

\textbf{Solution}

(a) The number of boys is $48$ and the number of girls is $72$. 

To find the ratio of boys to girls:

$\text{Ratio of boys to girls} = 48 : 72$

We simplify the ratio by dividing both numbers by their greatest common divisor, which is $24$:

$48 \div 24 = 2$
$72 \div 24 = 3$

So, the simplest ratio is $2 : 3$.

(b) Total number of participants:
$48 + 72 = 120$

The participants are divided into $6$ equal groups, so the number in each group is:
$120 \div 6 = 20$

\textbf{Answer:} 
(a) The ratio of boys to girls is $2:3$.
(b) There are $20$ participants in each group for the photo.

\section*{Question 3}
\textbf{Metadata}

\begin{itemize}
  \item Question ID: P6-RoFndRoWN\_P1-WNAdd2nd\_GPT4.1\_Recreation\_01
  \item Primary KC: RATIO | Finding ratio | finding the ratio of two or three given whole numbers
  \item Secondary KC: WHOLE NUMBERS | Addition | adding whole numbers
  \item Topic: Recreation such as sports, games, exercises, music, movie, dancing, painting, fishing and other recreation activities
  \item Grade: Primary 6
\end{itemize}

\textbf{Solution}

(a) The total number of guitar lesson students after new students joined:

$18 + 6 = 24$

So, there are 24 students in guitar lessons.

(b) The number of students in piano lessons is 12. The new ratio is $24:12$.

To simplify, divide both numbers by their greatest common divisor, which is 12:

$24 \div 12 = 2$

$12 \div 12 = 1$

So, the simplest ratio is $2:1$.

\textbf{Answer:}

(a) 24 students signed up for guitar lessons.

(b) The ratio is $2:1$.

\section*{Question 4}
\textbf{Metadata}

\begin{itemize}
  \item Question ID: P5-FrMulImIm\_P3-FrSmp\_GPT4.1\_Recreation\_01
  \item Primary KC: FRACTIONS | Multiplication | multiplying two improper fractions
  \item Secondary KC: FRACTIONS | Simplifying | expressing a fraction in its simplest form
  \item Topic: Recreation such as sports, games, exercises, music, movie, dancing, painting, fishing and other recreation activities
  \item Grade: Primary 5
\end{itemize}

\textbf{Solution}

Each student practises $\frac{7}{4}$ hours each day, and there are $\frac{9}{5}$ times as many days this month. To find the total hours each student spends:

Total hours $= \frac{7}{4} \times \frac{9}{5}$

First, multiply the numerators and denominators:

$\frac{7}{4} \times \frac{9}{5} = \frac{7 \times 9}{4 \times 5} = \frac{63}{20}$

Next, check if $\frac{63}{20}$ can be simplified. The greatest common divisor of 63 and 20 is 1, so it is already in its simplest form.

**Final Answer:** Each student spends $\frac{63}{20}$ hours practising in total this month.

\section*{Question 5}
\textbf{Metadata}

\begin{itemize}
  \item Question ID: O1-PcRep2q\_O1-PcCnv2Dc\_GPT4.1\_Recreation\_01
  \item Primary KC: PERCENTAGE | Representation and concept | comparing two quantities by percentage
  \item Secondary KC: PERCENTAGE | Conversion to decimals | expressing percentage as a decimal
  \item Topic: Recreation such as sports, games, exercises, music, movie, dancing, painting, fishing and other recreation activities
  \item Grade: Secondary O-level 1
\end{itemize}

\textbf{Solution}

(a) Sarah won $15$ out of $20$ matches.

The percentage of matches she won is:
\[
\frac{15}{20} \times 100\% = 75\%
\]

(b) To express this percentage as a decimal, divide $75\%$ by $100$:
\[
75\% = 0.75
\]

**Answer:**

(a) Sarah won $75\%$ of her matches.

(b) The percentage $75\%$ as a decimal is $0.75$.

\section*{Question 6}
\textbf{Metadata}

\begin{itemize}
  \item Question ID: P4-FrRepSet\_P3-FrCnvEq\_GPT4.1\_Recreation\_01
  \item Primary KC: FRACTIONS | Representation and concept | expressing a part of a set as a fraction
  \item Secondary KC: FRACTIONS | Conversion to equivalent fractions | Conversion to equivalent fractions (given either the denominator or the numerator)
  \item Topic: Recreation such as sports, games, exercises, music, movie, dancing, painting, fishing and other recreation activities
  \item Grade: Primary 4
\end{itemize}

\textbf{Solution}

(a) The fraction of the relay race team that are girls is $\dfrac{9}{24}$, since there are $9$ girls out of $24$ students.

(b) To write $\dfrac{9}{24}$ as an equivalent fraction with denominator $8$, we need to find a number $x$ such that $\dfrac{9}{24} = \dfrac{x}{8}$. 

Set up the equation:

$\dfrac{9}{24} = \dfrac{x}{8}$

Cross-multiply:

$9 \times 8 = 24 \times x$

$72 = 24x$

$x = \dfrac{72}{24} = 3$

So, an equivalent fraction is $\dfrac{3}{8}$.

Therefore, $\dfrac{9}{24} = \dfrac{3}{8}$.

\section*{Question 7}
\textbf{Metadata}

\begin{itemize}
  \item Question ID: P3-WNDivRmd3d\_P1-WNCmp\_GPT4.1\_Recreation\_01
  \item Primary KC: WHOLE NUMBERS | Division | dividing whole numbers up to 3 digits by 1 digit with remainder 
  \item Secondary KC: WHOLE NUMBERS | Comparison and ordering | comparing and ordering whole numbers
  \item Topic: Recreation such as sports, games, exercises, music, movie, dancing, painting, fishing and other recreation activities
  \item Grade: Primary 3
\end{itemize}

\textbf{Solution}

(a) To find out how many tables can be filled:

$127 \div 4 = 31$ remainder $3$.

So, 31 tables will be fully occupied, and $3$ children will not get a seat at a full table.

(b) For the checkers room:

$95 \div 4 = 23$ remainder $3$.

So, 23 tables will be fully occupied, and $3$ children will not get a seat at a full table.

Comparison:
$3$ children in the chess game and $3$ children in the checkers game are left without a seat at a full table. Thus, the number is the same, and neither group has more children without a seat. The difference is $3 - 3 = 0$.

\textbf{Answer:} Both games have the same number of children ($3$) left without a seat at a full table.

\section*{Question 8}
\textbf{Metadata}

\begin{itemize}
  \item Question ID: P6-RoFndDvqWN\_P1-WNSub2nd\_GPT4.1\_Recreation\_01
  \item Primary KC: RATIO | Finding divided quantities | dividing a given quantity in a given ratio
  \item Secondary KC: WHOLE NUMBERS | Subtraction | subtracting whole numbers
  \item Topic: Recreation such as sports, games, exercises, music, movie, dancing, painting, fishing and other recreation activities
  \item Grade: Primary 6
\end{itemize}

\textbf{Solution}

(a) First, find the total number of parts in the ratio:

$3 + 2 + 1 = 6$ parts

Each part is worth $\frac{120}{6} = 20$ points.

Siti's share: $3 \times 20 = 60$ points
Mei's share: $2 \times 20 = 40$ points
Raj's share: $1 \times 20 = 20$ points

(b) Siti spent \textdollar20, so she subtracts this from her share:

$60 - 20 = 40$

Siti had $40$ points left after buying the movie ticket.

\section*{Question 9}
\textbf{Metadata}

\begin{itemize}
  \item Question ID: P4-DcSub2d\_P4-DcCmp3d\_GPT4.1\_Recreation\_01
  \item Primary KC: DECIMALS | Subtraction | subtracting decimals (up to 2 decimal places)
  \item Secondary KC: DECIMALS | Comparison and ordering | comparing and ordering decimals up to 3 decimal places
  \item Topic: Recreation such as sports, games, exercises, music, movie, dancing, painting, fishing and other recreation activities
  \item Grade: Primary 4
\end{itemize}

\textbf{Solution}

(a) To find out how many more points Lina scored in the first round than in the second round, we subtract the second round score from the first round score:

$7.85 - 6.43 = 1.42$

So, Lina scored $1.42$ more points in the first round than in the second round.

(b) To arrange Lina's scores from lowest to highest, we compare $7.85$, $6.43$, and $7.802$.

First, compare $6.43$, $7.85$, and $7.802$. Since $6.43 < 7.802 < 7.85$,

the order from lowest to highest is:

$6.43$, $7.802$, $7.85$.

\section*{Question 10}
\textbf{Metadata}

\begin{itemize}
  \item Question ID: P5-DcMul3dK\_P4-DcRnd3d\_GPT4.1\_Recreation\_01
  \item Primary KC: DECIMALS | Multiplication | multiplying decimals (up to 3 decimal places) by 10, 100, 1000 and their multiples
  \item Secondary KC: DECIMALS | Rounding | rounding decimals up to 3 decimal places to the nearest whole number, 1 decimal place and 2 decimal places 
  \item Topic: Recreation such as sports, games, exercises, music, movie, dancing, painting, fishing and other recreation activities
  \item Grade: Primary 5
\end{itemize}

\textbf{Solution}

(a) To find the total minutes Amira spent, multiply the duration of one song by the number of songs:

$3.275 \times 20 = 65.5$

Therefore, Amira practised for $65.5$ minutes on Saturday.

(b) Rounding $65.5$ to the nearest whole number:

Since $0.5$ rounds up, $65.5$ rounds to $66$.

Amira spent $\textdollar66$ minutes practising, rounded to the nearest whole number.

(c) Rounding $65.5$ to $1$ decimal place:

There are no more digits after the $0.5$, so it stays as $65.5$.

Amira spent $65.5$ minutes practising, rounded to $1$ decimal place.

\section*{Question 11}
\textbf{Metadata}

\begin{itemize}
  \item Question ID: P6-FrDivPN\_P5-FrMul2nd\_GPT4.1\_Recreation\_01
  \item Primary KC: FRACTIONS | Division | dividing a proper fraction by a whole number
  \item Secondary KC: FRACTIONS | Multiplication | fraction multiplication
  \item Topic: Recreation such as sports, games, exercises, music, movie, dancing, painting, fishing and other recreation activities
  \item Grade: Primary 6
\end{itemize}

\textbf{Solution}

(a) Each friend's share is $\dfrac{3}{4} \div 3 = \dfrac{3}{4} \times \dfrac{1}{3} = \dfrac{3}{12} = \dfrac{1}{4}$ of the chocolate bar.

(b) Anna eats $\dfrac{2}{3}$ of one friend's share. So, Anna eats $\dfrac{2}{3} \times \dfrac{1}{4} = \dfrac{2}{12} = \dfrac{1}{6}$ of the chocolate bar.

Final answers:
(a) Each friend gets $\dfrac{1}{4}$ of the chocolate bar.
(b) Anna eats $\dfrac{1}{6}$ of the chocolate bar.

\section*{Question 12}
\textbf{Metadata}

\begin{itemize}
  \item Question ID: O1-AgSlvFrLr\_O1-AgRepEq\_GPT4.1\_Recreation\_01
  \item Primary KC: ALGEBRA | Solving | solving simple fractional equations that can be reduced to linear equations
  \item Secondary KC: ALGEBRA | Representation and concept | translation of simple real-world situations to equations
  \item Topic: Recreation such as sports, games, exercises, music, movie, dancing, painting, fishing and other recreation activities
  \item Grade: Secondary O-level 1
\end{itemize}

\textbf{Solution}

Let the number Sophie thinks of be $x$.

According to the problem, adding $3$ to half of her number gives $11$:
\[
\frac{1}{2}x + 3 = 11
\]

Subtract $3$ from both sides:
\[
\frac{1}{2}x = 8
\]

Multiply both sides by $2$ to solve for $x$:
\[
x = 16
\]

Therefore, Sophie thought of the number $16$.

\section*{Question 13}
\textbf{Metadata}

\begin{itemize}
  \item Question ID: P6-FrDivPN\_P3-FrSmp\_GPT4.1\_Recreation\_01
  \item Primary KC: FRACTIONS | Division | dividing a proper fraction by a whole number
  \item Secondary KC: FRACTIONS | Simplifying | expressing a fraction in its simplest form
  \item Topic: Recreation such as sports, games, exercises, music, movie, dancing, painting, fishing and other recreation activities
  \item Grade: Primary 6
\end{itemize}

\textbf{Solution}

Each person gets $\dfrac{3}{4} \div 3$ of the pizza.\newline

$\dfrac{3}{4} \div 3 = \dfrac{3}{4} \times \dfrac{1}{3} = \dfrac{3 \times 1}{4 \times 3} = \dfrac{3}{12}$\newline

Now, we simplify $\dfrac{3}{12}$ by dividing both numerator and denominator by $3$:\newline

$\dfrac{3 \div 3}{12 \div 3} = \dfrac{1}{4}$\newline

\textbf{Each person gets $\dfrac{1}{4}$ of the pizza.}

\section*{Question 14}
\textbf{Metadata}

\begin{itemize}
  \item Question ID: P4-DcDiv2d1d\_P4-DcCmp3d\_GPT4.1\_Recreation\_01
  \item Primary KC: DECIMALS | Division | dividing decimals (up to 2 decimal places) by a 1-digit whole number
  \item Secondary KC: DECIMALS | Comparison and ordering | comparing and ordering decimals up to 3 decimal places
  \item Topic: Recreation such as sports, games, exercises, music, movie, dancing, painting, fishing and other recreation activities
  \item Grade: Primary 4
\end{itemize}

\textbf{Solution}

(a) For each activity, divide the total minutes by $3$.

- Singing: $7.68 \div 3 = 2.56$ minutes per day
- Piano: $3.276 \div 3 = 1.092$ minutes per day
- Drumming: $5.13 \div 3 = 1.71$ minutes per day

(b) To order the daily times:
- Piano: $1.092$ (shortest)
- Drumming: $1.71$
- Singing: $2.56$ (longest)

So, the order from shortest to longest is: Piano, Drumming, Singing.

\section*{Question 15}
\textbf{Metadata}

\begin{itemize}
  \item Question ID: P6-PcFndWN\_P1-WNMul2nd\_GPT4.1\_Recreation\_01
  \item Primary KC: PERCENTAGE | Finding the whole | finding the whole given a part and the percentage
  \item Secondary KC: WHOLE NUMBERS | Multiplication | multiplying whole numbers
  \item Topic: Recreation such as sports, games, exercises, music, movie, dancing, painting, fishing and other recreation activities
  \item Grade: Primary 6
\end{itemize}

\textbf{Solution}

Let the total possible number of points in the basketball challenge be $x$.

Sarah scored 24 points, which is 40\% of $x$.

So, $0.4x = 24$

To find $x$, divide both sides by $0.4$:

\[
x = \frac{24}{0.4} = 60
\]

So, the total possible number of points in the basketball challenge is $60$.

After the challenge, the total possible points were multiplied by $2$ for the next game:

\[
\text{Total possible points after two games} = 60 \times 2 = 120
\]

\textbf{Answer:}

- The total possible number of points Sarah could have scored in the basketball challenge was $60$.
- The total possible number of points she could have scored after both games was $120$.

\section*{Question 16}
\textbf{Metadata}

\begin{itemize}
  \item Question ID: P4-WNDiv4d1d\_P1-WNMul2nd\_GPT4.1\_Recreation\_01
  \item Primary KC: WHOLE NUMBERS | Division | dividing whole numbers up to 4 digits by 1 digit
  \item Secondary KC: WHOLE NUMBERS | Multiplication | multiplying whole numbers
  \item Topic: Recreation such as sports, games, exercises, music, movie, dancing, painting, fishing and other recreation activities
  \item Grade: Primary 4
\end{itemize}

\textbf{Solution}

(a) To find out how many classes can be formed, divide $864$ students by $6$ students per class:\\

\[
864 \div 6 = 144
\]

So, $144$ classes can be formed.\\

(b) For the total number of music stands needed, multiply the number of classes by $12$ stands per class:\\

\[
144 \times 12 = 1,728
\]

Thus, $1,728$ music stands are needed for all the classes.

\section*{Question 17}
\textbf{Metadata}

\begin{itemize}
  \item Question ID: P5-DcMul3dK\_P4-DcSub2nd\_GPT4.1\_Recreation\_01
  \item Primary KC: DECIMALS | Multiplication | multiplying decimals (up to 3 decimal places) by 10, 100, 1000 and their multiples
  \item Secondary KC: DECIMALS | Subtraction | subtracting decimals
  \item Topic: Recreation such as sports, games, exercises, music, movie, dancing, painting, fishing and other recreation activities
  \item Grade: Primary 5
\end{itemize}

\textbf{Solution}

First, calculate the actual points Siti earned:

\[
\text{Points per note} = 0.175
\]
\[
\text{Number of notes} = 100
\]
\[
\text{Total points} = 0.175 \times 100 = 17.5
\]

The system added $2.35$ points by mistake, so we need to subtract this excess:

\[
\text{Corrected score} = 17.5 - 2.35 = 15.15
\]

\textbf{Answer:} Siti should have $15.15$ points after subtracting the wrongly added points.

\section*{Question 18}
\textbf{Metadata}

\begin{itemize}
  \item Question ID: P4-DcMul2d1d\_P4-DcCnv2Fr\_GPT4.1\_Recreation\_01
  \item Primary KC: DECIMALS | Multiplication | multiplying decimals (up to 2 decimal places) by a 1-digit whole number
  \item Secondary KC: DECIMALS | Conversion from decimals to fraction | expressing decimals as fractions
  \item Topic: Recreation such as sports, games, exercises, music, movie, dancing, painting, fishing and other recreation activities
  \item Grade: Primary 4
\end{itemize}

\textbf{Solution}

First, let's find the total distance Jason skates by multiplying the distance for one round by the number of rounds.\[ 1.25 \times 4 = 5 \]So, Jason skates a total of $5$ km.\newline\newline Now, to express $5$ km as a fraction, notice that $5 = \frac{5}{1}$.\newline\newline Therefore, Jason skates $5$ km in total, which can be expressed as $\frac{5}{1}$ km.

\section*{Question 19}
\textbf{Metadata}

\begin{itemize}
  \item Question ID: O1-RoRepFr\_P5-FrMul2nd\_GPT4.1\_Recreation\_01
  \item Primary KC: RATIO | Representation and concept | ratios involving fractions
  \item Secondary KC: FRACTIONS | Multiplication | fraction multiplication
  \item Topic: Recreation such as sports, games, exercises, music, movie, dancing, painting, fishing and other recreation activities
  \item Grade: Secondary O-level 1
\end{itemize}

\textbf{Solution}

Let the amount of blue paint used be $x$ litres. 

Given the ratio of blue paint to red paint is $\frac{2}{3} : 1$. This means $\frac{x}{\frac{3}{4}} = \frac{2}{3}$. 

To find $x$:
\[ \frac{x}{\frac{3}{4}} = \frac{2}{3} \]
Multiply both sides by $\frac{3}{4}$:
\[ x = \frac{2}{3} \times \frac{3}{4} \]
\[ x = \frac{2 \times 3}{3 \times 4} \]
\[ x = \frac{6}{12} \]
\[ x = \frac{1}{2} \]

So, the amount of blue paint used is $\frac{1}{2}$ litres.

\section*{Question 20}
\textbf{Metadata}

\begin{itemize}
  \item Question ID: P4-DcDiv2d1d\_P4-DcSub2nd\_GPT4.1\_Recreation\_01
  \item Primary KC: DECIMALS | Division | dividing decimals (up to 2 decimal places) by a 1-digit whole number
  \item Secondary KC: DECIMALS | Subtraction | subtracting decimals
  \item Topic: Recreation such as sports, games, exercises, music, movie, dancing, painting, fishing and other recreation activities
  \item Grade: Primary 4
\end{itemize}

\textbf{Solution}

First, let's find out how much Shawn pays each week by dividing the total fee by the number of weeks:

\[
\text{Fee per week} = \frac{36.84}{6} = 6.14
\]

After 4 weeks, Shawn has already paid:

\[
\text{Amount paid for 4 weeks} = 4 \times 6.14 = 24.56
\]

To find out how much more Shawn needs to pay, subtract what he has already paid from the total fee:

\[
\text{Amount left to pay} = 36.84 - 24.56 = 12.28
\]

So, Shawn has to pay \textdollar12.28 for the remaining 2 weeks.

\end{document}
