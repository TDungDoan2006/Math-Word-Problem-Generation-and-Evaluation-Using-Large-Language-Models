\documentclass{article}
\usepackage[utf8]{inputenc}
\usepackage{amsmath}
\usepackage{amsfonts}
\usepackage{amssymb}
\usepackage{graphicx}
\usepackage{hyperref}
\title{'Sarah Solutions recreation v8 v1'}
\author{Tien Dung Doan}
\begin{document}
\maketitle
\section*{Question 1}
\textbf{Metadata}

\begin{itemize}
  \item Question ID: P6-PcFndWN\_P1-WNMul2nd\_GPT4.1\_Recreation\_01
  \item Primary KC: PERCENTAGE | Finding the whole | finding the whole given a part and the percentage
  \item Secondary KC: WHOLE NUMBERS | Multiplication | multiplying whole numbers
  \item Topic: Recreation such as sports, games, exercises, music, movie, dancing, painting, fishing and other recreation activities
  \item Grade: Primary 6
\end{itemize}

\textbf{Solution}

Let the total possible number of points in the basketball challenge be $x$.

Sarah scored 24 points, which is 40\% of $x$.

So, $0.4x = 24$

To find $x$, divide both sides by $0.4$:

\[
x = \frac{24}{0.4} = 60
\]

So, the total possible number of points in the basketball challenge is $60$.

After the challenge, the total possible points were multiplied by $2$ for the next game:

\[
\text{Total possible points after two games} = 60 \times 2 = 120
\]

\textbf{Answer:}

- The total possible number of points Sarah could have scored in the basketball challenge was $60$.
- The total possible number of points she could have scored after both games was $120$.

\section*{Question 2}
\textbf{Metadata}

\begin{itemize}
  \item Question ID: O2-AgSlvIneq\_O2-AgRepIneq\_GPT4.1\_Recreation\_01
  \item Primary KC: ALGEBRA | Solving | solving simple linear inequalities with one variable
  \item Secondary KC: ALGEBRA | Representation and concept | translation of simple real-world situations to simple linear inequalities with one variable
  \item Topic: Recreation such as sports, games, exercises, music, movie, dancing, painting, fishing and other recreation activities
  \item Grade: Secondary O-level 2
\end{itemize}

\textbf{Solution}

The total amount Alicia has is \textdollar50. She wants to save at least \textdollar10 for snacks, so the money she can spend on movie tickets is at most \textdollar40.

Let $n$ be the number of movie tickets Alicia can buy. Each ticket costs \textdollar12. The amount she spends on tickets is $12n$.

We set up the inequality:
$$12n \leq 40$$

To find the largest possible value of $n$, solve the inequality:
$$n \leq \frac{40}{12}$$
$$n \leq 3.33$$

Since $n$ must be a whole number, the largest possible value of $n$ is $3$.

So, Alicia can buy at most $3$ movie tickets and still have at least \textdollar10 left for snacks.

\section*{Question 3}
\textbf{Metadata}

\begin{itemize}
  \item Question ID: O3-SPMulProb\_O3-SPFndPrCE\_GPT4.1\_Recreation\_01
  \item Primary KC: STATISTICS AND PROBABILITY | Multiplication | multiplication of probabilities
  \item Secondary KC: STATISTICS AND PROBABILITY | Finding probability | probability of simple combined events
  \item Topic: Recreation such as sports, games, exercises, music, movie, dancing, painting, fishing and other recreation activities
  \item Grade: Secondary O-level 3/4
\end{itemize}

\textbf{Solution}

First, calculate the probability that Adrian draws a red card.

- Total number of cards $= 5 + 7 = 12$
- Number of red cards $= 5$
- Probability of drawing a red card $= \frac{5}{12}$

Next, calculate the probability of rolling an even number on a fair six-sided dice.

- There are 3 even numbers on a dice: 2, 4, 6.
- Probability of rolling an even number $= \frac{3}{6} = \frac{1}{2}$

Since the card draw and dice roll are independent events, the probability that both events occur together is the product of their probabilities:

$$
\text{Probability} = \frac{5}{12} \times \frac{1}{2} = \frac{5}{24}
$$

Therefore, the probability that Adrian draws a red card and then rolls an even number is $\boxed{\frac{5}{24}}$.

\section*{Question 4}
\textbf{Metadata}

\begin{itemize}
  \item Question ID: P6-FrDivPN\_P2-FrSub2nd\_GPT4.1\_Recreation\_01
  \item Primary KC: FRACTIONS | Division | dividing a proper fraction by a whole number
  \item Secondary KC: FRACTIONS | Subtraction | subtracting fractions
  \item Topic: Recreation such as sports, games, exercises, music, movie, dancing, painting, fishing and other recreation activities
  \item Grade: Primary 6
\end{itemize}

\textbf{Solution}

(a) The total painting time is $\frac{3}{4}$ hour, and it was divided equally among 3 people.

So, each person got:

$\frac{3}{4} \div 3 = \frac{3}{4} \times \frac{1}{3} = \frac{3}{12} = \frac{1}{4}$

Each person got $\frac{1}{4}$ hour of painting time.

(b) Sarah left $\frac{1}{6}$ hour earlier than the others.

So, Sarah's actual painting time was:

$\frac{1}{4} - \frac{1}{6}$

First, make the denominators the same:

$\frac{1}{4} = \frac{3}{12}$
$\frac{1}{6} = \frac{2}{12}$

So,

$\frac{3}{12} - \frac{2}{12} = \frac{1}{12}$

Sarah painted for $\frac{1}{12}$ hour less than her friends.

\section*{Question 5}
\textbf{Metadata}

\begin{itemize}
  \item Question ID: O1-PcRepRvs\_O1-PcCnv2Fr\_GPT4.1\_Recreation\_01
  \item Primary KC: PERCENTAGE | Representation and concept | reverse percentages
  \item Secondary KC: PERCENTAGE | Conversion to fraction | expressing percentage as a fraction
  \item Topic: Recreation such as sports, games, exercises, music, movie, dancing, painting, fishing and other recreation activities
  \item Grade: Secondary O-level 1
\end{itemize}

\textbf{Solution}

Let $x$ be the maximum mark for the piano exam. 

Amy's score last month was $80\%$ of the maximum mark, which is $0.8x$.

This month, her score increased by $20\%$ compared to last month, so:

\[ \text{This month's score} = \text{Last month's score} + 20\% \text{ of last month's score} \]
\[ 96 = 0.8x + 20\% \times 0.8x \]

Express $20\%$ as a fraction: \( 20\% = \dfrac{20}{100} = \dfrac{1}{5} \).

Substitute:
\[ 96 = 0.8x + \left( \dfrac{1}{5} \times 0.8x \right) \]
\[ 96 = 0.8x + 0.16x \]
\[ 96 = 0.96x \]
\[ x = \dfrac{96}{0.96} \]
\[ x = 100 \]

\textbf{Answer:} The maximum mark for the piano exam was $100$.

\section*{Question 6}
\textbf{Metadata}

\begin{itemize}
  \item Question ID: P4-DcMul2d1d\_P4-DcCmp3d\_GPT4.1\_Recreation\_01
  \item Primary KC: DECIMALS | Multiplication | multiplying decimals (up to 2 decimal places) by a 1-digit whole number
  \item Secondary KC: DECIMALS | Comparison and ordering | comparing and ordering decimals up to 3 decimal places
  \item Topic: Recreation such as sports, games, exercises, music, movie, dancing, painting, fishing and other recreation activities
  \item Grade: Primary 4
\end{itemize}

\textbf{Solution}

Let us multiply the distance each player hit the ball by $4$:

For Player A:
$2.35 \times 4 = 9.40$

For Player B:
$2.171 \times 4 = 8.684$

For Player C:
$2.47 \times 4 = 9.88$

So, the total distances are:
Player A: $9.40$ m
Player B: $8.684$ m
Player C: $9.88$ m

Next, we arrange these totals from the shortest to the longest:
$8.684$ m (Player B), $9.40$ m (Player A), $9.88$ m (Player C)

**Final answer:**
Shortest to longest total distance: Player B, Player A, Player C.

\section*{Question 7}
\textbf{Metadata}

\begin{itemize}
  \item Question ID: O1-RoRepDc\_P4-DcAdd2nd\_GPT4.1\_Recreation\_01
  \item Primary KC: RATIO | Representation and concept | ratios involving decimals
  \item Secondary KC: DECIMALS | Addition | adding decimals
  \item Topic: Recreation such as sports, games, exercises, music, movie, dancing, painting, fishing and other recreation activities
  \item Grade: Secondary O-level 1
\end{itemize}

\textbf{Solution}

Let the number of girls be $g$.

The ratio of girls to boys is $1.5 : 2.5$. This means:
\[
\frac{g}{12.5} = \frac{1.5}{2.5}
\]

Solve for $g$:
\[
g = 12.5 \times \frac{1.5}{2.5} = 12.5 \times 0.6 = 7.5
\]
So, there are $7.5$ girls initially.

Now, $3.7$ more girls join:
\[
\text{New number of girls} = 7.5 + 3.7 = 11.2
\]

\textbf{Answer:}
There were initially $7.5$ girls in the badminton club. After $3.7$ more girls join, there are now $11.2$ girls in the club.

\section*{Question 8}
\textbf{Metadata}

\begin{itemize}
  \item Question ID: P5-FrMulImN\_P2-FrAdd2nd\_GPT4.1\_Recreation\_01
  \item Primary KC: FRACTIONS | Multiplication | multiplying a proper/improper fraction and a whole number
  \item Secondary KC: FRACTIONS | Addition | adding fractions
  \item Topic: Recreation such as sports, games, exercises, music, movie, dancing, painting, fishing and other recreation activities
  \item Grade: Primary 5
\end{itemize}

\textbf{Solution}

First, find the amount of paint Amanda used for the 5 painting sessions:

Amount used per session: $\frac{3}{4}$ of a tube

Total for 5 sessions: $5 \times \frac{3}{4} = \frac{15}{4}$ tubes

Then, add the amount used for touch-ups:

Touch-up amount: $\frac{2}{4}$ tubes

Total paint used: $\frac{15}{4} + \frac{2}{4} = \frac{17}{4}$ tubes

$\frac{17}{4}$ can be written as a mixed number:

$\frac{17}{4} = 4 \frac{1}{4}$

So, Amanda used a total of $4 \frac{1}{4}$ tubes of paint this month.

\section*{Question 9}
\textbf{Metadata}

\begin{itemize}
  \item Question ID: O1-PcRep2q\_O1-PcCnv2Fr\_GPT4.1\_Recreation\_01
  \item Primary KC: PERCENTAGE | Representation and concept | comparing two quantities by percentage
  \item Secondary KC: PERCENTAGE | Conversion to fraction | expressing percentage as a fraction
  \item Topic: Recreation such as sports, games, exercises, music, movie, dancing, painting, fishing and other recreation activities
  \item Grade: Secondary O-level 1
\end{itemize}

\textbf{Solution}

(a) 
Number of students who watched the band performance = $120$
Number of students who watched the dance performance = $150$

The difference in the number of students is $150 - 120 = 30$

The percentage by which the dance performance viewers is greater than the band performance viewers is:
\[
\frac{30}{120} \times 100\% = 25\%
\]

So, the number of students watching the dance performance is $25\%$ greater than those watching the band performance.

(b)
Express $25\%$ as a fraction:
\[
25\% = \frac{25}{100} = \frac{1}{4}
\]

Therefore, $25\%$ as a fraction in simplest form is $\frac{1}{4}$.

\section*{Question 10}
\textbf{Metadata}

\begin{itemize}
  \item Question ID: P5-FrMulImIm\_P5-FrCnv2Dc\_GPT4.1\_Recreation\_01
  \item Primary KC: FRACTIONS | Multiplication | multiplying two improper fractions
  \item Secondary KC: FRACTIONS | Conversion to decimals | expressing fractions as decimals
  \item Topic: Recreation such as sports, games, exercises, music, movie, dancing, painting, fishing and other recreation activities
  \item Grade: Primary 5
\end{itemize}

\textbf{Solution}

Sarah practised $\frac{7}{4}$ of her planned time on Monday, and $\frac{5}{3}$ times that on Tuesday.

First, find the fraction of her planned time practised on Tuesday:

$\frac{7}{4} \times \frac{5}{3} = \frac{7 \times 5}{4 \times 3} = \frac{35}{12}$

So, on Tuesday, Sarah practised $\frac{35}{12}$ of her planned time.

Next, express this fraction as a decimal:

$\frac{35}{12} = 2.9166... \approx 2.92$

Sarah completed about \textdollar2.92 times her planned practice time on Tuesday. 

**Final Answer:**

On Tuesday, Sarah completed $\frac{35}{12}$ of her planned practice time, which is approximately $2.92$ times her planned practice time.

\section*{Question 11}
\textbf{Metadata}

\begin{itemize}
  \item Question ID: P5-FrMulImN\_P5-FrCnv2Dc\_GPT4.1\_Recreation\_01
  \item Primary KC: FRACTIONS | Multiplication | multiplying a proper/improper fraction and a whole number
  \item Secondary KC: FRACTIONS | Conversion to decimals | expressing fractions as decimals
  \item Topic: Recreation such as sports, games, exercises, music, movie, dancing, painting, fishing and other recreation activities
  \item Grade: Primary 5
\end{itemize}

\textbf{Solution}

Sarah practises $\frac{3}{5}$ of an hour each day for 4 days.

Total hours spent $= 4 \times \frac{3}{5} = \frac{12}{5}$ hours.

Now, express $\frac{12}{5}$ as a decimal:

$\frac{12}{5} = 2.4$

Sarah spends 2.4 hours in total practising the piano over 4 days.

\section*{Question 12}
\textbf{Metadata}

\begin{itemize}
  \item Question ID: O1-AgRepExSq\_O1-AgEvlEx\_GPT4.1\_Recreation\_01
  \item Primary KC: ALGEBRA | Representation and concept | translation of simple real-world situations into quadratic algebraic expressions
  \item Secondary KC: ALGEBRA | Evaluation | evaluation of algebraic expressions and formulae
  \item Topic: Recreation such as sports, games, exercises, music, movie, dancing, painting, fishing and other recreation activities
  \item Grade: Secondary O-level 1
\end{itemize}

\textbf{Solution}

Given the cost expression: $2x^2 + 3x + 5$ and $x = 4$. 

Substitute $x = 4$ into the expression:

$2(4)^2 + 3(4) + 5 = 2(16) + 12 + 5 = 32 + 12 + 5 = 49$

Thus, the total cost for 4 hours is \textdollar49.

\section*{Question 13}
\textbf{Metadata}

\begin{itemize}
  \item Question ID: P5-PcRepWh\_P1-WNSub2nd\_GPT4.1\_Recreation\_01
  \item Primary KC: PERCENTAGE | Representation and concept | expressing a part of a whole as a percentage
  \item Secondary KC: WHOLE NUMBERS | Subtraction | subtracting whole numbers
  \item Topic: Recreation such as sports, games, exercises, music, movie, dancing, painting, fishing and other recreation activities
  \item Grade: Primary 5
\end{itemize}

\textbf{Solution}

(a) The number of students who went home after the lunch break is:

$60 - 45 = 15$

So, $15$ students went home.

(b) To find the percentage of students who stayed to continue painting:

\[
\text{Percentage stayed} = \left( \frac{45}{60} \right) \times 100\% = 75\%
\]

Therefore, $75\%$ of the students stayed to continue painting after the lunch break.

\section*{Question 14}
\textbf{Metadata}

\begin{itemize}
  \item Question ID: P6-PcFndChg\_P1-WNAdd2nd\_GPT4.1\_Recreation\_01
  \item Primary KC: PERCENTAGE | Finding change | finding percentage increase/decrease
  \item Secondary KC: WHOLE NUMBERS | Addition | adding whole numbers
  \item Topic: Recreation such as sports, games, exercises, music, movie, dancing, painting, fishing and other recreation activities
  \item Grade: Primary 6
\end{itemize}

\textbf{Solution}

(a) Number of students last month $= 40$
Number of new students $= 15$
Total number of students this month $= 40 + 15 = 55$

(b) The increase in the number of students is $15$.

The percentage increase $= \frac{\text{increase}}{\text{original number}} \times 100\% = \frac{15}{40} \times 100\%$

$= 0.375 \times 100\%$

$= 37.5\%$

To the nearest whole number, the percentage increase is $38\%$.

\section*{Question 15}
\textbf{Metadata}

\begin{itemize}
  \item Question ID: O1-RoRepFr\_P5-FrMul2nd\_GPT4.1\_Recreation\_01
  \item Primary KC: RATIO | Representation and concept | ratios involving fractions
  \item Secondary KC: FRACTIONS | Multiplication | fraction multiplication
  \item Topic: Recreation such as sports, games, exercises, music, movie, dancing, painting, fishing and other recreation activities
  \item Grade: Secondary O-level 1
\end{itemize}

\textbf{Solution}

Let the amount of blue paint used be $x$ litres. 

Given the ratio of blue paint to red paint is $\frac{2}{3} : 1$. This means $\frac{x}{\frac{3}{4}} = \frac{2}{3}$. 

To find $x$:
\[ \frac{x}{\frac{3}{4}} = \frac{2}{3} \]
Multiply both sides by $\frac{3}{4}$:
\[ x = \frac{2}{3} \times \frac{3}{4} \]
\[ x = \frac{2 \times 3}{3 \times 4} \]
\[ x = \frac{6}{12} \]
\[ x = \frac{1}{2} \]

So, the amount of blue paint used is $\frac{1}{2}$ litres.

\section*{Question 16}
\textbf{Metadata}

\begin{itemize}
  \item Question ID: P5-FrSubMix\_P5-FrCnv2Dc\_GPT4.1\_Recreation\_01
  \item Primary KC: FRACTIONS | Subtraction | subtracting mixed numbers
  \item Secondary KC: FRACTIONS | Conversion to decimals | expressing fractions as decimals
  \item Topic: Recreation such as sports, games, exercises, music, movie, dancing, painting, fishing and other recreation activities
  \item Grade: Primary 5
\end{itemize}

\textbf{Solution}

First, convert both mixed numbers to improper fractions:

For Saturday: $2\dfrac{3}{4} = 2 + \dfrac{3}{4} = \dfrac{8}{4} + \dfrac{3}{4} = \dfrac{11}{4}$

For Sunday: $1\dfrac{2}{5} = 1 + \dfrac{2}{5} = \dfrac{5}{5} + \dfrac{2}{5} = \dfrac{7}{5}$

Now, subtract the two:

Find a common denominator for $\dfrac{11}{4}$ and $\dfrac{7}{5}$, which is $20$.

$\dfrac{11}{4} = \dfrac{11 \times 5}{4 \times 5} = \dfrac{55}{20}$
$\dfrac{7}{5} = \dfrac{7 \times 4}{5 \times 4} = \dfrac{28}{20}$

Subtract: $\dfrac{55}{20} - \dfrac{28}{20} = \dfrac{27}{20}$

Write $\dfrac{27}{20}$ as a decimal:

$\dfrac{27}{20} = 1.35$

So, Rachel practised $1.35$ hours ($1$ hour $21$ minutes) more on Saturday than on Sunday.

\textbf{Answer:} Rachel practised $1.35$ hours more on Saturday than on Sunday.

\section*{Question 17}
\textbf{Metadata}

\begin{itemize}
  \item Question ID: P5-DcDiv3dK\_P4-DcSub2nd\_GPT4.1\_Recreation\_01
  \item Primary KC: DECIMALS | Division | dividing decimals (up to 3 decimal places) by 10, 100, 1000 and their multiples
  \item Secondary KC: DECIMALS | Subtraction | subtracting decimals
  \item Topic: Recreation such as sports, games, exercises, music, movie, dancing, painting, fishing and other recreation activities
  \item Grade: Primary 5
\end{itemize}

\textbf{Solution}

First, we find out how many bottles Sarah can fill:  \[ 14.250 \div 0.100 = 142.5 \]  This means she can fill 142 bottles, and there will be a remainder.  Amount used to fill bottles: \[ 142 \times 0.100 = 14.200 \text{ litres} \]  Leftover drink: \[ 14.250 - 14.200 = 0.050 \text{ litres} \]  After Sarah drank 0.075 litres, since she only had 0.050 litres left, it means she finishes the leftover drink and there is none left:  \[ 0.050 - 0.075 = -0.025 \]  However, since there cannot be negative drink left, Sarah could only drink 0.050 litres and there is \boxed{0} litres left after she drank.

\section*{Question 18}
\textbf{Metadata}

\begin{itemize}
  \item Question ID: P6-PcFndWN\_P1-WNDiv2nd\_GPT4.1\_Recreation\_01
  \item Primary KC: PERCENTAGE | Finding the whole | finding the whole given a part and the percentage
  \item Secondary KC: WHOLE NUMBERS | Division | dividing whole numbers
  \item Topic: Recreation such as sports, games, exercises, music, movie, dancing, painting, fishing and other recreation activities
  \item Grade: Primary 6
\end{itemize}

\textbf{Solution}

Let the total number of students in the school be $x$.\
Given that 35\% of the school's students is 42,\
\\
$0.35x = 42$\
\\
To find $x$, divide both sides by $0.35$:\
\\
$x = \frac{42}{0.35}$\
\\
$x = 120$\\
So, there are 120 students in the school.\\
If the students are divided into 6 equal groups:\\
$\text{Number of students in each group} = \frac{120}{6} = 20$\\
\textbf{Final answer:} There are 20 students in each group.

\section*{Question 19}
\textbf{Metadata}

\begin{itemize}
  \item Question ID: P4-DcMul2d1d\_P4-DcSub2nd\_GPT4.1\_Recreation\_01
  \item Primary KC: DECIMALS | Multiplication | multiplying decimals (up to 2 decimal places) by a 1-digit whole number
  \item Secondary KC: DECIMALS | Subtraction | subtracting decimals
  \item Topic: Recreation such as sports, games, exercises, music, movie, dancing, painting, fishing and other recreation activities
  \item Grade: Primary 4
\end{itemize}

\textbf{Solution}

(a) Number of hours per session: $1.25$

Number of sessions: $4$

Total hours spent $= 1.25 \times 4 = 5.00$

So, Lily spends $5.00$ hours at the workshop in one week.

(b) Hours planned: $6.5$

Hours already spent at workshop: $5.00$

Hours needed $= 6.5 - 5.00 = 1.5$

Therefore, Lily needs to dance $1.5$ more hours to meet her goal.

\section*{Question 20}
\textbf{Metadata}

\begin{itemize}
  \item Question ID: O3-SPAddProb\_O3-SPFndPrCE\_GPT4.1\_Recreation\_01
  \item Primary KC: STATISTICS AND PROBABILITY | Addition | addition of probabilities
  \item Secondary KC: STATISTICS AND PROBABILITY | Finding probability | probability of simple combined events
  \item Topic: Recreation such as sports, games, exercises, music, movie, dancing, painting, fishing and other recreation activities
  \item Grade: Secondary O-level 3/4
\end{itemize}

\textbf{Solution}

Let $A$ be the event that a student plays the piano, and $B$ be the event that a student plays the guitar. We are given:

\begin{align*}
P(A) &= \frac{18}{30} = 0.6 \\
P(B) &= \frac{15}{30} = 0.5 \\
P(A \cap B) &= \frac{6}{30} = 0.2
\end{align*}

We want to find $P(A \cup B)$, the probability that a student plays at least one instrument.

The addition rule for probabilities gives:

\[
P(A \cup B) = P(A) + P(B) - P(A \cap B)
\]

Substitute the values:
\[
P(A \cup B) = 0.6 + 0.5 - 0.2 = 0.9
\]

Therefore, the probability that a randomly chosen student plays at least one instrument is $0.9$ or $90\%$.

\section*{Question 21}
\textbf{Metadata}

\begin{itemize}
  \item Question ID: O1-PcFndRslt\_P1-WNAdd2nd\_GPT4.1\_Recreation\_01
  \item Primary KC: PERCENTAGE | Finding result after change | increasing/decreasing a quantity by a given percentage
  \item Secondary KC: WHOLE NUMBERS | Addition | adding whole numbers
  \item Topic: Recreation such as sports, games, exercises, music, movie, dancing, painting, fishing and other recreation activities
  \item Grade: Secondary O-level 1
\end{itemize}

\textbf{Solution}

(a) To find the number of members after a $15\%$ increase, calculate $15\%$ of $120$ and add it to the original number:

$15\% \times 120 = \frac{15}{100} \times 120 = 18$

Number of members after increase $= 120 + 18 = 138$

(b) After $25$ more people joined:

Total members $= 138 + 25 = 163$

\textbf{Final Answer:} There were $138$ members after the $15\%$ increase, and $163$ members after $25$ more people joined.

\section*{Question 22}
\textbf{Metadata}

\begin{itemize}
  \item Question ID: P3-WNMul3d1d\_P1-WNSub2nd\_GPT4.1\_Recreation\_01
  \item Primary KC: WHOLE NUMBERS | Multiplication | multiplying whole numbers up to 3 digits by 1 digit
  \item Secondary KC: WHOLE NUMBERS | Subtraction | subtracting whole numbers
  \item Topic: Recreation such as sports, games, exercises, music, movie, dancing, painting, fishing and other recreation activities
  \item Grade: Primary 3
\end{itemize}

\textbf{Solution}

First, calculate the total number of drumsticks needed:

Number of students $= 126$

Number of drumsticks per student $= 3$

Total drumsticks needed $= 126 \times 3 = 378$

Next, find out how many more drumsticks the school needs to buy:

Number of drumsticks the school already has $= 220$

Number of drumsticks to buy $= 378 - 220 = 158$

\textbf{Answer:} The school needs to buy $158$ more drumsticks.

\section*{Question 23}
\textbf{Metadata}

\begin{itemize}
  \item Question ID: O2-SPFndmdn\_O3-SPFndrng\_GPT4.1\_Recreation\_01
  \item Primary KC: STATISTICS AND PROBABILITY | Finding median | Finding median for a set of data
  \item Secondary KC: STATISTICS AND PROBABILITY | Finding range | finding range as measures of spread for a set of data 
  \item Topic: Recreation such as sports, games, exercises, music, movie, dancing, painting, fishing and other recreation activities
  \item Grade: Secondary O-level 2
\end{itemize}

\textbf{Solution}

(a) To find the median, first arrange the data in order:
$3, 4, 5, 6, 7, 8, 10$

There are $7$ data points. The median is the middle number, which is the $4$th value:

$\text{Median} = 6$

(b) To find the range, subtract the smallest value from the largest value:
$\text{Range} = 10 - 3 = 7$

Therefore,

- The median number of hours is $6$.
- The range of the number of hours is $7$.

\section*{Question 24}
\textbf{Metadata}

\begin{itemize}
  \item Question ID: P5-DcMul3dK\_P4-DcAdd2nd\_GPT4.1\_Recreation\_01
  \item Primary KC: DECIMALS | Multiplication | multiplying decimals (up to 3 decimal places) by 10, 100, 1000 and their multiples
  \item Secondary KC: DECIMALS | Addition | adding decimals
  \item Topic: Recreation such as sports, games, exercises, music, movie, dancing, painting, fishing and other recreation activities
  \item Grade: Primary 5
\end{itemize}

\textbf{Solution}

(a) The total mass of nuts in one bag is:\\
$0.125 + 0.25 + 0.4 = 0.775$ kg\\
\\
(b) For $10$ bags:\\
$0.775 \times 10 = 7.75$ kg\\
\\
(c) For two dance classes, she needs $10 \times 2 = 20$ bags.\\
Total mass needed = $0.775 \times 20 = 15.5$ kg\\
\\
\textbf{Final Answers:}\\
(a) $0.775$ kg\\
(b) $7.75$ kg\\
(c) $15.5$ kg

\section*{Question 25}
\textbf{Metadata}

\begin{itemize}
  \item Question ID: P6-FrDivPN\_P5-FrMul2nd\_GPT4.1\_Recreation\_01
  \item Primary KC: FRACTIONS | Division | dividing a proper fraction by a whole number
  \item Secondary KC: FRACTIONS | Multiplication | fraction multiplication
  \item Topic: Recreation such as sports, games, exercises, music, movie, dancing, painting, fishing and other recreation activities
  \item Grade: Primary 6
\end{itemize}

\textbf{Solution}

(a) Each friend's share is $\dfrac{3}{4} \div 3 = \dfrac{3}{4} \times \dfrac{1}{3} = \dfrac{3}{12} = \dfrac{1}{4}$ of the chocolate bar.

(b) Anna eats $\dfrac{2}{3}$ of one friend's share. So, Anna eats $\dfrac{2}{3} \times \dfrac{1}{4} = \dfrac{2}{12} = \dfrac{1}{6}$ of the chocolate bar.

Final answers:
(a) Each friend gets $\dfrac{1}{4}$ of the chocolate bar.
(b) Anna eats $\dfrac{1}{6}$ of the chocolate bar.

\section*{Question 26}
\textbf{Metadata}

\begin{itemize}
  \item Question ID: P5-DcDiv3dK\_P4-DcCmp3d\_GPT4.1\_Recreation\_01
  \item Primary KC: DECIMALS | Division | dividing decimals (up to 3 decimal places) by 10, 100, 1000 and their multiples
  \item Secondary KC: DECIMALS | Comparison and ordering | comparing and ordering decimals up to 3 decimal places
  \item Topic: Recreation such as sports, games, exercises, music, movie, dancing, painting, fishing and other recreation activities
  \item Grade: Primary 5
\end{itemize}

\textbf{Solution}

(a) Total distance = $12.450$ metres, Number of laps = $10$.

Distance per lap = $12.450 \div 10 = 1.245$ metres.

(b) Ordering the lap distances in ascending order:

$1.240$, $1.243$, $1.244$, $1.245$, $1.245$, $1.246$, $1.247$, $1.248$, $1.249$, $1.250$

(c) There are $10$ laps, so the middle values are the $5$th and $6$th lap distances when ordered.

The $5$th distance: $1.245$ metres

The $6$th distance: $1.246$ metres

Average of the middle two: $\frac{1.245 + 1.246}{2} = 1.2455$ metres

Thus, the lap distances exactly in the middle are $1.245$ metres and $1.246$ metres. Their average is $1.2455$ metres.

\end{document}
