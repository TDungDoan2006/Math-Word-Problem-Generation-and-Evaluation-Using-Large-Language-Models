\documentclass{article}
\usepackage[utf8]{inputenc}
\usepackage{amsmath}
\usepackage{amsfonts}
\usepackage{amssymb}
\usepackage{graphicx}
\usepackage{hyperref}
\title{100 Solutions recreation\_v8\_v1}
\author{Tien Dung Doan}
\begin{document}
\maketitle
\section*{Question 1}
\textbf{Metadata}

\begin{itemize}
  \item Question ID: P5-FrAddMix\_P5-FrCnv2Dc\_GPT4.1\_Recreation\_01
  \item Primary KC: FRACTIONS | Addition | adding mixed numbers
  \item Secondary KC: FRACTIONS | Conversion to decimals | expressing fractions as decimals
  \item Topic: Recreation such as sports, games, exercises, music, movie, dancing, painting, fishing and other recreation activities
  \item Grade: Primary 5
\end{itemize}

\textbf{Solution}

(a) First, add the mixed numbers:

$1\dfrac{3}{4} + 2\dfrac{2}{5}$

Convert the mixed numbers to improper fractions:

$1\dfrac{3}{4} = \frac{7}{4}$

$2\dfrac{2}{5} = \frac{12}{5}$

Find a common denominator for $4$ and $5$, which is $20$:

$\frac{7}{4} = \frac{35}{20}$

$\frac{12}{5} = \frac{48}{20}$

Add the fractions:

$\frac{35}{20}+\frac{48}{20}=\frac{83}{20}$

Convert $\frac{83}{20}$ back to a mixed number:

$83\div20=4$ remainder $3$, so $\frac{83}{20} = 4\dfrac{3}{20}$

So, Kelly practised piano for $4\dfrac{3}{20}$ hours in total on Monday and Tuesday.

(b) Express $4\dfrac{3}{20}$ as a decimal:

$\dfrac{3}{20} = 0.15$

So, $4 + 0.15 = 4.15$

Kelly practised piano for $4.15$ hours in total on Monday and Tuesday.

\section*{Question 2}
\textbf{Metadata}

\begin{itemize}
  \item Question ID: P6-FrDivPP\_P5-FrMul2nd\_GPT4.1\_Recreation\_01
  \item Primary KC: FRACTIONS | Division | dividing a whole number/proper fraction by a proper fraction
  \item Secondary KC: FRACTIONS | Multiplication | fraction multiplication
  \item Topic: Recreation such as sports, games, exercises, music, movie, dancing, painting, fishing and other recreation activities
  \item Grade: Primary 6
\end{itemize}

\textbf{Solution}

(a) To find how many songs Alvin can practise:

Number of songs $= \frac{3}{4} \div \frac{1}{8} = \frac{3}{4} \times \frac{8}{1} = \frac{3 \times 8}{4 \times 1} = \frac{24}{4} = 6$

So Alvin can practise $6$ songs.

(b) Time spent on the first song $= \frac{2}{3} \times \frac{1}{8} = \frac{2 \times 1}{3 \times 8} = \frac{2}{24} = \frac{1}{12}$ hour.

So Alvin spends $\frac{1}{12}$ hour on the first song.

\section*{Question 3}
\textbf{Metadata}

\begin{itemize}
  \item Question ID: P3-WNDivRmd3d\_P1-WNSub2nd\_GPT4.1\_Recreation\_01
  \item Primary KC: WHOLE NUMBERS | Division | dividing whole numbers up to 3 digits by 1 digit with remainder 
  \item Secondary KC: WHOLE NUMBERS | Subtraction | subtracting whole numbers
  \item Topic: Recreation such as sports, games, exercises, music, movie, dancing, painting, fishing and other recreation activities
  \item Grade: Primary 3
\end{itemize}

\textbf{Solution}

To find out how many stickers each student gets and how many are left:

$154 \div 6 = 25$ remainder $4$

Each student gets $25$ stickers, and the teacher has $4$ stickers left.

If the teacher keeps $2$ stickers for herself:

$4 - 2 = 2$

So, after keeping $2$ stickers, the teacher has $2$ stickers remaining.

\section*{Question 4}
\textbf{Metadata}

\begin{itemize}
  \item Question ID: P4-WNDiv4d1d\_P1-WNCmp\_GPT4.1\_Recreation\_01
  \item Primary KC: WHOLE NUMBERS | Division | dividing whole numbers up to 4 digits by 1 digit
  \item Secondary KC: WHOLE NUMBERS | Comparison and ordering | comparing and ordering whole numbers
  \item Topic: Recreation such as sports, games, exercises, music, movie, dancing, painting, fishing and other recreation activities
  \item Grade: Primary 4
\end{itemize}

\textbf{Solution}

(a) To find the number of children in each team, divide $1,356$ by $4$.\\
$1,356 \div 4 = 339$ \\ Children in each team: $339$\\
Leftover children: $1,356 - (4 \times 339) = 1,356 - 1,356 = 0$\\
So, there are no leftover children.\\
\\
(b) Comparing the number of children in each team ($339$) and the number of leftover children ($0$), we see that $339 > 0$.\
\\
Therefore, the number of children in each team is greater than the number of leftover children.

\section*{Question 5}
\textbf{Metadata}

\begin{itemize}
  \item Question ID: P4-DcDiv2d1d\_P4-DcCnv2Fr\_GPT4.1\_Recreation\_01
  \item Primary KC: DECIMALS | Division | dividing decimals (up to 2 decimal places) by a 1-digit whole number
  \item Secondary KC: DECIMALS | Conversion from decimals to fraction | expressing decimals as fractions
  \item Topic: Recreation such as sports, games, exercises, music, movie, dancing, painting, fishing and other recreation activities
  \item Grade: Primary 4
\end{itemize}

\textbf{Solution}

(a) The total cost for 6 packs is \textdollar9.84. To find the price of each pack:

\[
\text{Price per pack} = \frac{9.84}{6} = 1.64
\]

So, Siti paid \textdollar1.64 for each pack of guitar strings.

(b) To express 1.64 as a fraction:

\[
1.64 = \frac{164}{100} = \frac{41}{25}
\]

So, the price of each pack as a fraction in its simplest form is \(\frac{41}{25}\).

\section*{Question 6}
\textbf{Metadata}

\begin{itemize}
  \item Question ID: P6-RoFndDvqWN\_P1-WNAdd2nd\_GPT4.1\_Recreation\_01
  \item Primary KC: RATIO | Finding divided quantities | dividing a given quantity in a given ratio
  \item Secondary KC: WHOLE NUMBERS | Addition | adding whole numbers
  \item Topic: Recreation such as sports, games, exercises, music, movie, dancing, painting, fishing and other recreation activities
  \item Grade: Primary 6
\end{itemize}

\textbf{Solution}

First, we divide \textdollar180 in the ratio $2:3$ between Aaron and Bella. 

Find the total number of parts:
$2 + 3 = 5$

Aaron's share:
$\frac{2}{5} \times 180 = 72$

Bella's share:
$\frac{3}{5} \times 180 = 108$

So, Aaron paid \textdollar72 and Bella paid \textdollar108 for the tickets.

To find the total amount spent by the group (tickets + snacks):
$180 + 40 = 220$

The group spent a total of \textdollar220, including Charlie's contribution for snacks.

\section*{Question 7}
\textbf{Metadata}

\begin{itemize}
  \item Question ID: P5-FrMulPIm\_P2-FrSub2nd\_GPT4.1\_Recreation\_01
  \item Primary KC: FRACTIONS | Multiplication | multiplying a proper fraction and a proper/improper fractions
  \item Secondary KC: FRACTIONS | Subtraction | subtracting fractions
  \item Topic: Recreation such as sports, games, exercises, music, movie, dancing, painting, fishing and other recreation activities
  \item Grade: Primary 5
\end{itemize}

\textbf{Solution}

(a) To find how many hours Ella spent playing scales:

Ella practised piano for $\dfrac{3}{5}$ \text{hour}.
She spent $\dfrac{4}{7}$ of this time on scales:

$\dfrac{4}{7} \times \dfrac{3}{5} = \dfrac{4 \times 3}{7 \times 5} = \dfrac{12}{35}$ \text{hour}.

Answer: Ella spent $\dfrac{12}{35}$ \text{hour} playing scales.

(b) Ella practised piano $\dfrac{1}{5}$ \text{hour} less than Ben:

Amount of time Ben spent $=$ Amount of time Ella spent $+$ $\dfrac{1}{5}$ \text{hour}

Ella: $\dfrac{3}{5}$ \text{hour}

So, Ben: $\dfrac{3}{5} + \dfrac{1}{5} = \dfrac{4}{5}$ \text{hour}

Answer: Ben spent $\dfrac{4}{5}$ \text{hour} practising piano.

\section*{Question 8}
\textbf{Metadata}

\begin{itemize}
  \item Question ID: P4-WNMul4d1d\_P1-WNCmp\_GPT4.1\_Recreation\_01
  \item Primary KC: WHOLE NUMBERS | Multiplication | multiplying whole numbers up to 4 digits by 1 digit or up to 3 digits by 2 digits
  \item Secondary KC: WHOLE NUMBERS | Comparison and ordering | comparing and ordering whole numbers
  \item Topic: Recreation such as sports, games, exercises, music, movie, dancing, painting, fishing and other recreation activities
  \item Grade: Primary 4
\end{itemize}

\textbf{Solution}

(a) The total points the Red Team collected is $135 \times 7 = 945$ points.\\
(b) The total points the Blue Team collected is $248 \times 6 = 1,488$ points.\\
(c) Comparing the two totals, the Blue Team collected more points.\\
Difference in points: $1,488 - 945 = 543$ points.\\
\\
\textbf{Answer:}\\
(a) 945 points\\
(b) 1,488 points\\
(c) The Blue Team collected more points by 543 points.

\section*{Question 9}
\textbf{Metadata}

\begin{itemize}
  \item Question ID: O2-SPFndmdn\_O3-SPFndPctl\_GPT4.1\_Recreation\_01
  \item Primary KC: STATISTICS AND PROBABILITY | Finding median | Finding median for a set of data
  \item Secondary KC: STATISTICS AND PROBABILITY | Finding percentiles | finding percentiles for a set of data
  \item Topic: Recreation such as sports, games, exercises, music, movie, dancing, painting, fishing and other recreation activities
  \item Grade: Secondary O-level 2
\end{itemize}

\textbf{Solution}

(a) To find the median, first check that the data is arranged in ascending order. There are 11 numbers (an odd number), so the median is the value at position $\frac{11+1}{2}=6$ in the data set.\\

The 6th value is $16$.\\

\textbf{Median = 16}$.$\\

(b) To find the 25th percentile (first quartile, $Q_1$), use the position formula: $\frac{1}{4}(n+1)$, where $n=11$.\\

So, $Q_1$ is at position $\frac{1}{4}\times(11+1) = \frac{1}{4}\times 12=3$; the 3rd value is $10$.\\

To find the 75th percentile (third quartile, $Q_3$), use the position formula: $\frac{3}{4}(n+1)$.\\

$Q_3$ is at position $\frac{3}{4}\times 12=9$; the 9th value is $20$.\\

\textbf{25th percentile (Q$_1$) = 10}$.$\\
\textbf{75th percentile (Q$_3$) = 20}$.$

\section*{Question 10}
\textbf{Metadata}

\begin{itemize}
  \item Question ID: P4-WNMul4d1d\_P1-WNAdd2nd\_GPT4.1\_Recreation\_01
  \item Primary KC: WHOLE NUMBERS | Multiplication | multiplying whole numbers up to 4 digits by 1 digit or up to 3 digits by 2 digits
  \item Secondary KC: WHOLE NUMBERS | Addition | adding whole numbers
  \item Topic: Recreation such as sports, games, exercises, music, movie, dancing, painting, fishing and other recreation activities
  \item Grade: Primary 4
\end{itemize}

\textbf{Solution}

First, find the total number of water bottles given to all teams:

\[
235 \times 7 = 1,645
\]

Next, add the 58 spare bottles:

\[
1,645 + 58 = 1,703
\]

So, there will be 1,703 water bottles in all after adding the spare bottles.

\section*{Question 11}
\textbf{Metadata}

\begin{itemize}
  \item Question ID: P5-FrSubMix\_P2-FrAdd2nd\_GPT4.1\_Recreation\_01
  \item Primary KC: FRACTIONS | Subtraction | subtracting mixed numbers
  \item Secondary KC: FRACTIONS | Addition | adding fractions
  \item Topic: Recreation such as sports, games, exercises, music, movie, dancing, painting, fishing and other recreation activities
  \item Grade: Primary 5
\end{itemize}

\textbf{Solution}

(a) Total distance Sarah cycled:

$4\frac{2}{3} + 1\frac{5}{6}$

First, convert mixed numbers to improper fractions:
$4\frac{2}{3} = \frac{14}{3}$
$1\frac{5}{6} = \frac{11}{6}$

Find a common denominator:
$
\frac{14}{3} = \frac{28}{6}$

Add the fractions:
$\frac{28}{6} + \frac{11}{6} = \frac{39}{6} = 6\frac{1}{2}$

Sarah cycled a total of $6\frac{1}{2}$ km.

(b) Distance left to reach her goal:

$8\frac{1}{2} - 6\frac{1}{2}$

Convert to improper fractions:
$8\frac{1}{2} = \frac{17}{2}$
$6\frac{1}{2} = \frac{13}{2}$

Subtract:
$\frac{17}{2} - \frac{13}{2} = \frac{4}{2} = 2$

Sarah needed to cycle 2 km more to reach her goal.

\section*{Question 12}
\textbf{Metadata}

\begin{itemize}
  \item Question ID: O3-MXSub\_O3-MXAdd\_GPT4.1\_Recreation\_01
  \item Primary KC: MATRICES | Subtraction | subtraction of matrices
  \item Secondary KC: MATRICES | Addition | addition of matrices
  \item Topic: Recreation such as sports, games, exercises, music, movie, dancing, painting, fishing and other recreation activities
  \item Grade: Secondary O-level 3/4
\end{itemize}

\textbf{Solution}

(a) To find the corrected medal count for each house after the first day, we subtract matrix $C$ from $A$:
\[
A - C = \begin{bmatrix} 5 & 3 \\ 2 & 4 \\ 6 & 1 \end{bmatrix} - \begin{bmatrix} 0 & 1 \\ 0 & 1 \\ 0 & 1 \end{bmatrix} = \begin{bmatrix} 5-0 & 3-1 \\ 2-0 & 4-1 \\ 6-0 & 1-1 \end{bmatrix} = \begin{bmatrix} 5 & 2 \\ 2 & 3 \\ 6 & 0 \end{bmatrix}.
\]

(b) To find the total medals after both days, we add matrix $B$ to the result from part (a):
\[
\begin{bmatrix} 5 & 2 \\ 2 & 3 \\ 6 & 0 \end{bmatrix} + \begin{bmatrix} 3 & 2 \\ 1 & 2 \\ 2 & 2 \end{bmatrix} = \begin{bmatrix} 5+3 & 2+2 \\ 2+1 & 3+2 \\ 6+2 & 0+2 \end{bmatrix} = \begin{bmatrix} 8 & 4 \\ 3 & 5 \\ 8 & 2 \end{bmatrix}.
\]

**Final Answer:**

The corrected medal counts after the first day for the Red, Blue, and Green houses are:
\[
\begin{bmatrix} 5 & 2 \\ 2 & 3 \\ 6 & 0 \end{bmatrix}
\]

The total medal counts after both days are:
\[
\begin{bmatrix} 8 & 4 \\ 3 & 5 \\ 8 & 2 \end{bmatrix}
\]
where rows represent the houses (Red, Blue, Green) and columns represent Track and Field and Swimming medals, respectively.

\section*{Question 13}
\textbf{Metadata}

\begin{itemize}
  \item Question ID: P5-PcRepWh\_P1-WNAdd2nd\_GPT4.1\_Recreation\_01
  \item Primary KC: PERCENTAGE | Representation and concept | expressing a part of a whole as a percentage
  \item Secondary KC: WHOLE NUMBERS | Addition | adding whole numbers
  \item Topic: Recreation such as sports, games, exercises, music, movie, dancing, painting, fishing and other recreation activities
  \item Grade: Primary 5
\end{itemize}

\textbf{Solution}

(a) First, we find the total number of students who participated:

$120 + 80 + 100 = 300$

So, there were $300$ students who participated in these three sports altogether.

(b) To find the percentage of participants in the swimming event:

\[
\text{Percentage} = \frac{80}{300} \times 100\%
\]
$\frac{80}{300} = 0.2666\ldots$

$0.2666\ldots \times 100\% = 26.666\ldots\%$

Correct to the nearest whole number, this is $27\%$.

\textbf{Answer:}

(a) $300$ students 

(b) $27\%$ of the participants joined the swimming event.

\section*{Question 14}
\textbf{Metadata}

\begin{itemize}
  \item Question ID: P5-PcRepWh\_P1-WNMul2nd\_GPT4.1\_Recreation\_01
  \item Primary KC: PERCENTAGE | Representation and concept | expressing a part of a whole as a percentage
  \item Secondary KC: WHOLE NUMBERS | Multiplication | multiplying whole numbers
  \item Topic: Recreation such as sports, games, exercises, music, movie, dancing, painting, fishing and other recreation activities
  \item Grade: Primary 5
\end{itemize}

\textbf{Solution}

The total number of students is $250$. The number of students who played basketball is $70$.

The percentage of students who played basketball is:
\[
\frac{70}{250} \times 100\% = \frac{70 \times 100}{250}\% = \frac{7000}{250}\% = 28\%
\]

So, $28\%$ of the students played basketball.

\section*{Question 15}
\textbf{Metadata}

\begin{itemize}
  \item Question ID: P5-DcMul3dK\_P4-DcRnd3d\_GPT4.1\_Recreation\_01
  \item Primary KC: DECIMALS | Multiplication | multiplying decimals (up to 3 decimal places) by 10, 100, 1000 and their multiples
  \item Secondary KC: DECIMALS | Rounding | rounding decimals up to 3 decimal places to the nearest whole number, 1 decimal place and 2 decimal places 
  \item Topic: Recreation such as sports, games, exercises, music, movie, dancing, painting, fishing and other recreation activities
  \item Grade: Primary 5
\end{itemize}

\textbf{Solution}

(a) To find the total minutes Amira spent, multiply the duration of one song by the number of songs:

$3.275 \times 20 = 65.5$

Therefore, Amira practised for $65.5$ minutes on Saturday.

(b) Rounding $65.5$ to the nearest whole number:

Since $0.5$ rounds up, $65.5$ rounds to $66$.

Amira spent $\textdollar66$ minutes practising, rounded to the nearest whole number.

(c) Rounding $65.5$ to $1$ decimal place:

There are no more digits after the $0.5$, so it stays as $65.5$.

Amira spent $65.5$ minutes practising, rounded to $1$ decimal place.

\section*{Question 16}
\textbf{Metadata}

\begin{itemize}
  \item Question ID: O2-AgSlvSq1v\_O1-AgRepEq\_GPT4.1\_Recreation\_01
  \item Primary KC: ALGEBRA | Solving | solving quadratic equations in one variable
  \item Secondary KC: ALGEBRA | Representation and concept | translation of simple real-world situations to equations
  \item Topic: Recreation such as sports, games, exercises, music, movie, dancing, painting, fishing and other recreation activities
  \item Grade: Secondary O-level 2
\end{itemize}

\textbf{Solution}

Let $h = 15$, so $15 = -5t^2 + 20t$. Rearranging, we get:

$$
-5t^2 + 20t - 15 = 0
$$
Divide both sides by $-5$:
$$
t^2 - 4t + 3 = 0
$$
This is a quadratic equation. We solve it by factoring:

$$(t - 1)(t - 3) = 0$$

Thus, $t = 1$ or $t = 3$.

So, the ball reaches the height of \textdollar15$ metres after $1$ second and $3$ seconds.

\boxed{1 \text{ s and } 3 \text{ s}}

\section*{Question 17}
\textbf{Metadata}

\begin{itemize}
  \item Question ID: P5-FrSubMix\_P2-FrCmp\_GPT4.1\_Recreation\_01
  \item Primary KC: FRACTIONS | Subtraction | subtracting mixed numbers
  \item Secondary KC: FRACTIONS | Comparison and ordering | comparing and ordering fractions
  \item Topic: Recreation such as sports, games, exercises, music, movie, dancing, painting, fishing and other recreation activities
  \item Grade: Primary 5
\end{itemize}

\textbf{Solution}

(a) Total distance = $6$ km.

Jayden cycled $4 \frac{2}{3}$ km before his break.

Distance Jayden had left:
$6 - 4 \frac{2}{3} = 6 - \frac{14}{3} = \frac{18}{3} - \frac{14}{3} = \frac{4}{3}$ km

Sarah cycled $3 \frac{5}{6}$ km before her break.

Distance Sarah had left:
$6 - 3 \frac{5}{6} = 6 - \frac{23}{6} = \frac{36}{6} - \frac{23}{6} = \frac{13}{6}$ km

Compare the remaining distances:
Jayden: $\frac{4}{3} = \frac{8}{6}$ km
Sarah: $\frac{13}{6}$ km

How many more kilometres did Jayden have left after his break than Sarah?
$\frac{8}{6} - \frac{13}{6} = -\frac{5}{6}$ km

This means Jayden actually had $\frac{5}{6}$ km less to cycle after his break than Sarah.

(b) Jayden had $\frac{8}{6} = 1 \frac{1}{3}$ km left, and Sarah had $2 \frac{1}{6}$ km left. 

Since $\frac{8}{6} < \frac{13}{6}$, Jayden had less distance left to cycle after their break.

\textbf{Final answers:}

(a) Jayden had $\frac{5}{6}$ km less left to cycle after his break than Sarah did after hers.

(b) Jayden had less distance left to cycle after his break.

\section*{Question 18}
\textbf{Metadata}

\begin{itemize}
  \item Question ID: P6-FrDivPP\_P2-FrSub2nd\_GPT4.1\_Recreation\_01
  \item Primary KC: FRACTIONS | Division | dividing a whole number/proper fraction by a proper fraction
  \item Secondary KC: FRACTIONS | Subtraction | subtracting fractions
  \item Topic: Recreation such as sports, games, exercises, music, movie, dancing, painting, fishing and other recreation activities
  \item Grade: Primary 6
\end{itemize}

\textbf{Solution}

(a) To find the number of bookmarks Lina can make, divide the total length of ribbon by the length used for each bookmark:
\[
\text{Number of bookmarks} = \frac{3}{\frac{1}{4}} = 3 \times 4 = 12
\]
Lina can make $12$ bookmarks.

(b) Lina gives $\frac{2}{5}$ of the bookmarks away:
\[
\text{Bookmarks given} = \frac{2}{5} \times 12 = \frac{24}{5} = 4.8
\]
Since Lina cannot give away part of a bookmark, she gives away $4$ whole bookmarks (assuming bookmarks must be whole), or, if parts are allowed, $4.8$ bookmarks.

Bookmarks Lina has left:
\[
12 - 4.8 = 7.2
\]
\textbf{Lina has $7.2$ bookmarks left after giving some to her friends.}

(If only whole bookmarks can be given, then she gives $4$ bookmarks and has $8$ left.)

\section*{Question 19}
\textbf{Metadata}

\begin{itemize}
  \item Question ID: P5-FrMulMixN\_P3-FrSmp\_GPT4.1\_Recreation\_01
  \item Primary KC: FRACTIONS | Multiplication | multiplying a mixed number and a whole number
  \item Secondary KC: FRACTIONS | Simplifying | expressing a fraction in its simplest form
  \item Topic: Recreation such as sports, games, exercises, music, movie, dancing, painting, fishing and other recreation activities
  \item Grade: Primary 5
\end{itemize}

\textbf{Solution}

Let the amount of fruit punch in each jug be $2\dfrac{1}{2}$ litres. 

First, write $2\dfrac{1}{2}$ as an improper fraction:

$2\dfrac{1}{2} = 2 + \dfrac{1}{2} = \dfrac{4}{2} + \dfrac{1}{2} = \dfrac{5}{2}$

(a) Total amount of fruit punch for 6 jugs = $6 \times \dfrac{5}{2}$

$6 \times \dfrac{5}{2} = \dfrac{6 \times 5}{2} = \dfrac{30}{2}$

(b) Simplify $\dfrac{30}{2}$:

$\dfrac{30}{2} = 15$

Melissa will prepare 15 litres of fruit punch in total.

\section*{Question 20}
\textbf{Metadata}

\begin{itemize}
  \item Question ID: P4-DcSub2d\_P4-DcCnv2Fr\_GPT4.1\_Recreation\_01
  \item Primary KC: DECIMALS | Subtraction | subtracting decimals (up to 2 decimal places)
  \item Secondary KC: DECIMALS | Conversion from decimals to fraction | expressing decimals as fractions
  \item Topic: Recreation such as sports, games, exercises, music, movie, dancing, painting, fishing and other recreation activities
  \item Grade: Primary 4
\end{itemize}

\textbf{Solution}

(a) To find how many more kilometres Mandy swam on Monday, subtract the distance on Tuesday from the distance on Monday:

$$2.75 - 1.30 = 1.45$$

Mandy swam $1.45$ km more on Monday than on Tuesday.

(b) Express $1.45$ as a fraction:

$1.45 = 1 + 0.45 = 1 + \frac{45}{100}$

Simplify $\frac{45}{100}$:

$\frac{45}{100} = \frac{9}{20}$

So, $1.45 = 1\frac{9}{20}$

Final answer:

Mandy swam $1.45$ km more on Monday than on Tuesday, or $1\frac{9}{20}$ km.

\section*{Question 21}
\textbf{Metadata}

\begin{itemize}
  \item Question ID: P5-FrSubMix\_P3-FrSmp\_GPT4.1\_Recreation\_01
  \item Primary KC: FRACTIONS | Subtraction | subtracting mixed numbers
  \item Secondary KC: FRACTIONS | Simplifying | expressing a fraction in its simplest form
  \item Topic: Recreation such as sports, games, exercises, music, movie, dancing, painting, fishing and other recreation activities
  \item Grade: Primary 5
\end{itemize}

\textbf{Solution}

(a) The time Samuel spent sketching is:

$2\dfrac{3}{4} - 1\dfrac{2}{5}$

First, convert the mixed numbers to improper fractions:

$2\dfrac{3}{4} = \dfrac{8 + 3}{4} = \dfrac{11}{4}$

$1\dfrac{2}{5} = \dfrac{5 + 2}{5} = \dfrac{7}{5}$

Now subtract:

$\dfrac{11}{4} - \dfrac{7}{5}$

Find the common denominator:

LCM of $4$ and $5$ is $20$.

$\dfrac{11}{4} = \dfrac{11 \times 5}{4 \times 5} = \dfrac{55}{20}$

$\dfrac{7}{5} = \dfrac{7 \times 4}{5 \times 4} = \dfrac{28}{20}$

So,

$\dfrac{55}{20} - \dfrac{28}{20} = \dfrac{27}{20}$

(b) $\dfrac{27}{20}$ is an improper fraction. To express it as a mixed number in simplest form:

$\dfrac{27}{20} = 1\dfrac{7}{20}$

The fraction $\dfrac{7}{20}$ is already in its simplest form.

\textbf{Answer:}

(a) Samuel spent $\dfrac{27}{20}$ hours sketching.

(b) In simplest form, Samuel spent $1\dfrac{7}{20}$ hours sketching.

\section*{Question 22}
\textbf{Metadata}

\begin{itemize}
  \item Question ID: O1-PcFndRslt\_P1-WNMul2nd\_GPT4.1\_Recreation\_01
  \item Primary KC: PERCENTAGE | Finding result after change | increasing/decreasing a quantity by a given percentage
  \item Secondary KC: WHOLE NUMBERS | Multiplication | multiplying whole numbers
  \item Topic: Recreation such as sports, games, exercises, music, movie, dancing, painting, fishing and other recreation activities
  \item Grade: Secondary O-level 1
\end{itemize}

\textbf{Solution}

(a)
The original number of members = $80$
Percentage increase = $25\%$
Number of new members $= 25\%$ of $80 = \dfrac{25}{100} \times 80 = 20$

So, $20$ new members joined the club.

(b)
Each new member pays \textdollar15.

Total amount collected $= 20 \times 15 = 300$

So, the total amount collected is \textdollar300.

\section*{Question 23}
\textbf{Metadata}

\begin{itemize}
  \item Question ID: P4-DcSub2d\_P4-DcCmp3d\_GPT4.1\_Recreation\_01
  \item Primary KC: DECIMALS | Subtraction | subtracting decimals (up to 2 decimal places)
  \item Secondary KC: DECIMALS | Comparison and ordering | comparing and ordering decimals up to 3 decimal places
  \item Topic: Recreation such as sports, games, exercises, music, movie, dancing, painting, fishing and other recreation activities
  \item Grade: Primary 4
\end{itemize}

\textbf{Solution}

(a) To find out how many more points Lina scored in the first round than in the second round, we subtract the second round score from the first round score:

$7.85 - 6.43 = 1.42$

So, Lina scored $1.42$ more points in the first round than in the second round.

(b) To arrange Lina's scores from lowest to highest, we compare $7.85$, $6.43$, and $7.802$.

First, compare $6.43$, $7.85$, and $7.802$. Since $6.43 < 7.802 < 7.85$,

the order from lowest to highest is:

$6.43$, $7.802$, $7.85$.

\section*{Question 24}
\textbf{Metadata}

\begin{itemize}
  \item Question ID: P3-WNSub4d\_P1-WNCmp\_GPT4.1\_Recreation\_01
  \item Primary KC: WHOLE NUMBERS | Subtraction | subtracting whole numbers up to 4 digits
  \item Secondary KC: WHOLE NUMBERS | Comparison and ordering | comparing and ordering whole numbers
  \item Topic: Recreation such as sports, games, exercises, music, movie, dancing, painting, fishing and other recreation activities
  \item Grade: Primary 3
\end{itemize}

\textbf{Solution}

(a) The number of spectators still in the stadium after the relay race is obtained by subtraction:

\[
3250 - 1875 = 1375
\]

There were $1375$ spectators left after the relay race.

(b) We need to compare $1375$ with $2000$.

Since $1375 < 2000$, the organisers did NOT have enough spectators left to proceed with the next event.

\section*{Question 25}
\textbf{Metadata}

\begin{itemize}
  \item Question ID: P6-RoFndTmWN\_P1-WNAdd2nd\_GPT4.1\_Recreation\_01
  \item Primary KC: RATIO | Finding a missing term | finding the missing term in a pair of equivalent ratios
  \item Secondary KC: WHOLE NUMBERS | Addition | adding whole numbers
  \item Topic: Recreation such as sports, games, exercises, music, movie, dancing, painting, fishing and other recreation activities
  \item Grade: Primary 6
\end{itemize}

\textbf{Solution}

Let the number of boys be $4x$ and the number of girls be $5x$.

After $6$ more girls joined, the number of girls becomes $5x + 6$. The new ratio is $4:7$.

\[
\frac{4x}{5x + 6} = \frac{4}{7}
\]

Cross-multiply:
\[
4x \times 7 = 4 \times (5x + 6)
\]
\[
28x = 20x + 24
\]
\[
28x - 20x = 24
\]
\[
8x = 24
\]
\[
x = 3
\]

Therefore, the number of boys is:
\[
4x = 4 \times 3 = 12
\]

\textbf{There are 12 boys in the basketball team.}

\section*{Question 26}
\textbf{Metadata}

\begin{itemize}
  \item Question ID: P4-WNDiv4d1d\_P1-WNMul2nd\_GPT4.1\_Recreation\_01
  \item Primary KC: WHOLE NUMBERS | Division | dividing whole numbers up to 4 digits by 1 digit
  \item Secondary KC: WHOLE NUMBERS | Multiplication | multiplying whole numbers
  \item Topic: Recreation such as sports, games, exercises, music, movie, dancing, painting, fishing and other recreation activities
  \item Grade: Primary 4
\end{itemize}

\textbf{Solution}

(a) To find out how many classes can be formed, divide $864$ students by $6$ students per class:\\

\[
864 \div 6 = 144
\]

So, $144$ classes can be formed.\\

(b) For the total number of music stands needed, multiply the number of classes by $12$ stands per class:\\

\[
144 \times 12 = 1,728
\]

Thus, $1,728$ music stands are needed for all the classes.

\section*{Question 27}
\textbf{Metadata}

\begin{itemize}
  \item Question ID: P5-FrMulImIm\_P2-FrAdd2nd\_GPT4.1\_Recreation\_01
  \item Primary KC: FRACTIONS | Multiplication | multiplying two improper fractions
  \item Secondary KC: FRACTIONS | Addition | adding fractions
  \item Topic: Recreation such as sports, games, exercises, music, movie, dancing, painting, fishing and other recreation activities
  \item Grade: Primary 5
\end{itemize}

\textbf{Solution}

(a) Total hours spent playing badminton:

To find the total hours, multiply the duration of each game by the number of games:

$$\frac{7}{3} \times \frac{5}{2} = \frac{7 \times 5}{3 \times 2} = \frac{35}{6}$$

So, they spent $\frac{35}{6}$ hours playing badminton.

(b) Total time for snack breaks:

Add the two break durations:

$$\frac{2}{3} + \frac{3}{4}$$
Find a common denominator, which is $12$.
$$\frac{2}{3} = \frac{8}{12}$$
$$\frac{3}{4} = \frac{9}{12}$$
Add:
$$\frac{8}{12} + \frac{9}{12} = \frac{17}{12}$$

So, they spent $\frac{17}{12}$ hours on breaks.

(c) Total time at the gathering:

Add the total playing time and total break time:

$$\frac{35}{6} + \frac{17}{12}$$
Find a common denominator, which is $12$.
$$\frac{35}{6} = \frac{70}{12}$$
$$\frac{17}{12} = \frac{17}{12}$$
Add:
$$\frac{70}{12} + \frac{17}{12} = \frac{87}{12}$$

So, the total time spent at the gathering was $\frac{87}{12}$ hours.

\section*{Question 28}
\textbf{Metadata}

\begin{itemize}
  \item Question ID: P3-WNAdd4d\_P1-WNCmp\_GPT4.1\_Recreation\_01
  \item Primary KC: WHOLE NUMBERS | Addition | adding whole numbers up to 4 digits
  \item Secondary KC: WHOLE NUMBERS | Comparison and ordering | comparing and ordering whole numbers
  \item Topic: Recreation such as sports, games, exercises, music, movie, dancing, painting, fishing and other recreation activities
  \item Grade: Primary 3
\end{itemize}

\textbf{Solution}

(a) The total number of points for all three classes is:

$1,234 + 956 + 1,087 = 2,190 + 1,087 = 3,277$

So, the total number of points is $3,277$.

(b) Comparing the points:

\begin{align*}
&\text{Class A: } 1,234 \\
&\text{Class B: } 956 \\
&\text{Class C: } 1,087 
\end{align*}

Class A had the most points ($1,234$), and Class B had the least points ($956$).

\section*{Question 29}
\textbf{Metadata}

\begin{itemize}
  \item Question ID: P4-FrSubU12\_P2-FrAdd2nd\_GPT4.1\_Recreation\_01
  \item Primary KC: FRACTIONS | Subtraction | subtracting unlike fractions with two different denominators not exceeding 12
  \item Secondary KC: FRACTIONS | Addition | adding fractions
  \item Topic: Recreation such as sports, games, exercises, music, movie, dancing, painting, fishing and other recreation activities
  \item Grade: Primary 4
\end{itemize}

\textbf{Solution}

(a) Elaine spent $\frac{5}{6}$ hour painting and $\frac{1}{4}$ hour playing the piano.

To find the total time spent, add the two fractions:

$\frac{5}{6} + \frac{1}{4}$

Find the lowest common denominator (LCD) for $6$ and $4$, which is $12$.

$\frac{5}{6} = \frac{5 \times 2}{6 \times 2} = \frac{10}{12}$

$\frac{1}{4} = \frac{1 \times 3}{4 \times 3} = \frac{3}{12}$

Add the two fractions:

$\frac{10}{12} + \frac{3}{12} = \frac{13}{12}$

Elaine spent a total of $\frac{13}{12}$ hours on painting and playing the piano.


(b) To find how much longer she spent painting than playing the piano:

$\frac{5}{6} - \frac{1}{4}$

Using the LCD $12$:

$\frac{5}{6} = \frac{10}{12}$

$\frac{1}{4} = \frac{3}{12}$

$\frac{10}{12} - \frac{3}{12} = \frac{7}{12}$

Elaine spent $\frac{7}{12}$ hour longer painting than playing the piano.

\section*{Question 30}
\textbf{Metadata}

\begin{itemize}
  \item Question ID: P3-WNDiv3d1d\_P1-WNSub2nd\_GPT4.1\_Recreation\_01
  \item Primary KC: WHOLE NUMBERS | Division | dividing whole numbers up to 3 digits by 1 digit
  \item Secondary KC: WHOLE NUMBERS | Subtraction | subtracting whole numbers
  \item Topic: Recreation such as sports, games, exercises, music, movie, dancing, painting, fishing and other recreation activities
  \item Grade: Primary 3
\end{itemize}

\textbf{Solution}

First, divide $246$ tennis balls equally into boxes that hold $3$ balls each: 

\[
246 \div 3 = 82
\]
So, there are $82$ boxes, each with $3$ tennis balls packed in total. 

The total number of tennis balls in the boxes at first is $246$. Then, $18$ tennis balls were taken out for a game:

\[
246 - 18 = 228
\]

Therefore, $228$ tennis balls are left in the boxes after some are taken out.

\section*{Question 31}
\textbf{Metadata}

\begin{itemize}
  \item Question ID: P4-DcMul2d1d\_P4-DcRnd3d\_GPT4.1\_Recreation\_01
  \item Primary KC: DECIMALS | Multiplication | multiplying decimals (up to 2 decimal places) by a 1-digit whole number
  \item Secondary KC: DECIMALS | Rounding | rounding decimals up to 3 decimal places to the nearest whole number, 1 decimal place and 2 decimal places 
  \item Topic: Recreation such as sports, games, exercises, music, movie, dancing, painting, fishing and other recreation activities
  \item Grade: Primary 4
\end{itemize}

\textbf{Solution}

(a) Total number of hours Emma played badminton in $5$ days:

\[ 1.75 \times 5 = 8.75 \text{ hours} \]

(b) Rounding $8.75$:

- To the nearest whole number: $9$
- To $1$ decimal place: $8.8$

\textbf{Answers:}

(a) $8.75$ hours.

(b) $9$ hours (nearest whole number); $8.8$ hours (nearest $1$ decimal place).

\section*{Question 32}
\textbf{Metadata}

\begin{itemize}
  \item Question ID: O2-RoRepIvP\_P1-WNMul2nd\_GPT4.1\_Recreation\_01
  \item Primary KC: RATIO | Representation and concept | inverse proportion
  \item Secondary KC: WHOLE NUMBERS | Multiplication | multiplying whole numbers
  \item Topic: Recreation such as sports, games, exercises, music, movie, dancing, painting, fishing and other recreation activities
  \item Grade: Secondary O-level 2
\end{itemize}

\textbf{Solution}

Let $x$ be the number of dancers and $t$ be the number of hours required. Since the number of hours is inversely proportional to the number of dancers, we have 

$ x \times t = k $ for some constant $k$. 

For 4 dancers:
$ 4 \times 18 = k \implies k = 72 $

(a) For 6 dancers:
$ 6 \times t = 72 $
$ t = \frac{72}{6} = 12 $

\textbf{Answer:} If there are 6 dancers, they will need 12 hours to complete the choreography.

(b) If 2 more dancers join, the total number of dancers is $6+2=8$.
$ 8 \times t = 72 $
$ t = \frac{72}{8} = 9 $

\textbf{Answer:} If there are 8 dancers, they will need 9 hours to complete the choreography.

\section*{Question 33}
\textbf{Metadata}

\begin{itemize}
  \item Question ID: P4-WNMul4d1d\_P1-WNSub2nd\_GPT4.1\_Recreation\_01
  \item Primary KC: WHOLE NUMBERS | Multiplication | multiplying whole numbers up to 4 digits by 1 digit or up to 3 digits by 2 digits
  \item Secondary KC: WHOLE NUMBERS | Subtraction | subtracting whole numbers
  \item Topic: Recreation such as sports, games, exercises, music, movie, dancing, painting, fishing and other recreation activities
  \item Grade: Primary 4
\end{itemize}

\textbf{Solution}

First, calculate the total number of chairs:

\[
\text{Total chairs} = 125 \times 8 = 1000
\]

Next, subtract the number of students already seated from the total number of chairs to find the number of empty chairs:

\[
\text{Empty chairs} = 1000 - 97 = 903
\]

\textbf{Final answer:}

There are $903$ empty chairs before the performance begins.

\section*{Question 34}
\textbf{Metadata}

\begin{itemize}
  \item Question ID: P6-RoFndRoWN\_P1-WNAdd2nd\_GPT4.1\_Recreation\_01
  \item Primary KC: RATIO | Finding ratio | finding the ratio of two or three given whole numbers
  \item Secondary KC: WHOLE NUMBERS | Addition | adding whole numbers
  \item Topic: Recreation such as sports, games, exercises, music, movie, dancing, painting, fishing and other recreation activities
  \item Grade: Primary 6
\end{itemize}

\textbf{Solution}

(a) The total number of guitar lesson students after new students joined:

$18 + 6 = 24$

So, there are 24 students in guitar lessons.

(b) The number of students in piano lessons is 12. The new ratio is $24:12$.

To simplify, divide both numbers by their greatest common divisor, which is 12:

$24 \div 12 = 2$

$12 \div 12 = 1$

So, the simplest ratio is $2:1$.

\textbf{Answer:}

(a) 24 students signed up for guitar lessons.

(b) The ratio is $2:1$.

\section*{Question 35}
\textbf{Metadata}

\begin{itemize}
  \item Question ID: O1-AgRepnth\_O1-AgEvlEx\_GPT4.1\_Recreation\_01
  \item Primary KC: ALGEBRA | Representation and concept | recognising and representing patterns/relationships by finding an algebraic expression for the nth term
  \item Secondary KC: ALGEBRA | Evaluation | evaluation of algebraic expressions and formulae
  \item Topic: Recreation such as sports, games, exercises, music, movie, dancing, painting, fishing and other recreation activities
  \item Grade: Secondary O-level 1
\end{itemize}

\textbf{Solution}

(a) The number of jumps in week 1 is $12$. Each week, the number of jumps increases by $8$.

This is an arithmetic sequence with first term $a_1 = 12$ and common difference $d = 8$.

The general (nth) term is 
\[
a_n = 12 + (n-1) \times 8
\]

So the algebraic expression is: $a_n = 12 + 8(n-1)$.

(b) For $n = 10$, substitute into the expression:
\[
a_{10} = 12 + 8(10-1)
= 12 + 8 \times 9
= 12 + 72
= 84
\]

Nora will complete $84$ jumps in the 10th week.

\section*{Question 36}
\textbf{Metadata}

\begin{itemize}
  \item Question ID: P5-FrMulImN\_P2-FrSub2nd\_GPT4.1\_Recreation\_01
  \item Primary KC: FRACTIONS | Multiplication | multiplying a proper/improper fraction and a whole number
  \item Secondary KC: FRACTIONS | Subtraction | subtracting fractions
  \item Topic: Recreation such as sports, games, exercises, music, movie, dancing, painting, fishing and other recreation activities
  \item Grade: Primary 5
\end{itemize}

\textbf{Solution}

(a) Each dance routine takes $\frac{3}{4}$ hour. On Monday, students practise 5 dance routines.

$5 \times \frac{3}{4} = \frac{15}{4} = 3\frac{3}{4}$ hours$

On Monday, they spend $3\frac{3}{4}$ hours rehearsing.

(b) On Tuesday, they rehearse $2\frac{1}{2}$ hours.

$3\frac{3}{4} - 2\frac{1}{2}= \frac{15}{4} - \frac{5}{2}$

Convert $\frac{5}{2}$ to quarters: $\frac{5}{2} = \frac{10}{4}$

So,
$\frac{15}{4} - \frac{10}{4} = \frac{5}{4} = 1\frac{1}{4}$

They spend $1\frac{1}{4}$ hours more rehearsing on Monday than on Tuesday.

\section*{Question 37}
\textbf{Metadata}

\begin{itemize}
  \item Question ID: O1-RoRepDc\_P4-DcSub2nd\_GPT4.1\_Recreation\_01
  \item Primary KC: RATIO | Representation and concept | ratios involving decimals
  \item Secondary KC: DECIMALS | Subtraction | subtracting decimals
  \item Topic: Recreation such as sports, games, exercises, music, movie, dancing, painting, fishing and other recreation activities
  \item Grade: Secondary O-level 1
\end{itemize}

\textbf{Solution}

Let the time Mei spent practising free throws be $2.5x$ hours and the time spent on dribbling be $1.5x$ hours.

Total time is:

$$
2.5x + 1.5x = 4x
$$

We know that $4x = 2$ hours:

$$
4x = 2
$$
$$
x = 0.5
$$

So, time on free throws:

$$
2.5x = 2.5 \times 0.5 = 1.25 \text{ hours}
$$

Time on dribbling:

$$
1.5x = 1.5 \times 0.5 = 0.75 \text{ hours}
$$

Now, check the difference:

$$
1.25 - 0.75 = 0.5 \text{ hours}
$$

But the question says that the time spent on dribbling was $0.8$ hours less than on free throws, so there seems to be a discrepancy in the numbers. Let's adjust the problem to match the information.

Let $y$ be the time spent on dribbling.

Time on free throws: $y + 0.8$

Given the ratio:

$$
\frac{y + 0.8}{y} = \frac{2.5}{1.5}
$$

Cross-multiply:

$$
1.5(y + 0.8) = 2.5y
$$
$$
1.5y + 1.2 = 2.5y
$$
$$
1.2 = 2.5y - 1.5y = 1y
$$
$$
y = 1.2
$$

So, dribbling: $1.2$ hours

Free throws: $1.2 + 0.8 = 2.0$ hours

Check the total time: $1.2 + 2.0 = 3.2$ hours.

If you want the total to be 2 hours, perhaps the difference should be $0.5$ as in the first calculation. But for the numbers you gave: Ratio $2.5:1.5$, total $2$ hours, difference $0.5$ hours.

So:
Time spent on free throws: $1.25$ hours

Time spent on dribbling: $0.75$ hours

Mei spent $1.25$ hours on free throws and $0.75$ hours on dribbling.

\section*{Question 38}
\textbf{Metadata}

\begin{itemize}
  \item Question ID: P5-FrMulMixN\_P2-FrAdd2nd\_GPT4.1\_Recreation\_01
  \item Primary KC: FRACTIONS | Multiplication | multiplying a mixed number and a whole number
  \item Secondary KC: FRACTIONS | Addition | adding fractions
  \item Topic: Recreation such as sports, games, exercises, music, movie, dancing, painting, fishing and other recreation activities
  \item Grade: Primary 5
\end{itemize}

\textbf{Solution}

(a) The total number of hours Alyssa spent painting during the camp is:

$2\frac{1}{2} \times 3 = \frac{5}{2} \times 3 = \frac{15}{2} = 7\frac{1}{2}$ hours.

(b) On the last day, she spent $2\frac{1}{2}$ hours painting and $\frac{3}{4}$ hour drawing sketches. The total time is:

$2\frac{1}{2} + \frac{3}{4} = \frac{5}{2} + \frac{3}{4}$

To add these, convert $\frac{5}{2}$ to quarters: $\frac{5}{2} = \frac{10}{4}$

So $\frac{10}{4} + \frac{3}{4} = \frac{13}{4} = 3\frac{1}{4}$

Alyssa spent a total of $3\frac{1}{4}$ hours on art activities on the last day.

\section*{Question 39}
\textbf{Metadata}

\begin{itemize}
  \item Question ID: O3-MXMulSM\_O3-MXAdd\_GPT4.1\_Recreation\_01
  \item Primary KC: MATRICES | Multiplication | product of a scalar quantity and a matrix
  \item Secondary KC: MATRICES | Addition | addition of matrices
  \item Topic: Recreation such as sports, games, exercises, music, movie, dancing, painting, fishing and other recreation activities
  \item Grade: Secondary O-level 3/4
\end{itemize}

\textbf{Solution}

(a) To double every session for Group A, we multiply the matrix by the scalar $2$:

$2 \times \begin{pmatrix} 10 \\ 15 \\ 8 \end{pmatrix} = \begin{pmatrix} 20 \\ 30 \\ 16 \end{pmatrix}$

(b) To find the total for each type of exercise, we add the new Group A matrix to the Group B matrix:

$\begin{pmatrix} 20 \\ 30 \\ 16 \end{pmatrix} + \begin{pmatrix} 7 \\ 12 \\ 14 \end{pmatrix} = \begin{pmatrix} 20+7 \\ 30+12 \\ 16+14 \end{pmatrix} = \begin{pmatrix} 27 \\ 42 \\ 30 \end{pmatrix}$

Final answer: After the promotion, the total number of Swimming, Running, and Cycling sessions done by both groups are $27$, $42$, and $30$, respectively.

\section*{Question 40}
\textbf{Metadata}

\begin{itemize}
  \item Question ID: P6-RoFndDvqWN\_P1-WNSub2nd\_GPT4.1\_Recreation\_01
  \item Primary KC: RATIO | Finding divided quantities | dividing a given quantity in a given ratio
  \item Secondary KC: WHOLE NUMBERS | Subtraction | subtracting whole numbers
  \item Topic: Recreation such as sports, games, exercises, music, movie, dancing, painting, fishing and other recreation activities
  \item Grade: Primary 6
\end{itemize}

\textbf{Solution}

(a) First, find the total number of parts in the ratio:

$3 + 2 + 1 = 6$ parts

Each part is worth $\frac{120}{6} = 20$ points.

Siti's share: $3 \times 20 = 60$ points
Mei's share: $2 \times 20 = 40$ points
Raj's share: $1 \times 20 = 20$ points

(b) Siti spent \textdollar20, so she subtracts this from her share:

$60 - 20 = 40$

Siti had $40$ points left after buying the movie ticket.

\section*{Question 41}
\textbf{Metadata}

\begin{itemize}
  \item Question ID: P3-FrSubRl12\_P2-FrCmp\_GPT4.1\_Recreation\_01
  \item Primary KC: FRACTIONS | Subtraction | subtracting two related fractions within one whole with denominators of given fractions not exceeding 12
  \item Secondary KC: FRACTIONS | Comparison and ordering | comparing and ordering fractions
  \item Topic: Recreation such as sports, games, exercises, music, movie, dancing, painting, fishing and other recreation activities
  \item Grade: Primary 3
\end{itemize}

\textbf{Solution}

First, compare the fractions $\dfrac{5}{8}$ (Saturday) and $\dfrac{1}{8}$ (Sunday). 

$\dfrac{5}{8} > \dfrac{1}{8}$, so Jenny spent more time painting on Saturday.

Now, subtract the lesser fraction from the greater to find the difference:

$\dfrac{5}{8} - \dfrac{1}{8} = \dfrac{5 - 1}{8} = \dfrac{4}{8} = \dfrac{1}{2}$

Jenny spent $\dfrac{1}{2}$ hour more painting on Saturday than on Sunday.

\section*{Question 42}
\textbf{Metadata}

\begin{itemize}
  \item Question ID: P5-FrMulMixN\_P2-FrCmp\_GPT4.1\_Recreation\_01
  \item Primary KC: FRACTIONS | Multiplication | multiplying a mixed number and a whole number
  \item Secondary KC: FRACTIONS | Comparison and ordering | comparing and ordering fractions
  \item Topic: Recreation such as sports, games, exercises, music, movie, dancing, painting, fishing and other recreation activities
  \item Grade: Primary 5
\end{itemize}

\textbf{Solution}

(a) Jonathan's total practice time: \newline 
$2\frac{1}{2} \times 4$ 

First, convert $2\frac{1}{2}$ to an improper fraction: 
$2\frac{1}{2} = \frac{5}{2}$ 

Now multiply by 4:
$\frac{5}{2} \times 4 = \frac{5 \times 4}{2} = \frac{20}{2} = 10$

So, Jonathan spent 10 hours practising piano. 

(b) Emily practised for $9\frac{3}{4}$ hours. Convert this to an improper fraction:
$9\frac{3}{4} = \frac{39}{4}$

Jonathan's practice time was 10 hours. Compare 10 and $9\frac{3}{4}$:
$10 = \frac{40}{4}$, $9\frac{3}{4} = \frac{39}{4}$

Since $\frac{40}{4} > \frac{39}{4}$, Jonathan practised longer than Emily.

\textbf{Answer:} 
(a) Jonathan spent 10 hours practising piano. 
(b) Jonathan spent more time practising his instrument than Emily.

\section*{Question 43}
\textbf{Metadata}

\begin{itemize}
  \item Question ID: P6-RoFndRoWN\_P6-RoSmpWN\_GPT4.1\_Recreation\_01
  \item Primary KC: RATIO | Finding ratio | finding the ratio of two or three given whole numbers
  \item Secondary KC: RATIO | Simplifying | expressing a ratio in its simplest form
  \item Topic: Recreation such as sports, games, exercises, music, movie, dancing, painting, fishing and other recreation activities
  \item Grade: Primary 6
\end{itemize}

\textbf{Solution}

First, we find the ratio of students who played football : basketball : badminton:

Football : Basketball : Badminton $= 24 : 18 : 12$

To simplify, we find the highest common factor of 24, 18, and 12, which is $6$.

$24 \div 6 = 4$

$18 \div 6 = 3$

$12 \div 6 = 2$

So, the simplest form of the ratio is $4 : 3 : 2$.

\section*{Question 44}
\textbf{Metadata}

\begin{itemize}
  \item Question ID: P6-FrDivPP\_P2-FrAdd2nd\_GPT4.1\_Recreation\_01
  \item Primary KC: FRACTIONS | Division | dividing a whole number/proper fraction by a proper fraction
  \item Secondary KC: FRACTIONS | Addition | adding fractions
  \item Topic: Recreation such as sports, games, exercises, music, movie, dancing, painting, fishing and other recreation activities
  \item Grade: Primary 6
\end{itemize}

\textbf{Solution}

Part (a):

To find how many complete painting sets Linda can prepare with $6$ bottles of paint, we divide $6$ by $\frac{3}{4}$:

\[
6 \div \frac{3}{4} = 6 \times \frac{4}{3} = \frac{24}{3} = 8
\]

Linda can prepare $8$ complete painting sets with $6$ bottles of paint.

Part (b):

She receives an additional $\frac{1}{2}$ bottle of paint, so now she has:

\[
6 + \frac{1}{2} = \frac{12}{2} + \frac{1}{2} = \frac{13}{2}
\]

Now, divide the new total by $\frac{3}{4}$:

\[
\frac{13}{2} \div \frac{3}{4} = \frac{13}{2} \times \frac{4}{3} = \frac{52}{6} = \frac{26}{3} \approx 8.67
\]

Linda can make $8$ complete painting sets with $\frac{13}{2}$ bottles of paint (since she cannot prepare a fraction of a complete set).

\textbf{Answers:}

(a) $8$ complete sets

(b) $8$ complete sets (with some paint left over)

\section*{Question 45}
\textbf{Metadata}

\begin{itemize}
  \item Question ID: P4-DcDiv2d1d\_P4-DcRnd3d\_GPT4.1\_Recreation\_01
  \item Primary KC: DECIMALS | Division | dividing decimals (up to 2 decimal places) by a 1-digit whole number
  \item Secondary KC: DECIMALS | Rounding | rounding decimals up to 3 decimal places to the nearest whole number, 1 decimal place and 2 decimal places 
  \item Topic: Recreation such as sports, games, exercises, music, movie, dancing, painting, fishing and other recreation activities
  \item Grade: Primary 4
\end{itemize}

\textbf{Solution}

(a) To find the average time spent on each song, divide the total time by the number of songs:

\[
\text{Average time per song} = \frac{24.75}{5} = 4.95 \text{ minutes}
\]

(b) Rounding $4.95$ minutes:

- To the nearest whole number: $5$ minutes (since $0.95 \geq 0.5$)
- To 1 decimal place: $5.0$ minutes (since the second decimal, $5$, rounds $4.9$ up to $5.0$)
- To 2 decimal places: $4.95$ minutes (already 2 decimal places)

\[
\text{Answers:}
\]

- Nearest whole number: $5$
- 1 decimal place: $5.0$
- 2 decimal places: $4.95$

\section*{Question 46}
\textbf{Metadata}

\begin{itemize}
  \item Question ID: P5-DcMul3dK\_P4-DcCnv2Fr\_GPT4.1\_Recreation\_01
  \item Primary KC: DECIMALS | Multiplication | multiplying decimals (up to 3 decimal places) by 10, 100, 1000 and their multiples
  \item Secondary KC: DECIMALS | Conversion from decimals to fraction | expressing decimals as fractions
  \item Topic: Recreation such as sports, games, exercises, music, movie, dancing, painting, fishing and other recreation activities
  \item Grade: Primary 5
\end{itemize}

\textbf{Solution}

(a) Sarah swam $0.125$ km for each lap and completed $100$ laps. 

Total distance = $0.125 \times 100 = 12.5$ km

(b) To express $12.5$ as a fraction: 

$12.5 = \dfrac{125}{10} = \dfrac{25}{2}$

So, the total distance Sarah swam, as a fraction in its simplest form, is $\dfrac{25}{2}$ kilometres.

\section*{Question 47}
\textbf{Metadata}

\begin{itemize}
  \item Question ID: P5-FrMulPIm\_P2-FrAdd2nd\_GPT4.1\_Recreation\_01
  \item Primary KC: FRACTIONS | Multiplication | multiplying a proper fraction and a proper/improper fractions
  \item Secondary KC: FRACTIONS | Addition | adding fractions
  \item Topic: Recreation such as sports, games, exercises, music, movie, dancing, painting, fishing and other recreation activities
  \item Grade: Primary 5
\end{itemize}

\textbf{Solution}

(a) The fraction of the weekend Jia Wei spent on basketball is \[ \frac{3}{5} \times \frac{2}{3} = \frac{2}{5} \].\newline
(b) Jia Wei spent \( \frac{3}{5} \) of his weekend on badminton and \( \frac{2}{5} \) on basketball. Altogether, he spent \[ \frac{3}{5} + \frac{2}{5} = \frac{5}{5} = 1 \] \newline
So, Jia Wei spent the whole weekend playing badminton and basketball in total.

\section*{Question 48}
\textbf{Metadata}

\begin{itemize}
  \item Question ID: P6-PcFndChg\_P1-WNDiv2nd\_GPT4.1\_Recreation\_01
  \item Primary KC: PERCENTAGE | Finding change | finding percentage increase/decrease
  \item Secondary KC: WHOLE NUMBERS | Division | dividing whole numbers
  \item Topic: Recreation such as sports, games, exercises, music, movie, dancing, painting, fishing and other recreation activities
  \item Grade: Primary 6
\end{itemize}

\textbf{Solution}

(a) Initial number of students $=40$\newline Final number of students $=56$\newline Increase in number of students $=56-40=16$

Percentage increase $= \frac{\text{Increase}}{\text{Original number}} \times 100\% = \frac{16}{40} \times 100\% = 40\%$

So, the percentage increase in the number of students is $40\%$.

(b) Number of practice groups $=7$\newline Total students at end of term $=56$

Number of students per group $= \frac{56}{7} = 8$

So, there were $8$ students in each practice group at the end of the term.

\section*{Question 49}
\textbf{Metadata}

\begin{itemize}
  \item Question ID: P6-AgRepLrEx\_P6-AgEvlLrEx\_GPT4.1\_Recreation\_01
  \item Primary KC: ALGEBRA | Representation and concept | translation of real-world situations into linear algebraic expressions
  \item Secondary KC: ALGEBRA | Evaluation | evaluating linear expressions by substitution
  \item Topic: Recreation such as sports, games, exercises, music, movie, dancing, painting, fishing and other recreation activities
  \item Grade: Primary 6
\end{itemize}

\textbf{Solution}

Let $x$ be the amount Jia Wei saves each week. After 5 weeks, he has saved $5x$.

Each ticket costs \textdollar12, and he needs to buy 3 tickets.

Total amount needed $= 3 \times 12 = \textdollar36$.

Algebraic expression for total savings after 5 weeks: $5x$

To check if Jia Wei will have enough after 5 weeks if $x = 8$:

$5 \times 8 = 40$

He will have \textdollar40 after 5 weeks.

Since \textdollar40 $> $ \textdollar36, Jia Wei will have enough money to buy the 3 tickets.

\section*{Question 50}
\textbf{Metadata}

\begin{itemize}
  \item Question ID: O3-MXMul\_O3-MXSub\_GPT4.1\_Recreation\_01
  \item Primary KC: MATRICES | Multiplication | multiplication of matrices
  \item Secondary KC: MATRICES | Subtraction | subtraction of matrices
  \item Topic: Recreation such as sports, games, exercises, music, movie, dancing, painting, fishing and other recreation activities
  \item Grade: Secondary O-level 3/4
\end{itemize}

\textbf{Solution}

(a) First, subtract matrix $B$ from matrix $A$:

$C = A - B = \begin{pmatrix}14-10 & 3-2 & 7-8 \\ 16-13 & 5-3 & 6-5 \\ 20-18 & 4-6 & 9-10\end{pmatrix} = \begin{pmatrix}4 & 1 & -1 \\ 3 & 2 & 1 \\ 2 & -2 & -1\end{pmatrix}$

So the difference in each category per game is:
- Game 1: 4 more points, 1 more assist, 1 less rebound
- Game 2: 3 more points, 2 more assists, 1 more rebound
- Game 3: 2 more points, 2 less assists, 1 less rebound

(b) To find the weighted performance difference for each game, multiply each row of $C$ by the weights column matrix $W$:

The result is a $3 \times 1$ matrix:

For game 1:
$4 \times 2 + 1 \times 3 + (-1) \times 1 = 8 + 3 - 1 = 10$

For game 2:
$3 \times 2 + 2 \times 3 + 1 \times 1 = 6 + 6 + 1 = 13$

For game 3:
$2 \times 2 + (-2) \times 3 + (-1) \times 1 = 4 - 6 - 1 = -3$

So the weighted performance differences for Andy over Ben in each game are:
$\begin{pmatrix}10 \\ 13 \\ -3\end{pmatrix}$

For game 2, the weighted difference is $13$.

\section*{Question 51}
\textbf{Metadata}

\begin{itemize}
  \item Question ID: P6-FrDivPP\_P3-FrSmp\_GPT4.1\_Recreation\_01
  \item Primary KC: FRACTIONS | Division | dividing a whole number/proper fraction by a proper fraction
  \item Secondary KC: FRACTIONS | Simplifying | expressing a fraction in its simplest form
  \item Topic: Recreation such as sports, games, exercises, music, movie, dancing, painting, fishing and other recreation activities
  \item Grade: Primary 6
\end{itemize}

\textbf{Solution}

(a) To find the number of medals Tim can make, divide the total length of ribbon by the ribbon needed for each medal:
\[
6 \div \frac{3}{4} = 6 \times \frac{4}{3} = \frac{24}{3} = 8
\]
Tim can make $8$ medals.

(b) The answer as a fraction in its simplest form is $\boxed{8}$ medals (which is already a whole number and in its simplest form).

\section*{Question 52}
\textbf{Metadata}

\begin{itemize}
  \item Question ID: P4-WNMul4d1d\_P4-WNRnd5d\_GPT4.1\_Recreation\_01
  \item Primary KC: WHOLE NUMBERS | Multiplication | multiplying whole numbers up to 4 digits by 1 digit or up to 3 digits by 2 digits
  \item Secondary KC: WHOLE NUMBERS | Rounding | rounding whole numbers up to 100000 to the nearest 10, 100 or 1000 
  \item Topic: Recreation such as sports, games, exercises, music, movie, dancing, painting, fishing and other recreation activities
  \item Grade: Primary 4
\end{itemize}

\textbf{Solution}

(a) Rounding 156 to the nearest ten gives $160$. Rounding $47$ to the nearest ten gives $50$.

Estimated total number of dance steps $= 160 \times 50 = 8,000$

(b) The actual total number of dance steps is $156 \times 47$.

Calculate:
$156 \times 47 = (156 \times 40) + (156 \times 7) = 6,240 + 1,092 = 7,332$

So, the actual total number of dance steps is $7,332$.

\section*{Question 53}
\textbf{Metadata}

\begin{itemize}
  \item Question ID: P4-FrSubU12\_P3-FrSmp\_GPT4.1\_Recreation\_01
  \item Primary KC: FRACTIONS | Subtraction | subtracting unlike fractions with two different denominators not exceeding 12
  \item Secondary KC: FRACTIONS | Simplifying | expressing a fraction in its simplest form
  \item Topic: Recreation such as sports, games, exercises, music, movie, dancing, painting, fishing and other recreation activities
  \item Grade: Primary 4
\end{itemize}

\textbf{Solution}

To find how much longer Jasmine spent painting than playing the piano, subtract the time spent playing the piano from the time spent painting:

$\frac{5}{6} - \frac{1}{4}$

First, find a common denominator for 6 and 4. The least common denominator is 12.

Convert $\frac{5}{6}$ to twelfths:
$\frac{5}{6} = \frac{5 \times 2}{6 \times 2} = \frac{10}{12}$

Convert $\frac{1}{4}$ to twelfths:
$\frac{1}{4} = \frac{1 \times 3}{4 \times 3} = \frac{3}{12}$

Subtract:
$\frac{10}{12} - \frac{3}{12} = \frac{7}{12}$

$\frac{7}{12}$ is already in its simplest form.

\textbf{Answer:} Jasmine spent $\frac{7}{12}$ of an hour longer painting than playing the piano.

\section*{Question 54}
\textbf{Metadata}

\begin{itemize}
  \item Question ID: O3-BPOpr\_O3-BPRepPosI\_GPT4.1\_Recreation\_01
  \item Primary KC: BASE AND POWER | Operations | laws of indices
  \item Secondary KC: BASE AND POWER | Representation and concept  | positive indices that is not 1
  \item Topic: Recreation such as sports, games, exercises, music, movie, dancing, painting, fishing and other recreation activities
  \item Grade: Secondary O-level 3/4
\end{itemize}

\textbf{Solution}

(a) Since each song is played $2^n$ times in one week and $n=3$:

\[
2^3 = 8
\]
So, each song is played 8 times in one week.

(b) For all 3 songs, total number of plays is:
\[
3 \times 2^3
\]
Expressing $3 \times 2^3$ as a single term using indices:
This cannot be further combined using index laws since 3 is not a power of 2. But if each friend listens to 2 full rounds per day and they do this for 7 days, then:

- Number of rounds per week per friend: $2 \times 7 = 14$
- Each round has 3 songs, so songs played per friend: $14 \times 3 = 42$

By the problem definition:
- Number of times each song is played per week: $2^3 = 8$
- For 3 songs: $3 \times 2^3 = 24$

So, the answer is $3 \times 2^3 = 24$ plays in total.

(c) If in the next week, they double the number of times they listen to each song, express the new number of plays for each song using indices and simplify.
- New plays per song: $2 \times 2^3 = 2^1 \times 2^3 = 2^{1+3} = 2^4 = 16$ times for each song in one week.

**Summary of answers:**

(a) $8$

(b) $3 \times 2^3 = 24$

(c) $2^4 = 16$

\section*{Question 55}
\textbf{Metadata}

\begin{itemize}
  \item Question ID: P4-DcSub2d\_P4-DcAdd2nd\_GPT4.1\_Recreation\_01
  \item Primary KC: DECIMALS | Subtraction | subtracting decimals (up to 2 decimal places)
  \item Secondary KC: DECIMALS | Addition | adding decimals
  \item Topic: Recreation such as sports, games, exercises, music, movie, dancing, painting, fishing and other recreation activities
  \item Grade: Primary 4
\end{itemize}

\textbf{Solution}

First, let's find the total points Sarah scored in two games:

\[
8.75 + 9.60 = 18.35
\]

Sarah actually has $17.00$ points, so she must have lost some points. Let's find the difference between the total she could have gotten and the points she actually has:

\[
18.35 - 17.00 = 1.35
\]

So, she lost $1.35$ points and actually has $17.00$ points.

Now, let's find how many more points Sarah needs to reach $21.50$ points:

\[
21.50 - 17.00 = 4.50
\]

Sarah needs \textdollar4.50$ more points to reach \textdollar21.50$ points.

\section*{Question 56}
\textbf{Metadata}

\begin{itemize}
  \item Question ID: P6-FrDivPN\_P5-FrCnv2Dc\_GPT4.1\_Recreation\_01
  \item Primary KC: FRACTIONS | Division | dividing a proper fraction by a whole number
  \item Secondary KC: FRACTIONS | Conversion to decimals | expressing fractions as decimals
  \item Topic: Recreation such as sports, games, exercises, music, movie, dancing, painting, fishing and other recreation activities
  \item Grade: Primary 6
\end{itemize}

\textbf{Solution}

To find out how much blue paint each classmate gets, we divide $\dfrac{3}{5}$ by $4$:

\[
\dfrac{3}{5} \div 4 = \dfrac{3}{5} \times \dfrac{1}{4} = \dfrac{3}{20}
\]

So, each classmate gets $\dfrac{3}{20}$ of a bottle.

To express $\dfrac{3}{20}$ as a decimal:

\[
\dfrac{3}{20} = 0.15
\]

**Answer:** Each classmate gets $\dfrac{3}{20}$ of a bottle, which is $0.15$ of a bottle.

\section*{Question 57}
\textbf{Metadata}

\begin{itemize}
  \item Question ID: P6-RoFndTmWN\_P1-WNSub2nd\_GPT4.1\_Recreation\_01
  \item Primary KC: RATIO | Finding a missing term | finding the missing term in a pair of equivalent ratios
  \item Secondary KC: WHOLE NUMBERS | Subtraction | subtracting whole numbers
  \item Topic: Recreation such as sports, games, exercises, music, movie, dancing, painting, fishing and other recreation activities
  \item Grade: Primary 6
\end{itemize}

\textbf{Solution}

Let the number of matches Matthew wins be $5x$ and the number of matches Tom wins be $3x$. 

The total number of matches they win is given by
$5x + 3x = 8x = 72$
$\Rightarrow x = \frac{72}{8} = 9$

So,
Number of matches Matthew wins $= 5x = 5 \times 9 = 45$
Number of matches Tom wins $= 3x = 3 \times 9 = 27$

The difference in the number of matches they win:
$45 - 27 = 18$

Therefore, Matthew won $\boxed{45}$ matches.

\section*{Question 58}
\textbf{Metadata}

\begin{itemize}
  \item Question ID: P5-FrAddMix\_P2-FrCmp\_GPT4.1\_Recreation\_01
  \item Primary KC: FRACTIONS | Addition | adding mixed numbers
  \item Secondary KC: FRACTIONS | Comparison and ordering | comparing and ordering fractions
  \item Topic: Recreation such as sports, games, exercises, music, movie, dancing, painting, fishing and other recreation activities
  \item Grade: Primary 5
\end{itemize}

\textbf{Solution}

(a) Jing Wen's total practice time is 

$1\frac{1}{2} + 2\frac{1}{4} = \frac{3}{2} + \frac{9}{4}$\
First, express both fractions with a common denominator (4):\
$\frac{3}{2} = \frac{6}{4}$\
$\frac{6}{4} + \frac{9}{4} = \frac{15}{4}$\
$\frac{15}{4} = 3\frac{3}{4}$\
So, Jing Wen practised for $3\frac{3}{4}$ hours in total.\
\
(b) Arjun practised for $2\frac{2}{5}$ hours. To compare, convert both to improper fractions:\
$3\frac{3}{4} = \frac{15}{4}$\
$2\frac{2}{5} = \frac{12}{5}$\
Find a common denominator (20):\
$\frac{15}{4} = \frac{75}{20}$\
$\frac{12}{5} = \frac{48}{20}$\
\
$75 > 48$, so Jing Wen practised longer. The difference is $75 - 48 = 27$ parts out of 20, or $\frac{27}{20} = 1\frac{7}{20}$ hours.\
\
\textbf{Answer:}\
(a) Jing Wen practised for $3\frac{3}{4}$ hours.\
(b) Jing Wen practised longer by $1\frac{7}{20}$ hours.

\section*{Question 59}
\textbf{Metadata}

\begin{itemize}
  \item Question ID: P3-FrSubRl12\_P3-FrSmp\_GPT4.1\_Recreation\_01
  \item Primary KC: FRACTIONS | Subtraction | subtracting two related fractions within one whole with denominators of given fractions not exceeding 12
  \item Secondary KC: FRACTIONS | Simplifying | expressing a fraction in its simplest form
  \item Topic: Recreation such as sports, games, exercises, music, movie, dancing, painting, fishing and other recreation activities
  \item Grade: Primary 3
\end{itemize}

\textbf{Solution}

First, find the difference between the fractions of the set Alicia had and what she gave away:

$\frac{7}{12} - \frac{1}{4}$

Convert $\frac{1}{4}$ to a fraction with denominator 12:
$\frac{1}{4} = \frac{3}{12}$

Now subtract:
$\frac{7}{12} - \frac{3}{12} = \frac{4}{12}$

Simplify $\frac{4}{12}$ by dividing both numerator and denominator by 4:
$\frac{4 \div 4}{12 \div 4} = \frac{1}{3}$

Therefore, Alicia still has $\frac{1}{3}$ of the set.

\section*{Question 60}
\textbf{Metadata}

\begin{itemize}
  \item Question ID: O1-PcFndRslt\_P1-WNDiv2nd\_GPT4.1\_Recreation\_01
  \item Primary KC: PERCENTAGE | Finding result after change | increasing/decreasing a quantity by a given percentage
  \item Secondary KC: WHOLE NUMBERS | Division | dividing whole numbers
  \item Topic: Recreation such as sports, games, exercises, music, movie, dancing, painting, fishing and other recreation activities
  \item Grade: Secondary O-level 1
\end{itemize}

\textbf{Solution}

First, let's find the new number of students after the $25\%$ increase. 

$25\%$ of $40$ is: 

$\frac{25}{100} \times 40 = 10$

So, the new class size is: 

$40 + 10 = 50$

Now, we divide the $50$ students equally into groups of $5$:

$50 \div 5 = 10$

\textbf{Answer:} $10$ music groups can be formed after the increase in the number of students.

\section*{Question 61}
\textbf{Metadata}

\begin{itemize}
  \item Question ID: O1-RoRepFr\_P2-FrAdd2nd\_GPT4.1\_Recreation\_01
  \item Primary KC: RATIO | Representation and concept | ratios involving fractions
  \item Secondary KC: FRACTIONS | Addition | adding fractions
  \item Topic: Recreation such as sports, games, exercises, music, movie, dancing, painting, fishing and other recreation activities
  \item Grade: Secondary O-level 1
\end{itemize}

\textbf{Solution}

(a) The ratio of the fraction of time Evelyn spent dancing to the fraction of time Mark spent dancing is:

\[
\text{Ratio} = \frac{\frac{3}{5}}{\frac{2}{3}} = \frac{3}{5} \div \frac{2}{3} = \frac{3}{5} \times \frac{3}{2} = \frac{9}{10}
\]

So, $k = \frac{9}{10}$ and the ratio is $\frac{9}{10}:1$.

(b) The total fraction of their combined recreation time spent dancing is:

\[
\text{Total fraction} = \frac{3}{5} + \frac{2}{3}
\]
Find a common denominator for $5$ and $3$, which is $15$:

\[
\frac{3}{5} = \frac{9}{15},\quad \frac{2}{3} = \frac{10}{15}
\]
\[
\frac{9}{15} + \frac{10}{15} = \frac{19}{15}
\]

So, the total fraction is $\frac{19}{15}$.

Final answers:

(a) $k = \frac{9}{10}$

(b) $\frac{19}{15}$ of their combined recreation time was spent dancing.

\section*{Question 62}
\textbf{Metadata}

\begin{itemize}
  \item Question ID: P6-FrDivPN\_P2-FrCmp\_GPT4.1\_Recreation\_01
  \item Primary KC: FRACTIONS | Division | dividing a proper fraction by a whole number
  \item Secondary KC: FRACTIONS | Comparison and ordering | comparing and ordering fractions
  \item Topic: Recreation such as sports, games, exercises, music, movie, dancing, painting, fishing and other recreation activities
  \item Grade: Primary 6
\end{itemize}

\textbf{Solution}

(a) Each painter receives $\frac{3}{4} \div 5$ litre of blue paint.

$\frac{3}{4} \div 5 = \frac{3}{4} \times \frac{1}{5} = \frac{3}{20}$.

Each painter gets $\frac{3}{20}$ litre of blue paint.

(b) To compare $\frac{3}{20}$, $\frac{1}{10}$, and $\frac{1}{5}$, express all with a common denominator, $20$:

- $\frac{3}{20}$ (already denominator 20),
- $\frac{1}{10} = \frac{2}{20}$,
- $\frac{1}{5} = \frac{4}{20}$.

Arranging in ascending order:

$\frac{1}{10}$ (=$\frac{2}{20}$), $\frac{3}{20}$, $\frac{1}{5}$ (=$\frac{4}{20}$).

\textbf{Answer:}

(a) $\frac{3}{20}$ litre of blue paint per painter.

(b) Ascending order: $\frac{1}{10}$, $\frac{3}{20}$, $\frac{1}{5}$. 

\section*{Question 63}
\textbf{Metadata}

\begin{itemize}
  \item Question ID: P4-DcAdd2d\_P4-DcCnv2Fr\_GPT4.1\_Recreation\_01
  \item Primary KC: DECIMALS | Addition | adding decimals (up to 2 decimal places)
  \item Secondary KC: DECIMALS | Conversion from decimals to fraction | expressing decimals as fractions
  \item Topic: Recreation such as sports, games, exercises, music, movie, dancing, painting, fishing and other recreation activities
  \item Grade: Primary 4
\end{itemize}

\textbf{Solution}

(a) To find the total score, add the scores from both games:

\[
78.35 + 56.45 = 134.80
\]

Rachel's total score for both games is $134.80$.

(b) To express $134.80$ as a fraction:

$134.80 = 134 + 0.80$

$0.80$ as a fraction is $\frac{80}{100} = \frac{4}{5}$.

Therefore, $134.80 = 134\frac{4}{5}$, or as an improper fraction:

$134\frac{4}{5} = \frac{134 \times 5 + 4}{5} = \frac{670 + 4}{5} = \frac{674}{5}$.

So, Rachel's total score as a fraction in simplest form is $\frac{674}{5}$.

\section*{Question 64}
\textbf{Metadata}

\begin{itemize}
  \item Question ID: P5-DcMul3dK\_P4-DcCmp3d\_GPT4.1\_Recreation\_01
  \item Primary KC: DECIMALS | Multiplication | multiplying decimals (up to 3 decimal places) by 10, 100, 1000 and their multiples
  \item Secondary KC: DECIMALS | Comparison and ordering | comparing and ordering decimals up to 3 decimal places
  \item Topic: Recreation such as sports, games, exercises, music, movie, dancing, painting, fishing and other recreation activities
  \item Grade: Primary 5
\end{itemize}

\textbf{Solution}

(a) The allowed performance time for each finalist in the final round is: \
\[ 1.250 \times 10 = 12.500 \text{ minutes} \]\
So, each finalist is allowed \textdollar12.500 minutes to perform in the final round.\\
(b) The actual performance times are: $12.5$, $13.000$, and $12.750$ minutes.\
To compare and order these times (aligning them to 3 decimal places):\
\[ 12.500,\ 12.750,\ 13.000 \]
Therefore, the order from shortest to longest is:
$12.5$ minutes, $12.750$ minutes, $13.000$ minutes.

\section*{Question 65}
\textbf{Metadata}

\begin{itemize}
  \item Question ID: O3-BPOpr\_O3-BPRepFrI\_GPT4.1\_Recreation\_01
  \item Primary KC: BASE AND POWER | Operations | laws of indices
  \item Secondary KC: BASE AND POWER | Representation and concept  | fractional indices
  \item Topic: Recreation such as sports, games, exercises, music, movie, dancing, painting, fishing and other recreation activities
  \item Grade: Secondary O-level 3/4
\end{itemize}

\textbf{Solution}

First, the volume for song 2 is $V_2 = 2^2 = 4$. The volume for song 4 (if following the rule directly) would be $V_4 = 2^4 = 16$. However, the instructor wants $V_4$ to be half of $V_2$.\newline
Therefore, $V_4 = \frac{1}{2} \times V_2 = \frac{1}{2} \times 4 = 2$.\newline
Express $2$ in terms of indices of base $2$: $2 = 2^1$.\newline
Alternatively, with fractional indices, $2 = 4^{1/2} = (2^2)^{1/2} = 2^{2 \times 1/2} = 2^1$.\newline
Thus, the required volume for song 4 is $V_4 = (2^2)^{1/2}$, showing the fractional index representation.\newline
\textbf{Final Answer:} $V_4 = (2^2)^{1/2}$ or $2^{2 \times 1/2} = 2^1 = 2$.

\section*{Question 66}
\textbf{Metadata}

\begin{itemize}
  \item Question ID: P5-RtFndU\_P2-DcCnvD2N\_GPT4.1\_Recreation\_01
  \item Primary KC: RATE | Finding number of unit | finding number of units given rate and total amount
  \item Secondary KC: DECIMALS | Conversion to smaller units | converting a measurement from a larger unit in decimal form to a smaller unit
  \item Topic: Recreation such as sports, games, exercises, music, movie, dancing, painting, fishing and other recreation activities
  \item Grade: Primary 5
\end{itemize}

\textbf{Solution}

To find the number of bottles Siti filled, we need to know how many times $0.75$ litres fits into $4.5$ litres.

First, convert $4.5$ litres to millilitres:

$1$ litre $= 1000$ millilitres.

$4.5$ litres $= 4.5 \times 1000 = 4500$ millilitres.

Each bottle holds $0.75$ litres $= 0.75 \times 1000 = 750$ millilitres.

Now, divide the total amount of water by the amount one bottle can hold:

$\text{Number of bottles} = \dfrac{4500}{750} = 6$

\textbf{Answer:} Siti filled $6$ bottles.

\section*{Question 67}
\textbf{Metadata}

\begin{itemize}
  \item Question ID: P3-WNDiv3d1d\_P1-WNCmp\_GPT4.1\_Recreation\_01
  \item Primary KC: WHOLE NUMBERS | Division | dividing whole numbers up to 3 digits by 1 digit
  \item Secondary KC: WHOLE NUMBERS | Comparison and ordering | comparing and ordering whole numbers
  \item Topic: Recreation such as sports, games, exercises, music, movie, dancing, painting, fishing and other recreation activities
  \item Grade: Primary 3
\end{itemize}

\textbf{Solution}

(a) Number of students in each team: 
\[ 264 \div 6 = 44 \]
So, there were 44 students in each team.

(b) Number of students in each team from the other school:
\[ 275 \div 5 = 55 \]
So, each team in the other school had 55 students.

Comparing the number of students per team:
\[ 55 - 44 = 11 \]

The other school had more students in each team by 11 students.

\section*{Question 68}
\textbf{Metadata}

\begin{itemize}
  \item Question ID: P3-WNDivRmd3d\_P1-WNCmp\_GPT4.1\_Recreation\_01
  \item Primary KC: WHOLE NUMBERS | Division | dividing whole numbers up to 3 digits by 1 digit with remainder 
  \item Secondary KC: WHOLE NUMBERS | Comparison and ordering | comparing and ordering whole numbers
  \item Topic: Recreation such as sports, games, exercises, music, movie, dancing, painting, fishing and other recreation activities
  \item Grade: Primary 3
\end{itemize}

\textbf{Solution}

(a) To find out how many tables can be filled:

$127 \div 4 = 31$ remainder $3$.

So, 31 tables will be fully occupied, and $3$ children will not get a seat at a full table.

(b) For the checkers room:

$95 \div 4 = 23$ remainder $3$.

So, 23 tables will be fully occupied, and $3$ children will not get a seat at a full table.

Comparison:
$3$ children in the chess game and $3$ children in the checkers game are left without a seat at a full table. Thus, the number is the same, and neither group has more children without a seat. The difference is $3 - 3 = 0$.

\textbf{Answer:} Both games have the same number of children ($3$) left without a seat at a full table.

\section*{Question 69}
\textbf{Metadata}

\begin{itemize}
  \item Question ID: P4-DcMul2d1d\_P4-DcAdd2nd\_GPT4.1\_Recreation\_01
  \item Primary KC: DECIMALS | Multiplication | multiplying decimals (up to 2 decimal places) by a 1-digit whole number
  \item Secondary KC: DECIMALS | Addition | adding decimals
  \item Topic: Recreation such as sports, games, exercises, music, movie, dancing, painting, fishing and other recreation activities
  \item Grade: Primary 4
\end{itemize}

\textbf{Solution}

First, find the total number of players: \newline
$4$ friends + Amy herself $= 5$ players.\newline
Each player needs $3$ golf balls, so total golf balls needed: \newline
$5 \times 3 = 15$ golf balls.\newline
Each golf ball costs $\textdollar1.25$, so the total cost for golf balls is: \newline
$15 \times 1.25 = 18.75$\newline
Total cost for golf balls = $\textdollar18.75$.\newline
Each player also needs a scorecard at $\textdollar0.80$ each:\newline
$5 \times 0.80 = 4.00$\newline
Total cost for scorecards = $\textdollar4.00$.\newline
Add both costs together to find the total spent:\newline
$18.75 + 4.00 = 22.75$\newline
Amy spends $\textdollar22.75$ in total on the golf balls and scorecards.

\section*{Question 70}
\textbf{Metadata}

\begin{itemize}
  \item Question ID: P5-FrAddMix\_P3-FrSmp\_GPT4.1\_Recreation\_01
  \item Primary KC: FRACTIONS | Addition | adding mixed numbers
  \item Secondary KC: FRACTIONS | Simplifying | expressing a fraction in its simplest form
  \item Topic: Recreation such as sports, games, exercises, music, movie, dancing, painting, fishing and other recreation activities
  \item Grade: Primary 5
\end{itemize}

\textbf{Solution}

(a) First, add the whole numbers: $1 + 2 = 3$.

Next, add the fractions: $\frac{1}{4} + \frac{2}{3}$.

Find a common denominator. The least common multiple of $4$ and $3$ is $12$.

$\frac{1}{4} = \frac{3}{12}$, $\frac{2}{3} = \frac{8}{12}$

So, $\frac{3}{12} + \frac{8}{12} = \frac{11}{12}$

Add this to the sum of the whole numbers:

$3 + \frac{11}{12} = 3\frac{11}{12}$

(b) $3\frac{11}{12}$ is already in its simplest form because $11$ and $12$ have no common factor except $1$.

\textbf{Final Answer:}

Hui Ling spent a total of $3\frac{11}{12}$ hours playing both sports.

\section*{Question 71}
\textbf{Metadata}

\begin{itemize}
  \item Question ID: P5-FrMulImN\_P3-FrSmp\_GPT4.1\_Recreation\_01
  \item Primary KC: FRACTIONS | Multiplication | multiplying a proper/improper fraction and a whole number
  \item Secondary KC: FRACTIONS | Simplifying | expressing a fraction in its simplest form
  \item Topic: Recreation such as sports, games, exercises, music, movie, dancing, painting, fishing and other recreation activities
  \item Grade: Primary 5
\end{itemize}

\textbf{Solution}

Sarah practises $\frac{7}{4}$ hours each day for 5 days.

Total hours $= 5 \times \frac{7}{4} = \frac{35}{4}$ hours.

To express $\frac{35}{4}$ as a mixed number:

$\frac{35}{4} = 8\frac{3}{4}$

So, Sarah spends a total of $8\frac{3}{4}$ hours practising the piano in 5 days.

\section*{Question 72}
\textbf{Metadata}

\begin{itemize}
  \item Question ID: P5-FrMulImN\_P2-FrCmp\_GPT4.1\_Recreation\_01
  \item Primary KC: FRACTIONS | Multiplication | multiplying a proper/improper fraction and a whole number
  \item Secondary KC: FRACTIONS | Comparison and ordering | comparing and ordering fractions
  \item Topic: Recreation such as sports, games, exercises, music, movie, dancing, painting, fishing and other recreation activities
  \item Grade: Primary 5
\end{itemize}

\textbf{Solution}

(a) The total time Sarah spends practicing is:

$5 \times \frac{3}{4} = \frac{15}{4}$ hours.

$\frac{15}{4}$ hours can be converted to a mixed number:

$15 \div 4 = 3$ remainder $3$, so $\frac{15}{4} = 3\frac{3}{4}$ hours.

Sarah spends $3\frac{3}{4}$ hours practicing badminton in total.

(b) Compare $3\frac{3}{4}$ hours to Megan's $3$ hours.

$3\frac{3}{4}$ hours $= \frac{15}{4}$, $3$ hours $= \frac{12}{4}$.

$\frac{15}{4} - \frac{12}{4} = \frac{3}{4}$

Sarah spends $\frac{3}{4}$ hour more than Megan.

Therefore, Sarah spent more time practicing badminton by $\frac{3}{4}$ hour.

\section*{Question 73}
\textbf{Metadata}

\begin{itemize}
  \item Question ID: P3-FrSubRl12\_P2-FrAdd2nd\_GPT4.1\_Recreation\_01
  \item Primary KC: FRACTIONS | Subtraction | subtracting two related fractions within one whole with denominators of given fractions not exceeding 12
  \item Secondary KC: FRACTIONS | Addition | adding fractions
  \item Topic: Recreation such as sports, games, exercises, music, movie, dancing, painting, fishing and other recreation activities
  \item Grade: Primary 3
\end{itemize}

\textbf{Solution}

First, find the total time Sarah spent painting by adding $\dfrac{3}{4}$ hour and $\dfrac{1}{6}$ hour.

To add $\dfrac{3}{4} + \dfrac{1}{6}$, make their denominators the same. The lowest common denominator of $4$ and $6$ is $12$:

$\dfrac{3}{4} = \dfrac{3 \times 3}{4 \times 3} = \dfrac{9}{12}$

$\dfrac{1}{6} = \dfrac{1 \times 2}{6 \times 2} = \dfrac{2}{12}$

So, $\dfrac{9}{12} + \dfrac{2}{12} = \dfrac{11}{12}$

Sarah spent a total of $\dfrac{11}{12}$ hour painting.

Now, find out how much more time she could have painted before the $1$ hour ended by subtracting:

$1 - \dfrac{11}{12}$

Express $1$ as $\dfrac{12}{12}$:

$\dfrac{12}{12} - \dfrac{11}{12} = \dfrac{1}{12}$

Sarah could have painted for $\dfrac{1}{12}$ more hour before the workshop ended.

\section*{Question 74}
\textbf{Metadata}

\begin{itemize}
  \item Question ID: P5-RtFndT\_P2-DcCnvD2N\_GPT4.1\_Recreation\_01
  \item Primary KC: RATE | Finding total amount | finding total amount, given rate and number of units
  \item Secondary KC: DECIMALS | Conversion to smaller units | converting a measurement from a larger unit in decimal form to a smaller unit
  \item Topic: Recreation such as sports, games, exercises, music, movie, dancing, painting, fishing and other recreation activities
  \item Grade: Primary 5
\end{itemize}

\textbf{Solution}

First, we find the total cost using the given rate and time spent:

\[
\text{Total cost} = 3.50 \times 2.5 = 8.75
\]

Clara paid \textdollar8.75 in total.

Next, we convert \textdollar8.75 to cents:

\[
\text{textdollar}8.75 = 8.75 \times 100 = 875 \text{ cents}
\]

\textbf{Final Answer:} Clara paid \boxed{875} cents in total for the court rental.

\section*{Question 75}
\textbf{Metadata}

\begin{itemize}
  \item Question ID: P3-WNMul3d1d\_P1-WNAdd2nd\_GPT4.1\_Recreation\_01
  \item Primary KC: WHOLE NUMBERS | Multiplication | multiplying whole numbers up to 3 digits by 1 digit
  \item Secondary KC: WHOLE NUMBERS | Addition | adding whole numbers
  \item Topic: Recreation such as sports, games, exercises, music, movie, dancing, painting, fishing and other recreation activities
  \item Grade: Primary 3
\end{itemize}

\textbf{Solution}

First, find the total number of students in the 4 groups:

\[
125 \times 4 = 500
\]

Next, add the 38 new students who joined:

\[
500 + 38 = 538
\]

So, there are \textdollar538 students in the camp now.

\section*{Question 76}
\textbf{Metadata}

\begin{itemize}
  \item Question ID: O1-RoRepFr\_O1-RoSmpFr\_GPT4.1\_Recreation\_01
  \item Primary KC: RATIO | Representation and concept | ratios involving fractions
  \item Secondary KC: RATIO | Simplifying | converting a ratio involving fractions to its simplest form
  \item Topic: Recreation such as sports, games, exercises, music, movie, dancing, painting, fishing and other recreation activities
  \item Grade: Secondary O-level 1
\end{itemize}

\textbf{Solution}

First, write the ratio: $\frac{3}{4}:\frac{5}{6}$.

To simplify, find a common denominator for the fractions. The least common denominator of $4$ and $6$ is $12$.

Convert each fraction:

$\frac{3}{4} = \frac{3 \times 3}{4 \times 3} = \frac{9}{12}$

$\frac{5}{6} = \frac{5 \times 2}{6 \times 2} = \frac{10}{12}$

So, the ratio becomes $\frac{9}{12}:\frac{10}{12}$.

Remove the denominators by multiplying both parts of the ratio by $12$:

$9:10$

Hence, the simplest whole number ratio is $9:10$.

\section*{Question 77}
\textbf{Metadata}

\begin{itemize}
  \item Question ID: O1-RoRepDc\_O1-RoSmpDc\_GPT4.1\_Recreation\_01
  \item Primary KC: RATIO | Representation and concept | ratios involving decimals
  \item Secondary KC: RATIO | Simplifying | converting a ratio involving decimals to its simplest form
  \item Topic: Recreation such as sports, games, exercises, music, movie, dancing, painting, fishing and other recreation activities
  \item Grade: Secondary O-level 1
\end{itemize}

\textbf{Solution}

(a) The ratio of blue paint to red paint Benjamin used is $3.5 : 5.6$.

(b) To simplify $3.5 : 5.6$, divide both numbers by $1.4$ (their greatest common divisor):

\[
\frac{3.5}{1.4} = 2.5,\quad \frac{5.6}{1.4} = 4
\]

So, the simplified ratio is $2.5 : 4$. We can further multiply both terms by $2$ to clear the decimal:

\[
2.5 \times 2 = 5,\quad 4 \times 2 = 8
\]

Hence, the simplest whole number form is $5 : 8$.

\section*{Question 78}
\textbf{Metadata}

\begin{itemize}
  \item Question ID: O2-SPFndmean\_O3-BPRepSN\_GPT4.1\_Recreation\_01
  \item Primary KC: STATISTICS AND PROBABILITY | Finding mean deviation | calculation of the mean for a set of data
  \item Secondary KC: BASE AND POWER | Representation and concept  | use of standard form Ax10^n , where n is an integer, and 1<= A<= 10
  \item Topic: Recreation such as sports, games, exercises, music, movie, dancing, painting, fishing and other recreation activities
  \item Grade: Secondary O-level 2
\end{itemize}

\textbf{Solution}

(a) To find the mean, add all the numbers and divide by 5:

\begin{align*}
(3.2 \times 10^2) + (2.7 \times 10^2) + (4.0 \times 10^2) + (3.5 \times 10^2) + (2.6 \times 10^2) &= (3.2 + 2.7 + 4.0 + 3.5 + 2.6) \times 10^2 \\
&= 16.0 \times 10^2 \\
\end{align*}

So the total is $16.0 \times 10^2$ minutes.

The mean is:

\begin{align*}
\text{Mean} &= \frac{16.0 \times 10^2}{5} \\
&= 3.2 \times 10^2
\end{align*}

So, the mean number of minutes is $3.2 \times 10^2$ minutes.

(b) The mean deviation is the average of the absolute differences from the mean:

\begin{align*}
|3.2 \times 10^2 - 3.2 \times 10^2| &= 0 \\
|2.7 \times 10^2 - 3.2 \times 10^2| &= 0.5 \times 10^2 \\
|4.0 \times 10^2 - 3.2 \times 10^2| &= 0.8 \times 10^2 \\
|3.5 \times 10^2 - 3.2 \times 10^2| &= 0.3 \times 10^2 \\
|2.6 \times 10^2 - 3.2 \times 10^2| &= 0.6 \times 10^2 \\
\end{align*}

Now, sum them:

$0 + 0.5 \times 10^2 + 0.8 \times 10^2 + 0.3 \times 10^2 + 0.6 \times 10^2 = (0 + 0.5 + 0.8 + 0.3 + 0.6) \times 10^2 = 2.2 \times 10^2$

The mean deviation = $\frac{2.2 \times 10^2}{5} = 0.44 \times 10^2 = 4.4 \times 10^1$

**Final Answers:**

(a) $3.2 \times 10^2$ minutes

(b) $4.4 \times 10^1$ minutes

\section*{Question 79}
\textbf{Metadata}

\begin{itemize}
  \item Question ID: P4-WNDiv4d1d\_P1-WNSub2nd\_GPT4.1\_Recreation\_01
  \item Primary KC: WHOLE NUMBERS | Division | dividing whole numbers up to 4 digits by 1 digit
  \item Secondary KC: WHOLE NUMBERS | Subtraction | subtracting whole numbers
  \item Topic: Recreation such as sports, games, exercises, music, movie, dancing, painting, fishing and other recreation activities
  \item Grade: Primary 4
\end{itemize}

\textbf{Solution}

(a) To find the number of rows needed, divide the total number of children by the number of children per row:

\[
\text{Number of rows} = \frac{2,448}{6} = 408
\]

So, the organisers needed 408 rows.

(b) To find how many children were still at the concert:

\[
\text{Children remaining} = 2,448 - 253 = 2,195
\]

So, 2,195 children were still at the concert after some left.

\section*{Question 80}
\textbf{Metadata}

\begin{itemize}
  \item Question ID: O1-RoRepFr\_P2-FrSub2nd\_GPT4.1\_Recreation\_01
  \item Primary KC: RATIO | Representation and concept | ratios involving fractions
  \item Secondary KC: FRACTIONS | Subtraction | subtracting fractions
  \item Topic: Recreation such as sports, games, exercises, music, movie, dancing, painting, fishing and other recreation activities
  \item Grade: Secondary O-level 1
\end{itemize}

\textbf{Solution}

(a) Fraction of holiday spent playing the piano and painting:

\[
\frac{2}{5} + \frac{1}{4} = \frac{8}{20} + \frac{5}{20} = \frac{13}{20}
\]

Mei Lin spent $\frac{13}{20}$ of her holiday on both activities combined.

(b) Ratio of time spent playing the piano to painting:

\[
\text{Piano : Painting} = \frac{2}{5} : \frac{1}{4}
\]

To write as a ratio:
\[
\frac{2}{5} \div \frac{1}{4} = \frac{2}{5} \times \frac{4}{1} = \frac{8}{5}
\]

So the ratio is $8:5$.

Hence, $k = 8$. The ratio in its simplest form is $8:5$.

\section*{Question 81}
\textbf{Metadata}

\begin{itemize}
  \item Question ID: P3-FrAddRl12\_P3-FrSmp\_GPT4.1\_Recreation\_01
  \item Primary KC: FRACTIONS | Addition | adding two related fractions within one whole with denominators of given fractions not exceeding 12
  \item Secondary KC: FRACTIONS | Simplifying | expressing a fraction in its simplest form
  \item Topic: Recreation such as sports, games, exercises, music, movie, dancing, painting, fishing and other recreation activities
  \item Grade: Primary 3
\end{itemize}

\textbf{Solution}

Emily spent $\frac{3}{12}$ of an hour painting and $\frac{5}{12}$ of an hour drawing.

To find the total time spent, add the fractions:

$\frac{3}{12} + \frac{5}{12} = \frac{3+5}{12} = \frac{8}{12}$

Now, simplify $\frac{8}{12}$:

The greatest common divisor of 8 and 12 is 4.

$\frac{8 \div 4}{12 \div 4} = \frac{2}{3}$

So, Emily spent $\frac{2}{3}$ of an hour in total on painting and drawing.

\section*{Question 82}
\textbf{Metadata}

\begin{itemize}
  \item Question ID: O3-BPOpr\_O3-BPRepNegI\_GPT4.1\_Recreation\_01
  \item Primary KC: BASE AND POWER | Operations | laws of indices
  \item Secondary KC: BASE AND POWER | Representation and concept  | negative indices
  \item Topic: Recreation such as sports, games, exercises, music, movie, dancing, painting, fishing and other recreation activities
  \item Grade: Secondary O-level 3/4
\end{itemize}

\textbf{Solution}

(a) The score formula is $S = 3^a \times 3^{-b}$. Substitute $a = 5$ and $b = 2$:

\[
S = 3^5 \times 3^{-2}
\]

Using the law of indices $a^m \times a^n = a^{m+n}$:
\[
S = 3^{5 + (-2)} = 3^{3}
\]

So, expressed as a single power, the score is $3^3$.

(b) Now, calculate the value of $3^3$:
\[
3^3 = 3 \times 3 \times 3 = 27
\]

Therefore, the swimmer's final score is \textdollar27.

\section*{Question 83}
\textbf{Metadata}

\begin{itemize}
  \item Question ID: O1-AgSlvFrLr\_O1-AgRepEq\_GPT4.1\_Recreation\_01
  \item Primary KC: ALGEBRA | Solving | solving simple fractional equations that can be reduced to linear equations
  \item Secondary KC: ALGEBRA | Representation and concept | translation of simple real-world situations to equations
  \item Topic: Recreation such as sports, games, exercises, music, movie, dancing, painting, fishing and other recreation activities
  \item Grade: Secondary O-level 1
\end{itemize}

\textbf{Solution}

Let the number Sophie thinks of be $x$.

According to the problem, adding $3$ to half of her number gives $11$:
\[
\frac{1}{2}x + 3 = 11
\]

Subtract $3$ from both sides:
\[
\frac{1}{2}x = 8
\]

Multiply both sides by $2$ to solve for $x$:
\[
x = 16
\]

Therefore, Sophie thought of the number $16$.

\section*{Question 84}
\textbf{Metadata}

\begin{itemize}
  \item Question ID: O1-PcRep2q\_O1-PcCnv2Dc\_GPT4.1\_Recreation\_01
  \item Primary KC: PERCENTAGE | Representation and concept | comparing two quantities by percentage
  \item Secondary KC: PERCENTAGE | Conversion to decimals | expressing percentage as a decimal
  \item Topic: Recreation such as sports, games, exercises, music, movie, dancing, painting, fishing and other recreation activities
  \item Grade: Secondary O-level 1
\end{itemize}

\textbf{Solution}

(a) Sarah won $15$ out of $20$ matches.

The percentage of matches she won is:
\[
\frac{15}{20} \times 100\% = 75\%
\]

(b) To express this percentage as a decimal, divide $75\%$ by $100$:
\[
75\% = 0.75
\]

**Answer:**

(a) Sarah won $75\%$ of her matches.

(b) The percentage $75\%$ as a decimal is $0.75$.

\section*{Question 85}
\textbf{Metadata}

\begin{itemize}
  \item Question ID: O3-SPAddProb\_O2-SPRepPrSE\_GPT4.1\_Recreation\_01
  \item Primary KC: STATISTICS AND PROBABILITY | Addition | addition of probabilities
  \item Secondary KC: STATISTICS AND PROBABILITY | Representation and concept | probability of single events
  \item Topic: Recreation such as sports, games, exercises, music, movie, dancing, painting, fishing and other recreation activities
  \item Grade: Secondary O-level 3/4
\end{itemize}

\textbf{Solution}

(a) The probability that a student attends either the badminton or the painting session is the sum of the individual probabilities, since no student attends more than one activity: 

$P(\text{badminton or painting}) = P(\text{badminton}) + P(\text{painting})$

$= 0.3 + 0.2 = 0.5$

(b) The probability that a student attends any of the three activities is: 

$P(\text{badminton or painting or movie}) = P(\text{badminton}) + P(\text{painting}) + P(\text{movie})$

$= 0.3 + 0.2 + 0.4 = 0.9$

Therefore, the probability that a student does not attend any of these activities is:

$P(\text{none}) = 1 - P(\text{badminton or painting or movie})$

$= 1 - 0.9 = 0.1$

\section*{Question 86}
\textbf{Metadata}

\begin{itemize}
  \item Question ID: P5-FrMulImIm\_P2-FrCmp\_GPT4.1\_Recreation\_01
  \item Primary KC: FRACTIONS | Multiplication | multiplying two improper fractions
  \item Secondary KC: FRACTIONS | Comparison and ordering | comparing and ordering fractions
  \item Topic: Recreation such as sports, games, exercises, music, movie, dancing, painting, fishing and other recreation activities
  \item Grade: Primary 5
\end{itemize}

\textbf{Solution}

(a) First, find the number of skips Team B completes:
\[
\text{Team B:}\quad 20 \times \frac{7}{3} = \frac{20 \times 7}{3} = \frac{140}{3} \approx 46.67~\text{skips}
\]

Next, find the number of skips Team A completes:
\[
\text{Team A:}\quad \left(\frac{140}{3}\right) \times \frac{9}{4} = \frac{140 \times 9}{3 \times 4} = \frac{1260}{12} = 105~\text{skips}
\]

(b) Comparing the skips:
\begin{align*}
\text{Team C:} &\quad 20~\text{skips} \\
\text{Team B:} &\quad \frac{140}{3} \approx 46.67~\text{skips} \\
\text{Team A:} &\quad 105~\text{skips}
\end{align*}

Order from the least to the most skips:
\[
\text{Team C} < \text{Team B} < \text{Team A}
\]

\section*{Question 87}
\textbf{Metadata}

\begin{itemize}
  \item Question ID: P4-FrAddU12\_P2-FrCmp\_GPT4.1\_Recreation\_01
  \item Primary KC: FRACTIONS | Addition | adding unlike fractions with two different denominators not exceeding 12
  \item Secondary KC: FRACTIONS | Comparison and ordering | comparing and ordering fractions
  \item Topic: Recreation such as sports, games, exercises, music, movie, dancing, painting, fishing and other recreation activities
  \item Grade: Primary 4
\end{itemize}

\textbf{Solution}

(a) To find the total time Sarah spent, add $\frac{2}{5}$ and $\frac{1}{3}$. 

Find a common denominator:

The LCM of $5$ and $3$ is $15$.

Rewrite $\frac{2}{5}$: $\frac{2}{5} = \frac{2 \times 3}{5 \times 3} = \frac{6}{15}$

Rewrite $\frac{1}{3}$: $\frac{1}{3} = \frac{1 \times 5}{3 \times 5} = \frac{5}{15}$

Add them: $\frac{6}{15} + \frac{5}{15} = \frac{11}{15}$

Sarah spent $\frac{11}{15}$ of an hour in total.

(b) To compare $\frac{2}{5}$ and $\frac{1}{3}$, use the common denominator $15$:

$\frac{2}{5} = \frac{6}{15}$ and $\frac{1}{3} = \frac{5}{15}$

Since $6/15 > 5/15$, Sarah spent more time playing the piano.

Order from least to greatest: $\frac{1}{3}$, $\frac{2}{5}$.

\section*{Question 88}
\textbf{Metadata}

\begin{itemize}
  \item Question ID: P4-FrAddU12\_P3-FrSmp\_GPT4.1\_Recreation\_01
  \item Primary KC: FRACTIONS | Addition | adding unlike fractions with two different denominators not exceeding 12
  \item Secondary KC: FRACTIONS | Simplifying | expressing a fraction in its simplest form
  \item Topic: Recreation such as sports, games, exercises, music, movie, dancing, painting, fishing and other recreation activities
  \item Grade: Primary 4
\end{itemize}

\textbf{Solution}

First, we find the sum of the two fractions:

$\frac{3}{8} + \frac{2}{3}$

To add these, we need a common denominator. The denominators are $8$ and $3$, and their least common multiple is $24$.

Express $\frac{3}{8}$ with denominator $24$:

$\frac{3}{8} = \frac{3 \times 3}{8 \times 3} = \frac{9}{24}$

Express $\frac{2}{3}$ with denominator $24$:

$\frac{2}{3} = \frac{2 \times 8}{3 \times 8} = \frac{16}{24}$

Add the fractions:

$\frac{9}{24} + \frac{16}{24} = \frac{25}{24}$

$\frac{25}{24}$ is an improper fraction. It cannot be simplified further, but it can be written as:

$\frac{25}{24} = 1\frac{1}{24}$

Amir spent $\frac{25}{24}$ hours, or $1\frac{1}{24}$ hours, on both activities altogether.

\section*{Question 89}
\textbf{Metadata}

\begin{itemize}
  \item Question ID: P5-FrMulMixN\_P2-FrSub2nd\_GPT4.1\_Recreation\_01
  \item Primary KC: FRACTIONS | Multiplication | multiplying a mixed number and a whole number
  \item Secondary KC: FRACTIONS | Subtraction | subtracting fractions
  \item Topic: Recreation such as sports, games, exercises, music, movie, dancing, painting, fishing and other recreation activities
  \item Grade: Primary 5
\end{itemize}

\textbf{Solution}

(a) Each day, Sarah practised $2\frac{1}{2}$ hours. In 4 days, she practised:

\[
2\frac{1}{2} \times 4 = \left(\frac{5}{2}\right) \times 4 = \frac{5\times4}{2} = \frac{20}{2} = 10\text{ hours}
\]

(b) On Saturday, she practised $1\frac{3}{4}$ hours:

\[
10 - 1\frac{3}{4} = 10 - \frac{7}{4}
\]
Convert 10 to quarters:

\[
10 = \frac{40}{4}
\]
So,
\[
\frac{40}{4} - \frac{7}{4} = \frac{33}{4} = 8\frac{1}{4}
\]

Sarah practised $8\frac{1}{4}$ more hours from Monday to Thursday than on Saturday.

\section*{Question 90}
\textbf{Metadata}

\begin{itemize}
  \item Question ID: P3-WNDivRmd3d\_P1-WNMul2nd\_GPT4.1\_Recreation\_01
  \item Primary KC: WHOLE NUMBERS | Division | dividing whole numbers up to 3 digits by 1 digit with remainder 
  \item Secondary KC: WHOLE NUMBERS | Multiplication | multiplying whole numbers
  \item Topic: Recreation such as sports, games, exercises, music, movie, dancing, painting, fishing and other recreation activities
  \item Grade: Primary 3
\end{itemize}

\textbf{Solution}

Let us solve part (a) first:

To find how many teams can be formed, we divide $124$ by $6$.

$124\div6 = 20$ remainder $4$

This means $20$ teams can be formed, and $4$ children will be left without a team.

For part (b):

If each team needs $4$ paintbrushes, then the total number of paintbrushes needed is

$20 \times 4 = 80$

\textbf{Answers:}

(a) The teacher can form $20$ teams, and $4$ children will be left without a team.

(b) The teacher needs to prepare $80$ paintbrushes.

\section*{Question 91}
\textbf{Metadata}

\begin{itemize}
  \item Question ID: P4-WNDiv4d1d\_P1-WNAdd2nd\_GPT4.1\_Recreation\_01
  \item Primary KC: WHOLE NUMBERS | Division | dividing whole numbers up to 4 digits by 1 digit
  \item Secondary KC: WHOLE NUMBERS | Addition | adding whole numbers
  \item Topic: Recreation such as sports, games, exercises, music, movie, dancing, painting, fishing and other recreation activities
  \item Grade: Primary 4
\end{itemize}

\textbf{Solution}

First, divide the original number of children by the number of children per court:

$1,368 \div 6 = 228$

So, the organisers need $228$ courts.

Next, add the 290 new children to the original number:

$1,368 + 290 = 1,658$

Divide the new total by 6 to see how many courts are needed:

$1,658 \div 6 = 276$ remainder $2$

This means $276$ courts can each have 6 children, and there are 2 children left, so one more court is needed for them. Thus, the organisers will need $277$ courts in total.

Final answers:

- Courts needed at first: $228$
- New total number of children: $1,658$
- Courts needed after more children join: $277$

\section*{Question 92}
\textbf{Metadata}

\begin{itemize}
  \item Question ID: P6-FrDivPP\_P2-FrCmp\_GPT4.1\_Recreation\_01
  \item Primary KC: FRACTIONS | Division | dividing a whole number/proper fraction by a proper fraction
  \item Secondary KC: FRACTIONS | Comparison and ordering | comparing and ordering fractions
  \item Topic: Recreation such as sports, games, exercises, music, movie, dancing, painting, fishing and other recreation activities
  \item Grade: Primary 6
\end{itemize}

\textbf{Solution}

(a) To find how many pieces Jane got, divide the total width by the width of each piece:

$1 \div \frac{1}{4} = 1 \times \frac{4}{1} = 4$

Jane got $4$ pieces.

(b) For Emily:

$2 \div \frac{1}{3} = 2 \times 3 = 6$

Emily got $6$ pieces.

Comparing the number of pieces: Jane had $4$, Emily had $6$.

Arranging from least to greatest: $4, 6$.

Therefore, Emily had more pieces than Jane.

\section*{Question 93}
\textbf{Metadata}

\begin{itemize}
  \item Question ID: P5-RtFndT\_P2-DcCnvN2D\_GPT4.1\_Recreation\_01
  \item Primary KC: RATE | Finding total amount | finding total amount, given rate and number of units
  \item Secondary KC: DECIMALS | Conversion to larger units | converting a measurement from a smaller unit to a larger unit in decimal form
  \item Topic: Recreation such as sports, games, exercises, music, movie, dancing, painting, fishing and other recreation activities
  \item Grade: Primary 5
\end{itemize}

\textbf{Solution}

(a) To convert metres to kilometres, divide the number of metres by $1,000$:

$1,250 \div 1,000 = 1.25$

Farah swims $1.25$ kilometres.

(b) Each $100$ metres costs \textdollar1.20. Find the number of $100$ metre units in $1,250$ metres:

$1,250 \div 100 = 12.5$

Total cost $= 12.5 \times \textdollar1.20 = \textdollar15.00$

Farah has to pay \textdollar15.00 for her training session.

\section*{Question 94}
\textbf{Metadata}

\begin{itemize}
  \item Question ID: P6-RoFndRoWN\_P1-WNDiv2nd\_GPT4.1\_Recreation\_01
  \item Primary KC: RATIO | Finding ratio | finding the ratio of two or three given whole numbers
  \item Secondary KC: WHOLE NUMBERS | Division | dividing whole numbers
  \item Topic: Recreation such as sports, games, exercises, music, movie, dancing, painting, fishing and other recreation activities
  \item Grade: Primary 6
\end{itemize}

\textbf{Solution}

(a) The number of boys is $48$ and the number of girls is $72$. 

To find the ratio of boys to girls:

$\text{Ratio of boys to girls} = 48 : 72$

We simplify the ratio by dividing both numbers by their greatest common divisor, which is $24$:

$48 \div 24 = 2$
$72 \div 24 = 3$

So, the simplest ratio is $2 : 3$.

(b) Total number of participants:
$48 + 72 = 120$

The participants are divided into $6$ equal groups, so the number in each group is:
$120 \div 6 = 20$

\textbf{Answer:} 
(a) The ratio of boys to girls is $2:3$.
(b) There are $20$ participants in each group for the photo.

\section*{Question 95}
\textbf{Metadata}

\begin{itemize}
  \item Question ID: P4-FrRepSet\_P3-FrCnvEq\_GPT4.1\_Recreation\_01
  \item Primary KC: FRACTIONS | Representation and concept | expressing a part of a set as a fraction
  \item Secondary KC: FRACTIONS | Conversion to equivalent fractions | Conversion to equivalent fractions (given either the denominator or the numerator)
  \item Topic: Recreation such as sports, games, exercises, music, movie, dancing, painting, fishing and other recreation activities
  \item Grade: Primary 4
\end{itemize}

\textbf{Solution}

(a) The fraction of the relay race team that are girls is $\dfrac{9}{24}$, since there are $9$ girls out of $24$ students.

(b) To write $\dfrac{9}{24}$ as an equivalent fraction with denominator $8$, we need to find a number $x$ such that $\dfrac{9}{24} = \dfrac{x}{8}$. 

Set up the equation:

$\dfrac{9}{24} = \dfrac{x}{8}$

Cross-multiply:

$9 \times 8 = 24 \times x$

$72 = 24x$

$x = \dfrac{72}{24} = 3$

So, an equivalent fraction is $\dfrac{3}{8}$.

Therefore, $\dfrac{9}{24} = \dfrac{3}{8}$.

\section*{Question 96}
\textbf{Metadata}

\begin{itemize}
  \item Question ID: O2-RoRepDP\_P1-WNDiv2nd\_GPT4.1\_Recreation\_01
  \item Primary KC: RATIO | Representation and concept | direct proportion
  \item Secondary KC: WHOLE NUMBERS | Division | dividing whole numbers
  \item Topic: Recreation such as sports, games, exercises, music, movie, dancing, painting, fishing and other recreation activities
  \item Grade: Secondary O-level 2
\end{itemize}

\textbf{Solution}

(a) Let the number of basketball participants be $B$ and volleyball participants be $V = 48$.

According to the given ratio:
\[
\frac{B}{V} = \frac{5}{3}
\]
\[
B = V \times \frac{5}{3} = 48 \times \frac{5}{3} = 16 \times 5 = 80
\]
So, there are $80$ students participating in basketball.

(b) Each volleyball team has $6$ members. To find the number of full teams:
\[
\text{Number of full teams} = \frac{48}{6} = 8
\]
So, there can be $8$ full volleyball teams.

\section*{Question 97}
\textbf{Metadata}

\begin{itemize}
  \item Question ID: P6-RoFndRoWN\_P1-WNMul2nd\_GPT4.1\_Recreation\_01
  \item Primary KC: RATIO | Finding ratio | finding the ratio of two or three given whole numbers
  \item Secondary KC: WHOLE NUMBERS | Multiplication | multiplying whole numbers
  \item Topic: Recreation such as sports, games, exercises, music, movie, dancing, painting, fishing and other recreation activities
  \item Grade: Primary 6
\end{itemize}

\textbf{Solution}

First, we find the simplest ratio of the number of stickers collected by Amir, Ben, and Cindy. 

Amir : Ben : Cindy $=24:36:60$

To simplify, we find the highest common factor (HCF) of 24, 36, and 60.

The HCF is $12$.

\[
24 \div 12 = 2 \\ 
36 \div 12 = 3 \\ 
60 \div 12 = 5
\]

So, the ratio in simplest form is $2:3:5$.

For the second part, Amir gives each of his 24 stickers to 2 classmates:

Number of classmates $= 24 \times 2 = 48$

Therefore, the number of classmates who will receive a sticker from Amir is $48$. 

\textbf{Answers:}

\begin{itemize}
  \item (a) The simplest ratio is $2:3:5$.
  \item (b) $48$ classmates will receive a sticker from Amir.
\end{itemize}

\section*{Question 98}
\textbf{Metadata}

\begin{itemize}
  \item Question ID: O3-MXMulSM\_O3-MXMul\_GPT4.1\_Recreation\_01
  \item Primary KC: MATRICES | Multiplication | product of a scalar quantity and a matrix
  \item Secondary KC: MATRICES | Multiplication | multiplication of matrices
  \item Topic: Recreation such as sports, games, exercises, music, movie, dancing, painting, fishing and other recreation activities
  \item Grade: Secondary O-level 3/4
\end{itemize}

\textbf{Solution}

(a) To find the new points awarded to each player, multiply matrix $A$ by matrix $B$:

\[
A \times B = \begin{bmatrix} 6 & 4 \\ 3 & 7 \end{bmatrix} \times \begin{bmatrix} 3 & 0 \\ 0 & 2 \end{bmatrix}
\]

Matrix multiplication gives:
\[
\begin{bmatrix}
6 \times 3 + 4 \times 0 & 6 \times 0 + 4 \times 2 \\
3 \times 3 + 7 \times 0 & 3 \times 0 + 7 \times 2
\end{bmatrix}
= \begin{bmatrix}
18 & 8 \\
9 & 14
\end{bmatrix}
\]

So, the new points matrix is:
\[
\begin{bmatrix}
18 & 8 \\
9 & 14
\end{bmatrix}
\]

(b) If the coach doubles every player's total points, multiply the entire matrix by scalar $2$:

\[
2 \times \begin{bmatrix} 18 & 8 \\ 9 & 14 \end{bmatrix} = \begin{bmatrix} 36 & 16 \\ 18 & 28 \end{bmatrix}
\]

Therefore, the final matrix of points is:
\[
\boxed{\begin{bmatrix} 36 & 16 \\ 18 & 28 \end{bmatrix}}
\]

\section*{Question 99}
\textbf{Metadata}

\begin{itemize}
  \item Question ID: P5-RtFndU\_P2-DcCnvN2D\_GPT4.1\_Recreation\_01
  \item Primary KC: RATE | Finding number of unit | finding number of units given rate and total amount
  \item Secondary KC: DECIMALS | Conversion to larger units | converting a measurement from a smaller unit to a larger unit in decimal form
  \item Topic: Recreation such as sports, games, exercises, music, movie, dancing, painting, fishing and other recreation activities
  \item Grade: Primary 5
\end{itemize}

\textbf{Solution}

(a) To find the number of laps Sarah swam: \
$\text{Number of laps} = \frac{1500}{25} = 60$\
Sarah swam $60$ laps.\\
(b) To convert $1500$ metres to kilometres: \
$\text{Distance in kilometres} = \frac{1500}{1000} = 1.5$\
Sarah swam $1.5$ kilometres.

\section*{Question 100}
\textbf{Metadata}

\begin{itemize}
  \item Question ID: P6-RoFndRoWN\_P1-WNSub2nd\_GPT4.1\_Recreation\_01
  \item Primary KC: RATIO | Finding ratio | finding the ratio of two or three given whole numbers
  \item Secondary KC: WHOLE NUMBERS | Subtraction | subtracting whole numbers
  \item Topic: Recreation such as sports, games, exercises, music, movie, dancing, painting, fishing and other recreation activities
  \item Grade: Primary 6
\end{itemize}

\textbf{Solution}

Let the number of students who danced be $x$. 

We are told that the number of students who sang is 20 more than those who danced, so the number of students who sang is $x + 20$.

But from the problem, the actual numbers are:
Number of students who sang: $56$
Number of students who danced: $38$

Check: $56 - 38 = 18$, but the question says there are $20$ more. Since this does not tally, let's adjust the interpretation.

Actually, the problem can be:

"At the community sports hall, 45 children played badminton, and 33 children played table tennis. How many more children played badminton than table tennis? What is the ratio of the number of children who played badminton to those who played table tennis? Express your ratio in its simplest form."

Solution:
- Number of children who played badminton: $45$
- Number of children who played table tennis: $33$
- More children who played badminton: $45 - 33 = 12$

Ratio of badminton to table tennis:
$$\frac{45}{33} == \frac{15}{11}$$
So, the simplified ratio is $15:11$.

Final format:

{"word_problem": "At the community sports hall, 45 children played badminton and 33 children played table tennis. How many more children played badminton than table tennis? What is the ratio of the number of children who played badminton to those who played table tennis? Express your ratio in its simplest form.", "solution": "Number of children who played badminton: $45$\\Number of children who played table tennis: $33$\\More children who played badminton: $45 - 33 = 12$\\The ratio of children who played badminton to those who played table tennis is $45 : 33$.\\Simplify the ratio: $45 \div 3 = 15$, $33 \div 3 = 11$, so the simplest form is $15:11$."}

\end{document}
