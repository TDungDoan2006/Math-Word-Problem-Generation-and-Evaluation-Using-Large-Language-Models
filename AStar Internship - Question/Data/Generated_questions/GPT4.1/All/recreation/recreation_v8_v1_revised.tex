\documentclass{article}
\usepackage[utf8]{inputenc}
\usepackage{amsmath}
\usepackage{amsfonts}
\usepackage{amssymb}
\usepackage{graphicx}
\usepackage{hyperref}
\title{recreation\_v8\_v1}
\author{Tien Dung Doan}
\begin{document}
\maketitle
\section*{Question 1}
\textbf{Metadata}

\begin{itemize}
  \item Question ID: P3-WNAdd4d\_P1-WNCmp\_GPT4.1\_Recreation\_01
  \item Primary KC: WHOLE NUMBERS | Addition | adding whole numbers up to 4 digits
  \item Secondary KC: WHOLE NUMBERS | Comparison and ordering | comparing and ordering whole numbers
  \item Topic: Recreation such as sports, games, exercises, music, movie, dancing, painting, fishing and other recreation activities
  \item Grade: Primary 3
\end{itemize}

\textbf{Question}

During the school's Sports Day, there were three classes: Class A, Class B and Class C. 
Class A had $1,234$ points, Class B had $956$ points, and Class C had $1,087$ points by lunchtime.

(a) What was the total number of points all three classes had by lunchtime?

(b) Which class had the most points by lunchtime, and which class had the least?

\textbf{Solution}

(a) The total number of points for all three classes is:

$1,234 + 956 + 1,087 = 2,190 + 1,087 = 3,277$

So, the total number of points is $3,277$.

(b) Comparing the points:

\begin{align*}
&\text{Class A: } 1,234 \\
&\text{Class B: } 956 \\
&\text{Class C: } 1,087 
\end{align*}

Class A had the most points ($1,234$), and Class B had the least points ($956$).

\section*{Question 2}
\textbf{Metadata}

\begin{itemize}
  \item Question ID: P3-WNSub4d\_P1-WNCmp\_GPT4.1\_Recreation\_01
  \item Primary KC: WHOLE NUMBERS | Subtraction | subtracting whole numbers up to 4 digits
  \item Secondary KC: WHOLE NUMBERS | Comparison and ordering | comparing and ordering whole numbers
  \item Topic: Recreation such as sports, games, exercises, music, movie, dancing, painting, fishing and other recreation activities
  \item Grade: Primary 3
\end{itemize}

\textbf{Question}

During the Sports Day at Greenfield Primary School, there were $3250$ spectators in the stadium at the start of the relay race. When the race ended, $1875$ spectators had left. 

(a) How many spectators were still in the stadium after the relay race?

(b) If the organisers wanted at least $2000$ spectators to remain after the race to proceed with the next event, did they have enough spectators left? Give your answer by comparing the number found in part (a) with $2000$.

\textbf{Solution}

(a) The number of spectators still in the stadium after the relay race is obtained by subtraction:

$
3250 - 1875 = 1375
$

There were $1375$ spectators left after the relay race.

(b) We need to compare $1375$ with $2000$.

Since $1375 < 2000$, the organisers did NOT have enough spectators left to proceed with the next event.

\section*{Question 3}
\textbf{Metadata}

\begin{itemize}
  \item Question ID: P3-WNSub4d\_P1-WNAdd2nd\_GPT4.1\_Recreation\_01
  \item Primary KC: WHOLE NUMBERS | Subtraction | subtracting whole numbers up to 4 digits
  \item Secondary KC: WHOLE NUMBERS | Addition | adding whole numbers
  \item Topic: Recreation such as sports, games, exercises, music, movie, dancing, painting, fishing and other recreation activities
  \item Grade: Primary 3
\end{itemize}

\textbf{Question}

Darren had $1562$ stickers in his collection. After playing a sticker trading game with his friends, he gave away $738$ stickers. Later, he won $247$ stickers as a prize in another game. How many stickers does Darren have now?

\textbf{Solution}

First, find the number of stickers Darren has after giving away the stickers: 

$
1562 - 738 = 824
$

Next, add the stickers he won:

$
824 + 247 = 1071
$

Darren has $1071$ stickers now.

\section*{Question 4}
\textbf{Metadata}

\begin{itemize}
  \item Question ID: P3-WNDivRmd3d\_P1-WNCmp\_GPT4.1\_Recreation\_01
  \item Primary KC: WHOLE NUMBERS | Division | dividing whole numbers up to 3 digits by 1 digit with remainder 
  \item Secondary KC: WHOLE NUMBERS | Comparison and ordering | comparing and ordering whole numbers
  \item Topic: Recreation such as sports, games, exercises, music, movie, dancing, painting, fishing and other recreation activities
  \item Grade: Primary 3
\end{itemize}

\textbf{Question}

During a board games day, 127 children want to play chess. Each chess table can seat 4 players. 

(a) How many tables can be filled with 4 players at each table? How many children will not get a seat at a full table?

(b) Another game room has 95 children who want to play checkers. Each checkers table also seats 4 players. Compare the number of children left without a seat in both chess and checkers games. Which game has more children without a seat, and by how many?

\textbf{Solution}

(a) To find out how many tables can be filled:

$127 \div 4 = 31$ remainder $3$.

So, 31 tables will be fully occupied, and $3$ children will not get a seat at a full table.

(b) For the checkers room:

$95 \div 4 = 23$ remainder $3$.

So, 23 tables will be fully occupied, and $3$ children will not get a seat at a full table.

Comparison:
$3$ children in the chess game and $3$ children in the checkers game are left without a seat at a full table. Thus, the number is the same, and neither group has more children without a seat. The difference is $3 - 3 = 0$.

\textbf{Answer:} Both games have the same number of children ($3$) left without a seat at a full table.

\section*{Question 5}
\textbf{Metadata}

\begin{itemize}
  \item Question ID: P3-WNDivRmd3d\_P1-WNAdd2nd\_GPT4.1\_Recreation\_01
  \item Primary KC: WHOLE NUMBERS | Division | dividing whole numbers up to 3 digits by 1 digit with remainder 
  \item Secondary KC: WHOLE NUMBERS | Addition | adding whole numbers
  \item Topic: Recreation such as sports, games, exercises, music, movie, dancing, painting, fishing and other recreation activities
  \item Grade: Primary 3
\end{itemize}

\textbf{Question}

A group of 125 students is going to play a bowling game. Each bowling lane can hold 6 students. After filling up as many lanes as possible, the remaining students will play table tennis. Later, 9 more students join the group for table tennis. How many students are playing table tennis?

\textbf{Solution}

First, we divide the students among the bowling lanes:

Number of students: $125$
Number of students per lane: $6$

$125 \div 6 = 20$ lanes with a remainder of $5$ students.

So, $6 \times 20 = 120$ students are in the bowling lanes, and $125 - 120 = 5$ students do not have a lane and will play table tennis instead.

After that, $9$ more students join for table tennis:

Number of table tennis players = $5 + 9 = 14$

\textbf{Answer:} $14$ students are playing table tennis.

\section*{Question 6}
\textbf{Metadata}

\begin{itemize}
  \item Question ID: P3-WNDivRmd3d\_P1-WNSub2nd\_GPT4.1\_Recreation\_01
  \item Primary KC: WHOLE NUMBERS | Division | dividing whole numbers up to 3 digits by 1 digit with remainder 
  \item Secondary KC: WHOLE NUMBERS | Subtraction | subtracting whole numbers
  \item Topic: Recreation such as sports, games, exercises, music, movie, dancing, painting, fishing and other recreation activities
  \item Grade: Primary 3
\end{itemize}

\textbf{Question}

A music teacher has $154$ stickers to give as rewards to her students. She wants to divide the stickers equally among $6$ students. After giving out as many stickers as possible, how many stickers does she have left? Then, she decides to keep $2$ stickers for herself. How many stickers does she have now?

\textbf{Solution}

To find out how many stickers each student gets and how many are left:

$154 \div 6 = 25$ remainder $4$

Each student gets $25$ stickers, and the teacher has $4$ stickers left.

If the teacher keeps $2$ stickers for herself:

$4 - 2 = 2$

So, after keeping $2$ stickers, the teacher has $2$ stickers remaining.

\section*{Question 7}
\textbf{Metadata}

\begin{itemize}
  \item Question ID: P3-WNDivRmd3d\_P1-WNMul2nd\_GPT4.1\_Recreation\_01
  \item Primary KC: WHOLE NUMBERS | Division | dividing whole numbers up to 3 digits by 1 digit with remainder 
  \item Secondary KC: WHOLE NUMBERS | Multiplication | multiplying whole numbers
  \item Topic: Recreation such as sports, games, exercises, music, movie, dancing, painting, fishing and other recreation activities
  \item Grade: Primary 3
\end{itemize}

\textbf{Question}

A group of 124 children signed up for a painting workshop. The teacher wants to divide the children equally into teams for a painting competition, with each team consisting of 6 children. 

(a) How many teams can the teacher form, and how many children will be left without a team?

(b) If each team needs 4 paintbrushes, how many paintbrushes does the teacher need to prepare for all the teams?

\textbf{Solution}

Let us solve part (a) first:

To find how many teams can be formed, we divide $124$ by $6$.

$124\div6 = 20$ remainder $4$

This means $20$ teams can be formed, and $4$ children will be left without a team.

For part (b):

If each team needs $4$ paintbrushes, then the total number of paintbrushes needed is

$20 \times 4 = 80$

\textbf{Answers:}

(a) The teacher can form $20$ teams, and $4$ children will be left without a team.

(b) The teacher needs to prepare $80$ paintbrushes.

\section*{Question 8}
\textbf{Metadata}

\begin{itemize}
  \item Question ID: P3-WNMul3d1d\_P1-WNCmp\_GPT4.1\_Recreation\_01
  \item Primary KC: WHOLE NUMBERS | Multiplication | multiplying whole numbers up to 3 digits by 1 digit
  \item Secondary KC: WHOLE NUMBERS | Comparison and ordering | comparing and ordering whole numbers
  \item Topic: Recreation such as sports, games, exercises, music, movie, dancing, painting, fishing and other recreation activities
  \item Grade: Primary 3
\end{itemize}

\textbf{Question}

Sarah goes to a music class every week. Each week, she practices playing the piano for $45$ minutes each day for $4$ days. Her friend, David, practices the violin for $30$ minutes each day for $5$ days. 

Who spends more time practicing music in a week, and by how many minutes?

\textbf{Solution}

Sarah's total practice time in a week is $45 \times 4 = 180$ minutes.

David's total practice time in a week is $30 \times 5 = 150$ minutes.

Comparing $180$ minutes and $150$ minutes, Sarah spends more time practicing music.

The difference in practice time is $180 - 150 = 30$ minutes.

\textbf{Answer:} Sarah spends more time practicing music than David by $30$ minutes.

\section*{Question 9}
\textbf{Metadata}

\begin{itemize}
  \item Question ID: P3-WNMul3d1d\_P1-WNAdd2nd\_GPT4.1\_Recreation\_01
  \item Primary KC: WHOLE NUMBERS | Multiplication | multiplying whole numbers up to 3 digits by 1 digit
  \item Secondary KC: WHOLE NUMBERS | Addition | adding whole numbers
  \item Topic: Recreation such as sports, games, exercises, music, movie, dancing, painting, fishing and other recreation activities
  \item Grade: Primary 3
\end{itemize}

\textbf{Question}

At a music camp, there are 4 groups of students. Each group has 125 students. After the music session, 38 more students from another school joined the camp. How many students are there now in the camp?

\textbf{Solution}

First, find the total number of students in the 4 groups:

$
125 \times 4 = 500
$

Next, add the 38 new students who joined:

$
500 + 38 = 538
$

So, there are \textdollar538 students in the camp now.

\section*{Question 10}
\textbf{Metadata}

\begin{itemize}
  \item Question ID: P3-WNMul3d1d\_P1-WNSub2nd\_GPT4.1\_Recreation\_01
  \item Primary KC: WHOLE NUMBERS | Multiplication | multiplying whole numbers up to 3 digits by 1 digit
  \item Secondary KC: WHOLE NUMBERS | Subtraction | subtracting whole numbers
  \item Topic: Recreation such as sports, games, exercises, music, movie, dancing, painting, fishing and other recreation activities
  \item Grade: Primary 3
\end{itemize}

\textbf{Question}

A music school is organising a drumming class. There are 126 students signed up, and each student needs 3 drumsticks for the class. The school already has 220 drumsticks in its store. 

How many more drumsticks does the school need to buy to have enough for all the students?

\textbf{Solution}

First, calculate the total number of drumsticks needed:

Number of students $= 126$

Number of drumsticks per student $= 3$

Total drumsticks needed $= 126 \times 3 = 378$

Next, find out how many more drumsticks the school needs to buy:

Number of drumsticks the school already has $= 220$

Number of drumsticks to buy $= 378 - 220 = 158$

\textbf{Answer:} The school needs to buy $158$ more drumsticks.

\section*{Question 11}
\textbf{Metadata}

\begin{itemize}
  \item Question ID: P3-WNDiv3d1d\_P1-WNCmp\_GPT4.1\_Recreation\_01
  \item Primary KC: WHOLE NUMBERS | Division | dividing whole numbers up to 3 digits by 1 digit
  \item Secondary KC: WHOLE NUMBERS | Comparison and ordering | comparing and ordering whole numbers
  \item Topic: Recreation such as sports, games, exercises, music, movie, dancing, painting, fishing and other recreation activities
  \item Grade: Primary 3
\end{itemize}

\textbf{Question}

A group of 264 students participated in a school sports day. The students were divided equally into 6 teams for different games. Each team wore a different coloured shirt. 

(a) How many students were there in each team?

(b) Another school sent 275 students and divided them equally into 5 teams. Which school had more students in each team, and by how many?

\textbf{Solution}

(a) Number of students in each team: 
$ 264 \div 6 = 44 $
So, there were 44 students in each team.

(b) Number of students in each team from the other school:
$ 275 \div 5 = 55 $
So, each team in the other school had 55 students.

Comparing the number of students per team:
$ 55 - 44 = 11 $

The other school had more students in each team by 11 students.

\section*{Question 12}
\textbf{Metadata}

\begin{itemize}
  \item Question ID: P3-WNDiv3d1d\_P1-WNAdd2nd\_GPT4.1\_Recreation\_01
  \item Primary KC: WHOLE NUMBERS | Division | dividing whole numbers up to 3 digits by 1 digit
  \item Secondary KC: WHOLE NUMBERS | Addition | adding whole numbers
  \item Topic: Recreation such as sports, games, exercises, music, movie, dancing, painting, fishing and other recreation activities
  \item Grade: Primary 3
\end{itemize}

\textbf{Question}

A group of 168 children signed up for a painting workshop. The organiser wants to divide the children equally into 4 groups. After dividing all the children, 15 more children join the workshop. How many children are there in each group if the organiser still wants all groups to have the same number of children?

\textbf{Solution}

First, divide the original 168 children by 4: 

$168 \div 4 = 42$

So, each group had 42 children at first.

When 15 more children join, the total number of children becomes:

$168 + 15 = 183$

Now, divide 183 children into 4 equal groups:

$183 \div 4 = 45$ remainder $3$

Each group will have 45 children, and there will be 3 children left over. So, if the organiser wants all groups to have the same number of children, each group will have 45 children, and 3 children will not be assigned to any group.

\section*{Question 13}
\textbf{Metadata}

\begin{itemize}
  \item Question ID: P3-WNDiv3d1d\_P1-WNSub2nd\_GPT4.1\_Recreation\_01
  \item Primary KC: WHOLE NUMBERS | Division | dividing whole numbers up to 3 digits by 1 digit
  \item Secondary KC: WHOLE NUMBERS | Subtraction | subtracting whole numbers
  \item Topic: Recreation such as sports, games, exercises, music, movie, dancing, painting, fishing and other recreation activities
  \item Grade: Primary 3
\end{itemize}

\textbf{Question}

There are $246$ tennis balls that need to be packed equally into boxes for a school sports event. Each box can hold $3$ tennis balls. After packing all the tennis balls, $18$ tennis balls were taken out from some boxes to be used immediately for a game. How many tennis balls are left in the boxes after some are taken out?

\textbf{Solution}

First, divide $246$ tennis balls equally into boxes that hold $3$ balls each: 

$
246 \div 3 = 82
$
So, there are $82$ boxes, each with $3$ tennis balls packed in total. 

The total number of tennis balls in the boxes at first is $246$. Then, $18$ tennis balls were taken out for a game:

$
246 - 18 = 228
$

Therefore, $228$ tennis balls are left in the boxes after some are taken out.

\section*{Question 14}
\textbf{Metadata}

\begin{itemize}
  \item Question ID: P3-WNDiv3d1d\_P1-WNMul2nd\_GPT4.1\_Recreation\_01
  \item Primary KC: WHOLE NUMBERS | Division | dividing whole numbers up to 3 digits by 1 digit
  \item Secondary KC: WHOLE NUMBERS | Multiplication | multiplying whole numbers
  \item Topic: Recreation such as sports, games, exercises, music, movie, dancing, painting, fishing and other recreation activities
  \item Grade: Primary 3
\end{itemize}

\textbf{Question}

A music teacher has $168$ stickers to give to her $4$ students equally as a reward after a singing contest. Each student wants to stick their stickers onto sheets of paper, and each sheet can hold $6$ stickers. 

How many sheets does each student need to hold all of their stickers?

\textbf{Solution}

First, divide $168$ stickers among $4$ students:

$$168 \div 4 = 42$$

Each student receives $42$ stickers.

Next, find out how many sheets are needed for $42$ stickers if each sheet holds $6$ stickers:

$$42 \div 6 = 7$$

So each student needs $7$ sheets to hold all their stickers.

\section*{Question 15}
\textbf{Metadata}

\begin{itemize}
  \item Question ID: P3-FrAddRl12\_P2-FrCmp\_GPT4.1\_Recreation\_01
  \item Primary KC: FRACTIONS | Addition | adding two related fractions within one whole with denominators of given fractions not exceeding 12
  \item Secondary KC: FRACTIONS | Comparison and ordering | comparing and ordering fractions
  \item Topic: Recreation such as sports, games, exercises, music, movie, dancing, painting, fishing and other recreation activities
  \item Grade: Primary 3
\end{itemize}

\textbf{Question}

Amy participated in a music recital. She practised the piano for $\dfrac{3}{8}$ of an hour and then practised singing for $\dfrac{1}{8}$ of an hour.

(a) How much time did Amy spend practising piano and singing altogether?

(b) Her friend Ben practised music for $\dfrac{1}{2}$ of an hour. Who spent more time practising music, Amy or Ben?

\textbf{Solution}

(a) Amy spent $\dfrac{3}{8}$ hour on the piano and $\dfrac{1}{8}$ hour singing.

To find the total time she spent:
\$
\dfrac{3}{8} + \dfrac{1}{8} = \dfrac{3+1}{8} = \dfrac{4}{8} = \dfrac{1}{2} \text{ hour.}
\$

So, Amy spent $\dfrac{1}{2}$ hour practising piano and singing altogether.

(b) Ben practised for $\dfrac{1}{2}$ hour.

Comparing Amy’s and Ben’s total time:

Amy: $\dfrac{1}{2}$ hour  
Ben: $\dfrac{1}{2}$ hour

Both spent the same amount of time practising music.

\section*{Question 16}
\textbf{Metadata}

\begin{itemize}
  \item Question ID: P3-FrAddRl12\_P3-FrSmp\_GPT4.1\_Recreation\_01
  \item Primary KC: FRACTIONS | Addition | adding two related fractions within one whole with denominators of given fractions not exceeding 12
  \item Secondary KC: FRACTIONS | Simplifying | expressing a fraction in its simplest form
  \item Topic: Recreation such as sports, games, exercises, music, movie, dancing, painting, fishing and other recreation activities
  \item Grade: Primary 3
\end{itemize}

\textbf{Question}

During an art class, Emily spent $\frac{3}{12}$ of an hour painting and $\frac{5}{12}$ of an hour drawing. What fraction of an hour did Emily spend in total on painting and drawing? Express your answer in its simplest form.

\textbf{Solution}

Emily spent $\frac{3}{12}$ of an hour painting and $\frac{5}{12}$ of an hour drawing.

To find the total time spent, add the fractions:

$\frac{3}{12} + \frac{5}{12} = \frac{3+5}{12} = \frac{8}{12}$

Now, simplify $\frac{8}{12}$:

The greatest common divisor of 8 and 12 is 4.

$\frac{8 \div 4}{12 \div 4} = \frac{2}{3}$

So, Emily spent $\frac{2}{3}$ of an hour in total on painting and drawing.

\section*{Question 17}
\textbf{Metadata}

\begin{itemize}
  \item Question ID: P3-FrSubRl12\_P2-FrCmp\_GPT4.1\_Recreation\_01
  \item Primary KC: FRACTIONS | Subtraction | subtracting two related fractions within one whole with denominators of given fractions not exceeding 12
  \item Secondary KC: FRACTIONS | Comparison and ordering | comparing and ordering fractions
  \item Topic: Recreation such as sports, games, exercises, music, movie, dancing, painting, fishing and other recreation activities
  \item Grade: Primary 3
\end{itemize}

\textbf{Question}

Jenny joined a painting competition. She spent $\dfrac{5}{8}$ of an hour painting her picture on Saturday and $\dfrac{1}{8}$ of an hour painting on Sunday. On which day did Jenny spend more time painting, and by how much?

\textbf{Solution}

First, compare the fractions $\dfrac{5}{8}$ (Saturday) and $\dfrac{1}{8}$ (Sunday). 

$\dfrac{5}{8} > \dfrac{1}{8}$, so Jenny spent more time painting on Saturday.

Now, subtract the lesser fraction from the greater to find the difference:

$\dfrac{5}{8} - \dfrac{1}{8} = \dfrac{5 - 1}{8} = \dfrac{4}{8} = \dfrac{1}{2}$

Jenny spent $\dfrac{1}{2}$ hour more painting on Saturday than on Sunday.

\section*{Question 18}
\textbf{Metadata}

\begin{itemize}
  \item Question ID: P3-FrSubRl12\_P2-FrAdd2nd\_GPT4.1\_Recreation\_01
  \item Primary KC: FRACTIONS | Subtraction | subtracting two related fractions within one whole with denominators of given fractions not exceeding 12
  \item Secondary KC: FRACTIONS | Addition | adding fractions
  \item Topic: Recreation such as sports, games, exercises, music, movie, dancing, painting, fishing and other recreation activities
  \item Grade: Primary 3
\end{itemize}

\textbf{Question}

Sarah attended a painting workshop. She spent $\dfrac{3}{4}$ hour painting a landscape. Then she spent another $\dfrac{1}{6}$ hour painting a flower. How much total time did Sarah spend painting both pictures? If the workshop lasted $1$ hour, how much more time could Sarah have painted before the workshop ended?

\textbf{Solution}

First, find the total time Sarah spent painting by adding $\dfrac{3}{4}$ hour and $\dfrac{1}{6}$ hour.

To add $\dfrac{3}{4} + \dfrac{1}{6}$, make their denominators the same. The lowest common denominator of $4$ and $6$ is $12$:

$\dfrac{3}{4} = \dfrac{3 \times 3}{4 \times 3} = \dfrac{9}{12}$

$\dfrac{1}{6} = \dfrac{1 \times 2}{6 \times 2} = \dfrac{2}{12}$

So, $\dfrac{9}{12} + \dfrac{2}{12} = \dfrac{11}{12}$

Sarah spent a total of $\dfrac{11}{12}$ hour painting.

Now, find out how much more time she could have painted before the $1$ hour ended by subtracting:

$1 - \dfrac{11}{12}$

Express $1$ as $\dfrac{12}{12}$:

$\dfrac{12}{12} - \dfrac{11}{12} = \dfrac{1}{12}$

Sarah could have painted for $\dfrac{1}{12}$ more hour before the workshop ended.

\section*{Question 19}
\textbf{Metadata}

\begin{itemize}
  \item Question ID: P3-FrSubRl12\_P3-FrSmp\_GPT4.1\_Recreation\_01
  \item Primary KC: FRACTIONS | Subtraction | subtracting two related fractions within one whole with denominators of given fractions not exceeding 12
  \item Secondary KC: FRACTIONS | Simplifying | expressing a fraction in its simplest form
  \item Topic: Recreation such as sports, games, exercises, music, movie, dancing, painting, fishing and other recreation activities
  \item Grade: Primary 3
\end{itemize}

\textbf{Question}

Alicia played a game in which she needed to collect stickers. She collected $\frac{7}{12}$ of the stickers she needed to complete a set. After giving $\frac{1}{4}$ of her stickers to her friend, what fraction of the set does Alicia still have? Express your answer in its simplest form.

\textbf{Solution}

First, find the difference between the fractions of the set Alicia had and what she gave away:

$\frac{7}{12} - \frac{1}{4}$

Convert $\frac{1}{4}$ to a fraction with denominator 12:
$\frac{1}{4} = \frac{3}{12}$

Now subtract:
$\frac{7}{12} - \frac{3}{12} = \frac{4}{12}$

Simplify $\frac{4}{12}$ by dividing both numerator and denominator by 4:
$\frac{4 \div 4}{12 \div 4} = \frac{1}{3}$

Therefore, Alicia still has $\frac{1}{3}$ of the set.

\section*{Question 20}
\textbf{Metadata}

\begin{itemize}
  \item Question ID: P4-WNMul4d1d\_P1-WNCmp\_GPT4.1\_Recreation\_01
  \item Primary KC: WHOLE NUMBERS | Multiplication | multiplying whole numbers up to 4 digits by 1 digit or up to 3 digits by 2 digits
  \item Secondary KC: WHOLE NUMBERS | Comparison and ordering | comparing and ordering whole numbers
  \item Topic: Recreation such as sports, games, exercises, music, movie, dancing, painting, fishing and other recreation activities
  \item Grade: Primary 4
\end{itemize}

\textbf{Question}

During the school's Sports Day, there are 135 students in the Red Team and 248 students in the Blue Team. Each Red Team student collected 7 points, while each Blue Team student collected 6 points.\
\\
(a) How many points did the Red Team collect in total?\
\\
(b) How many points did the Blue Team collect in total?\
\\
(c) Which team collected more points, and by how many points?

\textbf{Solution}

(a) The total points the Red Team collected is $135 \times 7 = 945$ points.\\
(b) The total points the Blue Team collected is $248 \times 6 = 1,488$ points.\\
(c) Comparing the two totals, the Blue Team collected more points.\\
Difference in points: $1,488 - 945 = 543$ points.\\
\\
\textbf{Answer:}\\
(a) 945 points\\
(b) 1,488 points\\
(c) The Blue Team collected more points by 543 points.

\section*{Question 21}
\textbf{Metadata}

\begin{itemize}
  \item Question ID: P4-WNMul4d1d\_P1-WNAdd2nd\_GPT4.1\_Recreation\_01
  \item Primary KC: WHOLE NUMBERS | Multiplication | multiplying whole numbers up to 4 digits by 1 digit or up to 3 digits by 2 digits
  \item Secondary KC: WHOLE NUMBERS | Addition | adding whole numbers
  \item Topic: Recreation such as sports, games, exercises, music, movie, dancing, painting, fishing and other recreation activities
  \item Grade: Primary 4
\end{itemize}

\textbf{Question}

During a school sports day, each team received 235 small water bottles. There were 7 teams in total. After the event, the teachers decided to add 58 additional bottles to the total number of bottles used during the event to keep as spare for next year. How many water bottles will there be in all after adding the spare bottles?

\textbf{Solution}

First, find the total number of water bottles given to all teams:

$
235 \times 7 = 1,645
$

Next, add the 58 spare bottles:

$
1,645 + 58 = 1,703
$

So, there will be 1,703 water bottles in all after adding the spare bottles.

\section*{Question 22}
\textbf{Metadata}

\begin{itemize}
  \item Question ID: P4-WNMul4d1d\_P1-WNSub2nd\_GPT4.1\_Recreation\_01
  \item Primary KC: WHOLE NUMBERS | Multiplication | multiplying whole numbers up to 4 digits by 1 digit or up to 3 digits by 2 digits
  \item Secondary KC: WHOLE NUMBERS | Subtraction | subtracting whole numbers
  \item Topic: Recreation such as sports, games, exercises, music, movie, dancing, painting, fishing and other recreation activities
  \item Grade: Primary 4
\end{itemize}

\textbf{Question}

Sarah is helping to arrange chairs for a school music performance. There are $125$ rows of chairs, and each row has $8$ chairs. After setting up all the chairs, $97$ students have already taken their seats. How many empty chairs are there before the performance begins?

\textbf{Solution}

First, calculate the total number of chairs:

$
\text{Total chairs} = 125 \times 8 = 1000
$

Next, subtract the number of students already seated from the total number of chairs to find the number of empty chairs:

$
\text{Empty chairs} = 1000 - 97 = 903
$

\textbf{Final answer:}

There are $903$ empty chairs before the performance begins.

\section*{Question 23}
\textbf{Metadata}

\begin{itemize}
  \item Question ID: P4-WNMul4d1d\_P4-WNRnd5d\_GPT4.1\_Recreation\_01
  \item Primary KC: WHOLE NUMBERS | Multiplication | multiplying whole numbers up to 4 digits by 1 digit or up to 3 digits by 2 digits
  \item Secondary KC: WHOLE NUMBERS | Rounding | rounding whole numbers up to 100000 to the nearest 10, 100 or 1000 
  \item Topic: Recreation such as sports, games, exercises, music, movie, dancing, painting, fishing and other recreation activities
  \item Grade: Primary 4
\end{itemize}

\textbf{Question}

Elaine is organising a dance workshop. She expects 156 students to attend, and each student will practise 47 dance steps during the workshop. 

(a) Estimate the total number of dance steps all students will practise by first rounding the number of students to the nearest ten and the number of steps to the nearest ten, then multiplying them together.

(b) Calculate the actual total number of dance steps all students will practise.

\textbf{Solution}

(a) Rounding 156 to the nearest ten gives $160$. Rounding $47$ to the nearest ten gives $50$.

Estimated total number of dance steps $= 160 \times 50 = 8,000$

(b) The actual total number of dance steps is $156 \times 47$.

Calculate:
$156 \times 47 = (156 \times 40) + (156 \times 7) = 6,240 + 1,092 = 7,332$

So, the actual total number of dance steps is $7,332$.

\section*{Question 24}
\textbf{Metadata}

\begin{itemize}
  \item Question ID: P4-WNDiv4d1d\_P1-WNCmp\_GPT4.1\_Recreation\_01
  \item Primary KC: WHOLE NUMBERS | Division | dividing whole numbers up to 4 digits by 1 digit
  \item Secondary KC: WHOLE NUMBERS | Comparison and ordering | comparing and ordering whole numbers
  \item Topic: Recreation such as sports, games, exercises, music, movie, dancing, painting, fishing and other recreation activities
  \item Grade: Primary 4
\end{itemize}

\textbf{Question}

A group of $1,356$ children attended a music camp. The organisers want to divide all the children equally into $4$ teams for the final performance.\
\
(a) How many children will be in each team?\
\
(b) After dividing them into teams, which is greater: the number of children in each team or the number of leftover children? Explain your answer.

\textbf{Solution}

(a) To find the number of children in each team, divide $1,356$ by $4$.\\
$1,356 \div 4 = 339$ \\ Children in each team: $339$\\
Leftover children: $1,356 - (4 \times 339) = 1,356 - 1,356 = 0$\\
So, there are no leftover children.\\
\\
(b) Comparing the number of children in each team ($339$) and the number of leftover children ($0$), we see that $339 > 0$.\
\\
Therefore, the number of children in each team is greater than the number of leftover children.

\section*{Question 25}
\textbf{Metadata}

\begin{itemize}
  \item Question ID: P4-WNDiv4d1d\_P1-WNAdd2nd\_GPT4.1\_Recreation\_01
  \item Primary KC: WHOLE NUMBERS | Division | dividing whole numbers up to 4 digits by 1 digit
  \item Secondary KC: WHOLE NUMBERS | Addition | adding whole numbers
  \item Topic: Recreation such as sports, games, exercises, music, movie, dancing, painting, fishing and other recreation activities
  \item Grade: Primary 4
\end{itemize}

\textbf{Question}

A group of 1,368 children signed up for a large badminton tournament. Each badminton court can be used by 6 children at a time. The organisers want to divide all the children equally among the courts so that all courts are used by the same number of children. 

How many courts are needed? After all the children have played, another 290 children join the tournament. What is the new total number of children? How many courts will be needed now to accommodate all the children if the same rule is used?

\textbf{Solution}

First, divide the original number of children by the number of children per court:

$1,368 \div 6 = 228$

So, the organisers need $228$ courts.

Next, add the 290 new children to the original number:

$1,368 + 290 = 1,658$

Divide the new total by 6 to see how many courts are needed:

$1,658 \div 6 = 276$ remainder $2$

This means $276$ courts can each have 6 children, and there are 2 children left, so one more court is needed for them. Thus, the organisers will need $277$ courts in total.

Final answers:

- Courts needed at first: $228$
- New total number of children: $1,658$
- Courts needed after more children join: $277$

\section*{Question 26}
\textbf{Metadata}

\begin{itemize}
  \item Question ID: P4-WNDiv4d1d\_P1-WNSub2nd\_GPT4.1\_Recreation\_01
  \item Primary KC: WHOLE NUMBERS | Division | dividing whole numbers up to 4 digits by 1 digit
  \item Secondary KC: WHOLE NUMBERS | Subtraction | subtracting whole numbers
  \item Topic: Recreation such as sports, games, exercises, music, movie, dancing, painting, fishing and other recreation activities
  \item Grade: Primary 4
\end{itemize}

\textbf{Question}

A group of 2,448 children attended a music concert. The organisers wanted to seat the children in rows with 6 children in each row.

(a) How many rows of children did the organisers need to arrange?

Later, 253 children left the concert early.

(b) How many children were still at the concert after some left?

\textbf{Solution}

(a) To find the number of rows needed, divide the total number of children by the number of children per row:

$
\text{Number of rows} = \frac{2,448}{6} = 408
$

So, the organisers needed 408 rows.

(b) To find how many children were still at the concert:

$
\text{Children remaining} = 2,448 - 253 = 2,195
$

So, 2,195 children were still at the concert after some left.

\section*{Question 27}
\textbf{Metadata}

\begin{itemize}
  \item Question ID: P4-WNDiv4d1d\_P1-WNMul2nd\_GPT4.1\_Recreation\_01
  \item Primary KC: WHOLE NUMBERS | Division | dividing whole numbers up to 4 digits by 1 digit
  \item Secondary KC: WHOLE NUMBERS | Multiplication | multiplying whole numbers
  \item Topic: Recreation such as sports, games, exercises, music, movie, dancing, painting, fishing and other recreation activities
  \item Grade: Primary 4
\end{itemize}

\textbf{Question}

At a music school, there are $864$ students who want to join music classes. Each class must have $6$ students.\
(a) How many classes can the music school form?\
(b) If every class is given $12$ music stands, how many music stands are needed in total for all the classes?

\textbf{Solution}

(a) To find out how many classes can be formed, divide $864$ students by $6$ students per class:\\

$
864 \div 6 = 144
$

So, $144$ classes can be formed.\\

(b) For the total number of music stands needed, multiply the number of classes by $12$ stands per class:\\

$
144 \times 12 = 1,728
$

Thus, $1,728$ music stands are needed for all the classes.

\section*{Question 28}
\textbf{Metadata}

\begin{itemize}
  \item Question ID: P4-WNDiv4d1d\_P4-WNRnd5d\_GPT4.1\_Recreation\_01
  \item Primary KC: WHOLE NUMBERS | Division | dividing whole numbers up to 4 digits by 1 digit
  \item Secondary KC: WHOLE NUMBERS | Rounding | rounding whole numbers up to 100000 to the nearest 10, 100 or 1000 
  \item Topic: Recreation such as sports, games, exercises, music, movie, dancing, painting, fishing and other recreation activities
  \item Grade: Primary 4
\end{itemize}

\textbf{Question}

A group of friends collected $4362$ points by playing dance games over the weekend. They want to split the points equally among $7$ players. After dividing the points, to the nearest $10$, how many points will each player get?

\textbf{Solution}

First, divide $4362$ by $7$:

$
4362 \div 7 = 623.142857... 
$

Each player gets $623$ points, with some left over.

Next, round $623$ to the nearest $10$:

The last digit is $3$ (less than $5$), so we round down to $620$.

\textbf{Answer:}
To the nearest $10$, each player will get $\textdollar620$ points.

\section*{Question 29}
\textbf{Metadata}

\begin{itemize}
  \item Question ID: P4-FrRepSet\_P3-FrCnvEq\_GPT4.1\_Recreation\_01
  \item Primary KC: FRACTIONS | Representation and concept | expressing a part of a set as a fraction
  \item Secondary KC: FRACTIONS | Conversion to equivalent fractions | Conversion to equivalent fractions (given either the denominator or the numerator)
  \item Topic: Recreation such as sports, games, exercises, music, movie, dancing, painting, fishing and other recreation activities
  \item Grade: Primary 4
\end{itemize}

\textbf{Question}

During a school sports day, there are $24$ students in the relay race team. $9$ of them are girls. 

(a) What fraction of the relay race team are girls? 

(b) Express the fraction of girls as an equivalent fraction with denominator $8$.

\textbf{Solution}

(a) The fraction of the relay race team that are girls is $\dfrac{9}{24}$, since there are $9$ girls out of $24$ students.

(b) To write $\dfrac{9}{24}$ as an equivalent fraction with denominator $8$, we need to find a number $x$ such that $\dfrac{9}{24} = \dfrac{x}{8}$. 

Set up the equation:

$\dfrac{9}{24} = \dfrac{x}{8}$

Cross-multiply:

$9 \times 8 = 24 \times x$

$72 = 24x$

$x = \dfrac{72}{24} = 3$

So, an equivalent fraction is $\dfrac{3}{8}$.

Therefore, $\dfrac{9}{24} = \dfrac{3}{8}$.

\section*{Question 30}
\textbf{Metadata}

\begin{itemize}
  \item Question ID: P4-FrAddU12\_P2-FrCmp\_GPT4.1\_Recreation\_01
  \item Primary KC: FRACTIONS | Addition | adding unlike fractions with two different denominators not exceeding 12
  \item Secondary KC: FRACTIONS | Comparison and ordering | comparing and ordering fractions
  \item Topic: Recreation such as sports, games, exercises, music, movie, dancing, painting, fishing and other recreation activities
  \item Grade: Primary 4
\end{itemize}

\textbf{Question}

Sarah spent $\frac{2}{5}$ of an hour playing the piano and $\frac{1}{3}$ of an hour painting on Saturday afternoon. 

(a) How much time did Sarah spend on both activities altogether?

(b) Did she spend more time painting or playing the piano? Arrange the two fractions in order from least to greatest.

\textbf{Solution}

(a) To find the total time Sarah spent, add $\frac{2}{5}$ and $\frac{1}{3}$. 

Find a common denominator:

The LCM of $5$ and $3$ is $15$.

Rewrite $\frac{2}{5}$: $\frac{2}{5} = \frac{2 \times 3}{5 \times 3} = \frac{6}{15}$

Rewrite $\frac{1}{3}$: $\frac{1}{3} = \frac{1 \times 5}{3 \times 5} = \frac{5}{15}$

Add them: $\frac{6}{15} + \frac{5}{15} = \frac{11}{15}$

Sarah spent $\frac{11}{15}$ of an hour in total.

(b) To compare $\frac{2}{5}$ and $\frac{1}{3}$, use the common denominator $15$:

$\frac{2}{5} = \frac{6}{15}$ and $\frac{1}{3} = \frac{5}{15}$

Since $6/15 > 5/15$, Sarah spent more time playing the piano.

Order from least to greatest: $\frac{1}{3}$, $\frac{2}{5}$.

\section*{Question 31}
\textbf{Metadata}

\begin{itemize}
  \item Question ID: P4-FrAddU12\_P3-FrSmp\_GPT4.1\_Recreation\_01
  \item Primary KC: FRACTIONS | Addition | adding unlike fractions with two different denominators not exceeding 12
  \item Secondary KC: FRACTIONS | Simplifying | expressing a fraction in its simplest form
  \item Topic: Recreation such as sports, games, exercises, music, movie, dancing, painting, fishing and other recreation activities
  \item Grade: Primary 4
\end{itemize}

\textbf{Question}

Amir spent $\frac{3}{8}$ of an hour playing the guitar and $\frac{2}{3}$ of an hour painting on Sunday afternoon. 

What fraction of an hour did Amir spend on both activities altogether? Give your answer in its simplest form.

\textbf{Solution}

First, we find the sum of the two fractions:

$\frac{3}{8} + \frac{2}{3}$

To add these, we need a common denominator. The denominators are $8$ and $3$, and their least common multiple is $24$.

Express $\frac{3}{8}$ with denominator $24$:

$\frac{3}{8} = \frac{3 \times 3}{8 \times 3} = \frac{9}{24}$

Express $\frac{2}{3}$ with denominator $24$:

$\frac{2}{3} = \frac{2 \times 8}{3 \times 8} = \frac{16}{24}$

Add the fractions:

$\frac{9}{24} + \frac{16}{24} = \frac{25}{24}$

$\frac{25}{24}$ is an improper fraction. It cannot be simplified further, but it can be written as:

$\frac{25}{24} = 1\frac{1}{24}$

Amir spent $\frac{25}{24}$ hours, or $1\frac{1}{24}$ hours, on both activities altogether.

\section*{Question 32}
\textbf{Metadata}

\begin{itemize}
  \item Question ID: P4-FrSubU12\_P2-FrCmp\_GPT4.1\_Recreation\_01
  \item Primary KC: FRACTIONS | Subtraction | subtracting unlike fractions with two different denominators not exceeding 12
  \item Secondary KC: FRACTIONS | Comparison and ordering | comparing and ordering fractions
  \item Topic: Recreation such as sports, games, exercises, music, movie, dancing, painting, fishing and other recreation activities
  \item Grade: Primary 4
\end{itemize}

\textbf{Question}

Sarah spent $\dfrac{3}{4}$ hour practicing piano and $\dfrac{5}{12}$ hour painting on Saturday. 

(a) How much longer did Sarah spend practicing piano compared to painting?

(b) Did Sarah spend more time painting or practicing piano? Arrange the two activities in order from shortest to longest amount of time spent.

\textbf{Solution}

(a) To find out how much longer Sarah spent practicing piano, subtract $\dfrac{5}{12}$ from $\dfrac{3}{4}$.

First, find a common denominator. The denominators are $4$ and $12$. The lowest common denominator is $12$.

$\dfrac{3}{4} = \dfrac{3 \times 3}{4 \times 3} = \dfrac{9}{12}$

Now subtract:
$\dfrac{9}{12} - \dfrac{5}{12} = \dfrac{9 - 5}{12} = \dfrac{4}{12} = \dfrac{1}{3}$

Sarah spent $\dfrac{1}{3}$ hour more practicing piano than painting.

(b) Comparing $\dfrac{3}{4}$ hour (practicing piano) and $\dfrac{5}{12}$ hour (painting):

Convert $\dfrac{3}{4}$ to twelfths:
$\dfrac{3}{4} = \dfrac{9}{12}$

Since $\dfrac{9}{12} > \dfrac{5}{12}$, Sarah spent more time practicing piano.

Order from shortest to longest:

Painting ($\dfrac{5}{12}$ hour), then Practicing Piano ($\dfrac{3}{4}$ hour).

\section*{Question 33}
\textbf{Metadata}

\begin{itemize}
  \item Question ID: P4-FrSubU12\_P2-FrAdd2nd\_GPT4.1\_Recreation\_01
  \item Primary KC: FRACTIONS | Subtraction | subtracting unlike fractions with two different denominators not exceeding 12
  \item Secondary KC: FRACTIONS | Addition | adding fractions
  \item Topic: Recreation such as sports, games, exercises, music, movie, dancing, painting, fishing and other recreation activities
  \item Grade: Primary 4
\end{itemize}

\textbf{Question}

Elaine spent $\frac{5}{6}$ hour painting and $\frac{1}{4}$ hour playing the piano on Saturday afternoon. She spent a total of 1 hour doing these activities. 

(a) How much time in hours did Elaine spend painting and playing the piano together? 

(b) How much longer did she spend painting than playing the piano?

\textbf{Solution}

(a) Elaine spent $\frac{5}{6}$ hour painting and $\frac{1}{4}$ hour playing the piano.

To find the total time spent, add the two fractions:

$\frac{5}{6} + \frac{1}{4}$

Find the lowest common denominator (LCD) for $6$ and $4$, which is $12$.

$\frac{5}{6} = \frac{5 \times 2}{6 \times 2} = \frac{10}{12}$

$\frac{1}{4} = \frac{1 \times 3}{4 \times 3} = \frac{3}{12}$

Add the two fractions:

$\frac{10}{12} + \frac{3}{12} = \frac{13}{12}$

Elaine spent a total of $\frac{13}{12}$ hours on painting and playing the piano.


(b) To find how much longer she spent painting than playing the piano:

$\frac{5}{6} - \frac{1}{4}$

Using the LCD $12$:

$\frac{5}{6} = \frac{10}{12}$

$\frac{1}{4} = \frac{3}{12}$

$\frac{10}{12} - \frac{3}{12} = \frac{7}{12}$

Elaine spent $\frac{7}{12}$ hour longer painting than playing the piano.

\section*{Question 34}
\textbf{Metadata}

\begin{itemize}
  \item Question ID: P4-FrSubU12\_P3-FrSmp\_GPT4.1\_Recreation\_01
  \item Primary KC: FRACTIONS | Subtraction | subtracting unlike fractions with two different denominators not exceeding 12
  \item Secondary KC: FRACTIONS | Simplifying | expressing a fraction in its simplest form
  \item Topic: Recreation such as sports, games, exercises, music, movie, dancing, painting, fishing and other recreation activities
  \item Grade: Primary 4
\end{itemize}

\textbf{Question}

Jasmine spent $\frac{5}{6}$ of an hour painting and $\frac{1}{4}$ of an hour playing the piano on Saturday afternoon. How much longer did she spend painting than playing the piano? Express your answer in its simplest form.

\textbf{Solution}

To find how much longer Jasmine spent painting than playing the piano, subtract the time spent playing the piano from the time spent painting:

$\frac{5}{6} - \frac{1}{4}$

First, find a common denominator for 6 and 4. The least common denominator is 12.

Convert $\frac{5}{6}$ to twelfths:
$\frac{5}{6} = \frac{5 \times 2}{6 \times 2} = \frac{10}{12}$

Convert $\frac{1}{4}$ to twelfths:
$\frac{1}{4} = \frac{1 \times 3}{4 \times 3} = \frac{3}{12}$

Subtract:
$\frac{10}{12} - \frac{3}{12} = \frac{7}{12}$

$\frac{7}{12}$ is already in its simplest form.

\textbf{Answer:} Jasmine spent $\frac{7}{12}$ of an hour longer painting than playing the piano.

\section*{Question 35}
\textbf{Metadata}

\begin{itemize}
  \item Question ID: P4-DcAdd2d\_P4-DcCmp3d\_GPT4.1\_Recreation\_01
  \item Primary KC: DECIMALS | Addition | adding decimals (up to 2 decimal places)
  \item Secondary KC: DECIMALS | Comparison and ordering | comparing and ordering decimals up to 3 decimal places
  \item Topic: Recreation such as sports, games, exercises, music, movie, dancing, painting, fishing and other recreation activities
  \item Grade: Primary 4
\end{itemize}

\textbf{Question}

During a singing competition, Mei scored $8.75$ points in the first round, $7.90$ points in the second round, and $8.125$ points in the final round. Hui scored $7.85$ points in the first round, $8.20$ points in the second round, and $8.085$ points in the final round. 

(a) Calculate the total score Mei received for all three rounds.

(b) Who scored higher in the final round? Arrange Mei's and Hui's final round scores in order from smallest to largest.

\textbf{Solution}

(a) Mei's total score for all three rounds is: 

$8.75 + 7.90 + 8.125 = 16.65 + 8.125 = 24.775$

Mei scored $24.775$ points in total.

(b) Comparing final round scores:

Mei's final round score: $8.125$

Hui's final round score: $8.085$

Since $8.125 > 8.085$, Mei scored higher in the final round.

Arranging the final round scores from smallest to largest:

Hui: $8.085$, Mei: $8.125$.

\section*{Question 36}
\textbf{Metadata}

\begin{itemize}
  \item Question ID: P4-DcAdd2d\_P4-DcCnv2Fr\_GPT4.1\_Recreation\_01
  \item Primary KC: DECIMALS | Addition | adding decimals (up to 2 decimal places)
  \item Secondary KC: DECIMALS | Conversion from decimals to fraction | expressing decimals as fractions
  \item Topic: Recreation such as sports, games, exercises, music, movie, dancing, painting, fishing and other recreation activities
  \item Grade: Primary 4
\end{itemize}

\textbf{Question}

Rachel took part in a charity bowling event. In her first game, she scored $78.35$ points, and in her second game, she scored $56.45$ points. 

(a) What was Rachel's total score for both games?

(b) Express her total score as a fraction in its simplest form.

\textbf{Solution}

(a) To find the total score, add the scores from both games:

$
78.35 + 56.45 = 134.80
$

Rachel's total score for both games is $134.80$.

(b) To express $134.80$ as a fraction:

$134.80 = 134 + 0.80$

$0.80$ as a fraction is $\frac{80}{100} = \frac{4}{5}$.

Therefore, $134.80 = 134\frac{4}{5}$, or as an improper fraction:

$134\frac{4}{5} = \frac{134 \times 5 + 4}{5} = \frac{670 + 4}{5} = \frac{674}{5}$.

So, Rachel's total score as a fraction in simplest form is $\frac{674}{5}$.

\section*{Question 37}
\textbf{Metadata}

\begin{itemize}
  \item Question ID: P4-DcAdd2d\_P4-DcRnd3d\_GPT4.1\_Recreation\_01
  \item Primary KC: DECIMALS | Addition | adding decimals (up to 2 decimal places)
  \item Secondary KC: DECIMALS | Rounding | rounding decimals up to 3 decimal places to the nearest whole number, 1 decimal place and 2 decimal places 
  \item Topic: Recreation such as sports, games, exercises, music, movie, dancing, painting, fishing and other recreation activities
  \item Grade: Primary 4
\end{itemize}

\textbf{Question}

Jia Wei went swimming and recorded the distances he swam over three days. On Monday, he swam $1.28$ km; on Wednesday, he swam $2.53$ km; and on Friday, he swam $1.797$ km. 

(a) What was the total distance Jia Wei swam over the three days?

(b) Round the total distance to the nearest whole number, to 1 decimal place, and to 2 decimal places.

\textbf{Solution}

Step 1: Add up the distances:

$1.28 + 2.53 + 1.797 = 1.280 + 2.530 + 1.797 = 5.607$ km

Step 2: Round $5.607$ to the nearest whole number:
- The digit after the decimal is $6$ (which is 5 or more), so round up:
- Nearest whole number: $6$ km

Step 3: Round $5.607$ to 1 decimal place:
- The second digit after the decimal is $0$ (which is less than 5), so round down:
- $5.607$ rounded to 1 decimal place is $5.6$ km

Step 4: Round $5.607$ to 2 decimal places:
- The third digit after the decimal is $7$ (which is 5 or more), so round up:
- $5.607$ rounded to 2 decimal places is $5.61$ km

Final Answers:

(a) Jia Wei swam a total of $5.607$ km.

(b)
- Nearest whole number: $6$ km
- 1 decimal place: $5.6$ km
- 2 decimal places: $5.61$ km

\section*{Question 38}
\textbf{Metadata}

\begin{itemize}
  \item Question ID: P4-DcSub2d\_P4-DcCmp3d\_GPT4.1\_Recreation\_01
  \item Primary KC: DECIMALS | Subtraction | subtracting decimals (up to 2 decimal places)
  \item Secondary KC: DECIMALS | Comparison and ordering | comparing and ordering decimals up to 3 decimal places
  \item Topic: Recreation such as sports, games, exercises, music, movie, dancing, painting, fishing and other recreation activities
  \item Grade: Primary 4
\end{itemize}

\textbf{Question}

Lina played a music game where she earned points in decimals for each round. In the first round, she scored $7.85$ points. In the second round, she scored $6.43$ points. In the third round, she scored $7.802$ points.

(a) How many more points did Lina score in the first round than in the second round?

(b) Arrange Lina's scores from the lowest to the highest.

\textbf{Solution}

(a) To find out how many more points Lina scored in the first round than in the second round, we subtract the second round score from the first round score:

$7.85 - 6.43 = 1.42$

So, Lina scored $1.42$ more points in the first round than in the second round.

(b) To arrange Lina's scores from lowest to highest, we compare $7.85$, $6.43$, and $7.802$.

First, compare $6.43$, $7.85$, and $7.802$. Since $6.43 < 7.802 < 7.85$,

the order from lowest to highest is:

$6.43$, $7.802$, $7.85$.

\section*{Question 39}
\textbf{Metadata}

\begin{itemize}
  \item Question ID: P4-DcSub2d\_P4-DcCnv2Fr\_GPT4.1\_Recreation\_01
  \item Primary KC: DECIMALS | Subtraction | subtracting decimals (up to 2 decimal places)
  \item Secondary KC: DECIMALS | Conversion from decimals to fraction | expressing decimals as fractions
  \item Topic: Recreation such as sports, games, exercises, music, movie, dancing, painting, fishing and other recreation activities
  \item Grade: Primary 4
\end{itemize}

\textbf{Question}

Mandy is training for a swimming competition. On Monday, she swam $2.75$ km. On Tuesday, she swam $1.30$ km. 

(a) How many more kilometres did Mandy swim on Monday than on Tuesday? 

(b) Express your answer for part (a) as a fraction in its simplest form.

\textbf{Solution}

(a) To find how many more kilometres Mandy swam on Monday, subtract the distance on Tuesday from the distance on Monday:

$$2.75 - 1.30 = 1.45$$

Mandy swam $1.45$ km more on Monday than on Tuesday.

(b) Express $1.45$ as a fraction:

$1.45 = 1 + 0.45 = 1 + \frac{45}{100}$

Simplify $\frac{45}{100}$:

$\frac{45}{100} = \frac{9}{20}$

So, $1.45 = 1\frac{9}{20}$

Final answer:

Mandy swam $1.45$ km more on Monday than on Tuesday, or $1\frac{9}{20}$ km.

\section*{Question 40}
\textbf{Metadata}

\begin{itemize}
  \item Question ID: P4-DcSub2d\_P4-DcRnd3d\_GPT4.1\_Recreation\_01
  \item Primary KC: DECIMALS | Subtraction | subtracting decimals (up to 2 decimal places)
  \item Secondary KC: DECIMALS | Rounding | rounding decimals up to 3 decimal places to the nearest whole number, 1 decimal place and 2 decimal places 
  \item Topic: Recreation such as sports, games, exercises, music, movie, dancing, painting, fishing and other recreation activities
  \item Grade: Primary 4
\end{itemize}

\textbf{Question}

Sarah completed three rounds of swimming. In the first round, she swam $2.45$ km, and in the second round, she swam $1.38$ km. In the third round, she swam $0.755$ km. 

(a) What is the total distance Sarah swam in the first two rounds? 

(b) How much further did she swim in the first two rounds compared to the third round? Give your answer correct to $2$ decimal places. 

(c) Round the distance Sarah swam in the third round to the nearest whole number.

\textbf{Solution}

(a) Total distance in the first two rounds:

$2.45 + 1.38 = 3.83$ km

(b) Difference between the first two rounds and the third round:

$3.83 - 0.755 = 3.075$ km

Rounded to $2$ decimal places: $3.08$ km

(c) Distance in the third round to the nearest whole number:

$0.755$ km rounds up to $1$ km

\section*{Question 41}
\textbf{Metadata}

\begin{itemize}
  \item Question ID: P4-DcSub2d\_P4-DcAdd2nd\_GPT4.1\_Recreation\_01
  \item Primary KC: DECIMALS | Subtraction | subtracting decimals (up to 2 decimal places)
  \item Secondary KC: DECIMALS | Addition | adding decimals
  \item Topic: Recreation such as sports, games, exercises, music, movie, dancing, painting, fishing and other recreation activities
  \item Grade: Primary 4
\end{itemize}

\textbf{Question}

Sarah went bowling with her friends. She scored $8.75$ points in the first game and $9.60$ points in the second game. After both games, she realised she had $17.00$ points in total. How many more points did Sarah need to reach $21.50$ points in total after playing both games?

\textbf{Solution}

First, let's find the total points Sarah scored in two games:

$
8.75 + 9.60 = 18.35
$

Sarah actually has $17.00$ points, so she must have lost some points. Let's find the difference between the total she could have gotten and the points she actually has:

$
18.35 - 17.00 = 1.35
$

So, she lost $1.35$ points and actually has $17.00$ points.

Now, let's find how many more points Sarah needs to reach $21.50$ points:

$
21.50 - 17.00 = 4.50
$

Sarah needs \textdollar4.50$ more points to reach \textdollar21.50$ points.

\section*{Question 42}
\textbf{Metadata}

\begin{itemize}
  \item Question ID: P4-DcMul2d1d\_P4-DcCmp3d\_GPT4.1\_Recreation\_01
  \item Primary KC: DECIMALS | Multiplication | multiplying decimals (up to 2 decimal places) by a 1-digit whole number
  \item Secondary KC: DECIMALS | Comparison and ordering | comparing and ordering decimals up to 3 decimal places
  \item Topic: Recreation such as sports, games, exercises, music, movie, dancing, painting, fishing and other recreation activities
  \item Grade: Primary 4
\end{itemize}

\textbf{Question}

Amanda is helping her school organise a table tennis tournament. She recorded the distances, in metres, that three players hit the ball during practice. Player A hit the ball $2.35$ metres, Player B hit the ball $2.171$ metres, and Player C hit the ball $2.47$ metres. Each player hit the ball $4$ times, and all their hits were the same distance as their first try. 

(a) Find the total distance each player hit the ball by multiplying their distance by $4$.

(b) Arrange the total distances hit by the three players in order, from the shortest to the longest distance.

\textbf{Solution}

Let us multiply the distance each player hit the ball by $4$:

For Player A:
$2.35 \times 4 = 9.40$

For Player B:
$2.171 \times 4 = 8.684$

For Player C:
$2.47 \times 4 = 9.88$

So, the total distances are:
Player A: $9.40$ m
Player B: $8.684$ m
Player C: $9.88$ m

Next, we arrange these totals from the shortest to the longest:
$8.684$ m (Player B), $9.40$ m (Player A), $9.88$ m (Player C)

**Final answer:**
Shortest to longest total distance: Player B, Player A, Player C.

\section*{Question 43}
\textbf{Metadata}

\begin{itemize}
  \item Question ID: P4-DcMul2d1d\_P4-DcCnv2Fr\_GPT4.1\_Recreation\_01
  \item Primary KC: DECIMALS | Multiplication | multiplying decimals (up to 2 decimal places) by a 1-digit whole number
  \item Secondary KC: DECIMALS | Conversion from decimals to fraction | expressing decimals as fractions
  \item Topic: Recreation such as sports, games, exercises, music, movie, dancing, painting, fishing and other recreation activities
  \item Grade: Primary 4
\end{itemize}

\textbf{Question}

Jason enjoys rollerblading at the park. Each time he makes one round around the park, he skates $1.25$ km. If he completes $4$ rounds, how many kilometres does he skate in total? Express your answer as a decimal first. Then, express your answer as a fraction in its simplest form.

\textbf{Solution}

First, let's find the total distance Jason skates by multiplying the distance for one round by the number of rounds.$ 1.25 \times 4 = 5 $So, Jason skates a total of $5$ km.\newline\newline Now, to express $5$ km as a fraction, notice that $5 = \frac{5}{1}$.\newline\newline Therefore, Jason skates $5$ km in total, which can be expressed as $\frac{5}{1}$ km.

\section*{Question 44}
\textbf{Metadata}

\begin{itemize}
  \item Question ID: P4-DcMul2d1d\_P4-DcRnd3d\_GPT4.1\_Recreation\_01
  \item Primary KC: DECIMALS | Multiplication | multiplying decimals (up to 2 decimal places) by a 1-digit whole number
  \item Secondary KC: DECIMALS | Rounding | rounding decimals up to 3 decimal places to the nearest whole number, 1 decimal place and 2 decimal places 
  \item Topic: Recreation such as sports, games, exercises, music, movie, dancing, painting, fishing and other recreation activities
  \item Grade: Primary 4
\end{itemize}

\textbf{Question}

Emma played badminton for $1.75$ hours each day for $5$ days in a week. 

(a) What is the total number of hours Emma played badminton in $5$ days?

(b) Round your answer in part (a) to the nearest whole number and to $1$ decimal place.

\textbf{Solution}

(a) Total number of hours Emma played badminton in $5$ days:

$ 1.75 \times 5 = 8.75 \text{ hours} $

(b) Rounding $8.75$:

- To the nearest whole number: $9$
- To $1$ decimal place: $8.8$

\textbf{Answers:}

(a) $8.75$ hours.

(b) $9$ hours (nearest whole number); $8.8$ hours (nearest $1$ decimal place).

\section*{Question 45}
\textbf{Metadata}

\begin{itemize}
  \item Question ID: P4-DcMul2d1d\_P4-DcAdd2nd\_GPT4.1\_Recreation\_01
  \item Primary KC: DECIMALS | Multiplication | multiplying decimals (up to 2 decimal places) by a 1-digit whole number
  \item Secondary KC: DECIMALS | Addition | adding decimals
  \item Topic: Recreation such as sports, games, exercises, music, movie, dancing, painting, fishing and other recreation activities
  \item Grade: Primary 4
\end{itemize}

\textbf{Question}

Amy is organising a mini-golf game for her friends. Each player needs 3 golf balls and each golf ball costs $\textdollar1.25$. She is expecting 4 friends to join the game, including herself. \newline
After buying the golf balls, Amy also buys a scorecard for each friend at $\textdollar0.80$ each. \newline
How much does Amy spend in total on the golf balls and the scorecards?

\textbf{Solution}

First, find the total number of players: \newline
$4$ friends + Amy herself $= 5$ players.\newline
Each player needs $3$ golf balls, so total golf balls needed: \newline
$5 \times 3 = 15$ golf balls.\newline
Each golf ball costs $\textdollar1.25$, so the total cost for golf balls is: \newline
$15 \times 1.25 = 18.75$\newline
Total cost for golf balls = $\textdollar18.75$.\newline
Each player also needs a scorecard at $\textdollar0.80$ each:\newline
$5 \times 0.80 = 4.00$\newline
Total cost for scorecards = $\textdollar4.00$.\newline
Add both costs together to find the total spent:\newline
$18.75 + 4.00 = 22.75$\newline
Amy spends $\textdollar22.75$ in total on the golf balls and scorecards.

\section*{Question 46}
\textbf{Metadata}

\begin{itemize}
  \item Question ID: P4-DcMul2d1d\_P4-DcSub2nd\_GPT4.1\_Recreation\_01
  \item Primary KC: DECIMALS | Multiplication | multiplying decimals (up to 2 decimal places) by a 1-digit whole number
  \item Secondary KC: DECIMALS | Subtraction | subtracting decimals
  \item Topic: Recreation such as sports, games, exercises, music, movie, dancing, painting, fishing and other recreation activities
  \item Grade: Primary 4
\end{itemize}

\textbf{Question}

Lily joined a dance workshop. Each session of the workshop lasts for $1.25$ hours. She attends $4$ sessions in a week.

(a) How many hours does she spend at the workshop in one week?

(b) If she planned to spend $6.5$ hours on dancing for the week, how many more hours would she need to dance to meet her goal?

\textbf{Solution}

(a) Number of hours per session: $1.25$

Number of sessions: $4$

Total hours spent $= 1.25 \times 4 = 5.00$

So, Lily spends $5.00$ hours at the workshop in one week.

(b) Hours planned: $6.5$

Hours already spent at workshop: $5.00$

Hours needed $= 6.5 - 5.00 = 1.5$

Therefore, Lily needs to dance $1.5$ more hours to meet her goal.

\section*{Question 47}
\textbf{Metadata}

\begin{itemize}
  \item Question ID: P4-DcDiv2d1d\_P4-DcCmp3d\_GPT4.1\_Recreation\_01
  \item Primary KC: DECIMALS | Division | dividing decimals (up to 2 decimal places) by a 1-digit whole number
  \item Secondary KC: DECIMALS | Comparison and ordering | comparing and ordering decimals up to 3 decimal places
  \item Topic: Recreation such as sports, games, exercises, music, movie, dancing, painting, fishing and other recreation activities
  \item Grade: Primary 4
\end{itemize}

\textbf{Question}

During the school's Music Day, Sarah recorded the total duration she spent on three different music activities: singing, playing the piano, and drumming. 

She spent $7.68$ minutes singing, $3.276$ minutes playing the piano, and $5.13$ minutes drumming. She wants to practise each activity equally over 3 days, so she needs to divide the time for each activity by $3$ to find how many minutes she should spend on each activity per day.

(a) Calculate how many minutes Sarah should spend on each activity per day. 

(b) Rank the daily times for the three activities from the shortest to the longest.

\textbf{Solution}

(a) For each activity, divide the total minutes by $3$.

- Singing: $7.68 \div 3 = 2.56$ minutes per day
- Piano: $3.276 \div 3 = 1.092$ minutes per day
- Drumming: $5.13 \div 3 = 1.71$ minutes per day

(b) To order the daily times:
- Piano: $1.092$ (shortest)
- Drumming: $1.71$
- Singing: $2.56$ (longest)

So, the order from shortest to longest is: Piano, Drumming, Singing.

\section*{Question 48}
\textbf{Metadata}

\begin{itemize}
  \item Question ID: P4-DcDiv2d1d\_P4-DcCnv2Fr\_GPT4.1\_Recreation\_01
  \item Primary KC: DECIMALS | Division | dividing decimals (up to 2 decimal places) by a 1-digit whole number
  \item Secondary KC: DECIMALS | Conversion from decimals to fraction | expressing decimals as fractions
  \item Topic: Recreation such as sports, games, exercises, music, movie, dancing, painting, fishing and other recreation activities
  \item Grade: Primary 4
\end{itemize}

\textbf{Question}

Siti went to a music store and spent \textdollar9.84 to buy 6 identical packs of guitar strings. 

(a) How much did Siti pay for each pack of guitar strings?

(b) Express the price of each pack as a fraction in its simplest form.

\textbf{Solution}

(a) The total cost for 6 packs is \textdollar9.84. To find the price of each pack:

$
\text{Price per pack} = \frac{9.84}{6} = 1.64
$

So, Siti paid \textdollar1.64 for each pack of guitar strings.

(b) To express 1.64 as a fraction:

$
1.64 = \frac{164}{100} = \frac{41}{25}
$

So, the price of each pack as a fraction in its simplest form is \(\frac{41}{25}\).

\section*{Question 49}
\textbf{Metadata}

\begin{itemize}
  \item Question ID: P4-DcDiv2d1d\_P4-DcRnd3d\_GPT4.1\_Recreation\_01
  \item Primary KC: DECIMALS | Division | dividing decimals (up to 2 decimal places) by a 1-digit whole number
  \item Secondary KC: DECIMALS | Rounding | rounding decimals up to 3 decimal places to the nearest whole number, 1 decimal place and 2 decimal places 
  \item Topic: Recreation such as sports, games, exercises, music, movie, dancing, painting, fishing and other recreation activities
  \item Grade: Primary 4
\end{itemize}

\textbf{Question}

During a music competition, Hannah recorded the total time it took her to play 5 songs, which was $24.75$ minutes. She wants to know how long she spent, on average, on each song.

(a) How many minutes did she spend on each song if the time is divided equally among all 5 songs?

(b) Round your answer from (a) to the nearest whole number, to 1 decimal place, and to 2 decimal places.

\textbf{Solution}

(a) To find the average time spent on each song, divide the total time by the number of songs:

$
\text{Average time per song} = \frac{24.75}{5} = 4.95 \text{ minutes}
$

(b) Rounding $4.95$ minutes:

- To the nearest whole number: $5$ minutes (since $0.95 \geq 0.5$)
- To 1 decimal place: $5.0$ minutes (since the second decimal, $5$, rounds $4.9$ up to $5.0$)
- To 2 decimal places: $4.95$ minutes (already 2 decimal places)

$
\text{Answers:}
$

- Nearest whole number: $5$
- 1 decimal place: $5.0$
- 2 decimal places: $4.95$

\section*{Question 50}
\textbf{Metadata}

\begin{itemize}
  \item Question ID: P4-DcDiv2d1d\_P4-DcAdd2nd\_GPT4.1\_Recreation\_01
  \item Primary KC: DECIMALS | Division | dividing decimals (up to 2 decimal places) by a 1-digit whole number
  \item Secondary KC: DECIMALS | Addition | adding decimals
  \item Topic: Recreation such as sports, games, exercises, music, movie, dancing, painting, fishing and other recreation activities
  \item Grade: Primary 4
\end{itemize}

\textbf{Question}

Lisa bought 3 ribbons for her dance performance. The first ribbon is $1.25 \text{m}$ long, the second is $2.40 \text{m}$ long, and the third is $1.65 \text{m}$ long. She wants to divide all her ribbon equally among 4 friends, including herself. How long will each person get?

\textbf{Solution}

First, find the total length of ribbon Lisa has: 

$$1.25 + 2.40 + 1.65 = 5.30\text{ m}$$

Next, divide the total length by 4 to find out how much ribbon each person gets:

$$5.30 \div 4 = 1.325\text{ m}$$

So, each person will get $1.325 \text{ m}$ of ribbon.

\section*{Question 51}
\textbf{Metadata}

\begin{itemize}
  \item Question ID: P4-DcDiv2d1d\_P4-DcSub2nd\_GPT4.1\_Recreation\_01
  \item Primary KC: DECIMALS | Division | dividing decimals (up to 2 decimal places) by a 1-digit whole number
  \item Secondary KC: DECIMALS | Subtraction | subtracting decimals
  \item Topic: Recreation such as sports, games, exercises, music, movie, dancing, painting, fishing and other recreation activities
  \item Grade: Primary 4
\end{itemize}

\textbf{Question}

Shawn joined a music club that charges a total fee of $36.84 for 6 weeks of lessons. After paying for 4 weeks, he calculated how much more he needed to pay by subtracting the amount he had already paid from the total fee. If he pays the same amount each week, how much more does Shawn have to pay for the remaining 2 weeks?

\textbf{Solution}

First, let's find out how much Shawn pays each week by dividing the total fee by the number of weeks:

$
\text{Fee per week} = \frac{36.84}{6} = 6.14
$

After 4 weeks, Shawn has already paid:

$
\text{Amount paid for 4 weeks} = 4 \times 6.14 = 24.56
$

To find out how much more Shawn needs to pay, subtract what he has already paid from the total fee:

$
\text{Amount left to pay} = 36.84 - 24.56 = 12.28
$

So, Shawn has to pay \textdollar12.28 for the remaining 2 weeks.

\section*{Question 52}
\textbf{Metadata}

\begin{itemize}
  \item Question ID: P5-FrAddMix\_P2-FrCmp\_GPT4.1\_Recreation\_01
  \item Primary KC: FRACTIONS | Addition | adding mixed numbers
  \item Secondary KC: FRACTIONS | Comparison and ordering | comparing and ordering fractions
  \item Topic: Recreation such as sports, games, exercises, music, movie, dancing, painting, fishing and other recreation activities
  \item Grade: Primary 5
\end{itemize}

\textbf{Question}

Jing Wen and Arjun are practising the piano for their upcoming recital. On Saturday, Jing Wen practised for $1\frac{1}{2}$ hours in the morning and $2\frac{1}{4}$ hours in the afternoon. Arjun practised for $2\frac{2}{5}$ hours in total on the same day.\
\
(a) How many hours did Jing Wen practise the piano in total on Saturday?\
\
(b) Who practised longer, and by how much?

\textbf{Solution}

(a) Jing Wen's total practice time is 

$1\frac{1}{2} + 2\frac{1}{4} = \frac{3}{2} + \frac{9}{4}$\
First, express both fractions with a common denominator (4):\
$\frac{3}{2} = \frac{6}{4}$\
$\frac{6}{4} + \frac{9}{4} = \frac{15}{4}$\
$\frac{15}{4} = 3\frac{3}{4}$\
So, Jing Wen practised for $3\frac{3}{4}$ hours in total.\
\
(b) Arjun practised for $2\frac{2}{5}$ hours. To compare, convert both to improper fractions:\
$3\frac{3}{4} = \frac{15}{4}$\
$2\frac{2}{5} = \frac{12}{5}$\
Find a common denominator (20):\
$\frac{15}{4} = \frac{75}{20}$\
$\frac{12}{5} = \frac{48}{20}$\
\
$75 > 48$, so Jing Wen practised longer. The difference is $75 - 48 = 27$ parts out of 20, or $\frac{27}{20} = 1\frac{7}{20}$ hours.\
\
\textbf{Answer:}\
(a) Jing Wen practised for $3\frac{3}{4}$ hours.\
(b) Jing Wen practised longer by $1\frac{7}{20}$ hours.

\section*{Question 53}
\textbf{Metadata}

\begin{itemize}
  \item Question ID: P5-FrAddMix\_P3-FrSmp\_GPT4.1\_Recreation\_01
  \item Primary KC: FRACTIONS | Addition | adding mixed numbers
  \item Secondary KC: FRACTIONS | Simplifying | expressing a fraction in its simplest form
  \item Topic: Recreation such as sports, games, exercises, music, movie, dancing, painting, fishing and other recreation activities
  \item Grade: Primary 5
\end{itemize}

\textbf{Question}

Hui Ling played basketball for $1\frac{1}{4}$ hours and then played badminton for $2\frac{2}{3}$ hours on Saturday.

(a) How many hours did Hui Ling spend in total playing both sports? Express your answer as a mixed number.

(b) Express your answer in (a) in its simplest form.

\textbf{Solution}

(a) First, add the whole numbers: $1 + 2 = 3$.

Next, add the fractions: $\frac{1}{4} + \frac{2}{3}$.

Find a common denominator. The least common multiple of $4$ and $3$ is $12$.

$\frac{1}{4} = \frac{3}{12}$, $\frac{2}{3} = \frac{8}{12}$

So, $\frac{3}{12} + \frac{8}{12} = \frac{11}{12}$

Add this to the sum of the whole numbers:

$3 + \frac{11}{12} = 3\frac{11}{12}$

(b) $3\frac{11}{12}$ is already in its simplest form because $11$ and $12$ have no common factor except $1$.

\textbf{Final Answer:}

Hui Ling spent a total of $3\frac{11}{12}$ hours playing both sports.

\section*{Question 54}
\textbf{Metadata}

\begin{itemize}
  \item Question ID: P5-FrAddMix\_P5-FrCnv2Dc\_GPT4.1\_Recreation\_01
  \item Primary KC: FRACTIONS | Addition | adding mixed numbers
  \item Secondary KC: FRACTIONS | Conversion to decimals | expressing fractions as decimals
  \item Topic: Recreation such as sports, games, exercises, music, movie, dancing, painting, fishing and other recreation activities
  \item Grade: Primary 5
\end{itemize}

\textbf{Question}

Kelly practises piano every day. On Monday, she practised for $1\dfrac{3}{4}$ hours. On Tuesday, she practised for $2\dfrac{2}{5}$ hours. 

(a) How many hours did Kelly practise piano in total on Monday and Tuesday?

(b) Express your answer in (a) as a decimal.

\textbf{Solution}

(a) First, add the mixed numbers:

$1\dfrac{3}{4} + 2\dfrac{2}{5}$

Convert the mixed numbers to improper fractions:

$1\dfrac{3}{4} = \frac{7}{4}$

$2\dfrac{2}{5} = \frac{12}{5}$

Find a common denominator for $4$ and $5$, which is $20$:

$\frac{7}{4} = \frac{35}{20}$

$\frac{12}{5} = \frac{48}{20}$

Add the fractions:

$\frac{35}{20}+\frac{48}{20}=\frac{83}{20}$

Convert $\frac{83}{20}$ back to a mixed number:

$83\div20=4$ remainder $3$, so $\frac{83}{20} = 4\dfrac{3}{20}$

So, Kelly practised piano for $4\dfrac{3}{20}$ hours in total on Monday and Tuesday.

(b) Express $4\dfrac{3}{20}$ as a decimal:

$\dfrac{3}{20} = 0.15$

So, $4 + 0.15 = 4.15$

Kelly practised piano for $4.15$ hours in total on Monday and Tuesday.

\section*{Question 55}
\textbf{Metadata}

\begin{itemize}
  \item Question ID: P5-FrSubMix\_P2-FrCmp\_GPT4.1\_Recreation\_01
  \item Primary KC: FRACTIONS | Subtraction | subtracting mixed numbers
  \item Secondary KC: FRACTIONS | Comparison and ordering | comparing and ordering fractions
  \item Topic: Recreation such as sports, games, exercises, music, movie, dancing, painting, fishing and other recreation activities
  \item Grade: Primary 5
\end{itemize}

\textbf{Question}

Jayden and Sarah are competing in a game to see who can complete a 6 km bicycle ride the fastest. Jayden finished riding $4 \frac{2}{3}$ km before taking a break, and after his break, he completed the rest. 

Sarah, on the other hand, stopped for a break after riding $3 \frac{5}{6}$ km, and then finished the remaining distance in one go. 

(a) How many more kilometres did Jayden have left to finish after his break than Sarah did after her own break? 

(b) Who had less distance left to cycle after their break?

\textbf{Solution}

(a) Total distance = $6$ km.

Jayden cycled $4 \frac{2}{3}$ km before his break.

Distance Jayden had left:
$6 - 4 \frac{2}{3} = 6 - \frac{14}{3} = \frac{18}{3} - \frac{14}{3} = \frac{4}{3}$ km

Sarah cycled $3 \frac{5}{6}$ km before her break.

Distance Sarah had left:
$6 - 3 \frac{5}{6} = 6 - \frac{23}{6} = \frac{36}{6} - \frac{23}{6} = \frac{13}{6}$ km

Compare the remaining distances:
Jayden: $\frac{4}{3} = \frac{8}{6}$ km
Sarah: $\frac{13}{6}$ km

How many more kilometres did Jayden have left after his break than Sarah?
$\frac{8}{6} - \frac{13}{6} = -\frac{5}{6}$ km

This means Jayden actually had $\frac{5}{6}$ km less to cycle after his break than Sarah.

(b) Jayden had $\frac{8}{6} = 1 \frac{1}{3}$ km left, and Sarah had $2 \frac{1}{6}$ km left. 

Since $\frac{8}{6} < \frac{13}{6}$, Jayden had less distance left to cycle after their break.

\textbf{Final answers:}

(a) Jayden had $\frac{5}{6}$ km less left to cycle after his break than Sarah did after hers.

(b) Jayden had less distance left to cycle after his break.

\section*{Question 56}
\textbf{Metadata}

\begin{itemize}
  \item Question ID: P5-FrSubMix\_P2-FrAdd2nd\_GPT4.1\_Recreation\_01
  \item Primary KC: FRACTIONS | Subtraction | subtracting mixed numbers
  \item Secondary KC: FRACTIONS | Addition | adding fractions
  \item Topic: Recreation such as sports, games, exercises, music, movie, dancing, painting, fishing and other recreation activities
  \item Grade: Primary 5
\end{itemize}

\textbf{Question}

Sarah and her friends went for a morning cycling trip. She cycled $4\frac{2}{3}$ km before taking a break. After the break, she cycled another $1\frac{5}{6}$ km. 

(a) What was the total distance Sarah cycled?

(b) If her goal was to cycle $8\frac{1}{2}$ km that day, how much further did she need to cycle after the break?

\textbf{Solution}

(a) Total distance Sarah cycled:

$4\frac{2}{3} + 1\frac{5}{6}$

First, convert mixed numbers to improper fractions:
$4\frac{2}{3} = \frac{14}{3}$
$1\frac{5}{6} = \frac{11}{6}$

Find a common denominator:
$
\frac{14}{3} = \frac{28}{6}$

Add the fractions:
$\frac{28}{6} + \frac{11}{6} = \frac{39}{6} = 6\frac{1}{2}$

Sarah cycled a total of $6\frac{1}{2}$ km.

(b) Distance left to reach her goal:

$8\frac{1}{2} - 6\frac{1}{2}$

Convert to improper fractions:
$8\frac{1}{2} = \frac{17}{2}$
$6\frac{1}{2} = \frac{13}{2}$

Subtract:
$\frac{17}{2} - \frac{13}{2} = \frac{4}{2} = 2$

Sarah needed to cycle 2 km more to reach her goal.

\section*{Question 57}
\textbf{Metadata}

\begin{itemize}
  \item Question ID: P5-FrSubMix\_P3-FrSmp\_GPT4.1\_Recreation\_01
  \item Primary KC: FRACTIONS | Subtraction | subtracting mixed numbers
  \item Secondary KC: FRACTIONS | Simplifying | expressing a fraction in its simplest form
  \item Topic: Recreation such as sports, games, exercises, music, movie, dancing, painting, fishing and other recreation activities
  \item Grade: Primary 5
\end{itemize}

\textbf{Question}

Samuel joined an art class that lasted $2\dfrac{3}{4}$ hours on Saturday. He spent $1\dfrac{2}{5}$ hours painting and used the rest of the time to sketch. 

(a) How many hours did Samuel spend sketching?

(b) Express the number of hours spent sketching in its simplest fractional form.

\textbf{Solution}

(a) The time Samuel spent sketching is:

$2\dfrac{3}{4} - 1\dfrac{2}{5}$

First, convert the mixed numbers to improper fractions:

$2\dfrac{3}{4} = \dfrac{8 + 3}{4} = \dfrac{11}{4}$

$1\dfrac{2}{5} = \dfrac{5 + 2}{5} = \dfrac{7}{5}$

Now subtract:

$\dfrac{11}{4} - \dfrac{7}{5}$

Find the common denominator:

LCM of $4$ and $5$ is $20$.

$\dfrac{11}{4} = \dfrac{11 \times 5}{4 \times 5} = \dfrac{55}{20}$

$\dfrac{7}{5} = \dfrac{7 \times 4}{5 \times 4} = \dfrac{28}{20}$

So,

$\dfrac{55}{20} - \dfrac{28}{20} = \dfrac{27}{20}$

(b) $\dfrac{27}{20}$ is an improper fraction. To express it as a mixed number in simplest form:

$\dfrac{27}{20} = 1\dfrac{7}{20}$

The fraction $\dfrac{7}{20}$ is already in its simplest form.

\textbf{Answer:}

(a) Samuel spent $\dfrac{27}{20}$ hours sketching.

(b) In simplest form, Samuel spent $1\dfrac{7}{20}$ hours sketching.

\section*{Question 58}
\textbf{Metadata}

\begin{itemize}
  \item Question ID: P5-FrSubMix\_P5-FrCnv2Dc\_GPT4.1\_Recreation\_01
  \item Primary KC: FRACTIONS | Subtraction | subtracting mixed numbers
  \item Secondary KC: FRACTIONS | Conversion to decimals | expressing fractions as decimals
  \item Topic: Recreation such as sports, games, exercises, music, movie, dancing, painting, fishing and other recreation activities
  \item Grade: Primary 5
\end{itemize}

\textbf{Question}

Rachel is tracking her time spent practising the piano. On Saturday, she practised for $2\dfrac{3}{4}$ hours, and on Sunday, she practised for $1\dfrac{2}{5}$ hours. After both sessions, she wants to find out how much longer she practised on Saturday compared to Sunday, and write her answer in decimal form (hours, to 2 decimal places). How many more hours did Rachel practise on Saturday than on Sunday? Express your answer as a decimal.

\textbf{Solution}

First, convert both mixed numbers to improper fractions:

For Saturday: $2\dfrac{3}{4} = 2 + \dfrac{3}{4} = \dfrac{8}{4} + \dfrac{3}{4} = \dfrac{11}{4}$

For Sunday: $1\dfrac{2}{5} = 1 + \dfrac{2}{5} = \dfrac{5}{5} + \dfrac{2}{5} = \dfrac{7}{5}$

Now, subtract the two:

Find a common denominator for $\dfrac{11}{4}$ and $\dfrac{7}{5}$, which is $20$.

$\dfrac{11}{4} = \dfrac{11 \times 5}{4 \times 5} = \dfrac{55}{20}$
$\dfrac{7}{5} = \dfrac{7 \times 4}{5 \times 4} = \dfrac{28}{20}$

Subtract: $\dfrac{55}{20} - \dfrac{28}{20} = \dfrac{27}{20}$

Write $\dfrac{27}{20}$ as a decimal:

$\dfrac{27}{20} = 1.35$

So, Rachel practised $1.35$ hours ($1$ hour $21$ minutes) more on Saturday than on Sunday.

\textbf{Answer:} Rachel practised $1.35$ hours more on Saturday than on Sunday.

\section*{Question 59}
\textbf{Metadata}

\begin{itemize}
  \item Question ID: P5-FrMulImN\_P2-FrCmp\_GPT4.1\_Recreation\_01
  \item Primary KC: FRACTIONS | Multiplication | multiplying a proper/improper fraction and a whole number
  \item Secondary KC: FRACTIONS | Comparison and ordering | comparing and ordering fractions
  \item Topic: Recreation such as sports, games, exercises, music, movie, dancing, painting, fishing and other recreation activities
  \item Grade: Primary 5
\end{itemize}

\textbf{Question}

Sarah is practicing badminton. Each practice session lasts $\frac{3}{4}$ of an hour. She plans to attend 5 sessions this week. 

(a) How many hours will Sarah spend practicing badminton in total this week?

(b) After her 5 sessions, Sarah compares her practice time to her friend Megan, who practices for $3$ hours a week. Who spent more time practicing badminton, and by how much?

\textbf{Solution}

(a) The total time Sarah spends practicing is:

$5 \times \frac{3}{4} = \frac{15}{4}$ hours.

$\frac{15}{4}$ hours can be converted to a mixed number:

$15 \div 4 = 3$ remainder $3$, so $\frac{15}{4} = 3\frac{3}{4}$ hours.

Sarah spends $3\frac{3}{4}$ hours practicing badminton in total.

(b) Compare $3\frac{3}{4}$ hours to Megan's $3$ hours.

$3\frac{3}{4}$ hours $= \frac{15}{4}$, $3$ hours $= \frac{12}{4}$.

$\frac{15}{4} - \frac{12}{4} = \frac{3}{4}$

Sarah spends $\frac{3}{4}$ hour more than Megan.

Therefore, Sarah spent more time practicing badminton by $\frac{3}{4}$ hour.

\section*{Question 60}
\textbf{Metadata}

\begin{itemize}
  \item Question ID: P5-FrMulImN\_P2-FrAdd2nd\_GPT4.1\_Recreation\_01
  \item Primary KC: FRACTIONS | Multiplication | multiplying a proper/improper fraction and a whole number
  \item Secondary KC: FRACTIONS | Addition | adding fractions
  \item Topic: Recreation such as sports, games, exercises, music, movie, dancing, painting, fishing and other recreation activities
  \item Grade: Primary 5
\end{itemize}

\textbf{Question}

Amanda attends painting classes every week. Each time she paints, she uses $\frac{3}{4}$ of a tube of paint. If Amanda attended 5 painting sessions this month, how much paint did she use in total? After the fifth session, Amanda finds that she has used another $\frac{2}{4}$ of a tube of paint for touch-ups at home. What is the total amount of paint Amanda used this month?

\textbf{Solution}

First, find the amount of paint Amanda used for the 5 painting sessions:

Amount used per session: $\frac{3}{4}$ of a tube

Total for 5 sessions: $5 \times \frac{3}{4} = \frac{15}{4}$ tubes

Then, add the amount used for touch-ups:

Touch-up amount: $\frac{2}{4}$ tubes

Total paint used: $\frac{15}{4} + \frac{2}{4} = \frac{17}{4}$ tubes

$\frac{17}{4}$ can be written as a mixed number:

$\frac{17}{4} = 4 \frac{1}{4}$

So, Amanda used a total of $4 \frac{1}{4}$ tubes of paint this month.

\section*{Question 61}
\textbf{Metadata}

\begin{itemize}
  \item Question ID: P5-FrMulImN\_P2-FrSub2nd\_GPT4.1\_Recreation\_01
  \item Primary KC: FRACTIONS | Multiplication | multiplying a proper/improper fraction and a whole number
  \item Secondary KC: FRACTIONS | Subtraction | subtracting fractions
  \item Topic: Recreation such as sports, games, exercises, music, movie, dancing, painting, fishing and other recreation activities
  \item Grade: Primary 5
\end{itemize}

\textbf{Question}

A group of students is practicing for a dance performance. Each dance routine takes $\frac{3}{4}$ hour to rehearse.

On Monday, the students practise 5 dance routines. On Tuesday, they have enough time to rehearse only $2\frac{1}{2}$ hours in total. 

(a) How many hours do the students spend rehearsing on Monday?

(b) How much longer do they spend rehearsing on Monday than on Tuesday?

\textbf{Solution}

(a) Each dance routine takes $\frac{3}{4}$ hour. On Monday, students practise 5 dance routines.

$5 \times \frac{3}{4} = \frac{15}{4} = 3\frac{3}{4}$ hours$

On Monday, they spend $3\frac{3}{4}$ hours rehearsing.

(b) On Tuesday, they rehearse $2\frac{1}{2}$ hours.

$3\frac{3}{4} - 2\frac{1}{2}= \frac{15}{4} - \frac{5}{2}$

Convert $\frac{5}{2}$ to quarters: $\frac{5}{2} = \frac{10}{4}$

So,
$\frac{15}{4} - \frac{10}{4} = \frac{5}{4} = 1\frac{1}{4}$

They spend $1\frac{1}{4}$ hours more rehearsing on Monday than on Tuesday.

\section*{Question 62}
\textbf{Metadata}

\begin{itemize}
  \item Question ID: P5-FrMulImN\_P3-FrSmp\_GPT4.1\_Recreation\_01
  \item Primary KC: FRACTIONS | Multiplication | multiplying a proper/improper fraction and a whole number
  \item Secondary KC: FRACTIONS | Simplifying | expressing a fraction in its simplest form
  \item Topic: Recreation such as sports, games, exercises, music, movie, dancing, painting, fishing and other recreation activities
  \item Grade: Primary 5
\end{itemize}

\textbf{Question}

Sarah is practising the piano for an upcoming concert. Each day, she spends $\frac{7}{4}$ hours practising. If she practises for 5 days, how many hours does she spend practising in total? Express your answer as a mixed number in its simplest form.

\textbf{Solution}

Sarah practises $\frac{7}{4}$ hours each day for 5 days.

Total hours $= 5 \times \frac{7}{4} = \frac{35}{4}$ hours.

To express $\frac{35}{4}$ as a mixed number:

$\frac{35}{4} = 8\frac{3}{4}$

So, Sarah spends a total of $8\frac{3}{4}$ hours practising the piano in 5 days.

\section*{Question 63}
\textbf{Metadata}

\begin{itemize}
  \item Question ID: P5-FrMulImN\_P5-FrCnv2Dc\_GPT4.1\_Recreation\_01
  \item Primary KC: FRACTIONS | Multiplication | multiplying a proper/improper fraction and a whole number
  \item Secondary KC: FRACTIONS | Conversion to decimals | expressing fractions as decimals
  \item Topic: Recreation such as sports, games, exercises, music, movie, dancing, painting, fishing and other recreation activities
  \item Grade: Primary 5
\end{itemize}

\textbf{Question}

Sarah spends $\frac{3}{5}$ of an hour practising the piano each day. If she practises for 4 days, how many hours in total does she spend practising the piano? Express your answer as a decimal.

\textbf{Solution}

Sarah practises $\frac{3}{5}$ of an hour each day for 4 days.

Total hours spent $= 4 \times \frac{3}{5} = \frac{12}{5}$ hours.

Now, express $\frac{12}{5}$ as a decimal:

$\frac{12}{5} = 2.4$

Sarah spends 2.4 hours in total practising the piano over 4 days.

\section*{Question 64}
\textbf{Metadata}

\begin{itemize}
  \item Question ID: P5-FrMulPIm\_P2-FrCmp\_GPT4.1\_Recreation\_01
  \item Primary KC: FRACTIONS | Multiplication | multiplying a proper fraction and a proper/improper fractions
  \item Secondary KC: FRACTIONS | Comparison and ordering | comparing and ordering fractions
  \item Topic: Recreation such as sports, games, exercises, music, movie, dancing, painting, fishing and other recreation activities
  \item Grade: Primary 5
\end{itemize}

\textbf{Question}

Lina is learning to paint as a hobby. Last week, she completed $\frac{3}{5}$ of a painting in one session. This week, she spent $\frac{2}{3}$ as much time painting as she did last week. 

(a) What fraction of a painting did Lina complete this week?

(b) Compare the amount of painting Lina completed this week to last week. Which week did she complete more of her painting, and by how much?

\textbf{Solution}

(a) The fraction of a painting completed this week is:

$\frac{3}{5} \times \frac{2}{3} = \frac{3\times2}{5\times3} = \frac{6}{15} = \frac{2}{5}$.

So, Lina completed $\frac{2}{5}$ of a painting this week.

(b) Last week: $\frac{3}{5}$

This week: $\frac{2}{5}$

To compare, $\frac{3}{5} > \frac{2}{5}$.

Lina completed more of her painting last week.

The difference is: $\frac{3}{5} - \frac{2}{5} = \frac{1}{5}$.

Therefore, Lina completed $\frac{1}{5}$ more of a painting last week than this week.

\section*{Question 65}
\textbf{Metadata}

\begin{itemize}
  \item Question ID: P5-FrMulPIm\_P2-FrAdd2nd\_GPT4.1\_Recreation\_01
  \item Primary KC: FRACTIONS | Multiplication | multiplying a proper fraction and a proper/improper fractions
  \item Secondary KC: FRACTIONS | Addition | adding fractions
  \item Topic: Recreation such as sports, games, exercises, music, movie, dancing, painting, fishing and other recreation activities
  \item Grade: Primary 5
\end{itemize}

\textbf{Question}

Jia Wei spent $\frac{3}{5}$ of his weekend playing badminton and $\frac{2}{3}$ as much time on basketball as he did on badminton. \newline
(a) What fraction of his weekend did Jia Wei spend playing basketball? \newline
(b) What fraction of his weekend did Jia Wei spend playing badminton and basketball altogether?

\textbf{Solution}

(a) The fraction of the weekend Jia Wei spent on basketball is $ \frac{3}{5} \times \frac{2}{3} = \frac{2}{5} $.\newline
(b) Jia Wei spent \( \frac{3}{5} \) of his weekend on badminton and \( \frac{2}{5} \) on basketball. Altogether, he spent $ \frac{3}{5} + \frac{2}{5} = \frac{5}{5} = 1 $ \newline
So, Jia Wei spent the whole weekend playing badminton and basketball in total.

\section*{Question 66}
\textbf{Metadata}

\begin{itemize}
  \item Question ID: P5-FrMulPIm\_P2-FrSub2nd\_GPT4.1\_Recreation\_01
  \item Primary KC: FRACTIONS | Multiplication | multiplying a proper fraction and a proper/improper fractions
  \item Secondary KC: FRACTIONS | Subtraction | subtracting fractions
  \item Topic: Recreation such as sports, games, exercises, music, movie, dancing, painting, fishing and other recreation activities
  \item Grade: Primary 5
\end{itemize}

\textbf{Question}

Ella spent $\dfrac{3}{5}$ of an hour practising piano. Out of this time, she spent $\dfrac{4}{7}$ of it playing scales, and the rest playing songs. After practising, she realised she practised piano $\dfrac{1}{5}$ \text{hour} less than her friend Ben did.

(a) How many hours did Ella spend playing scales?

(b) How many hours did Ben spend practising piano?

\textbf{Solution}

(a) To find how many hours Ella spent playing scales:

Ella practised piano for $\dfrac{3}{5}$ \text{hour}.
She spent $\dfrac{4}{7}$ of this time on scales:

$\dfrac{4}{7} \times \dfrac{3}{5} = \dfrac{4 \times 3}{7 \times 5} = \dfrac{12}{35}$ \text{hour}.

Answer: Ella spent $\dfrac{12}{35}$ \text{hour} playing scales.

(b) Ella practised piano $\dfrac{1}{5}$ \text{hour} less than Ben:

Amount of time Ben spent $=$ Amount of time Ella spent $+$ $\dfrac{1}{5}$ \text{hour}

Ella: $\dfrac{3}{5}$ \text{hour}

So, Ben: $\dfrac{3}{5} + \dfrac{1}{5} = \dfrac{4}{5}$ \text{hour}

Answer: Ben spent $\dfrac{4}{5}$ \text{hour} practising piano.

\section*{Question 67}
\textbf{Metadata}

\begin{itemize}
  \item Question ID: P5-FrMulPIm\_P3-FrSmp\_GPT4.1\_Recreation\_01
  \item Primary KC: FRACTIONS | Multiplication | multiplying a proper fraction and a proper/improper fractions
  \item Secondary KC: FRACTIONS | Simplifying | expressing a fraction in its simplest form
  \item Topic: Recreation such as sports, games, exercises, music, movie, dancing, painting, fishing and other recreation activities
  \item Grade: Primary 5
\end{itemize}

\textbf{Question}

Sarah attends a weekly painting class. During each session, she uses $\frac{3}{4}$ of a tube of red paint. If she attends $\frac{5}{3}$ as many sessions this month as she did last month, what fraction of a tube of red paint did she use in total this month? Express your answer in its simplest form.

\textbf{Solution}

To find out how much red paint Sarah used this month, we need to multiply the number of sessions (in terms of last month) by the amount of paint used per session. 

She used $\frac{3}{4}$ of a tube per session and attended $\frac{5}{3}$ as many sessions. 

Total paint used $= \frac{3}{4} \times \frac{5}{3}$

$= \frac{3 \times 5}{4 \times 3}$

$= \frac{15}{12}$

Now, simplify $\frac{15}{12}$ by dividing both numerator and denominator by 3:

$\frac{15 \div 3}{12 \div 3} = \frac{5}{4}$

So, Sarah used $\frac{5}{4}$ tubes of red paint in total this month.

\section*{Question 68}
\textbf{Metadata}

\begin{itemize}
  \item Question ID: P5-FrMulPIm\_P5-FrCnv2Dc\_GPT4.1\_Recreation\_01
  \item Primary KC: FRACTIONS | Multiplication | multiplying a proper fraction and a proper/improper fractions
  \item Secondary KC: FRACTIONS | Conversion to decimals | expressing fractions as decimals
  \item Topic: Recreation such as sports, games, exercises, music, movie, dancing, painting, fishing and other recreation activities
  \item Grade: Primary 5
\end{itemize}

\textbf{Question}

During an art class, Sarah used $\frac{3}{4}$ of a bottle of blue paint to create the background of her painting. Later, she used $\frac{5}{6}$ of what was left in the bottle to add details.

(a) What fraction of the whole bottle did Sarah use for the details?

(b) Express your answer in (a) as a decimal, correct to 2 decimal places.

\textbf{Solution}

Let the amount of paint in the bottle be 1.

Sarah used $\frac{3}{4}$ of the bottle for the background, so paint remaining:

$1 - \frac{3}{4} = \frac{1}{4}$

Amount used for details:

$\frac{5}{6}$ of $\frac{1}{4} = \frac{5}{6} \times \frac{1}{4} = \frac{5}{24}$

(a) Sarah used $\frac{5}{24}$ of the whole bottle for the details.

(b) To express $\frac{5}{24}$ as a decimal:

$\frac{5}{24} = 0.2083$ (to 4 decimal places)

Correct to 2 decimal places: $0.21$

So, $\frac{5}{24} \approx 0.21$ (correct to 2 decimal places).

\section*{Question 69}
\textbf{Metadata}

\begin{itemize}
  \item Question ID: P5-FrMulImIm\_P2-FrCmp\_GPT4.1\_Recreation\_01
  \item Primary KC: FRACTIONS | Multiplication | multiplying two improper fractions
  \item Secondary KC: FRACTIONS | Comparison and ordering | comparing and ordering fractions
  \item Topic: Recreation such as sports, games, exercises, music, movie, dancing, painting, fishing and other recreation activities
  \item Grade: Primary 5
\end{itemize}

\textbf{Question}

During the school's annual sports day, two teams are competing in a skipping rope challenge. Team A completes $\frac{9}{4}$ times as many skips as Team B in each round. Team B completes $\frac{7}{3}$ times as many skips as Team C in each round. If Team C completes 20 skips in a round, 

(a) How many skips does Team A complete in a round? 

(b) Arrange the teams in order from the least to the most skips completed in a round.

\textbf{Solution}

(a) First, find the number of skips Team B completes:
$
\text{Team B:}\quad 20 \times \frac{7}{3} = \frac{20 \times 7}{3} = \frac{140}{3} \approx 46.67~\text{skips}
$

Next, find the number of skips Team A completes:
$
\text{Team A:}\quad \left(\frac{140}{3}\right) \times \frac{9}{4} = \frac{140 \times 9}{3 \times 4} = \frac{1260}{12} = 105~\text{skips}
$

(b) Comparing the skips:
\begin{align*}
\text{Team C:} &\quad 20~\text{skips} \\
\text{Team B:} &\quad \frac{140}{3} \approx 46.67~\text{skips} \\
\text{Team A:} &\quad 105~\text{skips}
\end{align*}

Order from the least to the most skips:
$
\text{Team C} < \text{Team B} < \text{Team A}
$

\section*{Question 70}
\textbf{Metadata}

\begin{itemize}
  \item Question ID: P5-FrMulImIm\_P2-FrAdd2nd\_GPT4.1\_Recreation\_01
  \item Primary KC: FRACTIONS | Multiplication | multiplying two improper fractions
  \item Secondary KC: FRACTIONS | Addition | adding fractions
  \item Topic: Recreation such as sports, games, exercises, music, movie, dancing, painting, fishing and other recreation activities
  \item Grade: Primary 5
\end{itemize}

\textbf{Question}

A group of friends decided to play badminton together. Each game lasts $\frac{7}{3}$ hours. They played $\frac{5}{2}$ games during their gathering, as some games were incomplete. 

(a) How many hours did they spend playing badminton in total? Express your answer as an improper fraction.

After playing, they enjoyed two snack breaks. The first break lasted $\frac{2}{3}$ hour and the second break lasted $\frac{3}{4}$ hour. 

(b) How many hours did they spend on snack breaks in total? Give your answer as an improper fraction.

(c) What was the total time spent at the gathering (playing and breaks)?

\textbf{Solution}

(a) Total hours spent playing badminton:

To find the total hours, multiply the duration of each game by the number of games:

$$\frac{7}{3} \times \frac{5}{2} = \frac{7 \times 5}{3 \times 2} = \frac{35}{6}$$

So, they spent $\frac{35}{6}$ hours playing badminton.

(b) Total time for snack breaks:

Add the two break durations:

$$\frac{2}{3} + \frac{3}{4}$$
Find a common denominator, which is $12$.
$$\frac{2}{3} = \frac{8}{12}$$
$$\frac{3}{4} = \frac{9}{12}$$
Add:
$$\frac{8}{12} + \frac{9}{12} = \frac{17}{12}$$

So, they spent $\frac{17}{12}$ hours on breaks.

(c) Total time at the gathering:

Add the total playing time and total break time:

$$\frac{35}{6} + \frac{17}{12}$$
Find a common denominator, which is $12$.
$$\frac{35}{6} = \frac{70}{12}$$
$$\frac{17}{12} = \frac{17}{12}$$
Add:
$$\frac{70}{12} + \frac{17}{12} = \frac{87}{12}$$

So, the total time spent at the gathering was $\frac{87}{12}$ hours.

\section*{Question 71}
\textbf{Metadata}

\begin{itemize}
  \item Question ID: P5-FrMulImIm\_P2-FrSub2nd\_GPT4.1\_Recreation\_01
  \item Primary KC: FRACTIONS | Multiplication | multiplying two improper fractions
  \item Secondary KC: FRACTIONS | Subtraction | subtracting fractions
  \item Topic: Recreation such as sports, games, exercises, music, movie, dancing, painting, fishing and other recreation activities
  \item Grade: Primary 5
\end{itemize}

\textbf{Question}

Maya attended a music camp where each session lasted $\frac{7}{4}$ hours. She attended $\frac{5}{3}$ of such sessions during the first day. On the second day, she spent $\frac{11}{6}$ hours practising the violin by herself. 

(a) How many hours did Maya spend in music sessions on the first day? 

(b) After the first day, how many more hours did she spend in music sessions compared to practising by herself on the second day?

\textbf{Solution}

(a) To find the number of hours Maya spent in music sessions on the first day, multiply:

$
\frac{7}{4} \times \frac{5}{3} = \frac{7 \times 5}{4 \times 3} = \frac{35}{12}\ \text{hours}
$

(b) To find how many more hours she spent in music sessions than practising:

$
\frac{35}{12} - \frac{11}{6}
$
First, convert $\frac{11}{6}$ to have a common denominator of 12:
$
\frac{11}{6} = \frac{11 \times 2}{6 \times 2} = \frac{22}{12}
$
Now subtract:
$
\frac{35}{12} - \frac{22}{12} = \frac{13}{12}\ \text{hours}
$

**Answer:**
(a) Maya spent $\frac{35}{12}$ hours in music sessions on the first day.

(b) She spent $\frac{13}{12}$ hours more in music sessions than practising by herself.

\section*{Question 72}
\textbf{Metadata}

\begin{itemize}
  \item Question ID: P5-FrMulImIm\_P3-FrSmp\_GPT4.1\_Recreation\_01
  \item Primary KC: FRACTIONS | Multiplication | multiplying two improper fractions
  \item Secondary KC: FRACTIONS | Simplifying | expressing a fraction in its simplest form
  \item Topic: Recreation such as sports, games, exercises, music, movie, dancing, painting, fishing and other recreation activities
  \item Grade: Primary 5
\end{itemize}

\textbf{Question}

A group of students is practising for a dance performance. Each student practises for $\frac{7}{4}$ hours each day. There are $\frac{9}{5}$ times as many practice days this month compared to last month. 

How many hours does each student spend practising in total this month? Give your answer as a fraction in its simplest form.

\textbf{Solution}

Each student practises $\frac{7}{4}$ hours each day, and there are $\frac{9}{5}$ times as many days this month. To find the total hours each student spends:

Total hours $= \frac{7}{4} \times \frac{9}{5}$

First, multiply the numerators and denominators:

$\frac{7}{4} \times \frac{9}{5} = \frac{7 \times 9}{4 \times 5} = \frac{63}{20}$

Next, check if $\frac{63}{20}$ can be simplified. The greatest common divisor of 63 and 20 is 1, so it is already in its simplest form.

**Final Answer:** Each student spends $\frac{63}{20}$ hours practising in total this month.

\section*{Question 73}
\textbf{Metadata}

\begin{itemize}
  \item Question ID: P5-FrMulImIm\_P5-FrCnv2Dc\_GPT4.1\_Recreation\_01
  \item Primary KC: FRACTIONS | Multiplication | multiplying two improper fractions
  \item Secondary KC: FRACTIONS | Conversion to decimals | expressing fractions as decimals
  \item Topic: Recreation such as sports, games, exercises, music, movie, dancing, painting, fishing and other recreation activities
  \item Grade: Primary 5
\end{itemize}

\textbf{Question}

Sarah participated in a dance competition where she needed to practise a routine. On Monday, she completed $\frac{7}{4}$ of her planned practice time, and on Tuesday, she managed to practise $\frac{5}{3}$ of that time. What fraction of her planned practice time did she complete on both days combined? Express your answer as a decimal.

\textbf{Solution}

Sarah practised $\frac{7}{4}$ of her planned time on Monday, and $\frac{5}{3}$ times that on Tuesday.

First, find the fraction of her planned time practised on Tuesday:

$\frac{7}{4} \times \frac{5}{3} = \frac{7 \times 5}{4 \times 3} = \frac{35}{12}$

So, on Tuesday, Sarah practised $\frac{35}{12}$ of her planned time.

Next, express this fraction as a decimal:

$\frac{35}{12} = 2.9166... \approx 2.92$

Sarah completed about \textdollar2.92 times her planned practice time on Tuesday. 

**Final Answer:**

On Tuesday, Sarah completed $\frac{35}{12}$ of her planned practice time, which is approximately $2.92$ times her planned practice time.

\section*{Question 74}
\textbf{Metadata}

\begin{itemize}
  \item Question ID: P5-FrMulMixN\_P2-FrCmp\_GPT4.1\_Recreation\_01
  \item Primary KC: FRACTIONS | Multiplication | multiplying a mixed number and a whole number
  \item Secondary KC: FRACTIONS | Comparison and ordering | comparing and ordering fractions
  \item Topic: Recreation such as sports, games, exercises, music, movie, dancing, painting, fishing and other recreation activities
  \item Grade: Primary 5
\end{itemize}

\textbf{Question}

Jonathan spent $2\frac{1}{2}$ hours practising piano each day for 4 days during the school holiday. His sister, Emily, practised violin for $9\frac{3}{4}$ hours in total over the same 4 days. \newline 
(a) How many hours did Jonathan spend practising piano over the 4 days? \newline 
(b) Who spent more time practising their instrument during these 4 days, Jonathan or Emily? Show your working clearly.

\textbf{Solution}

(a) Jonathan's total practice time: \newline 
$2\frac{1}{2} \times 4$ 

First, convert $2\frac{1}{2}$ to an improper fraction: 
$2\frac{1}{2} = \frac{5}{2}$ 

Now multiply by 4:
$\frac{5}{2} \times 4 = \frac{5 \times 4}{2} = \frac{20}{2} = 10$

So, Jonathan spent 10 hours practising piano. 

(b) Emily practised for $9\frac{3}{4}$ hours. Convert this to an improper fraction:
$9\frac{3}{4} = \frac{39}{4}$

Jonathan's practice time was 10 hours. Compare 10 and $9\frac{3}{4}$:
$10 = \frac{40}{4}$, $9\frac{3}{4} = \frac{39}{4}$

Since $\frac{40}{4} > \frac{39}{4}$, Jonathan practised longer than Emily.

\textbf{Answer:} 
(a) Jonathan spent 10 hours practising piano. 
(b) Jonathan spent more time practising his instrument than Emily.

\section*{Question 75}
\textbf{Metadata}

\begin{itemize}
  \item Question ID: P5-FrMulMixN\_P2-FrAdd2nd\_GPT4.1\_Recreation\_01
  \item Primary KC: FRACTIONS | Multiplication | multiplying a mixed number and a whole number
  \item Secondary KC: FRACTIONS | Addition | adding fractions
  \item Topic: Recreation such as sports, games, exercises, music, movie, dancing, painting, fishing and other recreation activities
  \item Grade: Primary 5
\end{itemize}

\textbf{Question}

Alyssa attended a weekend art camp where she painted for $2\frac{1}{2}$ hours each day. The camp lasted for 3 days.

(a) What is the total number of hours Alyssa spent painting during the camp?

On the last day, she also spent $\frac{3}{4}$ hour drawing sketches after painting.

(b) How much time in total did Alyssa spend on art activities (painting and sketching) on the last day?

\textbf{Solution}

(a) The total number of hours Alyssa spent painting during the camp is:

$2\frac{1}{2} \times 3 = \frac{5}{2} \times 3 = \frac{15}{2} = 7\frac{1}{2}$ hours.

(b) On the last day, she spent $2\frac{1}{2}$ hours painting and $\frac{3}{4}$ hour drawing sketches. The total time is:

$2\frac{1}{2} + \frac{3}{4} = \frac{5}{2} + \frac{3}{4}$

To add these, convert $\frac{5}{2}$ to quarters: $\frac{5}{2} = \frac{10}{4}$

So $\frac{10}{4} + \frac{3}{4} = \frac{13}{4} = 3\frac{1}{4}$

Alyssa spent a total of $3\frac{1}{4}$ hours on art activities on the last day.

\section*{Question 76}
\textbf{Metadata}

\begin{itemize}
  \item Question ID: P5-FrMulMixN\_P2-FrSub2nd\_GPT4.1\_Recreation\_01
  \item Primary KC: FRACTIONS | Multiplication | multiplying a mixed number and a whole number
  \item Secondary KC: FRACTIONS | Subtraction | subtracting fractions
  \item Topic: Recreation such as sports, games, exercises, music, movie, dancing, painting, fishing and other recreation activities
  \item Grade: Primary 5
\end{itemize}

\textbf{Question}

Sarah is practising piano for a concert. Each day, she spends $2\frac{1}{2}$ hours practising. She practised for 4 days in a week. 

(a) How many hours did Sarah spend practising in total over these 4 days?

(b) On Saturday, Sarah had to leave early and practised only $1\frac{3}{4}$ hours. How many more hours did she practise in total from Monday to Thursday than on Saturday?

\textbf{Solution}

(a) Each day, Sarah practised $2\frac{1}{2}$ hours. In 4 days, she practised:

$
2\frac{1}{2} \times 4 = \left(\frac{5}{2}\right) \times 4 = \frac{5\times4}{2} = \frac{20}{2} = 10\text{ hours}
$

(b) On Saturday, she practised $1\frac{3}{4}$ hours:

$
10 - 1\frac{3}{4} = 10 - \frac{7}{4}
$
Convert 10 to quarters:

$
10 = \frac{40}{4}
$
So,
$
\frac{40}{4} - \frac{7}{4} = \frac{33}{4} = 8\frac{1}{4}
$

Sarah practised $8\frac{1}{4}$ more hours from Monday to Thursday than on Saturday.

\section*{Question 77}
\textbf{Metadata}

\begin{itemize}
  \item Question ID: P5-FrMulMixN\_P3-FrSmp\_GPT4.1\_Recreation\_01
  \item Primary KC: FRACTIONS | Multiplication | multiplying a mixed number and a whole number
  \item Secondary KC: FRACTIONS | Simplifying | expressing a fraction in its simplest form
  \item Topic: Recreation such as sports, games, exercises, music, movie, dancing, painting, fishing and other recreation activities
  \item Grade: Primary 5
\end{itemize}

\textbf{Question}

Melissa is making fruit punch for her school's sports day. Each jug can hold $2\dfrac{1}{2}$ litres of fruit punch. She wants to prepare 6 jugs for the event. 

(a) What is the total amount of fruit punch Melissa will prepare? 

(b) Give your answer in its simplest fractional form, if necessary.

\textbf{Solution}

Let the amount of fruit punch in each jug be $2\dfrac{1}{2}$ litres. 

First, write $2\dfrac{1}{2}$ as an improper fraction:

$2\dfrac{1}{2} = 2 + \dfrac{1}{2} = \dfrac{4}{2} + \dfrac{1}{2} = \dfrac{5}{2}$

(a) Total amount of fruit punch for 6 jugs = $6 \times \dfrac{5}{2}$

$6 \times \dfrac{5}{2} = \dfrac{6 \times 5}{2} = \dfrac{30}{2}$

(b) Simplify $\dfrac{30}{2}$:

$\dfrac{30}{2} = 15$

Melissa will prepare 15 litres of fruit punch in total.

\section*{Question 78}
\textbf{Metadata}

\begin{itemize}
  \item Question ID: P5-FrMulMixN\_P5-FrCnv2Dc\_GPT4.1\_Recreation\_01
  \item Primary KC: FRACTIONS | Multiplication | multiplying a mixed number and a whole number
  \item Secondary KC: FRACTIONS | Conversion to decimals | expressing fractions as decimals
  \item Topic: Recreation such as sports, games, exercises, music, movie, dancing, painting, fishing and other recreation activities
  \item Grade: Primary 5
\end{itemize}

\textbf{Question}

Alicia attends a weekly dance class that lasts $1\frac{1}{2}$ hours each session. She will attend 4 such sessions this month. 

(a) What is the total number of hours Alicia will spend in dance classes this month?

(b) Express your answer from part (a) in decimal form.

\textbf{Solution}

(a) Alicia attends 4 sessions, each lasting $1\frac{1}{2}$ hours.

Let us convert the mixed number to an improper fraction:

$1\frac{1}{2} = \frac{3}{2}$

Now, multiply by 4:

$\frac{3}{2} \times 4 = \frac{3 \times 4}{2} = \frac{12}{2} = 6$

So, Alicia will spend $6$ hours in dance classes this month.

(b) Express $6$ hours as a decimal:

$6$ hours is already in decimal form.

\textbf{Final Answers:}

(a) $6$ hours

(b) $6$

\section*{Question 79}
\textbf{Metadata}

\begin{itemize}
  \item Question ID: P5-DcMul3dK\_P4-DcCmp3d\_GPT4.1\_Recreation\_01
  \item Primary KC: DECIMALS | Multiplication | multiplying decimals (up to 3 decimal places) by 10, 100, 1000 and their multiples
  \item Secondary KC: DECIMALS | Comparison and ordering | comparing and ordering decimals up to 3 decimal places
  \item Topic: Recreation such as sports, games, exercises, music, movie, dancing, painting, fishing and other recreation activities
  \item Grade: Primary 5
\end{itemize}

\textbf{Question}

During a music competition, each participant has a time limit to play their instrument. For the first round, each participant is given $1.250$ minutes to perform. In the finals, the organisers decided to increase the performance time for each finalist by $10$ times the first round's time. After the final round, the judges recorded the actual performance times of the top three finalists: $12.5$ minutes, $13.000$ minutes, and $12.750$ minutes.\
(a) What is the performance time allowed for each finalist in the final round?\
(b) Arrange the actual performance times of the top three finalists in order from shortest to longest.

\textbf{Solution}

(a) The allowed performance time for each finalist in the final round is: \
$ 1.250 \times 10 = 12.500 \text{ minutes} $\
So, each finalist is allowed \textdollar12.500 minutes to perform in the final round.\\
(b) The actual performance times are: $12.5$, $13.000$, and $12.750$ minutes.\
To compare and order these times (aligning them to 3 decimal places):\
$ 12.500,\ 12.750,\ 13.000 $
Therefore, the order from shortest to longest is:
$12.5$ minutes, $12.750$ minutes, $13.000$ minutes.

\section*{Question 80}
\textbf{Metadata}

\begin{itemize}
  \item Question ID: P5-DcMul3dK\_P4-DcCnv2Fr\_GPT4.1\_Recreation\_01
  \item Primary KC: DECIMALS | Multiplication | multiplying decimals (up to 3 decimal places) by 10, 100, 1000 and their multiples
  \item Secondary KC: DECIMALS | Conversion from decimals to fraction | expressing decimals as fractions
  \item Topic: Recreation such as sports, games, exercises, music, movie, dancing, painting, fishing and other recreation activities
  \item Grade: Primary 5
\end{itemize}

\textbf{Question}

Sarah goes swimming every Saturday. She swims 0.125 km each lap. Last Saturday, she swam 100 laps. 

(a) How many kilometres did Sarah swim in total last Saturday? 

(b) Express the total distance Sarah swam as a fraction in its simplest form.

\textbf{Solution}

(a) Sarah swam $0.125$ km for each lap and completed $100$ laps. 

Total distance = $0.125 \times 100 = 12.5$ km

(b) To express $12.5$ as a fraction: 

$12.5 = \dfrac{125}{10} = \dfrac{25}{2}$

So, the total distance Sarah swam, as a fraction in its simplest form, is $\dfrac{25}{2}$ kilometres.

\section*{Question 81}
\textbf{Metadata}

\begin{itemize}
  \item Question ID: P5-DcMul3dK\_P4-DcRnd3d\_GPT4.1\_Recreation\_01
  \item Primary KC: DECIMALS | Multiplication | multiplying decimals (up to 3 decimal places) by 10, 100, 1000 and their multiples
  \item Secondary KC: DECIMALS | Rounding | rounding decimals up to 3 decimal places to the nearest whole number, 1 decimal place and 2 decimal places 
  \item Topic: Recreation such as sports, games, exercises, music, movie, dancing, painting, fishing and other recreation activities
  \item Grade: Primary 5
\end{itemize}

\textbf{Question}

Amira is learning to play the piano. Each song she practises lasts about $3.275$ minutes. On Saturday, she played $20$ songs. 

(a) How many minutes did she spend practising on Saturday? 

(b) Rounded to the nearest whole number, how many minutes did Amira spend practising? 

(c) Rounded to $1$ decimal place, how many minutes did she spend practising?

\textbf{Solution}

(a) To find the total minutes Amira spent, multiply the duration of one song by the number of songs:

$3.275 \times 20 = 65.5$

Therefore, Amira practised for $65.5$ minutes on Saturday.

(b) Rounding $65.5$ to the nearest whole number:

Since $0.5$ rounds up, $65.5$ rounds to $66$.

Amira spent $\textdollar66$ minutes practising, rounded to the nearest whole number.

(c) Rounding $65.5$ to $1$ decimal place:

There are no more digits after the $0.5$, so it stays as $65.5$.

Amira spent $65.5$ minutes practising, rounded to $1$ decimal place.

\section*{Question 82}
\textbf{Metadata}

\begin{itemize}
  \item Question ID: P5-DcMul3dK\_P4-DcAdd2nd\_GPT4.1\_Recreation\_01
  \item Primary KC: DECIMALS | Multiplication | multiplying decimals (up to 3 decimal places) by 10, 100, 1000 and their multiples
  \item Secondary KC: DECIMALS | Addition | adding decimals
  \item Topic: Recreation such as sports, games, exercises, music, movie, dancing, painting, fishing and other recreation activities
  \item Grade: Primary 5
\end{itemize}

\textbf{Question}

Rachel is preparing treats for her dance class. She wants to pack small bags with an assortment of nuts. Each bag will contain $0.125$ kg of almonds, $0.25$ kg of cashews, and $0.4$ kg of walnuts.\
\\
If she needs to make $10$ of these treat bags,\\
(a) What is the total mass of nuts (in kg) in one bag?\\
(b) What is the total mass of nuts (in kg) needed to pack all $10$ bags?\\
(c) If Rachel wants to prepare the same number of bags for two different dance classes, how much nuts does she need in total for both classes?

\textbf{Solution}

(a) The total mass of nuts in one bag is:\\
$0.125 + 0.25 + 0.4 = 0.775$ kg\\
\\
(b) For $10$ bags:\\
$0.775 \times 10 = 7.75$ kg\\
\\
(c) For two dance classes, she needs $10 \times 2 = 20$ bags.\\
Total mass needed = $0.775 \times 20 = 15.5$ kg\\
\\
\textbf{Final Answers:}\\
(a) $0.775$ kg\\
(b) $7.75$ kg\\
(c) $15.5$ kg

\section*{Question 83}
\textbf{Metadata}

\begin{itemize}
  \item Question ID: P5-DcMul3dK\_P4-DcSub2nd\_GPT4.1\_Recreation\_01
  \item Primary KC: DECIMALS | Multiplication | multiplying decimals (up to 3 decimal places) by 10, 100, 1000 and their multiples
  \item Secondary KC: DECIMALS | Subtraction | subtracting decimals
  \item Topic: Recreation such as sports, games, exercises, music, movie, dancing, painting, fishing and other recreation activities
  \item Grade: Primary 5
\end{itemize}

\textbf{Question}

Siti is playing a music game where she earns points by hitting notes correctly. For each correct note, she earns $0.175$ points. She played a song with $100$ notes and got every note correct. 

After the song, her friend noticed that her total score displayed was $17.5$ points. However, there was a system error and $2.35$ points were wrongly added to her score. 

How many points should Siti have after subtracting the points that were wrongly added?

\textbf{Solution}

First, calculate the actual points Siti earned:

$
\text{Points per note} = 0.175
$
$
\text{Number of notes} = 100
$
$
\text{Total points} = 0.175 \times 100 = 17.5
$

The system added $2.35$ points by mistake, so we need to subtract this excess:

$
\text{Corrected score} = 17.5 - 2.35 = 15.15
$

\textbf{Answer:} Siti should have $15.15$ points after subtracting the wrongly added points.

\section*{Question 84}
\textbf{Metadata}

\begin{itemize}
  \item Question ID: P5-DcDiv3dK\_P4-DcCmp3d\_GPT4.1\_Recreation\_01
  \item Primary KC: DECIMALS | Division | dividing decimals (up to 3 decimal places) by 10, 100, 1000 and their multiples
  \item Secondary KC: DECIMALS | Comparison and ordering | comparing and ordering decimals up to 3 decimal places
  \item Topic: Recreation such as sports, games, exercises, music, movie, dancing, painting, fishing and other recreation activities
  \item Grade: Primary 5
\end{itemize}

\textbf{Question}

At a swimming competition, Sarah swam a total distance of $12.450$ metres over $10$ equal laps.

(a) What was the distance Sarah swam in each lap?

Later, the official times for each swimmer's lap are reported. Sarah's average lap distances for her $10$ laps are: $1.245$, $1.247$, $1.243$, $1.250$, $1.240$, $1.246$, $1.244$, $1.249$, $1.245$, and $1.248$ metres. 

(b) Arrange Sarah's lap distances in order from the shortest to the longest lap.

(c) Which lap distance was exactly in the middle after you arranged them?

\textbf{Solution}

(a) Total distance = $12.450$ metres, Number of laps = $10$.

Distance per lap = $12.450 \div 10 = 1.245$ metres.

(b) Ordering the lap distances in ascending order:

$1.240$, $1.243$, $1.244$, $1.245$, $1.245$, $1.246$, $1.247$, $1.248$, $1.249$, $1.250$

(c) There are $10$ laps, so the middle values are the $5$th and $6$th lap distances when ordered.

The $5$th distance: $1.245$ metres

The $6$th distance: $1.246$ metres

Average of the middle two: $\frac{1.245 + 1.246}{2} = 1.2455$ metres

Thus, the lap distances exactly in the middle are $1.245$ metres and $1.246$ metres. Their average is $1.2455$ metres.

\section*{Question 85}
\textbf{Metadata}

\begin{itemize}
  \item Question ID: P5-DcDiv3dK\_P4-DcCnv2Fr\_GPT4.1\_Recreation\_01
  \item Primary KC: DECIMALS | Division | dividing decimals (up to 3 decimal places) by 10, 100, 1000 and their multiples
  \item Secondary KC: DECIMALS | Conversion from decimals to fraction | expressing decimals as fractions
  \item Topic: Recreation such as sports, games, exercises, music, movie, dancing, painting, fishing and other recreation activities
  \item Grade: Primary 5
\end{itemize}

\textbf{Question}

Sarah is playing a song on the piano that lasts $0.125$ hours. After the song, her teacher asks her to divide the length of the song equally among $1000$ students so that each student gets to practise the same duration.

(a) How many hours does each student get to practise the song?

(b) Express the duration each student practises as a fraction in its simplest form.

\textbf{Solution}

(a) To find how many hours each student gets to practise, divide $0.125$ hours by $1000$:

$$ 0.125 \div 1000 = 0.000125 $$

So, each student gets $0.000125$ hours to practise the song.

(b) To express $0.000125$ as a fraction:

$$ 0.000125 = \frac{125}{1000000} $$

Now, simplify $\frac{125}{1000000}$:
- Divide numerator and denominator by $125$:
$$ \frac{125 \div 125}{1000000 \div 125} = \frac{1}{8000} $$

So, each student practises $\frac{1}{8000}$ hours.

\section*{Question 86}
\textbf{Metadata}

\begin{itemize}
  \item Question ID: P5-DcDiv3dK\_P4-DcRnd3d\_GPT4.1\_Recreation\_01
  \item Primary KC: DECIMALS | Division | dividing decimals (up to 3 decimal places) by 10, 100, 1000 and their multiples
  \item Secondary KC: DECIMALS | Rounding | rounding decimals up to 3 decimal places to the nearest whole number, 1 decimal place and 2 decimal places 
  \item Topic: Recreation such as sports, games, exercises, music, movie, dancing, painting, fishing and other recreation activities
  \item Grade: Primary 5
\end{itemize}

\textbf{Question}

Jia Wen recorded the time she took to complete each lap while swimming for her training. She swam a total distance of $500$ metres and her total time taken was $202.457$ seconds.

(a) What was Jia Wen's average time in seconds for each $10$ metres of swimming? Express your answer as a decimal correct to $3$ decimal places.

(b) Round your answer in (a) to the nearest whole number, $1$ decimal place, and $2$ decimal places.

\textbf{Solution}

(a) To find Jia Wen's average time per $10$ metres:

Total time $=202.457$ seconds

Total distance $=500$ metres

Number of $10$ metre segments in $500$ metres $= 500 \div 10 = 50$

Average time per $10$ metres $= 202.457 \div 50$

$202.457 \div 50 = 4.04914$

So, to $3$ decimal places:

Average time per $10$ metres $= 4.049$ seconds

(b) Rounding $4.049$:

- To the nearest whole number: $4$
- To $1$ decimal place: $4.0$
- To $2$ decimal places: $4.05$

\textbf{Answers:}

(a) $4.049$ seconds

(b) To the nearest whole number: $4$; to $1$ decimal place: $4.0$; to $2$ decimal places: $4.05$

\section*{Question 87}
\textbf{Metadata}

\begin{itemize}
  \item Question ID: P5-DcDiv3dK\_P4-DcAdd2nd\_GPT4.1\_Recreation\_01
  \item Primary KC: DECIMALS | Division | dividing decimals (up to 3 decimal places) by 10, 100, 1000 and their multiples
  \item Secondary KC: DECIMALS | Addition | adding decimals
  \item Topic: Recreation such as sports, games, exercises, music, movie, dancing, painting, fishing and other recreation activities
  \item Grade: Primary 5
\end{itemize}

\textbf{Question}

Wei Lin attended a swimming class where she swam a total distance of $8.250$ metres on Monday and $3.750$ metres on Wednesday. After her lessons each day, her coach asked her to divide the total distance she swam that week equally among $10$ days, as part of a new training plan. What is the average distance, in metres, that Wei Lin should plan to swim each day, rounded to three decimal places?

\textbf{Solution}

First, find the total distance Wei Lin swam over the two days:

$
8.250 + 3.750 = 12.000
$

Next, divide the total distance by $10$ to find the average distance per day:

$
12.000 \div 10 = 1.200
$

Therefore, Wei Lin should plan to swim \(1.200\) metres each day.

\section*{Question 88}
\textbf{Metadata}

\begin{itemize}
  \item Question ID: P5-DcDiv3dK\_P4-DcSub2nd\_GPT4.1\_Recreation\_01
  \item Primary KC: DECIMALS | Division | dividing decimals (up to 3 decimal places) by 10, 100, 1000 and their multiples
  \item Secondary KC: DECIMALS | Subtraction | subtracting decimals
  \item Topic: Recreation such as sports, games, exercises, music, movie, dancing, painting, fishing and other recreation activities
  \item Grade: Primary 5
\end{itemize}

\textbf{Question}

Sarah is organizing a badminton tournament and she has bought 14.250 litres of sports drink. She plans to fill bottles, each with 100 ml (0.100 litres) of the drink. After filling as many bottles as possible, she found that there was some sports drink left. If she drank 0.075 litres of the remaining drink, how much sports drink was left after she drank it? Express your answer in litres.

\textbf{Solution}

First, we find out how many bottles Sarah can fill:  $ 14.250 \div 0.100 = 142.5 $  This means she can fill 142 bottles, and there will be a remainder.  Amount used to fill bottles: $ 142 \times 0.100 = 14.200 \text{ litres} $  Leftover drink: $ 14.250 - 14.200 = 0.050 \text{ litres} $  After Sarah drank 0.075 litres, since she only had 0.050 litres left, it means she finishes the leftover drink and there is none left:  $ 0.050 - 0.075 = -0.025 $  However, since there cannot be negative drink left, Sarah could only drink 0.050 litres and there is \boxed{0} litres left after she drank.

\section*{Question 89}
\textbf{Metadata}

\begin{itemize}
  \item Question ID: P5-PcRepWh\_P1-WNAdd2nd\_GPT4.1\_Recreation\_01
  \item Primary KC: PERCENTAGE | Representation and concept | expressing a part of a whole as a percentage
  \item Secondary KC: WHOLE NUMBERS | Addition | adding whole numbers
  \item Topic: Recreation such as sports, games, exercises, music, movie, dancing, painting, fishing and other recreation activities
  \item Grade: Primary 5
\end{itemize}

\textbf{Question}

During the school’s annual Sports Day, there were $120$ students who participated in running, $80$ students in swimming, and $100$ students in basketball. 

(a) How many students participated in these three sports altogether?

(b) What percentage of the total participants joined the swimming event? Express your answer correct to the nearest whole number.

\textbf{Solution}

(a) First, we find the total number of students who participated:

$120 + 80 + 100 = 300$

So, there were $300$ students who participated in these three sports altogether.

(b) To find the percentage of participants in the swimming event:

$
\text{Percentage} = \frac{80}{300} \times 100\%
$
$\frac{80}{300} = 0.2666\ldots$

$0.2666\ldots \times 100\% = 26.666\ldots\%$

Correct to the nearest whole number, this is $27\%$.

\textbf{Answer:}

(a) $300$ students 

(b) $27\%$ of the participants joined the swimming event.

\section*{Question 90}
\textbf{Metadata}

\begin{itemize}
  \item Question ID: P5-PcRepWh\_P1-WNSub2nd\_GPT4.1\_Recreation\_01
  \item Primary KC: PERCENTAGE | Representation and concept | expressing a part of a whole as a percentage
  \item Secondary KC: WHOLE NUMBERS | Subtraction | subtracting whole numbers
  \item Topic: Recreation such as sports, games, exercises, music, movie, dancing, painting, fishing and other recreation activities
  \item Grade: Primary 5
\end{itemize}

\textbf{Question}

During the school holidays, a group of 60 students attended a painting workshop. After the lunch break, 45 students returned to continue painting, while the rest went home. 

(a) How many students went home after the lunch break?

(b) What percentage of the students stayed to continue painting after the lunch break?

\textbf{Solution}

(a) The number of students who went home after the lunch break is:

$60 - 45 = 15$

So, $15$ students went home.

(b) To find the percentage of students who stayed to continue painting:

$
\text{Percentage stayed} = \left( \frac{45}{60} \right) \times 100\% = 75\%
$

Therefore, $75\%$ of the students stayed to continue painting after the lunch break.

\section*{Question 91}
\textbf{Metadata}

\begin{itemize}
  \item Question ID: P5-PcRepWh\_P1-WNMul2nd\_GPT4.1\_Recreation\_01
  \item Primary KC: PERCENTAGE | Representation and concept | expressing a part of a whole as a percentage
  \item Secondary KC: WHOLE NUMBERS | Multiplication | multiplying whole numbers
  \item Topic: Recreation such as sports, games, exercises, music, movie, dancing, painting, fishing and other recreation activities
  \item Grade: Primary 5
\end{itemize}

\textbf{Question}

During the school sports day, there were 250 students participating in various games. Out of these, 70 students played basketball. What percentage of the students played basketball?

\textbf{Solution}

The total number of students is $250$. The number of students who played basketball is $70$.

The percentage of students who played basketball is:
$
\frac{70}{250} \times 100\% = \frac{70 \times 100}{250}\% = \frac{7000}{250}\% = 28\%
$

So, $28\%$ of the students played basketball.

\section*{Question 92}
\textbf{Metadata}

\begin{itemize}
  \item Question ID: P5-PcRepWh\_P1-WNDiv2nd\_GPT4.1\_Recreation\_01
  \item Primary KC: PERCENTAGE | Representation and concept | expressing a part of a whole as a percentage
  \item Secondary KC: WHOLE NUMBERS | Division | dividing whole numbers
  \item Topic: Recreation such as sports, games, exercises, music, movie, dancing, painting, fishing and other recreation activities
  \item Grade: Primary 5
\end{itemize}

\textbf{Question}

Sarah participated in a painting competition with 80 participants. At the end of the competition, she found out that 32 participants received certificates of participation. 

What percentage of the participants received certificates of participation?

\textbf{Solution}

First, we need to find what fraction of the participants received certificates. 

$\frac{32}{80} = 0.4$

To express this as a percentage, multiply by $100$:

$0.4 \times 100 = 40$

So, $40\%$ of the participants received certificates of participation.

\section*{Question 93}
\textbf{Metadata}

\begin{itemize}
  \item Question ID: P5-RtFndR\_P2-DcCnvN2D\_GPT4.1\_Recreation\_01
  \item Primary KC: RATE | Finding rate | finding rate given total amount and number of units
  \item Secondary KC: DECIMALS | Conversion to larger units | converting a measurement from a smaller unit to a larger unit in decimal form
  \item Topic: Recreation such as sports, games, exercises, music, movie, dancing, painting, fishing and other recreation activities
  \item Grade: Primary 5
\end{itemize}

\textbf{Question}

During a dance rehearsal, Emily practised for a total of $150$ minutes over $5$ days. Each day, she practised for the same amount of time. 

(a) What was the average number of minutes Emily practised each day?

(b) Express the average practice time per day in hours, giving your answer in decimal form.

\textbf{Solution}

(a) To find the average number of minutes Emily practised each day, divide the total time by the number of days:

$
\text{Average minutes per day} = \frac{150}{5} = 30 \text{ minutes}
$

(b) To convert $30$ minutes to hours:

There are $60$ minutes in $1$ hour.

$
\text{Hours per day} = \frac{30}{60} = 0.5 \text{ hours}
$

So, the average practice time per day is $0.5$ hours.

\section*{Question 94}
\textbf{Metadata}

\begin{itemize}
  \item Question ID: P5-RtFndR\_P2-DcCnvD2N\_GPT4.1\_Recreation\_01
  \item Primary KC: RATE | Finding rate | finding rate given total amount and number of units
  \item Secondary KC: DECIMALS | Conversion to smaller units | converting a measurement from a larger unit in decimal form to a smaller unit
  \item Topic: Recreation such as sports, games, exercises, music, movie, dancing, painting, fishing and other recreation activities
  \item Grade: Primary 5
\end{itemize}

\textbf{Question}

The children in a music club played the piano for a total of $6.4$ hours across $8$ days during the school holidays. Every day, they played for the same amount of time. 

(a) How many hours did they play the piano each day?

(b) Express the number of hours each day in minutes.

\textbf{Solution}

(a) The total number of hours played is $6.4$ hours. The piano was played over $8$ days.

The number of hours played each day $= \dfrac{6.4}{8} = 0.8$ hours.

(b) $1$ hour $= 60$ minutes.

$0.8$ hours $= 0.8 \times 60 = 48$ minutes.

\textbf{Answer:}
(a) $0.8$ hours per day.
(b) $48$ minutes per day.

\section*{Question 95}
\textbf{Metadata}

\begin{itemize}
  \item Question ID: P5-RtFndT\_P2-DcCnvN2D\_GPT4.1\_Recreation\_01
  \item Primary KC: RATE | Finding total amount | finding total amount, given rate and number of units
  \item Secondary KC: DECIMALS | Conversion to larger units | converting a measurement from a smaller unit to a larger unit in decimal form
  \item Topic: Recreation such as sports, games, exercises, music, movie, dancing, painting, fishing and other recreation activities
  \item Grade: Primary 5
\end{itemize}

\textbf{Question}

A swimming pool charges \textdollar1.20 for every 100 metres swum. During a training session, Farah swims 1,250 metres. 

(a) Convert the total distance Farah swims into kilometres, giving your answer in decimal form. 
(b) How much does Farah have to pay for her training session?

\textbf{Solution}

(a) To convert metres to kilometres, divide the number of metres by $1,000$:

$1,250 \div 1,000 = 1.25$

Farah swims $1.25$ kilometres.

(b) Each $100$ metres costs \textdollar1.20. Find the number of $100$ metre units in $1,250$ metres:

$1,250 \div 100 = 12.5$

Total cost $= 12.5 \times \textdollar1.20 = \textdollar15.00$

Farah has to pay \textdollar15.00 for her training session.

\section*{Question 96}
\textbf{Metadata}

\begin{itemize}
  \item Question ID: P5-RtFndT\_P2-DcCnvD2N\_GPT4.1\_Recreation\_01
  \item Primary KC: RATE | Finding total amount | finding total amount, given rate and number of units
  \item Secondary KC: DECIMALS | Conversion to smaller units | converting a measurement from a larger unit in decimal form to a smaller unit
  \item Topic: Recreation such as sports, games, exercises, music, movie, dancing, painting, fishing and other recreation activities
  \item Grade: Primary 5
\end{itemize}

\textbf{Question}

Clara paid \textdollar3.50 per hour to rent a tennis court. She played tennis for $2.5$ hours on Saturday afternoon. How much did she pay in total for the court rental? Give your answer in cents.

\textbf{Solution}

First, we find the total cost using the given rate and time spent:

$
\text{Total cost} = 3.50 \times 2.5 = 8.75
$

Clara paid \textdollar8.75 in total.

Next, we convert \textdollar8.75 to cents:

$
\text{textdollar}8.75 = 8.75 \times 100 = 875 \text{ cents}
$

\textbf{Final Answer:} Clara paid \boxed{875} cents in total for the court rental.

\section*{Question 97}
\textbf{Metadata}

\begin{itemize}
  \item Question ID: P5-RtFndU\_P2-DcCnvN2D\_GPT4.1\_Recreation\_01
  \item Primary KC: RATE | Finding number of unit | finding number of units given rate and total amount
  \item Secondary KC: DECIMALS | Conversion to larger units | converting a measurement from a smaller unit to a larger unit in decimal form
  \item Topic: Recreation such as sports, games, exercises, music, movie, dancing, painting, fishing and other recreation activities
  \item Grade: Primary 5
\end{itemize}

\textbf{Question}

During a school swimming competition, Sarah swam a total of $1500$ metres. Each lap in the pool is $25$ metres long.\
(a) How many laps did Sarah swim? \
(b) If the total distance Sarah swam is to be converted to kilometres, what is the distance in kilometres expressed in decimal form?

\textbf{Solution}

(a) To find the number of laps Sarah swam: \
$\text{Number of laps} = \frac{1500}{25} = 60$\
Sarah swam $60$ laps.\\
(b) To convert $1500$ metres to kilometres: \
$\text{Distance in kilometres} = \frac{1500}{1000} = 1.5$\
Sarah swam $1.5$ kilometres.

\section*{Question 98}
\textbf{Metadata}

\begin{itemize}
  \item Question ID: P5-RtFndU\_P2-DcCnvD2N\_GPT4.1\_Recreation\_01
  \item Primary KC: RATE | Finding number of unit | finding number of units given rate and total amount
  \item Secondary KC: DECIMALS | Conversion to smaller units | converting a measurement from a larger unit in decimal form to a smaller unit
  \item Topic: Recreation such as sports, games, exercises, music, movie, dancing, painting, fishing and other recreation activities
  \item Grade: Primary 5
\end{itemize}

\textbf{Question}

Siti is filling bottles with water for a school sports event. Each bottle holds $0.75$ litres of water. She filled a total of $4.5$ litres of water. How many bottles did she fill?

\textbf{Solution}

To find the number of bottles Siti filled, we need to know how many times $0.75$ litres fits into $4.5$ litres.

First, convert $4.5$ litres to millilitres:

$1$ litre $= 1000$ millilitres.

$4.5$ litres $= 4.5 \times 1000 = 4500$ millilitres.

Each bottle holds $0.75$ litres $= 0.75 \times 1000 = 750$ millilitres.

Now, divide the total amount of water by the amount one bottle can hold:

$\text{Number of bottles} = \dfrac{4500}{750} = 6$

\textbf{Answer:} Siti filled $6$ bottles.

\section*{Question 99}
\textbf{Metadata}

\begin{itemize}
  \item Question ID: P6-FrDivPN\_P2-FrCmp\_GPT4.1\_Recreation\_01
  \item Primary KC: FRACTIONS | Division | dividing a proper fraction by a whole number
  \item Secondary KC: FRACTIONS | Comparison and ordering | comparing and ordering fractions
  \item Topic: Recreation such as sports, games, exercises, music, movie, dancing, painting, fishing and other recreation activities
  \item Grade: Primary 6
\end{itemize}

\textbf{Question}

Emily is organising a painting workshop for her friends. She has $\frac{3}{4}$ litre of blue paint and wants to share it equally among $5$ painters. After dividing the blue paint, each painter also receives $\frac{1}{10}$ litre of red paint. 

(a) How much blue paint does each painter get? 

(b) Arrange the following quantities in ascending order: the amount of blue paint each painter receives, the amount of red paint each painter receives, and $\frac{1}{5}$ litre.

\textbf{Solution}

(a) Each painter receives $\frac{3}{4} \div 5$ litre of blue paint.

$\frac{3}{4} \div 5 = \frac{3}{4} \times \frac{1}{5} = \frac{3}{20}$.

Each painter gets $\frac{3}{20}$ litre of blue paint.

(b) To compare $\frac{3}{20}$, $\frac{1}{10}$, and $\frac{1}{5}$, express all with a common denominator, $20$:

- $\frac{3}{20}$ (already denominator 20),
- $\frac{1}{10} = \frac{2}{20}$,
- $\frac{1}{5} = \frac{4}{20}$.

Arranging in ascending order:

$\frac{1}{10}$ (=$\frac{2}{20}$), $\frac{3}{20}$, $\frac{1}{5}$ (=$\frac{4}{20}$).

\textbf{Answer:}

(a) $\frac{3}{20}$ litre of blue paint per painter.

(b) Ascending order: $\frac{1}{10}$, $\frac{3}{20}$, $\frac{1}{5}$. 

\section*{Question 100}
\textbf{Metadata}

\begin{itemize}
  \item Question ID: P6-FrDivPN\_P2-FrAdd2nd\_GPT4.1\_Recreation\_01
  \item Primary KC: FRACTIONS | Division | dividing a proper fraction by a whole number
  \item Secondary KC: FRACTIONS | Addition | adding fractions
  \item Topic: Recreation such as sports, games, exercises, music, movie, dancing, painting, fishing and other recreation activities
  \item Grade: Primary 6
\end{itemize}

\textbf{Question}

Sarah and her three friends are sharing some watermelon slices after their music lesson. Sarah has $\frac{3}{4}$ of a watermelon left, and she wants to divide it equally among herself and her three friends. After dividing, each person gets the same amount of watermelon. Later, one more friend joins them, and Sarah decides to add another $\frac{1}{8}$ of a watermelon she found in her lunchbox to what each person already has. 

How much watermelon does each person have now?

\textbf{Solution}

First, find the amount of watermelon each person gets after dividing $\frac{3}{4}$ by 4:

$
\frac{3}{4} \div 4 = \frac{3}{4} \times \frac{1}{4} = \frac{3}{16}
$

So, each person gets $\frac{3}{16}$ of a watermelon.

When the fifth friend joins, Sarah adds $\frac{1}{8}$ of a watermelon to what each person already has:

Amount each person has = $\frac{3}{16} + \frac{1}{8}$

Convert $\frac{1}{8}$ to sixteenths:

$
\frac{1}{8} = \frac{2}{16}
$

Add the fractions:

$
\frac{3}{16} + \frac{2}{16} = \frac{5}{16}
$

Each person now has $\frac{5}{16}$ of a watermelon.

\section*{Question 101}
\textbf{Metadata}

\begin{itemize}
  \item Question ID: P6-FrDivPN\_P2-FrSub2nd\_GPT4.1\_Recreation\_01
  \item Primary KC: FRACTIONS | Division | dividing a proper fraction by a whole number
  \item Secondary KC: FRACTIONS | Subtraction | subtracting fractions
  \item Topic: Recreation such as sports, games, exercises, music, movie, dancing, painting, fishing and other recreation activities
  \item Grade: Primary 6
\end{itemize}

\textbf{Question}

During the weekend, Sarah spent $\frac{3}{4}$ hour painting with her friends. She divided her painting time equally among herself and her 2 friends. Later, Sarah realised she had to leave $\frac{1}{6}$ hour earlier than the others. 

(a) How much painting time did each person get?

(b) After leaving early, how much painting time did Sarah actually have compared to her friends?

\textbf{Solution}

(a) The total painting time is $\frac{3}{4}$ hour, and it was divided equally among 3 people.

So, each person got:

$\frac{3}{4} \div 3 = \frac{3}{4} \times \frac{1}{3} = \frac{3}{12} = \frac{1}{4}$

Each person got $\frac{1}{4}$ hour of painting time.

(b) Sarah left $\frac{1}{6}$ hour earlier than the others.

So, Sarah's actual painting time was:

$\frac{1}{4} - \frac{1}{6}$

First, make the denominators the same:

$\frac{1}{4} = \frac{3}{12}$
$\frac{1}{6} = \frac{2}{12}$

So,

$\frac{3}{12} - \frac{2}{12} = \frac{1}{12}$

Sarah painted for $\frac{1}{12}$ hour less than her friends.

\section*{Question 102}
\textbf{Metadata}

\begin{itemize}
  \item Question ID: P6-FrDivPN\_P3-FrSmp\_GPT4.1\_Recreation\_01
  \item Primary KC: FRACTIONS | Division | dividing a proper fraction by a whole number
  \item Secondary KC: FRACTIONS | Simplifying | expressing a fraction in its simplest form
  \item Topic: Recreation such as sports, games, exercises, music, movie, dancing, painting, fishing and other recreation activities
  \item Grade: Primary 6
\end{itemize}

\textbf{Question}

Olivia and her two friends are sharing $\dfrac{3}{4}$ of a pizza equally among themselves after playing basketball. How much of the pizza does each person get? Give your answer in its simplest form.

\textbf{Solution}

Each person gets $\dfrac{3}{4} \div 3$ of the pizza.\newline

$\dfrac{3}{4} \div 3 = \dfrac{3}{4} \times \dfrac{1}{3} = \dfrac{3 \times 1}{4 \times 3} = \dfrac{3}{12}$\newline

Now, we simplify $\dfrac{3}{12}$ by dividing both numerator and denominator by $3$:\newline

$\dfrac{3 \div 3}{12 \div 3} = \dfrac{1}{4}$\newline

\textbf{Each person gets $\dfrac{1}{4}$ of the pizza.}

\section*{Question 103}
\textbf{Metadata}

\begin{itemize}
  \item Question ID: P6-FrDivPN\_P5-FrCnv2Dc\_GPT4.1\_Recreation\_01
  \item Primary KC: FRACTIONS | Division | dividing a proper fraction by a whole number
  \item Secondary KC: FRACTIONS | Conversion to decimals | expressing fractions as decimals
  \item Topic: Recreation such as sports, games, exercises, music, movie, dancing, painting, fishing and other recreation activities
  \item Grade: Primary 6
\end{itemize}

\textbf{Question}

Sarah participated in a painting competition and used $\dfrac{3}{5}$ of a bottle of blue paint. She decided to share this amount equally among 4 of her classmates for their own art projects. How much blue paint does each classmate get? Express your answer as a fraction and then as a decimal.

\textbf{Solution}

To find out how much blue paint each classmate gets, we divide $\dfrac{3}{5}$ by $4$:

$
\dfrac{3}{5} \div 4 = \dfrac{3}{5} \times \dfrac{1}{4} = \dfrac{3}{20}
$

So, each classmate gets $\dfrac{3}{20}$ of a bottle.

To express $\dfrac{3}{20}$ as a decimal:

$
\dfrac{3}{20} = 0.15
$

**Answer:** Each classmate gets $\dfrac{3}{20}$ of a bottle, which is $0.15$ of a bottle.

\section*{Question 104}
\textbf{Metadata}

\begin{itemize}
  \item Question ID: P6-FrDivPN\_P5-FrMul2nd\_GPT4.1\_Recreation\_01
  \item Primary KC: FRACTIONS | Division | dividing a proper fraction by a whole number
  \item Secondary KC: FRACTIONS | Multiplication | fraction multiplication
  \item Topic: Recreation such as sports, games, exercises, music, movie, dancing, painting, fishing and other recreation activities
  \item Grade: Primary 6
\end{itemize}

\textbf{Question}

Anna has $\dfrac{3}{4}$ of a chocolate bar. She wants to share this chocolate bar equally among 3 friends at a picnic so that each friend receives the same amount. After giving each friend their share, Anna decides to eat only $\dfrac{2}{3}$ of one friend's share. 

(a) How much of the chocolate bar does each friend get? 

(b) How much of the chocolate bar does Anna end up eating?

\textbf{Solution}

(a) Each friend's share is $\dfrac{3}{4} \div 3 = \dfrac{3}{4} \times \dfrac{1}{3} = \dfrac{3}{12} = \dfrac{1}{4}$ of the chocolate bar.

(b) Anna eats $\dfrac{2}{3}$ of one friend's share. So, Anna eats $\dfrac{2}{3} \times \dfrac{1}{4} = \dfrac{2}{12} = \dfrac{1}{6}$ of the chocolate bar.

Final answers:
(a) Each friend gets $\dfrac{1}{4}$ of the chocolate bar.
(b) Anna eats $\dfrac{1}{6}$ of the chocolate bar.

\section*{Question 105}
\textbf{Metadata}

\begin{itemize}
  \item Question ID: P6-FrDivPP\_P2-FrCmp\_GPT4.1\_Recreation\_01
  \item Primary KC: FRACTIONS | Division | dividing a whole number/proper fraction by a proper fraction
  \item Secondary KC: FRACTIONS | Comparison and ordering | comparing and ordering fractions
  \item Topic: Recreation such as sports, games, exercises, music, movie, dancing, painting, fishing and other recreation activities
  \item Grade: Primary 6
\end{itemize}

\textbf{Question}

During a painting activity, Jane and her friends had a canvas that is $1$ unit wide. Jane divided the canvas into pieces, each piece having a width of $\frac{1}{4}$ unit, to share equally among her group.

(a) How many pieces did Jane get when she divided the $1$ unit wide canvas into pieces of $\frac{1}{4}$ unit each?

Later, Emily cut another canvas of $2$ units wide into pieces of $\frac{1}{3}$ unit each.

(b) Who had more pieces: Jane or Emily? Arrange the number of pieces from least to greatest.

\textbf{Solution}

(a) To find how many pieces Jane got, divide the total width by the width of each piece:

$1 \div \frac{1}{4} = 1 \times \frac{4}{1} = 4$

Jane got $4$ pieces.

(b) For Emily:

$2 \div \frac{1}{3} = 2 \times 3 = 6$

Emily got $6$ pieces.

Comparing the number of pieces: Jane had $4$, Emily had $6$.

Arranging from least to greatest: $4, 6$.

Therefore, Emily had more pieces than Jane.

\section*{Question 106}
\textbf{Metadata}

\begin{itemize}
  \item Question ID: P6-FrDivPP\_P2-FrAdd2nd\_GPT4.1\_Recreation\_01
  \item Primary KC: FRACTIONS | Division | dividing a whole number/proper fraction by a proper fraction
  \item Secondary KC: FRACTIONS | Addition | adding fractions
  \item Topic: Recreation such as sports, games, exercises, music, movie, dancing, painting, fishing and other recreation activities
  \item Grade: Primary 6
\end{itemize}

\textbf{Question}

Linda was organising art kits for a school painting activity. She had $6$ bottles of paint. Each painting set requires $\frac{3}{4}$ of a bottle of paint. 

(a) How many complete painting sets could she prepare with $6$ bottles of paint?

Later, her friend gave her another $\frac{1}{2}$ bottle of paint. 

(b) After receiving the extra paint, how many complete sets can Linda prepare in total?

\textbf{Solution}

Part (a):

To find how many complete painting sets Linda can prepare with $6$ bottles of paint, we divide $6$ by $\frac{3}{4}$:

$
6 \div \frac{3}{4} = 6 \times \frac{4}{3} = \frac{24}{3} = 8
$

Linda can prepare $8$ complete painting sets with $6$ bottles of paint.

Part (b):

She receives an additional $\frac{1}{2}$ bottle of paint, so now she has:

$
6 + \frac{1}{2} = \frac{12}{2} + \frac{1}{2} = \frac{13}{2}
$

Now, divide the new total by $\frac{3}{4}$:

$
\frac{13}{2} \div \frac{3}{4} = \frac{13}{2} \times \frac{4}{3} = \frac{52}{6} = \frac{26}{3} \approx 8.67
$

Linda can make $8$ complete painting sets with $\frac{13}{2}$ bottles of paint (since she cannot prepare a fraction of a complete set).

\textbf{Answers:}

(a) $8$ complete sets

(b) $8$ complete sets (with some paint left over)

\section*{Question 107}
\textbf{Metadata}

\begin{itemize}
  \item Question ID: P6-FrDivPP\_P2-FrSub2nd\_GPT4.1\_Recreation\_01
  \item Primary KC: FRACTIONS | Division | dividing a whole number/proper fraction by a proper fraction
  \item Secondary KC: FRACTIONS | Subtraction | subtracting fractions
  \item Topic: Recreation such as sports, games, exercises, music, movie, dancing, painting, fishing and other recreation activities
  \item Grade: Primary 6
\end{itemize}

\textbf{Question}

Lina has a ribbon that is $3$ metres long. She uses $\frac{1}{4}$ metre of ribbon for each bookmark she makes.

(a) How many bookmarks can Lina make from the $3$ metres of ribbon?

Later, Lina gives $\frac{2}{5}$ of the bookmarks she made to her friends.

(b) How many bookmarks does Lina have left after giving some to her friends?

\textbf{Solution}

(a) To find the number of bookmarks Lina can make, divide the total length of ribbon by the length used for each bookmark:
$
\text{Number of bookmarks} = \frac{3}{\frac{1}{4}} = 3 \times 4 = 12
$
Lina can make $12$ bookmarks.

(b) Lina gives $\frac{2}{5}$ of the bookmarks away:
$
\text{Bookmarks given} = \frac{2}{5} \times 12 = \frac{24}{5} = 4.8
$
Since Lina cannot give away part of a bookmark, she gives away $4$ whole bookmarks (assuming bookmarks must be whole), or, if parts are allowed, $4.8$ bookmarks.

Bookmarks Lina has left:
$
12 - 4.8 = 7.2
$
\textbf{Lina has $7.2$ bookmarks left after giving some to her friends.}

(If only whole bookmarks can be given, then she gives $4$ bookmarks and has $8$ left.)

\section*{Question 108}
\textbf{Metadata}

\begin{itemize}
  \item Question ID: P6-FrDivPP\_P3-FrSmp\_GPT4.1\_Recreation\_01
  \item Primary KC: FRACTIONS | Division | dividing a whole number/proper fraction by a proper fraction
  \item Secondary KC: FRACTIONS | Simplifying | expressing a fraction in its simplest form
  \item Topic: Recreation such as sports, games, exercises, music, movie, dancing, painting, fishing and other recreation activities
  \item Grade: Primary 6
\end{itemize}

\textbf{Question}

Tim has $6$ metres of ribbon to use for making medals for a school sports day. Each medal requires $\frac{3}{4}$ metre of ribbon. 

(a) How many medals can Tim make with all the ribbon he has?

(b) Express the answer as a fraction in its simplest form, if he can only make a whole number of medals.

\textbf{Solution}

(a) To find the number of medals Tim can make, divide the total length of ribbon by the ribbon needed for each medal:
$
6 \div \frac{3}{4} = 6 \times \frac{4}{3} = \frac{24}{3} = 8
$
Tim can make $8$ medals.

(b) The answer as a fraction in its simplest form is $\boxed{8}$ medals (which is already a whole number and in its simplest form).

\section*{Question 109}
\textbf{Metadata}

\begin{itemize}
  \item Question ID: P6-FrDivPP\_P5-FrCnv2Dc\_GPT4.1\_Recreation\_01
  \item Primary KC: FRACTIONS | Division | dividing a whole number/proper fraction by a proper fraction
  \item Secondary KC: FRACTIONS | Conversion to decimals | expressing fractions as decimals
  \item Topic: Recreation such as sports, games, exercises, music, movie, dancing, painting, fishing and other recreation activities
  \item Grade: Primary 6
\end{itemize}

\textbf{Question}

Megan is practising for a dance recital. She has practices that last $2$ hours in total each week. Each solo dance routine takes $\frac{3}{4}$ hour to complete. 

(a) How many solo dance routines can Megan perform in $2$ hours?  
(b) Express the fraction you get in part (a) as a decimal, rounded to $2$ decimal places.

\textbf{Solution}

(a) To find how many solo dance routines Megan can perform in $2$ hours, we divide $2$ by $\frac{3}{4}$: 

$
\text{Number of routines} = \frac{2}{\frac{3}{4}}
$
$
= 2 \div \frac{3}{4}
$
$
= 2 \times \frac{4}{3}
$
$
= \frac{8}{3}
$

So, Megan can perform $\frac{8}{3}$ solo dance routines in $2$ hours.

(b) To express $\frac{8}{3}$ as a decimal:

$
\frac{8}{3} = 2.666\ldots
$

Rounded to 2 decimal places, $\frac{8}{3} \approx 2.67$.

\textbf{Final Answers:}  
(a) $\frac{8}{3}$ solo dance routines  
(b) $2.67$ solo dance routines (rounded to 2 decimal places) 

\section*{Question 110}
\textbf{Metadata}

\begin{itemize}
  \item Question ID: P6-FrDivPP\_P5-FrMul2nd\_GPT4.1\_Recreation\_01
  \item Primary KC: FRACTIONS | Division | dividing a whole number/proper fraction by a proper fraction
  \item Secondary KC: FRACTIONS | Multiplication | fraction multiplication
  \item Topic: Recreation such as sports, games, exercises, music, movie, dancing, painting, fishing and other recreation activities
  \item Grade: Primary 6
\end{itemize}

\textbf{Question}

Alvin has $\frac{3}{4}$ hour to practise playing the piano, and each song he wants to play takes $\frac{1}{8}$ hour to practise. After dividing his time equally among as many songs as possible, Alvin decides to spend $\frac{2}{3}$ of the practise time on the first song he learns. 

(a) How many songs can Alvin practise in $\frac{3}{4}$ hour if each requires $\frac{1}{8}$ hour?

(b) How much time, in hours, does he spend on the first song?

\textbf{Solution}

(a) To find how many songs Alvin can practise:

Number of songs $= \frac{3}{4} \div \frac{1}{8} = \frac{3}{4} \times \frac{8}{1} = \frac{3 \times 8}{4 \times 1} = \frac{24}{4} = 6$

So Alvin can practise $6$ songs.

(b) Time spent on the first song $= \frac{2}{3} \times \frac{1}{8} = \frac{2 \times 1}{3 \times 8} = \frac{2}{24} = \frac{1}{12}$ hour.

So Alvin spends $\frac{1}{12}$ hour on the first song.

\section*{Question 111}
\textbf{Metadata}

\begin{itemize}
  \item Question ID: P6-PcFndWN\_P1-WNAdd2nd\_GPT4.1\_Recreation\_01
  \item Primary KC: PERCENTAGE | Finding the whole | finding the whole given a part and the percentage
  \item Secondary KC: WHOLE NUMBERS | Addition | adding whole numbers
  \item Topic: Recreation such as sports, games, exercises, music, movie, dancing, painting, fishing and other recreation activities
  \item Grade: Primary 6
\end{itemize}

\textbf{Question}

At a school sports carnival, $45$ students participated in the basketball game. These students made up $30\%$ of all the students who took part in the carnival. Later, $15$ more students joined for the relay race. What was the total number of students who participated in the carnival after the relay race participants joined?

\textbf{Solution}

Let the total number of students who took part in the carnival originally be $x$.\newline
It is given that $45$ students represent $30\%$ of all the students.\newline
So, $30\% \times x = 45$\newline
$ 0.3x = 45 $\newline
$ x = \frac{45}{0.3} = 150 $\newline
Thus, originally, there were $150$ students at the carnival.\newline
When $15$ more students joined for the relay race, the total number of students became:\newline
$ 150 + 15 = 165 $\newline
\textbf{Therefore, the total number of students who participated in the carnival after the relay race participants joined was $165$.}

\section*{Question 112}
\textbf{Metadata}

\begin{itemize}
  \item Question ID: P6-PcFndWN\_P1-WNSub2nd\_GPT4.1\_Recreation\_01
  \item Primary KC: PERCENTAGE | Finding the whole | finding the whole given a part and the percentage
  \item Secondary KC: WHOLE NUMBERS | Subtraction | subtracting whole numbers
  \item Topic: Recreation such as sports, games, exercises, music, movie, dancing, painting, fishing and other recreation activities
  \item Grade: Primary 6
\end{itemize}

\textbf{Question}

During a school sports event, $20\%$ of the participants played basketball. After $12$ students left the event, the number of participants who played basketball remained the same, but now, the basketball players made up $25\%$ of the remaining participants. How many students were at the event initially?

\textbf{Solution}

Let the initial number of participants be $x$.

$20\%$ of participants played basketball, so the number of basketball players is $0.2x$.

After $12$ students left (none were basketball players), the remaining number of participants is $x - 12$.

Now, the basketball players make up $25\%$ of the remaining group:

$0.2x = 25\% \times (x - 12)$

$0.2x = 0.25(x - 12)$

$0.2x = 0.25x - 3$

$0.25x - 0.2x = 3$

$0.05x = 3$

$x = \frac{3}{0.05} = 60$

\textbf{Therefore, there were initially $60$ students at the sports event.}

\section*{Question 113}
\textbf{Metadata}

\begin{itemize}
  \item Question ID: P6-PcFndWN\_P1-WNMul2nd\_GPT4.1\_Recreation\_01
  \item Primary KC: PERCENTAGE | Finding the whole | finding the whole given a part and the percentage
  \item Secondary KC: WHOLE NUMBERS | Multiplication | multiplying whole numbers
  \item Topic: Recreation such as sports, games, exercises, music, movie, dancing, painting, fishing and other recreation activities
  \item Grade: Primary 6
\end{itemize}

\textbf{Question}

During a school sports day, Sarah found that she had scored 24 points in a basketball challenge, which was 40\% of her total possible points. After the challenge, she played another game where the total possible points were multiplied by 2. 

What was the total possible number of points Sarah could have scored in the basketball challenge? What was the total possible number of points she could have scored after playing both games (expressed as a whole number)?

\textbf{Solution}

Let the total possible number of points in the basketball challenge be $x$.

Sarah scored 24 points, which is 40\% of $x$.

So, $0.4x = 24$

To find $x$, divide both sides by $0.4$:

$
x = \frac{24}{0.4} = 60
$

So, the total possible number of points in the basketball challenge is $60$.

After the challenge, the total possible points were multiplied by $2$ for the next game:

$
\text{Total possible points after two games} = 60 \times 2 = 120
$

\textbf{Answer:}

- The total possible number of points Sarah could have scored in the basketball challenge was $60$.
- The total possible number of points she could have scored after both games was $120$.

\section*{Question 114}
\textbf{Metadata}

\begin{itemize}
  \item Question ID: P6-PcFndWN\_P1-WNDiv2nd\_GPT4.1\_Recreation\_01
  \item Primary KC: PERCENTAGE | Finding the whole | finding the whole given a part and the percentage
  \item Secondary KC: WHOLE NUMBERS | Division | dividing whole numbers
  \item Topic: Recreation such as sports, games, exercises, music, movie, dancing, painting, fishing and other recreation activities
  \item Grade: Primary 6
\end{itemize}

\textbf{Question}

There are 42 students in a dance club. This represents 35\% of the total number of students in their school. After dividing the school into 6 equal-sized groups for a sports event, how many students are there in each group?

\textbf{Solution}

Let the total number of students in the school be $x$.\
Given that 35\% of the school's students is 42,\
\\
$0.35x = 42$\
\\
To find $x$, divide both sides by $0.35$:\
\\
$x = \frac{42}{0.35}$\
\\
$x = 120$\\
So, there are 120 students in the school.\\
If the students are divided into 6 equal groups:\\
$\text{Number of students in each group} = \frac{120}{6} = 20$\\
\textbf{Final answer:} There are 20 students in each group.

\section*{Question 115}
\textbf{Metadata}

\begin{itemize}
  \item Question ID: P6-PcFndChg\_P1-WNAdd2nd\_GPT4.1\_Recreation\_01
  \item Primary KC: PERCENTAGE | Finding change | finding percentage increase/decrease
  \item Secondary KC: WHOLE NUMBERS | Addition | adding whole numbers
  \item Topic: Recreation such as sports, games, exercises, music, movie, dancing, painting, fishing and other recreation activities
  \item Grade: Primary 6
\end{itemize}

\textbf{Question}

In a music club, there were 40 students last month. This month, 15 new students joined the club. 

(a) What is the total number of students in the club this month?

(b) By what percentage did the club’s membership increase from last month to this month? (Round your answer to the nearest whole number.)

\textbf{Solution}

(a) Number of students last month $= 40$
Number of new students $= 15$
Total number of students this month $= 40 + 15 = 55$

(b) The increase in the number of students is $15$.

The percentage increase $= \frac{\text{increase}}{\text{original number}} \times 100\% = \frac{15}{40} \times 100\%$

$= 0.375 \times 100\%$

$= 37.5\%$

To the nearest whole number, the percentage increase is $38\%$.

\section*{Question 116}
\textbf{Metadata}

\begin{itemize}
  \item Question ID: P6-PcFndChg\_P1-WNSub2nd\_GPT4.1\_Recreation\_01
  \item Primary KC: PERCENTAGE | Finding change | finding percentage increase/decrease
  \item Secondary KC: WHOLE NUMBERS | Subtraction | subtracting whole numbers
  \item Topic: Recreation such as sports, games, exercises, music, movie, dancing, painting, fishing and other recreation activities
  \item Grade: Primary 6
\end{itemize}

\textbf{Question}

During the June holidays, Amir played football every week. In the first week, he played for 12 hours. In the second week, he played for 18 hours.

(a) How many more hours did Amir play football in the second week than the first week?

(b) What is the percentage increase in the number of hours he played football from the first week to the second week?

\textbf{Solution}

(a) Number of hours played in the second week: $18$

Number of hours played in the first week: $12$

Difference in hours: $18 - 12 = 6$

Amir played $6$ more hours in the second week than the first week.

(b) Percentage increase $= \frac{\text{Increase}}{\text{Original amount}} \times 100\%$

Increase in hours $= 6$

Original amount (first week) $= 12$

Percentage increase $= \frac{6}{12} \times 100\% = 0.5 \times 100\% = 50\%$

The number of hours Amir played football increased by $50\%$ from the first week to the second week.

\section*{Question 117}
\textbf{Metadata}

\begin{itemize}
  \item Question ID: P6-PcFndChg\_P1-WNMul2nd\_GPT4.1\_Recreation\_01
  \item Primary KC: PERCENTAGE | Finding change | finding percentage increase/decrease
  \item Secondary KC: WHOLE NUMBERS | Multiplication | multiplying whole numbers
  \item Topic: Recreation such as sports, games, exercises, music, movie, dancing, painting, fishing and other recreation activities
  \item Grade: Primary 6
\end{itemize}

\textbf{Question}

At a local sports shop, the price of a tennis racket was \textdollar55 last month. This month, the shop owner increases the price of each tennis racket by 20\%. A school wants to buy 8 tennis rackets this month for their students. 

What is the total amount they need to pay for 8 tennis rackets at the new price?

\textbf{Solution}

First, we find the percentage increase in the price of one tennis racket:

The amount of increase $= 20\% \times 55 = \frac{20}{100} \times 55 = 11$

New price of one racket $= 55 + 11 = \textdollar66$

Total amount for 8 rackets $= 66 \times 8 = \textdollar528$

Therefore, the school needs to pay \textdollar528 for 8 tennis rackets at the new price.

\section*{Question 118}
\textbf{Metadata}

\begin{itemize}
  \item Question ID: P6-PcFndChg\_P1-WNDiv2nd\_GPT4.1\_Recreation\_01
  \item Primary KC: PERCENTAGE | Finding change | finding percentage increase/decrease
  \item Secondary KC: WHOLE NUMBERS | Division | dividing whole numbers
  \item Topic: Recreation such as sports, games, exercises, music, movie, dancing, painting, fishing and other recreation activities
  \item Grade: Primary 6
\end{itemize}

\textbf{Question}

Claire joined a dance class and initially, there were 40 students in the class. At the end of the term, the number of students increased to 56. The students were then divided equally into 7 practice groups.

(a) What was the percentage increase in the number of students in the dance class?

(b) How many students were there in each practice group at the end of the term?

\textbf{Solution}

(a) Initial number of students $=40$\newline Final number of students $=56$\newline Increase in number of students $=56-40=16$

Percentage increase $= \frac{\text{Increase}}{\text{Original number}} \times 100\% = \frac{16}{40} \times 100\% = 40\%$

So, the percentage increase in the number of students is $40\%$.

(b) Number of practice groups $=7$\newline Total students at end of term $=56$

Number of students per group $= \frac{56}{7} = 8$

So, there were $8$ students in each practice group at the end of the term.

\section*{Question 119}
\textbf{Metadata}

\begin{itemize}
  \item Question ID: P6-RoFndDvqWN\_P1-WNAdd2nd\_GPT4.1\_Recreation\_01
  \item Primary KC: RATIO | Finding divided quantities | dividing a given quantity in a given ratio
  \item Secondary KC: WHOLE NUMBERS | Addition | adding whole numbers
  \item Topic: Recreation such as sports, games, exercises, music, movie, dancing, painting, fishing and other recreation activities
  \item Grade: Primary 6
\end{itemize}

\textbf{Question}

A group of friends decided to buy tickets for an outdoor music concert, and the total cost was \textdollar180. They agreed to split the amount in the ratio $2:3$ between Aaron and Bella. Later, Charlie joined them and paid \textdollar40 for snacks for the group. How much did each of Aaron and Bella pay for the tickets? What was the total amount the group spent, including Charlie's contribution?

\textbf{Solution}

First, we divide \textdollar180 in the ratio $2:3$ between Aaron and Bella. 

Find the total number of parts:
$2 + 3 = 5$

Aaron's share:
$\frac{2}{5} \times 180 = 72$

Bella's share:
$\frac{3}{5} \times 180 = 108$

So, Aaron paid \textdollar72 and Bella paid \textdollar108 for the tickets.

To find the total amount spent by the group (tickets + snacks):
$180 + 40 = 220$

The group spent a total of \textdollar220, including Charlie's contribution for snacks.

\section*{Question 120}
\textbf{Metadata}

\begin{itemize}
  \item Question ID: P6-RoFndDvqWN\_P1-WNSub2nd\_GPT4.1\_Recreation\_01
  \item Primary KC: RATIO | Finding divided quantities | dividing a given quantity in a given ratio
  \item Secondary KC: WHOLE NUMBERS | Subtraction | subtracting whole numbers
  \item Topic: Recreation such as sports, games, exercises, music, movie, dancing, painting, fishing and other recreation activities
  \item Grade: Primary 6
\end{itemize}

\textbf{Question}

Siti and her two friends shared the points they earned in a basketball game. They decided to divide the total of $120$ points in the ratio $3:2:1$ among Siti, Mei, and Raj, respectively. After receiving her share, Siti bought a movie ticket for \textdollar20. 

(a) How many points did each person receive?

(b) After buying the movie ticket, how many points did Siti have left?

\textbf{Solution}

(a) First, find the total number of parts in the ratio:

$3 + 2 + 1 = 6$ parts

Each part is worth $\frac{120}{6} = 20$ points.

Siti's share: $3 \times 20 = 60$ points
Mei's share: $2 \times 20 = 40$ points
Raj's share: $1 \times 20 = 20$ points

(b) Siti spent \textdollar20, so she subtracts this from her share:

$60 - 20 = 40$

Siti had $40$ points left after buying the movie ticket.

\section*{Question 121}
\textbf{Metadata}

\begin{itemize}
  \item Question ID: P6-RoFndRoWN\_P1-WNAdd2nd\_GPT4.1\_Recreation\_01
  \item Primary KC: RATIO | Finding ratio | finding the ratio of two or three given whole numbers
  \item Secondary KC: WHOLE NUMBERS | Addition | adding whole numbers
  \item Topic: Recreation such as sports, games, exercises, music, movie, dancing, painting, fishing and other recreation activities
  \item Grade: Primary 6
\end{itemize}

\textbf{Question}

A group of friends attended a music camp. On the first day, 18 students signed up for guitar lessons, and 12 students signed up for piano lessons. Later, another 6 students joined the guitar lessons. 

(a) What is the total number of students who signed up for guitar lessons after the new students joined?

(b) What is the ratio of the number of students in guitar lessons to the number of students in piano lessons after the new students joined? Express your answer in its simplest form.

\textbf{Solution}

(a) The total number of guitar lesson students after new students joined:

$18 + 6 = 24$

So, there are 24 students in guitar lessons.

(b) The number of students in piano lessons is 12. The new ratio is $24:12$.

To simplify, divide both numbers by their greatest common divisor, which is 12:

$24 \div 12 = 2$

$12 \div 12 = 1$

So, the simplest ratio is $2:1$.

\textbf{Answer:}

(a) 24 students signed up for guitar lessons.

(b) The ratio is $2:1$.

\section*{Question 122}
\textbf{Metadata}

\begin{itemize}
  \item Question ID: P6-RoFndRoWN\_P1-WNSub2nd\_GPT4.1\_Recreation\_01
  \item Primary KC: RATIO | Finding ratio | finding the ratio of two or three given whole numbers
  \item Secondary KC: WHOLE NUMBERS | Subtraction | subtracting whole numbers
  \item Topic: Recreation such as sports, games, exercises, music, movie, dancing, painting, fishing and other recreation activities
  \item Grade: Primary 6
\end{itemize}

\textbf{Question}

During the school talent contest, 56 students participated in singing, 38 students participated in dancing, and a certain number of students participated only in both. If there are 20 more students who sang than those who danced, what is the ratio of the number of students who danced to the number of students who sang? Express your answer in its simplest form.

\textbf{Solution}

Let the number of students who danced be $x$. 

We are told that the number of students who sang is 20 more than those who danced, so the number of students who sang is $x + 20$.

But from the problem, the actual numbers are:
Number of students who sang: $56$
Number of students who danced: $38$

Check: $56 - 38 = 18$, but the question says there are $20$ more. Since this does not tally, let's adjust the interpretation.

Actually, the problem can be:

"At the community sports hall, 45 children played badminton, and 33 children played table tennis. How many more children played badminton than table tennis? What is the ratio of the number of children who played badminton to those who played table tennis? Express your ratio in its simplest form."

Solution:
- Number of children who played badminton: $45$
- Number of children who played table tennis: $33$
- More children who played badminton: $45 - 33 = 12$

Ratio of badminton to table tennis:
$$rac{45}{33} = rac{15}{11}$$
So, the simplified ratio is $15:11$.

Final format:

{"word_problem": "At the community sports hall, 45 children played badminton and 33 children played table tennis. How many more children played badminton than table tennis? What is the ratio of the number of children who played badminton to those who played table tennis? Express your ratio in its simplest form.", "solution": "Number of children who played badminton: $45$\\Number of children who played table tennis: $33$\\More children who played badminton: $45 - 33 = 12$\\The ratio of children who played badminton to those who played table tennis is $45 : 33$.\\Simplify the ratio: $45 \div 3 = 15$, $33 \div 3 = 11$, so the simplest form is $15:11$."}

\section*{Question 123}
\textbf{Metadata}

\begin{itemize}
  \item Question ID: P6-RoFndRoWN\_P1-WNMul2nd\_GPT4.1\_Recreation\_01
  \item Primary KC: RATIO | Finding ratio | finding the ratio of two or three given whole numbers
  \item Secondary KC: WHOLE NUMBERS | Multiplication | multiplying whole numbers
  \item Topic: Recreation such as sports, games, exercises, music, movie, dancing, painting, fishing and other recreation activities
  \item Grade: Primary 6
\end{itemize}

\textbf{Question}

At a school sports day, three friends, Amir, Ben, and Cindy, collected stickers for cheering their teams. Amir collected 24 stickers, Ben collected 36 stickers, and Cindy collected 60 stickers. What is the ratio of the number of stickers Amir collected to Ben to Cindy in its simplest form? After that, if Amir decides to give each sticker to 2 of his classmates (so each sticker is given to 2 people), how many classmates will receive a sticker from Amir in total?

\textbf{Solution}

First, we find the simplest ratio of the number of stickers collected by Amir, Ben, and Cindy. 

Amir : Ben : Cindy $=24:36:60$

To simplify, we find the highest common factor (HCF) of 24, 36, and 60.

The HCF is $12$.

$
24 \div 12 = 2 \\ 
36 \div 12 = 3 \\ 
60 \div 12 = 5
$

So, the ratio in simplest form is $2:3:5$.

For the second part, Amir gives each of his 24 stickers to 2 classmates:

Number of classmates $= 24 \times 2 = 48$

Therefore, the number of classmates who will receive a sticker from Amir is $48$. 

\textbf{Answers:}

\begin{itemize}
  \item (a) The simplest ratio is $2:3:5$.
  \item (b) $48$ classmates will receive a sticker from Amir.
\end{itemize}

\section*{Question 124}
\textbf{Metadata}

\begin{itemize}
  \item Question ID: P6-RoFndRoWN\_P1-WNDiv2nd\_GPT4.1\_Recreation\_01
  \item Primary KC: RATIO | Finding ratio | finding the ratio of two or three given whole numbers
  \item Secondary KC: WHOLE NUMBERS | Division | dividing whole numbers
  \item Topic: Recreation such as sports, games, exercises, music, movie, dancing, painting, fishing and other recreation activities
  \item Grade: Primary 6
\end{itemize}

\textbf{Question}

During a school music competition, there are 48 boys and 72 girls taking part. After the competition, the total number of participants is divided equally among 6 groups for a group photo. 

(a) What is the ratio of boys to girls in the competition? 

(b) How many participants are there in each group for the photo?

\textbf{Solution}

(a) The number of boys is $48$ and the number of girls is $72$. 

To find the ratio of boys to girls:

$\text{Ratio of boys to girls} = 48 : 72$

We simplify the ratio by dividing both numbers by their greatest common divisor, which is $24$:

$48 \div 24 = 2$
$72 \div 24 = 3$

So, the simplest ratio is $2 : 3$.

(b) Total number of participants:
$48 + 72 = 120$

The participants are divided into $6$ equal groups, so the number in each group is:
$120 \div 6 = 20$

\textbf{Answer:} 
(a) The ratio of boys to girls is $2:3$.
(b) There are $20$ participants in each group for the photo.

\section*{Question 125}
\textbf{Metadata}

\begin{itemize}
  \item Question ID: P6-RoFndRoWN\_P6-RoSmpWN\_GPT4.1\_Recreation\_01
  \item Primary KC: RATIO | Finding ratio | finding the ratio of two or three given whole numbers
  \item Secondary KC: RATIO | Simplifying | expressing a ratio in its simplest form
  \item Topic: Recreation such as sports, games, exercises, music, movie, dancing, painting, fishing and other recreation activities
  \item Grade: Primary 6
\end{itemize}

\textbf{Question}

At a community sports day, 24 students played football, 18 students played basketball, and 12 students played badminton. What is the ratio of students who played football to basketball to badminton, expressed in its simplest form?

\textbf{Solution}

First, we find the ratio of students who played football : basketball : badminton:

Football : Basketball : Badminton $= 24 : 18 : 12$

To simplify, we find the highest common factor of 24, 18, and 12, which is $6$.

$24 \div 6 = 4$

$18 \div 6 = 3$

$12 \div 6 = 2$

So, the simplest form of the ratio is $4 : 3 : 2$.

\section*{Question 126}
\textbf{Metadata}

\begin{itemize}
  \item Question ID: P6-RoFndTmWN\_P1-WNAdd2nd\_GPT4.1\_Recreation\_01
  \item Primary KC: RATIO | Finding a missing term | finding the missing term in a pair of equivalent ratios
  \item Secondary KC: WHOLE NUMBERS | Addition | adding whole numbers
  \item Topic: Recreation such as sports, games, exercises, music, movie, dancing, painting, fishing and other recreation activities
  \item Grade: Primary 6
\end{itemize}

\textbf{Question}

At a sports event, the ratio of boys to girls in the basketball team is $4:5$. After $6$ more girls joined the team, the new ratio of boys to girls became $4:7$. How many boys are there in the basketball team?

\textbf{Solution}

Let the number of boys be $4x$ and the number of girls be $5x$.

After $6$ more girls joined, the number of girls becomes $5x + 6$. The new ratio is $4:7$.

$
\frac{4x}{5x + 6} = \frac{4}{7}
$

Cross-multiply:
$
4x \times 7 = 4 \times (5x + 6)
$
$
28x = 20x + 24
$
$
28x - 20x = 24
$
$
8x = 24
$
$
x = 3
$

Therefore, the number of boys is:
$
4x = 4 \times 3 = 12
$

\textbf{There are 12 boys in the basketball team.}

\section*{Question 127}
\textbf{Metadata}

\begin{itemize}
  \item Question ID: P6-RoFndTmWN\_P1-WNSub2nd\_GPT4.1\_Recreation\_01
  \item Primary KC: RATIO | Finding a missing term | finding the missing term in a pair of equivalent ratios
  \item Secondary KC: WHOLE NUMBERS | Subtraction | subtracting whole numbers
  \item Topic: Recreation such as sports, games, exercises, music, movie, dancing, painting, fishing and other recreation activities
  \item Grade: Primary 6
\end{itemize}

\textbf{Question}

Matthew and Tom are participating in a chess tournament. The ratio of the number of matches Matthew wins to the number of matches Tom wins is $5:3$. After the tournament, Matthew realised he has won 18 more matches than Tom. If the total number of matches they win together is $72$, how many matches did Matthew win?

\textbf{Solution}

Let the number of matches Matthew wins be $5x$ and the number of matches Tom wins be $3x$. 

The total number of matches they win is given by
$5x + 3x = 8x = 72$
$\Rightarrow x = \frac{72}{8} = 9$

So,
Number of matches Matthew wins $= 5x = 5 \times 9 = 45$
Number of matches Tom wins $= 3x = 3 \times 9 = 27$

The difference in the number of matches they win:
$45 - 27 = 18$

Therefore, Matthew won $\boxed{45}$ matches.

\section*{Question 128}
\textbf{Metadata}

\begin{itemize}
  \item Question ID: P6-AgRepLrEx\_P6-AgSmpLrEx\_GPT4.1\_Recreation\_01
  \item Primary KC: ALGEBRA | Representation and concept | translation of real-world situations into linear algebraic expressions
  \item Secondary KC: ALGEBRA | Simplifying | simplifying linear expressions
  \item Topic: Recreation such as sports, games, exercises, music, movie, dancing, painting, fishing and other recreation activities
  \item Grade: Primary 6
\end{itemize}

\textbf{Question}

Sarah is organising a music concert. She buys some tickets for her friends. Each ticket costs $x$. If she buys 3 more tickets, she gets a discount of $2$ per ticket. 

Write an algebraic expression for the total cost if Sarah buys $n$ tickets without any discount. Then, write and simplify an algebraic expression for the total cost if she buys $(n + 3)$ tickets with the discount applied to each ticket.

\textbf{Solution}

If Sarah buys $n$ tickets at the regular price, the total cost is $nx$.

If Sarah buys $(n + 3)$ tickets with a discount of $2$ per ticket, the cost per ticket becomes $(x - 2)$.

So, the total cost = $(n + 3)(x - 2)$

Let's expand and simplify this expression:

$\begin{align*}
(n + 3)(x - 2) &= n(x - 2) + 3(x - 2) \\
&= nx - 2n + 3x - 6 \\
&= nx + 3x - 2n - 6
\end{align*}$

Therefore, the total cost if Sarah buys $(n + 3)$ tickets with the discount is $nx + 3x - 2n - 6$.

\section*{Question 129}
\textbf{Metadata}

\begin{itemize}
  \item Question ID: P6-AgRepLrEx\_P6-AgEvlLrEx\_GPT4.1\_Recreation\_01
  \item Primary KC: ALGEBRA | Representation and concept | translation of real-world situations into linear algebraic expressions
  \item Secondary KC: ALGEBRA | Evaluation | evaluating linear expressions by substitution
  \item Topic: Recreation such as sports, games, exercises, music, movie, dancing, painting, fishing and other recreation activities
  \item Grade: Primary 6
\end{itemize}

\textbf{Question}

Jia Wei is saving money to buy tickets for a movie with his friends. He decides to save the same amount of money every week. Let $x$ be the amount in \textdollar he saves each week. After 5 weeks, he will have saved a total of $5x$.

If each movie ticket costs \textdollar12 and he needs to buy tickets for himself and 2 friends, write an algebraic expression to represent the total amount of money Jia Wei needs to save.

If Jia Wei saves \textdollar8 each week, will he have enough money after 5 weeks to buy the 3 tickets? Show your working.

\textbf{Solution}

Let $x$ be the amount Jia Wei saves each week. After 5 weeks, he has saved $5x$.

Each ticket costs \textdollar12, and he needs to buy 3 tickets.

Total amount needed $= 3 \times 12 = \textdollar36$.

Algebraic expression for total savings after 5 weeks: $5x$

To check if Jia Wei will have enough after 5 weeks if $x = 8$:

$5 \times 8 = 40$

He will have \textdollar40 after 5 weeks.

Since \textdollar40 $> $ \textdollar36, Jia Wei will have enough money to buy the 3 tickets.

\section*{Question 130}
\textbf{Metadata}

\begin{itemize}
  \item Question ID: P6-AgSlvLrN\_P6-AgRepLrEx\_GPT4.1\_Recreation\_01
  \item Primary KC: ALGEBRA | Solving simple linear equations | solving linear equations involving whole number coefficient and one variable only
  \item Secondary KC: ALGEBRA | Representation and concept | translation of real-world situations into linear algebraic expressions
  \item Topic: Recreation such as sports, games, exercises, music, movie, dancing, painting, fishing and other recreation activities
  \item Grade: Primary 6
\end{itemize}

\textbf{Question}

Jason is buying tickets to a movie for himself and some friends. Each ticket costs $\textdollar12$. Jason spends a total of $\textdollar60$ on tickets. Let $n$ represent the number of tickets Jason buys. Translate this situation into an algebraic equation and solve to find out how many tickets Jason buys.

\textbf{Solution}

Let $n$ be the number of tickets Jason buys. Each ticket costs $\textdollar12$, so the total cost is $12n$. Jason spends $\textdollar60$ on tickets. 

So, we write the equation:
$$
12n = 60
$$
Divide both sides by 12:
$$
n = \frac{60}{12}
$$
$$
n = 5
$$
Therefore, Jason buys $5$ tickets.

\section*{Question 131}
\textbf{Metadata}

\begin{itemize}
  \item Question ID: O1-RoRepFr\_P2-FrAdd2nd\_GPT4.1\_Recreation\_01
  \item Primary KC: RATIO | Representation and concept | ratios involving fractions
  \item Secondary KC: FRACTIONS | Addition | adding fractions
  \item Topic: Recreation such as sports, games, exercises, music, movie, dancing, painting, fishing and other recreation activities
  \item Grade: Secondary O-level 1
\end{itemize}

\textbf{Question}

Evelyn and Mark are members of a dance club. Last month, Evelyn spent $\frac{3}{5}$ of her recreation time dancing and the rest playing badminton. Mark spent $\frac{2}{3}$ of his recreation time dancing and the rest playing badminton. 

The ratio of the fraction of time Evelyn spent dancing to the fraction of time Mark spent dancing is $k:1$. 

(a) Find the value of $k$.

(b) What is the total fraction of their combined recreation time that was spent dancing?

Give your answers in the simplest form.

\textbf{Solution}

(a) The ratio of the fraction of time Evelyn spent dancing to the fraction of time Mark spent dancing is:

$
\text{Ratio} = \frac{\frac{3}{5}}{\frac{2}{3}} = \frac{3}{5} \div \frac{2}{3} = \frac{3}{5} \times \frac{3}{2} = \frac{9}{10}
$

So, $k = \frac{9}{10}$ and the ratio is $\frac{9}{10}:1$.

(b) The total fraction of their combined recreation time spent dancing is:

$
\text{Total fraction} = \frac{3}{5} + \frac{2}{3}
$
Find a common denominator for $5$ and $3$, which is $15$:

$
\frac{3}{5} = \frac{9}{15},\quad \frac{2}{3} = \frac{10}{15}
$
$
\frac{9}{15} + \frac{10}{15} = \frac{19}{15}
$

So, the total fraction is $\frac{19}{15}$.

Final answers:

(a) $k = \frac{9}{10}$

(b) $\frac{19}{15}$ of their combined recreation time was spent dancing.

\section*{Question 132}
\textbf{Metadata}

\begin{itemize}
  \item Question ID: O1-RoRepFr\_P2-FrSub2nd\_GPT4.1\_Recreation\_01
  \item Primary KC: RATIO | Representation and concept | ratios involving fractions
  \item Secondary KC: FRACTIONS | Subtraction | subtracting fractions
  \item Topic: Recreation such as sports, games, exercises, music, movie, dancing, painting, fishing and other recreation activities
  \item Grade: Secondary O-level 1
\end{itemize}

\textbf{Question}

During a school holiday, Mei Lin spent her time doing recreation activities. She spent $\dfrac{2}{5}$ of her holiday playing the piano and $\dfrac{1}{4}$ of her holiday painting. 

(a) What fraction of her holiday did Mei Lin spend on both activities combined?

(b) The ratio of time Mei Lin spent playing the piano to painting is $k : 1$. Find the value of $k$ in its simplest form.

\textbf{Solution}

(a) Fraction of holiday spent playing the piano and painting:

$
\frac{2}{5} + \frac{1}{4} = \frac{8}{20} + \frac{5}{20} = \frac{13}{20}
$

Mei Lin spent $\frac{13}{20}$ of her holiday on both activities combined.

(b) Ratio of time spent playing the piano to painting:

$
\text{Piano : Painting} = \frac{2}{5} : \frac{1}{4}
$

To write as a ratio:
$
\frac{2}{5} \div \frac{1}{4} = \frac{2}{5} \times \frac{4}{1} = \frac{8}{5}
$

So the ratio is $8:5$.

Hence, $k = 8$. The ratio in its simplest form is $8:5$.

\section*{Question 133}
\textbf{Metadata}

\begin{itemize}
  \item Question ID: O1-RoRepFr\_P5-FrMul2nd\_GPT4.1\_Recreation\_01
  \item Primary KC: RATIO | Representation and concept | ratios involving fractions
  \item Secondary KC: FRACTIONS | Multiplication | fraction multiplication
  \item Topic: Recreation such as sports, games, exercises, music, movie, dancing, painting, fishing and other recreation activities
  \item Grade: Secondary O-level 1
\end{itemize}

\textbf{Question}

At a painting workshop, the ratio of the amount of blue paint used to the amount of red paint used is $\frac{2}{3} : 1$. If the artists use $\frac{3}{4}$ litres of red paint, how many litres of blue paint do they use? Express your answer as a fraction in its simplest form.

\textbf{Solution}

Let the amount of blue paint used be $x$ litres. 

Given the ratio of blue paint to red paint is $\frac{2}{3} : 1$. This means $\frac{x}{\frac{3}{4}} = \frac{2}{3}$. 

To find $x$:
$ \frac{x}{\frac{3}{4}} = \frac{2}{3} $
Multiply both sides by $\frac{3}{4}$:
$ x = \frac{2}{3} \times \frac{3}{4} $
$ x = \frac{2 \times 3}{3 \times 4} $
$ x = \frac{6}{12} $
$ x = \frac{1}{2} $

So, the amount of blue paint used is $\frac{1}{2}$ litres.

\section*{Question 134}
\textbf{Metadata}

\begin{itemize}
  \item Question ID: O1-RoRepFr\_P6-FrDiv2nd\_GPT4.1\_Recreation\_01
  \item Primary KC: RATIO | Representation and concept | ratios involving fractions
  \item Secondary KC: FRACTIONS | Division | fraction division
  \item Topic: Recreation such as sports, games, exercises, music, movie, dancing, painting, fishing and other recreation activities
  \item Grade: Secondary O-level 1
\end{itemize}

\textbf{Question}

During a holiday camp, a group of students participated in two recreation activities: painting and dancing. The ratio of the number of students who chose painting to those who chose dancing was $\frac{3}{4} : 2$. If a total of $\frac{5}{6}$ as many students participated in painting as in dancing, and there were $24$ students who chose dancing, how many students participated in painting?

\textbf{Solution}

Let the number of students who chose painting be $p$ and those who chose dancing be $d$. From the problem, the ratio of students who chose painting to dancing is $\frac{3}{4} : 2$. This means:

$ \frac{p}{d} = \frac{\frac{3}{4}}{2} = \frac{3}{4} \times \frac{1}{2} = \frac{3}{8} $

Alternatively, we are also told that there are $\frac{5}{6}$ as many students in painting as in dancing:
$ p = \frac{5}{6}d $
But we are told $d = 24$.

Step 1: Substitute $d = 24$ into $p = \frac{5}{6}d$
$ p = \frac{5}{6} \times 24 = 5 \times 4 = 20 $

Step 2: Check that the given ratio holds with the numbers:
$ \frac{p}{d} = \frac{20}{24} = \frac{5}{6} $
But the stated ratio at the start was $\frac{3}{8}$, so let's check:

Alternatively, since ratio $p$ to $d$ is $\frac{3}{4} : 2 = \frac{3}{4} \div 2 = \frac{3}{8}$.
So the number of students who chose painting is:
$p = \frac{3}{8}d = \frac{3}{8} \times 24 = 9$
But we also have $p = \frac{5}{6}d = 20$.

The two constraints cannot be simultaneously true unless the two ratios are the same. Let's rephrase with the correct logical link:

#### Corrected Problem Statement
During a holiday camp, a group of students participated in two activities: painting and dancing. The ratio of the number of students who chose painting to those who chose dancing was $\frac{3}{4} : 2$. If $24$ students participated in dancing, how many students participated in painting?

#### Solution
Let the number of students who chose painting be $p$.

Given ratio: $\dfrac{p}{24} = \dfrac{3}{4} \div 2 = \dfrac{3}{8}$.
$p = \frac{3}{8} \times 24 = 9$
So, \textbf{9 students participated in painting}.

#### If required to both use ratios involving fractions and fraction division (synergy):
Suppose the number of students who participated in painting is $\frac{3}{4}$ times the number who participated in dancing. If there were $24$ students who participated in dancing, how many participated in painting?

$\text{Number in painting} = \frac{3}{4} \times 24 = 18$

Therefore, $18$ students participated in painting.

\section*{Question 135}
\textbf{Metadata}

\begin{itemize}
  \item Question ID: O1-RoRepFr\_O1-RoSmpFr\_GPT4.1\_Recreation\_01
  \item Primary KC: RATIO | Representation and concept | ratios involving fractions
  \item Secondary KC: RATIO | Simplifying | converting a ratio involving fractions to its simplest form
  \item Topic: Recreation such as sports, games, exercises, music, movie, dancing, painting, fishing and other recreation activities
  \item Grade: Secondary O-level 1
\end{itemize}

\textbf{Question}

At a music camp, the ratio of the time Anna spends practising the piano to the time she spends playing the violin in one week is $\frac{3}{4}:\frac{5}{6}$. 

Express this ratio in its simplest whole number form.

\textbf{Solution}

First, write the ratio: $\frac{3}{4}:\frac{5}{6}$.

To simplify, find a common denominator for the fractions. The least common denominator of $4$ and $6$ is $12$.

Convert each fraction:

$\frac{3}{4} = \frac{3 \times 3}{4 \times 3} = \frac{9}{12}$

$\frac{5}{6} = \frac{5 \times 2}{6 \times 2} = \frac{10}{12}$

So, the ratio becomes $\frac{9}{12}:\frac{10}{12}$.

Remove the denominators by multiplying both parts of the ratio by $12$:

$9:10$

Hence, the simplest whole number ratio is $9:10$.

\section*{Question 136}
\textbf{Metadata}

\begin{itemize}
  \item Question ID: O1-RoRepDc\_P4-DcAdd2nd\_GPT4.1\_Recreation\_01
  \item Primary KC: RATIO | Representation and concept | ratios involving decimals
  \item Secondary KC: DECIMALS | Addition | adding decimals
  \item Topic: Recreation such as sports, games, exercises, music, movie, dancing, painting, fishing and other recreation activities
  \item Grade: Secondary O-level 1
\end{itemize}

\textbf{Question}

At a community sports event, the number of girls to boys in the badminton club is in the ratio $1.5 : 2.5$. If there are $12.5$ boys, how many girls are there? Later, $3.7$ more girls join the club. What is the new total number of girls in the badminton club?

\textbf{Solution}

Let the number of girls be $g$.

The ratio of girls to boys is $1.5 : 2.5$. This means:
$
\frac{g}{12.5} = \frac{1.5}{2.5}
$

Solve for $g$:
$
g = 12.5 \times \frac{1.5}{2.5} = 12.5 \times 0.6 = 7.5
$
So, there are $7.5$ girls initially.

Now, $3.7$ more girls join:
$
\text{New number of girls} = 7.5 + 3.7 = 11.2
$

\textbf{Answer:}
There were initially $7.5$ girls in the badminton club. After $3.7$ more girls join, there are now $11.2$ girls in the club.

\section*{Question 137}
\textbf{Metadata}

\begin{itemize}
  \item Question ID: O1-RoRepDc\_P4-DcSub2nd\_GPT4.1\_Recreation\_01
  \item Primary KC: RATIO | Representation and concept | ratios involving decimals
  \item Secondary KC: DECIMALS | Subtraction | subtracting decimals
  \item Topic: Recreation such as sports, games, exercises, music, movie, dancing, painting, fishing and other recreation activities
  \item Grade: Secondary O-level 1
\end{itemize}

\textbf{Question}

During a school basketball training session, the ratio of the time Mei spent practising free throws to the time she spent practising dribbling was $2.5:1.5$. If Mei spent a total of 2 hours on both activities and spent $0.8$ hours less on dribbling than on free throws, how much time did she spend on each activity?

\textbf{Solution}

Let the time Mei spent practising free throws be $2.5x$ hours and the time spent on dribbling be $1.5x$ hours.

Total time is:

$$
2.5x + 1.5x = 4x
$$

We know that $4x = 2$ hours:

$$
4x = 2
$$
$$
x = 0.5
$$

So, time on free throws:

$$
2.5x = 2.5 \times 0.5 = 1.25 \text{ hours}
$$

Time on dribbling:

$$
1.5x = 1.5 \times 0.5 = 0.75 \text{ hours}
$$

Now, check the difference:

$$
1.25 - 0.75 = 0.5 \text{ hours}
$$

But the question says that the time spent on dribbling was $0.8$ hours less than on free throws, so there seems to be a discrepancy in the numbers. Let's adjust the problem to match the information.

Let $y$ be the time spent on dribbling.

Time on free throws: $y + 0.8$

Given the ratio:

$$
\frac{y + 0.8}{y} = \frac{2.5}{1.5}
$$

Cross-multiply:

$$
1.5(y + 0.8) = 2.5y
$$
$$
1.5y + 1.2 = 2.5y
$$
$$
1.2 = 2.5y - 1.5y = 1y
$$
$$
y = 1.2
$$

So, dribbling: $1.2$ hours

Free throws: $1.2 + 0.8 = 2.0$ hours

Check the total time: $1.2 + 2.0 = 3.2$ hours.

If you want the total to be 2 hours, perhaps the difference should be $0.5$ as in the first calculation. But for the numbers you gave: Ratio $2.5:1.5$, total $2$ hours, difference $0.5$ hours.

So:
Time spent on free throws: $1.25$ hours

Time spent on dribbling: $0.75$ hours

Mei spent $1.25$ hours on free throws and $0.75$ hours on dribbling.

\section*{Question 138}
\textbf{Metadata}

\begin{itemize}
  \item Question ID: O1-RoRepDc\_O1-RoSmpDc\_GPT4.1\_Recreation\_01
  \item Primary KC: RATIO | Representation and concept | ratios involving decimals
  \item Secondary KC: RATIO | Simplifying | converting a ratio involving decimals to its simplest form
  \item Topic: Recreation such as sports, games, exercises, music, movie, dancing, painting, fishing and other recreation activities
  \item Grade: Secondary O-level 1
\end{itemize}

\textbf{Question}

Benjamin and Chloe joined a painting activity and used blue and red paint to mix colors. Benjamin used $3.5$ litres of blue paint and $5.6$ litres of red paint, while Chloe used $2.1$ litres of blue paint and $3.5$ litres of red paint. 

(a) Write the ratio of blue paint to red paint used by Benjamin in the form $a : b$ where $a$ and $b$ are decimals.

(b) Express the ratio in part (a) in its simplest whole number form.

\textbf{Solution}

(a) The ratio of blue paint to red paint Benjamin used is $3.5 : 5.6$.

(b) To simplify $3.5 : 5.6$, divide both numbers by $1.4$ (their greatest common divisor):

$
\frac{3.5}{1.4} = 2.5,\quad \frac{5.6}{1.4} = 4
$

So, the simplified ratio is $2.5 : 4$. We can further multiply both terms by $2$ to clear the decimal:

$
2.5 \times 2 = 5,\quad 4 \times 2 = 8
$

Hence, the simplest whole number form is $5 : 8$.

\section*{Question 139}
\textbf{Metadata}

\begin{itemize}
  \item Question ID: O1-PcRep2q\_O1-PcCnv2Fr\_GPT4.1\_Recreation\_01
  \item Primary KC: PERCENTAGE | Representation and concept | comparing two quantities by percentage
  \item Secondary KC: PERCENTAGE | Conversion to fraction | expressing percentage as a fraction
  \item Topic: Recreation such as sports, games, exercises, music, movie, dancing, painting, fishing and other recreation activities
  \item Grade: Secondary O-level 1
\end{itemize}

\textbf{Question}

During a school talent show, 120 students watched a band performance while 150 students watched a dance performance. 

(a) By what percentage is the number of students watching the dance performance greater than the number watching the band performance? Express your answer correct to 1 decimal place.

(b) Express your answer to (a) as a fraction in its simplest form.

\textbf{Solution}

(a) 
Number of students who watched the band performance = $120$
Number of students who watched the dance performance = $150$

The difference in the number of students is $150 - 120 = 30$

The percentage by which the dance performance viewers is greater than the band performance viewers is:
$
\frac{30}{120} \times 100\% = 25\%
$

So, the number of students watching the dance performance is $25\%$ greater than those watching the band performance.

(b)
Express $25\%$ as a fraction:
$
25\% = \frac{25}{100} = \frac{1}{4}
$

Therefore, $25\%$ as a fraction in simplest form is $\frac{1}{4}$.

\section*{Question 140}
\textbf{Metadata}

\begin{itemize}
  \item Question ID: O1-PcRep2q\_O1-PcCnv2Dc\_GPT4.1\_Recreation\_01
  \item Primary KC: PERCENTAGE | Representation and concept | comparing two quantities by percentage
  \item Secondary KC: PERCENTAGE | Conversion to decimals | expressing percentage as a decimal
  \item Topic: Recreation such as sports, games, exercises, music, movie, dancing, painting, fishing and other recreation activities
  \item Grade: Secondary O-level 1
\end{itemize}

\textbf{Question}

During a weekend badminton tournament, Sarah played 20 matches and won 15 of them. 

(a) What percentage of matches did Sarah win?

(b) Express this percentage as a decimal.

\textbf{Solution}

(a) Sarah won $15$ out of $20$ matches.

The percentage of matches she won is:
$
\frac{15}{20} \times 100\% = 75\%
$

(b) To express this percentage as a decimal, divide $75\%$ by $100$:
$
75\% = 0.75
$

**Answer:**

(a) Sarah won $75\%$ of her matches.

(b) The percentage $75\%$ as a decimal is $0.75$.

\section*{Question 141}
\textbf{Metadata}

\begin{itemize}
  \item Question ID: O1-PcFndRslt\_P1-WNAdd2nd\_GPT4.1\_Recreation\_01
  \item Primary KC: PERCENTAGE | Finding result after change | increasing/decreasing a quantity by a given percentage
  \item Secondary KC: WHOLE NUMBERS | Addition | adding whole numbers
  \item Topic: Recreation such as sports, games, exercises, music, movie, dancing, painting, fishing and other recreation activities
  \item Grade: Secondary O-level 1
\end{itemize}

\textbf{Question}

At a community sports club, there were originally $120$ members. During the start of the new year, the club's membership increased by $15\%$. After the increase, $25$ more people joined the club.

(a) How many members were there after the $15\%$ increase?

(b) What is the total number of members in the club after the extra $25$ people joined?

\textbf{Solution}

(a) To find the number of members after a $15\%$ increase, calculate $15\%$ of $120$ and add it to the original number:

$15\% \times 120 = \frac{15}{100} \times 120 = 18$

Number of members after increase $= 120 + 18 = 138$

(b) After $25$ more people joined:

Total members $= 138 + 25 = 163$

\textbf{Final Answer:} There were $138$ members after the $15\%$ increase, and $163$ members after $25$ more people joined.

\section*{Question 142}
\textbf{Metadata}

\begin{itemize}
  \item Question ID: O1-PcFndRslt\_P1-WNSub2nd\_GPT4.1\_Recreation\_01
  \item Primary KC: PERCENTAGE | Finding result after change | increasing/decreasing a quantity by a given percentage
  \item Secondary KC: WHOLE NUMBERS | Subtraction | subtracting whole numbers
  \item Topic: Recreation such as sports, games, exercises, music, movie, dancing, painting, fishing and other recreation activities
  \item Grade: Secondary O-level 1
\end{itemize}

\textbf{Question}

Ryan joined a music club with 60 members at the start of the year. In April, the number of members increased by 15\%, but later in July, 10 members left the club.

(a) How many members were there in the club after the increase in April?

(b) What was the final number of members remaining after 10 members left in July?

\textbf{Solution}

(a) Number of members after increase in April:

Number increased by $15\%$:
$60 \times \frac{15}{100} = 9$

Total members after increase: $60 + 9 = 69$

(b) Members left in July: $10$

Number of members remaining: $69 - 10 = 59$

\textbf{Final answer:}

(a) $69$ members

(b) $59$ members

\section*{Question 143}
\textbf{Metadata}

\begin{itemize}
  \item Question ID: O1-PcFndRslt\_P1-WNMul2nd\_GPT4.1\_Recreation\_01
  \item Primary KC: PERCENTAGE | Finding result after change | increasing/decreasing a quantity by a given percentage
  \item Secondary KC: WHOLE NUMBERS | Multiplication | multiplying whole numbers
  \item Topic: Recreation such as sports, games, exercises, music, movie, dancing, painting, fishing and other recreation activities
  \item Grade: Secondary O-level 1
\end{itemize}

\textbf{Question}

A music club had $80$ members at the beginning of the year. During the annual recruitment drive, the number of members increased by $25\%$. Each new member pays a subscription fee of \textdollar15.

(a) How many new members joined the club after the recruitment drive?

(b) What is the total amount collected from all the new members as subscription fees?

\textbf{Solution}

(a)
The original number of members = $80$
Percentage increase = $25\%$
Number of new members $= 25\%$ of $80 = \dfrac{25}{100} \times 80 = 20$

So, $20$ new members joined the club.

(b)
Each new member pays \textdollar15.

Total amount collected $= 20 \times 15 = 300$

So, the total amount collected is \textdollar300.

\section*{Question 144}
\textbf{Metadata}

\begin{itemize}
  \item Question ID: O1-PcFndRslt\_P1-WNDiv2nd\_GPT4.1\_Recreation\_01
  \item Primary KC: PERCENTAGE | Finding result after change | increasing/decreasing a quantity by a given percentage
  \item Secondary KC: WHOLE NUMBERS | Division | dividing whole numbers
  \item Topic: Recreation such as sports, games, exercises, music, movie, dancing, painting, fishing and other recreation activities
  \item Grade: Secondary O-level 1
\end{itemize}

\textbf{Question}

A music class had $40$ students enrolled at the start of the year. The instructor decided to increase the size of the class by $25\%$ for the next term to allow more students to join. After the increase, the instructor plans to divide all the students equally into music groups, each containing $5$ students. 

How many music groups can be formed after the increase in the number of students?

\textbf{Solution}

First, let's find the new number of students after the $25\%$ increase. 

$25\%$ of $40$ is: 

$\frac{25}{100} \times 40 = 10$

So, the new class size is: 

$40 + 10 = 50$

Now, we divide the $50$ students equally into groups of $5$:

$50 \div 5 = 10$

\textbf{Answer:} $10$ music groups can be formed after the increase in the number of students.

\section*{Question 145}
\textbf{Metadata}

\begin{itemize}
  \item Question ID: O1-PcRepRvs\_O1-PcCnv2Fr\_GPT4.1\_Recreation\_01
  \item Primary KC: PERCENTAGE | Representation and concept | reverse percentages
  \item Secondary KC: PERCENTAGE | Conversion to fraction | expressing percentage as a fraction
  \item Topic: Recreation such as sports, games, exercises, music, movie, dancing, painting, fishing and other recreation activities
  \item Grade: Secondary O-level 1
\end{itemize}

\textbf{Question}

Last month, Amy scored $80\%$ in her piano exam. This month, after practising more, her score increased by a certain percentage. If her score this month is $96$ marks and this increase represents exactly $20\%$ more than her score last month, what was the maximum mark for the exam? Express $20\%$ as a fraction in your workings.

\textbf{Solution}

Let $x$ be the maximum mark for the piano exam. 

Amy's score last month was $80\%$ of the maximum mark, which is $0.8x$.

This month, her score increased by $20\%$ compared to last month, so:

$ \text{This month's score} = \text{Last month's score} + 20\% \text{ of last month's score} $
$ 96 = 0.8x + 20\% \times 0.8x $

Express $20\%$ as a fraction: \( 20\% = \dfrac{20}{100} = \dfrac{1}{5} \).

Substitute:
$ 96 = 0.8x + \left( \dfrac{1}{5} \times 0.8x \right) $
$ 96 = 0.8x + 0.16x $
$ 96 = 0.96x $
$ x = \dfrac{96}{0.96} $
$ x = 100 $

\textbf{Answer:} The maximum mark for the piano exam was $100$.

\section*{Question 146}
\textbf{Metadata}

\begin{itemize}
  \item Question ID: O1-PcRepRvs\_O1-PcCnv2Dc\_GPT4.1\_Recreation\_01
  \item Primary KC: PERCENTAGE | Representation and concept | reverse percentages
  \item Secondary KC: PERCENTAGE | Conversion to decimals | expressing percentage as a decimal
  \item Topic: Recreation such as sports, games, exercises, music, movie, dancing, painting, fishing and other recreation activities
  \item Grade: Secondary O-level 1
\end{itemize}

\textbf{Question}

Eileen attended a dance workshop. After a 15\% discount, she paid \textdollar102 for the registration fee. Express 15\% as a decimal, and then find the original price of the registration fee before the discount.

\textbf{Solution}

First, express 15\% as a decimal: 

15\% = \frac{15}{100} = 0.15.

Let the original price be $x$.

After a 15\% discount, Eileen paid 85\% of the original price:

$102 = 0.85x$

To find $x$, divide both sides by $0.85$:

$x = \frac{102}{0.85} = 120$

The original price of the registration fee was \textdollar120.

\section*{Question 147}
\textbf{Metadata}

\begin{itemize}
  \item Question ID: O1-AgRepExSq\_O1-AgEvlEx\_GPT4.1\_Recreation\_01
  \item Primary KC: ALGEBRA | Representation and concept | translation of simple real-world situations into quadratic algebraic expressions
  \item Secondary KC: ALGEBRA | Evaluation | evaluation of algebraic expressions and formulae
  \item Topic: Recreation such as sports, games, exercises, music, movie, dancing, painting, fishing and other recreation activities
  \item Grade: Secondary O-level 1
\end{itemize}

\textbf{Question}

Sarah is planning a mini-golf game for her friends. She notices that the cost, in dollars, to book the golf course for $x$ hours is given by the expression $2x^2 + 3x + 5$. If she wants to book the golf course for 4 hours, what will be the total cost in \textdollar?

\textbf{Solution}

Given the cost expression: $2x^2 + 3x + 5$ and $x = 4$. 

Substitute $x = 4$ into the expression:

$2(4)^2 + 3(4) + 5 = 2(16) + 12 + 5 = 32 + 12 + 5 = 49$

Thus, the total cost for 4 hours is \textdollar49.

\section*{Question 148}
\textbf{Metadata}

\begin{itemize}
  \item Question ID: O1-AgRepnth\_O1-AgEvlEx\_GPT4.1\_Recreation\_01
  \item Primary KC: ALGEBRA | Representation and concept | recognising and representing patterns/relationships by finding an algebraic expression for the nth term
  \item Secondary KC: ALGEBRA | Evaluation | evaluation of algebraic expressions and formulae
  \item Topic: Recreation such as sports, games, exercises, music, movie, dancing, painting, fishing and other recreation activities
  \item Grade: Secondary O-level 1
\end{itemize}

\textbf{Question}

A fitness club organises a weekly jump rope challenge. In the first week, Nora completes 12 jumps. Each week after that, she completes 8 more jumps than the previous week.

Let $n$ represent the week number.

(a) Find an algebraic expression in terms of $n$ for the number of jumps Nora completes in the $n$th week.

(b) How many jumps will Nora complete in the 10th week?

\textbf{Solution}

(a) The number of jumps in week 1 is $12$. Each week, the number of jumps increases by $8$.

This is an arithmetic sequence with first term $a_1 = 12$ and common difference $d = 8$.

The general (nth) term is 
$
a_n = 12 + (n-1) \times 8
$

So the algebraic expression is: $a_n = 12 + 8(n-1)$.

(b) For $n = 10$, substitute into the expression:
$
a_{10} = 12 + 8(10-1)
= 12 + 8 \times 9
= 12 + 72
= 84
$

Nora will complete $84$ jumps in the 10th week.

\section*{Question 149}
\textbf{Metadata}

\begin{itemize}
  \item Question ID: O1-AgSlvFrLr\_O1-AgRepEq\_GPT4.1\_Recreation\_01
  \item Primary KC: ALGEBRA | Solving | solving simple fractional equations that can be reduced to linear equations
  \item Secondary KC: ALGEBRA | Representation and concept | translation of simple real-world situations to equations
  \item Topic: Recreation such as sports, games, exercises, music, movie, dancing, painting, fishing and other recreation activities
  \item Grade: Secondary O-level 1
\end{itemize}

\textbf{Question}

Sophie and her friends are playing a number game. Sophie thinks of a number. She tells her friend Rachel that if she adds $3$ to half of her number, the result will be $11$. What number did Sophie think of? Translate the situation into an algebraic equation and solve it.

\textbf{Solution}

Let the number Sophie thinks of be $x$.

According to the problem, adding $3$ to half of her number gives $11$:
$
\frac{1}{2}x + 3 = 11
$

Subtract $3$ from both sides:
$
\frac{1}{2}x = 8
$

Multiply both sides by $2$ to solve for $x$:
$
x = 16
$

Therefore, Sophie thought of the number $16$.

\section*{Question 150}
\textbf{Metadata}

\begin{itemize}
  \item Question ID: O2-RoRepDP\_P1-WNMul2nd\_GPT4.1\_Recreation\_01
  \item Primary KC: RATIO | Representation and concept | direct proportion
  \item Secondary KC: WHOLE NUMBERS | Multiplication | multiplying whole numbers
  \item Topic: Recreation such as sports, games, exercises, music, movie, dancing, painting, fishing and other recreation activities
  \item Grade: Secondary O-level 2
\end{itemize}

\textbf{Question}

A fitness club is organising a group exercise class. The number of exercise mats needed is directly proportional to the number of participants. If 6 mats are required for a class of 8 participants, how many mats would be needed for a class of 30 participants? Then, if each mat costs \textdollar25, calculate the total cost to provide mats for the 30 participants.

\textbf{Solution}

Let the number of mats needed be $m$ and the number of participants be $p$. Since $m$ is directly proportional to $p$, we have $m = k \times p$, where $k$ is the constant of proportionality.\newline From the information, $6 = k \times 8$, so $k = \frac{6}{8} = \frac{3}{4}$.\newline For 30 participants: $m = \frac{3}{4} \times 30 = 22.5$.\newline Since the number of mats must be a whole number, the organiser should prepare 23 mats.\newline The total cost is $23 \times 25 = 575$.\newline \textbf{Answer:} 23 mats are needed, and the total cost is \textdollar575.

\section*{Question 151}
\textbf{Metadata}

\begin{itemize}
  \item Question ID: O2-RoRepDP\_P1-WNDiv2nd\_GPT4.1\_Recreation\_01
  \item Primary KC: RATIO | Representation and concept | direct proportion
  \item Secondary KC: WHOLE NUMBERS | Division | dividing whole numbers
  \item Topic: Recreation such as sports, games, exercises, music, movie, dancing, painting, fishing and other recreation activities
  \item Grade: Secondary O-level 2
\end{itemize}

\textbf{Question}

At a school sports day, the number of students participating in basketball is directly proportional to the number of students participating in volleyball. If the ratio of basketball participants to volleyball participants is $5 : 3$, and there are $48$ volleyball participants, each volleyball team must have exactly $6$ members. 

(a) How many students are participating in basketball?

(b) How many full volleyball teams can be formed?

\textbf{Solution}

(a) Let the number of basketball participants be $B$ and volleyball participants be $V = 48$.

According to the given ratio:
$
\frac{B}{V} = \frac{5}{3}
$
$
B = V \times \frac{5}{3} = 48 \times \frac{5}{3} = 16 \times 5 = 80
$
So, there are $80$ students participating in basketball.

(b) Each volleyball team has $6$ members. To find the number of full teams:
$
\text{Number of full teams} = \frac{48}{6} = 8
$
So, there can be $8$ full volleyball teams.

\section*{Question 152}
\textbf{Metadata}

\begin{itemize}
  \item Question ID: O2-RoRepIvP\_P1-WNMul2nd\_GPT4.1\_Recreation\_01
  \item Primary KC: RATIO | Representation and concept | inverse proportion
  \item Secondary KC: WHOLE NUMBERS | Multiplication | multiplying whole numbers
  \item Topic: Recreation such as sports, games, exercises, music, movie, dancing, painting, fishing and other recreation activities
  \item Grade: Secondary O-level 2
\end{itemize}

\textbf{Question}

A group of friends is preparing a dance performance for their school's annual concert. The teacher observes that when 4 dancers practice together, it takes them 18 hours to complete the entire choreography. The teacher wants to reduce the total practice time by increasing the number of dancers practicing together. 

(a) If the number of dancers is increased to 6, how many hours will they need to complete the choreography, assuming the amount of work is inversely proportional to the number of dancers?

(b) If 2 more dancers join the 6, how many total hours of practice will the group require then?

Give your answers clearly.

\textbf{Solution}

Let $x$ be the number of dancers and $t$ be the number of hours required. Since the number of hours is inversely proportional to the number of dancers, we have 

$ x \times t = k $ for some constant $k$. 

For 4 dancers:
$ 4 \times 18 = k \implies k = 72 $

(a) For 6 dancers:
$ 6 \times t = 72 $
$ t = \frac{72}{6} = 12 $

\textbf{Answer:} If there are 6 dancers, they will need 12 hours to complete the choreography.

(b) If 2 more dancers join, the total number of dancers is $6+2=8$.
$ 8 \times t = 72 $
$ t = \frac{72}{8} = 9 $

\textbf{Answer:} If there are 8 dancers, they will need 9 hours to complete the choreography.

\section*{Question 153}
\textbf{Metadata}

\begin{itemize}
  \item Question ID: O2-RoRepIvP\_P1-WNDiv2nd\_GPT4.1\_Recreation\_01
  \item Primary KC: RATIO | Representation and concept | inverse proportion
  \item Secondary KC: WHOLE NUMBERS | Division | dividing whole numbers
  \item Topic: Recreation such as sports, games, exercises, music, movie, dancing, painting, fishing and other recreation activities
  \item Grade: Secondary O-level 2
\end{itemize}

\textbf{Question}

A group of students plans to play a board game at a recreation centre. The game can be completed in $120$ minutes if there are $4$ players, with each player contributing equally to the gameplay. 

(a) If the number of players is increased to $6$, and the players still contribute equally, how long will it take to complete the same game? 

(b) After the game, the group wants to divide the cost of renting the board game, which is \textdollar36, equally among all the $6$ players. How much does each player need to pay?

\textbf{Solution}

(a) Since the time taken to complete the game is inversely proportional to the number of players, we let $t$ be the time taken and $n$ be the number of players:

$t \propto \frac{1}{n}$

So, $t_1 n_1 = t_2 n_2$

Given $t_1 = 120$ min, $n_1 = 4$, $n_2 = 6$, so:

$120 \times 4 = t_2 \times 6$

$480 = 6 t_2$

$t_2 = \frac{480}{6} = 80$

So, it will take $80$ minutes for $6$ players to complete the game.

(b) The cost of renting the board game is \textdollar36, divided equally among $6$ players:

Amount each player pays $= \frac{\textdollar36}{6} = \textdollar6$

Each player needs to pay \textdollar6.

\section*{Question 154}
\textbf{Metadata}

\begin{itemize}
  \item Question ID: O2-AgSlvIneq\_O2-AgRepIneq\_GPT4.1\_Recreation\_01
  \item Primary KC: ALGEBRA | Solving | solving simple linear inequalities with one variable
  \item Secondary KC: ALGEBRA | Representation and concept | translation of simple real-world situations to simple linear inequalities with one variable
  \item Topic: Recreation such as sports, games, exercises, music, movie, dancing, painting, fishing and other recreation activities
  \item Grade: Secondary O-level 2
\end{itemize}

\textbf{Question}

During the school holidays, Alicia wants to watch movies at the cinema. Each movie ticket costs \textdollar12. Her mother gave her \textdollar50 for movie tickets, and Alicia also wants to save at least \textdollar10 from this amount for snacks. Let $n$ represent the number of movie tickets Alicia can buy. Write down an inequality to represent this situation and find the largest possible value of $n$.

\textbf{Solution}

The total amount Alicia has is \textdollar50. She wants to save at least \textdollar10 for snacks, so the money she can spend on movie tickets is at most \textdollar40.

Let $n$ be the number of movie tickets Alicia can buy. Each ticket costs \textdollar12. The amount she spends on tickets is $12n$.

We set up the inequality:
$$12n \leq 40$$

To find the largest possible value of $n$, solve the inequality:
$$n \leq \frac{40}{12}$$
$$n \leq 3.33$$

Since $n$ must be a whole number, the largest possible value of $n$ is $3$.

So, Alicia can buy at most $3$ movie tickets and still have at least \textdollar10 left for snacks.

\section*{Question 155}
\textbf{Metadata}

\begin{itemize}
  \item Question ID: O2-AgSlvSq1v\_O1-AgRepEq\_GPT4.1\_Recreation\_01
  \item Primary KC: ALGEBRA | Solving | solving quadratic equations in one variable
  \item Secondary KC: ALGEBRA | Representation and concept | translation of simple real-world situations to equations
  \item Topic: Recreation such as sports, games, exercises, music, movie, dancing, painting, fishing and other recreation activities
  \item Grade: Secondary O-level 2
\end{itemize}

\textbf{Question}

During a school sports day, a basketball is thrown upwards from the ground and its height $h$, in metres, after $t$ seconds is given by the equation $h = -5t^2 + 20t$. 

A banner is placed at a height of \textdollar15$ metres. How many seconds after the ball is thrown does it reach the height of \textdollar15$ metres? 

(Express your answers to 2 decimal places.)

\textbf{Solution}

Let $h = 15$, so $15 = -5t^2 + 20t$. Rearranging, we get:

$$
-5t^2 + 20t - 15 = 0
$$
Divide both sides by $-5$:
$$
t^2 - 4t + 3 = 0
$$
This is a quadratic equation. We solve it by factoring:

$$(t - 1)(t - 3) = 0$$

Thus, $t = 1$ or $t = 3$.

So, the ball reaches the height of \textdollar15$ metres after $1$ second and $3$ seconds.

\boxed{1 \text{ s and } 3 \text{ s}}

\section*{Question 156}
\textbf{Metadata}

\begin{itemize}
  \item Question ID: O2-AgSlvLr2v\_O1-AgRepEq\_GPT4.1\_Recreation\_01
  \item Primary KC: ALGEBRA | Solving | solving linear equations in two variables
  \item Secondary KC: ALGEBRA | Representation and concept | translation of simple real-world situations to equations
  \item Topic: Recreation such as sports, games, exercises, music, movie, dancing, painting, fishing and other recreation activities
  \item Grade: Secondary O-level 2
\end{itemize}

\textbf{Question}

Ali and Ben went to a sports centre to play badminton and rent rackets. The total amount they paid for court rental and racket rental was \textdollar28. The court rental costs \textdollar4 per hour, and the racket rental costs \textdollar2 per racket. 

If Ali and Ben rented the court for $x$ hours and rented a total of $y$ rackets, write two equations to represent this situation. If they played for 3 hours and rented a total of 5 rackets, how much did they spend in total?

\textbf{Solution}

Let $x$ be the number of hours the court was rented, and $y$ be the number of rackets rented. 

The total amount spent is:
$ \textdollar 4 \times x + \textdollar 2 \times y = \textdollar 28 $

If $x = 3$ and $y = 5$,
$ \textdollar 4 \times 3 + \textdollar 2 \times 5 = \textdollar 12 + \textdollar 10 = \textdollar 22 $

So, if they played for 3 hours and rented 5 rackets, they spent \textdollar22 in total.

Equation representing the situation: 
$ 4x + 2y = 28 $
Substitute $x=3$, $y=5$:
$ 4(3) + 2(5) = 12 + 10 = 22 $

Hence, they spent \textdollar22.

\section*{Question 157}
\textbf{Metadata}

\begin{itemize}
  \item Question ID: O2-SPFndmdn\_O2-SPFndmode\_GPT4.1\_Recreation\_01
  \item Primary KC: STATISTICS AND PROBABILITY | Finding median | Finding median for a set of data
  \item Secondary KC: STATISTICS AND PROBABILITY | Finding mode | Finding mode for a set of data
  \item Topic: Recreation such as sports, games, exercises, music, movie, dancing, painting, fishing and other recreation activities
  \item Grade: Secondary O-level 2
\end{itemize}

\textbf{Question}

A group of 9 students recorded the number of badminton matches they played during a school holiday. The numbers of matches played by the students are: $3$, $5$, $2$, $6$, $3$, $7$, $5$, $3$, $4$.

(a) What is the median number of matches played?

(b) What is the mode of the number of matches played?

\textbf{Solution}

(a) To find the median, first arrange the data in order:

$2,\ 3,\ 3,\ 3,\ 4,\ 5,\ 5,\ 6,\ 7$

Since there are 9 data points (an odd number), the median is the value at position $\frac{9+1}{2} = 5$.

So, the median number of matches played is $4$.

(b) The mode is the number that occurs most frequently.

In the list $2,\ 3,\ 3,\ 3,\ 4,\ 5,\ 5,\ 6,\ 7$, the number $3$ appears $3$ times, which is more frequent than any other number.

Therefore, the mode of the number of matches played is $3$.

\section*{Question 158}
\textbf{Metadata}

\begin{itemize}
  \item Question ID: O2-SPFndmdn\_O3-SPFndPctl\_GPT4.1\_Recreation\_01
  \item Primary KC: STATISTICS AND PROBABILITY | Finding median | Finding median for a set of data
  \item Secondary KC: STATISTICS AND PROBABILITY | Finding percentiles | finding percentiles for a set of data
  \item Topic: Recreation such as sports, games, exercises, music, movie, dancing, painting, fishing and other recreation activities
  \item Grade: Secondary O-level 2
\end{itemize}

\textbf{Question}

A group of 11 friends recorded the number of basketball games each played during the June holidays. The numbers of games played were:\\

$5,\ 8,\ 10,\ 12,\ 14,\ 16,\ 17,\ 19,\ 20,\ 22,\ 25$\\

(a) Find the median number of games played.\\

(b) Find the 25th percentile (first quartile) and the 75th percentile (third quartile) for the number of games played.

\textbf{Solution}

(a) To find the median, first check that the data is arranged in ascending order. There are 11 numbers (an odd number), so the median is the value at position $\frac{11+1}{2}=6$ in the data set.\\

The 6th value is $16$.\\

\textbf{Median = 16}$.$\\

(b) To find the 25th percentile (first quartile, $Q_1$), use the position formula: $\frac{1}{4}(n+1)$, where $n=11$.\\

So, $Q_1$ is at position $\frac{1}{4}\times(11+1) = \frac{1}{4}\times 12=3$; the 3rd value is $10$.\\

To find the 75th percentile (third quartile, $Q_3$), use the position formula: $\frac{3}{4}(n+1)$.\\

$Q_3$ is at position $\frac{3}{4}\times 12=9$; the 9th value is $20$.\\

\textbf{25th percentile (Q$_1$) = 10}$.$\\
\textbf{75th percentile (Q$_3$) = 20}$.$

\section*{Question 159}
\textbf{Metadata}

\begin{itemize}
  \item Question ID: O2-SPFndmdn\_O3-SPFndrng\_GPT4.1\_Recreation\_01
  \item Primary KC: STATISTICS AND PROBABILITY | Finding median | Finding median for a set of data
  \item Secondary KC: STATISTICS AND PROBABILITY | Finding range | finding range as measures of spread for a set of data 
  \item Topic: Recreation such as sports, games, exercises, music, movie, dancing, painting, fishing and other recreation activities
  \item Grade: Secondary O-level 2
\end{itemize}

\textbf{Question}

During the school holidays, a group of friends tracked the number of hours they spent playing basketball each week for 7 weeks. The hours recorded were: $3$, $7$, $5$, $8$, $4$, $6$, $10$.

(a) Find the median number of hours the friends spent playing basketball per week.

(b) Find the range of the number of hours spent playing basketball per week.

\textbf{Solution}

(a) To find the median, first arrange the data in order:
$3, 4, 5, 6, 7, 8, 10$

There are $7$ data points. The median is the middle number, which is the $4$th value:

$\text{Median} = 6$

(b) To find the range, subtract the smallest value from the largest value:
$\text{Range} = 10 - 3 = 7$

Therefore,

- The median number of hours is $6$.
- The range of the number of hours is $7$.

\section*{Question 160}
\textbf{Metadata}

\begin{itemize}
  \item Question ID: O2-SPFndmean\_O3-BPRepSN\_GPT4.1\_Recreation\_01
  \item Primary KC: STATISTICS AND PROBABILITY | Finding mean deviation | calculation of the mean for a set of data
  \item Secondary KC: BASE AND POWER | Representation and concept  | use of standard form $A\times10^n$ , where n is an integer, and $1<= A<= 10$
  \item Topic: Recreation such as sports, games, exercises, music, movie, dancing, painting, fishing and other recreation activities
  \item Grade: Secondary O-level 2
\end{itemize}

\textbf{Question}

A group of 5 students recorded the number of minutes they spent exercising each day for a week. The recorded times, written in standard form, are $3.2 \times 10^2$, $2.7 \times 10^2$, $4.0 \times 10^2$, $3.5 \times 10^2$, and $2.6 \times 10^2$ minutes. 

(a) Find the mean number of minutes spent exercising per day. Express your answer in standard form.

(b) Find the mean deviation of the data. Give your answer in standard form.

\textbf{Solution}

(a) To find the mean, add all the numbers and divide by 5:

\begin{align*}
(3.2 \times 10^2) + (2.7 \times 10^2) + (4.0 \times 10^2) + (3.5 \times 10^2) + (2.6 \times 10^2) &= (3.2 + 2.7 + 4.0 + 3.5 + 2.6) \times 10^2 \\
&= 16.0 \times 10^2 \\
\end{align*}

So the total is $16.0 \times 10^2$ minutes.

The mean is:

\begin{align*}
\text{Mean} &= \frac{16.0 \times 10^2}{5} \\
&= 3.2 \times 10^2
\end{align*}

So, the mean number of minutes is $3.2 \times 10^2$ minutes.

(b) The mean deviation is the average of the absolute differences from the mean:

\begin{align*}
|3.2 \times 10^2 - 3.2 \times 10^2| &= 0 \\
|2.7 \times 10^2 - 3.2 \times 10^2| &= 0.5 \times 10^2 \\
|4.0 \times 10^2 - 3.2 \times 10^2| &= 0.8 \times 10^2 \\
|3.5 \times 10^2 - 3.2 \times 10^2| &= 0.3 \times 10^2 \\
|2.6 \times 10^2 - 3.2 \times 10^2| &= 0.6 \times 10^2 \\
\end{align*}

Now, sum them:

$0 + 0.5 \times 10^2 + 0.8 \times 10^2 + 0.3 \times 10^2 + 0.6 \times 10^2 = (0 + 0.5 + 0.8 + 0.3 + 0.6) \times 10^2 = 2.2 \times 10^2$

The mean deviation = $\frac{2.2 \times 10^2}{5} = 0.44 \times 10^2 = 4.4 \times 10^1$

**Final Answers:**

(a) $3.2 \times 10^2$ minutes

(b) $4.4 \times 10^1$ minutes

\section*{Question 161}
\textbf{Metadata}

\begin{itemize}
  \item Question ID: O3-BPOpr\_O3-BPRepPosI\_GPT4.1\_Recreation\_01
  \item Primary KC: BASE AND POWER | Operations | laws of indices
  \item Secondary KC: BASE AND POWER | Representation and concept  | positive indices that is not 1
  \item Topic: Recreation such as sports, games, exercises, music, movie, dancing, painting, fishing and other recreation activities
  \item Grade: Secondary O-level 3/4
\end{itemize}

\textbf{Question}

During the school holidays, a group of friends played a music playlist on repeat. The playlist has 3 songs and each friend wants to listen to the playlist through 2 full rounds each day. Consider that the number of times each song is played in one week can be represented by $2^n$, where $n$ is a positive integer greater than 1.

(a) If $n = 3$, how many times is each song played in one week?

(b) Write the total number of times all 3 songs are played in one week as a single term using indices and simplify your answer.

\textbf{Solution}

(a) Since each song is played $2^n$ times in one week and $n=3$:

$
2^3 = 8
$
So, each song is played 8 times in one week.

(b) For all 3 songs, total number of plays is:
$
3 \times 2^3
$
Expressing $3 \times 2^3$ as a single term using indices:
This cannot be further combined using index laws since 3 is not a power of 2. But if each friend listens to 2 full rounds per day and they do this for 7 days, then:

- Number of rounds per week per friend: $2 \times 7 = 14$
- Each round has 3 songs, so songs played per friend: $14 \times 3 = 42$

By the problem definition:
- Number of times each song is played per week: $2^3 = 8$
- For 3 songs: $3 \times 2^3 = 24$

So, the answer is $3 \times 2^3 = 24$ plays in total.

(c) If in the next week, they double the number of times they listen to each song, express the new number of plays for each song using indices and simplify.
- New plays per song: $2 \times 2^3 = 2^1 \times 2^3 = 2^{1+3} = 2^4 = 16$ times for each song in one week.

**Summary of answers:**

(a) $8$

(b) $3 \times 2^3 = 24$

(c) $2^4 = 16$

\section*{Question 162}
\textbf{Metadata}

\begin{itemize}
  \item Question ID: O3-BPOpr\_O3-BPRepNegI\_GPT4.1\_Recreation\_01
  \item Primary KC: BASE AND POWER | Operations | laws of indices
  \item Secondary KC: BASE AND POWER | Representation and concept  | negative indices
  \item Topic: Recreation such as sports, games, exercises, music, movie, dancing, painting, fishing and other recreation activities
  \item Grade: Secondary O-level 3/4
\end{itemize}

\textbf{Question}

A swimming club is organising a tournament where the score of each participant in the final round is calculated using the formula $S = 3^a \times 3^{-b}$, where $a$ is the number of laps completed and $b$ is the number of penalty laps added for a false start. 

A swimmer completes $5$ laps but receives a penalty of $2$ penalty laps. 

(a) Express the swimmer's score $S$ as a single power of $3$.

(b) Calculate the swimmer's final score.

\textbf{Solution}

(a) The score formula is $S = 3^a \times 3^{-b}$. Substitute $a = 5$ and $b = 2$:

$
S = 3^5 \times 3^{-2}
$

Using the law of indices $a^m \times a^n = a^{m+n}$:
$
S = 3^{5 + (-2)} = 3^{3}
$

So, expressed as a single power, the score is $3^3$.

(b) Now, calculate the value of $3^3$:
$
3^3 = 3 \times 3 \times 3 = 27
$

Therefore, the swimmer's final score is \textdollar27.

\section*{Question 163}
\textbf{Metadata}

\begin{itemize}
  \item Question ID: O3-BPOpr\_O3-BPRepFrI\_GPT4.1\_Recreation\_01
  \item Primary KC: BASE AND POWER | Operations | laws of indices
  \item Secondary KC: BASE AND POWER | Representation and concept  | fractional indices
  \item Topic: Recreation such as sports, games, exercises, music, movie, dancing, painting, fishing and other recreation activities
  \item Grade: Secondary O-level 3/4
\end{itemize}

\textbf{Question}

During a dance fitness class, Fiona set up a playlist where the volume in decibels, $V$, increases according to the rule $V = 2^n$, where $n$ represents the $n$-th song in the list. After 3 energetic songs, the instructor wants the volume for the 4th song to be half as loud as the volume of the 2nd song, expressed in terms of the same rule. Express the required volume for the 4th song, $V_4$, using laws of indices and represent your final answer using fractional indices.

\textbf{Solution}

First, the volume for song 2 is $V_2 = 2^2 = 4$. The volume for song 4 (if following the rule directly) would be $V_4 = 2^4 = 16$. However, the instructor wants $V_4$ to be half of $V_2$.\newline
Therefore, $V_4 = \frac{1}{2} \times V_2 = \frac{1}{2} \times 4 = 2$.\newline
Express $2$ in terms of indices of base $2$: $2 = 2^1$.\newline
Alternatively, with fractional indices, $2 = 4^{1/2} = (2^2)^{1/2} = 2^{2 \times 1/2} = 2^1$.\newline
Thus, the required volume for song 4 is $V_4 = (2^2)^{1/2}$, showing the fractional index representation.\newline
\textbf{Final Answer:} $V_4 = (2^2)^{1/2}$ or $2^{2 \times 1/2} = 2^1 = 2$.

\section*{Question 164}
\textbf{Metadata}

\begin{itemize}
  \item Question ID: O3-STOprUn\_O3-STOprIns\_GPT4.1\_Recreation\_01
  \item Primary KC: SET | Set operations | union of two sets
  \item Secondary KC: SET | Set operations | intersection of two sets
  \item Topic: Recreation such as sports, games, exercises, music, movie, dancing, painting, fishing and other recreation activities
  \item Grade: Secondary O-level 3/4
\end{itemize}

\textbf{Question}

In a school, students can choose to join either the Science Club, the Music Club, or both. It was found that 38 students joined the Science Club, 26 students joined the Music Club, and 14 students joined both clubs. 

(a) Represent this information using set notation, letting $S$ be the set of students in the Science Club and $M$ be the set of students in the Music Club.

(b) How many students joined either the Science Club or the Music Club or both?

\textbf{Solution}

(a) Let $S$ represent the set of students in the Science Club and $M$ the set of students in the Music Club. The information can be expressed as:
$ |S| = 38,\quad |M| = 26,\quad |S \cap M| = 14. $

(b) The number of students who joined either the Science Club or the Music Club or both is the number of students in the union of the two sets:
$ |S \cup M| = |S| + |M| - |S \cap M|. $
Substitute the given values:
$ |S \cup M| = 38 + 26 - 14 = 50. $
Therefore, $50$ students joined either the Science Club or the Music Club or both.

\section*{Question 165}
\textbf{Metadata}

\begin{itemize}
  \item Question ID: O3-MXMulSM\_O3-MXAdd\_GPT4.1\_Recreation\_01
  \item Primary KC: MATRICES | Multiplication | product of a scalar quantity and a matrix
  \item Secondary KC: MATRICES | Addition | addition of matrices
  \item Topic: Recreation such as sports, games, exercises, music, movie, dancing, painting, fishing and other recreation activities
  \item Grade: Secondary O-level 3/4
\end{itemize}

\textbf{Question}

A fitness club records the number of different types of exercises done by two groups during a week. The data for Group A and Group B are represented by the following matrices:

$\text{Group A} = \begin{pmatrix} 10 \\ 15 \\ 8 \end{pmatrix}$

$\text{Group B} = \begin{pmatrix} 7 \\ 12 \\ 14 \end{pmatrix}$

where the rows represent the number of times Swimming, Running, and Cycling were each completed.

During a holiday promotion, the club wants to double the sessions for Group A and then find out the total combined number of exercises for each activity from both groups. 

(a) Write an expression to show the new matrix for Group A after doubling all sessions with a scalar multiplication. 

(b) Using matrix addition, find the total number of each exercise done by both groups after the promotion.

\textbf{Solution}

(a) To double every session for Group A, we multiply the matrix by the scalar $2$:

$2 \times \begin{pmatrix} 10 \\ 15 \\ 8 \end{pmatrix} = \begin{pmatrix} 20 \\ 30 \\ 16 \end{pmatrix}$

(b) To find the total for each type of exercise, we add the new Group A matrix to the Group B matrix:

$\begin{pmatrix} 20 \\ 30 \\ 16 \end{pmatrix} + \begin{pmatrix} 7 \\ 12 \\ 14 \end{pmatrix} = \begin{pmatrix} 20+7 \\ 30+12 \\ 16+14 \end{pmatrix} = \begin{pmatrix} 27 \\ 42 \\ 30 \end{pmatrix}$

Final answer: After the promotion, the total number of Swimming, Running, and Cycling sessions done by both groups are $27$, $42$, and $30$, respectively.

\section*{Question 166}
\textbf{Metadata}

\begin{itemize}
  \item Question ID: O3-MXMulSM\_O3-MXSub\_GPT4.1\_Recreation\_01
  \item Primary KC: MATRICES | Multiplication | product of a scalar quantity and a matrix
  \item Secondary KC: MATRICES | Subtraction | subtraction of matrices
  \item Topic: Recreation such as sports, games, exercises, music, movie, dancing, painting, fishing and other recreation activities
  \item Grade: Secondary O-level 3/4
\end{itemize}

\textbf{Question}

A music club keeps track of the number of hours that two teams, Team X and Team Y, spent practising various instruments over a week. The data for each team is represented by a matrix as follows:
$
\text{Team X: }
M = \begin{bmatrix} 2 & 4 \\ 3 & 5 \end{bmatrix}
$
and
$
\text{Team Y: }
N = \begin{bmatrix} 1 & 2 \\ 2 & 3 \end{bmatrix}
$
where the rows represent the instruments (row 1: piano, row 2: violin) and the columns represent two different days (column 1: Monday, column 2: Wednesday).

The school principal decides to double the total practice hours for Team X and subtract the hours that Team Y practised. Find the resulting matrix after doubling Team X's matrix and subtracting Team Y's matrix from it.

\textbf{Solution}

First, calculate the product of a scalar and Team X's matrix:

$
2M = 2 \times \begin{bmatrix} 2 & 4 \\ 3 & 5 \end{bmatrix} = \begin{bmatrix} 4 & 8 \\ 6 & 10 \end{bmatrix}
$

Next, subtract Team Y's matrix:

$
2M - N = \begin{bmatrix} 4 & 8 \\ 6 & 10 \end{bmatrix} - \begin{bmatrix} 1 & 2 \\ 2 & 3 \end{bmatrix}
= \begin{bmatrix} 4 - 1 & 8 - 2 \\ 6 - 2 & 10 - 3 \end{bmatrix}
= \begin{bmatrix} 3 & 6 \\ 4 & 7 \end{bmatrix}
$

Therefore, the resulting matrix is:
$
\boxed{\begin{bmatrix} 3 & 6 \\ 4 & 7 \end{bmatrix}}
$
This matrix represents the final recorded hours for each instrument and day after the calculations.

\section*{Question 167}
\textbf{Metadata}

\begin{itemize}
  \item Question ID: O3-MXMulSM\_O3-MXMul\_GPT4.1\_Recreation\_01
  \item Primary KC: MATRICES | Multiplication | product of a scalar quantity and a matrix
  \item Secondary KC: MATRICES | Multiplication | multiplication of matrices
  \item Topic: Recreation such as sports, games, exercises, music, movie, dancing, painting, fishing and other recreation activities
  \item Grade: Secondary O-level 3/4
\end{itemize}

\textbf{Question}

A table tennis academy tracks the number of matches won by its players in two different tournaments. The results are recorded in the following matrix:
$
A = \begin{bmatrix} 6 & 4 \\ 3 & 7 \end{bmatrix}
$
where the rows represent Player 1 and Player 2, and the columns represent Tournament X and Tournament Y, respectively.

During training, a coach decides to award each player triple the number of points for every win in Tournament X and double the number of points for every win in Tournament Y. This can be expressed as the following diagonal matrix:
$
B = \begin{bmatrix} 3 & 0 \\ 0 & 2 \end{bmatrix}
$

(a) Find the new points matrix awarded to each player by multiplying the matrix $A$ by the matrix $B$.

(b) If, for motivation, the coach decides to further double every player's total points (found in part a), what will be the final matrix of points?

\textbf{Solution}

(a) To find the new points awarded to each player, multiply matrix $A$ by matrix $B$:

$
A \times B = \begin{bmatrix} 6 & 4 \\ 3 & 7 \end{bmatrix} \times \begin{bmatrix} 3 & 0 \\ 0 & 2 \end{bmatrix}
$

Matrix multiplication gives:
$
\begin{bmatrix}
6 \times 3 + 4 \times 0 & 6 \times 0 + 4 \times 2 \\
3 \times 3 + 7 \times 0 & 3 \times 0 + 7 \times 2
\end{bmatrix}
= \begin{bmatrix}
18 & 8 \\
9 & 14
\end{bmatrix}
$

So, the new points matrix is:
$
\begin{bmatrix}
18 & 8 \\
9 & 14
\end{bmatrix}
$

(b) If the coach doubles every player's total points, multiply the entire matrix by scalar $2$:

$
2 \times \begin{bmatrix} 18 & 8 \\ 9 & 14 \end{bmatrix} = \begin{bmatrix} 36 & 16 \\ 18 & 28 \end{bmatrix}
$

Therefore, the final matrix of points is:
$
\boxed{\begin{bmatrix} 36 & 16 \\ 18 & 28 \end{bmatrix}}
$

\section*{Question 168}
\textbf{Metadata}

\begin{itemize}
  \item Question ID: O3-MXSub\_O3-MXAdd\_GPT4.1\_Recreation\_01
  \item Primary KC: MATRICES | Subtraction | subtraction of matrices
  \item Secondary KC: MATRICES | Addition | addition of matrices
  \item Topic: Recreation such as sports, games, exercises, music, movie, dancing, painting, fishing and other recreation activities
  \item Grade: Secondary O-level 3/4
\end{itemize}

\textbf{Question}

During a school sports carnival, three houses—Red, Blue, and Green—recorded the number of medals won in two types of sports: Track and Field, and Swimming. Their medal counts after the first day were given by the matrix
$
A = \begin{bmatrix} 5 & 3 \\ 2 & 4 \\ 6 & 1 \end{bmatrix}
$
where the rows represent Red, Blue, and Green houses respectively, and the columns represent Track and Field medals (first column) and Swimming medals (second column).

On the second day, the houses won additional medals as shown below:
$
B = \begin{bmatrix} 3 & 2 \\ 1 & 2 \\ 2 & 2 \end{bmatrix}
$

Unfortunately, due to a scoring error, the judges realised that the initial results for the first day for Swimming were overstated by:
$
C = \begin{bmatrix} 0 & 1 \\ 0 & 1 \\ 0 & 1 \end{bmatrix}
$

(a) Find the corrected medal count for each house after the first day.

(b) Calculate the total medals for each house after both days, using the corrected first day results.

\textbf{Solution}

(a) To find the corrected medal count for each house after the first day, we subtract matrix $C$ from $A$:
$
A - C = \begin{bmatrix} 5 & 3 \\ 2 & 4 \\ 6 & 1 \end{bmatrix} - \begin{bmatrix} 0 & 1 \\ 0 & 1 \\ 0 & 1 \end{bmatrix} = \begin{bmatrix} 5-0 & 3-1 \\ 2-0 & 4-1 \\ 6-0 & 1-1 \end{bmatrix} = \begin{bmatrix} 5 & 2 \\ 2 & 3 \\ 6 & 0 \end{bmatrix}.
$

(b) To find the total medals after both days, we add matrix $B$ to the result from part (a):
$
\begin{bmatrix} 5 & 2 \\ 2 & 3 \\ 6 & 0 \end{bmatrix} + \begin{bmatrix} 3 & 2 \\ 1 & 2 \\ 2 & 2 \end{bmatrix} = \begin{bmatrix} 5+3 & 2+2 \\ 2+1 & 3+2 \\ 6+2 & 0+2 \end{bmatrix} = \begin{bmatrix} 8 & 4 \\ 3 & 5 \\ 8 & 2 \end{bmatrix}.
$

**Final Answer:**

The corrected medal counts after the first day for the Red, Blue, and Green houses are:
$
\begin{bmatrix} 5 & 2 \\ 2 & 3 \\ 6 & 0 \end{bmatrix}
$

The total medal counts after both days are:
$
\begin{bmatrix} 8 & 4 \\ 3 & 5 \\ 8 & 2 \end{bmatrix}
$
where rows represent the houses (Red, Blue, Green) and columns represent Track and Field and Swimming medals, respectively.

\section*{Question 169}
\textbf{Metadata}

\begin{itemize}
  \item Question ID: O3-MXMul\_O3-MXAdd\_GPT4.1\_Recreation\_01
  \item Primary KC: MATRICES | Multiplication | multiplication of matrices
  \item Secondary KC: MATRICES | Addition | addition of matrices
  \item Topic: Recreation such as sports, games, exercises, music, movie, dancing, painting, fishing and other recreation activities
  \item Grade: Secondary O-level 3/4
\end{itemize}

\textbf{Question}

A sports club has two teams, Team A and Team B, participating in three different types of activities: basketball, badminton, and table tennis. The number of hours each member of Team A spends weekly on each activity is represented by matrix $A$:
$
A = \begin{pmatrix} 2 & 4 & 3 \\ 1 & 3 & 2 \end{pmatrix}
$
where the first row corresponds to Alice, the second row corresponds to Ben, and the columns represent basketball, badminton, and table tennis respectively.

Similarly, the weekly hours for Team B are given by matrix $B$:
$
B = \begin{pmatrix} 3 & 2 & 4 \\ 2 & 1 & 5 \end{pmatrix}
$
where the first row is Cindy, and the second row is Dan. 

(a) The club wants to find out the total number of hours spent on each activity by all four members. Write the combined matrix $C$ by adding $A$ and $B$.

(b) The coaches want to introduce a new training scheme represented by the matrix $M$:
$
M = \begin{pmatrix} 1 \\ 0.5 \\ 0.8 \end{pmatrix}
$
which reflects the impact factor (weightage) assigned to basketball, badminton, and table tennis respectively. Find the product $C \times M$ and explain what the result represents.

\textbf{Solution}

(a) To find the combined matrix $C$, we add $A$ and $B$:
$
C = A + B = \begin{pmatrix} 2 & 4 & 3 \\ 1 & 3 & 2 \end{pmatrix} + \begin{pmatrix} 3 & 2 & 4 \\ 2 & 1 & 5 \end{pmatrix} = \begin{pmatrix} 5 & 6 & 7 \\ 3 & 4 & 7 \end{pmatrix}
$
So, $C = \begin{pmatrix} 5 & 6 & 7 \\ 3 & 4 & 7 \end{pmatrix}$. Each row gives the total weekly hours spent by each member on the three activities.

(b) To find the matrix product $C \times M$:
$
\begin{pmatrix} 5 & 6 & 7 \\ 3 & 4 & 7 \end{pmatrix} \times \begin{pmatrix} 1 \\ 0.5 \\ 0.8 \end{pmatrix} = \begin{pmatrix} 5 \times 1 + 6 \times 0.5 + 7 \times 0.8 \\ 3 \times 1 + 4 \times 0.5 + 7 \times 0.8 \end{pmatrix}
$
Calculating each entry:

First row: $5 \times 1 = 5$; $6 \times 0.5 = 3$; $7 \times 0.8 = 5.6$. Sum: $5 + 3 + 5.6 = 13.6$

Second row: $3 \times 1 = 3$; $4 \times 0.5 = 2$; $7 \times 0.8 = 5.6$. Sum: $3 + 2 + 5.6 = 10.6$

So the result is:
$
C \times M = \begin{pmatrix} 13.6 \\ 10.6 \end{pmatrix}
$
This means Alice and Cindy together have a total weighted activity time of $13.6$ units, and Ben and Dan together have a total weighted activity time of $10.6$ units per week when considering the impact factors of each sport.

\section*{Question 170}
\textbf{Metadata}

\begin{itemize}
  \item Question ID: O3-MXMul\_O3-MXSub\_GPT4.1\_Recreation\_01
  \item Primary KC: MATRICES | Multiplication | multiplication of matrices
  \item Secondary KC: MATRICES | Subtraction | subtraction of matrices
  \item Topic: Recreation such as sports, games, exercises, music, movie, dancing, painting, fishing and other recreation activities
  \item Grade: Secondary O-level 3/4
\end{itemize}

\textbf{Question}

The coach of a basketball team is analyzing the performance of two of his players, Andy and Ben, over three games. He records the number of points scored, assists, and rebounds in each game for both players. The data is arranged in matrices as follows:

For Andy:

$A = \begin{pmatrix}14 & 3 & 7 \\ 16 & 5 & 6 \\ 20 & 4 & 9\end{pmatrix}$

(Columns represent: points, assists, rebounds; rows represent: game 1, 2, and 3)

For Ben:

$B = \begin{pmatrix}10 & 2 & 8 \\ 13 & 3 & 5 \\ 18 & 6 & 10\end{pmatrix}$

(a) Calculate the matrix $C = A - B$ to find the difference in performance for each category in each game between Andy and Ben.

(b) Suppose the coach wants to assign weights to the three categories by the matrix $W = \begin{pmatrix}2 \\ 3 \\ 1\end{pmatrix}$, representing the importance of points, assists, and rebounds respectively. Compute the weighted performance difference for each game using the product $C \times W$.

What is the weighted performance difference for Andy over Ben in game 2?

\textbf{Solution}

(a) First, subtract matrix $B$ from matrix $A$:

$C = A - B = \begin{pmatrix}14-10 & 3-2 & 7-8 \\ 16-13 & 5-3 & 6-5 \\ 20-18 & 4-6 & 9-10\end{pmatrix} = \begin{pmatrix}4 & 1 & -1 \\ 3 & 2 & 1 \\ 2 & -2 & -1\end{pmatrix}$

So the difference in each category per game is:
- Game 1: 4 more points, 1 more assist, 1 less rebound
- Game 2: 3 more points, 2 more assists, 1 more rebound
- Game 3: 2 more points, 2 less assists, 1 less rebound

(b) To find the weighted performance difference for each game, multiply each row of $C$ by the weights column matrix $W$:

The result is a $3 \times 1$ matrix:

For game 1:
$4 \times 2 + 1 \times 3 + (-1) \times 1 = 8 + 3 - 1 = 10$

For game 2:
$3 \times 2 + 2 \times 3 + 1 \times 1 = 6 + 6 + 1 = 13$

For game 3:
$2 \times 2 + (-2) \times 3 + (-1) \times 1 = 4 - 6 - 1 = -3$

So the weighted performance differences for Andy over Ben in each game are:
$\begin{pmatrix}10 \\ 13 \\ -3\end{pmatrix}$

For game 2, the weighted difference is $13$.

\section*{Question 171}
\textbf{Metadata}

\begin{itemize}
  \item Question ID: O3-SPFndQtl\_O3-SPFndIQR\_GPT4.1\_Recreation\_01
  \item Primary KC: STATISTICS AND PROBABILITY | Finding quartiles | finding quartiles for a set of data
  \item Secondary KC: STATISTICS AND PROBABILITY | Finding range | finding interquartile range as measures of spread for a set of data 
  \item Topic: Recreation such as sports, games, exercises, music, movie, dancing, painting, fishing and other recreation activities
  \item Grade: Secondary O-level 3/4
\end{itemize}

\textbf{Question}

A group of 9 students recorded the number of minutes they spent on their daily exercise over a week. The times (in minutes) were: $30, 40, 45, 50, 55, 60, 70, 75, 80$. 

(a) Find the lower quartile ($Q_1$), the median ($Q_2$), and the upper quartile ($Q_3$) of the data set.

(b) Calculate the interquartile range of the exercise times.

\textbf{Solution}

First, arrange the data in ascending order (already done): $30, 40, 45, 50, 55, 60, 70, 75, 80$.

There are $n = 9$ data values.

(a) **Finding Quartiles:**

- The median ($Q_2$) is the middle value, which is the $5^{\text{th}}$ value: $55$.
- To find $Q_1$, consider the lower half: $30, 40, 45, 50$ (first 4 values). $Q_1$ is the median of these 4 values: average of $40$ and $45$. 

$
Q_1 = \frac{40 + 45}{2} = 42.5
$

- To find $Q_3$, take the upper half: $60, 70, 75, 80$. $Q_3$ is the median of these 4 values: average of $70$ and $75$.

$
Q_3 = \frac{70 + 75}{2} = 72.5
$

Quartiles are:
- $Q_1 = 42.5$
- $Q_2 = 55$
- $Q_3 = 72.5$

(b) **Interquartile Range:**

$
\text{Interquartile range} = Q_3 - Q_1 = 72.5 - 42.5 = 30
$

**Answer:**
- The lower quartile is $42.5$ minutes, the median is $55$ minutes, and the upper quartile is $72.5$ minutes.
- The interquartile range is $30$ minutes.

\section*{Question 172}
\textbf{Metadata}

\begin{itemize}
  \item Question ID: O3-SPFndstd\_O2-SPFndmean\_GPT4.1\_Recreation\_01
  \item Primary KC: STATISTICS AND PROBABILITY | Finding standard deviation | calculation of the standard deviation for a set of data
  \item Secondary KC: STATISTICS AND PROBABILITY | Finding mean deviation | calculation of the mean for a set of data
  \item Topic: Recreation such as sports, games, exercises, music, movie, dancing, painting, fishing and other recreation activities
  \item Grade: Secondary O-level 3/4
\end{itemize}

\textbf{Question}

A group of 5 friends tracked the number of hours they spent exercising last week. The recorded hours were: $2$, $5$, $4$, $3$, and $6$ hours. 

(a) Calculate the mean number of hours the friends spent exercising last week. 

(b) Find the standard deviation, correct to 2 decimal places, of the number of hours spent exercising last week.

\textbf{Solution}

(a) To calculate the mean:

Let the data be: $x_1 = 2$, $x_2 = 5$, $x_3 = 4$, $x_4 = 3$, $x_5 = 6$.

Mean $\overline{x} = \dfrac{2 + 5 + 4 + 3 + 6}{5} = \dfrac{20}{5} = 4$ hours.

(b) To calculate the standard deviation:

First, subtract the mean from each data point and square the result:

$(2 - 4)^2 = 4$

$(5 - 4)^2 = 1$

$(4 - 4)^2 = 0$

$(3 - 4)^2 = 1$

$(6 - 4)^2 = 4$

Sum of squared deviations: $4 + 1 + 0 + 1 + 4 = 10$

Variance $= \dfrac{10}{5} = 2$

Standard deviation $= \sqrt{2} \approx 1.41$ (correct to 2 decimal places).

\textbf{Final Answers:}

(a) Mean $= 4$ hours.

(b) Standard deviation $= 1.41$ hours (correct to 2 decimal places).

\section*{Question 173}
\textbf{Metadata}

\begin{itemize}
  \item Question ID: O3-SPAddProb\_O2-SPRepPrSE\_GPT4.1\_Recreation\_01
  \item Primary KC: STATISTICS AND PROBABILITY | Addition | addition of probabilities
  \item Secondary KC: STATISTICS AND PROBABILITY | Representation and concept | probability of single events
  \item Topic: Recreation such as sports, games, exercises, music, movie, dancing, painting, fishing and other recreation activities
  \item Grade: Secondary O-level 3/4
\end{itemize}

\textbf{Question}

At a school’s monthly recreation day, students can choose to participate in three main activities: badminton, painting, or movie screening. The probability that a randomly chosen student attends the badminton session is $0.3$, and the probability that a student attends the painting session is $0.2$. 

Assume no student attends more than one activity. 

(a) What is the probability that a randomly chosen student attends either the badminton or the painting session? 

(b) If the probability that a student attends the movie screening is $0.4$, what is the probability that a student does not attend any of these three activities?

\textbf{Solution}

(a) The probability that a student attends either the badminton or the painting session is the sum of the individual probabilities, since no student attends more than one activity: 

$P(\text{badminton or painting}) = P(\text{badminton}) + P(\text{painting})$

$= 0.3 + 0.2 = 0.5$

(b) The probability that a student attends any of the three activities is: 

$P(\text{badminton or painting or movie}) = P(\text{badminton}) + P(\text{painting}) + P(\text{movie})$

$= 0.3 + 0.2 + 0.4 = 0.9$

Therefore, the probability that a student does not attend any of these activities is:

$P(\text{none}) = 1 - P(\text{badminton or painting or movie})$

$= 1 - 0.9 = 0.1$

\section*{Question 174}
\textbf{Metadata}

\begin{itemize}
  \item Question ID: O3-SPAddProb\_O3-SPFndPrCE\_GPT4.1\_Recreation\_01
  \item Primary KC: STATISTICS AND PROBABILITY | Addition | addition of probabilities
  \item Secondary KC: STATISTICS AND PROBABILITY | Finding probability | probability of simple combined events
  \item Topic: Recreation such as sports, games, exercises, music, movie, dancing, painting, fishing and other recreation activities
  \item Grade: Secondary O-level 3/4
\end{itemize}

\textbf{Question}

In a music club, there are 30 students. During a weekly activity, each student can choose to play either the piano or the guitar, or both. Out of the 30 students, 18 choose to play the piano, 15 choose to play the guitar, and 6 choose to play both instruments. If one student is chosen at random, what is the probability that the student plays at least one of the two instruments (the piano or the guitar)?

\textbf{Solution}

Let $A$ be the event that a student plays the piano, and $B$ be the event that a student plays the guitar. We are given:

\begin{align*}
P(A) &= \frac{18}{30} = 0.6 \\
P(B) &= \frac{15}{30} = 0.5 \\
P(A \cap B) &= \frac{6}{30} = 0.2
\end{align*}

We want to find $P(A \cup B)$, the probability that a student plays at least one instrument.

The addition rule for probabilities gives:

$
P(A \cup B) = P(A) + P(B) - P(A \cap B)
$

Substitute the values:
$
P(A \cup B) = 0.6 + 0.5 - 0.2 = 0.9
$

Therefore, the probability that a randomly chosen student plays at least one instrument is $0.9$ or $90\%$.

\section*{Question 175}
\textbf{Metadata}

\begin{itemize}
  \item Question ID: O3-SPMulProb\_O2-SPRepPrSE\_GPT4.1\_Recreation\_01
  \item Primary KC: STATISTICS AND PROBABILITY | Multiplication | multiplication of probabilities
  \item Secondary KC: STATISTICS AND PROBABILITY | Representation and concept | probability of single events
  \item Topic: Recreation such as sports, games, exercises, music, movie, dancing, painting, fishing and other recreation activities
  \item Grade: Secondary O-level 3/4
\end{itemize}

\textbf{Question}

During the school sports day, Sarah and Amir are taking part in two independent games: basketball free throw and table tennis match. 

The probability that Sarah scores a basket during a free throw is $0.8$. The probability that Amir wins his first table tennis match is $0.5$.

(a) Find the probability that Sarah scores a basket and Amir wins his first match.

(b) What is the probability that only one of them succeeds in their event?

\textbf{Solution}

(a) Since the two events are independent, the probability that both Sarah scores a basket and Amir wins his match is given by the multiplication of their individual probabilities:

$
P(\text{Sarah scores and Amir wins}) = P(\text{Sarah scores}) \times P(\text{Amir wins}) = 0.8 \times 0.5 = 0.4
$

So, the probability is $0.4$.

(b) The probability that only one of them succeeds means:

- Sarah scores and Amir does not win, OR
- Sarah does not score and Amir wins.

First, calculate the probability for each scenario:

Sarah scores and Amir does not win:
$
P(\text{Sarah scores}) \times P(\text{Amir does not win}) = 0.8 \times (1-0.5) = 0.8 \times 0.5 = 0.4
$

Sarah does not score and Amir wins:
$
P(\text{Sarah does not score}) \times P(\text{Amir wins}) = (1-0.8) \times 0.5 = 0.2 \times 0.5 = 0.1
$

Adding these probabilities together:
$
P(\text{Only one succeeds}) = 0.4 + 0.1 = 0.5
$

So, the probability that only one of them succeeds in their event is $0.5$.

\section*{Question 176}
\textbf{Metadata}

\begin{itemize}
  \item Question ID: O3-SPMulProb\_O3-SPFndPrCE\_GPT4.1\_Recreation\_01
  \item Primary KC: STATISTICS AND PROBABILITY | Multiplication | multiplication of probabilities
  \item Secondary KC: STATISTICS AND PROBABILITY | Finding probability | probability of simple combined events
  \item Topic: Recreation such as sports, games, exercises, music, movie, dancing, painting, fishing and other recreation activities
  \item Grade: Secondary O-level 3/4
\end{itemize}

\textbf{Question}

Adrian is participating in a recreational board game competition. In the game, he draws one card from a deck that has 5 red cards and 7 blue cards. Immediately after, he rolls a fair six-sided dice. 

What is the probability that Adrian draws a red card and then rolls an even number on the dice?

\textbf{Solution}

First, calculate the probability that Adrian draws a red card.

- Total number of cards $= 5 + 7 = 12$
- Number of red cards $= 5$
- Probability of drawing a red card $= \frac{5}{12}$

Next, calculate the probability of rolling an even number on a fair six-sided dice.

- There are 3 even numbers on a dice: 2, 4, 6.
- Probability of rolling an even number $= \frac{3}{6} = \frac{1}{2}$

Since the card draw and dice roll are independent events, the probability that both events occur together is the product of their probabilities:

$$
\text{Probability} = \frac{5}{12} \times \frac{1}{2} = \frac{5}{24}
$$

Therefore, the probability that Adrian draws a red card and then rolls an even number is $\boxed{\frac{5}{24}}$.

\end{document}
