\documentclass{article}
\usepackage[utf8]{inputenc}
\usepackage{amsmath}
\usepackage{amsfonts}
\usepackage{amssymb}
\usepackage{graphicx}
\usepackage{hyperref}
\title{20 Solutions Services MWPs by GPT4.1 v7 v1}
\author{Tien Dung Doan}
\begin{document}
\maketitle
\section*{Question 1}
\textbf{Metadata}

\begin{itemize}
  \item Question ID: O3-SPFndQtl\_O3-SPFndIQR\_GPT4.1\_Services\_03
  \item Primary KC: STATISTICS AND PROBABILITY | Finding quartiles | finding quartiles for a set of data
  \item Secondary KC: STATISTICS AND PROBABILITY | Finding range | finding interquartile range as measures of spread for a set of data 
  \item Topic: Services such as installation, maintenance, repairing, cleaning, laundry, hotel, retail, e-commerce, streaming services, digital services etc.
  \item Grade: Secondary O-level 3/4
\end{itemize}

\textbf{Solution}

(a) Arrange the data in order (already arranged):\\
$110, 120, 150, 155, 160, 170, 180, 185, 200$\\

Number of data points $n = 9$\\
\\
Median ($Q_2$) is the $\frac{n+1}{2} = \frac{10}{2} = 5$th value:\\
$Q_2 = 160$\\
\\
Lower half (excluding median): $110, 120, 150, 155$\\
Upper half (excluding median): $170, 180, 185, 200$\\

Lower quartile ($Q_1$): median of $110, 120, 150, 155$\\
There are 4 values, so $Q_1$ is the average of the 2nd and 3rd values.\\
$Q_1 = \frac{120 + 150}{2} = 135$\\

Upper quartile ($Q_3$): median of $170, 180, 185, 200$\\
$Q_3 = \frac{180 + 185}{2} = 182.5$\\

(b) Interquartile range (IQR)\\
$IQR = Q_3 - Q_1 = 182.5 - 135 = 47.5$\\

Therefore,\\
$Q_1 = \textdollar 135$\\
$Q_2 = \textdollar 160$\\
$Q_3 = \textdollar 182.5$\\
$IQR = \textdollar 47.5$.

\section*{Question 2}
\textbf{Metadata}

\begin{itemize}
  \item Question ID: P5-FrMulImN\_P5-FrCnv2Dc\_GPT4.1\_Services\_03
  \item Primary KC: FRACTIONS | Multiplication | multiplying a proper/improper fraction and a whole number
  \item Secondary KC: FRACTIONS | Conversion to decimals | expressing fractions as decimals
  \item Topic: Services such as installation, maintenance, repairing, cleaning, laundry, hotel, retail, e-commerce, streaming services, digital services etc.
  \item Grade: Primary 5
\end{itemize}

\textbf{Solution}

(a) Time taken to clean one carpet $= \frac{3}{5}$ hour.

Time taken to clean 8 carpets $= 8 \times \frac{3}{5} = \frac{24}{5}$ hours.

(b) To express $\frac{24}{5}$ as a decimal:

$\frac{24}{5} = 4.8$

So, the total time taken is $4.8$ hours.

\section*{Question 3}
\textbf{Metadata}

\begin{itemize}
  \item Question ID: P4-FrAddU12\_P3-FrSmp\_GPT4.1\_Services\_04
  \item Primary KC: FRACTIONS | Addition | adding unlike fractions with two different denominators not exceeding 12
  \item Secondary KC: FRACTIONS | Simplifying | expressing a fraction in its simplest form
  \item Topic: Services such as installation, maintenance, repairing, cleaning, laundry, hotel, retail, e-commerce, streaming services, digital services etc.
  \item Grade: Primary 4
\end{itemize}

\textbf{Solution}

To find the total fraction of the day Ginny spent washing and ironing, we need to add $\frac{3}{8}$ and $\frac{5}{12}$. 

First, find the lowest common multiple (LCM) of 8 and 12, which is 24.

\[
\frac{3}{8} = \frac{3 \times 3}{8 \times 3} = \frac{9}{24} \\
\frac{5}{12} = \frac{5 \times 2}{12 \times 2} = \frac{10}{24}
\]

Now add the fractions:
\[
\frac{9}{24} + \frac{10}{24} = \frac{19}{24}
\]

The fraction $\frac{19}{24}$ cannot be simplified further as 19 and 24 have no common factors other than 1.

**Answer:** Ginny spent $\frac{19}{24}$ of her workday washing and ironing, expressed in its simplest form.

\section*{Question 4}
\textbf{Metadata}

\begin{itemize}
  \item Question ID: P4-FrRepMixIm\_P3-FrCnvEq\_GPT4.1\_Services\_04
  \item Primary KC: FRACTIONS | Representation and concept | mixed numbers and improper fractions
  \item Secondary KC: FRACTIONS | Conversion to equivalent fractions | Conversion to equivalent fractions (given either the denominator or the numerator)
  \item Topic: Services such as installation, maintenance, repairing, cleaning, laundry, hotel, retail, e-commerce, streaming services, digital services etc.
  \item Grade: Primary 4
\end{itemize}

\textbf{Solution}

First, change the mixed number $1\dfrac{1}{2}$ to an improper fraction: 

$1\dfrac{1}{2} = \dfrac{2 \times 1 + 1}{2} = \dfrac{3}{2}$.

Next, we need to convert $\dfrac{3}{2}$ to an equivalent fraction with denominator 4:

$\dfrac{3}{2} = \dfrac{3 \times 2}{2 \times 2} = \dfrac{6}{4}$.

Mr. Tan cleaned $\dfrac{7}{4}$ hours on Monday and $\dfrac{6}{4}$ hours on Tuesday.

Total hours spent cleaning:

$\dfrac{7}{4} + \dfrac{6}{4} = \dfrac{13}{4}$.

He spent a total of $\dfrac{13}{4}$ hours cleaning rooms over both days, reported as an improper fraction with denominator $4$. 

\textbf{Final Answer:} $\dfrac{13}{4}$ hours.

\section*{Question 5}
\textbf{Metadata}

\begin{itemize}
  \item Question ID: O1-FDDiv\_O1-FDAdd\_GPT4.1\_Services\_04
  \item Primary KC: FRACTIONS AND DECIMALS | Division | Division involving one fraction and one decimal number
  \item Secondary KC: FRACTIONS AND DECIMALS | Addition | Addition involving one fraction and one decimal number
  \item Topic: Services such as installation, maintenance, repairing, cleaning, laundry, hotel, retail, e-commerce, streaming services, digital services etc.
  \item Grade: Secondary O-level 1
\end{itemize}

\textbf{Solution}

(a) The number of hotel rooms that can be fully cleaned is calculated by dividing the total amount of disinfectant by the amount needed per room:\\
\\
$\displaystyle \text{Number of rooms} = \dfrac{\dfrac{7}{2}}{0.75}$\\
First, express $0.75$ as a fraction: $0.75 = \dfrac{3}{4}$\\
So,\\
$\displaystyle \dfrac{7}{2} \div \dfrac{3}{4} = \dfrac{7}{2} \times \dfrac{4}{3} = \dfrac{7 \times 4}{2 \times 3} = \dfrac{28}{6} = \dfrac{14}{3} \approx 4.666...$\\
Therefore, the company can fully clean $4$ hotel rooms (since partial rooms cannot be fully cleaned).\\
\\
(b) Amount of disinfectant left after cleaning $4$ rooms:\\
$\displaystyle \text{Used disinfectant} = 4 \times 0.75 = 3.00$ litres\\
$\displaystyle \text{Disinfectant remaining} = \dfrac{7}{2} - 3.00 = 3.5 - 3.0 = 0.5$ litres\\
After ordering an additional $1.2$ litres:\\
$\displaystyle \text{New total amount} = 0.5 + 1.2 = 1.7$ litres\\
\\
\boxed{\text{The new total amount of disinfectant the company will have is } 1.7 \text{ litres.}}

\section*{Question 6}
\textbf{Metadata}

\begin{itemize}
  \item Question ID: P3-WNDivRmd3d\_P1-WNAdd2nd\_GPT4.1\_Services\_04
  \item Primary KC: WHOLE NUMBERS | Division | dividing whole numbers up to 3 digits by 1 digit with remainder 
  \item Secondary KC: WHOLE NUMBERS | Addition | adding whole numbers
  \item Topic: Services such as installation, maintenance, repairing, cleaning, laundry, hotel, retail, e-commerce, streaming services, digital services etc.
  \item Grade: Primary 3
\end{itemize}

\textbf{Solution}

(a) To find how many towels can go in each box, divide $275$ by $4$:

\[ 275 \div 4 = 68 \text{ R } 3 \]

So, each box can contain $68$ towels.

(b) The remainder is $3$, so $3$ towels will be left unpacked.

After receiving $19$ more towels:

\[ 3 \text{ (leftover towels) } + 19 = 22 \]

So, the company will have $22$ towels in total after adding the ones left unpacked and the new towels received.

\section*{Question 7}
\textbf{Metadata}

\begin{itemize}
  \item Question ID: P4-DcSub2d\_P4-DcRnd3d\_GPT4.1\_Services\_04
  \item Primary KC: DECIMALS | Subtraction | subtracting decimals (up to 2 decimal places)
  \item Secondary KC: DECIMALS | Rounding | rounding decimals up to 3 decimal places to the nearest whole number, 1 decimal place and 2 decimal places 
  \item Topic: Services such as installation, maintenance, repairing, cleaning, laundry, hotel, retail, e-commerce, streaming services, digital services etc.
  \item Grade: Primary 4
\end{itemize}

\textbf{Solution}

(a) Amount Alice paid extra:

$42.80 - 37.459 = 5.341$

So, Alice paid \textdollar5.341 more than the promotional price.

(b) Rounding the promotional price $37.459$:
- To the nearest whole number: $37$
- To 1 decimal place: $37.5$
- To 2 decimal places: $37.46$

Final answers:
(a) \textdollar5.341
(b) Nearest dollar: \textdollar37; nearest 1 decimal place: \textdollar37.5; nearest 2 decimal places: \textdollar37.46

\section*{Question 8}
\textbf{Metadata}

\begin{itemize}
  \item Question ID: P5-FrMulPIm\_P5-FrCnv2Dc\_GPT4.1\_Services\_03
  \item Primary KC: FRACTIONS | Multiplication | multiplying a proper fraction and a proper/improper fractions
  \item Secondary KC: FRACTIONS | Conversion to decimals | expressing fractions as decimals
  \item Topic: Services such as installation, maintenance, repairing, cleaning, laundry, hotel, retail, e-commerce, streaming services, digital services etc.
  \item Grade: Primary 5
\end{itemize}

\textbf{Solution}

Jasmine sends $\frac{3}{4}$ kg for each load. During the sale week, the laundry cleans $\frac{5}{2}$ times as much:

$\frac{3}{4} \times \frac{5}{2} = \frac{3 \times 5}{4 \times 2} = \frac{15}{8}$ kg.

Now, to convert $\frac{15}{8}$ to a decimal, divide $15$ by $8$:

$\frac{15}{8} = 1.875$

Therefore, the laundry cleans $1.875$ kg of Jasmine's clothes in total during the sale week.

\section*{Question 9}
\textbf{Metadata}

\begin{itemize}
  \item Question ID: P6-PcFndChg\_P1-WNDiv2nd\_GPT4.1\_Services\_03
  \item Primary KC: PERCENTAGE | Finding change | finding percentage increase/decrease
  \item Secondary KC: WHOLE NUMBERS | Division | dividing whole numbers
  \item Topic: Services such as installation, maintenance, repairing, cleaning, laundry, hotel, retail, e-commerce, streaming services, digital services etc.
  \item Grade: Primary 6
\end{itemize}

\textbf{Solution}

First, we find the increase in the number of subscribers:

\[
\text{Increase} = 1,170 - 900 = 270
\]

Divide this increase equally among 5 support teams:

\[
\text{Subscribers per team} = \frac{270}{5} = 54
\]

Next, we calculate the percentage increase:

\[
\text{Percentage increase} = \frac{\text{Increase}}{\text{Original number}} \times 100\% = \frac{270}{900} \times 100\% = 30\% 
\]

\textbf{Answers:} Each support team will be credited for 54 additional subscribers, and the percentage increase in the number of subscribers from January to June is $30\%$. 

\section*{Question 10}
\textbf{Metadata}

\begin{itemize}
  \item Question ID: P5-FrAddMix\_P3-FrSmp\_GPT4.1\_Services\_03
  \item Primary KC: FRACTIONS | Addition | adding mixed numbers
  \item Secondary KC: FRACTIONS | Simplifying | expressing a fraction in its simplest form
  \item Topic: Services such as installation, maintenance, repairing, cleaning, laundry, hotel, retail, e-commerce, streaming services, digital services etc.
  \item Grade: Primary 5
\end{itemize}

\textbf{Solution}

Let us first add the two mixed numbers:

$2\frac{1}{4} + 1\frac{2}{3}$

Step 1: Convert each mixed number to an improper fraction.

$2\frac{1}{4} = 2 + \frac{1}{4} = \frac{8}{4} + \frac{1}{4} = \frac{9}{4}$

$1\frac{2}{3} = 1 + \frac{2}{3} = \frac{3}{3} + \frac{2}{3} = \frac{5}{3}$

Step 2: Add the fractions. First, find the common denominator:

LCM of 4 and 3 is 12.

$\frac{9}{4} = \frac{9 \times 3}{4 \times 3} = \frac{27}{12}$

$\frac{5}{3} = \frac{5 \times 4}{3 \times 4} = \frac{20}{12}$

So,

$\frac{27}{12} + \frac{20}{12} = \frac{47}{12}$

Step 3: Convert $\frac{47}{12}$ back to a mixed number.

$47 \div 12 = 3$ remainder $11$

So,

$\frac{47}{12} = 3\frac{11}{12}$

Step 4: Check if $\frac{11}{12}$ can be simplified. $11$ and $12$ have no common factors other than $1$, so it is in its simplest form.

Answer:

(a) The cleaners spent a total of $3\frac{11}{12}$ hours cleaning both rooms.

(b) $3\frac{11}{12}$ is already in its simplest form.

\section*{Question 11}
\textbf{Metadata}

\begin{itemize}
  \item Question ID: P5-FrMulPIm\_P2-FrCmp\_GPT4.1\_Services\_03
  \item Primary KC: FRACTIONS | Multiplication | multiplying a proper fraction and a proper/improper fractions
  \item Secondary KC: FRACTIONS | Comparison and ordering | comparing and ordering fractions
  \item Topic: Services such as installation, maintenance, repairing, cleaning, laundry, hotel, retail, e-commerce, streaming services, digital services etc.
  \item Grade: Primary 5
\end{itemize}

\textbf{Solution}

(a) The fraction of all the rooms Lina cleaned after lunch is:

She cleaned $\frac{7}{9}$ \, \text{of what was cleaned in the morning}$, so
\[
\text{Fraction of all rooms cleaned after lunch} = \frac{7}{9} \times \frac{3}{5} = \frac{7 \times 3}{9 \times 5} = \frac{21}{45}.
\]
This fraction can be simplified by dividing both numerator and denominator by $3$:
\[
\frac{21}{45} = \frac{7}{15}.
\]
So, Lina cleaned $\frac{7}{15}$ of all the rooms after lunch.

(b) To compare and order $\frac{3}{5}, \frac{7}{9},$ and $\frac{7}{15}$:

First, convert each fraction to have a common denominator:
- The least common multiple of $5, 9, 15$ is $45$.

\[
\frac{3}{5} = \frac{3 \times 9}{5 \times 9} = \frac{27}{45}
\]
\[
\frac{7}{9} = \frac{7 \times 5}{9 \times 5} = \frac{35}{45}
\]
\[
\frac{7}{15} = \frac{7 \times 3}{15 \times 3} = \frac{21}{45}
\]
Now compare $\frac{21}{45}$, $\frac{27}{45}$, $\frac{35}{45}$.

\[
\frac{7}{15} < \frac{3}{5} < \frac{7}{9}.
\]
Thus, in order from least to greatest:
\[
\boxed{\frac{7}{15},\ \frac{3}{5},\ \frac{7}{9}}
\]

\section*{Question 12}
\textbf{Metadata}

\begin{itemize}
  \item Question ID: P6-RoFndDvqWN\_P1-WNSub2nd\_GPT4.1\_Services\_01
  \item Primary KC: RATIO | Finding divided quantities | dividing a given quantity in a given ratio
  \item Secondary KC: WHOLE NUMBERS | Subtraction | subtracting whole numbers
  \item Topic: Services such as installation, maintenance, repairing, cleaning, laundry, hotel, retail, e-commerce, streaming services, digital services etc.
  \item Grade: Primary 6
\end{itemize}

\textbf{Solution}

(a) The total payment is $\textdollar540$, to be divided in the ratio $5:4$.

Total parts = $5 + 4 = 9$

Each part is $\frac{\textdollar540}{9} = \textdollar60$

Team A's share: $5 \times \textdollar60 = \textdollar300$

Team B's share: $4 \times \textdollar60 = \textdollar240$

So, Team A receives $\textdollar300$ and Team B receives $\textdollar240$.

(b) Team A pays $\textdollar30$ for cleaning supplies.

Amount left for Team A: $\textdollar300 - \textdollar30 = \textdollar270$

\textbf{Answer:}

(a) Team A receives $\textdollar300$ and Team B receives $\textdollar240$ before any expenses.

(b) After paying for the cleaning supplies, Team A has $\textdollar270$ left.

\section*{Question 13}
\textbf{Metadata}

\begin{itemize}
  \item Question ID: P6-PcFndChg\_P1-WNMul2nd\_GPT4.1\_Services\_03
  \item Primary KC: PERCENTAGE | Finding change | finding percentage increase/decrease
  \item Secondary KC: WHOLE NUMBERS | Multiplication | multiplying whole numbers
  \item Topic: Services such as installation, maintenance, repairing, cleaning, laundry, hotel, retail, e-commerce, streaming services, digital services etc.
  \item Grade: Primary 6
\end{itemize}

\textbf{Solution}

(a) Last month: Number of rooms cleaned $= 25$.

Amount earned $= 25 \times 80 = 2000$.

So, the company earned $\textdollar2000$ last month.

(b) This month: Number of rooms cleaned $= 30$.

Amount earned $= 30 \times 80 = 2400$.

So, the company earned $\textdollar2400$ this month.

(c) The increase in number of rooms cleaned $= 30 - 25 = 5$.

Percentage increase $= \dfrac{5}{25} \times 100\% = 20\%$.

So, the percentage increase in the number of rooms cleaned is $20\%$.

\section*{Question 14}
\textbf{Metadata}

\begin{itemize}
  \item Question ID: P5-RtFndR\_P2-DcCnvN2D\_GPT4.1\_Services\_03
  \item Primary KC: RATE | Finding rate | finding rate given total amount and number of units
  \item Secondary KC: DECIMALS | Conversion to larger units | converting an amount or measurement from a smaller unit to a larger unit in decimal form
  \item Topic: Services such as installation, maintenance, repairing, cleaning, laundry, hotel, retail, e-commerce, streaming services, digital services etc.
  \item Grade: Primary 5
\end{itemize}

\textbf{Solution}

(a) First, find the average amount earned for washing $1$ car:

\[
\text{Average per 1 car} = \frac{\textdollar324}{27} = \textdollar12
\]

(b) To find the average amount earned for washing $10$ cars, convert the amount to a larger unit:

\[
\text{Average per 10 cars} = \text{Average per car} \times 10 = \textdollar12 \times 10 = \textdollar120
\]

Therefore,

(a) The average amount earned for washing $1$ car is $\textdollar12$.

(b) The average amount earned for washing $10$ cars is $\textdollar120$. (in decimal form: $\textdollar120.00$)


\section*{Question 15}
\textbf{Metadata}

\begin{itemize}
  \item Question ID: P3-WNMul3d1d\_P1-WNCmp\_GPT4.1\_Services\_04
  \item Primary KC: WHOLE NUMBERS | Multiplication | multiplying whole numbers up to 3 digits by 1 digit
  \item Secondary KC: WHOLE NUMBERS | Comparison and ordering | comparing and ordering whole numbers
  \item Topic: Services such as installation, maintenance, repairing, cleaning, laundry, hotel, retail, e-commerce, streaming services, digital services etc.
  \item Grade: Primary 3
\end{itemize}

\textbf{Solution}

(a) To find out how many hours one worker takes to wash all $126$ shirts, we divide $126$ by $4$.

$126 \div 4 = 31\ \text{remainder}\ 2.$

This means:
$4 \times 31 = 124$

So, $31$ hours to wash $124$ shirts, with $2$ shirts remaining.

It takes $31$ hours for one worker to wash $124$ shirts. For the $2$ remaining shirts, the worker needs $1$ more hour (as $1$ hour is required to wash up to $4$ shirts).

So, total hours needed $= 31 + 1 = 32$ hours.

\textbf{Answer:} It takes $32$ hours for one worker to wash all $126$ shirts.

(b) Compare $126$ shirts and $148$ shirts.

$148 > 126$

\textbf{Answer:} The laundry shop washed more shirts on the day they received $148$ shirts.

\section*{Question 16}
\textbf{Metadata}

\begin{itemize}
  \item Question ID: P6-RoFndRoWN\_P1-WNAdd2nd\_GPT4.1\_Services\_03
  \item Primary KC: RATIO | Finding ratio | finding the ratio of two or three given whole numbers
  \item Secondary KC: WHOLE NUMBERS | Addition | adding whole numbers
  \item Topic: Services such as installation, maintenance, repairing, cleaning, laundry, hotel, retail, e-commerce, streaming services, digital services etc.
  \item Grade: Primary 6
\end{itemize}

\textbf{Solution}

(a) The number of rooms for Contract A, B, and C are $18$, $27$, and $15$ respectively. 

To find the ratio:
$18:27:15$
Divide each number by their highest common factor, which is $3$.

$\dfrac{18}{3} : \dfrac{27}{3} : \dfrac{15}{3} = 6:9:5$

So, the ratio is $6:9:5$.

(b) To find the total number of rooms cleaned:
$18 + 27 + 15 = 60$

So, the total number of rooms cleaned is $60$.

\section*{Question 17}
\textbf{Metadata}

\begin{itemize}
  \item Question ID: P5-FrMulImN\_P3-FrSmp\_GPT4.1\_Services\_03
  \item Primary KC: FRACTIONS | Multiplication | multiplying a proper/improper fraction and a whole number
  \item Secondary KC: FRACTIONS | Simplifying | expressing a fraction in its simplest form
  \item Topic: Services such as installation, maintenance, repairing, cleaning, laundry, hotel, retail, e-commerce, streaming services, digital services etc.
  \item Grade: Primary 5
\end{itemize}

\textbf{Solution}

To find out how many litres of juice there are in total, multiply the fraction of a litre in one jug by the number of jugs:

$\frac{7}{8} \times 12 = \frac{7 \times 12}{8}$

$= \frac{84}{8}$

Now, simplify $\frac{84}{8}$:

Divide the numerator and denominator by 4:

$\frac{84 \div 4}{8 \div 4} = \frac{21}{2}$

So, the hotel has $\frac{21}{2}$ litres of juice in total.

\section*{Question 18}
\textbf{Metadata}

\begin{itemize}
  \item Question ID: P6-FrDivPN\_P5-FrCnv2Dc\_GPT4.1\_Services\_08
  \item Primary KC: FRACTIONS | Division | dividing a proper fraction by a whole number
  \item Secondary KC: FRACTIONS | Conversion to decimals | expressing fractions as decimals
  \item Topic: Services such as installation, maintenance, repairing, cleaning, laundry, hotel, retail, e-commerce, streaming services, digital services etc.
  \item Grade: Primary 6
\end{itemize}

\textbf{Solution}

Let the total work be $\frac{3}{4}$ of the office, to be divided by 5 staff members.

(a) Fraction of the office each staff member cleans:

$\frac{3}{4} \div 5 = \frac{3}{4} \times \frac{1}{5} = \frac{3}{20}$

(b) Express $\frac{3}{20}$ as a decimal:

$\frac{3}{20} = \frac{3 \div 1}{20 \div 1} = 0.15$

So each staff member will clean $\frac{3}{20}$ (or $0.15$) of the office.

\section*{Question 19}
\textbf{Metadata}

\begin{itemize}
  \item Question ID: O3-MXMulSM\_O3-MXSub\_GPT4.1\_Services\_03
  \item Primary KC: MATRICES | Multiplication | product of a scalar quantity and a matrix
  \item Secondary KC: MATRICES | Subtraction | subtraction of matrices
  \item Topic: Services such as installation, maintenance, repairing, cleaning, laundry, hotel, retail, e-commerce, streaming services, digital services etc.
  \item Grade: Secondary O-level 3/4
\end{itemize}

\textbf{Solution}

First, double the order using scalar multiplication:

$2A = 2 \times \begin{bmatrix}12 & 18 \\ 8 & 9\end{bmatrix} = \begin{bmatrix}24 & 36 \\ 16 & 18\end{bmatrix}$

Next, subtract matrix $B$ from the doubled order:

$2A - B = \begin{bmatrix}24 & 36 \\ 16 & 18\end{bmatrix} - \begin{bmatrix}0 & 5 \\ 0 & 3\end{bmatrix} = \begin{bmatrix}24 - 0 & 36 - 5 \\ 16 - 0 & 18 - 3\end{bmatrix} = \begin{bmatrix}24 & 31 \\ 16 & 15\end{bmatrix}$

Thus, the number of cleaning supplies to be sent next week is:
- Branch 1: $24$ bottles of cleaning spray, $16$ mops
- Branch 2: $31$ bottles of cleaning spray, $15$ mops

\section*{Question 20}
\textbf{Metadata}

\begin{itemize}
  \item Question ID: P3-WNDiv3d1d\_P1-WNMul2nd\_GPT4.1\_Services\_04
  \item Primary KC: WHOLE NUMBERS | Division | dividing whole numbers up to 3 digits by 1 digit
  \item Secondary KC: WHOLE NUMBERS | Multiplication | multiplying whole numbers
  \item Topic: Services such as installation, maintenance, repairing, cleaning, laundry, hotel, retail, e-commerce, streaming services, digital services etc.
  \item Grade: Primary 3
\end{itemize}

\textbf{Solution}

(a) To find out how many shirts each washing machine will wash, divide the total number of shirts by the number of machines:

$$128 \div 4 = 32$$

Each washing machine will wash $32$ shirts.

(b) Each washing machine spends $2$ hours washing $32$ shirts. There are $4$ machines, so the total number of hours spent by all machines is:

$$4 \times 2 = 8$$

All $4$ machines spend a total of $8$ hours washing the shirts.

\end{document}
