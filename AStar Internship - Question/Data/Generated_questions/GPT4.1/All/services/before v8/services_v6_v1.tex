\documentclass{article}
\usepackage[utf8]{inputenc}
\usepackage{amsmath}
\usepackage{amsfonts}
\usepackage{amssymb}
\usepackage{graphicx}
\usepackage{hyperref}
\title{All Services MWPs by GPT4.1}
\author{Tien Dung Doan}
\begin{document}
\maketitle
\section*{Question 1}
\textbf{Metadata}

\begin{itemize}
  \item Question ID: P3-WNAdd4d\_P1-WNCmp\_GPT4.1\_Services\_03
  \item Primary KC: WHOLE NUMBERS | Addition | adding whole numbers up to 4 digits
  \item Secondary KC: WHOLE NUMBERS | Comparison and ordering | comparing and ordering whole numbers
  \item Topic: Services such as installation, maintenance, repairing, cleaning, laundry, e-commerce
  \item Grade: Primary 3
\end{itemize}

\textbf{Question}

A cleaning company received three orders to clean offices in one day. The orders were for rooms of 1,245 square feet, 2,380 square feet, and 980 square feet, respectively. 

(a) What is the total area, in square feet, the cleaning company needs to clean that day? 

(b) Another company cleaned a total of 4,500 square feet that same day. Which company cleaned a greater area, and by how many square feet?

\textbf{Solution}

Step (a): Find the total area cleaned by the first company: 

$1,245 + 2,380 + 980 = 4,605$ sq ft 

Step (b): Compare 4,605 sq ft (first company) and 4,500 sq ft (second company). 

$4,605 > 4,500$, so the first company cleaned a greater area. 

Difference: $4,605 - 4,500 = 105$ sq ft. 

Answer: The first company cleaned a greater area by 105 square feet.

\section*{Question 2}
\textbf{Metadata}

\begin{itemize}
  \item Question ID: P3-WNAdd4d\_P1-WNAdd2nd\_GPT4.1\_Services\_02
  \item Primary KC: WHOLE NUMBERS | Addition | adding whole numbers up to 4 digits
  \item Secondary KC: WHOLE NUMBERS | Addition | adding whole numbers
  \item Topic: Services such as installation, maintenance, repairing, cleaning, laundry, e-commerce
  \item Grade: Primary 3
\end{itemize}

\textbf{Question}

Mr. Lim runs a cleaning service. On Monday, he cleaned 1,236 offices. On Tuesday, he cleaned 2,405 offices. On Wednesday, he cleaned 789 offices. How many offices did Mr. Lim clean altogether from Monday to Wednesday?

\textbf{Solution}

Add the number of offices cleaned each day:

$
\begin{align*}
\text{Total offices cleaned} &= 1,236 + 2,405 + 789 \\
&= (1,236 + 2,405) + 789 \\
&= 3,641 + 789 \\
&= 4,430 \\
\end{align*}
$

Mr. Lim cleaned \(4,430\) offices altogether from Monday to Wednesday.

\section*{Question 3}
\textbf{Metadata}

\begin{itemize}
  \item Question ID: P3-WNSub4d\_P1-WNCmp\_GPT4.1\_Services\_02
  \item Primary KC: WHOLE NUMBERS | Subtraction | subtracting whole numbers up to 4 digits
  \item Secondary KC: WHOLE NUMBERS | Comparison and ordering | comparing and ordering whole numbers
  \item Topic: Services such as installation, maintenance, repairing, cleaning, laundry, e-commerce
  \item Grade: Primary 3
\end{itemize}

\textbf{Question}

A laundry shop cleaned 2,568 shirts in the first week of June and 1,972 shirts in the second week of June. 

(a) How many more shirts did the laundry shop clean in the first week than in the second week?

(b) The laundry shop also cleaned 2,645 shirts in the third week. Arrange the number of shirts cleaned in all three weeks in order, from the greatest to the least.

\textbf{Solution}

(a) To find how many more shirts were cleaned in the first week than in the second week, subtract:

2,568 - 1,972 = 596

So, the laundry shop cleaned 596 more shirts in the first week than in the second week.

(b) The number of shirts cleaned in each week:
First week: 2,568
Second week: 1,972
Third week: 2,645

Ordering from the greatest to the least:
2,645 (third week), 2,568 (first week), 1,972 (second week)

\section*{Question 4}
\textbf{Metadata}

\begin{itemize}
  \item Question ID: P3-WNSub4d\_P1-WNAdd2nd\_GPT4.1\_Services\_01
  \item Primary KC: WHOLE NUMBERS | Subtraction | subtracting whole numbers up to 4 digits
  \item Secondary KC: WHOLE NUMBERS | Addition | adding whole numbers
  \item Topic: Services such as installation, maintenance, repairing, cleaning, laundry, e-commerce
  \item Grade: Primary 3
\end{itemize}

\textbf{Question}

A washing machine repair shop received $3,240$ service requests in January and $2,715$ service requests in February. After February, the shop completed $4,532$ of these service requests in total. How many service requests are left to be completed at the end of February?

\textbf{Solution}

First, find the total number of service requests received in January and February:

$3,240 + 2,715 = 5,955$

Next, subtract the number of completed service requests from the total received:

$5,955 - 4,532 = 1,423$

Answer: There are $1,423$ service requests left to be completed at the end of February.

\section*{Question 5}
\textbf{Metadata}

\begin{itemize}
  \item Question ID: P3-WNSub4d\_P1-WNSub2nd\_GPT4.1\_Services\_01
  \item Primary KC: WHOLE NUMBERS | Subtraction | subtracting whole numbers up to 4 digits
  \item Secondary KC: WHOLE NUMBERS | Subtraction | subtracting whole numbers
  \item Topic: Services such as installation, maintenance, repairing, cleaning, laundry, e-commerce
  \item Grade: Primary 3
\end{itemize}

\textbf{Question}

A cleaning company was hired to clean 2,800 offices in a week. After finishing the first three days, the workers had already cleaned 1,263 offices. On the fourth day, they managed to clean another 480 offices. How many offices are left for the company to clean for the rest of the week?

\textbf{Solution}

First, find the total number of offices cleaned in the first four days:

$1,263 + 480 = 1,743$

Next, subtract this total from the original number of offices:

$2,800 - 1,743 = 1,057$

So, there are \(1,057\) offices left for the company to clean for the rest of the week.

\section*{Question 6}
\textbf{Metadata}

\begin{itemize}
  \item Question ID: P3-WNDivRmd3d\_P1-WNCmp\_GPT4.1\_Services\_01
  \item Primary KC: WHOLE NUMBERS | Division | dividing whole numbers up to 3 digits by 1 digit with remainder 
  \item Secondary KC: WHOLE NUMBERS | Comparison and ordering | comparing and ordering whole numbers
  \item Topic: Services such as installation, maintenance, repairing, cleaning, laundry, e-commerce
  \item Grade: Primary 3
\end{itemize}

\textbf{Question}

A cleaning company receives 227 requests for home cleaning services in a week. They send these requests equally to 4 cleaning teams.

(a) How many requests does each team get, and how many requests are left over?

(b) The team that receives the most requests in another branch cleaned 58 homes last week. Compare the number of requests each team in this branch received this week to the 58 homes cleaned by the other team. Which number is larger?

\textbf{Solution}

(a)
We divide 227 by 4:

227 \div 4 = 56 \text{ remainder } 3.

So, each team gets 56 requests, with 3 requests left over.

Answer: Each team gets 56 requests, and there are 3 requests left over.

(b)
Each team in this branch receives 56 requests. Compare 56 to 58:

56 < 58.

So, 58 is larger.

Answer: 58 is larger; the other team cleaned more homes than the number of requests each team received in this branch.

\section*{Question 7}
\textbf{Metadata}

\begin{itemize}
  \item Question ID: P3-WNDivRmd3d\_P1-WNAdd2nd\_GPT4.1\_Services\_01
  \item Primary KC: WHOLE NUMBERS | Division | dividing whole numbers up to 3 digits by 1 digit with remainder 
  \item Secondary KC: WHOLE NUMBERS | Addition | adding whole numbers
  \item Topic: Services such as installation, maintenance, repairing, cleaning, laundry, e-commerce
  \item Grade: Primary 3
\end{itemize}

\textbf{Question}

A washing company received \( 287 \) shirts from a hotel to wash. The company has baskets that can each hold \( 6 \) shirts. The workers want to place an equal number of shirts in each basket. 

After filling all the baskets, how many shirts will be left over?

If the hotel sends another \( 45 \) shirts the next day, how many shirts does the company have in total now, including the leftover shirts?

\textbf{Solution}

First, divide \( 287 \) shirts by \( 6 \) to find out how many baskets can be filled and how many shirts will be left over:

$ 287 \div 6 = 47 \text{ remainder } 5 $

That means \( 47 \) baskets can be filled, and \( 5 \) shirts will be left over.

Next, if the hotel sends another \( 45 \) shirts, we add the leftover shirts and the new shirts:

$ 5 + 45 = 50 $

So, there are \( 50 \) shirts in total now.

\section*{Question 8}
\textbf{Metadata}

\begin{itemize}
  \item Question ID: P3-WNDivRmd3d\_P1-WNSub2nd\_GPT4.1\_Services\_01
  \item Primary KC: WHOLE NUMBERS | Division | dividing whole numbers up to 3 digits by 1 digit with remainder 
  \item Secondary KC: WHOLE NUMBERS | Subtraction | subtracting whole numbers
  \item Topic: Services such as installation, maintenance, repairing, cleaning, laundry, e-commerce
  \item Grade: Primary 3
\end{itemize}

\textbf{Question}

A laundry shop received 153 dirty shirts to wash in one day. The staff must divide the shirts equally into baskets, and each basket can hold 4 shirts. After all the baskets are filled, the staff realise that some shirts are left without a basket. Later in the day, 8 more shirts are collected from another customer and need to be washed as well. 

(a) How many baskets will be completely filled? How many shirts will be left without a basket after dividing?

(b) After washing the shirts from part (a) and the 8 extra shirts, how many shirts in total does the laundry shop have left to wash?


\textbf{Solution}

Let us solve each part step by step.

(a) To find how many baskets are completely filled, divide 153 by 4:

153 \div 4 = 38\text{ remainder }1

So, 38 baskets are completely filled and 1 shirt is left without a basket.

(b) Next, 8 more shirts arrive. The staff needs to wash the 1 leftover shirt from part (a) and the 8 new shirts, totaling:

1 + 8 = 9

Therefore, the laundry shop has 9 shirts left to wash after the first round.

\section*{Question 9}
\textbf{Metadata}

\begin{itemize}
  \item Question ID: P3-WNDivRmd3d\_P1-WNMul2nd\_GPT4.1\_Services\_01
  \item Primary KC: WHOLE NUMBERS | Division | dividing whole numbers up to 3 digits by 1 digit with remainder 
  \item Secondary KC: WHOLE NUMBERS | Multiplication | multiplying whole numbers
  \item Topic: Services such as installation, maintenance, repairing, cleaning, laundry, e-commerce
  \item Grade: Primary 3
\end{itemize}

\textbf{Question}

Aisha is helping her mother do the laundry. There are 125 socks that need to be washed, and each washing machine can wash 4 socks at a time. After dividing the socks into washing machines, how many washing machines will be full, and how many socks will be left for the next round? If Aisha needs to wash the leftover socks in another round, how many socks will she have after 3 such rounds (including the leftovers each time)?

\textbf{Solution}

First, divide 125 socks by 4:

$125 \div 4 = 31$ with a remainder of $1$.

So, 31 washing machines will be full, and 1 sock will be left for the next round.

If Aisha collects leftover socks after each round and repeats:
- After 1st round, 1 sock is left.
- After 2nd round, 1 sock is left again (since 1 sock cannot fill a washing machine).
- After 3rd round, still 1 sock remains.

If we multiply the number of washing machines used each round ($31$) by 3 rounds, Aisha will have washed:

$31 \times 4 \times 3 = 372$ socks if she had that many, but since she only had 125 socks, in 3 rounds, she still just completes washing the socks, with 1 sock left each round.

Therefore, after 3 rounds, she will still have 1 sock left unwashed, as 1 sock cannot fill a washing machine by itself.

\section*{Question 10}
\textbf{Metadata}

\begin{itemize}
  \item Question ID: P3-WNMul3d1d\_P1-WNCmp\_GPT4.1\_Services\_01
  \item Primary KC: WHOLE NUMBERS | Multiplication | multiplying whole numbers up to 3 digits by 1 digit
  \item Secondary KC: WHOLE NUMBERS | Comparison and ordering | comparing and ordering whole numbers
  \item Topic: Services such as installation, maintenance, repairing, cleaning, laundry, e-commerce
  \item Grade: Primary 3
\end{itemize}

\textbf{Question}

A laundry shop receives 185 shirts from a hotel each day to be washed. The shop uses a washing machine that can wash only 1 shirt at a time. If the shop has 4 washing machines working at the same time, how many shirts can the shop wash in one hour (if each washing machine washes 1 shirt every hour)? On another day, the shop receives 780 shirts from another hotel. Compare the number of shirts from the two hotels. Which hotel sends more shirts and by how many?

\textbf{Solution}

First, find out how many shirts can be washed in one hour with 4 washing machines: $ 1 \text{ shirt per machine} \times 4 \text{ machines} = 4 \text{ shirts per hour} $ 

Next, compare the number of shirts from Hotel 1 and Hotel 2:
* Hotel 1: 185 shirts
* Hotel 2: 780 shirts

780 (Hotel 2) is more than 185 (Hotel 1).

To find by how many shirts more:
$ 780 - 185 = 595 $

Hotel 2 sends 595 more shirts than Hotel 1.

\section*{Question 11}
\textbf{Metadata}

\begin{itemize}
  \item Question ID: P3-WNMul3d1d\_P1-WNAdd2nd\_GPT4.1\_Services\_01
  \item Primary KC: WHOLE NUMBERS | Multiplication | multiplying whole numbers up to 3 digits by 1 digit
  \item Secondary KC: WHOLE NUMBERS | Addition | adding whole numbers
  \item Topic: Services such as installation, maintenance, repairing, cleaning, laundry, e-commerce
  \item Grade: Primary 3
\end{itemize}

\textbf{Question}

A cleaning company charges $7 to clean one room in an office building. There are 3 offices, and each office has 4 rooms. After cleaning all the rooms in the 3 offices, the company also cleans 6 extra meeting rooms for $7 each. How much did the company earn in total for cleaning all the rooms?

\textbf{Solution}

First, find the total number of rooms in the 3 offices:

$
\text{Number of rooms in 3 offices} = 3 \times 4 = 12
$

Next, calculate the cost for cleaning these 12 rooms:

$
\text{Cost for office rooms} = 12 \times 7 = 84
$

Now, find the cost for cleaning the 6 extra meeting rooms:

$
\text{Cost for meeting rooms} = 6 \times 7 = 42
$

Add both amounts to get the total earnings:

$
\text{Total earnings} = 84 + 42 = 126
$

So, the company earned \$126 in total.

\section*{Question 12}
\textbf{Metadata}

\begin{itemize}
  \item Question ID: P3-WNMul3d1d\_P1-WNSub2nd\_GPT4.1\_Services\_01
  \item Primary KC: WHOLE NUMBERS | Multiplication | multiplying whole numbers up to 3 digits by 1 digit
  \item Secondary KC: WHOLE NUMBERS | Subtraction | subtracting whole numbers
  \item Topic: Services such as installation, maintenance, repairing, cleaning, laundry, e-commerce
  \item Grade: Primary 3
\end{itemize}

\textbf{Question}

A cleaning company charges \$7 to clean one sofa. In a week, Mrs Tan asks the company to clean 4 sofas in her living room. However, Mrs Tan finds a coupon that gives her \$5 off the total cleaning bill. 

What is the final amount Mrs Tan has to pay after using the coupon?

\textbf{Solution}

First, find the total cost to clean 4 sofas by multiplying the price for one sofa by 4:

$
7 \times 4 = 28
$

So, the total cost before discount is \$28.

Next, subtract the coupon value:

$
28 - 5 = 23
$

Therefore, Mrs Tan has to pay \$23 after using the coupon.

\section*{Question 13}
\textbf{Metadata}

\begin{itemize}
  \item Question ID: P3-WNMul3d1d\_P1-WNMul2nd\_GPT4.1\_Services\_01
  \item Primary KC: WHOLE NUMBERS | Multiplication | multiplying whole numbers up to 3 digits by 1 digit
  \item Secondary KC: WHOLE NUMBERS | Multiplication | multiplying whole numbers
  \item Topic: Services such as installation, maintenance, repairing, cleaning, laundry, e-commerce
  \item Grade: Primary 3
\end{itemize}

\textbf{Question}

A laundry shop charges $8 to wash one shirt. Mr. Tan brought 126 shirts for his company to be washed. 

(a) How much did it cost to wash all the shirts?

(b) If Mr. Tan needs to wash the same number of shirts every month for 4 months, how much will he pay in total for 4 months?

\textbf{Solution}

(a) To find the cost to wash 126 shirts:

$126 \times 8 = 1,008$

So, it will cost $1,008 to wash all the shirts.

(b) Amount paid in 4 months:

$1,008 \times 4 = 4,032$

Mr. Tan will pay $4,032 in total for 4 months.

\section*{Question 14}
\textbf{Metadata}

\begin{itemize}
  \item Question ID: P3-WNDiv3d1d\_P1-WNCmp\_GPT4.1\_Services\_01
  \item Primary KC: WHOLE NUMBERS | Division | dividing whole numbers up to 3 digits by 1 digit
  \item Secondary KC: WHOLE NUMBERS | Comparison and ordering | comparing and ordering whole numbers
  \item Topic: Services such as installation, maintenance, repairing, cleaning, laundry, e-commerce
  \item Grade: Primary 3
\end{itemize}

\textbf{Question}

A cleaning company needs to clean 624 classrooms in a school. Each team of cleaners can clean 8 classrooms in one day. 

(a) How many teams are needed if the work must be done in one day?

(b) All the teams are given one classroom each to start with on the first hour. List the room numbers in order from the smallest to the largest that are cleaned by the first five teams in the first hour.

\textbf{Solution}

(a) To find the number of teams needed, divide 624 classrooms by 8 classrooms per team:
$
\frac{624}{8} = 78
$
So, 78 teams are needed to clean all the classrooms in one day.

(b) The first five teams will clean classrooms numbered 1, 2, 3, 4, and 5 in the first hour. Arranged from smallest to largest, the room numbers are: 1, 2, 3, 4, 5.

\section*{Question 15}
\textbf{Metadata}

\begin{itemize}
  \item Question ID: P3-WNDiv3d1d\_P1-WNAdd2nd\_GPT4.1\_Services\_01
  \item Primary KC: WHOLE NUMBERS | Division | dividing whole numbers up to 3 digits by 1 digit
  \item Secondary KC: WHOLE NUMBERS | Addition | adding whole numbers
  \item Topic: Services such as installation, maintenance, repairing, cleaning, laundry, e-commerce
  \item Grade: Primary 3
\end{itemize}

\textbf{Question}

A cleaning company needs to clean 156 chairs for a big event. They have 4 cleaners, and each cleaner will clean the same number of chairs. After they finish, 12 more chairs are delivered last minute for cleaning. How many chairs did each cleaner clean at first? After adding the 12 extra chairs, how many chairs did the cleaners clean in total?

\textbf{Solution}

First, divide the total number of chairs by the number of cleaners:

$
156 \div 4 = 39
$

Each cleaner cleaned 39 chairs at first.

Next, add the 12 extra chairs:

$
156 + 12 = 168
$

The cleaners cleaned a total of 168 chairs after the extra chairs were delivered.

\section*{Question 16}
\textbf{Metadata}

\begin{itemize}
  \item Question ID: P3-WNDiv3d1d\_P1-WNSub2nd\_GPT4.1\_Services\_01
  \item Primary KC: WHOLE NUMBERS | Division | dividing whole numbers up to 3 digits by 1 digit
  \item Secondary KC: WHOLE NUMBERS | Subtraction | subtracting whole numbers
  \item Topic: Services such as installation, maintenance, repairing, cleaning, laundry, e-commerce
  \item Grade: Primary 3
\end{itemize}

\textbf{Question}

A cleaning company received 216 dirty towels from a hotel to wash. They can wash the towels by placing 8 towels in each washing machine at a time. After washing, 35 towels were found to still be dirty and needed to be rewashed. How many washing machines of towels were cleaned properly on the first wash?

\textbf{Solution}

First, divide the total number of towels by the number of towels in each machine: 

$$216 \div 8 = 27 \text{ loads}$$

Next, subtract the number of towels that were still dirty after the first wash:

$$216 - 35 = 181$$

Now, find how many loads of 8 towels are in the 181 clean towels:

$$181 \div 8 = 22 \text{ loads} \, \text{(with a remainder of 5 towels)}$$

So, \boxed{22} washing machines of towels were cleaned properly on the first wash.

\section*{Question 17}
\textbf{Metadata}

\begin{itemize}
  \item Question ID: P3-WNDiv3d1d\_P1-WNMul2nd\_GPT4.1\_Services\_01
  \item Primary KC: WHOLE NUMBERS | Division | dividing whole numbers up to 3 digits by 1 digit
  \item Secondary KC: WHOLE NUMBERS | Multiplication | multiplying whole numbers
  \item Topic: Services such as installation, maintenance, repairing, cleaning, laundry, e-commerce
  \item Grade: Primary 3
\end{itemize}

\textbf{Question}

A laundry company receives 216 shirts to wash. The shirts are to be packed equally into boxes, with each box containing 6 shirts. 

(a) How many boxes does the company need to pack all the shirts? 

(b) If one of their workers can carry 4 boxes at a time, how many shirts can the worker carry in one trip?

\textbf{Solution}

(a) To find out how many boxes are needed, divide 216 by 6:

$$
216 \div 6 = 36
$$

So, the company needs **36 boxes**.

(b) One box contains 6 shirts. If the worker carries 4 boxes, the total number of shirts is:

$$
4 \times 6 = 24
$$

So, the worker can carry **24 shirts** in one trip.

\section*{Question 18}
\textbf{Metadata}

\begin{itemize}
  \item Question ID: P3-FrAddRl12\_P2-FrCmp\_GPT4.1\_Services\_01
  \item Primary KC: FRACTIONS | Addition | adding two related fractions within one whole with denominators of given fractions not exceeding 12
  \item Secondary KC: FRACTIONS | Comparison and ordering | comparing and ordering fractions
  \item Topic: Services such as installation, maintenance, repairing, cleaning, laundry, e-commerce
  \item Grade: Primary 3
\end{itemize}

\textbf{Question}

A washing company cleaned $\frac{2}{6}$ of a large carpet in the morning and $\frac{3}{6}$ of the carpet in the afternoon. 

(a) What fraction of the carpet did the company clean in total?

(b) After cleaning, did the company clean more than half of the carpet? Explain your answer.

\textbf{Solution}

(a) To find the total fraction cleaned, add the two fractions:

$\frac{2}{6} + \frac{3}{6} = \frac{2+3}{6} = \frac{5}{6}$

So, the company cleaned $\frac{5}{6}$ of the carpet in total.

(b) To compare $\frac{5}{6}$ with half, we compare it to $\frac{1}{2}$.

Convert $\frac{1}{2}$ to a denominator of 6:

$\frac{1}{2} = \frac{3}{6}$

Since $\frac{5}{6} > \frac{3}{6}$, $\frac{5}{6}$ is more than half.

Therefore, the company cleaned more than half of the carpet.

\section*{Question 19}
\textbf{Metadata}

\begin{itemize}
  \item Question ID: P3-FrAddRl12\_P2-FrAdd2nd\_GPT4.1\_Services\_01
  \item Primary KC: FRACTIONS | Addition | adding two related fractions within one whole with denominators of given fractions not exceeding 12
  \item Secondary KC: FRACTIONS | Addition | adding fractions
  \item Topic: Services such as installation, maintenance, repairing, cleaning, laundry, e-commerce
  \item Grade: Primary 3
\end{itemize}

\textbf{Question}

A washing machine repairman spends $\frac{3}{8}$ of an hour fixing the motor and $\frac{1}{8}$ of an hour cleaning the filter. How much time did he spend on both tasks together?

\textbf{Solution}

To find the total time spent, add the two fractions:

$\frac{3}{8} + \frac{1}{8} = \frac{3+1}{8} = \frac{4}{8}$

Simplify $\frac{4}{8}$:

$\frac{4}{8} = \frac{1}{2}$

So, the repairman spent $\frac{1}{2}$ hour on both tasks together.

\section*{Question 20}
\textbf{Metadata}

\begin{itemize}
  \item Question ID: P3-FrAddRl12\_P3-FrSmp\_GPT4.1\_Services\_01
  \item Primary KC: FRACTIONS | Addition | adding two related fractions within one whole with denominators of given fractions not exceeding 12
  \item Secondary KC: FRACTIONS | Simplifying | expressing a fraction in its simplest form
  \item Topic: Services such as installation, maintenance, repairing, cleaning, laundry, e-commerce
  \item Grade: Primary 3
\end{itemize}

\textbf{Question}

Mrs. Lee hired a cleaning service to help clean her house. The cleaners spent $\frac{2}{6}$ of an hour cleaning the living room and $\frac{3}{6}$ of an hour cleaning the kitchen. What fraction of an hour did they spend altogether cleaning both rooms? Give your answer in its simplest form.

\textbf{Solution}

To find the total time spent cleaning both rooms, add the two fractions:

$\frac{2}{6} + \frac{3}{6} = \frac{2 + 3}{6} = \frac{5}{6}$

The fraction $\frac{5}{6}$ is already in its simplest form. 

So, the cleaners spent $\frac{5}{6}$ of an hour cleaning both rooms.

\section*{Question 21}
\textbf{Metadata}

\begin{itemize}
  \item Question ID: P3-FrSubRl12\_P2-FrCmp\_GPT4.1\_Services\_01
  \item Primary KC: FRACTIONS | Subtraction | subtracting two related fractions within one whole with denominators of given fractions not exceeding 12
  \item Secondary KC: FRACTIONS | Comparison and ordering | comparing and ordering fractions
  \item Topic: Services such as installation, maintenance, repairing, cleaning, laundry, e-commerce
  \item Grade: Primary 3
\end{itemize}

\textbf{Question}

A washing machine repairman used $\frac{7}{12}$ of a bottle of cleaning solution to clean one machine and $\frac{5}{12}$ of the bottle to clean another machine. 

(a) How much more cleaning solution did the repairman use for the first machine than for the second machine?

(b) After cleaning both machines, did the repairman use more than half of the bottle of cleaning solution? Show your working.

\textbf{Solution}

(a) To find out how much more solution was used for the first machine than the second, subtract: 

$\frac{7}{12} - \frac{5}{12} = \frac{2}{12} = \frac{1}{6}$

The repairman used $\frac{1}{6}$ of a bottle more for the first machine.

(b) Total amount of cleaning solution used: 

$\frac{7}{12} + \frac{5}{12} = \frac{12}{12} = 1$ (a whole bottle)

Half of a bottle is $\frac{1}{2} = \frac{6}{12}$.

Since 1 (or $\frac{12}{12}$) is more than $\frac{6}{12}$, the repairman used more than half of the bottle.

\section*{Question 22}
\textbf{Metadata}

\begin{itemize}
  \item Question ID: P3-FrSubRl12\_P2-FrAdd2nd\_GPT4.1\_Services\_01
  \item Primary KC: FRACTIONS | Subtraction | subtracting two related fractions within one whole with denominators of given fractions not exceeding 12
  \item Secondary KC: FRACTIONS | Addition | adding fractions
  \item Topic: Services such as installation, maintenance, repairing, cleaning, laundry, e-commerce
  \item Grade: Primary 3
\end{itemize}

\textbf{Question}

Aisha was helping her mother do laundry. In the morning, she washed \( \frac{7}{12} \) of the dirty clothes. In the afternoon, she washed another \( \frac{1}{4} \) of the dirty clothes. 

(a) What fraction of the dirty clothes did Aisha wash in total? 

(b) If all the dirty clothes needed to be washed, what fraction of the dirty clothes were still unwashed after the afternoon?

\textbf{Solution}

(a) To find the total fraction of dirty clothes Aisha washed, we add the two fractions:

$
\frac{7}{12} + \frac{1}{4} = \frac{7}{12} + \frac{3}{12} = \frac{10}{12} = \frac{5}{6}
$

So, Aisha washed \( \frac{5}{6} \) of the dirty clothes in total.

(b) To find the fraction of clothes still unwashed:

$
1 - \frac{5}{6} = \frac{6}{6} - \frac{5}{6} = \frac{1}{6}
$

So, \( \frac{1}{6} \) of the dirty clothes were still unwashed after the afternoon.

\section*{Question 23}
\textbf{Metadata}

\begin{itemize}
  \item Question ID: P3-FrSubRl12\_P2-FrSub2nd\_GPT4.1\_Services\_01
  \item Primary KC: FRACTIONS | Subtraction | subtracting two related fractions within one whole with denominators of given fractions not exceeding 12
  \item Secondary KC: FRACTIONS | Subtraction | subtracting fractions
  \item Topic: Services such as installation, maintenance, repairing, cleaning, laundry, e-commerce
  \item Grade: Primary 3
\end{itemize}

\textbf{Question}

Mrs Tan hired a cleaning service to clean her apartment. On Monday, the cleaners finished \( \frac{7}{12} \) of the total cleaning work. On Tuesday, they cleaned another \( \frac{4}{12} \) of the work. 

How much more of the cleaning work was done on Monday than on Tuesday?

\textbf{Solution}

Amount cleaned on Monday: \( \frac{7}{12} \)
Amount cleaned on Tuesday: \( \frac{4}{12} \)

To find how much more was done on Monday:
$
\frac{7}{12} - \frac{4}{12} = \frac{3}{12}
$

So, \( \frac{3}{12} \) more of the cleaning work was done on Monday than on Tuesday. This can be simplified to \( \frac{1}{4} \).

\section*{Question 24}
\textbf{Metadata}

\begin{itemize}
  \item Question ID: P3-FrSubRl12\_P3-FrSmp\_GPT4.1\_Services\_01
  \item Primary KC: FRACTIONS | Subtraction | subtracting two related fractions within one whole with denominators of given fractions not exceeding 12
  \item Secondary KC: FRACTIONS | Simplifying | expressing a fraction in its simplest form
  \item Topic: Services such as installation, maintenance, repairing, cleaning, laundry, e-commerce
  \item Grade: Primary 3
\end{itemize}

\textbf{Question}

A cleaning company completed \( \frac{7}{12} \) of a school's annual cleaning tasks by the end of June. By the end of September, they have completed \( \frac{11}{12} \) of the tasks for the year. 

How much more of the annual cleaning tasks did the company complete from the end of June to the end of September? Express your answer in its simplest form.

\textbf{Solution}

To find out how much more of the tasks were completed, subtract the fraction completed by June from the fraction completed by September:

$
\frac{11}{12} - \frac{7}{12} = \frac{11 - 7}{12} = \frac{4}{12}
$

Simplify \( \frac{4}{12} \):

$
\frac{4}{12} = \frac{1}{3}
$

So, the company completed \( \frac{1}{3} \) more of the annual cleaning tasks from the end of June to the end of September.

\section*{Question 25}
\textbf{Metadata}

\begin{itemize}
  \item Question ID: P4-WNMul4d1d\_P1-WNCmp\_GPT4.1\_Services\_01
  \item Primary KC: WHOLE NUMBERS | Multiplication | multiplying whole numbers up to 4 digits by 1 digit or up to 3 digits by 2 digits
  \item Secondary KC: WHOLE NUMBERS | Comparison and ordering | comparing and ordering whole numbers
  \item Topic: Services such as installation, maintenance, repairing, cleaning, laundry, e-commerce
  \item Grade: Primary 4
\end{itemize}

\textbf{Question}

A cleaning service company accepts two different jobs in one week. For Job A, the company cleans 128 houses, and each house pays $36 for cleaning. For Job B, the company cleans 85 offices, and each office pays $48 for cleaning. 

(a) How much money did the company earn from Job A? 

(b) How much money did the company earn from Job B? 

(c) Which job earned the company more money, and by how much?


\textbf{Solution}

(a) From Job A: 
128 \times 36 = 4,608
The company earned $4,608 from Job A.

(b) From Job B: 
85 \times 48 = 4,080
The company earned $4,080 from Job B.

(c) Job A earned more money.
$4,608 - $4,080 = $528

So, Job A earned $528 more than Job B.

\section*{Question 26}
\textbf{Metadata}

\begin{itemize}
  \item Question ID: P4-WNMul4d1d\_P1-WNAdd2nd\_GPT4.1\_Services\_01
  \item Primary KC: WHOLE NUMBERS | Multiplication | multiplying whole numbers up to 4 digits by 1 digit or up to 3 digits by 2 digits
  \item Secondary KC: WHOLE NUMBERS | Addition | adding whole numbers
  \item Topic: Services such as installation, maintenance, repairing, cleaning, laundry, e-commerce
  \item Grade: Primary 4
\end{itemize}

\textbf{Question}

A laundry shop received 126 bags of clothes from a hotel for cleaning. Each bag contains 38 shirts. The shop also received 1,275 shirts from another hotel. How many shirts in total did the laundry shop receive from both hotels?

\textbf{Solution}

First, find the number of shirts in the 126 bags:

$126 \times 38 = 4,788$ shirts

The number of shirts from the other hotel is $1,275$.

Total number of shirts received:

$4,788 + 1,275 = 6,063$

**Final answer:** The laundry shop received a total of $6,063$ shirts from both hotels.

\section*{Question 27}
\textbf{Metadata}

\begin{itemize}
  \item Question ID: P4-WNMul4d1d\_P1-WNSub2nd\_GPT4.1\_Services\_01
  \item Primary KC: WHOLE NUMBERS | Multiplication | multiplying whole numbers up to 4 digits by 1 digit or up to 3 digits by 2 digits
  \item Secondary KC: WHOLE NUMBERS | Subtraction | subtracting whole numbers
  \item Topic: Services such as installation, maintenance, repairing, cleaning, laundry, e-commerce
  \item Grade: Primary 4
\end{itemize}

\textbf{Question}

A cleaning company has 325 offices to clean each week. Each office requires 8 cleaning cloths. 

(a) How many cleaning cloths does the company need for all the offices each week?

After cleaning one week, the company found that 1,650 cleaning cloths were damaged and could not be used again.

(b) How many cleaning cloths are left after removing the damaged ones?

\textbf{Solution}

Part (a):
Each office needs 8 cleaning cloths.
Number of offices = 325

Total cleaning cloths needed = $325 \times 8 = 2,600$

Part (b):
Number of cloths left after removing damaged ones:
Total cloths used = 2,600
Damaged cloths = 1,650

Cloths left = $2,600 - 1,650 = 950$

**Final answers:**
(a) The company needs 2,600 cleaning cloths each week.
(b) There are 950 cleaning cloths left after removing the damaged ones.

\section*{Question 28}
\textbf{Metadata}

\begin{itemize}
  \item Question ID: P4-WNMul4d1d\_P1-WNMul2nd\_GPT4.1\_Services\_01
  \item Primary KC: WHOLE NUMBERS | Multiplication | multiplying whole numbers up to 4 digits by 1 digit or up to 3 digits by 2 digits
  \item Secondary KC: WHOLE NUMBERS | Multiplication | multiplying whole numbers
  \item Topic: Services such as installation, maintenance, repairing, cleaning, laundry, e-commerce
  \item Grade: Primary 4
\end{itemize}

\textbf{Question}

A cleaning service company charges \$135 for cleaning one office unit. Last week, the company cleaned 18 office units of the same size. 

(a) How much did the company earn from cleaning all 18 office units last week?

(b) If the company plans to clean the same number of office units each week for 7 weeks, how much would they earn in total after 7 weeks?

\textbf{Solution}

Step (a):

Amount earned from 1 office unit = \$135

Number of office units = 18

Total amount earned last week = 135 \times 18 = 2,430

So, the company earned \$2,430 last week.

Step (b):

Amount earned each week = \$2,430

Number of weeks = 7

Total amount for 7 weeks = 2,430 \times 7 = 17,010

So, the company would earn \$17,010 in total after 7 weeks.

\section*{Question 29}
\textbf{Metadata}

\begin{itemize}
  \item Question ID: P4-WNMul4d1d\_P4-WNRnd5d\_GPT4.1\_Services\_01
  \item Primary KC: WHOLE NUMBERS | Multiplication | multiplying whole numbers up to 4 digits by 1 digit or up to 3 digits by 2 digits
  \item Secondary KC: WHOLE NUMBERS | Rounding | rounding whole numbers up to 100000 to the nearest 10, 100 or 1000 
  \item Topic: Services such as installation, maintenance, repairing, cleaning, laundry, e-commerce
  \item Grade: Primary 4
\end{itemize}

\textbf{Question}

A laundry company washes \( 1,368 \) shirts each day for a busy hotel. The company operates every day for a month with 28 days. 

(a) How many shirts does the company wash in 28 days?

(b) Round your answer to the nearest thousand.

\textbf{Solution}

(a) Number of shirts washed in 28 days: 
\$1,368 \times 28 = 38,304\$ 
So, the company washes \(38,304\) shirts in 28 days.

(b) Rounding \(38,304\) to the nearest thousand:
Since the hundreds digit is 3 (which is less than 5), we round down.
\(38,304 \approx 38,000\) (to the nearest thousand).

\section*{Question 30}
\textbf{Metadata}

\begin{itemize}
  \item Question ID: P4-WNDiv4d1d\_P1-WNCmp\_GPT4.1\_Services\_01
  \item Primary KC: WHOLE NUMBERS | Division | dividing whole numbers up to 4 digits by 1 digit
  \item Secondary KC: WHOLE NUMBERS | Comparison and ordering | comparing and ordering whole numbers
  \item Topic: Services such as installation, maintenance, repairing, cleaning, laundry, e-commerce
  \item Grade: Primary 4
\end{itemize}

\textbf{Question}

A cleaning company accepted 2,568 laundry items to wash in one day. The items are packed equally into baskets, each basket holding 4 items. 

(a) How many baskets are needed to pack all the laundry items?

(b) If another company washed 2,430 items and packed them equally into baskets with 5 items each, which company needed more baskets? Explain your answer by comparing the numbers of baskets required by each company.

\textbf{Solution}

Let's solve part (a):

Number of baskets needed $= \frac{2,568}{4}$

$2,568 \div 4 = 642$

So, 642 baskets are needed for the first company.

Now, part (b):

For the second company:

Number of baskets $= \frac{2,430}{5}$

$2,430 \div 5 = 486$

Comparing the two:

642 baskets (first company) vs 486 baskets (second company)

642 > 486

Therefore, the first company needed more baskets than the second company.

\section*{Question 31}
\textbf{Metadata}

\begin{itemize}
  \item Question ID: P4-WNDiv4d1d\_P1-WNAdd2nd\_GPT4.1\_Services\_01
  \item Primary KC: WHOLE NUMBERS | Division | dividing whole numbers up to 4 digits by 1 digit
  \item Secondary KC: WHOLE NUMBERS | Addition | adding whole numbers
  \item Topic: Services such as installation, maintenance, repairing, cleaning, laundry, e-commerce
  \item Grade: Primary 4
\end{itemize}

\textbf{Question}

A cleaning company received an order to clean 1,836 chairs for a school event. The company has 6 workers, and they want to split the chairs equally among all the workers.

After cleaning, the workers cleaned 257 more chairs for another small office on the same day.

(a) How many chairs did each worker clean for the school event?

(b) What is the total number of chairs cleaned by all the workers on that day?

\textbf{Solution}

(a) To find out how many chairs each worker cleaned for the school event:

$1,836 \div 6 = 306$

So, each worker cleaned 306 chairs for the school event.

(b) First, find the total number of chairs cleaned on the same day:

Total number of chairs cleaned $= 1,836$ (school event) $+ 257$ (office) $= 2,093$

So, the total number of chairs cleaned by all the workers on that day is $2,093$.

\section*{Question 32}
\textbf{Metadata}

\begin{itemize}
  \item Question ID: P4-WNDiv4d1d\_P1-WNSub2nd\_GPT4.1\_Services\_01
  \item Primary KC: WHOLE NUMBERS | Division | dividing whole numbers up to 4 digits by 1 digit
  \item Secondary KC: WHOLE NUMBERS | Subtraction | subtracting whole numbers
  \item Topic: Services such as installation, maintenance, repairing, cleaning, laundry, e-commerce
  \item Grade: Primary 4
\end{itemize}

\textbf{Question}

A company received 3,672 laundry shirts for cleaning. The manager wants to divide the shirts equally among 6 workers. After the workers finished their share of shirts, 128 shirts were found to have stains that needed to be cleaned again. 

How many shirts did each worker clean at first? After setting aside the 128 stained shirts, how many cleaned shirts are left?

\textbf{Solution}

First, we divide 3,672 shirts by 6 workers:

$
\frac{3,672}{6} = 612
$

Each worker cleaned 612 shirts at first.

Next, we subtract the 128 stained shirts from the total cleaned shirts:

$
3,672 - 128 = 3,544
$

After setting aside the 128 stained shirts, there are 3,544 cleaned shirts left.

\section*{Question 33}
\textbf{Metadata}

\begin{itemize}
  \item Question ID: P4-WNDiv4d1d\_P1-WNMul2nd\_GPT4.1\_Services\_01
  \item Primary KC: WHOLE NUMBERS | Division | dividing whole numbers up to 4 digits by 1 digit
  \item Secondary KC: WHOLE NUMBERS | Multiplication | multiplying whole numbers
  \item Topic: Services such as installation, maintenance, repairing, cleaning, laundry, e-commerce
  \item Grade: Primary 4
\end{itemize}

\textbf{Question}

A cleaning company has a contract to clean 2,352 office desks every week. They have 6 workers who share the work equally. 

(a) How many desks does each worker need to clean in a week?

If each worker cleans the same number of desks every day for 4 working days in a week,

(b) How many desks does each worker clean each day?

\textbf{Solution}

(a) To find out how many desks each worker needs to clean in a week:

$$
\text{Number of desks each worker cleans in a week} = \frac{2352}{6} = 392
$$
So, each worker needs to clean 392 desks in a week.

(b) To find out how many desks each worker cleans each day, divide the weekly workload by 4:

$$
\text{Number of desks each worker cleans each day} = \frac{392}{4} = 98
$$
So, each worker cleans 98 desks each day.

\section*{Question 34}
\textbf{Metadata}

\begin{itemize}
  \item Question ID: P4-WNDiv4d1d\_P4-WNRnd5d\_GPT4.1\_Services\_01
  \item Primary KC: WHOLE NUMBERS | Division | dividing whole numbers up to 4 digits by 1 digit
  \item Secondary KC: WHOLE NUMBERS | Rounding | rounding whole numbers up to 100000 to the nearest 10, 100 or 1000 
  \item Topic: Services such as installation, maintenance, repairing, cleaning, laundry, e-commerce
  \item Grade: Primary 4
\end{itemize}

\textbf{Question}

A cleaning company received an order to clean 7,964 chairs for an event. The company wants to split the work equally among their 6 teams. 

(a) How many chairs will each team need to clean? 

(b) Round the number of chairs each team needs to clean to the nearest 10.

\textbf{Solution}

Step 1: Divide the total number of chairs by the number of teams:

$\displaystyle \frac{7,964}{6} = 1,327$ remainder $2$.

So, each team will need to clean $1,327$ chairs, and there are 2 chairs left over.

Step 2: Round the number of chairs each team cleans to the nearest 10:

$1,327$ rounded to the nearest 10 is $1,330$ (since the last digit, 7, is 5 or more, we round up).

Final answers:
(a) $1,327$ chairs per team
(b) $1,330$ chairs per team (rounded to the nearest 10)

\section*{Question 35}
\textbf{Metadata}

\begin{itemize}
  \item Question ID: P4-FrRepMixIm\_P3-FrCnvEq\_GPT4.1\_Services\_01
  \item Primary KC: FRACTIONS | Representation and concept | mixed numbers and improper fractions
  \item Secondary KC: FRACTIONS | Conversion to equivalent fractions | Conversion to equivalent fractions (given either the denominator or the numerator)
  \item Topic: Services such as installation, maintenance, repairing, cleaning, laundry, e-commerce
  \item Grade: Primary 4
\end{itemize}

\textbf{Question}

A laundry service washed $2\dfrac{3}{4}$ baskets of clothes in the morning and $\dfrac{11}{4}$ baskets in the afternoon. 

(a) Write both amounts as improper fractions. 

(b) To compare them easily, express both improper fractions with a denominator of 8. 

Which period did the laundry service wash more baskets of clothes?

\textbf{Solution}

(a) $2\dfrac{3}{4}$ as an improper fraction is $\dfrac{11}{4}$ (since $2 \times 4 + 3 = 8 + 3 = 11$).

So, both the morning and afternoon are $\dfrac{11}{4}$ baskets!

(b) To express $\dfrac{11}{4}$ with denominator 8, multiply numerator and denominator by 2:

$\dfrac{11}{4} \times \dfrac{2}{2} = \dfrac{22}{8}$

So, both $2\dfrac{3}{4}$ and $\dfrac{11}{4}$ become $\dfrac{22}{8}$.

Therefore, the laundry service washed the same number of baskets in the morning and afternoon.

\section*{Question 36}
\textbf{Metadata}

\begin{itemize}
  \item Question ID: P4-FrRepSet\_P3-FrCnvEq\_GPT4.1\_Services\_01
  \item Primary KC: FRACTIONS | Representation and concept | fraction as part of a set 
  \item Secondary KC: FRACTIONS | Conversion to equivalent fractions | Conversion to equivalent fractions (given either the denominator or the numerator)
  \item Topic: Services such as installation, maintenance, repairing, cleaning, laundry, e-commerce
  \item Grade: Primary 4
\end{itemize}

\textbf{Question}

A laundromat has 12 washing machines. During the weekend, \( \frac{2}{3} \) of the machines were busy washing clothes for customers. 

(a) How many machines were being used? 

(b) Out of those machines being used, \( \frac{1}{2} \) were used for washing bedsheets. What fraction of all the machines were used to wash bedsheets? Write your answer as an equivalent fraction with denominator 12.

\textbf{Solution}

(a) To find how many machines were being used: 

\( \frac{2}{3} \times 12 = 8 \) machines.

(b) Out of these 8 machines, \( \frac{1}{2} \times 8 = 4 \) machines were used to wash bedsheets.

To find the fraction of all machines used to wash bedsheets:

\( \frac{4}{12} \) of all machines = \( \frac{1}{3} \) (after simplifying), but since the question asks for a fraction with denominator 12, the answer is \( \boxed{\frac{4}{12}} \).

\section*{Question 37}
\textbf{Metadata}

\begin{itemize}
  \item Question ID: P4-FrAddU12\_P2-FrCmp\_GPT4.1\_Services\_01
  \item Primary KC: FRACTIONS | Addition | adding unlike fractions with two different denominators not exceeding 12
  \item Secondary KC: FRACTIONS | Comparison and ordering | comparing and ordering fractions
  \item Topic: Services such as installation, maintenance, repairing, cleaning, laundry, e-commerce
  \item Grade: Primary 4
\end{itemize}

\textbf{Question}

A cleaning company received two orders for cleaning two different offices. For Office A, the cleaners spent $\frac{2}{3}$ of an hour. For Office B, they spent $\frac{5}{12}$ of an hour. 

(a) How much time did the cleaners spend cleaning both offices in total? 

(b) Which office took a longer time to clean, and by how much?

\textbf{Solution}

**(a)** To find the total time spent, add the two fractions:

$
\frac{2}{3} + \frac{5}{12}
$

The denominators are different. The least common denominator for 3 and 12 is 12.

$
\frac{2}{3} = \frac{2 \times 4}{3 \times 4} = \frac{8}{12}
$

So,
$
\frac{8}{12} + \frac{5}{12} = \frac{13}{12}
$

$\frac{13}{12}$ hours is the total time, which can also be written as $1\frac{1}{12}$ hours.

**(b)** To compare the two times:

$
\frac{2}{3} = \frac{8}{12}\text{ and }\frac{5}{12}
$

$\frac{2}{3}$ is greater than $\frac{5}{12}$.

The difference:
$
\frac{8}{12} - \frac{5}{12} = \frac{3}{12} = \frac{1}{4}
$

**Conclusion:**

- The cleaners took a longer time cleaning Office A.
- Office A took $\frac{1}{4}$ hour longer to clean than Office B.

\section*{Question 38}
\textbf{Metadata}

\begin{itemize}
  \item Question ID: P4-FrAddU12\_P2-FrAdd2nd\_GPT4.1\_Services\_01
  \item Primary KC: FRACTIONS | Addition | adding unlike fractions with two different denominators not exceeding 12
  \item Secondary KC: FRACTIONS | Addition | adding fractions
  \item Topic: Services such as installation, maintenance, repairing, cleaning, laundry, e-commerce
  \item Grade: Primary 4
\end{itemize}

\textbf{Question}

Aisha is helping her mother do the laundry on a Sunday morning. She uses $\frac{3}{8}$ of a box of detergent to wash some clothes. Later, she washes another load and uses $\frac{1}{6}$ of the box. 

How much of the box of detergent did Aisha use altogether?

\textbf{Solution}

To find the total amount of detergent used, we need to add $\frac{3}{8}$ and $\frac{1}{6}$ together.

Let's find a common denominator for $8$ and $6$. The smallest common multiple is $24$.

$\frac{3}{8} = \frac{3 \times 3}{8 \times 3} = \frac{9}{24}$

$\frac{1}{6} = \frac{1 \times 4}{6 \times 4} = \frac{4}{24}$

Now, add the fractions:
$\frac{9}{24} + \frac{4}{24} = \frac{13}{24}$

Aisha used $\frac{13}{24}$ of the box of detergent altogether.

\section*{Question 39}
\textbf{Metadata}

\begin{itemize}
  \item Question ID: P4-FrAddU12\_P3-FrSmp\_GPT4.1\_Services\_01
  \item Primary KC: FRACTIONS | Addition | adding unlike fractions with two different denominators not exceeding 12
  \item Secondary KC: FRACTIONS | Simplifying | expressing a fraction in its simplest form
  \item Topic: Services such as installation, maintenance, repairing, cleaning, laundry, e-commerce
  \item Grade: Primary 4
\end{itemize}

\textbf{Question}

A washing machine repairman fixed two machines in a day. For the first machine, he finished \(\frac{3}{4}\) of the repair work by lunchtime. For the second machine, he finished \(\frac{2}{3}\) of the repair work before lunch as well. What fraction of the total repair work for both machines did he complete by lunchtime? Give your answer in its simplest form.

\textbf{Solution}

Total fraction of work done by lunchtime = \(\frac{3}{4} + \frac{2}{3}\).

Find a common denominator (12):
$
\frac{3}{4} = \frac{9}{12}, \qquad \frac{2}{3} = \frac{8}{12}
$
So,
$
\frac{9}{12} + \frac{8}{12} = \frac{17}{12}
$

Expressing \(\frac{17}{12}\) as a mixed number, we have \(1\frac{5}{12}\).

**Answer:** The repairman completed \(1\frac{5}{12}\) of the total repair work for both machines by lunchtime.

\section*{Question 40}
\textbf{Metadata}

\begin{itemize}
  \item Question ID: P4-FrSubU12\_P2-FrCmp\_GPT4.1\_Services\_01
  \item Primary KC: FRACTIONS | Subtraction | subtracting unlike fractions with two different denominators not exceeding 12
  \item Secondary KC: FRACTIONS | Comparison and ordering | comparing and ordering fractions
  \item Topic: Services such as installation, maintenance, repairing, cleaning, laundry, e-commerce
  \item Grade: Primary 4
\end{itemize}

\textbf{Question}

A laundry service received \( \frac{3}{4} \) of a basket of white clothes and \( \frac{5}{6} \) of a basket of coloured clothes from a customer in the morning. After cleaning, the customer found that \( \frac{1}{3} \) of a basket of coloured clothes were missing. 

1) How many baskets of coloured clothes did the customer get back after cleaning?  
2) Did the customer get back more white clothes or coloured clothes after cleaning? Show your working.

\textbf{Solution}

1) The amount of coloured clothes the customer got back after cleaning:

$
\frac{5}{6} - \frac{1}{3}
$

Find the common denominator (6):
$
\frac{5}{6} - \frac{2}{6} = \frac{3}{6} = \frac{1}{2}
$

So, the customer got back \( \frac{1}{2} \) of a basket of coloured clothes after cleaning.

2) Compare \( \frac{3}{4} \) (white clothes) and \( \frac{1}{2} \) (coloured clothes):

Convert \( \frac{3}{4} \) and \( \frac{1}{2} \) to have a common denominator (4):
$
\frac{3}{4} > \frac{2}{4} = \frac{1}{2}
$

Therefore, the customer got back more white clothes than coloured clothes after cleaning.

\section*{Question 41}
\textbf{Metadata}

\begin{itemize}
  \item Question ID: P4-FrSubU12\_P2-FrAdd2nd\_GPT4.1\_Services\_01
  \item Primary KC: FRACTIONS | Subtraction | subtracting unlike fractions with two different denominators not exceeding 12
  \item Secondary KC: FRACTIONS | Addition | adding fractions
  \item Topic: Services such as installation, maintenance, repairing, cleaning, laundry, e-commerce
  \item Grade: Primary 4
\end{itemize}

\textbf{Question}

A cleaning company uses $\frac{3}{4}$ litre of detergent to clean the floor of a flat and $\frac{5}{12}$ litre to clean the windows of the same flat. After cleaning both, they realised they needed to use an extra $\frac{1}{6}$ litre of detergent to remove some tough stains. 

(a) How much detergent did they use in total to clean the flat?

(b) If the company had brought only 2 litres of detergent at the start, how much detergent do they have left after cleaning the flat?


\textbf{Solution}

(a) To find the total detergent used, add all the quantities:

$\frac{3}{4} + \frac{5}{12} + \frac{1}{6}$

First, find the common denominator. The denominators are 4, 12, and 6. The lowest common multiple is 12.

$\frac{3}{4} = \frac{9}{12}$

$\frac{5}{12} = \frac{5}{12}$

$\frac{1}{6} = \frac{2}{12}$

Add them together:
$\frac{9}{12} + \frac{5}{12} + \frac{2}{12} = \frac{16}{12}$

Simplify $\frac{16}{12}$:
$\frac{16}{12} = \frac{4}{3} \text{ litres}$

So, $\frac{4}{3}$ litres (or $1\frac{1}{3}$ litres) of detergent was used.

(b) The company brought 2 litres.

Amount left:
$2 - \frac{4}{3}$

Write 2 as a fraction with denominator 3:
$2 = \frac{6}{3}$

So,
$\frac{6}{3} - \frac{4}{3} = \frac{2}{3}$

The company has $\frac{2}{3}$ litre of detergent left after cleaning the flat.

\section*{Question 42}
\textbf{Metadata}

\begin{itemize}
  \item Question ID: P4-FrSubU12\_P2-FrSub2nd\_GPT4.1\_Services\_01
  \item Primary KC: FRACTIONS | Subtraction | subtracting unlike fractions with two different denominators not exceeding 12
  \item Secondary KC: FRACTIONS | Subtraction | subtracting fractions
  \item Topic: Services such as installation, maintenance, repairing, cleaning, laundry, e-commerce
  \item Grade: Primary 4
\end{itemize}

\textbf{Question}

A washing machine repairman fixed  \( \frac{5}{6} \)  of the washing machines in a laundry shop in the morning. In the afternoon, he fixed another  \( \frac{1}{4} \)  of the machines. 

How much more of the machines did he fix in the morning than in the afternoon?

\textbf{Solution}

To find out how much more of the machines the repairman fixed in the morning than in the afternoon, we need to subtract the fraction of machines fixed in the afternoon from the fraction fixed in the morning:

$
\frac{5}{6} - \frac{1}{4}
$

First, find the lowest common denominator (LCD) for 6 and 4, which is 12.

Convert both fractions to have a denominator of 12:

$
\frac{5}{6} = \frac{5\times2}{6\times2} = \frac{10}{12}
$
$
\frac{1}{4} = \frac{1\times3}{4\times3} = \frac{3}{12}
$

Now, subtract:

$
\frac{10}{12} - \frac{3}{12} = \frac{7}{12}
$

So, the repairman fixed \( \frac{7}{12} \) more of the machines in the morning than in the afternoon.

\section*{Question 43}
\textbf{Metadata}

\begin{itemize}
  \item Question ID: P4-FrSubU12\_P3-FrSmp\_GPT4.1\_Services\_01
  \item Primary KC: FRACTIONS | Subtraction | subtracting unlike fractions with two different denominators not exceeding 12
  \item Secondary KC: FRACTIONS | Simplifying | expressing a fraction in its simplest form
  \item Topic: Services such as installation, maintenance, repairing, cleaning, laundry, e-commerce
  \item Grade: Primary 4
\end{itemize}

\textbf{Question}

A cleaning service company received a request to clean 7/8 of a large office building. After working for a day, the cleaners managed to complete 2/3 of the cleaning job that was requested. What fraction of the cleaning job is still left for the cleaners to do? Express your answer in its simplest form.

\textbf{Solution}

Fraction of cleaning job left = Total requested cleaning - Completed cleaning

= \frac{7}{8} - \frac{2}{3}

To subtract, find the common denominator (24):
\frac{7}{8} = \frac{21}{24}, \quad \frac{2}{3} = \frac{16}{24}

So, \frac{21}{24} - \frac{16}{24} = \frac{5}{24}

Answer: \boxed{\frac{5}{24}} of the cleaning job is still left for the cleaners to do.

\section*{Question 44}
\textbf{Metadata}

\begin{itemize}
  \item Question ID: P4-DcAdd2d\_P4-DcCmp3d\_GPT4.1\_Services\_01
  \item Primary KC: DECIMALS | Addition | adding decimals (up to 2 decimal places)
  \item Secondary KC: DECIMALS | Comparison and ordering | comparing and ordering decimals up to 3 decimal places
  \item Topic: Services such as installation, maintenance, repairing, cleaning, laundry, e-commerce
  \item Grade: Primary 4
\end{itemize}

\textbf{Question}

A cleaning company offers three types of cleaning services. 

- Standard cleaning costs $12.50. 
- Deep cleaning costs $23.75. 
- Carpet cleaning costs $18.485.

Mrs Tan wants to book two services. She can either book Standard cleaning and Deep cleaning, or Deep cleaning and Carpet cleaning. 

(a) What is the total cost if she books Standard cleaning and Deep cleaning?

(b) What is the total cost if she books Deep cleaning and Carpet cleaning? (Round your final answer to 2 decimal places if needed.)

(c) Compare the totals in parts (a) and (b). Which combination costs more?

\textbf{Solution}

(a) Total cost for Standard cleaning and Deep cleaning:

$$
12.50 + 23.75 = 36.25
$$

(b) Total cost for Deep cleaning and Carpet cleaning:

$$
23.75 + 18.485 = 42.235
$$

Rounding $42.235$ to 2 decimal places: $42.24$

(c) Comparing $36.25$ and $42.24$:

Since $42.24 > 36.25$, booking Deep cleaning and Carpet cleaning costs more.

\section*{Question 45}
\textbf{Metadata}

\begin{itemize}
  \item Question ID: P4-DcAdd2d\_P4-DcCnv2Fr\_GPT4.1\_Services\_01
  \item Primary KC: DECIMALS | Addition | adding decimals (up to 2 decimal places)
  \item Secondary KC: DECIMALS | Conversion from decimals to fraction | expressing decimals as fractions
  \item Topic: Services such as installation, maintenance, repairing, cleaning, laundry, e-commerce
  \item Grade: Primary 4
\end{itemize}

\textbf{Question}

Mr. Lim had his air conditioner serviced. He paid \$42.75 for the servicing fee and \$8.50 for the cleaning solution. 

(a) How much did Mr. Lim pay in total for the service and cleaning solution?

(b) Express the amount paid for the cleaning solution (\$8.50) as a fraction in its simplest form.

\textbf{Solution}

(a) Total amount paid = Service fee + Cleaning solution cost\\
= \$42.75 + \$8.50\\
= \$51.25

(b) \$8.50 can be written as 8.50 or \( \frac{850}{100} \).\\
Simplifying, \( \frac{850}{100} = \frac{17}{2} \).\\
So, \$8.50 = \( \frac{17}{2} \) as a fraction.

\section*{Question 46}
\textbf{Metadata}

\begin{itemize}
  \item Question ID: P4-DcAdd2d\_P4-DcRnd3d\_GPT4.1\_Services\_01
  \item Primary KC: DECIMALS | Addition | adding decimals (up to 2 decimal places)
  \item Secondary KC: DECIMALS | Rounding | rounding decimals up to 3 decimal places to the nearest whole number, 1 decimal place and 2 decimal places 
  \item Topic: Services such as installation, maintenance, repairing, cleaning, laundry, e-commerce
  \item Grade: Primary 4
\end{itemize}

\textbf{Question}

A cleaning service charges \$12.45 for cleaning a living room and \$18.30 for cleaning a kitchen. Mrs Lee hires the service to clean both her living room and kitchen.

(a) How much does Mrs Lee need to pay in total for both rooms?

(b) Round the total amount to the nearest dollar.

(c) Round the total amount to 1 decimal place.

(d) Round the total amount to 2 decimal places.

\textbf{Solution}

(a) Total amount = \$12.45 + \$18.30 = \$30.75.

(b) Rounding \$30.75 to the nearest dollar:
The digit in the tenths place is 7, which is 5 or more, so we round up.
\$
\boxed{31}
\$

(c) Rounding \$30.75 to 1 decimal place:
The digit in the hundredths place is 5, which is 5 or more, so we round up the tenths place.
\$
\boxed{30.8}
\$

(d) Rounding \$30.75 to 2 decimal places:
There are already 2 decimal places, so it stays as:
\$
\boxed{30.75}
\$

\section*{Question 47}
\textbf{Metadata}

\begin{itemize}
  \item Question ID: P4-DcAdd2d\_P4-DcAdd2nd\_GPT4.1\_Services\_01
  \item Primary KC: DECIMALS | Addition | adding decimals (up to 2 decimal places)
  \item Secondary KC: DECIMALS | Addition | adding decimals
  \item Topic: Services such as installation, maintenance, repairing, cleaning, laundry, e-commerce
  \item Grade: Primary 4
\end{itemize}

\textbf{Question}

A washing machine repairman in Singapore charges \$25.50 for a basic repair service. If a customer also needs an extra part replaced that costs \$13.75, how much does the customer pay in total for the repair service and the extra part? 

If the customer also wants to add a cleaning service for \$9.40, what is the total amount the customer has to pay for all three services?

\textbf{Solution}

First, we add the cost of the basic repair service and the extra part:

$$
25.50 + 13.75 = 39.25
$$

So, the customer pays \$39.25 for the repair service and the extra part.

Next, add the cleaning service:
$$
39.25 + 9.40 = 48.65
$$

Therefore, the total amount the customer has to pay for all three services is \$48.65.

\section*{Question 48}
\textbf{Metadata}

\begin{itemize}
  \item Question ID: P4-DcSub2d\_P4-DcCmp3d\_GPT4.1\_Services\_01
  \item Primary KC: DECIMALS | Subtraction | subtracting decimals (up to 2 decimal places)
  \item Secondary KC: DECIMALS | Comparison and ordering | comparing and ordering decimals up to 3 decimal places
  \item Topic: Services such as installation, maintenance, repairing, cleaning, laundry, e-commerce
  \item Grade: Primary 4
\end{itemize}

\textbf{Question}

A laundry shop charged $15.80 for washing and $7.25 for ironing a set of clothes. After the service, May noticed that the bill listed the discounted price for washing as $13.95. 

(a) How much did May save on washing after the discount?

(b) Arrange the following amounts in order from the smallest to the largest: $7.25, $13.950, $15.800.

\textbf{Solution}

(a) May saved $15.80 - $13.95 = $1.85 on washing after the discount.

(b) Arrange from smallest to largest:
$7.25,
$13.950,
$15.800.

So the order is: $7.25,
$13.950,
$15.800.

\section*{Question 49}
\textbf{Metadata}

\begin{itemize}
  \item Question ID: P4-DcSub2d\_P4-DcCnv2Fr\_GPT4.1\_Services\_01
  \item Primary KC: DECIMALS | Subtraction | subtracting decimals (up to 2 decimal places)
  \item Secondary KC: DECIMALS | Conversion from decimals to fraction | expressing decimals as fractions
  \item Topic: Services such as installation, maintenance, repairing, cleaning, laundry, e-commerce
  \item Grade: Primary 4
\end{itemize}

\textbf{Question}

A cleaning service charges \$45.75 for a full house cleaning. If Mr. Lim has a discount coupon for \$12.40, how much does he need to pay after using the coupon? Express your answer in both decimal and as a fraction in its simplest form.

\textbf{Solution}

First, subtract the coupon amount from the total charge:

$
45.75 - 12.40 = 33.35
$

So, Mr. Lim needs to pay \$33.35.

To express 33.35 as a fraction:

33.35 can be written as \( \frac{3335}{100} \).

Simplify \( \frac{3335}{100} \):

Since 3335 and 100 have no common factors apart from 1, the fraction is already in its simplest form.

**Final answers:**

- Decimal: \$33.35
- Fraction: \( \frac{3335}{100} \)

\section*{Question 50}
\textbf{Metadata}

\begin{itemize}
  \item Question ID: P4-DcSub2d\_P4-DcRnd3d\_GPT4.1\_Services\_01
  \item Primary KC: DECIMALS | Subtraction | subtracting decimals (up to 2 decimal places)
  \item Secondary KC: DECIMALS | Rounding | rounding decimals up to 3 decimal places to the nearest whole number, 1 decimal place and 2 decimal places 
  \item Topic: Services such as installation, maintenance, repairing, cleaning, laundry, e-commerce
  \item Grade: Primary 4
\end{itemize}

\textbf{Question}

A washing service charges $15.80$ for cleaning a large mattress and $7.35$ for a small pillow. Mrs. Tan sends 1 large mattress and 1 small pillow for cleaning. 

(a) How much more did it cost to clean the mattress than the pillow? 

(b) The total cost before rounding is the sum of both items. Round the total cost to the nearest dollar.

\textbf{Solution}

(a) Amount difference: $15.80 - 7.35 = 8.45$. 

It costs $\$8.45$ more to clean the mattress than the pillow. 

(b) Total cost before rounding: $15.80 + 7.35 = 23.15$. 

Rounding $23.15$ to the nearest dollar: $23.15$ is closer to $23$ than to $24$, so the answer is $\$23$.

\section*{Question 51}
\textbf{Metadata}

\begin{itemize}
  \item Question ID: P4-DcSub2d\_P4-DcAdd2nd\_GPT4.1\_Services\_01
  \item Primary KC: DECIMALS | Subtraction | subtracting decimals (up to 2 decimal places)
  \item Secondary KC: DECIMALS | Addition | adding decimals
  \item Topic: Services such as installation, maintenance, repairing, cleaning, laundry, e-commerce
  \item Grade: Primary 4
\end{itemize}

\textbf{Question}

A washing machine repair service charges \$52.80 for fixing a machine. If the customer also wants to add a cleaning service that costs \$13.50, what is the total amount the customer has to pay? If the customer has already paid \$60, how much more does the customer need to pay to cover the entire bill?

\textbf{Solution}

First, add the repair cost and the cleaning service cost:

$
\$52.80 + \$13.50 = \$66.30
$

This is the total amount to be paid. The customer has already paid \$60. To find out how much more needs to be paid, subtract \$60 from \$66.30:

$
\$66.30 - \$60.00 = \$6.30
$

So, the customer needs to pay \$6.30 more.

\section*{Question 52}
\textbf{Metadata}

\begin{itemize}
  \item Question ID: P4-DcSub2d\_P4-DcSub2nd\_GPT4.1\_Services\_01
  \item Primary KC: DECIMALS | Subtraction | subtracting decimals (up to 2 decimal places)
  \item Secondary KC: DECIMALS | Subtraction | subtracting decimals
  \item Topic: Services such as installation, maintenance, repairing, cleaning, laundry, e-commerce
  \item Grade: Primary 4
\end{itemize}

\textbf{Question}

Amy hired a cleaning company to clean her house. The total charge for the cleaning service was $85.60. She paid a deposit of $23.45 when she made the booking. How much does Amy still need to pay after the deposit?

\textbf{Solution}

To find out how much Amy still needs to pay, subtract the deposit from the total charge:

$
\begin{align*}
\text{Amount left to pay} &= \$85.60 - \$23.45 \\
&= \$62.15
\end{align*}
$

Amy still needs to pay $62.15.

\section*{Question 53}
\textbf{Metadata}

\begin{itemize}
  \item Question ID: P4-DcMul2d1d\_P4-DcCmp3d\_GPT4.1\_Services\_01
  \item Primary KC: DECIMALS | Multiplication | multiplying decimals (up to 2 decimal places) by a 1-digit whole number
  \item Secondary KC: DECIMALS | Comparison and ordering | comparing and ordering decimals up to 3 decimal places
  \item Topic: Services such as installation, maintenance, repairing, cleaning, laundry, e-commerce
  \item Grade: Primary 4
\end{itemize}

\textbf{Question}

A cleaning company charges \$3.75 per hour to clean one room. Mrs Tan hires the company to clean 4 rooms in her house, and each room takes one hour to clean. 

After the cleaning, Mrs Tan compares the amounts she would need to pay if the company raised their price to \$3.852 per hour next month.

(a) How much does Mrs Tan pay this month for cleaning the 4 rooms?

(b) Calculate the total amount she would have to pay next month with the new price for 4 rooms.

(c) Which amount is greater, and by how much? Arrange the two total amounts in ascending order.

\textbf{Solution}

(a) The current price per room = $3.75. Number of rooms = 4.

Total amount this month = $3.75 \times 4 = $15.00

(b) Next month's price per room = $3.852. Number of rooms = 4.

Total amount next month = $3.852 \times 4 = $15.408

(c) Comparing the two amounts:

$15.00 < $15.408

The amount next month is greater.

Difference = $15.408 - $15.00 = $0.408

Arranged in ascending order:

$15.00, $15.408

So, Mrs Tan will pay $0.408 more next month if she hires the company at the new rate.

\section*{Question 54}
\textbf{Metadata}

\begin{itemize}
  \item Question ID: P4-DcMul2d1d\_P4-DcCnv2Fr\_GPT4.1\_Services\_01
  \item Primary KC: DECIMALS | Multiplication | multiplying decimals (up to 2 decimal places) by a 1-digit whole number
  \item Secondary KC: DECIMALS | Conversion from decimals to fraction | expressing decimals as fractions
  \item Topic: Services such as installation, maintenance, repairing, cleaning, laundry, e-commerce
  \item Grade: Primary 4
\end{itemize}

\textbf{Question}

A cleaning company charges \(3.75\) dollars per hour to clean a classroom. If Mrs. Lee hires the company for \(4\) hours, how much will she pay in total? Express your final answer both as a decimal and as a fraction in simplest form.

\textbf{Solution}

First, multiply the hourly rate by the number of hours:

$
3.75 \times 4 = 15.00
$

So, Mrs. Lee will pay \(15.00\) dollars in total.

Next, express \(15.00\) as a fraction:

- \(15.00 = 15\)
- As a fraction in simplest form, this is \(\boxed{15}\) or \(\frac{15}{1}\).

Thus, Mrs. Lee will pay \(15.00\) dollars (decimal) or \(\frac{15}{1}\) (fraction) in total.

\section*{Question 55}
\textbf{Metadata}

\begin{itemize}
  \item Question ID: P4-DcMul2d1d\_P4-DcRnd3d\_GPT4.1\_Services\_01
  \item Primary KC: DECIMALS | Multiplication | multiplying decimals (up to 2 decimal places) by a 1-digit whole number
  \item Secondary KC: DECIMALS | Rounding | rounding decimals up to 3 decimal places to the nearest whole number, 1 decimal place and 2 decimal places 
  \item Topic: Services such as installation, maintenance, repairing, cleaning, laundry, e-commerce
  \item Grade: Primary 4
\end{itemize}

\textbf{Question}

A cleaning company charges \$7.85 per hour to clean an apartment. Mrs. Lee hired the company for 4 hours to clean her home.

(a) How much did Mrs. Lee have to pay in total before rounding?

(b) Round the total amount Mrs. Lee has to pay to the nearest dollar.

(c) Round the total amount Mrs. Lee has to pay to 1 decimal place.

\textbf{Solution}

Let us solve each part step by step.

(a) Total amount before rounding:
$
\text{Total amount} = \$7.85 \times 4 = \$31.40
$
So, Mrs. Lee has to pay \$31.40 in total before rounding.

(b) Round \$31.40 to the nearest dollar:
The digit at the tenths place is 4 (in 0.40), which is less than 5. So, round down.
$
\$31.40 \approx \$31\text{ (to the nearest dollar)}
$

(c) Round \$31.40 to 1 decimal place:
The digit at the hundredths place is 0 (in 0.40), so no need to round up.
$
\$31.40 \approx \$31.4 \text{ (to 1 decimal place)}
$

**Answers:**
(a) \$31.40  (b) \$31  (c) \$31.4

\section*{Question 56}
\textbf{Metadata}

\begin{itemize}
  \item Question ID: P4-DcMul2d1d\_P4-DcAdd2nd\_GPT4.1\_Services\_01
  \item Primary KC: DECIMALS | Multiplication | multiplying decimals (up to 2 decimal places) by a 1-digit whole number
  \item Secondary KC: DECIMALS | Addition | adding decimals
  \item Topic: Services such as installation, maintenance, repairing, cleaning, laundry, e-commerce
  \item Grade: Primary 4
\end{itemize}

\textbf{Question}

A cleaning company charges \$2.50 to clean one window. Mrs Tan wants to clean 6 windows in her house. She also asks the company to clean 3 mirrors, and each mirror costs \$1.80 to clean. What is the total cost Mrs Tan has to pay to the cleaning company?

\textbf{Solution}

First, calculate the total cost to clean the windows: 

6 \times \$2.50 = \$15.00

Next, calculate the total cost to clean the mirrors: 

3 \times \$1.80 = \$5.40

Now, add the costs together to find the total amount Mrs Tan has to pay: 

\$15.00 + \$5.40 = \$20.40

So, Mrs Tan has to pay \$20.40 in total.

\section*{Question 57}
\textbf{Metadata}

\begin{itemize}
  \item Question ID: P4-DcMul2d1d\_P4-DcSub2nd\_GPT4.1\_Services\_01
  \item Primary KC: DECIMALS | Multiplication | multiplying decimals (up to 2 decimal places) by a 1-digit whole number
  \item Secondary KC: DECIMALS | Subtraction | subtracting decimals
  \item Topic: Services such as installation, maintenance, repairing, cleaning, laundry, e-commerce
  \item Grade: Primary 4
\end{itemize}

\textbf{Question}

A laundromat charges \$2.35 to wash one shirt. Mrs Tan wants to wash 4 shirts for her family. However, she uses a voucher that gives her a discount of \$1.50 on the total washing cost. How much does Mrs Tan need to pay after using the voucher?

\textbf{Solution}

First, we find the total cost to wash 4 shirts:

$
\$2.35 \times 4 = \$9.40
$

Next, we subtract the voucher discount:

$
\$9.40 - \$1.50 = \$7.90
$

So, Mrs Tan needs to pay \$7.90 after using the voucher.

\section*{Question 58}
\textbf{Metadata}

\begin{itemize}
  \item Question ID: P4-DcDiv2d1d\_P4-DcCmp3d\_GPT4.1\_Services\_01
  \item Primary KC: DECIMALS | Division | dividing decimals (up to 2 decimal places) by a 1-digit whole number
  \item Secondary KC: DECIMALS | Comparison and ordering | comparing and ordering decimals up to 3 decimal places
  \item Topic: Services such as installation, maintenance, repairing, cleaning, laundry, e-commerce
  \item Grade: Primary 4
\end{itemize}

\textbf{Question}

A laundry shop cleaned three bags of clothes. 

The total weight of the first bag was \(3.45\) kg and it was divided evenly among 5 washing machines. The total weight of the second bag was \(2.88\) kg and it was divided evenly among 4 washing machines. The total weight of the third bag was \(4.29\) kg and it was divided evenly among 6 washing machines. 

(a) Find the weight of clothes in each washing machine for each bag, giving your answers in decimals up to 2 decimal places. 

(b) Arrange the three bags' per-machine weights from heaviest to lightest.

\textbf{Solution}

(a)

First bag: \(3.45 \div 5 = 0.69\) kg per machine.

Second bag: \(2.88 \div 4 = 0.72\) kg per machine.

Third bag: \(4.29 \div 6 = 0.715\) kg per machine (rounded to 2 decimal places: 0.72 kg, but let’s compare using 3 decimals for ordering).

So, the weights are:

- First bag: 0.69 kg
- Second bag: 0.72 kg
- Third bag: 0.715 kg

(b)

Comparing: 0.72, 0.715, 0.69

From heaviest to lightest:

Second bag (0.720 kg), Third bag (0.715 kg), First bag (0.690 kg).

\section*{Question 59}
\textbf{Metadata}

\begin{itemize}
  \item Question ID: P4-DcDiv2d1d\_P4-DcCnv2Fr\_GPT4.1\_Services\_01
  \item Primary KC: DECIMALS | Division | dividing decimals (up to 2 decimal places) by a 1-digit whole number
  \item Secondary KC: DECIMALS | Conversion from decimals to fraction | expressing decimals as fractions
  \item Topic: Services such as installation, maintenance, repairing, cleaning, laundry, e-commerce
  \item Grade: Primary 4
\end{itemize}

\textbf{Question}

A washing service charges \$9.60 to clean 8 shirts. 

(a) How much does it cost to clean 1 shirt? Give your answer as a decimal. 

(b) Express the cost of cleaning 1 shirt as a fraction in its simplest form.

\textbf{Solution}

(a) To find the cost of cleaning 1 shirt: 

$
\text{Cost per shirt} = \frac{9.60}{8} = 1.20
$
So, it costs \$1.20 to clean 1 shirt.

(b) Express $1.20 as a fraction:

$
1.20 = \frac{120}{100} = \frac{6}{5}
$

Thus, the cost of cleaning 1 shirt as a fraction is \( \boxed{\frac{6}{5}} \) dollars.

\section*{Question 60}
\textbf{Metadata}

\begin{itemize}
  \item Question ID: P4-DcDiv2d1d\_P4-DcRnd3d\_GPT4.1\_Services\_01
  \item Primary KC: DECIMALS | Division | dividing decimals (up to 2 decimal places) by a 1-digit whole number
  \item Secondary KC: DECIMALS | Rounding | rounding decimals up to 3 decimal places to the nearest whole number, 1 decimal place and 2 decimal places 
  \item Topic: Services such as installation, maintenance, repairing, cleaning, laundry, e-commerce
  \item Grade: Primary 4
\end{itemize}

\textbf{Question}

A washing and ironing service charges a total of $38.76 to clean 6 shirts. Each shirt costs the same amount to clean. 

(a) What is the cost to clean 1 shirt? Give your answer correct to 2 decimal places. 

(b) Round your answer in (a) to the nearest whole number and also to 1 decimal place.

\textbf{Solution}

(a) Cost to clean 1 shirt:

$
\text{Cost of 1 shirt} = \frac{38.76}{6} = 6.46
$

So, the cost to clean 1 shirt is \$6.46 (to 2 decimal places).

(b) Rounding $6.46:

- To the nearest whole number: $6.46 \approx 6
- To the nearest 1 decimal place: $6.46 \approx 6.5

\section*{Question 61}
\textbf{Metadata}

\begin{itemize}
  \item Question ID: P4-DcDiv2d1d\_P4-DcAdd2nd\_GPT4.1\_Services\_01
  \item Primary KC: DECIMALS | Division | dividing decimals (up to 2 decimal places) by a 1-digit whole number
  \item Secondary KC: DECIMALS | Addition | adding decimals
  \item Topic: Services such as installation, maintenance, repairing, cleaning, laundry, e-commerce
  \item Grade: Primary 4
\end{itemize}

\textbf{Question}

A group of friends wants to have their air-conditioners serviced. The service company charges $38.40 for cleaning 1 air-conditioner. 

(a) If they want to split the cost of cleaning 1 air-conditioner equally among 4 friends, how much does each friend pay?

(b) After cleaning, each friend also buys a bottle of cleaning spray that costs $7.25 each. What is the total amount each friend spends in total, for both their share of the cleaning and the bottle of spray?


\textbf{Solution}

(a) To find how much each friend pays for cleaning 1 air-conditioner, divide $38.40 by 4:

$38.40 \div 4 = $9.60

Each friend pays $9.60 for the cleaning.

(b) Each friend buys a bottle of spray for $7.25. Add this to their cleaning share:

$9.60 + $7.25 = $16.85

Each friend spends $16.85 in total.

\section*{Question 62}
\textbf{Metadata}

\begin{itemize}
  \item Question ID: P4-DcDiv2d1d\_P4-DcSub2nd\_GPT4.1\_Services\_01
  \item Primary KC: DECIMALS | Division | dividing decimals (up to 2 decimal places) by a 1-digit whole number
  \item Secondary KC: DECIMALS | Subtraction | subtracting decimals
  \item Topic: Services such as installation, maintenance, repairing, cleaning, laundry, e-commerce
  \item Grade: Primary 4
\end{itemize}

\textbf{Question}

A cleaning company charges \$36.75 to clean 5 identical hotel rooms. 

(a) How much does it cost to clean 1 hotel room? 

(b) If a customer paid \$10.20 for cleaning a room and received \$1.95 as a refund, how much did the customer actually pay for the cleaning service?


\textbf{Solution}

(a) To find the cost to clean 1 hotel room, divide \$36.75 by 5:

$
\frac{36.75}{5} = 7.35
$

So, it costs \$7.35 to clean 1 hotel room.

(b) The customer initially paid \$10.20 and got refunded \$1.95. The actual amount paid is:

$
10.20 - 1.95 = 8.25
$

So, the customer actually paid \$8.25 for the cleaning service.

\section*{Question 63}
\textbf{Metadata}

\begin{itemize}
  \item Question ID: P5-FrAddMix\_P2-FrCmp\_GPT4.1\_Services\_01
  \item Primary KC: FRACTIONS | Addition | adding mixed numbers
  \item Secondary KC: FRACTIONS | Comparison and ordering | comparing and ordering fractions
  \item Topic: Services such as installation, maintenance, repairing, cleaning, laundry, e-commerce
  \item Grade: Primary 5
\end{itemize}

\textbf{Question}

Mr. Tan runs a cleaning service. On Monday, he cleaned 2 houses. For the first house, he spent $1\dfrac{3}{4}$ hours cleaning the kitchen and $2\dfrac{1}{2}$ hours cleaning the living room. For the second house, he spent $3\dfrac{1}{5}$ hours cleaning altogether. 

(a) How many hours did Mr. Tan spend in total cleaning the first house?

(b) Did Mr. Tan spend more time cleaning the first house or the second house? Arrange the total cleaning times for both houses from least to greatest.

\textbf{Solution}

(a) Hours spent cleaning the first house:
$1\dfrac{3}{4} + 2\dfrac{1}{2} = \frac{7}{4} + \frac{5}{2}$
$= \frac{7}{4} + \frac{10}{4}$
$= \frac{17}{4} = 4\dfrac{1}{4}$ hours

(b) Second house: $3\dfrac{1}{5} = \frac{16}{5}$ hours.
First house: $4\dfrac{1}{4} = \frac{17}{4}$ hours.

To compare:
Convert both to decimals or a common denominator.
$\frac{17}{4} = 4.25$
$\frac{16}{5} = 3.2$

So $4\dfrac{1}{4}$ > $3\dfrac{1}{5}$.

Arranged from least to greatest: $3\dfrac{1}{5}$ hours, $4\dfrac{1}{4}$ hours.

Mr. Tan spent more time cleaning the first house.

\section*{Question 64}
\textbf{Metadata}

\begin{itemize}
  \item Question ID: P5-FrAddMix\_P2-FrAdd2nd\_GPT4.1\_Services\_01
  \item Primary KC: FRACTIONS | Addition | adding mixed numbers
  \item Secondary KC: FRACTIONS | Addition | adding fractions
  \item Topic: Services such as installation, maintenance, repairing, cleaning, laundry, e-commerce
  \item Grade: Primary 5
\end{itemize}

\textbf{Question}

A washing machine repair technician spent $2\dfrac{1}{4}$ hours repairing one customer's machine in the morning. Later that day, he spent another $1\dfrac{2}{5}$ hours cleaning and maintaining another washing machine. If he also spent $\dfrac{3}{4}$ hours writing up service reports for both jobs, what is the total amount of time he spent on these three tasks during the day?

\textbf{Solution}

First, add the mixed numbers for the repair and cleaning time:  

$2\dfrac{1}{4} + 1\dfrac{2}{5} = (2 + 1) + (\dfrac{1}{4} + \dfrac{2}{5}) = 3 + (\dfrac{1}{4} + \dfrac{2}{5})$

To add $\dfrac{1}{4}$ and $\dfrac{2}{5}$, find a common denominator (20):

$\dfrac{1}{4} = \dfrac{5}{20}$ and $\dfrac{2}{5} = \dfrac{8}{20}$

$\dfrac{5}{20} + \dfrac{8}{20} = \dfrac{13}{20}$

So, total for first two tasks = $3\dfrac{13}{20}$ hours.

Now, add the service report time $\dfrac{3}{4}$:

$3\dfrac{13}{20} + \dfrac{3}{4}$

Convert $\dfrac{3}{4}$ to twentieths: $\dfrac{3}{4} = \dfrac{15}{20}$

$\dfrac{13}{20} + \dfrac{15}{20} = \dfrac{28}{20} = 1\dfrac{8}{20} = 1\dfrac{2}{5}$

Final total: $3 + 1\dfrac{2}{5} = 4\dfrac{2}{5}$

Therefore, the technician spent $\boxed{4\dfrac{2}{5}}$ hours in total on these three tasks during the day.

\section*{Question 65}
\textbf{Metadata}

\begin{itemize}
  \item Question ID: P5-FrAddMix\_P3-FrSmp\_GPT4.1\_Services\_01
  \item Primary KC: FRACTIONS | Addition | adding mixed numbers
  \item Secondary KC: FRACTIONS | Simplifying | expressing a fraction in its simplest form
  \item Topic: Services such as installation, maintenance, repairing, cleaning, laundry, e-commerce
  \item Grade: Primary 5
\end{itemize}

\textbf{Question}

A washing machine repairman spent $2\dfrac{1}{3}$ hours repairing a washing machine in a customer's house in the morning. In the afternoon, he spent another $1\dfrac{3}{4}$ hours cleaning and maintaining a different washing machine. 

(a) How many hours did he spend on both jobs altogether? 

(b) Express your answer in its simplest form.

\textbf{Solution}

First, we add the two mixed numbers:

$2\dfrac{1}{3} + 1\dfrac{3}{4}$ 

Convert both mixed numbers to improper fractions:

$2\dfrac{1}{3} = \dfrac{2 \times 3 + 1}{3} = \dfrac{7}{3}$

$1\dfrac{3}{4} = \dfrac{1 \times 4 + 3}{4} = \dfrac{7}{4}$

Now add them:

$\dfrac{7}{3} + \dfrac{7}{4}$

Find the common denominator (12):

$\dfrac{7 \times 4}{3 \times 4} + \dfrac{7 \times 3}{4 \times 3} = \dfrac{28}{12} + \dfrac{21}{12} = \dfrac{49}{12}$

Express $\dfrac{49}{12}$ as a mixed number:

$49 \div 12 = 4$ remainder $1$

So $\dfrac{49}{12} = 4\dfrac{1}{12}$

Therefore, the repairman spent $\boxed{4\dfrac{1}{12}\ \text{hours}}$ on both jobs altogether. The fraction $\dfrac{1}{12}$ is already in its simplest form.

\section*{Question 66}
\textbf{Metadata}

\begin{itemize}
  \item Question ID: P5-FrAddMix\_P5-FrCnv2Dc\_GPT4.1\_Services\_01
  \item Primary KC: FRACTIONS | Addition | adding mixed numbers
  \item Secondary KC: FRACTIONS | Conversion to decimals | expressing fractions as decimals
  \item Topic: Services such as installation, maintenance, repairing, cleaning, laundry, e-commerce
  \item Grade: Primary 5
\end{itemize}

\textbf{Question}

A washing machine technician serviced two different washing machines in the morning. The first machine took $1\dfrac{3}{4}$ hours to service, and the second machine took $2\dfrac{2}{5}$ hours. 

(a) How many hours did the technician spend in total on both washing machines? Give your answer as a mixed number.  

(b) Express the total time spent as a decimal, correct to 2 decimal places.

\textbf{Solution}

(a) First, add the mixed numbers:

$1\dfrac{3}{4} = 1 + \dfrac{3}{4}$

$2\dfrac{2}{5} = 2 + \dfrac{2}{5}$

Let's add the whole numbers:
$1 + 2 = 3$

Now add the fractions $\dfrac{3}{4} + \dfrac{2}{5}$.
Find a common denominator:
$\dfrac{3}{4} = \dfrac{15}{20}$
$\dfrac{2}{5} = \dfrac{8}{20}$

So, $\dfrac{15}{20} + \dfrac{8}{20} = \dfrac{23}{20}$

$\dfrac{23}{20}$ is equal to $1\dfrac{3}{20}$

Add to the whole number part:
$3 + 1\dfrac{3}{20} = 4\dfrac{3}{20}$

**Answer to (a):**
$4\dfrac{3}{20}$ hours

(b) Convert $4\dfrac{3}{20}$ to a decimal:
$4\dfrac{3}{20} = 4 + \dfrac{3}{20}$

$\dfrac{3}{20} = 0.15$

So, total time = $4 + 0.15 = 4.15$

**Rounded to 2 decimal places:**
$4.15$ hours

**Final answers:**
(a) $4\dfrac{3}{20}$ hours
(b) $4.15$ hours

\section*{Question 67}
\textbf{Metadata}

\begin{itemize}
  \item Question ID: P5-FrSubMix\_P2-FrCmp\_GPT4.1\_Services\_01
  \item Primary KC: FRACTIONS | Subtraction | subtracting mixed numbers
  \item Secondary KC: FRACTIONS | Comparison and ordering | comparing and ordering fractions
  \item Topic: Services such as installation, maintenance, repairing, cleaning, laundry, e-commerce
  \item Grade: Primary 5
\end{itemize}

\textbf{Question}

A cleaning company sends two workers, Adam and Bryan, to clean two similar-sized classrooms. Adam spends $2\dfrac{3}{4}$ hours cleaning his classroom, while Bryan spends $3\dfrac{1}{2}$ hours cleaning his classroom. 

(a) How much longer did Bryan spend cleaning than Adam?

(b) Arrange the following three times in order from the shortest to the longest:

$2\dfrac{3}{4}$ hours, $3\dfrac{1}{2}$ hours, $3\dfrac{1}{4}$ hours.

\textbf{Solution}

(a) Let's subtract the time Adam spent from the time Bryan spent:

$3\dfrac{1}{2} - 2\dfrac{3}{4}$

Convert to improper fractions:

$3\dfrac{1}{2} = \dfrac{7}{2}$

$2\dfrac{3}{4} = \dfrac{11}{4}$

We need a common denominator (4):

$\dfrac{7}{2} = \dfrac{14}{4}$

$\dfrac{14}{4} - \dfrac{11}{4} = \dfrac{3}{4}$

So, Bryan spent $\dfrac{3}{4}$ hour longer cleaning than Adam.

(b) Convert all times to improper fractions (with common denominator 4):

$2\dfrac{3}{4} = \dfrac{11}{4}$

$3\dfrac{1}{2} = \dfrac{7}{2} = \dfrac{14}{4}$

$3\dfrac{1}{4} = \dfrac{13}{4}$

Now, order:

$\dfrac{11}{4}$ (2$\dfrac{3}{4}$ hours), $\dfrac{13}{4}$ (3$\dfrac{1}{4}$ hours), $\dfrac{14}{4}$ (3$\dfrac{1}{2}$ hours).

So, the order from shortest to longest:

$2\dfrac{3}{4}$ hours, $3\dfrac{1}{4}$ hours, $3\dfrac{1}{2}$ hours.

\section*{Question 68}
\textbf{Metadata}

\begin{itemize}
  \item Question ID: P5-FrSubMix\_P2-FrAdd2nd\_GPT4.1\_Services\_01
  \item Primary KC: FRACTIONS | Subtraction | subtracting mixed numbers
  \item Secondary KC: FRACTIONS | Addition | adding fractions
  \item Topic: Services such as installation, maintenance, repairing, cleaning, laundry, e-commerce
  \item Grade: Primary 5
\end{itemize}

\textbf{Question}

A laundry shop washed \(4 \frac{1}{2}\) baskets of clothes in the morning. In the afternoon, they washed another \(2 \frac{2}{3}\) baskets of clothes. At the end of the day, they delivered \(5 \frac{1}{4}\) baskets of clothes back to their customers.\
\
(a) How many baskets of clothes did the laundry shop wash altogether?\
\
(b) After delivering the baskets back to the customers, how many baskets of clothes were left at the shop?\


\textbf{Solution}

(a) To find the total number of baskets washed, add the morning and afternoon counts:\\
$
4 \frac{1}{2} + 2 \frac{2}{3} = \frac{9}{2} + \frac{8}{3}
$
To add the fractions, get a common denominator (6):\
$
\frac{9}{2} = \frac{27}{6},\ \quad \frac{8}{3} = \frac{16}{6}\\
\frac{27}{6} + \frac{16}{6} = \frac{43}{6}
$
Now convert back to a mixed number:\
$
\frac{43}{6} = 7 \frac{1}{6}
$
So, they washed \(7 \frac{1}{6}\) baskets of clothes altogether.\
\
(b) To find how many baskets were left after delivery:\
$
7 \frac{1}{6} - 5 \frac{1}{4}
$
First, convert to improper fractions:\
$
7 \frac{1}{6} = \frac{43}{6},\ \quad 5 \frac{1}{4} = \frac{21}{4}
$
Get a common denominator (12):\
$
\frac{43}{6} = \frac{86}{12},\ \quad \frac{21}{4} = \frac{63}{12}\\
\frac{86}{12} - \frac{63}{12} = \frac{23}{12}\ (
1 \frac{11}{12}\text{ baskets}
)
$
So, \(1 \frac{11}{12}\) baskets of clothes were left at the shop after delivery.

\section*{Question 69}
\textbf{Metadata}

\begin{itemize}
  \item Question ID: P5-FrSubMix\_P2-FrSub2nd\_GPT4.1\_Services\_01
  \item Primary KC: FRACTIONS | Subtraction | subtracting mixed numbers
  \item Secondary KC: FRACTIONS | Subtraction | subtracting fractions
  \item Topic: Services such as installation, maintenance, repairing, cleaning, laundry, e-commerce
  \item Grade: Primary 5
\end{itemize}

\textbf{Question}

A cleaning company completed a spring cleaning for a customer’s flat. The total time taken was $3\dfrac{2}{3}$ hours. In the first part of the job, the company spent $1\dfrac{3}{8}$ hours cleaning the kitchen. They then spent $\dfrac{5}{12}$ hours cleaning the bathroom. How much time did the company spend cleaning the rest of the flat?

\textbf{Solution}

First, find the total time spent cleaning the kitchen and bathroom:

$1\dfrac{3}{8} + \dfrac{5}{12}$

Convert $1\dfrac{3}{8}$ to an improper fraction:
$1\dfrac{3}{8} = \dfrac{8}{8} + \dfrac{3}{8} = \dfrac{11}{8}$

Add $\dfrac{11}{8} + \dfrac{5}{12}$:
Find a common denominator (
LCM of 8 and 12 is 24):

$\dfrac{11}{8} = \dfrac{33}{24}$
$\dfrac{5}{12} = \dfrac{10}{24}$

$\dfrac{33}{24} + \dfrac{10}{24} = \dfrac{43}{24}$

Now, subtract the time spent on kitchen and bathroom from the total time:

$3\dfrac{2}{3} - \dfrac{43}{24}$

Convert $3\dfrac{2}{3}$ to improper fraction:
$3\dfrac{2}{3} = \dfrac{9}{3} + \dfrac{2}{3} = \dfrac{11}{3}$
Convert $\dfrac{11}{3}$ to base 24:
$\dfrac{11}{3} = \dfrac{88}{24}$

Now subtract:
$\dfrac{88}{24} - \dfrac{43}{24} = \dfrac{45}{24} = 1\dfrac{21}{24} = 1\dfrac{7}{8}$

Therefore, the company spent $1\dfrac{7}{8}$ hours cleaning the rest of the flat.

\section*{Question 70}
\textbf{Metadata}

\begin{itemize}
  \item Question ID: P5-FrSubMix\_P3-FrSmp\_GPT4.1\_Services\_01
  \item Primary KC: FRACTIONS | Subtraction | subtracting mixed numbers
  \item Secondary KC: FRACTIONS | Simplifying | expressing a fraction in its simplest form
  \item Topic: Services such as installation, maintenance, repairing, cleaning, laundry, e-commerce
  \item Grade: Primary 5
\end{itemize}

\textbf{Question}

A washing machine repairman spent $3\dfrac{3}{4}$ hours installing a new washing machine at a customer's home. After finishing the installation, he took $1\dfrac{5}{8}$ hours to clean up and check his tools. How many hours longer did the installation take than the clean-up and checking, and express your answer in its simplest form?

\textbf{Solution}

First, convert the mixed numbers to improper fractions:

Installation time: $3\dfrac{3}{4} = \dfrac{15}{4}$
Clean-up and checking time: $1\dfrac{5}{8} = \dfrac{13}{8}$

To subtract, convert $\dfrac{15}{4}$ to have a denominator of 8:
$\dfrac{15}{4} = \dfrac{30}{8}$

Subtract:
$\dfrac{30}{8} - \dfrac{13}{8} = \dfrac{17}{8}$

Express $\dfrac{17}{8}$ as a mixed number:
$\dfrac{17}{8} = 2\dfrac{1}{8}$

So, the installation took $2\dfrac{1}{8}$ hours longer than the clean-up and checking, and this fraction is already in its simplest form.

\section*{Question 71}
\textbf{Metadata}

\begin{itemize}
  \item Question ID: P5-FrSubMix\_P5-FrCnv2Dc\_GPT4.1\_Services\_01
  \item Primary KC: FRACTIONS | Subtraction | subtracting mixed numbers
  \item Secondary KC: FRACTIONS | Conversion to decimals | expressing fractions as decimals
  \item Topic: Services such as installation, maintenance, repairing, cleaning, laundry, e-commerce
  \item Grade: Primary 5
\end{itemize}

\textbf{Question}

A washing company delivered 4\( \frac{3}{5} \) litres of detergent to a laundry shop. After cleaning some clothes, the shop had only 2\( \frac{2}{5} \) litres of detergent left.

(a) How many litres of detergent were used?

(b) Express the amount of detergent used as a decimal.

\textbf{Solution}

(a) To find out how much detergent was used, subtract the amount left from the amount delivered:

$
4\frac{3}{5} - 2\frac{2}{5} = (4 - 2) + \left(\frac{3}{5} - \frac{2}{5}\right) = 2 + \frac{1}{5} = 2\frac{1}{5}
$

(b) \(2\frac{1}{5}\) as a decimal:
$
\frac{1}{5} = 0.2
$
So,
$
2\frac{1}{5} = 2.2
$

Final answers:
(a) \(2\frac{1}{5}\) litres
(b) 2.2 litres

\section*{Question 72}
\textbf{Metadata}

\begin{itemize}
  \item Question ID: P5-FrMulImN\_P2-FrCmp\_GPT4.1\_Services\_01
  \item Primary KC: FRACTIONS | Multiplication | multiplying a proper/improper fraction and a whole number
  \item Secondary KC: FRACTIONS | Comparison and ordering | comparing and ordering fractions
  \item Topic: Services such as installation, maintenance, repairing, cleaning, laundry, e-commerce
  \item Grade: Primary 5
\end{itemize}

\textbf{Question}

A cleaning service uses \(\frac{3}{4}\) litre of detergent to clean one classroom. (a) How much detergent will they use to clean 6 classrooms? (b) Another cleaning company uses \(\frac{2}{3}\) litre of detergent for each classroom. Who uses more detergent to clean 6 classrooms, and by how much?

\textbf{Solution}

(a) Amount of detergent needed for 6 classrooms:

$ 6 \times \frac{3}{4} = \frac{18}{4} = \frac{9}{2} = 4\frac{1}{2} \text{ litres} $

(b) The other company:

$ 6 \times \frac{2}{3} = \frac{12}{3} = 4 \text{ litres} $

\(4\frac{1}{2} \) litres (first company) vs \(4\) litres (second company). 
So, the first company uses more detergent.

Difference:
$
4\frac{1}{2} - 4 = \frac{9}{2} - \frac{8}{2} = \frac{1}{2} \text{ litre}$

Therefore, the first company uses \(\frac{1}{2}\) litre more detergent than the second company to clean 6 classrooms.

\section*{Question 73}
\textbf{Metadata}

\begin{itemize}
  \item Question ID: P5-FrMulImN\_P2-FrAdd2nd\_GPT4.1\_Services\_01
  \item Primary KC: FRACTIONS | Multiplication | multiplying a proper/improper fraction and a whole number
  \item Secondary KC: FRACTIONS | Addition | adding fractions
  \item Topic: Services such as installation, maintenance, repairing, cleaning, laundry, e-commerce
  \item Grade: Primary 5
\end{itemize}

\textbf{Question}

A cleaning company charges $\frac{2}{3}$ of an hour to clean one office room. If they need to clean 5 identical office rooms, how many hours will they spend cleaning all the rooms?

After cleaning the rooms, they also spend another $\frac{1}{4}$ hour sweeping the corridor and $\frac{1}{6}$ hour picking up rubbish. What is the total time, in hours, that the company spent on the job?

\textbf{Solution}

First, calculate the time spent cleaning 5 rooms:

$\frac{2}{3} \times 5 = \frac{2 \times 5}{3} = \frac{10}{3}$ hours$ = 3\frac{1}{3}$ hours$

Next, add the time spent on sweeping and picking up rubbish:

Sweeping: $\frac{1}{4}$ hour
Picking up rubbish: $\frac{1}{6}$ hour

Find a common denominator to add $\frac{1}{4} + \frac{1}{6}$.

The lowest common denominator is 12.
$\frac{1}{4} = \frac{3}{12}$
$\frac{1}{6} = \frac{2}{12}$
$\frac{3}{12} + \frac{2}{12} = \frac{5}{12}$

Total time spent:
$\frac{10}{3} + \frac{5}{12}$

Convert $\frac{10}{3}$ to twelfths: $\frac{10}{3} = \frac{40}{12}$
$\frac{40}{12} + \frac{5}{12} = \frac{45}{12}$

Simplify $\frac{45}{12}$:

$\frac{45}{12} = 3 \dfrac{9}{12} = 3\dfrac{3}{4}$ hours

Therefore, the company spent a total of $3\dfrac{3}{4}$ hours on the job.

\section*{Question 74}
\textbf{Metadata}

\begin{itemize}
  \item Question ID: P5-FrMulImN\_P2-FrSub2nd\_GPT4.1\_Services\_01
  \item Primary KC: FRACTIONS | Multiplication | multiplying a proper/improper fraction and a whole number
  \item Secondary KC: FRACTIONS | Subtraction | subtracting fractions
  \item Topic: Services such as installation, maintenance, repairing, cleaning, laundry, e-commerce
  \item Grade: Primary 5
\end{itemize}

\textbf{Question}

A washing company uses $\frac{4}{5}$ litre of detergent to clean one large carpet. Last week, they received an order to clean 7 large carpets at a hotel. 

(a) What is the total amount of detergent needed to clean all 7 carpets?

(b) After cleaning 7 carpets, they found that they already had $\frac{3}{10}$ litre of detergent in their container. How much more detergent must they add to have the total amount needed for 7 carpets?


\textbf{Solution}

(a) Amount of detergent needed for 1 carpet $= \frac{4}{5}$ litre

Amount needed for 7 carpets $= 7 \times \frac{4}{5} = \frac{28}{5}$ litres $= 5\frac{3}{5}$ litres.

(b) Amount of detergent still needed $= \frac{28}{5} - \frac{3}{10}$ litres

First, express both fractions with a common denominator:

$\frac{28}{5} = \frac{56}{10}$

$\frac{56}{10} - \frac{3}{10} = \frac{53}{10}$ litres $= 5\frac{3}{10}$ litres.

So, they must add $5\frac{3}{10}$ litres more detergent to have enough for 7 carpets.

\section*{Question 75}
\textbf{Metadata}

\begin{itemize}
  \item Question ID: P5-FrMulImN\_P3-FrSmp\_GPT4.1\_Services\_01
  \item Primary KC: FRACTIONS | Multiplication | multiplying a proper/improper fraction and a whole number
  \item Secondary KC: FRACTIONS | Simplifying | expressing a fraction in its simplest form
  \item Topic: Services such as installation, maintenance, repairing, cleaning, laundry, e-commerce
  \item Grade: Primary 5
\end{itemize}

\textbf{Question}

In a cleaning company, each employee can clean \( \frac{3}{4} \) of a room in 1 hour. If Jonathan works for 8 hours, how many rooms will he clean in total? Express your answer as a fraction in its simplest form.

\textbf{Solution}

Jonathan cleans \( \frac{3}{4} \) of a room in 1 hour. In 8 hours, he will clean:

$
8 \times \frac{3}{4} = \frac{24}{4} = 6
$

So, Jonathan will clean \( 6 \) rooms in total.

(If the answer before simplification was a non-whole fraction, we would further express it in its simplest form. In this case, the answer is already a whole number.)

\section*{Question 76}
\textbf{Metadata}

\begin{itemize}
  \item Question ID: P5-FrMulImN\_P5-FrCnv2Dc\_GPT4.1\_Services\_01
  \item Primary KC: FRACTIONS | Multiplication | multiplying a proper/improper fraction and a whole number
  \item Secondary KC: FRACTIONS | Conversion to decimals | expressing fractions as decimals
  \item Topic: Services such as installation, maintenance, repairing, cleaning, laundry, e-commerce
  \item Grade: Primary 5
\end{itemize}

\textbf{Question}

A cleaning company charges $\frac{3}{4}$ of a dollar per shirt for laundry service. If Mrs. Ng sends 16 shirts to be cleaned, 

(a) How much does she have to pay, in dollars, as a fraction?

(b) Express the amount found in (a) as a decimal.

\textbf{Solution}

(a) Amount to pay as a fraction:

$= 16 \times \frac{3}{4}$

$= \frac{16 \times 3}{4}$

$= \frac{48}{4}$

$= 12$

So Mrs. Ng has to pay $12 as a fraction (or just 12 dollars).

(b) Expressing the fraction as a decimal:

$12$ as a decimal is $12.00$

Therefore, Mrs. Ng has to pay $12.00 in total.

\section*{Question 77}
\textbf{Metadata}

\begin{itemize}
  \item Question ID: P5-FrMulImN\_P5-FrMul2nd\_GPT4.1\_Services\_01
  \item Primary KC: FRACTIONS | Multiplication | multiplying a proper/improper fraction and a whole number
  \item Secondary KC: FRACTIONS | Multiplication | fraction multiplication
  \item Topic: Services such as installation, maintenance, repairing, cleaning, laundry, e-commerce
  \item Grade: Primary 5
\end{itemize}

\textbf{Question}

A cleaning company is hired to clean 5 offices. Each office is cleaned by using $\frac{3}{4}$ of a bottle of cleaning solution. 

(a) How many bottles of cleaning solution are needed to clean all 5 offices?

After cleaning the offices, the company also uses $\frac{2}{3}$ of the amount of solution used earlier to clean the pantry. 

(b) How many bottles of cleaning solution are used for cleaning the pantry?

\textbf{Solution}

(a) Each office uses $\frac{3}{4}$ bottle, and there are 5 offices.

Total solution needed: $5 \times \frac{3}{4} = \frac{15}{4}$ bottles = $3\frac{3}{4}$ bottles.

(b) The pantry uses $\frac{2}{3}$ of what was used for the offices.

Amount for pantry:

$\frac{2}{3} \times \frac{15}{4} = \frac{2 \times 15}{3 \times 4} = \frac{30}{12} = \frac{5}{2}$ bottles = $2\frac{1}{2}$ bottles.

**Final answers:**

(a) $3\frac{3}{4}$ bottles are needed for the offices.

(b) $2\frac{1}{2}$ bottles are used for the pantry.

\section*{Question 78}
\textbf{Metadata}

\begin{itemize}
  \item Question ID: P5-FrMulPIm\_P2-FrCmp\_GPT4.1\_Services\_01
  \item Primary KC: FRACTIONS | Multiplication | multiplying a proper fraction and a proper/improper fractions
  \item Secondary KC: FRACTIONS | Comparison and ordering | comparing and ordering fractions
  \item Topic: Services such as installation, maintenance, repairing, cleaning, laundry, e-commerce
  \item Grade: Primary 5
\end{itemize}

\textbf{Question}


A laundry service received two different orders. For the first order, $\frac{3}{4}$ of the total load of clothes required washing, and for the second order, $\frac{5}{6}$ of the total load required washing. If both orders each had the same total load of 48 kg, find:

(a) How many kilograms of clothes needed washing in each order?

(b) Compare the two amounts and state which order required washing a greater mass of clothes, and by how many kilograms.


\textbf{Solution}


(a) For the first order:

$\frac{3}{4} \times 48 = 36$ kg of clothes needed washing.

For the second order:

$\frac{5}{6} \times 48 = 40$ kg of clothes needed washing.

(b) Comparing the two results, $40$ kg (second order) is more than $36$ kg (first order).

Difference $= 40 - 36 = 4$ kg.

The second order required washing a greater mass of clothes by 4 kg.


\section*{Question 79}
\textbf{Metadata}

\begin{itemize}
  \item Question ID: P5-FrMulPIm\_P2-FrAdd2nd\_GPT4.1\_Services\_01
  \item Primary KC: FRACTIONS | Multiplication | multiplying a proper fraction and a proper/improper fractions
  \item Secondary KC: FRACTIONS | Addition | adding fractions
  \item Topic: Services such as installation, maintenance, repairing, cleaning, laundry, e-commerce
  \item Grade: Primary 5
\end{itemize}

\textbf{Question}

A cleaning company uses $\frac{3}{4}$ of a bottle of cleaning solution for each house they clean. On Monday, they cleaned $2\frac{1}{2}$ houses in the morning, and in the afternoon they used $\frac{1}{3}$ of a bottle more. 

How many bottles of cleaning solution did they use in total on Monday?

(Give your answer as a fraction in its simplest form.)

\textbf{Solution}

First, calculate the amount of cleaning solution used for the houses cleaned in the morning:

Number of houses cleaned = $2\frac{1}{2} = \frac{5}{2}$

Amount used = $\frac{3}{4} \times \frac{5}{2} = \frac{3\times5}{4\times2} = \frac{15}{8}$

In the afternoon, they used $\frac{1}{3}$ more bottles.

Total used = $\frac{15}{8} + \frac{1}{3}$

To add these, find a common denominator (24):

$\frac{15}{8} = \frac{45}{24}$

$\frac{1}{3} = \frac{8}{24}$

Total used = $\frac{45}{24} + \frac{8}{24} = \frac{53}{24}$

Therefore, they used $\frac{53}{24}$ bottles of cleaning solution in total on Monday.

\section*{Question 80}
\textbf{Metadata}

\begin{itemize}
  \item Question ID: P5-FrMulPIm\_P2-FrSub2nd\_GPT4.1\_Services\_01
  \item Primary KC: FRACTIONS | Multiplication | multiplying a proper fraction and a proper/improper fractions
  \item Secondary KC: FRACTIONS | Subtraction | subtracting fractions
  \item Topic: Services such as installation, maintenance, repairing, cleaning, laundry, e-commerce
  \item Grade: Primary 5
\end{itemize}

\textbf{Question}

A handyman is cleaning two types of windows in an office building. He spends $\frac{3}{4}$ of an hour cleaning each small window. On one floor, there are $\frac{5}{2}$ times as many large windows as there are small windows. If he spends the same amount of time on each large window as on each small window, and the total time he spends cleaning both types of windows is $8\frac{1}{2}$ hours, how much longer does he spend cleaning the large windows than the small windows on that floor?

\textbf{Solution}

Let the number of small windows be $x$.\
Time spent cleaning small windows $= x \times \frac{3}{4}$ hours.\
Number of large windows $= \frac{5}{2}x.$\
Time spent cleaning large windows $= \frac{5}{2}x \times \frac{3}{4} = x \times \frac{5}{2} \times \frac{3}{4} = x \times \frac{15}{8}$ hours.\
\
Total time spent is $x \times \frac{3}{4} + x \times \frac{15}{8} = 8\frac{1}{2}$ hours.\
$= x \left(\frac{3}{4} + \frac{15}{8}\right) = 8\frac{1}{2}$\
$= x \left(\frac{6}{8} + \frac{15}{8}\right) = 8\frac{1}{2}$\
$= x \times \frac{21}{8} = \frac{17}{2}$\
\
So $x = \frac{17}{2} \div \frac{21}{8} = \frac{17}{2} \times \frac{8}{21} = \frac{17 \times 8}{2 \times 21} = \frac{136}{42} = \frac{68}{21}$\
\
Time spent cleaning large windows $= x \times \frac{15}{8} = \frac{68}{21} \times \frac{15}{8} = \frac{68 \times 15}{21 \times 8} = \frac{1020}{168} = \frac{255}{42}$ hours.\
\
Time spent cleaning small windows $= x \times \frac{3}{4} = \frac{68}{21} \times \frac{3}{4} = \frac{68 \times 3}{21 \times 4} = \frac{204}{84} = \frac{51}{21}$ hours.\
\
Difference $= \frac{255}{42} - \frac{51}{21} = \frac{255}{42} - \frac{102}{42} = \frac{153}{42} = \frac{51}{14}$ hours.\
\
Therefore, the handyman spends $\boxed{\frac{51}{14}}$ hours longer cleaning the large windows than the small windows.

\section*{Question 81}
\textbf{Metadata}

\begin{itemize}
  \item Question ID: P5-FrMulPIm\_P3-FrSmp\_GPT4.1\_Services\_01
  \item Primary KC: FRACTIONS | Multiplication | multiplying a proper fraction and a proper/improper fractions
  \item Secondary KC: FRACTIONS | Simplifying | expressing a fraction in its simplest form
  \item Topic: Services such as installation, maintenance, repairing, cleaning, laundry, e-commerce
  \item Grade: Primary 5
\end{itemize}

\textbf{Question}

A laundry shop uses \( \frac{2}{5} \) of a bottle of detergent to wash a load of uniform clothes. Yesterday, they washed \( \frac{7}{3} \) loads for a large customer order. 

(a) How much of the detergent bottle did they use in total? Express your answer as a fraction.

(b) Express your answer in (a) in its simplest form.

\textbf{Solution}

(a) To find the total amount of detergent used, multiply:$ \frac{2}{5} \times \frac{7}{3} = \frac{2 \times 7}{5 \times 3} = \frac{14}{15} \text{ (of a bottle)} $

(b) The fraction \( \frac{14}{15} \) is already in its simplest form because 14 and 15 have no common factors other than 1.

**Final answer:** The laundry shop used \( \frac{14}{15} \) of a detergent bottle in total.

\section*{Question 82}
\textbf{Metadata}

\begin{itemize}
  \item Question ID: P5-FrMulPIm\_P5-FrCnv2Dc\_GPT4.1\_Services\_01
  \item Primary KC: FRACTIONS | Multiplication | multiplying a proper fraction and a proper/improper fractions
  \item Secondary KC: FRACTIONS | Conversion to decimals | expressing fractions as decimals
  \item Topic: Services such as installation, maintenance, repairing, cleaning, laundry, e-commerce
  \item Grade: Primary 5
\end{itemize}

\textbf{Question}

Mrs Lim hires a carpet cleaning service to clean her home. The company charges $120 to clean the entire house. During the cleaning, they managed to clean $\frac{3}{5}$ of the house in the morning and finished $\frac{3}{4}$ of the cleaned part before lunch. 

(a) What fraction of the whole house was cleaned before lunch?

(b) Express your answer in part (a) as a decimal, correct to 2 decimal places. 

(c) How much did Mrs Lim have to pay for the part of the house cleaned before lunch? Express your answer in dollars and cents, correct to 2 decimal places.

\textbf{Solution}

(a) The fraction of the whole house cleaned before lunch is $\frac{3}{5} \times \frac{3}{4} = \frac{9}{20}$.

(b) $\frac{9}{20} = 0.45$ (correct to 2 decimal places).

(c) Amount to pay = $0.45 \times 120 = 54$.
So, Mrs Lim has to pay $\$54.00$ for the part of the house cleaned before lunch.

\section*{Question 83}
\textbf{Metadata}

\begin{itemize}
  \item Question ID: P5-FrMulPIm\_P5-FrMul2nd\_GPT4.1\_Services\_01
  \item Primary KC: FRACTIONS | Multiplication | multiplying a proper fraction and a proper/improper fractions
  \item Secondary KC: FRACTIONS | Multiplication | fraction multiplication
  \item Topic: Services such as installation, maintenance, repairing, cleaning, laundry, e-commerce
  \item Grade: Primary 5
\end{itemize}

\textbf{Question}

A cleaner was hired to clean all the windows of a school. On the first day, she managed to clean $\frac{3}{5}$ of the windows. The following day, she cleaned $\frac{4}{7}$ of the remaining windows. 

What fraction of all the windows did she clean on the second day?

\textbf{Solution}

Let the total number of windows be 1.

Fraction cleaned on the first day: $\frac{3}{5}$. 

Fraction remaining after the first day: $1 - \frac{3}{5} = \frac{2}{5}$. 

On the second day, she cleaned $\frac{4}{7}$ of the remaining windows:

$\frac{4}{7} \times \frac{2}{5} = \frac{4 \times 2}{7 \times 5} = \frac{8}{35}$.

So, the cleaner cleaned $\frac{8}{35}$ of all the windows on the second day.

\section*{Question 84}
\textbf{Metadata}

\begin{itemize}
  \item Question ID: P5-FrMulImIm\_P2-FrCmp\_GPT4.1\_Services\_01
  \item Primary KC: FRACTIONS | Multiplication | multiplying two improper fractions
  \item Secondary KC: FRACTIONS | Comparison and ordering | comparing and ordering fractions
  \item Topic: Services such as installation, maintenance, repairing, cleaning, laundry, e-commerce
  \item Grade: Primary 5
\end{itemize}

\textbf{Question}

A cleaning service offers two types of large mop packages for offices. 

Package A uses $\frac{7}{3}$ litres of cleaning solution for each room and cleans $\frac{10}{4}$ rooms per floor in a building. Package B uses $\frac{5}{2}$ litres of solution for each room and cleans $\frac{11}{4}$ rooms per floor. 

For one floor, calculate the total amount of cleaning solution used by each package. Then, compare the total amounts and state which package uses more cleaning solution per floor.

\textbf{Solution}

For Package A:
$
\text{Total solution used} = \frac{7}{3} \times \frac{10}{4} = \frac{7 \times 10}{3 \times 4} = \frac{70}{12} = \frac{35}{6} \text{ litres}
$

For Package B:
$
\text{Total solution used} = \frac{5}{2} \times \frac{11}{4} = \frac{5 \times 11}{2 \times 4} = \frac{55}{8} \text{ litres}
$

To compare $\frac{35}{6}$ and $\frac{55}{8}$, find a common denominator:

$\frac{35}{6} = \frac{35 \times 4}{6 \times 4} = \frac{140}{24}$

$\frac{55}{8} = \frac{55 \times 3}{8 \times 3} = \frac{165}{24}$

Since $140 < 165$, $\frac{35}{6} < \frac{55}{8}$.

Therefore, **Package B uses more cleaning solution per floor.**

\section*{Question 85}
\textbf{Metadata}

\begin{itemize}
  \item Question ID: P5-FrMulImIm\_P2-FrAdd2nd\_GPT4.1\_Services\_01
  \item Primary KC: FRACTIONS | Multiplication | multiplying two improper fractions
  \item Secondary KC: FRACTIONS | Addition | adding fractions
  \item Topic: Services such as installation, maintenance, repairing, cleaning, laundry, e-commerce
  \item Grade: Primary 5
\end{itemize}

\textbf{Question}

A cleaning company uses $\frac{9}{4}$ litres of cleaning solution to clean each floor of an office building. If they clean $\frac{7}{3}$ floors in one morning, how many litres of cleaning solution do they use?

Later, the manager decides to clean $\frac{5}{6}$ more litres on top of what was already used in the morning. What is the total number of litres of cleaning solution used after adding the extra amount?

\textbf{Solution}

First, calculate how much cleaning solution is used in the morning:

$\frac{9}{4} \times \frac{7}{3} = \frac{9 \times 7}{4 \times 3} = \frac{63}{12}$ litres

Simplify $\frac{63}{12}$ by dividing both numerator and denominator by 3:

$\frac{63 \div 3}{12 \div 3} = \frac{21}{4}$ litres

Next, add the extra $\frac{5}{6}$ litres:

Find a common denominator (LCD of 4 and 6 is 12):
$\frac{21}{4} = \frac{21 \times 3}{4 \times 3} = \frac{63}{12}$
$\frac{5}{6} = \frac{5 \times 2}{6 \times 2} = \frac{10}{12}$

Add both fractions:
$\frac{63}{12} + \frac{10}{12} = \frac{73}{12}$ litres

$\frac{73}{12}$ as a mixed number:
$73 \div 12 = 6$ remainder $1$, so $6\frac{1}{12}$ litres.

**Final Answer:**
The cleaning company used a total of $6\frac{1}{12}$ litres of cleaning solution.

\section*{Question 86}
\textbf{Metadata}

\begin{itemize}
  \item Question ID: P5-FrMulImIm\_P2-FrSub2nd\_GPT4.1\_Services\_01
  \item Primary KC: FRACTIONS | Multiplication | multiplying two improper fractions
  \item Secondary KC: FRACTIONS | Subtraction | subtracting fractions
  \item Topic: Services such as installation, maintenance, repairing, cleaning, laundry, e-commerce
  \item Grade: Primary 5
\end{itemize}

\textbf{Question}

Sarah runs a small cleaning service. On Saturday, she used $\frac{9}{4}$ litres of cleaning solution to clean an office that was $\frac{7}{3}$ times larger than a regular room. 

(a) How many litres of cleaning solution would she use to clean the regular-sized room? 

(b) If on Sunday, she used $2\frac{1}{6}$ litres less cleaning solution than what she used for the regular-sized room, how many litres of cleaning solution did she use on Sunday?

\textbf{Solution}

Let $x$ be the amount of cleaning solution needed for a regular-sized room. 

Since the office is $\frac{7}{3}$ times larger than a regular room, Sarah used $\frac{7}{3}$ times as much cleaning solution:

$\frac{7}{3} \times x = \frac{9}{4}$

So,

$x = \frac{9}{4} \div \frac{7}{3} = \frac{9}{4} \times \frac{3}{7} = \frac{27}{28}$ litres.

(a) **She would use $\frac{27}{28}$ litres of cleaning solution for a regular-sized room.**

On Sunday, she used $2\frac{1}{6}$ litres less than this amount.

$2\frac{1}{6} = \frac{13}{6}$

So, the amount used on Sunday:

$\frac{27}{28} - \frac{13}{6}$

First, find a common denominator (the LCM of 28 and 6 is 84):

$\frac{27}{28} = \frac{81}{84}$

$\frac{13}{6} = \frac{182}{84}$

Now subtract:

$\frac{81}{84} - \frac{182}{84} = -\frac{101}{84}$

Since the answer is negative, this means $2\frac{1}{6}$ litres is more than the amount needed for a regular room, and so she did not use any cleaning solution on Sunday (which is not realistic) or the question asks for the difference in the other direction.

But let's correct: She used $2\frac{1}{6}$ litres less than what she used for the regular-sized room, so:

Regular room: $\frac{27}{28}$ litres

Amount used on Sunday: $\frac{27}{28} - 2\frac{1}{6}$

Convert both to improper fractions with common denominators:

$2\frac{1}{6} = \frac{13}{6} = \frac{182}{84}$

$\frac{27}{28} = \frac{81}{84}$

$\frac{81}{84} - \frac{182}{84} = -\frac{101}{84}$

Since subtracting a larger number from a smaller gives a negative answer, check if the problem should be: 'She used $2\frac{1}{6}$ litres less THAN what she used for the regular-sized room,' or 'She used $2\frac{1}{6}$ litres in total, which is less than what she used for the regular-sized room.'

If the latter, then: $\frac{27}{28} - 2\frac{1}{6} = $ negative so not possible. 

Alternatively, if the question is: "On Sunday, she used $2\frac{1}{6}$ litres of cleaning solution. How much less is this than the amount she would use for a regular room?"

Then:

$\frac{27}{28} - 2\frac{1}{6} = \frac{81}{84} - \frac{182}{84} = -\frac{101}{84}$ (again negative)

Therefore, let's adjust (b):

(b) On Sunday, Sarah used $2\frac{1}{6}$ litres of cleaning solution. How much less is this than the amount she would need for a regular-sized room?

$\frac{27}{28} - 2\frac{1}{6} = \frac{81}{84} - \frac{182}{84} = -\frac{101}{84}$

A negative answer means $2\frac{1}{6}$ litres is more than $\frac{27}{28}$ litres.

Alternatively, how much more is $2\frac{1}{6}$ litres than $\frac{27}{28}$ litres?

$2\frac{1}{6} - \frac{27}{28} = \frac{182}{84} - \frac{81}{84} = \frac{101}{84}$ litres.

Therefore, $2\frac{1}{6}$ litres is $\frac{101}{84}$ litres more than $\frac{27}{28}$ litres.

So the final questions should be:

(a) $\frac{27}{28}$ litres

(b) $2\frac{1}{6}$ litres is $\frac{101}{84}$ litres more than $\frac{27}{28}$ litres.

\section*{Question 87}
\textbf{Metadata}

\begin{itemize}
  \item Question ID: P5-FrMulImIm\_P3-FrSmp\_GPT4.1\_Services\_01
  \item Primary KC: FRACTIONS | Multiplication | multiplying two improper fractions
  \item Secondary KC: FRACTIONS | Simplifying | expressing a fraction in its simplest form
  \item Topic: Services such as installation, maintenance, repairing, cleaning, laundry, e-commerce
  \item Grade: Primary 5
\end{itemize}

\textbf{Question}

A cleaning company uses bottles of cleaning solution that are each $\frac{7}{3}$ litres. For a special cleaning project, the company needs to prepare $\frac{5}{2}$ bottles of solution. 

(a) What is the total volume of cleaning solution needed for the project (in litres)? Express your answer as a fraction.

(b) Express your answer from (a) in its simplest form.

\textbf{Solution}

(a) Each bottle contains $\frac{7}{3}$ litres, and $\frac{5}{2}$ bottles are needed.

So, the total volume needed = $\frac{7}{3} \times \frac{5}{2} = \frac{35}{6}$ litres.

(b) $\frac{35}{6}$ is already in its simplest form because 35 and 6 have no common factors other than 1.

Final Answer: $\frac{35}{6}$ litres.

\section*{Question 88}
\textbf{Metadata}

\begin{itemize}
  \item Question ID: P5-FrMulImIm\_P5-FrCnv2Dc\_GPT4.1\_Services\_01
  \item Primary KC: FRACTIONS | Multiplication | multiplying two improper fractions
  \item Secondary KC: FRACTIONS | Conversion to decimals | expressing fractions as decimals
  \item Topic: Services such as installation, maintenance, repairing, cleaning, laundry, e-commerce
  \item Grade: Primary 5
\end{itemize}

\textbf{Question}

A cleaning service uses $\dfrac{9}{4}$ litres of cleaning solution for each large room and $\dfrac{7}{3}$ litres for each small room. If on Saturday, a customer booked the cleaning of one large room and one small room, how many litres of cleaning solution were used in total? Express your answer as a decimal.


\textbf{Solution}

First, we need to find the total amount of cleaning solution used by adding the amount used for one large room and one small room:

$\dfrac{9}{4} + \dfrac{7}{3}$

Find a common denominator (12):

$\dfrac{9}{4} = \dfrac{27}{12}$

$\dfrac{7}{3} = \dfrac{28}{12}$

Add:

$\dfrac{27}{12} + \dfrac{28}{12} = \dfrac{55}{12}$

Now, convert $\dfrac{55}{12}$ to a decimal:

$55 \div 12 = 4.583...$

Correct to 2 decimal places (as is common for litres):

$\dfrac{55}{12} = 4.58$ litres (rounded to 2 decimal places)

**Answer: 4.58 litres of cleaning solution were used in total.**


\section*{Question 89}
\textbf{Metadata}

\begin{itemize}
  \item Question ID: P5-FrMulImIm\_P5-FrMul2nd\_GPT4.1\_Services\_01
  \item Primary KC: FRACTIONS | Multiplication | multiplying two improper fractions
  \item Secondary KC: FRACTIONS | Multiplication | fraction multiplication
  \item Topic: Services such as installation, maintenance, repairing, cleaning, laundry, e-commerce
  \item Grade: Primary 5
\end{itemize}

\textbf{Question}

A cleaning company uses a special liquid to clean floors. Each bottle of the cleaning liquid can clean $\dfrac{9}{5}$ rooms. If one team uses $\dfrac{7}{4}$ bottles in a day, how many rooms can the team clean with the liquid used in a day?

\textbf{Solution}

To find the total number of rooms that can be cleaned, multiply the number of rooms per bottle by the number of bottles used:

$
\text{Rooms cleaned} = \dfrac{9}{5} \times \dfrac{7}{4} = \dfrac{9 \times 7}{5 \times 4} = \dfrac{63}{20}
$

So, the team can clean $\dfrac{63}{20}$ rooms in a day with the cleaning liquid used.

\section*{Question 90}
\textbf{Metadata}

\begin{itemize}
  \item Question ID: P5-FrMulMixN\_P2-FrCmp\_GPT4.1\_Services\_01
  \item Primary KC: FRACTIONS | Multiplication | multiplying a mixed number and a whole number
  \item Secondary KC: FRACTIONS | Comparison and ordering | comparing and ordering fractions
  \item Topic: Services such as installation, maintenance, repairing, cleaning, laundry, e-commerce
  \item Grade: Primary 5
\end{itemize}

\textbf{Question}

A laundry shop washes 3\,\frac{1}{2} baskets of clothes for each customer. On Monday, 4 customers brought in their dirty laundry.\
\
(a) How many baskets of clothes did the shop wash in total on Monday?\
\
(b) On Tuesday, the shop washed 14 baskets of clothes. On which day did the shop wash more baskets of clothes? Show your working clearly.

\textbf{Solution}

\textbf{(a) Number of baskets washed for 4 customers:}\\
$ 3\,\frac{1}{2} \times 4 = \frac{7}{2} \times 4 = \frac{7 \times 4}{2} = \frac{28}{2} = 14 $\
So, the shop washed 14 baskets of clothes on Monday.\\
\textbf{(b) Comparison:}\\
On Monday: 14 baskets.\\
On Tuesday: 14 baskets.\\
Since 14 = 14, the shop washed the same number of baskets on both days.

\section*{Question 91}
\textbf{Metadata}

\begin{itemize}
  \item Question ID: P5-FrMulMixN\_P2-FrAdd2nd\_GPT4.1\_Services\_01
  \item Primary KC: FRACTIONS | Multiplication | multiplying a mixed number and a whole number
  \item Secondary KC: FRACTIONS | Addition | adding fractions
  \item Topic: Services such as installation, maintenance, repairing, cleaning, laundry, e-commerce
  \item Grade: Primary 5
\end{itemize}

\textbf{Question}

A cleaning company offers carpet cleaning services. Siti worked for $3\frac{1}{2}$ hours each day for 5 days this week. 

(a) How many hours did Siti work in total over the 5 days?

On Saturday, she helped her friend clean for $2\frac{2}{3}$ hours. 

(b) What is the total number of hours Siti spent cleaning this week, including Saturday?

\textbf{Solution}

Part (a):

Siti worked $3\frac{1}{2}$ hours each day for 5 days.

First, convert the mixed number to an improper fraction:
$3\frac{1}{2} = \frac{7}{2}$

Total hours in 5 days:
$5 \times \frac{7}{2} = \frac{35}{2} = 17\frac{1}{2}$ hours

Part (b):

On Saturday: $2\frac{2}{3} = \frac{8}{3}$ hours

Add to the total from part (a):
$17\frac{1}{2} + 2\frac{2}{3}$
Convert both to improper fractions:
$17\frac{1}{2} = \frac{35}{2}$
$2\frac{2}{3} = \frac{8}{3}$

Find a common denominator (6):
$\frac{35}{2} = \frac{105}{6}$
$\frac{8}{3} = \frac{16}{6}$

$\frac{105}{6} + \frac{16}{6} = \frac{121}{6} = 20\frac{1}{6}$

**Answer:**
(a) $17\frac{1}{2}$ hours
(b) $20\frac{1}{6}$ hours

\section*{Question 92}
\textbf{Metadata}

\begin{itemize}
  \item Question ID: P5-FrMulMixN\_P2-FrSub2nd\_GPT4.1\_Services\_01
  \item Primary KC: FRACTIONS | Multiplication | multiplying a mixed number and a whole number
  \item Secondary KC: FRACTIONS | Subtraction | subtracting fractions
  \item Topic: Services such as installation, maintenance, repairing, cleaning, laundry, e-commerce
  \item Grade: Primary 5
\end{itemize}

\textbf{Question}

A cleaning company uses $2\dfrac{1}{2}$ litres of detergent to clean one floor of an office building. If the cleaners need to clean 4 floors, how many litres of detergent will they need in total?

After cleaning, they realise they have only $8\dfrac{3}{4}$ litres of detergent left. How much detergent did they use during the cleaning of the 4 floors?

\textbf{Solution}

First, we calculate the total amount of detergent used to clean 4 floors:

$2\dfrac{1}{2} \times 4 = \left(\frac{5}{2}\right) \times 4 = \frac{5 \times 4}{2} = \frac{20}{2} = 10$ litres.

So, they used 10 litres of detergent to clean 4 floors.

Next, to find out how much detergent was used, we subtract the amount left from the original amount used.

Amount used = Total used - Amount left

$10 - 8\dfrac{3}{4} = 10 - \frac{35}{4} = \frac{40}{4} - \frac{35}{4} = \frac{5}{4} = 1\dfrac{1}{4}$ litres.

Therefore, the cleaners used $1\dfrac{1}{4}$ litres of detergent during the cleaning of the 4 floors.

\section*{Question 93}
\textbf{Metadata}

\begin{itemize}
  \item Question ID: P5-FrMulMixN\_P3-FrSmp\_GPT4.1\_Services\_01
  \item Primary KC: FRACTIONS | Multiplication | multiplying a mixed number and a whole number
  \item Secondary KC: FRACTIONS | Simplifying | expressing a fraction in its simplest form
  \item Topic: Services such as installation, maintenance, repairing, cleaning, laundry, e-commerce
  \item Grade: Primary 5
\end{itemize}

\textbf{Question}

A cleaning company charges $2\dfrac{1}{4}$ hours to clean one storey of an office building. If they need to clean 3 storeys, how many hours in total will they take? Give your answer as a simplified mixed number.

\textbf{Solution}

First, write $2\dfrac{1}{4}$ as an improper fraction:

$2\dfrac{1}{4} = \frac{9}{4}$

Multiply by 3 storeys:

$3 \times \frac{9}{4} = \frac{27}{4}$

Convert $\frac{27}{4}$ to a mixed number:

$27 \div 4 = 6$ remainder $3$

So, $\frac{27}{4} = 6\dfrac{3}{4}$

$3$ and $4$ have no common factors other than $1$, so the fraction is already in its simplest form.

**Answer:** The cleaners will take $6\dfrac{3}{4}$ hours in total to clean all 3 storeys.

\section*{Question 94}
\textbf{Metadata}

\begin{itemize}
  \item Question ID: P5-FrMulMixN\_P5-FrCnv2Dc\_GPT4.1\_Services\_01
  \item Primary KC: FRACTIONS | Multiplication | multiplying a mixed number and a whole number
  \item Secondary KC: FRACTIONS | Conversion to decimals | expressing fractions as decimals
  \item Topic: Services such as installation, maintenance, repairing, cleaning, laundry, e-commerce
  \item Grade: Primary 5
\end{itemize}

\textbf{Question}

Alice runs a professional carpet cleaning service. For each job, she uses $2\frac{1}{2}$ litres of cleaning solution per room. She has been hired to clean 6 rooms in a condominium. 

(a) How many litres of cleaning solution will Alice need in total? 

(b) Express the total amount of cleaning solution needed as a decimal.

\textbf{Solution}

(a) First, convert the mixed number to an improper fraction:

$2\frac{1}{2} = \frac{5}{2}$

Multiply by the number of rooms:

$\frac{5}{2} \times 6 = \frac{5 \times 6}{2} = \frac{30}{2} = 15$

So, Alice will need $15$ litres in total.

(b) Express as a decimal: $15$ litres is already a decimal (15.0 litres).

\textbf{Final Answers:} 
(a) $15$ litres
(b) $15.0$ litres

\section*{Question 95}
\textbf{Metadata}

\begin{itemize}
  \item Question ID: P5-FrMulMixN\_P5-FrMul2nd\_GPT4.1\_Services\_01
  \item Primary KC: FRACTIONS | Multiplication | multiplying a mixed number and a whole number
  \item Secondary KC: FRACTIONS | Multiplication | fraction multiplication
  \item Topic: Services such as installation, maintenance, repairing, cleaning, laundry, e-commerce
  \item Grade: Primary 5
\end{itemize}

\textbf{Question}

A cleaning company charges $\frac{2}{3}$ hours to clean each sofa. Last Saturday, they cleaned $3\frac{1}{2}$ sofas for a customer. 

(a) How many hours did the company spend cleaning all the sofas?

(b) If the company employs 4 cleaners to do the job together, and they each work the same amount of time, how many hours did each cleaner work?

\textbf{Solution}

(a) First, multiply the number of sofas by the time taken for each sofa:

$3\frac{1}{2} \times \frac{2}{3}$ 

Convert $3\frac{1}{2}$ to an improper fraction:
$3\frac{1}{2} = \frac{7}{2}$

Now multiply:
$\frac{7}{2} \times \frac{2}{3} = \frac{7 \times 2}{2 \times 3} = \frac{14}{6}$  
Simplify $\frac{14}{6} = \frac{7}{3}$ hours

$\frac{7}{3}$ hours $= 2\frac{1}{3}$ hours

So, the company spent $2\frac{1}{3}$ hours cleaning all the sofas.

(b) If 4 cleaners share this total time equally:

Each cleaner worked $\frac{2\frac{1}{3}}{4}$ hours

Convert $2\frac{1}{3}$ to an improper fraction:
$2\frac{1}{3} = \frac{7}{3}$

So, $\frac{7}{3} \div 4 = \frac{7}{3} \times \frac{1}{4} = \frac{7}{12}$ hours per cleaner.

Therefore, each cleaner worked $\frac{7}{12}$ of an hour.

\section*{Question 96}
\textbf{Metadata}

\begin{itemize}
  \item Question ID: P5-DcMul3dK\_P4-DcCmp3d\_GPT4.1\_Services\_01
  \item Primary KC: DECIMALS | Multiplication | multiplying decimals (up to 3 decimal places) by 10, 100, 1000 and their multiples
  \item Secondary KC: DECIMALS | Comparison and ordering | comparing and ordering decimals up to 3 decimal places
  \item Topic: Services such as installation, maintenance, repairing, cleaning, laundry, e-commerce
  \item Grade: Primary 5
\end{itemize}

\textbf{Question}

A cleaning company charges \$1.275 per square metre to clean office carpets. Mr. Lim has three offices to clean: Office A is 45 square metres, Office B is 38 square metres, and Office C is 51 square metres.

(a) Find the cleaning cost for each office, rounding your answers to the nearest cent.

(b) Arrange the cleaning costs for the three offices in order from least to greatest.

\textbf{Solution}

(a) 
\begin{align*}
\text{Office A:}\quad & 1.275 \times 45 = 57.375 \text{ (dollars)} \rightarrow \$57.38 \\
\text{Office B:}\quad & 1.275 \times 38 = 48.45 \text{ (dollars)} \rightarrow \$48.45 \\
\text{Office C:}\quad & 1.275 \times 51 = 65.025 \text{ (dollars)} \rightarrow \$65.03 \\
\end{align*}

(b) 
Comparing the rounded costs:
\$48.45 ~<~ \$57.38 ~<~ \$65.03

So, the order from least to greatest is: Office B, Office A, Office C.

\section*{Question 97}
\textbf{Metadata}

\begin{itemize}
  \item Question ID: P5-DcMul3dK\_P4-DcCnv2Fr\_GPT4.1\_Services\_01
  \item Primary KC: DECIMALS | Multiplication | multiplying decimals (up to 3 decimal places) by 10, 100, 1000 and their multiples
  \item Secondary KC: DECIMALS | Conversion from decimals to fraction | expressing decimals as fractions
  \item Topic: Services such as installation, maintenance, repairing, cleaning, laundry, e-commerce
  \item Grade: Primary 5
\end{itemize}

\textbf{Question}

A cleaning service company charges \$0.275 per square metre to clean offices. Mr. Lim wants to clean an office space that is 1200 square metres large. 

(a) What is the total cleaning cost Mr. Lim has to pay? 

(b) Express the cost per square metre, $0.275, as a fraction in its simplest form.

\textbf{Solution}

Part (a): 

Total cleaning cost = cost per square metre $\times$ number of square metres 

Total cost = $0.275 \times 1200$

$0.275 \times 1200 = 330$

So, Mr. Lim has to pay \$330.

Part (b): 

$0.275 = \frac{275}{1000}$

Simplifying by dividing both the numerator and the denominator by 25,

$\frac{275 \div 25}{1000 \div 25} = \frac{11}{40}$

So, $0.275$ as a fraction in simplest form is $\frac{11}{40}$.

\section*{Question 98}
\textbf{Metadata}

\begin{itemize}
  \item Question ID: P5-DcMul3dK\_P4-DcRnd3d\_GPT4.1\_Services\_01
  \item Primary KC: DECIMALS | Multiplication | multiplying decimals (up to 3 decimal places) by 10, 100, 1000 and their multiples
  \item Secondary KC: DECIMALS | Rounding | rounding decimals up to 3 decimal places to the nearest whole number, 1 decimal place and 2 decimal places 
  \item Topic: Services such as installation, maintenance, repairing, cleaning, laundry, e-commerce
  \item Grade: Primary 5
\end{itemize}

\textbf{Question}

A cleaning company charges \$5.278 per square metre to clean a carpet. Mrs. Lee wants to clean the carpets in her office, which has a total area of 100 square metres. 

(a) What is the total cost, in dollars, to clean the entire office carpet?

(b) Round your answer in (a) to the nearest whole number.

(c) If Mrs. Lee decides to clean only 25 square metres, what is the total cost, rounded to 1 decimal place?

\textbf{Solution}

(a) Total cost = 100 \times 5.278 = 527.8. So, the total cost is \$527.80.

(b) Rounded to the nearest whole number: 527.80 rounds up to 528.

(c) Cost for 25 square metres = 25 \times 5.278 = 131.95. 
Rounded to 1 decimal place: 132.0

Final answers:
(a) \$527.80
(b) \$528
(c) \$132.0

\section*{Question 99}
\textbf{Metadata}

\begin{itemize}
  \item Question ID: P5-DcMul3dK\_P4-DcAdd2nd\_GPT4.1\_Services\_01
  \item Primary KC: DECIMALS | Multiplication | multiplying decimals (up to 3 decimal places) by 10, 100, 1000 and their multiples
  \item Secondary KC: DECIMALS | Addition | adding decimals
  \item Topic: Services such as installation, maintenance, repairing, cleaning, laundry, e-commerce
  \item Grade: Primary 5
\end{itemize}

\textbf{Question}

A laundry shop in Singapore charges $3.280$ for washing one blanket and $5.470$ for washing one bed sheet. If Mr. Lee sends $5$ blankets and $3$ bed sheets for washing, how much does he need to pay in total?

\textbf{Solution}

First, find the total cost for washing the blankets:

$3.280 \times 5 = 16.400$

Next, find the total cost for washing the bed sheets:

$5.470 \times 3 = 16.410$

Now, add the two totals together:

$16.400 + 16.410 = 32.810$

Therefore, Mr. Lee needs to pay $\$32.810$ in total.

\section*{Question 100}
\textbf{Metadata}

\begin{itemize}
  \item Question ID: P5-DcMul3dK\_P4-DcSub2nd\_GPT4.1\_Services\_01
  \item Primary KC: DECIMALS | Multiplication | multiplying decimals (up to 3 decimal places) by 10, 100, 1000 and their multiples
  \item Secondary KC: DECIMALS | Subtraction | subtracting decimals
  \item Topic: Services such as installation, maintenance, repairing, cleaning, laundry, e-commerce
  \item Grade: Primary 5
\end{itemize}

\textbf{Question}

A cleaning company charges $6.450 per square metre to clean office carpets.

Mr. Lim hired the company to clean 120 square metres of carpet in his office. The company gave him a discount of $142.20 for being a regular customer.

(a) What is the total cost to clean 120 square metres of carpet before the discount?

(b) How much did Mr. Lim pay after the discount?

\textbf{Solution}

(a) The cost to clean 1 square metre is $6.450.
To find the total cost for 120 square metres:

$6.450 \times 120 = $
First, multiply 6.450 by 1000 (to see the decimal shift):
6.450 \times 1000 = 6450

But 120 is 100 + 20.
6.450 \times 100 = 645.0
6.450 \times 20 = 129.0
Add them up: 645.0 + 129.0 = $774.0
So, the total cost before the discount is $774.00.

(b) He gets a discount of $142.20.
Amount to pay:
$774.00 - $142.20 = $631.80

So, Mr. Lim pays $631.80 after the discount.

\section*{Question 101}
\textbf{Metadata}

\begin{itemize}
  \item Question ID: P5-DcDiv3dK\_P4-DcCmp3d\_GPT4.1\_Services\_01
  \item Primary KC: DECIMALS | Division | dividing decimals (up to 3 decimal places) by 10, 100, 1000 and their multiples
  \item Secondary KC: DECIMALS | Comparison and ordering | comparing and ordering decimals up to 3 decimal places
  \item Topic: Services such as installation, maintenance, repairing, cleaning, laundry, e-commerce
  \item Grade: Primary 5
\end{itemize}

\textbf{Question}

A laundry shop received a large order to clean towels. The total weight of the towels was $36.480$ kg. The shop owner needs to divide the towels equally into $100$ laundry bags before sending them for cleaning.

(a) What is the weight of towels, in kilograms, in each laundry bag?

After loading all the bags, the owner compares the weight of towels in each bag to three other amounts: $0.36$ kg, $0.364$ kg, and $0.365$ kg.

(b) Arrange the following amounts, including the weight of towels in each laundry bag, in order from least to greatest:
$0.36$ kg, $0.364$ kg, $0.365$ kg, and the weight of towels in each laundry bag.

\textbf{Solution}

(a) To find the weight of towels in each laundry bag:

Weight in each bag $= \dfrac{36.480\ \text{kg}}{100} = 0.3648\ \text{kg}$

(b) To compare:
\begin{align*}
0.36 & = 0.3600 \\
0.364 & = 0.3640 \\
0.365 & = 0.3650 \\
0.3648 & \text{ (weight in each bag)}
\end{align*}

Arranging from least to greatest:

$0.36$ kg $< 0.364$ kg $< 0.3648$ kg $< 0.365$ kg

So, the order is:
$
0.36\ \text{kg},\ 0.364\ \text{kg},\ 0.3648\ \text{kg},\ 0.365\ \text{kg}
$

\section*{Question 102}
\textbf{Metadata}

\begin{itemize}
  \item Question ID: P5-DcDiv3dK\_P4-DcCnv2Fr\_GPT4.1\_Services\_01
  \item Primary KC: DECIMALS | Division | dividing decimals (up to 3 decimal places) by 10, 100, 1000 and their multiples
  \item Secondary KC: DECIMALS | Conversion from decimals to fraction | expressing decimals as fractions
  \item Topic: Services such as installation, maintenance, repairing, cleaning, laundry, e-commerce
  \item Grade: Primary 5
\end{itemize}

\textbf{Question}

A cleaning service charges \$24.680 for a job. The total cost will be shared equally among 100 customers. 

(a) How much does each customer need to pay? Express your answer in decimal form.

(b) Express the amount each customer needs to pay as a fraction in its simplest form.

\textbf{Solution}

(a) Each customer pays \$24.680 \div 100 = \$0.2468.

(b) To express 0.2468 as a fraction:
0.2468 = \frac{2468}{10000}

Simplify \frac{2468}{10000}:
Divide numerator and denominator by 4: \frac{2468 \div 4}{10000 \div 4} = \frac{617}{2500}

So, each customer pays \$0.2468, or \frac{617}{2500} as a fraction.

\section*{Question 103}
\textbf{Metadata}

\begin{itemize}
  \item Question ID: P5-DcDiv3dK\_P4-DcRnd3d\_GPT4.1\_Services\_01
  \item Primary KC: DECIMALS | Division | dividing decimals (up to 3 decimal places) by 10, 100, 1000 and their multiples
  \item Secondary KC: DECIMALS | Rounding | rounding decimals up to 3 decimal places to the nearest whole number, 1 decimal place and 2 decimal places 
  \item Topic: Services such as installation, maintenance, repairing, cleaning, laundry, e-commerce
  \item Grade: Primary 5
\end{itemize}

\textbf{Question}

A cleaning company charges $15.750 for cleaning one room. A school hires the company to clean 100 rooms. 

(a) What is the amount charged for cleaning one room when divided by 100?

(b) Round your answer in (a) to the nearest whole number, 1 decimal place, and 2 decimal places.

\textbf{Solution}

(a) To find the amount charged for cleaning one room divided by 100:

$
\frac{15.750}{100} = 0.1575
$

(b) Now, round 0.1575:

- To the nearest whole number: 0
- To 1 decimal place: 0.2
- To 2 decimal places: 0.16

\section*{Question 104}
\textbf{Metadata}

\begin{itemize}
  \item Question ID: P5-DcDiv3dK\_P4-DcAdd2nd\_GPT4.1\_Services\_01
  \item Primary KC: DECIMALS | Division | dividing decimals (up to 3 decimal places) by 10, 100, 1000 and their multiples
  \item Secondary KC: DECIMALS | Addition | adding decimals
  \item Topic: Services such as installation, maintenance, repairing, cleaning, laundry, e-commerce
  \item Grade: Primary 5
\end{itemize}

\textbf{Question}

A laundry company charged $12.75 to clean 5 shirts. Another customer brought in 13 shirts and paid the same rate per shirt. 

(a) How much did it cost to clean 1 shirt? 

(b) What was the total cost for cleaning 13 shirts at this rate?

(c) If the customer also paid an additional $2.40 for express service, what was the total amount the customer paid for 13 shirts, including the express fee?

\textbf{Solution}

**(a) Cost to clean 1 shirt:**

$ \text{Cost per shirt} = \frac{12.75}{5} = 2.55 $

**(b) Total cost for 13 shirts:**

$ \text{Total cost for 13 shirts} = 13 \times 2.55 = 33.15 $

**(c) Total amount including express fee:**

$ 33.15 + 2.40 = 35.55 $

**Final Answer:**

(a) $2.55  
(b) $33.15  
(c) $35.55

\section*{Question 105}
\textbf{Metadata}

\begin{itemize}
  \item Question ID: P5-DcDiv3dK\_P4-DcSub2nd\_GPT4.1\_Services\_01
  \item Primary KC: DECIMALS | Division | dividing decimals (up to 3 decimal places) by 10, 100, 1000 and their multiples
  \item Secondary KC: DECIMALS | Subtraction | subtracting decimals
  \item Topic: Services such as installation, maintenance, repairing, cleaning, laundry, e-commerce
  \item Grade: Primary 5
\end{itemize}

\textbf{Question}

A laundry shop charges $7.500 to wash and iron 10 shirts. Lina brought 10 shirts to the shop and was charged the full amount. Later, she realised she only needed 7 shirts washed and ironed. If the shop subtracts the cost of the extra 3 shirts at the same rate per shirt from her total bill, how much should Lina pay after the adjustment?

\textbf{Solution}

First, find the cost to wash and iron 1 shirt:

$
\text{Cost per shirt} = \frac{7.500}{10} = 0.750
$

Cost of 3 extra shirts:

$
\text{Cost of 3 shirts} = 0.750 \times 3 = 2.250
$

Subtract this amount from the initial bill:

$
\text{Amount Lina should pay} = 7.500 - 2.250 = 5.250
$

Therefore, Lina should pay $5.250 after the adjustment.

\section*{Question 106}
\textbf{Metadata}

\begin{itemize}
  \item Question ID: P5-PcRepWh\_P1-WNAdd2nd\_GPT4.1\_Services\_01
  \item Primary KC: PERCENTAGE | Representation and concept | expressing a part of a whole as a percentage
  \item Secondary KC: WHOLE NUMBERS | Addition | adding whole numbers
  \item Topic: Services such as installation, maintenance, repairing, cleaning, laundry, e-commerce
  \item Grade: Primary 5
\end{itemize}

\textbf{Question}

A cleaning company cleaned 60 offices in one week. Out of these, 15 offices requested window cleaning as an extra service. 

(a) What percentage of the offices requested window cleaning?

(b) If another 5 offices request window cleaning the next week, what will be the new percentage of offices requesting window cleaning out of the total 60 offices?

\textbf{Solution}

(a) To find the percentage of offices that requested window cleaning:

Number of offices that requested window cleaning = 15 \\ 
Total offices cleaned = 60 

$\text{Percentage} = \frac{15}{60} \times 100\% = 25\%$

(b) Additional offices requesting window cleaning = 5 \\ 
Total offices requesting window cleaning next week = 15 + 5 = 20

$\text{New percentage} = \frac{20}{60} \times 100\% = 33\frac{1}{3}\%$

Final answers:
(a) 25% of the offices requested window cleaning.

(b) The new percentage is 33\frac{1}{3}% of the offices.

\section*{Question 107}
\textbf{Metadata}

\begin{itemize}
  \item Question ID: P5-PcRepWh\_P1-WNSub2nd\_GPT4.1\_Services\_01
  \item Primary KC: PERCENTAGE | Representation and concept | expressing a part of a whole as a percentage
  \item Secondary KC: WHOLE NUMBERS | Subtraction | subtracting whole numbers
  \item Topic: Services such as installation, maintenance, repairing, cleaning, laundry, e-commerce
  \item Grade: Primary 5
\end{itemize}

\textbf{Question}

A cleaning company received a contract to clean 200 offices in a shopping mall. After 1 week, they had cleaned 150 of the offices.

(a) What percentage of the offices have been cleaned after 1 week?

(b) How many offices still need to be cleaned?

\textbf{Solution}

(a) Number of offices cleaned = 150

Total number of offices = 200

Percentage cleaned $= \dfrac{150}{200} \times 100\% = 75\%$

So, 75\% of the offices have been cleaned after 1 week.

(b) Offices still to be cleaned $= 200 - 150 = 50$

So, 50 offices still need to be cleaned.

\section*{Question 108}
\textbf{Metadata}

\begin{itemize}
  \item Question ID: P5-PcRepWh\_P1-WNMul2nd\_GPT4.1\_Services\_01
  \item Primary KC: PERCENTAGE | Representation and concept | expressing a part of a whole as a percentage
  \item Secondary KC: WHOLE NUMBERS | Multiplication | multiplying whole numbers
  \item Topic: Services such as installation, maintenance, repairing, cleaning, laundry, e-commerce
  \item Grade: Primary 5
\end{itemize}

\textbf{Question}

A laundry shop cleaned 40 shirts on Monday. On Tuesday, the shop cleaned 25\% more shirts than on Monday.

(a) How many shirts did the shop clean on Tuesday?

(b) What was the total number of shirts cleaned on both Monday and Tuesday?

Express your answers clearly and show your working.

\textbf{Solution}

(a) On Monday, the shop cleaned 40 shirts.

25\% of 40 = \frac{25}{100} \times 40 = 10

So, on Tuesday, the shop cleaned: 40 + 10 = 50 shirts

Answer: The shop cleaned 50 shirts on Tuesday.

(b) Total number of shirts cleaned = 40 (Monday) + 50 (Tuesday) = 90

Answer: 90 shirts were cleaned in total on both days.

\section*{Question 109}
\textbf{Metadata}

\begin{itemize}
  \item Question ID: P5-PcRepWh\_P1-WNDiv2nd\_GPT4.1\_Services\_01
  \item Primary KC: PERCENTAGE | Representation and concept | expressing a part of a whole as a percentage
  \item Secondary KC: WHOLE NUMBERS | Division | dividing whole numbers
  \item Topic: Services such as installation, maintenance, repairing, cleaning, laundry, e-commerce
  \item Grade: Primary 5
\end{itemize}

\textbf{Question}

A cleaning company was hired to clean a block of $80$ offices. On Monday, their team managed to clean $32$ offices. 

(a) What percentage of the offices did they clean on Monday?

(b) If the team decides to divide the remaining offices equally among $3$ more cleaning days, how many offices will they need to clean each day?

\textbf{Solution}

(a) The percentage of offices cleaned on Monday is given by 

$ \frac{32}{80} \times 100\% = 40\% $

(b) Number of offices remaining to be cleaned: $80 - 32 = 48$.

Dividing equally among $3$ more days: 

$ \frac{48}{3} = 16 $

So, they need to clean $16$ offices each day for the next $3$ days.

\section*{Question 110}
\textbf{Metadata}

\begin{itemize}
  \item Question ID: P5-RtFndR\_P2-DcCnvN2D\_GPT4.1\_Services\_01
  \item Primary KC: RATE | Finding rate | finding rate given total amount and number of units
  \item Secondary KC: DECIMALS | Conversion to larger units | converting an amount or measurement from a smaller unit to a larger unit in decimal form
  \item Topic: Services such as installation, maintenance, repairing, cleaning, laundry, e-commerce
  \item Grade: Primary 5
\end{itemize}

\textbf{Question}

A cleaning company charges a total of $96 for cleaning 8 offices. Each office varies in size, but the cleaning time for each is 45 minutes. 

(a) What is the rate charged per office?

(b) Express the total cleaning time for all 8 offices in hours, in decimal form.

\textbf{Solution}

Part (a):
Total cost = $96
Number of offices = 8
Rate per office = $96 \div 8 = $12

Part (b):
Cleaning time per office = 45 minutes
Total time for 8 offices = 8 \times 45 = 360 minutes
1 hour = 60 minutes
Total time in hours = 360 \div 60 = 6.0 hours

Final answers:
(a) The rate charged per office is $12.
(b) The total cleaning time is 6.0 hours.

\section*{Question 111}
\textbf{Metadata}

\begin{itemize}
  \item Question ID: P5-RtFndR\_P2-DcCnvD2N\_GPT4.1\_Services\_01
  \item Primary KC: RATE | Finding rate | finding rate given total amount and number of units
  \item Secondary KC: DECIMALS | Conversion to smaller units | converting an amount or measurement from a larger unit in decimal form to a smaller unit
  \item Topic: Services such as installation, maintenance, repairing, cleaning, laundry, e-commerce
  \item Grade: Primary 5
\end{itemize}

\textbf{Question}

A laundry shop charges $56.40 to clean 12.5 kilograms of clothes. 

(a) What is the rate charged per kilogram of clothes cleaned? 

(b) Express your answer in (a) in cents per 100 grams.

\textbf{Solution}

(a) Rate per kilogram = \( \frac{56.40}{12.5} = 4.512 \) dollars per kilogram.

(b) 1 kilogram = 1000 grams, so 100 grams = 0.1 kilogram.

Amount charged for 0.1 kilogram = \( 4.512 \times 0.1 = 0.4512 \) dollars.

To convert dollars to cents, \( 0.4512 \times 100 = 45.12 \) cents.

Thus, the rate is 45.12 cents per 100 grams.

\section*{Question 112}
\textbf{Metadata}

\begin{itemize}
  \item Question ID: P5-RtFndT\_P2-DcCnvN2D\_GPT4.1\_Services\_01
  \item Primary KC: RATE | Finding total amount | finding total amount, given rate and number of units
  \item Secondary KC: DECIMALS | Conversion to larger units | converting an amount or measurement from a smaller unit to a larger unit in decimal form
  \item Topic: Services such as installation, maintenance, repairing, cleaning, laundry, e-commerce
  \item Grade: Primary 5
\end{itemize}

\textbf{Question}

A cleaning company charges \$2.50 for every 100 grams of laundry cleaned. Mdm Lim sent in a bag of clothes that weighs 1.4 kilograms. How much does Mdm Lim have to pay to clean her laundry? 

(Give your answer in dollars and cents.)

\textbf{Solution}

First, convert the weight from kilograms to grams:

1.4 kg = 1.4 \times 1000 = 1400 grams.

Next, find how many 100 grams are in 1400 grams:

1400 \div 100 = 14

The rate is \$2.50 per 100 grams, so the total amount to pay is:

14 \times \$2.50 = \$35.00

Therefore, Mdm Lim has to pay \$35.00 to clean her laundry.

\section*{Question 113}
\textbf{Metadata}

\begin{itemize}
  \item Question ID: P5-RtFndT\_P2-DcCnvD2N\_GPT4.1\_Services\_01
  \item Primary KC: RATE | Finding total amount | finding total amount, given rate and number of units
  \item Secondary KC: DECIMALS | Conversion to smaller units | converting an amount or measurement from a larger unit in decimal form to a smaller unit
  \item Topic: Services such as installation, maintenance, repairing, cleaning, laundry, e-commerce
  \item Grade: Primary 5
\end{itemize}

\textbf{Question}

A cleaning company charges $1.25 per square metre to clean a carpet. Mrs Tan has a carpet that measures 16.5 square metres. How much does Mrs Tan have to pay the company to clean her carpet? Give your answer in cents.

\textbf{Solution}

First, find the total cost by multiplying the rate by the number of square metres:

$1.25 \times 16.5 = $20.625

Next, to convert dollars to cents, multiply by 100:

$20.625 \times 100 = 2062.5$ cents

Since money is usually rounded to the nearest cent, Mrs Tan has to pay 2063 cents.

\section*{Question 114}
\textbf{Metadata}

\begin{itemize}
  \item Question ID: P5-RtFndU\_P2-DcCnvN2D\_GPT4.1\_Services\_01
  \item Primary KC: RATE | Finding number of unit | finding number of units given rate and total amount
  \item Secondary KC: DECIMALS | Conversion to larger units | converting an amount or measurement from a smaller unit to a larger unit in decimal form
  \item Topic: Services such as installation, maintenance, repairing, cleaning, laundry, e-commerce
  \item Grade: Primary 5
\end{itemize}

\textbf{Question}

A cleaning company charges \$0.35 for washing each small towel. After a hotel event, the staff collected a total of 168 towels that needed to be washed. The total amount paid to the cleaning company for washing these towels was \$58.80. 

(a) How many towels were washed by the company? 

(b) Each small towel weighs 250 grams. What is the total mass of all 168 towels in kilograms? (Give your answer in decimal form.)

\textbf{Solution}

(a) Let the number of towels washed be $n$. 

The rate for washing 1 towel = $\$0.35$.

Total amount paid = $\$58.80$.

So, $n = \dfrac{58.80}{0.35} = 168$ towels.

(b) Each towel weighs 250 grams.

Total mass = $168 \times 250 = 42\ 000$ grams.

To convert grams to kilograms, divide by 1000:

$42\ 000$ grams $\div 1000 = 42$ kilograms.

**Answers:**
(a) 168 towels
(b) 42 kilograms

\section*{Question 115}
\textbf{Metadata}

\begin{itemize}
  \item Question ID: P5-RtFndU\_P2-DcCnvD2N\_GPT4.1\_Services\_01
  \item Primary KC: RATE | Finding number of unit | finding number of units given rate and total amount
  \item Secondary KC: DECIMALS | Conversion to smaller units | converting an amount or measurement from a larger unit in decimal form to a smaller unit
  \item Topic: Services such as installation, maintenance, repairing, cleaning, laundry, e-commerce
  \item Grade: Primary 5
\end{itemize}

\textbf{Question}

A company charges \$3.50 for cleaning 1 square metre of carpet. Mrs Lim wants to clean her living room carpet, which has an area of 12.6 square metres. 

(a) Convert the area of the carpet from square metres to square decimetres. (1 square metre = 100 square decimetres)

(b) How much will Mrs Lim have to pay for cleaning her carpet?

\textbf{Solution}

(a) Area in square decimetres = 12.6 \times 100 = 1260 \text{ square decimetres}.

(b) Cost to clean 1 square metre = \$3.50

Total area = 12.6 square metres

Total cost = 12.6 \times 3.50 = \$44.10

\textbf{Answer:} 
(a) 1260 square decimetres 
(b) Mrs Lim will have to pay \$44.10.

\section*{Question 116}
\textbf{Metadata}

\begin{itemize}
  \item Question ID: P6-FrDivPN\_P2-FrCmp\_GPT4.1\_Services\_06
  \item Primary KC: FRACTIONS | Division | dividing a proper fraction by a whole number
  \item Secondary KC: FRACTIONS | Comparison and ordering | comparing and ordering fractions
  \item Topic: Services such as installation, maintenance, repairing, cleaning, laundry, e-commerce
  \item Grade: Primary 6
\end{itemize}

\textbf{Question}

A cleaning company received an order to clean 5 identical apartment units in a condominium. The cleaning team has \( \frac{3}{4} \) litres of cleaning solution that needs to be divided equally among the 5 units. 

(a) How much cleaning solution will be used in each apartment unit? 

(b) Another team has \( \frac{2}{3} \) litres of cleaning solution to divide equally among the same number of units. Which team will use more cleaning solution per apartment unit? Show your working.

\textbf{Solution}

**(a) Amount of cleaning solution per unit:**

To find the amount used per unit:

$ \text{Amount per unit} = \frac{3}{4} \div 5 = \frac{3}{4} \times \frac{1}{5} = \frac{3}{20} \text{ litres} $

**(b) Comparison with the other team:**

Amount per unit for the second team:

$ \text{Amount per unit (second team)} = \frac{2}{3} \div 5 = \frac{2}{3} \times \frac{1}{5} = \frac{2}{15} \text{ litres} $

Now compare \( \frac{3}{20} \) and \( \frac{2}{15} \):

Let's find a common denominator. The lowest common multiple of 20 and 15 is 60.

$ \frac{3}{20} = \frac{9}{60} $
$ \frac{2}{15} = \frac{8}{60} $

Since \( \frac{9}{60} > \frac{8}{60} \), each apartment cleaned by the first team receives more cleaning solution.

**Final Answer:**
- Each apartment unit from the first team gets \( \frac{3}{20} \) litres.
- The first team uses more cleaning solution per apartment unit.

\section*{Question 117}
\textbf{Metadata}

\begin{itemize}
  \item Question ID: P6-FrDivPN\_P2-FrAdd2nd\_GPT4.1\_Services\_01
  \item Primary KC: FRACTIONS | Division | dividing a proper fraction by a whole number
  \item Secondary KC: FRACTIONS | Addition | adding fractions
  \item Topic: Services such as installation, maintenance, repairing, cleaning, laundry, e-commerce
  \item Grade: Primary 6
\end{itemize}

\textbf{Question}

A cleaning company is hired to clean 3 schools. Each school needs to have \( \frac{2}{3} \) of its classrooms cleaned each day. The company divides the job equally among 4 cleaning teams. After the first day, the company realises it can clean an additional \( \frac{1}{6} \) of the classrooms in each school without exceeding their schedule. 

(a) What fraction of the classrooms in one school does each team clean in one day?

(b) What fraction of the classrooms in one school can be cleaned in one day after including the additional classrooms?

(c) What is the total fraction of the classrooms that can be cleaned in all 3 schools in one day after the change?

\textbf{Solution}

(a) Each school needs \( \frac{2}{3} \) of its classrooms cleaned each day. The company has 4 teams, so each team cleans \( \frac{2}{3} \div 4 = \frac{2}{3} \times \frac{1}{4} = \frac{2}{12} = \frac{1}{6} \) of the classrooms in one school per day.

(b) The company can clean an additional \( \frac{1}{6} \) of the classrooms in each school per day. So, the total fraction cleaned per day in one school is:

$
\frac{2}{3} + \frac{1}{6} = \frac{4}{6} + \frac{1}{6} = \frac{5}{6}
$

(c) For all 3 schools, the total fraction of classrooms cleaned per day is:

$
3 \times \frac{5}{6} = \frac{15}{6} = 2\frac{1}{2}
$

So, a total equivalent to 2 whole schools and half of another school’s classrooms are cleaned each day after the change.

\section*{Question 118}
\textbf{Metadata}

\begin{itemize}
  \item Question ID: P6-FrDivPN\_P2-FrSub2nd\_GPT4.1\_Services\_01
  \item Primary KC: FRACTIONS | Division | dividing a proper fraction by a whole number
  \item Secondary KC: FRACTIONS | Subtraction | subtracting fractions
  \item Topic: Services such as installation, maintenance, repairing, cleaning, laundry, e-commerce
  \item Grade: Primary 6
\end{itemize}

\textbf{Question}

A laundry shop uses $\dfrac{3}{4}$ of a bottle of detergent to wash 6 loads of clothes. 

(a) How much detergent does the laundry shop use for 1 load of clothes?

After washing 1 load, the shop realises they need to use $\dfrac{1}{8}$ of a bottle less detergent for the next load. 

(b) How much detergent will be used for the next load?

\textbf{Solution}

Part (a):

To find out how much detergent is used per load, we divide $\dfrac{3}{4}$ by 6.

$\dfrac{3}{4} \div 6 = \dfrac{3}{4} \times \dfrac{1}{6} = \dfrac{3}{24} = \dfrac{1}{8}$

So, detergent used for 1 load is $\dfrac{1}{8}$ of a bottle.

---

Part (b):

For the next load, the laundry shop will use $\dfrac{1}{8}$ of a bottle less detergent.
The amount of detergent used for the next load:

$\dfrac{1}{8} - \dfrac{1}{8} = 0$

So, no detergent would be used for the next load according to this calculation. (If the intention was to subtract a smaller amount, e.g., $\dfrac{1}{16}$, adjust accordingly. But with $\dfrac{1}{8} - \dfrac{1}{8} = 0$.)

\section*{Question 119}
\textbf{Metadata}

\begin{itemize}
  \item Question ID: P6-FrDivPN\_P3-FrSmp\_GPT4.1\_Services\_06
  \item Primary KC: FRACTIONS | Division | dividing a proper fraction by a whole number
  \item Secondary KC: FRACTIONS | Simplifying | expressing a fraction in its simplest form
  \item Topic: Services such as installation, maintenance, repairing, cleaning, laundry, e-commerce
  \item Grade: Primary 6
\end{itemize}

\textbf{Question}

A cleaning company uses $\frac{3}{4}$ litre of detergent to clean 1 house. The company has exactly $5$ houses left to clean today. If the available detergent needs to be divided equally among all the houses, how many litres of detergent will each house receive? Express your answer in its simplest form.

\textbf{Solution}

The total detergent to be shared among the 5 houses is $\frac{3}{4}$ litre. Each house receives:

$
\frac{3}{4} \div 5 = \frac{3}{4} \times \frac{1}{5} = \frac{3}{20}
$

So, each house will receive $\frac{3}{20}$ litre of detergent in its simplest form.

\section*{Question 120}
\textbf{Metadata}

\begin{itemize}
  \item Question ID: P6-FrDivPN\_P5-FrCnv2Dc\_GPT4.1\_Services\_06
  \item Primary KC: FRACTIONS | Division | dividing a proper fraction by a whole number
  \item Secondary KC: FRACTIONS | Conversion to decimals | expressing fractions as decimals
  \item Topic: Services such as installation, maintenance, repairing, cleaning, laundry, e-commerce
  \item Grade: Primary 6
\end{itemize}

\textbf{Question}

A cleaning company uses $\frac{3}{4}$ litres of detergent to clean 6 identical offices.\

(a) How much detergent, in litres, is used to clean each office?\

(b) Express the answer from part (a) as a decimal.

\textbf{Solution}

(a) Amount of detergent used for each office = $\frac{3}{4} \div 6 = \frac{3}{4} \times \frac{1}{6} = \frac{3}{24} = \frac{1}{8}$ litres.\

(b) $\frac{1}{8} = 0.125$ litres.\

Final answer: Each office uses $\frac{1}{8}$ litres, which is $0.125$ litres.

\section*{Question 121}
\textbf{Metadata}

\begin{itemize}
  \item Question ID: P6-FrDivPN\_P5-FrMul2nd\_GPT4.1\_Services\_01
  \item Primary KC: FRACTIONS | Division | dividing a proper fraction by a whole number
  \item Secondary KC: FRACTIONS | Multiplication | fraction multiplication
  \item Topic: Services such as installation, maintenance, repairing, cleaning, laundry, e-commerce
  \item Grade: Primary 6
\end{itemize}

\textbf{Question}

A cleaning service company had \( \frac{3}{4} \) of a bottle of cleaning solution left at the end of a day's work. The next morning, the team decided to share the remaining solution equally among 5 cleaning staff before starting their jobs. 

(a) How much of the cleaning solution did each staff receive? 

(b) If each staff needed to use twice as much solution for their bigger job that day, how much cleaning solution would each staff need in total?

\textbf{Solution}

(a) To find the amount each staff received, divide the remaining cleaning solution by 5:

$
\frac{3}{4} \div 5 = \frac{3}{4} \times \frac{1}{5} = \frac{3}{20}
$

So, each staff received \(\frac{3}{20}\) of a bottle.

(b) If each staff needed twice as much:

$
2 \times \frac{3}{20} = \frac{6}{20} = \frac{3}{10}
$

Each staff would need \(\frac{3}{10}\) of a bottle for their bigger job.

\section*{Question 122}
\textbf{Metadata}

\begin{itemize}
  \item Question ID: P6-FrDivPP\_P2-FrCmp\_GPT4.1\_Services\_06
  \item Primary KC: FRACTIONS | Division | dividing a whole number/proper fraction by a proper fraction
  \item Secondary KC: FRACTIONS | Comparison and ordering | comparing and ordering fractions
  \item Topic: Services such as installation, maintenance, repairing, cleaning, laundry, e-commerce
  \item Grade: Primary 6
\end{itemize}

\textbf{Question}

A team of cleaners is tasked with cleaning several rooms in a hotel. It takes \( \frac{2}{3} \) of an hour to clean one room. 

The cleaning supervisor needs to prepare a schedule for cleaning 8 rooms.

(a) How many \( \frac{2}{3} \) hours are there in 8 hours?

(b) Arrange the following fractions in order from least to greatest, based on the answers in (a): \( \frac{8}{3}, \frac{12}{2}, \frac{24}{6} \).


\textbf{Solution}

(a) To find how many \( \frac{2}{3} \) hours are in 8 hours, we divide 8 by \( \frac{2}{3} \):

$
8 \div \frac{2}{3} = 8 \times \frac{3}{2} = 12
$

So, there are 12 intervals of \( \frac{2}{3} \) hours in 8 hours.

(b) Convert each fraction to their simplest form:
- \( \frac{8}{3} = 2 \frac{2}{3} \)
- \( \frac{12}{2} = 6 \)
- \( \frac{24}{6} = 4 \)

Now, order them from least to greatest:
$
\frac{8}{3} < \frac{24}{6} < \frac{12}{2}
$

So, the order is: \( \frac{8}{3}, \frac{24}{6}, \frac{12}{2} \).


\section*{Question 123}
\textbf{Metadata}

\begin{itemize}
  \item Question ID: P6-FrDivPP\_P2-FrAdd2nd\_GPT4.1\_Services\_01
  \item Primary KC: FRACTIONS | Division | dividing a whole number/proper fraction by a proper fraction
  \item Secondary KC: FRACTIONS | Addition | adding fractions
  \item Topic: Services such as installation, maintenance, repairing, cleaning, laundry, e-commerce
  \item Grade: Primary 6
\end{itemize}

\textbf{Question}

A cleaning company uses $\frac{3}{4}$ litres of detergent for every room it cleans. One morning, the company receives an order to clean 9 rooms in an office building. Before starting work, the supervisor finds that there are only 6 litres of detergent left. 

(a) How many rooms can the company clean with the 6 litres of detergent? 

(b) After using up the 6 litres, another $\frac{1}{2}$ litre of detergent is bought. What is the total amount of detergent the company has now? 

(c) Based on your answer in (b), how many more rooms in total can be cleaned now?

\textbf{Solution}

(a) Each room requires $\frac{3}{4}$ litres of detergent. To find how many rooms can be cleaned with 6 litres, divide 6 by $\frac{3}{4}$:

$6 \div \frac{3}{4} = 6 \times \frac{4}{3} = \frac{24}{3} = 8$ rooms.

(b) After using up the 6 litres, another $\frac{1}{2}$ litre is bought. The total amount of detergent now is $0 + \frac{1}{2} = \frac{1}{2}$ litre.

(c) With $\frac{1}{2}$ litre, the number of rooms that can be cleaned is:

$\frac{1}{2} \div \frac{3}{4} = \frac{1}{2} \times \frac{4}{3} = \frac{4}{6} = \frac{2}{3}$ room.

So, after using up the first 6 litres, only $\frac{2}{3}$ of a room can be cleaned with the $\frac{1}{2}$ litre of new detergent.

\section*{Question 124}
\textbf{Metadata}

\begin{itemize}
  \item Question ID: P6-FrDivPP\_P2-FrSub2nd\_GPT4.1\_Services\_01
  \item Primary KC: FRACTIONS | Division | dividing a whole number/proper fraction by a proper fraction
  \item Secondary KC: FRACTIONS | Subtraction | subtracting fractions
  \item Topic: Services such as installation, maintenance, repairing, cleaning, laundry, e-commerce
  \item Grade: Primary 6
\end{itemize}

\textbf{Question}

A cleaning company uses $\frac{2}{3}$ litres of cleaning solution for each classroom they clean. They have a total of 8 litres of cleaning solution in stock. 

(a) How many classrooms can they clean with the solution they currently have?

After using some of the solution to clean a certain number of classrooms, they realise they need to save $1\frac{1}{3}$ litres for an emergency.

(b) How many more litres of cleaning solution can they use before reaching the emergency reserve?

(c) With the remaining usable solution, how many more classrooms can they clean?

\textbf{Solution}

(a) To find out how many classrooms can be cleaned with 8 litres of solution, divide 8 by $\frac{2}{3}$:

$8 \div \frac{2}{3} = 8 \times \frac{3}{2} = 12$

So, they can clean 12 classrooms.

(b) To find out how much solution they can still use before reaching the emergency reserve, subtract the amount reserved from the total:

$8 - 1\frac{1}{3} = 8 - \frac{4}{3} = \frac{24}{3} - \frac{4}{3} = \frac{20}{3} = 6\frac{2}{3}$ litres

So, they can use $6\frac{2}{3}$ litres before reaching the reserve.

(c) With $6\frac{2}{3}$ litres, the number of classrooms they can clean is:

$6\frac{2}{3} \div \frac{2}{3} = \frac{20}{3} \div \frac{2}{3} = \frac{20}{3} \times \frac{3}{2} = \frac{20 \times 3}{3 \times 2} = \frac{60}{6} = 10$

So, with the remaining usable solution, they can clean 10 more classrooms.

\section*{Question 125}
\textbf{Metadata}

\begin{itemize}
  \item Question ID: P6-FrDivPP\_P3-FrSmp\_GPT4.1\_Services\_06
  \item Primary KC: FRACTIONS | Division | dividing a whole number/proper fraction by a proper fraction
  \item Secondary KC: FRACTIONS | Simplifying | expressing a fraction in its simplest form
  \item Topic: Services such as installation, maintenance, repairing, cleaning, laundry, e-commerce
  \item Grade: Primary 6
\end{itemize}

\textbf{Question}

Aisha is doing laundry for her family. She has 6 litres of detergent, and each washing cycle uses $\frac{3}{4}$ litre of detergent. 

(a) How many complete washing cycles can she run with the detergent she has?

(b) After dividing, express your answer as a fraction in its simplest form where appropriate.

\textbf{Solution}

Let the total amount of detergent be $6$ litres. Each cycle uses $\frac{3}{4}$ litre.

(a) The number of washing cycles Aisha can run is:

$6 \div \frac{3}{4} = 6 \times \frac{4}{3} = \frac{6 \times 4}{3} = \frac{24}{3} = 8$

So, Aisha can run 8 complete washing cycles.

(b) The working above is already in simplest form. The answer is **8 washing cycles**.

\section*{Question 126}
\textbf{Metadata}

\begin{itemize}
  \item Question ID: P6-FrDivPP\_P5-FrCnv2Dc\_GPT4.1\_Services\_06
  \item Primary KC: FRACTIONS | Division | dividing a whole number/proper fraction by a proper fraction
  \item Secondary KC: FRACTIONS | Conversion to decimals | expressing fractions as decimals
  \item Topic: Services such as installation, maintenance, repairing, cleaning, laundry, e-commerce
  \item Grade: Primary 6
\end{itemize}

\textbf{Question}

A cleaning company has 6 litres of cleaning solution. For each cleaning job, \( \frac{3}{4} \) litre of solution is used. 

(a) How many cleaning jobs can be completed with 6 litres of the solution?

(b) Express the amount of solution used for each job as a decimal.

\textbf{Solution}

(a) To find the number of cleaning jobs that can be completed, divide the total volume by the amount used per job: 

$
6 \div \frac{3}{4} = 6 \times \frac{4}{3} = \frac{24}{3} = 8.
$

So, 8 cleaning jobs can be completed.

(b) The amount of solution used for each job is \( \frac{3}{4} \) litre.
$
\frac{3}{4} = 0.75
$

Thus, each job uses 0.75 litre of cleaning solution.

\section*{Question 127}
\textbf{Metadata}

\begin{itemize}
  \item Question ID: P6-FrDivPP\_P5-FrMul2nd\_GPT4.1\_Services\_01
  \item Primary KC: FRACTIONS | Division | dividing a whole number/proper fraction by a proper fraction
  \item Secondary KC: FRACTIONS | Multiplication | fraction multiplication
  \item Topic: Services such as installation, maintenance, repairing, cleaning, laundry, e-commerce
  \item Grade: Primary 6
\end{itemize}

\textbf{Question}

A cleaning company uses $\frac{3}{4}$ of a bottle of cleaning solution for each classroom they clean.

One day, they need to clean 5 classrooms. After cleaning these classrooms, the company manager discovers that half of the total amount of cleaning solution used was actually mixed with water in a ratio of $\frac{1}{2}$ cleaning solution to $\frac{1}{2}$ water.

(a) How many bottles of cleaning solution did they use in total for the 5 classrooms?

(b) How many bottles of cleaning solution were mixed with water?

\textbf{Solution}

**(a) Find the total number of bottles used:**

Amount used per classroom: $\frac{3}{4}$ bottle
Number of classrooms: $5$

Total bottles used $= 5 \times \frac{3}{4} = \frac{15}{4}$ bottles

**(b) Find how many bottles were mixed with water:**

Half the total solution was mixed with water:
$\frac{1}{2} \times \frac{15}{4} = \frac{15}{8}$ bottles

**Final Answers:**
(a) $\frac{15}{4}$ bottles of cleaning solution were used in total.
(b) $\frac{15}{8}$ bottles of cleaning solution were mixed with water.

\section*{Question 128}
\textbf{Metadata}

\begin{itemize}
  \item Question ID: P6-PcFndWN\_P1-WNAdd2nd\_GPT4.1\_Services\_01
  \item Primary KC: PERCENTAGE | Finding the whole | finding the whole given a part and the percentage
  \item Secondary KC: WHOLE NUMBERS | Addition | adding whole numbers
  \item Topic: Services such as installation, maintenance, repairing, cleaning, laundry, e-commerce
  \item Grade: Primary 6
\end{itemize}

\textbf{Question}

Sarah runs a small business that offers cleaning services. One month, she received two orders: one for cleaning an office and another for cleaning a house. She completed the office cleaning first and charged $\$120$ for it. This amount made up $40\%$ of her total earnings for that month. After finishing the house cleaning, she earned her total for the month. 

(a) How much did Sarah earn in total from both cleaning jobs that month? 

(b) What was the amount Sarah received for cleaning the house?

\textbf{Solution}

Let Sarah's total earnings for the month be $x$ dollars.

(a) We know that $40\%$ of $x$ is $\$120$:
\$
0.4x = 120\\
x = \frac{120}{0.4} = 300
\$
So, Sarah earned $\$300$ in total from both jobs.

(b) The amount she received for cleaning the house is:
\$
\$300 - \$120 = \$180
\$

**Answer:**
(a) Sarah earned $\$300$ in total from both jobs.

(b) The amount received for cleaning the house was $\$180$.

\section*{Question 129}
\textbf{Metadata}

\begin{itemize}
  \item Question ID: P6-PcFndWN\_P1-WNSub2nd\_GPT4.1\_Services\_01
  \item Primary KC: PERCENTAGE | Finding the whole | finding the whole given a part and the percentage
  \item Secondary KC: WHOLE NUMBERS | Subtraction | subtracting whole numbers
  \item Topic: Services such as installation, maintenance, repairing, cleaning, laundry, e-commerce
  \item Grade: Primary 6
\end{itemize}

\textbf{Question}

A cleaning company offers a special package for office cleaning. This month, a client paid $\$150$ for the cleaning service, which covered 60\% of the total invoice after subtracting a $\$40$ discount the company gave for early payment. What was the original (whole) invoice amount before the discount was applied?

\textbf{Solution}

Let the original invoice amount be $x$.

After subtracting the discount, the client pays $x - 40$.

$\$150$ is 60\% of this amount:

$0.6(x - 40) = 150$

$\Rightarrow x - 40 = \frac{150}{0.6}$

$\Rightarrow x - 40 = 250$

$\Rightarrow x = 250 + 40 = 290$

So, the original invoice amount before the discount was \boxed{\$290}.

\section*{Question 130}
\textbf{Metadata}

\begin{itemize}
  \item Question ID: P6-PcFndWN\_P1-WNMul2nd\_GPT4.1\_Services\_01
  \item Primary KC: PERCENTAGE | Finding the whole | finding the whole given a part and the percentage
  \item Secondary KC: WHOLE NUMBERS | Multiplication | multiplying whole numbers
  \item Topic: Services such as installation, maintenance, repairing, cleaning, laundry, e-commerce
  \item Grade: Primary 6
\end{itemize}

\textbf{Question}

A cleaning company charges $42 to clean an office room. This amount is 35\% of the total weekly cleaning fee for an office building with multiple rooms. If there are 5 such office rooms in the building and each is cleaned once per week, what is the total weekly cleaning fee for the building?

Give your answer in dollars.

\textbf{Solution}

Let the total weekly cleaning fee for the building be $x$ dollars. We are told that the cleaning of one office room costs $42 and this is 35\% of the whole weekly cleaning fee.

So, $42 = 35\%$ of $x$.

Convert 35\% to its decimal form:
35\% = $\frac{35}{100} = 0.35$

So, $42 = 0.35x$

To find $x$:
$ x = \frac{42}{0.35} = 120 $

But this is the total weekly cleaning fee if there was only one office room. However, there are 5 rooms, each charged $42$ per week.

So, total cost for 5 rooms: $\ 42 \times 5 = 210$

Now, $210$ is 35\% of the total weekly cleaning fee for the entire building.
Let the total weekly cleaning fee be $y$.

$210 = 0.35y$
$ y = \frac{210}{0.35} = 600 $

So, the total weekly cleaning fee for the building is $600$.

\section*{Question 131}
\textbf{Metadata}

\begin{itemize}
  \item Question ID: P6-PcFndWN\_P1-WNDiv2nd\_GPT4.1\_Services\_01
  \item Primary KC: PERCENTAGE | Finding the whole | finding the whole given a part and the percentage
  \item Secondary KC: WHOLE NUMBERS | Division | dividing whole numbers
  \item Topic: Services such as installation, maintenance, repairing, cleaning, laundry, e-commerce
  \item Grade: Primary 6
\end{itemize}

\textbf{Question}

A cleaning company completed 20\% of its weekly laundry jobs on Monday, which amounts to 16 jobs. The company needs to assign the remaining jobs equally among the next 4 days (Tuesday to Friday). How many laundry jobs should be done each day from Tuesday to Friday?

\textbf{Solution}

Let the total number of laundry jobs for the week be $x$. Since 20\% of the jobs equals 16, we have:

$20\% \times x = 16$

$\Rightarrow \frac{20}{100} \times x = 16$

$\Rightarrow \frac{1}{5}x = 16$

$\Rightarrow x = 16 \times 5$

$\Rightarrow x = 80$

So, there are 80 laundry jobs in total.

The remaining jobs to be done from Tuesday to Friday is:

$80 - 16 = 64$

To assign these 64 jobs equally over the next 4 days:

$64 \div 4 = 16$

\textbf{Answer:} Each day from Tuesday to Friday, the company should complete 16 laundry jobs.

\section*{Question 132}
\textbf{Metadata}

\begin{itemize}
  \item Question ID: P6-PcFndChg\_P1-WNAdd2nd\_GPT4.1\_Services\_01
  \item Primary KC: PERCENTAGE | Finding change | finding percentage increase/decrease
  \item Secondary KC: WHOLE NUMBERS | Addition | adding whole numbers
  \item Topic: Services such as installation, maintenance, repairing, cleaning, laundry, e-commerce
  \item Grade: Primary 6
\end{itemize}

\textbf{Question}

A washing machine repair service charges $120 for a basic repair. If additional parts are needed, each part costs $30. Mrs Tan’s washing machine needed two additional parts for the repair. After the service, the company informed Mrs Tan that there would be a 15\% increase in the total cost due to an urgent service request. 

(a) What is the total cost before the percentage increase?

(b) What is the total cost after the 15\% increase?

\textbf{Solution}

Let’s solve step-by-step:

(a) To find the total cost before the percentage increase:

Cost of basic repair = $120 

Number of parts needed = 2 

Cost of additional parts = $30 \times 2 = $60 

Total cost before increase = $120 + $60 = $180

(b) To find the total cost after a 15\% increase:

Percentage increase = 15\% of $180 = \frac{15}{100} \times 180 = $27

Total cost after increase = $180 + $27 = $207

**Final Answers:**

(a) $180

(b) $207

\section*{Question 133}
\textbf{Metadata}

\begin{itemize}
  \item Question ID: P6-PcFndChg\_P1-WNSub2nd\_GPT4.1\_Services\_01
  \item Primary KC: PERCENTAGE | Finding change | finding percentage increase/decrease
  \item Secondary KC: WHOLE NUMBERS | Subtraction | subtracting whole numbers
  \item Topic: Services such as installation, maintenance, repairing, cleaning, laundry, e-commerce
  \item Grade: Primary 6
\end{itemize}

\textbf{Question}

A company charged Mrs Lim \$120 for installing an air-conditioner last year. This year, due to higher costs, the company increased the installation charge by \$18. \\ 
(a) What is the new installation charge this year? \\ 
(b) What is the percentage increase in the installation charge, correct to the nearest whole number?

\textbf{Solution}

(a) The new installation charge is: \\ 
\$120 + \$18 = \$138. \\ 
(b) Percentage increase = \frac{\text{increase}}{\text{original amount}} \times 100\% \\ 
= \frac{18}{120} \times 100\% = 15\%. \\ 
\text{Final Answers:} \\ 
(a) \$138 \\ 
(b) 15\%

\section*{Question 134}
\textbf{Metadata}

\begin{itemize}
  \item Question ID: P6-PcFndChg\_P1-WNMul2nd\_GPT4.1\_Services\_01
  \item Primary KC: PERCENTAGE | Finding change | finding percentage increase/decrease
  \item Secondary KC: WHOLE NUMBERS | Multiplication | multiplying whole numbers
  \item Topic: Services such as installation, maintenance, repairing, cleaning, laundry, e-commerce
  \item Grade: Primary 6
\end{itemize}

\textbf{Question}

A cleaning company charged $120 for a basic cleaning service last year. This year, the company increased its price by 15\%. A resident hires the company to clean her apartment 4 times this year. 

(a) What is the new price of the cleaning service this year?

(b) How much does the resident pay in total for 4 cleaning sessions this year?

\textbf{Solution}

(a) First, find the percentage increase:

15\% of $120 = \frac{15}{100} \times 120 = 18

The new price = Old price + Increase = 120 + 18 = $138

(b) The total cost for 4 cleaning sessions = 4 \times 138 = $552

\textbf{Final Answers:}
(a) The new price is $138 per cleaning session.
(b) The resident pays $552 in total for 4 sessions.

\section*{Question 135}
\textbf{Metadata}

\begin{itemize}
  \item Question ID: P6-PcFndChg\_P1-WNDiv2nd\_GPT4.1\_Services\_01
  \item Primary KC: PERCENTAGE | Finding change | finding percentage increase/decrease
  \item Secondary KC: WHOLE NUMBERS | Division | dividing whole numbers
  \item Topic: Services such as installation, maintenance, repairing, cleaning, laundry, e-commerce
  \item Grade: Primary 6
\end{itemize}

\textbf{Question}

A cleaning service company charged Mrs Lee $240 to clean her apartment last year. This year, the company divided the total service among 4 sessions and increased the price per session by 25\% due to rising costs. What is the new total amount Mrs Lee has to pay for cleaning her apartment this year?

\textbf{Solution}

First, let's find the price per session last year:

$\text{Last year's total charge} = \$240$

$\text{Number of sessions} = 4$

$\text{Price per session last year} = \frac{240}{4} = \$60$

Next, there is a 25\% increase in the price per session:

$\text{Amount of increase per session} = 25\% \times 60 = \frac{25}{100} \times 60 = 15$

$\text{New price per session} = 60 + 15 = \$75$

Finally, the new total amount for 4 sessions is:

$\text{New total amount} = 75 \times 4 = \$300$

\boxed{\$300} is the new total amount Mrs Lee has to pay for cleaning her apartment this year.

\section*{Question 136}
\textbf{Metadata}

\begin{itemize}
  \item Question ID: P6-RoFndDvqWN\_P6-RoFndTmWN\_GPT4.1\_Services\_06
  \item Primary KC: RATIO | Finding divided quantities | dividing a given quantity in a given ratio
  \item Secondary KC: RATIO | Finding a missing term | finding the missing term in a pair of equivalent ratios
  \item Topic: Services such as installation, maintenance, repairing, cleaning, laundry, e-commerce
  \item Grade: Primary 6
\end{itemize}

\textbf{Question}

A cleaning company has a contract to clean 180 offices. The company divides the task between Team A and Team B in the ratio of $2:3$.

(a) How many offices does each team need to clean?

Later, Team A finishes its part faster and the manager wants Team B to clean the remaining offices at the same speed as Team A. It is found that Team B can clean 18 offices in the same time Team A cleans 12 offices. If Team B continues at this rate, how many offices can Team B clean when Team A finishes 28 offices?

\textbf{Solution}

(a) Let the number of offices Team A needs to clean be $2x$ and Team B be $3x$.

$2x + 3x = 180$
$5x = 180$
$x = 36$

Team A: $2 \times 36 = 72$ offices
Team B: $3 \times 36 = 108$ offices

(b) Team B: Team A = $18:12$
To find how many offices Team B can clean when Team A finishes 28 offices:

First, simplify the ratio:
$18:12 = 3:2$

Let Team B's offices be $y$ when Team A finishes 28 offices:
$y:28 = 3:2$

So $\frac{y}{28} = \frac{3}{2}$
$y = 28 \times \frac{3}{2} = 42$

Therefore, when Team A finishes 28 offices, Team B can clean 42 offices.

\section*{Question 137}
\textbf{Metadata}

\begin{itemize}
  \item Question ID: P6-RoFndRoWN\_P1-WNAdd2nd\_GPT4.1\_Services\_01
  \item Primary KC: RATIO | Finding ratio | finding the ratio of two or three given whole numbers
  \item Secondary KC: WHOLE NUMBERS | Addition | adding whole numbers
  \item Topic: Services such as installation, maintenance, repairing, cleaning, laundry, e-commerce
  \item Grade: Primary 6
\end{itemize}

\textbf{Question}

A cleaning company received 16 requests for home cleaning and 24 requests for office cleaning in one week. After adding 10 more home cleaning requests that came in later, what is the ratio of the number of home cleaning requests to the number of office cleaning requests? Express your answer in its simplest form.

\textbf{Solution}

First, add the additional home cleaning requests: 

16 (original home requests) + 10 (new requests) = 26 home cleaning requests.

The number of office cleaning requests is 24.

The ratio of the number of home cleaning requests to office cleaning requests is:

$26:24$

To simplify the ratio, divide both numbers by their highest common factor, which is 2:

$\frac{26}{2} : \frac{24}{2} = 13:12$

So, the simplest form of the ratio is $13:12$.

\section*{Question 138}
\textbf{Metadata}

\begin{itemize}
  \item Question ID: P6-RoFndRoWN\_P1-WNSub2nd\_GPT4.1\_Services\_01
  \item Primary KC: RATIO | Finding ratio | finding the ratio of two or three given whole numbers
  \item Secondary KC: WHOLE NUMBERS | Subtraction | subtracting whole numbers
  \item Topic: Services such as installation, maintenance, repairing, cleaning, laundry, e-commerce
  \item Grade: Primary 6
\end{itemize}

\textbf{Question}

A laundry company received 85 shirts and 65 trousers to wash on Monday. After finishing all the washing on Monday, they found that 18 shirts and 12 trousers had stains and needed to be washed again on Tuesday. 

(a) How many shirts and trousers were clean after Monday?

(b) Find the ratio of the number of clean shirts to clean trousers after Monday, in its simplest form.

\textbf{Solution}

(a) Number of clean shirts = 85 - 18 = 67\
Number of clean trousers = 65 - 12 = 53

(b) The ratio of the number of clean shirts to clean trousers is 67 : 53. 

This ratio is already in its simplest form, since 67 and 53 have no common factors other than 1.

\textbf{Final answers:}
(a) 67 shirts and 53 trousers were clean after Monday.
(b) The ratio of clean shirts to clean trousers is 67 : 53.

\section*{Question 139}
\textbf{Metadata}

\begin{itemize}
  \item Question ID: P6-RoFndRoWN\_P1-WNMul2nd\_GPT4.1\_Services\_01
  \item Primary KC: RATIO | Finding ratio | finding the ratio of two or three given whole numbers
  \item Secondary KC: WHOLE NUMBERS | Multiplication | multiplying whole numbers
  \item Topic: Services such as installation, maintenance, repairing, cleaning, laundry, e-commerce
  \item Grade: Primary 6
\end{itemize}

\textbf{Question}

A cleaning company received 48 requests for window cleaning and 72 requests for floor cleaning in a month. 

(a) Find the ratio of window cleaning requests to floor cleaning requests in its simplest form. 

(b) If each window cleaning request takes 2 hours and each floor cleaning request takes 3 hours, what is the total number of hours needed to complete all the cleaning requests?

\textbf{Solution}

(a) The ratio of window cleaning requests to floor cleaning requests is $48:72$. 

To simplify, divide both numbers by 24: 
$\frac{48}{24} : \frac{72}{24} = 2:3$.

So, the ratio is $2:3$.

(b) Total hours for window cleaning: $48 \times 2 = 96$ hours.

Total hours for floor cleaning: $72 \times 3 = 216$ hours.

Total number of hours needed: $96 + 216 = 312$ hours.

\section*{Question 140}
\textbf{Metadata}

\begin{itemize}
  \item Question ID: P6-RoFndRoWN\_P1-WNDiv2nd\_GPT4.1\_Services\_01
  \item Primary KC: RATIO | Finding ratio | finding the ratio of two or three given whole numbers
  \item Secondary KC: WHOLE NUMBERS | Division | dividing whole numbers
  \item Topic: Services such as installation, maintenance, repairing, cleaning, laundry, e-commerce
  \item Grade: Primary 6
\end{itemize}

\textbf{Question}

A cleaning company sends 15 workers to clean three office buildings. The number of workers sent to Building A, Building B, and Building C are in the ratio 2 : 3 : 5. 

(a) Find the number of workers sent to each building.

(b) If the number of workers sent to Building C is equally divided into 5 cleaning teams, how many workers are there in each team?

\textbf{Solution}

(a) Let the number of workers sent to Building A, B, and C be 2x, 3x, and 5x respectively.

The total ratio parts = 2 + 3 + 5 = 10 parts

Since 15 workers in total = 10 parts,

1 part = 15 \div 10 = 1.5 workers

\text{Number of workers in Building A} = 2 \times 1.5 = 3

\text{Number of workers in Building B} = 3 \times 1.5 = 4.5

\text{Number of workers in Building C} = 5 \times 1.5 = 7.5

However, since the number of workers must be whole numbers, the possible distribution could be:

- A: 3 workers
- B: 4 or 5 workers (assigning the half worker to B)
- C: 8 workers

But based on the strict ratio division with whole numbers, the closest numbers that fit the total and maintain ratio are A: 3, B: 4, C: 8.

(b) Number of workers in Building C = 8. The workers are divided equally into 5 cleaning teams:

Number of workers in each team = 8 \div 5 = 1.6 workers.

Since we cannot have a fraction of a worker, the problem may intend that 3 teams have 2 workers and 2 teams have 1 worker each. So, each team has either 1 or 2 workers.

\section*{Question 141}
\textbf{Metadata}

\begin{itemize}
  \item Question ID: P6-RoFndRoWN\_P6-RoSmpWN\_GPT4.1\_Services\_06
  \item Primary KC: RATIO | Finding ratio | finding the ratio of two or three given whole numbers
  \item Secondary KC: RATIO | Simplifying | expressing a ratio in its simplest form
  \item Topic: Services such as installation, maintenance, repairing, cleaning, laundry, e-commerce
  \item Grade: Primary 6
\end{itemize}

\textbf{Question}

A cleaning company sent 12 staff members to clean an office building. Out of these staff, 7 were responsible for cleaning offices, 3 were responsible for cleaning the toilets, and the rest were assigned to clean the corridors. 

(a) Find the ratio of the number of staff cleaning offices to those cleaning toilets to those cleaning corridors. 

(b) Express this ratio in its simplest form.

\textbf{Solution}

(a) Number of staff cleaning offices = 7 
Number of staff cleaning toilets = 3 
Total staff = 12 
Number of staff cleaning corridors = 12 - 7 - 3 = 2 

So, the ratio of staff cleaning offices : toilets : corridors is 7 : 3 : 2.

(b) To simplify, we look for the highest common factor among 7, 3, and 2. Since 7, 3, and 2 have no common factors other than 1, the ratio is already in its simplest form: 

$\boxed{7 : 3 : 2}$

\section*{Question 142}
\textbf{Metadata}

\begin{itemize}
  \item Question ID: P6-AgRepLrEx\_P6-AgSmpLrEx\_GPT4.1\_Services\_06
  \item Primary KC: ALGEBRA | Representation and concept | translation of simple real-world situations into linear algebraic expressions
  \item Secondary KC: ALGEBRA | Simplifying | simplifying simple linear expressions
  \item Topic: Services such as installation, maintenance, repairing, cleaning, laundry, e-commerce
  \item Grade: Primary 6
\end{itemize}

\textbf{Question}

A company charges $8 for every air conditioner it installs, plus a fixed service fee of $25. 

(a) If the company installs $x$ air conditioners for a customer, write an algebraic expression for the total amount the customer has to pay.

(b) Simplify the expression you have written in part (a).

\textbf{Solution}

(a) Let $x$ be the number of air conditioners installed.

The total cost is made up of the installation fee for each air conditioner plus the fixed service fee.

So, **Total cost** $= 8x + 25$

(b) The expression $8x + 25$ is already in its simplest form. Therefore, the simplified algebraic expression for the total amount is **$8x + 25$**.

\section*{Question 143}
\textbf{Metadata}

\begin{itemize}
  \item Question ID: P6-AgRepLrEx\_P6-AgEvlLrEx\_GPT4.1\_Services\_06
  \item Primary KC: ALGEBRA | Representation and concept | translation of simple real-world situations into linear algebraic expressions
  \item Secondary KC: ALGEBRA | Evaluation | evaluating simple linear expressions by substitution
  \item Topic: Services such as installation, maintenance, repairing, cleaning, laundry, e-commerce
  \item Grade: Primary 6
\end{itemize}

\textbf{Question}

A cleaning company charges \$15 for each room they clean and a fixed fee of \$25 for each visit. Let $x$ be the number of rooms cleaned in one visit. 

(a) Write down an algebraic expression for the total cost, in dollars, for cleaning $x$ rooms in one visit.

(b) Find the total cost if they clean 4 rooms in one visit.

\textbf{Solution}

(a) The cost for cleaning $x$ rooms is \$15 per room, which is $15x$. There is also a fixed fee of \$25 per visit. So, the total cost is: 

Total cost $= 15x + 25$

(b) When $x = 4$:

Total cost $= 15(4) + 25 = 60 + 25 = \$85$

So, the total cost for cleaning 4 rooms in one visit is \$85.

\section*{Question 144}
\textbf{Metadata}

\begin{itemize}
  \item Question ID: P6-AgSlvLrN\_P6-AgRepLrEx\_GPT4.1\_Services\_06
  \item Primary KC: ALGEBRA | Solving simple linear equations | solving simple linear equations involving whole number coefficient and one variable only
  \item Secondary KC: ALGEBRA | Representation and concept | translation of simple real-world situations into linear algebraic expressions
  \item Topic: Services such as installation, maintenance, repairing, cleaning, laundry, e-commerce
  \item Grade: Primary 6
\end{itemize}

\textbf{Question}

A washing machine repair company charges a fixed fee of \$30 for each service, plus \$12 for every hour spent repairing a machine. If the total bill for repairing Mrs. Tan's washing machine was \$78, write an equation to represent this situation. Then, find out how many hours the company spent repairing Mrs. Tan's washing machine.

\textbf{Solution}

Let $x$ be the number of hours spent repairing.\
The total cost is: \$30 + 12x = 78.\
Subtract 30 from both sides: \\ 12x = 48.\
Divide both sides by 12: \\ x = 4.\
\textbf{The company spent 4 hours repairing Mrs. Tan's washing machine.}

\section*{Question 145}
\textbf{Metadata}

\begin{itemize}
  \item Question ID: O1-FDSub\_O1-FDAdd\_GPT4.1\_Services\_02
  \item Primary KC: FRACTIONS AND DECIMALS | Subtraction | Subtraction involving one fraction and one decimal number
  \item Secondary KC: FRACTIONS AND DECIMALS | Addition | Addition involving one fraction and one decimal number
  \item Topic: Services such as installation, maintenance, repairing, cleaning, laundry, e-commerce
  \item Grade: Secondary O-level 1
\end{itemize}

\textbf{Question}

A cleaning company charges $10.50$ dollars for a basic cleaning service and $\frac{1}{4}$ of the price of the basic package for sanitising the kitchen. If Mrs Lim paid $15$ dollars in total for both services, how much more did she pay for the sanitising service compared to the basic cleaning service?

\textbf{Solution}

Let the cost of the sanitising service be $x$ dollars. 

The total paid is:
$
10.50 + x = 15
$
$
x = 15 - 10.50 = 4.50
$

We are told that the cost of sanitising is $\frac{1}{4}$ of the price of the basic package:
$
x = \frac{1}{4} \times 10.50 = \frac{10.50}{4} = 2.625
$

However, $x$ found from the total paid is $4.50$ dollars, which means the sanitising service in reality cost $4.50$. To find *how much more* Mrs Lim paid for the sanitising service compared to the basic cleaning service:
$
\text{Amount more paid} = 4.50 - 2.625 = 1.875
$

But the question actually asks: 'how much more did she pay for the sanitising service compared to the basic cleaning service?'. Since the basic cleaning service costs $10.50$, and the sanitising service costs $4.50$:
$
\text{Difference} = 10.50 - 4.50 = 6.00
$

Therefore, Mrs Lim paid $\$6.00$ more for the basic cleaning service compared to the sanitising service.

\section*{Question 146}
\textbf{Metadata}

\begin{itemize}
  \item Question ID: O1-FDMul\_O1-FDAdd\_GPT4.1\_Services\_02
  \item Primary KC: FRACTIONS AND DECIMALS | Multiplication | Multiplication involving one fraction and one decimal number
  \item Secondary KC: FRACTIONS AND DECIMALS | Addition | Addition involving one fraction and one decimal number
  \item Topic: Services such as installation, maintenance, repairing, cleaning, laundry, e-commerce
  \item Grade: Secondary O-level 1
\end{itemize}

\textbf{Question}

A window cleaning company charges \( \frac{2}{3} \) times the base rate for cleaning windows that are above the second floor. The base rate per window is \$5.80. If Anna wants to clean 4 windows above the second floor and also wants an additional ground floor window cleaned, which costs \$3.50, what is the total amount she will have to pay for all the cleaning? Give your answer in dollars and cents.

\textbf{Solution}

First, calculate the cost for one window above the second floor: 

$
\text{Cost per window above 2nd floor} = \frac{2}{3} \times 5.80 = \frac{2 \times 5.80}{3}
$

Calculate \(2 \times 5.80 = 11.60\). So,
$
\frac{11.60}{3} = 3.8666... \approx \$3.87
$

There are 4 such windows:
$
4 \times 3.87 = \$15.48
$

Cost for the additional ground floor window: \$3.50.

Total cost:
$
\$15.48 + \$3.50 = \$18.98
$

So, Anna will have to pay \$18.98 in total for all the cleaning.

\section*{Question 147}
\textbf{Metadata}

\begin{itemize}
  \item Question ID: O1-FDMul\_O1-FDSub\_GPT4.1\_Services\_02
  \item Primary KC: FRACTIONS AND DECIMALS | Multiplication | Multiplication involving one fraction and one decimal number
  \item Secondary KC: FRACTIONS AND DECIMALS | Subtraction | Subtraction involving one fraction and one decimal number
  \item Topic: Services such as installation, maintenance, repairing, cleaning, laundry, e-commerce
  \item Grade: Secondary O-level 1
\end{itemize}

\textbf{Question}

A cleaning service charges $\frac{2}{3}$ times the standard hourly rate for cleaning apartments booked through an online app. The standard hourly rate is \$18.50. If Ali hires the service and his total bill (before discount) for cleaning is \$44.40, by how much did Ali save due to the app discount? Express your answer in dollars, giving your answer as a decimal number.

\textbf{Solution}

First, calculate the discounted hourly rate:  

$\frac{2}{3} \times 18.50 = \frac{2 \times 18.50}{3} = \frac{37}{3} = 12.33\overline{3}$ dollars per hour.

Let $h$ be the number of hours of cleaning Ali hired. From the normal rate:

$18.50 \times h = 44.40$

$h = \frac{44.40}{18.50} = 2.4$ hours

Now calculate the discounted bill:

Discounted bill $= 12.33\overline{3} \times 2.4 = 29.6$

Ali's savings:

Savings $= 44.40 - 29.60 = 14.80$

Ali saved \$14.80 due to the app discount.

\section*{Question 148}
\textbf{Metadata}

\begin{itemize}
  \item Question ID: O1-FDDiv\_O1-FDAdd\_GPT4.1\_Services\_02
  \item Primary KC: FRACTIONS AND DECIMALS | Division | Division involving one fraction and one decimal number
  \item Secondary KC: FRACTIONS AND DECIMALS | Addition | Addition involving one fraction and one decimal number
  \item Topic: Services such as installation, maintenance, repairing, cleaning, laundry, e-commerce
  \item Grade: Secondary O-level 1
\end{itemize}

\textbf{Question}

A professional cleaning service charges 2.4 hours to clean a small office. Sarah worked for $1\frac{1}{2}$ hours on one office and another $0.6$ hours helping to clean a conference room. 

(a) What is the total time, in hours, that Sarah worked on both rooms? 

(b) The supervisor wants to know how many offices like the small office could be cleaned with the total time Sarah spent. Give your answer as a decimal up to 2 decimal places.

\textbf{Solution}

Let’s solve the problem step by step:

(a) The total time worked by Sarah:

Sarah worked $1\frac{1}{2}$ hours (which is $1.5$ hours) and $0.6$ hours.

Total time $= 1.5 + 0.6 = 2.1$ hours

(b) Number of small offices that can be cleaned with $2.1$ hours, if one office needs $2.4$ hours:

Number of offices $= \dfrac{2.1}{2.4}$

Write $2.1$ as a decimal and $2.4$ is already a decimal:

Number of offices $= \dfrac{2.1}{2.4}$

Divide $2.1 \div 2.4$

$2.1 \div 2.4 = 0.875$

Thus, Sarah could have cleaned $0.88$ (to 2 decimal places) small offices with the total time worked.

**Final Answers:**

(a) $2.1$ hours

(b) $0.88$ offices (rounded to 2 decimal places)

\section*{Question 149}
\textbf{Metadata}

\begin{itemize}
  \item Question ID: O1-FDDiv\_O1-FDSub\_GPT4.1\_Services\_02
  \item Primary KC: FRACTIONS AND DECIMALS | Division | Division involving one fraction and one decimal number
  \item Secondary KC: FRACTIONS AND DECIMALS | Subtraction | Subtraction involving one fraction and one decimal number
  \item Topic: Services such as installation, maintenance, repairing, cleaning, laundry, e-commerce
  \item Grade: Secondary O-level 1
\end{itemize}

\textbf{Question}

A cleaning company charges \(1.2\) litres of cleaning solution for every job they do. After completing a big project, they realised they had used \( \frac{9}{4} \) litres of cleaning solution. 

(a) How many jobs did they complete if all jobs used the same amount of solution? Give your answer correct to 2 decimal places.

After the last job, the company noticed the leftover cleaning solution in the bottle was \(1.2 - 0.75\) litres.

(b) How much cleaning solution (in litres) was left after the last job?

\textbf{Solution}

\textbf{Part (a):}

Given:
- \(1.2\) litres are used per job
- Total cleaning solution used: \(\frac{9}{4}\) litres

To find: Number of jobs = Total used \( \div \) Amount per job

Number of jobs = \( \frac{9}{4} \div 1.2 \)

First, convert \(1.2\) to a fraction:
$
1.2 = \frac{12}{10} = \frac{6}{5}
$
Now, perform the division:
$
\frac{9}{4} \div \frac{6}{5} = \frac{9}{4} \times \frac{5}{6} = \frac{9 \times 5}{4 \times 6} = \frac{45}{24} = \frac{15}{8} = 1.875
$

\textbf{Answer to (a):} The company completed \(1.88\) jobs (to 2 decimal places).

\textbf{Part (b):}
Leftover cleaning solution after the last job:
$
1.2 - 0.75 = 0.45 \text{ litres}
$

\textbf{Answer to (b):} \(0.45\) litres of cleaning solution was left after the last job.

\section*{Question 150}
\textbf{Metadata}

\begin{itemize}
  \item Question ID: O1-FDDiv\_O1-FDMul\_GPT4.1\_Services\_02
  \item Primary KC: FRACTIONS AND DECIMALS | Division | Division involving one fraction and one decimal number
  \item Secondary KC: FRACTIONS AND DECIMALS | Multiplication | Multiplication involving one fraction and one decimal number
  \item Topic: Services such as installation, maintenance, repairing, cleaning, laundry, e-commerce
  \item Grade: Secondary O-level 1
\end{itemize}

\textbf{Question}

A cleaning company uses 0.75 litres of cleaning solution for every $\frac{3}{4}$ of a room they clean. 

(a) How many litres of cleaning solution does the company use to clean 3 rooms?

(b) If the company has 9 litres of cleaning solution left, how many full rooms can they clean with this amount?


\textbf{Solution}

Let’s solve the word problem step by step:

(a) **Total solution used to clean 3 rooms:**

First, determine the amount of solution used to clean 1 room:
- For $\frac{3}{4}$ of a room: 0.75 litres are used

To find solution for 1 full room:

$\text{For 1 room} = 0.75 \div \frac{3}{4} = 0.75 \times \frac{4}{3} = \frac{3}{4} \times \frac{4}{3}$

But, it's $0.75 \div \frac{3}{4} = $ (Convert $0.75$ to fraction: $\frac{3}{4}$)

So $\frac{3}{4} \div \frac{3}{4} = 1$ litre for 1 room.

To clean 3 rooms:
$1 \times 3 = 3$ litres

**Answer to (a):** The company uses 3 litres of cleaning solution to clean 3 rooms.

---

(b) **Finding how many full rooms can be cleaned with 9 litres:**
- Amount needed for 1 room = 1 litre (from above)
- $9 \div 1 = 9$

**Answer to (b):** The company can clean 9 full rooms with 9 litres of cleaning solution.


\section*{Question 151}
\textbf{Metadata}

\begin{itemize}
  \item Question ID: O1-RoRepFr\_P2-FrAdd2nd\_GPT4.1\_Services\_01
  \item Primary KC: RATIO | Representation and concept | ratios involving fractions
  \item Secondary KC: FRACTIONS | Addition | adding fractions
  \item Topic: Services such as installation, maintenance, repairing, cleaning, laundry, e-commerce
  \item Grade: Secondary O-level 1
\end{itemize}

\textbf{Question}

Mrs Tan hired a cleaning service to clean her house. On Monday, the cleaners spent $\frac{3}{8}$ of the total cleaning hours in the living room and $\frac{1}{4}$ of the total hours in the kitchen. The remaining time was spent cleaning the bedrooms. 

(a) What fraction of the total cleaning hours was spent cleaning the bedrooms?

(b) What is the ratio of the fraction of hours spent cleaning the living room to the fraction of hours spent cleaning the bedrooms? Express your answer in its simplest form.

\textbf{Solution}

(a) Time spent in the living room: $\frac{3}{8}$ of total hours

Time spent in the kitchen: $\frac{1}{4}$ of total hours

Total fraction spent cleaning living room and kitchen:

$\frac{3}{8} + \frac{1}{4} = \frac{3}{8} + \frac{2}{8} = \frac{5}{8}$

Fraction spent cleaning bedrooms:

$1 - \frac{5}{8} = \frac{8}{8} - \frac{5}{8} = \frac{3}{8}$

So, $\boxed{\frac{3}{8}}$ of the total cleaning hours was spent cleaning the bedrooms.

(b) Ratio of living room to bedrooms:

Fraction spent in living room : fraction spent in bedrooms = $\frac{3}{8} : \frac{3}{8}$

This simplifies to $1 : 1$.

So, the ratio is $\boxed{1 : 1}$.

\section*{Question 152}
\textbf{Metadata}

\begin{itemize}
  \item Question ID: O1-RoRepFr\_P2-FrSub2nd\_GPT4.1\_Services\_01
  \item Primary KC: RATIO | Representation and concept | ratios involving fractions
  \item Secondary KC: FRACTIONS | Subtraction | subtracting fractions
  \item Topic: Services such as installation, maintenance, repairing, cleaning, laundry, e-commerce
  \item Grade: Secondary O-level 1
\end{itemize}

\textbf{Question}

A washing service offers two types of laundry detergents: Premium and Standard. For a certain load of laundry, the recommended amount of detergent to use is given in the ratio $\frac{2}{3} : \frac{1}{3}$ (Premium : Standard) by volume. If a total of $\frac{4}{5}$ litres of detergent is to be used for a load, and the amount of Premium detergent used is $\frac{1}{2}$ litre more than the amount of Standard detergent used, how many litres of Standard detergent are needed for the load?


\textbf{Solution}

Let the amount of Premium detergent used be $P$ litres and the amount of Standard detergent used be $S$ litres.

The ratio $P : S = \frac{2}{3} : \frac{1}{3}$. This means:
$$
\frac{P}{S} = \frac{\frac{2}{3}}{\frac{1}{3}} = 2
$$
So $P = 2S$.

We know that the total amount is:
$$
P + S = \frac{4}{5}
$$
And also,
$$
P = S + \frac{1}{2}
$$

Substitute $P = 2S$ into $P = S + \frac{1}{2}$:
$$
2S = S + \frac{1}{2} \\
2S - S = \frac{1}{2} \\
S = \frac{1}{2}
$$

Therefore, the amount of Standard detergent used is $\boxed{\frac{1}{2}}$ litres.

\section*{Question 153}
\textbf{Metadata}

\begin{itemize}
  \item Question ID: O1-RoRepFr\_P5-FrMul2nd\_GPT4.1\_Services\_01
  \item Primary KC: RATIO | Representation and concept | ratios involving fractions
  \item Secondary KC: FRACTIONS | Multiplication | fraction multiplication
  \item Topic: Services such as installation, maintenance, repairing, cleaning, laundry, e-commerce
  \item Grade: Secondary O-level 1
\end{itemize}

\textbf{Question}

A laundry service uses a special detergent mixture. For every $\frac{2}{5}$ litre of concentrated detergent, they mix it with $3$ times as much water by volume. If a customer asks for a batch with $\frac{3}{4}$ litre of concentrated detergent, how much water should the service add to maintain the ratio? Express your answer as a fraction in its simplest form.

\textbf{Solution}

The ratio of concentrated detergent to water is $\frac{2}{5}:3 \times \frac{2}{5}$, or $\frac{2}{5}:\frac{6}{5}$.

The ratio means that for every $\frac{2}{5}$ litre of detergent, $\frac{6}{5}$ litres of water are needed.

If a customer wants a batch with $\frac{3}{4}$ litre of detergent, we first need to find out how many times $\frac{2}{5}$ fits into $\frac{3}{4}$:

Number of 'units': $\frac{3}{4} \div \frac{2}{5} = \frac{3}{4} \times \frac{5}{2} = \frac{15}{8}$.

So, the total water needed is $\frac{15}{8}$ times $\frac{6}{5}$:

$\frac{15}{8} \times \frac{6}{5} = \frac{15 \times 6}{8 \times 5} = \frac{90}{40} = \frac{9}{4}$.

Therefore, the service should add $\boxed{\frac{9}{4}}$ litres of water.

\section*{Question 154}
\textbf{Metadata}

\begin{itemize}
  \item Question ID: O1-RoRepFr\_P6-FrDiv2nd\_GPT4.1\_Services\_01
  \item Primary KC: RATIO | Representation and concept | ratios involving fractions
  \item Secondary KC: FRACTIONS | Division | fraction division
  \item Topic: Services such as installation, maintenance, repairing, cleaning, laundry, e-commerce
  \item Grade: Secondary O-level 1
\end{itemize}

\textbf{Question}

A cleaning company provides a special washing service for curtains. For every $\frac{3}{4}$ hour it spends cleaning, it uses detergent and water in the ratio of $\frac{2}{3} : 1$ by volume. If a job took the company $3\frac{1}{2}$ hours to complete, and the company used a total of $7$ litres of water, how many litres of detergent did they use for this job?

\textbf{Solution}

First, determine how many $\frac{3}{4}$ hour intervals are in $3\frac{1}{2}$ hours:

$3\frac{1}{2} \div \frac{3}{4} = \frac{7}{2} \div \frac{3}{4} = \frac{7}{2} \times \frac{4}{3} = \frac{28}{6} = \frac{14}{3}$

So, there are $\frac{14}{3}$ intervals of $\frac{3}{4}$ hour each.

The detergent to water ratio by volume is $\frac{2}{3} : 1$. That means, for every $\frac{2}{3}$ litre of detergent, $1$ litre of water is used.

Let the volume of water used per interval be $x$ litres.

Over $\frac{14}{3}$ intervals, the total water used is:

$\left( \frac{14}{3} \right)x = 7$ litres

Solving for $x$:

$x = 7 \times \frac{3}{14} = \frac{21}{14} = 1.5$

So, in each $\frac{3}{4}$ hour interval, $1.5$ litres of water are used. The amount of detergent per interval is proportional to the ratio:

If water is $1$ litre, detergent is $\frac{2}{3}$ litre
So, if water is $1.5$ litres:

Detergent $= 1.5 \times \frac{2}{3} = 1$ litre$\/$interval

Total detergent used over $\frac{14}{3}$ intervals:

$\frac{14}{3} \times 1 = \frac{14}{3}$ litres $

\rightarrow$ $\frac{14}{3} = 4\frac{2}{3}$ litres

\textbf{Final Answer:}

The company used $\boxed{4\frac{2}{3}}$ litres of detergent for this job.

\section*{Question 155}
\textbf{Metadata}

\begin{itemize}
  \item Question ID: O1-RoRepFr\_O1-RoSmpFr\_GPT4.1\_Services\_02
  \item Primary KC: RATIO | Representation and concept | ratios involving fractions
  \item Secondary KC: RATIO | Simplifying | converting a ratio involving fractions to its simplest form
  \item Topic: Services such as installation, maintenance, repairing, cleaning, laundry, e-commerce
  \item Grade: Secondary O-level 1
\end{itemize}

\textbf{Question}

A laundry shop received two types of detergents, A and B, to use for washing clothes. For every $\frac{3}{4}$ litre of detergent A, the shop used $\frac{1}{2}$ litre of detergent B in the same washing cycle.

(a) Express the amount of detergent A used to the amount of detergent B used as a ratio in the form $a:b$ where $a$ and $b$ are fractions.

(b) Simplify this ratio to its simplest whole number form.

\textbf{Solution}

(a) Amount of detergent A : Amount of detergent B = $\frac{3}{4}:\frac{1}{2}$

(b) To simplify $\frac{3}{4}:\frac{1}{2}$:

Step 1: Express both fractions with a common denominator if needed.

$\frac{3}{4}:\frac{1}{2}$

Step 2: Convert to a:b form by dividing both terms by $\frac{1}{2}$:

$\frac{3}{4} \div \frac{1}{2} : \frac{1}{2} \div \frac{1}{2}$

$= (\frac{3}{4} \times \frac{2}{1}) : 1$

$= (\frac{6}{4}) : 1$

$= (\frac{3}{2}) : 1$

Now, multiply both sides by 2 to convert the ratio into the simplest whole number form:

$(\frac{3}{2} \times 2) : (1 \times 2) = 3 : 2$

Final Answer:

(a) The ratio of detergent A used to detergent B used is $\frac{3}{4}:\frac{1}{2}$.

(b) The simplest whole number form of the ratio is $3:2$.

\section*{Question 156}
\textbf{Metadata}

\begin{itemize}
  \item Question ID: O1-RoRepDc\_P4-DcAdd2nd\_GPT4.1\_Services\_01
  \item Primary KC: RATIO | Representation and concept | ratios involving decimals
  \item Secondary KC: DECIMALS | Addition | adding decimals
  \item Topic: Services such as installation, maintenance, repairing, cleaning, laundry, e-commerce
  \item Grade: Secondary O-level 1
\end{itemize}

\textbf{Question}

A company offers a cleaning service where 0.6 litres of cleaning liquid is needed for every 1.5 m$^2$ of floor area. Chloe is cleaning two rooms, Room A and Room B. Room A has an area of 3.2 m$^2$ and Room B has an area of 2.8 m$^2$.

(a) What is the total area, in m$^2$, that Chloe needs to clean?

(b) Using the ratio given, how many litres of cleaning liquid will Chloe need in total to clean both rooms?

(c) If Chloe only has a bottle containing 2.5 litres of cleaning liquid, does she have enough to clean both rooms? Show your working.

\textbf{Solution}

Let us solve each part step by step.

(a) Total area to clean:
$
\text{Total area} = 3.2\ \text{m}^2 + 2.8\ \text{m}^2 = 6.0\ \text{m}^2
$

(b) The ratio is 0.6 litres : 1.5 m$^2$. First, find how many litres are needed for 6.0 m$^2$.

Find the amount of liquid needed per 1 m$^2$:
$
\frac{0.6}{1.5} = 0.4\ \text{litres per m}^2
$

So for 6.0 m$^2$:
$
0.4 \times 6.0 = 2.4\ \text{litres}
$

(c) Chloe has 2.5 litres.

Amount required = 2.4 litres.

Since 2.5 litres > 2.4 litres, \textbf{Chloe has enough cleaning liquid} to clean both rooms.

\section*{Question 157}
\textbf{Metadata}

\begin{itemize}
  \item Question ID: O1-RoRepDc\_P4-DcSub2nd\_GPT4.1\_Services\_01
  \item Primary KC: RATIO | Representation and concept | ratios involving decimals
  \item Secondary KC: DECIMALS | Subtraction | subtracting decimals
  \item Topic: Services such as installation, maintenance, repairing, cleaning, laundry, e-commerce
  \item Grade: Secondary O-level 1
\end{itemize}

\textbf{Question}

At a laundry shop, the ratio of the mass of detergent (in kg) to the mass of clothes (in kg) used for a wash is $0.15:2.4$. 

(a) If a customer brings in $9.6$ kg of clothes, how much detergent (in kg) should be used for the wash, according to the shop's ratio?

(b) After the wash, $0.58$ kg of detergent is left unused from the amount prepared. How much detergent (in kg) was actually used?

\textbf{Solution}

(a) First, for every $2.4$ kg of clothes, $0.15$ kg of detergent is required. 

Let the amount of detergent needed for $9.6$ kg of clothes be $x$ kg.

$ 
\frac{0.15}{2.4} = \frac{x}{9.6}
$

Cross-multiply:

$ 
0.15 \times 9.6 = 2.4 \times x 
\implies 1.44 = 2.4x
\implies x = \frac{1.44}{2.4} = 0.6
$

So, $0.6$ kg of detergent should be used.

(b) If $0.58$ kg of detergent is left unused from the $0.6$ kg prepared, then the amount actually used is

$
0.6 - 0.58 = 0.02 \text{ kg}
$

So, $0.02$ kg of detergent was actually used.

\section*{Question 158}
\textbf{Metadata}

\begin{itemize}
  \item Question ID: O1-RoRepDc\_O1-RoSmpDc\_GPT4.1\_Services\_02
  \item Primary KC: RATIO | Representation and concept | ratios involving decimals
  \item Secondary KC: RATIO | Simplifying | converting a ratio involving decimals to its simplest form
  \item Topic: Services such as installation, maintenance, repairing, cleaning, laundry, e-commerce
  \item Grade: Secondary O-level 1
\end{itemize}

\textbf{Question}

Trina manages an e-commerce business that sells two types of cleaning solutions. In a particular week, she sold 1.5 litres of Solution A for installation services and 2.4 litres of Solution B for maintenance services. 

(a) Express the ratio of the amount of Solution A sold to the amount of Solution B sold in that week in the form $a:b$, where $a$ and $b$ are decimals. 

(b) Convert this ratio to its simplest form using whole numbers.

\textbf{Solution}

(a) The amount of Solution A sold to Solution B sold is $1.5:2.4$.

(b) To simplify $1.5:2.4$:

First, we can write both numbers in tenths: $1.5:2.4 = \frac{15}{10} : \frac{24}{10}$

Multiply both sides by 10 to remove decimals: $15:24$

Now, simplify $15:24$ by dividing both numbers by their greatest common divisor, which is 3:

$\frac{15}{3} : \frac{24}{3} = 5:8$

So, the simplest form of the ratio is $5:8$.

\section*{Question 159}
\textbf{Metadata}

\begin{itemize}
  \item Question ID: O1-PcRep2q\_O1-PcCnv2Fr\_GPT4.1\_Services\_02
  \item Primary KC: PERCENTAGE | Representation and concept | comparing two quantities by percentage
  \item Secondary KC: PERCENTAGE | Conversion to fraction | expressing percentage as a fraction
  \item Topic: Services such as installation, maintenance, repairing, cleaning, laundry, e-commerce
  \item Grade: Secondary O-level 1
\end{itemize}

\textbf{Question}

A laundry shop washed 120 shirts on Monday and 150 shirts on Tuesday. 

(a) By what percentage did the number of shirts washed on Tuesday increase compared to Monday?

(b) Express this percentage increase as a fraction in its simplest form.

\textbf{Solution}

Let the number of shirts washed on Monday be $120$ and on Tuesday be $150$.

(a) The increase in the number of shirts is $150 - 120 = 30$. 

The percentage increase compared to Monday is: 
$$
\text{Percentage increase} = \frac{30}{120} \times 100\% = 25\%
$$

(b) To express $25\%$ as a fraction:
$$
25\% = \frac{25}{100} = \frac{1}{4}
$$

So, the percentage increase from Monday to Tuesday is $25\%$ or $\frac{1}{4}$.

\section*{Question 160}
\textbf{Metadata}

\begin{itemize}
  \item Question ID: O1-PcRep2q\_O1-PcCnv2Dc\_GPT4.1\_Services\_02
  \item Primary KC: PERCENTAGE | Representation and concept | comparing two quantities by percentage
  \item Secondary KC: PERCENTAGE | Conversion to decimals | expressing percentage as a decimal
  \item Topic: Services such as installation, maintenance, repairing, cleaning, laundry, e-commerce
  \item Grade: Secondary O-level 1
\end{itemize}

\textbf{Question}

A cleaning company charges a monthly fee of \$400 for regular cleaning services to a client. To encourage long-term engagement, the company offers a 15\% discount if the client agrees to a one-year contract, and the discounted rate is paid monthly. 

(a) By what percentage is the monthly fee reduced if the client chooses the one-year contract?

(b) Express this percentage reduction as a decimal, and use it to calculate the new monthly fee after the discount.

(c) How much will the client pay altogether in one year if they choose the one-year contract?

\textbf{Solution}

Let the original monthly fee be \$400. 

(a) The percentage reduction is given: 15\%.

(b) To express 15\% as a decimal: 

$15\% = \frac{15}{100} = 0.15$

(c) New monthly fee after discount:

$\text{Reduction} = 0.15 \times 400 = \$60$

$\text{Discounted monthly fee} = 400 - 60 = \$340$

Total for one year:

$\$340 \times 12 = \$4,080$

Therefore, the client will pay \$4,080 altogether if they sign the one-year contract.

\section*{Question 161}
\textbf{Metadata}

\begin{itemize}
  \item Question ID: O1-PcFndRslt\_P1-WNAdd2nd\_GPT4.1\_Services\_01
  \item Primary KC: PERCENTAGE | Finding result after change | increasing/decreasing a quantity by a given percentage
  \item Secondary KC: WHOLE NUMBERS | Addition | adding whole numbers
  \item Topic: Services such as installation, maintenance, repairing, cleaning, laundry, e-commerce
  \item Grade: Secondary O-level 1
\end{itemize}

\textbf{Question}

A cleaning company charges $150 to clean a house. During a special promotion, they increased their basic service charge by 10%. On one Saturday, they also offered an additional carpet cleaning service for $40. Find the total amount a customer has to pay if they choose both the promoted basic service and the additional carpet cleaning service on that Saturday.

\textbf{Solution}

First, calculate the increased basic service charge:

Original basic service charge = $150
Increase = 10% of $150 = $150 \times 0.10 = $15
Increased service charge = $150 + $15 = $165

Next, add the cost of the additional carpet cleaning service:

Total amount = Increased service charge + Carpet cleaning service
Total amount = $165 + $40 = $205

Therefore, the customer has to pay \boxed{\$205}.

\section*{Question 162}
\textbf{Metadata}

\begin{itemize}
  \item Question ID: O1-PcFndRslt\_P1-WNSub2nd\_GPT4.1\_Services\_01
  \item Primary KC: PERCENTAGE | Finding result after change | increasing/decreasing a quantity by a given percentage
  \item Secondary KC: WHOLE NUMBERS | Subtraction | subtracting whole numbers
  \item Topic: Services such as installation, maintenance, repairing, cleaning, laundry, e-commerce
  \item Grade: Secondary O-level 1
\end{itemize}

\textbf{Question}

A laundry service normally charges $\$40$ to clean a set of curtains. This month, they are offering a $15\%$ discount for cleaning curtains. \newline
(a) What is the discounted price for cleaning a set of curtains? \newline
(b) If Mrs Tan has $\$100$ and pays for the laundry service at the discounted price, how much money does she have left after paying for the service?

\textbf{Solution}

(a) To find the discounted price, we find $15\%$ of $\$40$ and subtract it from $\$40$. \newline
$15\%$ of $\$40 = 0.15 \times 40 = \$6$. \newline
Discounted price $= \$40 - \$6 = \$34$. \newline
(b) Mrs Tan starts with $\$100$ and pays $\$34$. \newline
Money left $= 100 - 34 = \$66$. \newline
\textbf{Answer:} The discounted price is $\$34$. Mrs Tan has $\$66$ left after paying for the service.

\section*{Question 163}
\textbf{Metadata}

\begin{itemize}
  \item Question ID: O1-PcFndRslt\_P1-WNMul2nd\_GPT4.1\_Services\_01
  \item Primary KC: PERCENTAGE | Finding result after change | increasing/decreasing a quantity by a given percentage
  \item Secondary KC: WHOLE NUMBERS | Multiplication | multiplying whole numbers
  \item Topic: Services such as installation, maintenance, repairing, cleaning, laundry, e-commerce
  \item Grade: Secondary O-level 1
\end{itemize}

\textbf{Question}

A washing machine repair service charges $80 for a standard maintenance session. During a promotion, the company announces that the price for each session will be decreased by $15\%$ for customers who book 3 or more sessions at once. 

(a) What is the price of each session after the $15\%$ discount? 

(b) If a customer books 4 maintenance sessions during the promotion, what is the total amount the customer has to pay?

\textbf{Solution}

(a) To find the new price per session after a $15\%$ decrease:\\
Decrease amount per session $= 15\%\ \times\ 80 = \frac{15}{100} \times 80 = 12$\\

New price per session $= 80 - 12 = 68$\\

(b) For 4 sessions:\\
Total amount to pay $= 4 \times 68 = 272$\\

\textbf{Final answers:} 
(a) $68 per session 
(b) $272 for 4 sessions.

\section*{Question 164}
\textbf{Metadata}

\begin{itemize}
  \item Question ID: O1-PcFndRslt\_P1-WNDiv2nd\_GPT4.1\_Services\_01
  \item Primary KC: PERCENTAGE | Finding result after change | increasing/decreasing a quantity by a given percentage
  \item Secondary KC: WHOLE NUMBERS | Division | dividing whole numbers
  \item Topic: Services such as installation, maintenance, repairing, cleaning, laundry, e-commerce
  \item Grade: Secondary O-level 1
\end{itemize}

\textbf{Question}

A company charges a fee for cleaning services. Last month, the company charged a total of $800 for cleaning 20 apartments. This month, the company decided to increase its cleaning fee by 15% due to higher costs. 

(a) What is the new cleaning fee per apartment after the increase?

(b) If 5 apartments were cleaned at the new rate, how much did the company collect in total from these 5 apartments?

\textbf{Solution}

First, we find the cleaning fee per apartment last month:

$
\text{Original price per apartment} = \frac{\$800}{20} = \$40
$

Next, the cleaning fee is increased by 15%. The increase is:

$
\text{Increase} = 15\% \times \$40 = 0.15 \times 40 = \$6
$

Therefore, the new fee per apartment is:

$
\text{New fee per apartment} = \$40 + \$6 = \$46
$

(b) Total collected from 5 apartments at the new rate:

$
\text{Total} = 5 \times \$46 = \$230
$

\textbf{Summary:}
(a) The new cleaning fee per apartment is $46.
(b) The company collected $230 from cleaning 5 apartments at the new rate.

\section*{Question 165}
\textbf{Metadata}

\begin{itemize}
  \item Question ID: O1-PcRepRvs\_O1-PcCnv2Fr\_GPT4.1\_Services\_02
  \item Primary KC: PERCENTAGE | Representation and concept | reverse percentages
  \item Secondary KC: PERCENTAGE | Conversion to fraction | expressing percentage as a fraction
  \item Topic: Services such as installation, maintenance, repairing, cleaning, laundry, e-commerce
  \item Grade: Secondary O-level 1
\end{itemize}

\textbf{Question}

A cleaning company charged a total bill of \$180 for cleaning services after applying a 20\% discount for first-time customers. What was the original price before the discount? Express the 20\% discount as a fraction in your working.

\textbf{Solution}

Let the original price be \( x \). The discount given is 20\%, which can be expressed as a fraction: \( 20\% = \frac{20}{100} = \frac{1}{5} \).

After applying a 20\% discount, the customer pays 80\% of the original price:

$
\text{Amount paid} = 80\% \text{ of the original price} = \frac{80}{100} \times x = \frac{4}{5}x
$

We are told the total bill after the discount is \$180:

$
\frac{4}{5}x = 180
$

Solve for \( x \):
$
x = 180 \times \frac{5}{4} = 225
$

Therefore, the original price before the discount was \$225.

\section*{Question 166}
\textbf{Metadata}

\begin{itemize}
  \item Question ID: O1-PcRepRvs\_O1-PcCnv2Dc\_GPT4.1\_Services\_02
  \item Primary KC: PERCENTAGE | Representation and concept | reverse percentages
  \item Secondary KC: PERCENTAGE | Conversion to decimals | expressing percentage as a decimal
  \item Topic: Services such as installation, maintenance, repairing, cleaning, laundry, e-commerce
  \item Grade: Secondary O-level 1
\end{itemize}

\textbf{Question}

A company charges customers for air conditioner maintenance. Last month, they ran a promotion, offering a 15\% discount on the usual service fee. During the promotion, Ms. Tan paid \$204 for her service after the discount was applied. 

(a) Express 15\% as a decimal.

(b) What was the usual service fee before the discount?

\textbf{Solution}

(a) 15\% as a decimal is \(\frac{15}{100} = 0.15\).

(b) Let the usual service fee be \(x\) dollars. After a 15\% discount, Ms. Tan paid \$204.

The discounted amount is \(100\% - 15\% = 85\%\) of the usual fee.

Expressing 85\% as a decimal: \(0.85\)

So, \(0.85x = 204\)

\(x = \frac{204}{0.85} = 240\)

The usual service fee before the discount was \$240.

\section*{Question 167}
\textbf{Metadata}

\begin{itemize}
  \item Question ID: O1-AgRepExSq\_O1-AgEvlEx\_GPT4.1\_Services\_02
  \item Primary KC: ALGEBRA | Representation and concept | translation of simple real-world situations into quadratic algebraic expressions
  \item Secondary KC: ALGEBRA | Evaluation | evaluation of algebraic expressions and formulae
  \item Topic: Services such as installation, maintenance, repairing, cleaning, laundry, e-commerce
  \item Grade: Secondary O-level 1
\end{itemize}

\textbf{Question}

A cleaning company charges a fixed call-out fee of \$20 and an additional cost based on the area of the carpet to be cleaned. The cost per square metre is \$x. A client has a rectangular carpet measuring $(a+2)$ metres long and $(a)$ metres wide. 

(a) Express, in terms of \(a\) and \(x\), the total amount (in dollars) the client will have to pay, and simplify your answer as much as possible.

(b) If \(a = 4\) and \(x = 3\), evaluate the total amount the client will pay.

\textbf{Solution}

(a) Area of the carpet = $(a+2) \times a = a^2 + 2a$ square metres.

Cost for cleaning = cost per square metre $\times$ area = $x(a^2 + 2a)$

Total amount = fixed fee + cleaning cost 
= $20 + x(a^2 + 2a)$

(b) When $a = 4$, $x = 3$:

Area = $(4 + 2) \times 4 = 6 \times 4 = 24$ sq m

Cleaning cost = $3 \times 24 = 72$

Total amount = $20 + 72 = 92$

So, the client will pay $\$92$.

\section*{Question 168}
\textbf{Metadata}

\begin{itemize}
  \item Question ID: O1-AgRepnth\_O1-AgEvlEx\_GPT4.1\_Services\_02
  \item Primary KC: ALGEBRA | Representation and concept | recognising and representing patterns/relationships by finding an algebraic expression for the nth term
  \item Secondary KC: ALGEBRA | Evaluation | evaluation of algebraic expressions and formulae
  \item Topic: Services such as installation, maintenance, repairing, cleaning, laundry, e-commerce
  \item Grade: Secondary O-level 1
\end{itemize}

\textbf{Question}

A cleaning company charges its clients according to the following pattern: for the first cleaning session, a client pays $40. For each subsequent session, the cost increases by $5. 

(a) Let $n$ be the number of cleaning sessions attended by a client. Write down an algebraic expression, in terms of $n$, for the cost of the $n$th cleaning session.

(b) Using your expression from part (a), find the cost of the 10th cleaning session.

\textbf{Solution}

(a) The cost for the first session is $40$ and each subsequent session increases by $5$. This forms an arithmetic sequence where the first term $a_1 = 40$ and the common difference $d = 5$.

The $n$th term of an arithmetic sequence is given by:
$ a_n = a_1 + (n-1)d $

Substituting the values, the algebraic expression for the cost of the $n$th session is:
$ \text{Cost} = 40 + (n-1) \times 5 $
$ \text{Cost} = 40 + 5n - 5 $
$ \text{Cost} = 5n + 35 $

(b) For the 10th session, substitute $n = 10$ into the expression:
$ \text{Cost} = 5(10) + 35 = 50 + 35 = 85 $

Therefore, the cost of the 10th cleaning session is $\$85$.

\section*{Question 169}
\textbf{Metadata}

\begin{itemize}
  \item Question ID: O1-AgSlvFrLr\_O1-AgRepEq\_GPT4.1\_Services\_02
  \item Primary KC: ALGEBRA | Solving | solving simple fractional equations that can be reduced to linear equations
  \item Secondary KC: ALGEBRA | Representation and concept | translation of simple real-world situations to equations
  \item Topic: Services such as installation, maintenance, repairing, cleaning, laundry, e-commerce
  \item Grade: Secondary O-level 1
\end{itemize}

\textbf{Question}

A washing machine repair service charges a fixed fee of $ \$20 $ for transportation and an additional $ \frac{3}{4} $ of the time spent (in hours) multiplied by $ \$40 $ per hour for labour. If the total amount charged for a repair job is $ \$65 $, and the time spent is $ x $ hours, write down an equation for the total cost, and find the value of $ x $.

\textbf{Solution}

Let the total cost be $ T $. According to the problem,$\begin{align*}
T &= 20 + \frac{3}{4} \times 40x \\
T &= 20 + 30x
\end{align*}$

We are told the total charged is $ \$65 $.
$65 = 20 + 30x$
Subtract 20 from both sides:
$65 - 20 = 30x$
$45 = 30x$
Divide both sides by 30:
$x = \frac{45}{30} = 1.5$

So, the time spent on the repair was $ 1.5 $ hours.

\section*{Question 170}
\textbf{Metadata}

\begin{itemize}
  \item Question ID: O2-RoRepDP\_P1-WNMul2nd\_GPT4.1\_Services\_01
  \item Primary KC: RATIO | Representation and concept | direct proportion
  \item Secondary KC: WHOLE NUMBERS | Multiplication | multiplying whole numbers
  \item Topic: Services such as installation, maintenance, repairing, cleaning, laundry, e-commerce
  \item Grade: Secondary O-level 2
\end{itemize}

\textbf{Question}

Mr. Tan owns a cleaning service company. For a small apartment, the ratio of cleaning solution to water used is $2 : 5$. If his team needs to prepare the solution for 8 small apartments and each apartment requires $6$ litres of the cleaning solution mixture, 

\begin{enumerate}
  \item How many litres of cleaning solution and water will be needed for all 8 apartments?
  \item If the company has a container that can hold up to $20$ litres, how many full containers will be required to store the entire mixture prepared for the 8 apartments?
\end{enumerate}


\textbf{Solution}

Step 1: Total mixture needed for 8 apartments:
Each apartment requires $6$ litres, so $8 \times 6 = 48$ litres in total.

Step 2: Find litres of cleaning solution and water.
Ratio of cleaning solution to water is $2:5$.
Total parts $= 2 + 5 = 7$ parts.

Each part $= \frac{48}{7} \approx 6.857$ litres.

Cleaning solution $= 2 \times 6.857 \approx 13.714$ litres.
Water $= 5 \times 6.857 \approx 34.286$ litres.

Step 3: Calculate number of full containers needed.
Each container can hold $20$ litres.
Total mixture $= 48$ litres.
Number of full containers $= \lceil \frac{48}{20} \rceil = \lceil 2.4 \rceil = 3$ containers.

\textbf{Final Answers:}
\begin{enumerate}
  \item The team will need approximately $13.71$ litres of cleaning solution and $34.29$ litres of water.
  \item $3$ full containers will be needed to store the entire mixture.
\end{enumerate}


\section*{Question 171}
\textbf{Metadata}

\begin{itemize}
  \item Question ID: O2-RoRepDP\_P1-WNDiv2nd\_GPT4.1\_Services\_01
  \item Primary KC: RATIO | Representation and concept | direct proportion
  \item Secondary KC: WHOLE NUMBERS | Division | dividing whole numbers
  \item Topic: Services such as installation, maintenance, repairing, cleaning, laundry, e-commerce
  \item Grade: Secondary O-level 2
\end{itemize}

\textbf{Question}

A laundry shop offers a promotional deal: for every 3 shirts cleaned, they charge $12.\
\\
(a) What is the cost to clean 15 shirts? \\n(b) If Angela has $48, how many shirts can she have cleaned, under this promotion?

\textbf{Solution}

(a) The shop charges $12 for every 3 shirts.\\
To find out how many sets of 3 shirts are there in 15 shirts: \\n$15 \div 3 = 5$ \\nSo, there are 5 sets of 3 shirts.\\
The total cost is: $12 \times 5 = $60\\
Answer: The cost to clean 15 shirts is $60.\\
\\
(b) To find how many sets Angela can clean with $48: \\n$48 \div 12 = 4$ \\nShe can pay for 4 sets, and each set is 3 shirts: \\n$4 \times 3 = 12$ \\nAnswer: Angela can have 12 shirts cleaned.

\section*{Question 172}
\textbf{Metadata}

\begin{itemize}
  \item Question ID: O2-RoRepIvP\_P1-WNMul2nd\_GPT4.1\_Services\_01
  \item Primary KC: RATIO | Representation and concept | inverse proportion
  \item Secondary KC: WHOLE NUMBERS | Multiplication | multiplying whole numbers
  \item Topic: Services such as installation, maintenance, repairing, cleaning, laundry, e-commerce
  \item Grade: Secondary O-level 2
\end{itemize}

\textbf{Question}

A group of workers is hired to clean all the classrooms in a school. If 4 workers can clean all the classrooms in 6 hours, how long would it take for 8 workers to clean all the classrooms, assuming they all work at the same rate and work together? Additionally, how many classroom-hours (number of classrooms cleaned per hour) does the group of 8 workers achieve if they complete the task at the calculated time?

\textbf{Solution}

Let the time taken by 8 workers be $x$ hours. Since the number of workers is inversely proportional to the time taken (inverse proportion), we have:

$4 \times 6 = 8 \times x$

$24 = 8x$

$x = \frac{24}{8} = 3$ hours

So, 8 workers would take 3 hours to clean all the classrooms.

Next, let's find the number of classroom-hours for the group of 8 workers. The total number of classroom-hours completed by 8 workers in 3 hours is:

$8 \text{ workers} \times 3 \text{ hours} = 24$ classroom-hours.

Therefore, the group of 8 workers achieves 24 classroom-hours when they clean all the classrooms in 3 hours.

\section*{Question 173}
\textbf{Metadata}

\begin{itemize}
  \item Question ID: O2-RoRepIvP\_P1-WNDiv2nd\_GPT4.1\_Services\_01
  \item Primary KC: RATIO | Representation and concept | inverse proportion
  \item Secondary KC: WHOLE NUMBERS | Division | dividing whole numbers
  \item Topic: Services such as installation, maintenance, repairing, cleaning, laundry, e-commerce
  \item Grade: Secondary O-level 2
\end{itemize}

\textbf{Question}

A cleaning company is hired to clean 320 identical classrooms in a school. If 8 cleaners can finish cleaning all the classrooms in 20 hours, how many hours will it take 10 cleaners to finish cleaning the same number of classrooms, assuming they all work at the same rate and the work is shared equally? 

Additionally, how many classrooms will each cleaner clean if the 10 cleaners work together in that time?

\textbf{Solution}

Since the number of cleaners and the time taken are inversely proportional (more cleaners means less time needed), we set up the equation:

$
8 \text{ cleaners} \times 20 \text{ hours} = 10 \text{ cleaners} \times x \text{ hours}
$

So,
$
8 \times 20 = 10x \\
160 = 10x \\
x = \frac{160}{10} = 16
$

Therefore, it will take 10 cleaners 16 hours to clean all the classrooms.

Now, to find the number of classrooms each cleaner will clean:

$
\text{Total classrooms} = 320 \\
\text{Total cleaners} = 10 \\
\text{Each cleaner cleans: } \frac{320}{10} = 32 \text{ classrooms}
$

**Final answers:**
- It will take 16 hours for 10 cleaners to clean all 320 classrooms.
- Each cleaner will clean 32 classrooms.

\section*{Question 174}
\textbf{Metadata}

\begin{itemize}
  \item Question ID: O2-AgSlvIneq\_O2-AgRepIneq\_GPT4.1\_Services\_01
  \item Primary KC: ALGEBRA | Solving | solving simple linear inequalities with one variable
  \item Secondary KC: ALGEBRA | Representation and concept | translation of simple real-world situations to simple linear inequalities with one variable
  \item Topic: Services such as installation, maintenance, repairing, cleaning, laundry, e-commerce
  \item Grade: Secondary O-level 2
\end{itemize}

\textbf{Question}

A cleaning company charges a fixed fee of \$30 plus \$6 for each hour of cleaning. If a customer has a budget of at most \$72 for cleaning services, write an inequality to represent the greatest number of hours, $h$, the customer can afford. Then, solve the inequality to find the maximum number of hours of cleaning the customer can purchase.


\textbf{Solution}

Let $h$ represent the number of hours of cleaning. The total cost for $h$ hours is given by $30 + 6h$. Since the customer can spend at most $72$, we have:
$
30 + 6h \leq 72
$
Subtract 30 from both sides:
$
6h \leq 42
$
Divide both sides by 6:
$
h \leq 7
$
The customer can afford at most 7 hours of cleaning service.


\section*{Question 175}
\textbf{Metadata}

\begin{itemize}
  \item Question ID: O2-AgSlvSq1v\_O1-AgRepEq\_GPT4.1\_Services\_01
  \item Primary KC: ALGEBRA | Solving | solving quadratic equations in one variable
  \item Secondary KC: ALGEBRA | Representation and concept | translation of simple real-world situations to equations
  \item Topic: Services such as installation, maintenance, repairing, cleaning, laundry, e-commerce
  \item Grade: Secondary O-level 2
\end{itemize}

\textbf{Question}

A washing machine repair company charges customers a basic visiting fee of \$30, plus an additional amount that depends on the number of hours spent repairing the machine. The total repair time needed for a certain washing machine can be represented by $h$ hours. The company collected a total of \$150 for one particular repair job. Later, the manager realised that the extra repair work required meant the total time spent, $h$, also led to a quadratic relationship: the additional amount charged (in dollars) is $10h^2$. 

Form an equation based on this situation, and solve for $h$, the number of hours spent on that repair job.

\textbf{Solution}

The total cost charged to the customer was \$150, which consists of a \$30 basic fee and an additional fee of $10h^2$.

So, we can write the equation:

\begin{align*}
150 &= 30 + 10h^2 \\
150 - 30 &= 10h^2 \\
120 &= 10h^2 \\
\frac{120}{10} &= h^2 \\
12 &= h^2 \\
h &= \sqrt{12} \\
h &= 2\sqrt{3}
\end{align*}

So, the number of hours spent on the repair job was $2\sqrt{3}$ hours (about 3.46 hours, to 2 decimal places).

\section*{Question 176}
\textbf{Metadata}

\begin{itemize}
  \item Question ID: O2-AgSlvLr2v\_O1-AgRepEq\_GPT4.1\_Services\_01
  \item Primary KC: ALGEBRA | Solving | solving linear equations in two variables
  \item Secondary KC: ALGEBRA | Representation and concept | translation of simple real-world situations to equations
  \item Topic: Services such as installation, maintenance, repairing, cleaning, laundry, e-commerce
  \item Grade: Secondary O-level 2
\end{itemize}

\textbf{Question}

A cleaning company offers two types of cleaning packages for homes. Package A charges a basic fee of \$40 plus \$12 per room cleaned. Package B charges a basic fee of \$30 plus \$15 per room cleaned. Nicole needs her 3-room flat cleaned but is unsure which package is cheaper, so she wants to find out for how many rooms (let this number be $x$) both packages will cost the same. Write an equation to represent the total cost of each package and solve for $x$ to find the number of rooms for which the costs are equal. Then, determine which package Nicole should choose for her 3-room flat.

\textbf{Solution}

Let $x$ be the number of rooms to be cleaned.\
\
Cost of Package A: \$40 + 12x\
Cost of Package B: \$30 + 15x\
\
To find when the costs are the same, set the equations equal:\
$
40 + 12x = 30 + 15x
$
Subtract 12x from both sides:\
$
40 = 30 + 3x
$
Subtract 30 from both sides:\
$
10 = 3x
$
Divide both sides by 3:\
$
x = \frac{10}{3}\approx 3.33
$
So, for about 3.33 rooms, the two packages cost the same.\
\
For Nicole's 3-room flat:\
\
Package A: $40 + 12(3) = 40 + 36 = \$76$\
Package B: $30 + 15(3) = 30 + 45 = \$75$\
\
Package B is cheaper for Nicole's 3-room flat.

\section*{Question 177}
\textbf{Metadata}

\begin{itemize}
  \item Question ID: O2-SPFndmdn\_O2-SPFndmode\_GPT4.1\_Services\_01
  \item Primary KC: STATISTICS AND PROBABILITY | Finding median | Finding median for a set of data
  \item Secondary KC: STATISTICS AND PROBABILITY | Finding mode | Finding mode for a set of data
  \item Topic: Services such as installation, maintenance, repairing, cleaning, laundry, e-commerce
  \item Grade: Secondary O-level 2
\end{itemize}

\textbf{Question}

A laundry shop recorded the number of shirts washed in each of the last 7 days as follows: 18, 22, 18, 25, 20, 18, and 24.  

(a) Find the mode of the number of shirts washed.  
(b) Find the median number of shirts washed in a day.

\textbf{Solution}

(a) To find the mode, list the data: 18, 22, 18, 25, 20, 18, 24.  
The number 18 appears 3 times, and all other numbers appear once.  
**Mode = 18**

(b) To find the median, first arrange the data in order: 18, 18, 18, 20, 22, 24, 25.  
There are 7 data values (an odd number), so the median is the 4th value.  
**Median = 20**

\section*{Question 178}
\textbf{Metadata}

\begin{itemize}
  \item Question ID: O2-SPFndmdn\_O3-SPFndPctl\_GPT4.1\_Services\_01
  \item Primary KC: STATISTICS AND PROBABILITY | Finding median | Finding median for a set of data
  \item Secondary KC: STATISTICS AND PROBABILITY | Finding percentiles | finding percentiles for a set of data
  \item Topic: Services such as installation, maintenance, repairing, cleaning, laundry, e-commerce
  \item Grade: Secondary O-level 2
\end{itemize}

\textbf{Question}

A laundry service collected the number of clothes washed for 9 randomly selected customers in one day. The numbers of clothes washed were: 14, 18, 12, 20, 16, 22, 15, 19, and 17.

(a) Find the median number of clothes washed for these customers.

(b) Calculate the 75th percentile (upper quartile) of the number of clothes washed for this data set.

\textbf{Solution}

First, arrange the data in ascending order:

12, 14, 15, 16, 17, 18, 19, 20, 22

(a) **Median:**
Since there are 9 data points (odd number), the median is the middle value, which is the 5th number:

Median = 17

(b) **75th percentile (upper quartile, Q3):**
Using the percentile formula: \( P = \frac{n+1}{100}\times k \), where k=75 for the 75th percentile and n=9.

$
P = \frac{9+1}{100} \times 75 = \frac{10}{100} \times 75 = 7.5
$
So, the 7.5th value (between the 7th and 8th values):

7th value = 19
8th value = 20

Interpolate:

75th percentile = 19 + 0.5 \times (20 - 19) = 19.5

**Final answers:**
- Median = 17
- 75th percentile = 19.5

\section*{Question 179}
\textbf{Metadata}

\begin{itemize}
  \item Question ID: O2-SPFndmdn\_O3-SPFndrng\_GPT4.1\_Services\_01
  \item Primary KC: STATISTICS AND PROBABILITY | Finding median | Finding median for a set of data
  \item Secondary KC: STATISTICS AND PROBABILITY | Finding range | finding range as measures of spread for a set of data 
  \item Topic: Services such as installation, maintenance, repairing, cleaning, laundry, e-commerce
  \item Grade: Secondary O-level 2
\end{itemize}

\textbf{Question}

A company offers laundry services for clothes in Singapore. Over the past 7 days, the number of laundry orders they received per day was as follows: 18, 25, 22, 27, 20, 24, and 26. 

(a) What is the median number of laundry orders received in a day over these 7 days?  
(b) What is the range of the number of laundry orders received per day?

\textbf{Solution}

First, arrange the data in ascending order: 18, 20, 22, 24, 25, 26, 27.  
(a) The median is the middle value in an ordered set. Since there are 7 data points, the median is the 4th value:

Median = 24.

(b) The range is the difference between the largest and smallest values:  
Range = 27 - 18 = 9.

**Final answers:**
(a) Median = 24 
(b) Range = 9

\section*{Question 180}
\textbf{Metadata}

\begin{itemize}
  \item Question ID: O2-SPFndmean\_O3-BPRepSN\_GPT4.1\_Services\_01
  \item Primary KC: STATISTICS AND PROBABILITY | Finding mean deviation | calculation of the mean for a set of data
  \item Secondary KC: BASE AND POWER | Representation and concept  | use of standard form Ax10^n , where n is an integer, and 1<= A<= 10
  \item Topic: Services such as installation, maintenance, repairing, cleaning, laundry, e-commerce
  \item Grade: Secondary O-level 2
\end{itemize}

\textbf{Question}

A company provides home cleaning services. Over 5 days, the number of homes cleaned each day was recorded. The results are as follows:

Monday: $3.2 \times 10^2$ homes  
Tuesday: $2.8 \times 10^2$ homes  
Wednesday: $3.5 \times 10^2$ homes  
Thursday: $3.0 \times 10^2$ homes  
Friday: $2.5 \times 10^2$ homes

(a) Express the total number of homes cleaned over the 5 days in standard form.

(b) Find the mean number of homes cleaned daily, leaving your answer in standard form.

(c) Calculate the mean deviation of the number of homes cleaned from the mean number found in (b). Express all answers in standard form.

\textbf{Solution}

(a) Total number of homes cleaned:

$= (3.2 + 2.8 + 3.5 + 3.0 + 2.5) \times 10^2$

$= 15.0 \times 10^2$

$= 1.5 \times 10^3$ (in standard form)

(b) Mean number of homes cleaned per day:

Mean $= \dfrac{1.5 \times 10^3}{5} = 0.3 \times 10^3$

$= 3.0 \times 10^2$ (in standard form)

(c) Mean deviation:

Find the absolute difference from the mean for each day:

Monday: $|3.2 \times 10^2 - 3.0 \times 10^2| = 0.2 \times 10^2 = 2.0 \times 10^1$

Tuesday: $|2.8 \times 10^2 - 3.0 \times 10^2| = 0.2 \times 10^2 = 2.0 \times 10^1$

Wednesday: $|3.5 \times 10^2 - 3.0 \times 10^2| = 0.5 \times 10^2 = 5.0 \times 10^1$

Thursday: $|3.0 \times 10^2 - 3.0 \times 10^2| = 0$

Friday: $|2.5 \times 10^2 - 3.0 \times 10^2| = 0.5 \times 10^2 = 5.0 \times 10^1$

Sum of deviations:

$= 2.0 \times 10^1 + 2.0 \times 10^1 + 5.0 \times 10^1 + 0 + 5.0 \times 10^1$

$= (2.0 + 2.0 + 5.0 + 0 + 5.0) \times 10^1$

$= 14.0 \times 10^1$

Mean deviation $= \dfrac{14.0 \times 10^1}{5} = 2.8 \times 10^1$ (in standard form)

Final Answers:

(a) $1.5 \times 10^3$ homes

(b) $3.0 \times 10^2$ homes

(c) $2.8 \times 10^1$ homes

\section*{Question 181}
\textbf{Metadata}

\begin{itemize}
  \item Question ID: O3-BPRepPosI\_O3-BPRepNegI\_GPT4.1\_Services\_01
  \item Primary KC: BASE AND POWER | Representation and concept  | positive indices that is not 1
  \item Secondary KC: BASE AND POWER | Representation and concept  | negative indices
  \item Topic: Services such as installation, maintenance, repairing, cleaning, laundry, e-commerce
  \item Grade: Secondary O-level 3/4
\end{itemize}

\textbf{Question}

A company provides regular air-conditioner cleaning services. For each cleaning session, the company uses a special disinfectant that is effective for $2^3$ days. However, if the cleaning is not done regularly, the effectiveness of the disinfectant decreases, and every time the cleaning interval doubles, the effectiveness period of the disinfectant is halved, following the rule $2^{-n}$, where $n$ is the number of times the interval is doubled.

(a) What is the effectiveness period in days after the cleaning interval has been doubled twice?

(b) If the company originally offers a cleaning package every $2^3$ days, but a customer chooses to clean every $2^5$ days instead, what is the new effectiveness period of the disinfectant?

\textbf{Solution}

Let us use the rules of indices to solve the problems.

(a) The initial effectiveness period: $2^3 = 8$ days.
Each time the interval is doubled, the period is multiplied by $2^{-1}$ (halved). After doubling twice: $2^{-2} = \dfrac{1}{2^2} = \dfrac{1}{4}$.
So, new effectiveness period $= 8 \times \dfrac{1}{4} = 2$ days.

(b) To go from every $2^3$ days to every $2^5$ days, the interval is doubled $2$ times ($2^5 / 2^3 = 2^{5-3} = 2^2 = 4$ times, or doubled twice).
So, as above, effectiveness period $= 8 \times \dfrac{1}{4} = 2$ days.

**Answers:**
(a) 2 days
(b) 2 days

\section*{Question 182}
\textbf{Metadata}

\begin{itemize}
  \item Question ID: O3-BPRepPosI\_O3-BPRepFrI\_GPT4.1\_Services\_01
  \item Primary KC: BASE AND POWER | Representation and concept  | positive indices that is not 1
  \item Secondary KC: BASE AND POWER | Representation and concept  | fractional indices
  \item Topic: Services such as installation, maintenance, repairing, cleaning, laundry, e-commerce
  \item Grade: Secondary O-level 3/4
\end{itemize}

\textbf{Question}

In a laundry service, every washing machine is designed to handle an increasing workload as it ages. Suppose the efficiency (in percentage) of a washing machine after $n$ years is given by $E(n) = 100 \times (0.9)^n$. After some improvements, the laundry service decides to use a special cleaning solution that, according to the manufacturer, will further change the efficiency such that it is now $E'(n) = E(n)^{\frac{3}{2}}$.

(a) Find the efficiency of a washing machine after 2 years, before using the special cleaning solution, expressing your answer as a power with a positive index that is not 1.

(b) Write the new efficiency of the washing machine after 2 years, $E'(2)$, as a single power (with a fractional index) of $0.9$.

(c) Calculate the value of $E'(2)$ correct to 2 decimal places.

\textbf{Solution}

Let us solve each part step by step.

(a) The efficiency after 2 years is:

\begin{align*}
E(2) &= 100 \times (0.9)^2 \\
     &= 100 \times 0.9^2
\end{align*}
This expresses the answer as a power with a positive index that is not 1.

(b) The new efficiency after 2 years is:
\begin{align*}
E'(2) &= [E(2)]^{\frac{3}{2}} \\
      &= [100 \times 0.9^2]^{\frac{3}{2}} \\
      &= 100^{\frac{3}{2}} \times (0.9^2)^{\frac{3}{2}} \\ 
      &= (100^{\frac{3}{2}}) \times (0.9)^{2 \times \frac{3}{2}} \\
      &= (100^{\frac{3}{2}}) \times (0.9)^3
\end{align*}
$E'(2)$ as a single power of $0.9$ is $(100^{\frac{3}{2}}) \times 0.9^3$.

(c) Now calculating the value:

First, $100^{3/2} = (100^{1/2})^3 = (10)^3 = 1000$

$0.9^3 = 0.9 \times 0.9 \times 0.9 = 0.729$

So,
$
E'(2) = 1000 \times 0.729 = 729
$

Therefore, the new efficiency of the washing machine after 2 years, correct to 2 decimal places, is $729.00$.

\section*{Question 183}
\textbf{Metadata}

\begin{itemize}
  \item Question ID: O3-BPRepNegI\_O3-BPRepPosI\_GPT4.1\_Services\_01
  \item Primary KC: BASE AND POWER | Representation and concept  | negative indices
  \item Secondary KC: BASE AND POWER | Representation and concept  | positive indices that is not 1
  \item Topic: Services such as installation, maintenance, repairing, cleaning, laundry, e-commerce
  \item Grade: Secondary O-level 3/4
\end{itemize}

\textbf{Question}

A cleaning company charges a base fee that decreases by half every year after a promotional period using the equation \(C = 200 \times (\frac{1}{2})^n\), where \(C\) is the cleaning fee (in dollars) for an annual contract after \(n\) years past the 2-year promotional period. 

After 3 more years (so, \(n = 3\)), they offer a special bonus: for every cleaning session after this point, the number of services doubles compared to the previous session, represented by \(S = 4 \times 2^k\), where \(S\) is the number of services in the \(k\)th session after year 5. 

(a) What is the cleaning fee after 5 years (that is, calculate the fee for \(n = 3\))? 

(b) How many services will a customer receive at the 3rd session after year 5 (that is, what is \(S\) when \(k = 3\))?

\textbf{Solution}

(a) To find the cleaning fee after 5 years (for \(n = 3\)), use the negative index formula:
$
C = 200 \times (\frac{1}{2})^3 = 200 \times \frac{1}{8} = 25
$
So, the cleaning fee after 5 years is \(\boxed{\$25}\).

(b) To find the number of services at the 3rd session after year 5 (with \(k = 3\)), use the positive index formula:
$
S = 4 \times 2^3 = 4 \times 8 = 32
$
So, the customer will receive \(\boxed{32}\) services at the 3rd session after year 5.

\section*{Question 184}
\textbf{Metadata}

\begin{itemize}
  \item Question ID: O3-BPRepNegI\_O3-BPRepFrI\_GPT4.1\_Services\_01
  \item Primary KC: BASE AND POWER | Representation and concept  | negative indices
  \item Secondary KC: BASE AND POWER | Representation and concept  | fractional indices
  \item Topic: Services such as installation, maintenance, repairing, cleaning, laundry, e-commerce
  \item Grade: Secondary O-level 3/4
\end{itemize}

\textbf{Question}

A smart appliance repair company uses a formula to estimate the time (in hours) required to repair a model of air conditioner based on how many years old ($x$) the machine is. The estimated repair time is given by
$
t = 4^{-x} + 8^{\frac{1}{3}}
$
where $x$ is a negative whole number because the company tracks the age of these machines as 'years since new', counting back from 0.

(a) If an air conditioner was bought 2 years ago ($x = -2$), calculate the estimated repair time.

(b) If a newer model was bought just last year ($x = -1$), what is the difference in estimated repair time between the two machines?

\textbf{Solution}

(a) For $x = -2$:
$
t = 4^{-(-2)} + 8^{\frac{1}{3}} = 4^2 + 8^{\frac{1}{3}}
$
Now,
$
4^2 = 16
$
$
8^{\frac{1}{3}} = \sqrt[3]{8} = 2
$
So,
$
t = 16 + 2 = 18\text{ hours}
$

(b) For $x = -1$:
$
t = 4^{-(-1)} + 8^{\frac{1}{3}} = 4^1 + 8^{\frac{1}{3}}
$
$
4^1 = 4
$
$
8^{\frac{1}{3}} = 2
$
So,
$
t = 4 + 2 = 6\text{ hours}
$

Difference in estimated repair time:
$
18 - 6 = 12\text{ hours}
$

**Final Answer:**
(a) The estimated repair time for the first machine is \(18\) hours.
(b) The difference in estimated repair time is \(12\) hours.

\section*{Question 185}
\textbf{Metadata}

\begin{itemize}
  \item Question ID: O3-BPRepFrI\_O3-BPRepPosI\_GPT4.1\_Services\_01
  \item Primary KC: BASE AND POWER | Representation and concept  | fractional indices
  \item Secondary KC: BASE AND POWER | Representation and concept  | positive indices that is not 1
  \item Topic: Services such as installation, maintenance, repairing, cleaning, laundry, e-commerce
  \item Grade: Secondary O-level 3/4
\end{itemize}

\textbf{Question}

A cleaning service company uses a special formula that determines the effectiveness $E$ of a cleaning agent after $n$ uses. The formula is:

$
E = 64^{\frac{2}{3}} \times 4^n
$

where $E$ is the effectiveness value and $n$ must be a positive integer.

(a) Express $64^{\frac{2}{3}}$ as a positive integer using the properties of indices.

(b) If the cleaning agent is used 3 times $(n=3)$, what is the effectiveness value $E$ in terms of a positive integer?

\textbf{Solution}

Part (a):

$64^{\frac{2}{3}}$
We know that $64 = 4^3$.
Thus,
$
64^{\frac{2}{3}} = (4^3)^{\frac{2}{3}} = 4^{3 \times \frac{2}{3}} = 4^2 = 16
$

So, $64^{\frac{2}{3}} = 16$.

Part (b):

For $n=3$:
$
E = 16 \times 4^3
$
$
4^3 = 4 \times 4 \times 4 = 64
$
$
E = 16 \times 64 = 1024
$

The effectiveness value $E$ after 3 uses is **1024**.

\section*{Question 186}
\textbf{Metadata}

\begin{itemize}
  \item Question ID: O3-BPRepFrI\_O3-BPRepNegI\_GPT4.1\_Services\_01
  \item Primary KC: BASE AND POWER | Representation and concept  | fractional indices
  \item Secondary KC: BASE AND POWER | Representation and concept  | negative indices
  \item Topic: Services such as installation, maintenance, repairing, cleaning, laundry, e-commerce
  \item Grade: Secondary O-level 3/4
\end{itemize}

\textbf{Question}

A washing machine repair service charges $60 per visit for the first year. For each subsequent year, the cost per visit changes according to the formula $C = 60 \times \left(\frac{3}{2}\right)^{-n}$, where $n$ is the number of years after the first year.

(a) Express the cost per visit in the fourth year as a single number using a fractional index.

(b) Simplify this expression by rewriting the fractional index as a negative index.

(c) Calculate the cost per visit in the fourth year, giving your answer correct to 2 decimal places.

\textbf{Solution}

(a) In the fourth year, $n = 3$ (since the first year is $n = 0$).

Cost per visit, $C = 60 \times \left(\frac{3}{2}\right)^{-3}$

(b) We can rewrite $\left(\frac{3}{2}\right)^{-3}$ as a fractional index with a negative power:

$\left(\frac{3}{2}\right)^{-3} = \left(\frac{2}{3}\right)^{3}$

So, $C = 60 \times \left(\frac{2}{3}\right)^3$

(c) Calculate $\left(\frac{2}{3}\right)^3 = \frac{2^3}{3^3} = \frac{8}{27}$

Therefore, $C = 60 \times \frac{8}{27} = \frac{480}{27} = 17.78$ (rounded to 2 decimal places).

So, the cost per visit in the fourth year is $\boxed{17.78}$ dollars.

\section*{Question 187}
\textbf{Metadata}

\begin{itemize}
  \item Question ID: O3-BPOpr\_O3-BPRepPosI\_GPT4.1\_Services\_01
  \item Primary KC: BASE AND POWER | Operations | laws of indices
  \item Secondary KC: BASE AND POWER | Representation and concept  | positive indices that is not 1
  \item Topic: Services such as installation, maintenance, repairing, cleaning, laundry, e-commerce
  \item Grade: Secondary O-level 3/4
\end{itemize}

\textbf{Question}

A laundry service charges according to the number of kilograms of clothes washed. The cost in dollars for washing $n$ kilograms of clothes can be calculated using the formula $C = 2^n$, where $n$ is a positive integer greater than 1. 

(a) If a customer sends 3 kg of clothes, calculate the cost.

(b) If another customer sends 5 kg of clothes and claims she should pay twice as much as the first customer, check if her statement is correct using the laws of indices.

\textbf{Solution}

(a) Using $C = 2^n$, for $n = 3$:

$$
C = 2^3 = 8
$$
So, the cost is \$8.

(b) For $n = 5$:

$$
C = 2^5 = 32
$$
The first customer's cost is $2^3 = 8$.

Let us check if the second customer pays twice as much:
$$
8 \times 2 = 16
$$
But the second customer pays $32,\ $ which is not twice, but actually four times the cost.

Alternatively, using the law of indices:
$$
2^5 = 2^3 \times 2^2 = 8 \times 4 = 32
$$
So, the second customer pays four times the amount, not twice the amount. Her statement is incorrect.

\section*{Question 188}
\textbf{Metadata}

\begin{itemize}
  \item Question ID: O3-BPOpr\_O3-BPRepNegI\_GPT4.1\_Services\_01
  \item Primary KC: BASE AND POWER | Operations | laws of indices
  \item Secondary KC: BASE AND POWER | Representation and concept  | negative indices
  \item Topic: Services such as installation, maintenance, repairing, cleaning, laundry, e-commerce
  \item Grade: Secondary O-level 3/4
\end{itemize}

\textbf{Question}

A local air-conditioning service company keeps track of the number of air-conditioner filters it cleans and reuses each year. On average, the company cleans 2 air-conditioner filters per service and serves 4 times as many homes each year compared to the previous year. 

If $f$ represents the number of filters cleaned in the first year, express the number of filters cleaned in the $n$-th year in terms of $f$, using the laws of indices. 

In the $6$th year, due to improvements, every filter can now be reused for $2^{-2}$ as many services as initially (in the first year). Express the number of filters required per service in the $6$th year as a power of $2$, and explain the meaning of your answer with reference to negative indices.

\textbf{Solution}

The number of filters cleaned each year is multiplied by 4 every year. So, in the $n$-th year, the number of filters cleaned, $F_n= f \times 4^{n-1}$. 

Let us show this with laws of indices:
- After 1 year: $f$
- After 2 years: $f \times 4$
- After 3 years: $f \times 4^2$
- After $n$ years: $f \times 4^{n-1}$

In the $6$th year, the number of filters reused per service is $2^{-2}$ as many as in the first year. Originally, it was 1 filter per service. Now in the $6$th year:

Number of filters required per service $= 1 \times 2^{-2} = 2^{-2}$

Recall: $2^{-2} = \dfrac{1}{2^2} = \dfrac{1}{4}$.

This means in the $6$th year, the company needs only $\dfrac{1}{4}$ of a filter per service compared to the first year, or equivalently every filter can be used 4 times more. The negative index indicates division: $2^{-2}$ means divide by $2^2$, reducing the number of filters needed per service.

\section*{Question 189}
\textbf{Metadata}

\begin{itemize}
  \item Question ID: O3-BPOpr\_O3-BPRepFrI\_GPT4.1\_Services\_01
  \item Primary KC: BASE AND POWER | Operations | laws of indices
  \item Secondary KC: BASE AND POWER | Representation and concept  | fractional indices
  \item Topic: Services such as installation, maintenance, repairing, cleaning, laundry, e-commerce
  \item Grade: Secondary O-level 3/4
\end{itemize}

\textbf{Question}

A cleaning company offers a special machine that cleans carpets. The manufacturer claims the machine cleans an area of $4^{3/2}$ square metres in one hour. The cleaning efficiency (area cleaned per hour) doubles each year due to improvements in technology. 

(a) Express $4^{3/2}$ as a single integer.

(b) If the efficiency doubles each year, use the laws of indices to express the cleaning area for the machine after three years as a power of 2. What is the area, in square metres, the machine can clean per hour after three years?

\textbf{Solution}

(a) $4^{3/2} = (4^{1/2})^3 = (2)^3 = 8$. So the machine cleans 8 square metres per hour.

(b) Each year the cleaning area doubles, so after three years, the area cleaned per hour is $8 \times 2^3 = 8 \times 8 = 64$ square metres.
Alternatively, using indices:
$4^{3/2} \times 2^3 = (2^2)^{3/2} \times 2^3 = 2^{2 \times 3/2} \times 2^3 = 2^3 \times 2^3 = 2^{3+3} = 2^6 = 64$.

So, after three years, the machine can clean $64$ square metres per hour.

\section*{Question 190}
\textbf{Metadata}

\begin{itemize}
  \item Question ID: O3-STOprUn\_O3-STOprIns\_GPT4.1\_Services\_01
  \item Primary KC: SET | Set operations | union of two sets
  \item Secondary KC: SET | Set operations | intersection of two sets
  \item Topic: Services such as installation, maintenance, repairing, cleaning, laundry, e-commerce
  \item Grade: Secondary O-level 3/4
\end{itemize}

\textbf{Question}

A company offers two types of home services: aircon maintenance and aircon cleaning. In a particular week, 70 households engaged the company for aircon maintenance, 50 households engaged the company for aircon cleaning, and 30 households engaged the company for both services.

(a) How many households engaged the company for at least one of the two services that week?

(b) How many households engaged the company for only one type of service?

(c) If the total number of unique households served that week is 90, how many households did not engage the company for either aircon maintenance or cleaning?

\textbf{Solution}

(a) Let $A$ be the set of households that engaged aircon maintenance, and $B$ be the set of households that engaged aircon cleaning. We are asked for $|A \cup B|$.

By the principle of inclusion-exclusion:

$$|A \cup B| = |A| + |B| - |A \cap B|$$
$$|A \cup B| = 70 + 50 - 30 = 90$$

So, 90 households engaged the company for at least one of the two services.

(b) Households that engaged only one type of service: \newline
Those who engaged only maintenance: $|A| - |A \cap B| = 70 - 30 = 40$ \newline
Those who engaged only cleaning: $|B| - |A \cap B| = 50 - 30 = 20$\newline
Thus, the number of households that engaged the company for only one type of service is:
$$40 + 20 = 60$$

(c) If the total number of unique households served that week is 90, and all these engaged at least one service, then $90 - 90 = 0$ households did not engage either service. If, however, the total population of possible households is greater than 90, let us say $N$ households, the number that did not engage is $N - 90$. But as phrased, with 'unique households served is 90' and $|A \cup B| = 90$, all served households took at least one service, so the answer is zero.

\section*{Question 191}
\textbf{Metadata}

\begin{itemize}
  \item Question ID: O3-MXMulSM\_O3-MXAdd\_GPT4.1\_Services\_01
  \item Primary KC: MATRICES | Multiplication | product of a scalar quantity and a matrix
  \item Secondary KC: MATRICES | Addition | addition of matrices
  \item Topic: Services such as installation, maintenance, repairing, cleaning, laundry, e-commerce
  \item Grade: Secondary O-level 3/4
\end{itemize}

\textbf{Question}

A cleaning company manages the number of hours its workers spend at two different sites in a week. The hours are represented in the following matrix:

$
A = \begin{bmatrix} 12 & 9 \\ 8 & 15 \end{bmatrix}
$

where the first row represents the hours by Team 1 at Site A and Site B respectively, and the second row represents the hours by Team 2 at Site A and Site B respectively.

The company decides to increase the hours of all teams at both sites by 20\% next week (i.e., multiply the matrix by a scalar). After the increase, an additional matrix

$
B = \begin{bmatrix} 1 & 2 \\ 0 & 1 \end{bmatrix}
$

representing extra assistance hours is to be added to the increased hours.

(a) Find the new matrix representing the increased hours after the 20\% increment.

(b) Calculate the final total hours matrix after including the extra assistance hours.

\textbf{Solution}

(a) First, increase all hours by 20\%. This is multiplying each entry of matrix $A$ by 1.2:

$
1.2 \times A = 1.2 \times \begin{bmatrix}12 & 9 \\ 8 & 15\end{bmatrix} = \begin{bmatrix}1.2 \times 12 & 1.2 \times 9 \\ 1.2 \times 8 & 1.2 \times 15\end{bmatrix} = \begin{bmatrix}14.4 & 10.8 \\ 9.6 & 18\end{bmatrix}
$

(b) Add the extra assistance hours:

$
\begin{bmatrix}14.4 & 10.8 \\ 9.6 & 18\end{bmatrix} + \begin{bmatrix}1 & 2 \\ 0 & 1\end{bmatrix} = \begin{bmatrix}14.4 + 1 & 10.8 + 2 \\ 9.6 + 0 & 18 + 1\end{bmatrix} = \begin{bmatrix}15.4 & 12.8 \\ 9.6 & 19\end{bmatrix}
$

So, the final total hours matrix is:

$
\boxed{\begin{bmatrix}15.4 & 12.8 \\ 9.6 & 19\end{bmatrix}}
$
This matrix shows the final total hours worked by Team 1 and Team 2 at Site A and Site B respectively after the increase and addition of extra hours.

\section*{Question 192}
\textbf{Metadata}

\begin{itemize}
  \item Question ID: O3-MXMulSM\_O3-MXSub\_GPT4.1\_Services\_01
  \item Primary KC: MATRICES | Multiplication | product of a scalar quantity and a matrix
  \item Secondary KC: MATRICES | Subtraction | subtraction of matrices
  \item Topic: Services such as installation, maintenance, repairing, cleaning, laundry, e-commerce
  \item Grade: Secondary O-level 3/4
\end{itemize}

\textbf{Question}

A cleaning services company manages three types of cleaning jobs each week — residential, office, and commercial. The number of jobs completed over two weeks is represented by the following matrices, where the first row is Week 1 and the second row is Week 2:

$
A = \begin{bmatrix} 12 & 15 & 8 \\ 14 & 11 & 10 \end{bmatrix}
$

The company introduces a new policy where they double the number of commercial cleaning jobs for both weeks, but the number of residential and office jobs remains unchanged. Represent this change as a matrix operation. After the change, another team completed the following number of jobs:

$
B = \begin{bmatrix} 3 & 2 & 1 \\ 2 & 1 & 2 \end{bmatrix}
$

Find the matrix that shows the difference in the number of jobs (after the change) between the company's teams for each job type and week.

\textbf{Solution}

First, we represent the change: doubling the number of commercial cleaning jobs (the third column) in matrix A. We can create a scalar matrix:

$
C = \begin{bmatrix} 1 & 1 & 2 \\ 1 & 1 & 2 \end{bmatrix}
$

Now, multiply each entry in A by the corresponding scalar in C (since the question asks for doubling only the third column):

$
A' = \begin{bmatrix} 12 \times 1 & 15 \times 1 & 8 \times 2 \\ 14 \times 1 & 11 \times 1 & 10 \times 2 \end{bmatrix} = \begin{bmatrix} 12 & 15 & 16 \\ 14 & 11 & 20 \end{bmatrix}
$

Now, subtract matrix B from A' to find the difference in jobs:

$
D = A' - B = \begin{bmatrix} 12-3 & 15-2 & 16-1 \\ 14-2 & 11-1 & 20-2 \end{bmatrix} = \begin{bmatrix} 9 & 13 & 15 \\ 12 & 10 & 18 \end{bmatrix}
$

So, the matrix showing the difference in the number of jobs between the teams for each job type and week is:

$
\boxed{\begin{bmatrix} 9 & 13 & 15 \\ 12 & 10 & 18 \end{bmatrix}}
$


\section*{Question 193}
\textbf{Metadata}

\begin{itemize}
  \item Question ID: O3-MXMulSM\_O3-MXMul\_GPT4.1\_Services\_01
  \item Primary KC: MATRICES | Multiplication | product of a scalar quantity and a matrix
  \item Secondary KC: MATRICES | Multiplication | multiplication of matrices
  \item Topic: Services such as installation, maintenance, repairing, cleaning, laundry, e-commerce
  \item Grade: Secondary O-level 3/4
\end{itemize}

\textbf{Question}

A cleaning company stores data about its weekly services using matrices. The number of cleaning tasks for three types of services (residential, commercial, industrial) over two weeks is recorded in matrix $A$:

$
A = \begin{bmatrix} 12 & 15 & 18 \\ 14 & 16 & 20 \end{bmatrix}
$

The cost (in dollars) for each service type is shown in vector $B$:

$
B = \begin{bmatrix} 30 \\ 50 \\ 80 \end{bmatrix}
$

If the company increases the charges for all services by 10% after the two weeks, answer the following:

1. Find the new cost vector $B'$ after the 10% increase is represented as a scalar multiple of $B$.
2. Calculate the total earnings in each week after the price increase, by multiplying the weekly service matrix by the new price vector.

\textbf{Solution}

1. To represent a 10% increase, multiply the original price vector $B$ by the scalar $1.1$:

$
B' = 1.1 \times B = 1.1 \begin{bmatrix} 30 \\ 50 \\ 80 \end{bmatrix} = \begin{bmatrix} 33 \\ 55 \\ 88 \end{bmatrix}
$

2. The earnings for each week are calculated by multiplying the matrix $A$ and the new price vector $B'$ (noting that $A$ has rows as weeks and columns as service types):

$
\text{Total earnings in each week} = A \times B'
$

Calculate:

Week 1: $[12,\ 15,\ 18] \cdot \begin{bmatrix} 33 \\ 55 \\ 88 \end{bmatrix}$

$= 12 \times 33 + 15 \times 55 + 18 \times 88$
$= 396 + 825 + 1584 = 2805$

Week 2: $[14,\ 16,\ 20] \cdot \begin{bmatrix} 33 \\ 55 \\ 88 \end{bmatrix}$

$= 14 \times 33 + 16 \times 55 + 20 \times 88$
$= 462 + 880 + 1760 = 3102$

$
\text{Total earnings after price increase:}
$

- Week 1: \$2,805
- Week 2: \$3,102

\section*{Question 194}
\textbf{Metadata}

\begin{itemize}
  \item Question ID: O3-MXSub\_O3-MXAdd\_GPT4.1\_Services\_01
  \item Primary KC: MATRICES | Subtraction | subtraction of matrices
  \item Secondary KC: MATRICES | Addition | addition of matrices
  \item Topic: Services such as installation, maintenance, repairing, cleaning, laundry, e-commerce
  \item Grade: Secondary O-level 3/4
\end{itemize}

\textbf{Question}

A cleaning company records the number of different types of cleaning jobs completed in two different weeks. The numbers are shown in matrices below, where the columns represent 'Office Cleaning', 'Home Cleaning', and 'Carpet Cleaning' jobs:

In Week 1: $
A = \begin{bmatrix} 12 & 9 & 5 \\ 8 & 6 & 7 \end{bmatrix}
$
In Week 2: $
B = \begin{bmatrix} 10 & 7 & 6 \\ 9 & 8 & 5 \end{bmatrix}
$

(a) Calculate the total number of each type of cleaning job completed in the two weeks combined.

(b) Calculate the change in the number of each type of cleaning job from Week 1 to Week 2.

\textbf{Solution}

(a) To find the total number of each type of cleaning job completed in the two weeks combined, we add the two matrices:

$
A + B = \begin{bmatrix} 12 & 9 & 5 \\ 8 & 6 & 7 \end{bmatrix} + \begin{bmatrix} 10 & 7 & 6 \\ 9 & 8 & 5 \end{bmatrix} = \begin{bmatrix} 12+10 & 9+7 & 5+6 \\ 8+9 & 6+8 & 7+5 \end{bmatrix} = \begin{bmatrix} 22 & 16 & 11 \\ 17 & 14 & 12 \end{bmatrix}
$

(b) To calculate the change in the number of each type of job from Week 1 to Week 2, subtract the Week 1 matrix from the Week 2 matrix:

$
B - A = \begin{bmatrix} 10 & 7 & 6 \\ 9 & 8 & 5 \end{bmatrix} - \begin{bmatrix} 12 & 9 & 5 \\ 8 & 6 & 7 \end{bmatrix} = \begin{bmatrix} 10-12 & 7-9 & 6-5 \\ 9-8 & 8-6 & 5-7 \end{bmatrix} = \begin{bmatrix} -2 & -2 & 1 \\ 1 & 2 & -2 \end{bmatrix}
$

So, for each job type in each row, this matrix shows the increase (+) or decrease (−) in the number of jobs from Week 1 to Week 2.

\section*{Question 195}
\textbf{Metadata}

\begin{itemize}
  \item Question ID: O3-MXMul\_O3-MXMulSM\_GPT4.1\_Services\_01
  \item Primary KC: MATRICES | Multiplication | multiplication of matrices
  \item Secondary KC: MATRICES | Multiplication | product of a scalar quantity and a matrix
  \item Topic: Services such as installation, maintenance, repairing, cleaning, laundry, e-commerce
  \item Grade: Secondary O-level 3/4
\end{itemize}

\textbf{Question}

A company in Singapore offers three types of laundry services: Basic Wash, Premium Wash, and Express Wash. The number of each service offered in two different branches, Jurong and Bishan, during a week is given in matrix $A$:

$$
A = \begin{bmatrix} 30 & 20 & 15 \\ 25 & 18 & 12 \end{bmatrix}
$$

where the rows represent the branches (first row: Jurong, second row: Bishan), and the columns represent the type of service (first column: Basic Wash, second column: Premium Wash, third column: Express Wash).

Each service has a different price:

- Basic Wash: $4 per order
- Premium Wash: $7 per order
- Express Wash: $10 per order

Let the price vector be $P = \begin{bmatrix} 4 \\ 7 \\ 10 \end{bmatrix}$.

(a) Find the total revenue matrix for each branch by multiplying matrix $A$ by vector $P$.

(b) If the company gives a 10% discount on all services for a special promotion, represent this using a scalar multiplication and calculate the total revenue matrix after the discount.

\textbf{Solution}

(a) To find the total revenue for each branch, we multiply matrix $A$ by column vector $P$:

$$
AP = \begin{bmatrix} 30 & 20 & 15 \\ 25 & 18 & 12 \end{bmatrix}
\times
\begin{bmatrix} 4 \\ 7 \\ 10 \end{bmatrix}
$$

For Jurong branch:
$$(30 \times 4) + (20 \times 7) + (15 \times 10) = 120 + 140 + 150 = 410$$

For Bishan branch:
$$(25 \times 4) + (18 \times 7) + (12 \times 10) = 100 + 126 + 120 = 346$$

So, the total revenue matrix is:
$$
AP = \begin{bmatrix} 410 \\ 346 \end{bmatrix}
$$

(b) A 10% discount means each value is multiplied by 0.9 (i.e., 90% of the original revenue). This is scalar multiplication:

$$
0.9 \times \begin{bmatrix} 410 \\ 346 \end{bmatrix} = \begin{bmatrix} 369 \\ 311.4 \end{bmatrix}
$$

So, after the discount, the total revenue from Jurong branch is $369, and from Bishan branch is $311.40.

\section*{Question 196}
\textbf{Metadata}

\begin{itemize}
  \item Question ID: O3-MXMul\_O3-MXAdd\_GPT4.1\_Services\_01
  \item Primary KC: MATRICES | Multiplication | multiplication of matrices
  \item Secondary KC: MATRICES | Addition | addition of matrices
  \item Topic: Services such as installation, maintenance, repairing, cleaning, laundry, e-commerce
  \item Grade: Secondary O-level 3/4
\end{itemize}

\textbf{Question}

A cleaning company provides two types of cleaning services: Basic and Premium. The number of hours taken (in hours) by each of their 3 teams (Team A, Team B, Team C) on two different days is represented by the following matrices:

On Monday:
$
M = \begin{pmatrix} 2 & 3 \\ 4 & 1 \\ 3 & 2 \end{pmatrix}
$
On Tuesday:
$
T = \begin{pmatrix} 1 & 4 \\ 2 & 3 \\ 5 & 1 \end{pmatrix}
$
Each type of service has a different hourly rate. The hourly rates (in dollars) for Basic and Premium services are represented by the matrix:
$
R = \begin{pmatrix} 20 & 30 \\ \end{pmatrix}
$

(a) Find the total number of hours spent by each team on both days for each service by adding matrices $M$ and $T$.

(b) Calculate the total earnings of each team from both services for both days by multiplying the matrix obtained in part (a) by the rate matrix $R^T$ (the transpose of $R$).

\textbf{Solution}

Part (a):
Sum the matrices $M$ and $T$ to find total hours spent by each team on each service for both days:
$
M + T = \begin{pmatrix} 2+1 & 3+4 \\ 4+2 & 1+3 \\ 3+5 & 2+1 \end{pmatrix} = \begin{pmatrix} 3 & 7 \\ 6 & 4 \\ 8 & 3 \end{pmatrix}
$

So, the total hours matrix is:
$
H = \begin{pmatrix} 3 & 7 \\ 6 & 4 \\ 8 & 3 \end{pmatrix}
$

Part (b):
Hourly rate matrix $R = \begin{pmatrix} 20 & 30 \end{pmatrix}$. Its transpose is $R^T = \begin{pmatrix} 20 \\ 30 \end{pmatrix}$.

Multiply the matrix $H$ with $R^T$ to get total earnings for each team:
$
\begin{pmatrix} 3 & 7 \\ 6 & 4 \\ 8 & 3 \end{pmatrix} \times \begin{pmatrix} 20 \\ 30 \end{pmatrix} = \begin{pmatrix} (3\times20)+(7\times30) \\ (6\times20)+(4\times30) \\ (8\times20)+(3\times30) \end{pmatrix} = \begin{pmatrix} 60+210 \\ 120+120 \\ 160+90 \end{pmatrix} = \begin{pmatrix} 270 \\ 240 \\ 250 \end{pmatrix}
$

Therefore, the total earnings for each team are:

- Team A: $270
- Team B: $240
- Team C: $250



\section*{Question 197}
\textbf{Metadata}

\begin{itemize}
  \item Question ID: O3-MXMul\_O3-MXSub\_GPT4.1\_Services\_01
  \item Primary KC: MATRICES | Multiplication | multiplication of matrices
  \item Secondary KC: MATRICES | Subtraction | subtraction of matrices
  \item Topic: Services such as installation, maintenance, repairing, cleaning, laundry, e-commerce
  \item Grade: Secondary O-level 3/4
\end{itemize}

\textbf{Question}

A cleaning company provides services to two offices, Office A and Office B, every week. The tasks are divided among two teams: Team X and Team Y. The number of hours (per week) taken by each team for vacuuming and mopping in each office is represented by the following matrices:

For Week 1:
$
M_1 = \begin{bmatrix} 4 & 3 \\ 5 & 2 \end{bmatrix}
$

For Week 2:
$
M_2 = \begin{bmatrix} 6 & 4 \\ 3 & 5 \end{bmatrix}
$

where the first row represents Team X (vacuuming and mopping), and the second row represents Team Y (vacuuming and mopping).

The company wants to calculate the total man-hours for Week 2 by multiplying the following matrix of tasks (columns: vacuuming, mopping) by the cost per hour for each task:
$
C = \begin{bmatrix} 12 & 10 \\ 12 & 10 \end{bmatrix}
$

(a) Find the cost matrix for Week 2 by multiplying \( M_2 \) by \( C \).

(b) The company wants to know by how much the cost (in dollars) has increased or decreased from Week 1 to Week 2. Calculate the difference matrix by using the result from (a) and subtracting the corresponding cost matrix for Week 1 (computed in a similar way).

\textbf{Solution}

(a) First, multiply the matrices for Week 2:
Given:
$
M_2 = \begin{bmatrix} 6 & 4 \\ 3 & 5 \end{bmatrix},
C = \begin{bmatrix} 12 & 10 \\ 12 & 10 \end{bmatrix}
$

Matrix multiplication \( M_2 \times C \):
$
\begin{bmatrix} 6 & 4 \\ 3 & 5 \end{bmatrix} \times \begin{bmatrix} 12 & 10 \\ 12 & 10 \end{bmatrix}
= \begin{bmatrix}
(6 \times 12 + 4 \times 12) & (6 \times 10 + 4 \times 10) \\
(3 \times 12 + 5 \times 12) & (3 \times 10 + 5 \times 10)
\end{bmatrix}
= \begin{bmatrix} (72 + 48) & (60 + 40) \\ (36 + 60) & (30 + 50) \end{bmatrix}
= \begin{bmatrix} 120 & 100 \\ 96 & 80 \end{bmatrix}
$
So, the cost matrix for Week 2 is:
\(
\begin{bmatrix} 120 & 100 \\ 96 & 80 \end{bmatrix}
\) dollars.

(b) Calculate the cost matrix for Week 1 by multiplying \( M_1 \) by \( C \):
$
M_1 = \begin{bmatrix} 4 & 3 \\ 5 & 2 \end{bmatrix}
$
$
M_1 \times C = \begin{bmatrix} 4 & 3 \\ 5 & 2 \end{bmatrix} \times \begin{bmatrix} 12 & 10 \\ 12 & 10 \end{bmatrix}
= \begin{bmatrix}
(4 \times 12 + 3 \times 12) & (4 \times 10 + 3 \times 10) \\
(5 \times 12 + 2 \times 12) & (5 \times 10 + 2 \times 10)
\end{bmatrix}
= \begin{bmatrix} (48 + 36) & (40 + 30) \\ (60 + 24) & (50 + 20) \end{bmatrix}
= \begin{bmatrix} 84 & 70 \\ 84 & 70 \end{bmatrix}
$

Now, subtract the Week 1 cost matrix from the Week 2 cost matrix:
$
\begin{bmatrix} 120 & 100 \\ 96 & 80 \end{bmatrix} - \begin{bmatrix} 84 & 70 \\ 84 & 70 \end{bmatrix} = \begin{bmatrix} 120-84 & 100-70 \\ 96-84 & 80-70 \end{bmatrix} = \begin{bmatrix} 36 & 30 \\ 12 & 10 \end{bmatrix}
$

**Final Answer:**
(a) The cost matrix for Week 2 is \( \begin{bmatrix} 120 & 100 \\ 96 & 80 \end{bmatrix} \) dollars.

(b) The difference in cost from Week 1 to Week 2 is \( \begin{bmatrix} 36 & 30 \\ 12 & 10 \end{bmatrix} \) dollars, showing the amount by which cost increased for each team and task.

\section*{Question 198}
\textbf{Metadata}

\begin{itemize}
  \item Question ID: O3-SPFndQtl\_O3-SPFndIQR\_GPT4.1\_Services\_01
  \item Primary KC: STATISTICS AND PROBABILITY | Finding quartiles | finding quartiles for a set of data
  \item Secondary KC: STATISTICS AND PROBABILITY | Finding range | finding interquartile range as measures of spread for a set of data 
  \item Topic: Services such as installation, maintenance, repairing, cleaning, laundry, e-commerce
  \item Grade: Secondary O-level 3/4
\end{itemize}

\textbf{Question}

A cleaning company recorded the number of homes cleaned by each of its 9 employees in one week. The numbers of homes cleaned are: $7, 9, 10, 10, 12, 15, 16, 17, 18$. 

(a) Find the first quartile ($Q_1$), median ($Q_2$), and third quartile ($Q_3$) for the set of data. 

(b) Calculate the interquartile range (IQR) for the data set and explain what it represents in this context.

\textbf{Solution}

To find the quartiles:

Step 1: Arrange data in ascending order (already done): $7, 9, 10, 10, 12, 15, 16, 17, 18$

Step 2: Calculate the median ($Q_2$):
There are 9 values. Median is the 5th value:
$Q_2 = 12$

Step 3: Find $Q_1$ (lower quartile):
Lower half (before median): $7, 9, 10, 10$
Median of these 4 values is the average of 2nd and 3rd values:
$Q_1 = \frac{9+10}{2} = 9.5$

Step 4: Find $Q_3$ (upper quartile):
Upper half (after median): $15, 16, 17, 18$
Median of these 4 values is the average of 2nd and 3rd values:
$Q_3 = \frac{16+17}{2} = 16.5$

(a) Quartiles:
$Q_1 = 9.5$, $Q_2 = 12$, $Q_3 = 16.5$

(b) Interquartile Range (IQR):
$IQR = Q_3 - Q_1 = 16.5 - 9.5 = 7$

The IQR of 7 means that the middle 50% of the employees cleaned between 9.5 and 16.5 homes. This measures the spread of the number of homes cleaned by the employees, excluding extreme values.

\section*{Question 199}
\textbf{Metadata}

\begin{itemize}
  \item Question ID: O3-SPFndstd\_O2-SPFndmean\_GPT4.1\_Services\_01
  \item Primary KC: STATISTICS AND PROBABILITY | Finding standard deviation | calculation of the standard deviation for a set of data
  \item Secondary KC: STATISTICS AND PROBABILITY | Finding mean deviation | calculation of the mean for a set of data
  \item Topic: Services such as installation, maintenance, repairing, cleaning, laundry, e-commerce
  \item Grade: Secondary O-level 3/4
\end{itemize}

\textbf{Question}

A laundry shop records the number of shirts cleaned each day for a week as follows: 52, 48, 55, 50, 60, 53, and 52.

(a) Calculate the mean number of shirts cleaned per day.

(b) Using your answer from (a), calculate the standard deviation of the number of shirts cleaned per day, correct to 2 decimal places.

\textbf{Solution}

Let the number of shirts cleaned each day be: 52, 48, 55, 50, 60, 53, 52.

(a) The mean is given by:
$
\text{Mean} = \frac{52+48+55+50+60+53+52}{7} = \frac{370}{7} = 52.857 \approx 52.86
$

(b) The standard deviation, $s$, is given by:
$
s = \sqrt{\frac{1}{n} \sum_{i=1}^{n} (x_i - \overline{x})^2}
$
Plug in the values:
$
\begin{align*}
(52 - 52.857)^2 &= (-0.857)^2 = 0.735 \\
(48 - 52.857)^2 &= (-4.857)^2 = 23.602 \\
(55 - 52.857)^2 &= (2.143)^2 = 4.594 \\
(50 - 52.857)^2 &= (-2.857)^2 = 8.174 \\
(60 - 52.857)^2 &= (7.143)^2 = 51.047 \\
(53 - 52.857)^2 &= (0.143)^2 = 0.020 \\
(52 - 52.857)^2 &= (-0.857)^2 = 0.735 \\
\end{align*}
$
Sum of squares:
$
0.735 + 23.602 + 4.594 + 8.174 + 51.047 + 0.020 + 0.735 = 88.907
$
So,
$
s = \sqrt{\frac{88.907}{7}} = \sqrt{12.701} = 3.566 \approx 3.57
$

**Answers:**

(a) The mean is $52.86$ shirts (rounded to 2 decimal places).

(b) The standard deviation is $3.57$ shirts (rounded to 2 decimal places).

\section*{Question 200}
\textbf{Metadata}

\begin{itemize}
  \item Question ID: O3-SPAddProb\_O2-SPRepPrSE\_GPT4.1\_Services\_01
  \item Primary KC: STATISTICS AND PROBABILITY | Addition | addition of probabilities
  \item Secondary KC: STATISTICS AND PROBABILITY | Representation and concept | probability of single events
  \item Topic: Services such as installation, maintenance, repairing, cleaning, laundry, e-commerce
  \item Grade: Secondary O-level 3/4
\end{itemize}

\textbf{Question}

A laundry service company records that the probability that a customer brings in a stained shirt for laundry on a Monday is $0.13$, and the probability that the same customer brings in a torn shirt for repair on that day is $0.07$. Given that no customer brings in both a stained and a torn shirt at the same time on Monday, what is the probability that a customer brings in either a stained shirt or a torn shirt on a Monday?

Express your answer as a decimal.

\textbf{Solution}

Let $P(S)$ be the probability that a customer brings in a stained shirt, and $P(T)$ be the probability that a customer brings in a torn shirt. 

Given:

$P(S) = 0.13$

$P(T) = 0.07$

Since no customer brings in both a stained and a torn shirt together on Monday, these two events are mutually exclusive.

So, the probability that a customer brings in either a stained shirt or a torn shirt is:

$$P(S \text{ or } T) = P(S) + P(T) = 0.13 + 0.07 = 0.20$$

Thus, the probability that a customer brings in either a stained shirt or a torn shirt for laundry or repair on a Monday is $0.20$.

\section*{Question 201}
\textbf{Metadata}

\begin{itemize}
  \item Question ID: O3-SPAddProb\_O3-SPFndPrCE\_GPT4.1\_Services\_01
  \item Primary KC: STATISTICS AND PROBABILITY | Addition | addition of probabilities
  \item Secondary KC: STATISTICS AND PROBABILITY | Finding probability | probability of simple combined events
  \item Topic: Services such as installation, maintenance, repairing, cleaning, laundry, e-commerce
  \item Grade: Secondary O-level 3/4
\end{itemize}

\textbf{Question}

A home cleaning company offers two types of services: window cleaning and aircon servicing. The probability that a client books window cleaning is $0.45$. The probability that a client books aircon servicing is $0.35$. The probability that a client books both services is $0.20$.

(a) What is the probability that a client books at least one of the two services?

(b) What is the probability that a client books only one of the two services?

\textbf{Solution}

(a) To find the probability that a client books at least one of the two services, we use the formula for the probability of the union of two events:

$$
P(A \text{ or } B) = P(A) + P(B) - P(A \text{ and } B)
$$
Substitute the given values:

$$
P(\text{Window cleaning or Aircon servicing}) = 0.45 + 0.35 - 0.20 = 0.60
$$
So, the probability that a client books at least one service is $0.60$.

(b) The probability that a client books only one service (window cleaning or aircon servicing, but not both) is:

$$
P(\text{only one}) = P(\text{Window only}) + P(\text{Aircon only})
$$

Where

$$
P(\text{Window only}) = P(\text{Window}) - P(\text{Both}) = 0.45 - 0.20 = 0.25
$$
$$
P(\text{Aircon only}) = P(\text{Aircon}) - P(\text{Both}) = 0.35 - 0.20 = 0.15
$$
So,
$$
P(\text{only one}) = 0.25 + 0.15 = 0.40
$$

Therefore, the probability that a client books only one of the two services is $0.40$.

\section*{Question 202}
\textbf{Metadata}

\begin{itemize}
  \item Question ID: O3-SPMulProb\_O2-SPRepPrSE\_GPT4.1\_Services\_01
  \item Primary KC: STATISTICS AND PROBABILITY | Multiplication | multiplication of probabilities
  \item Secondary KC: STATISTICS AND PROBABILITY | Representation and concept | probability of single events
  \item Topic: Services such as installation, maintenance, repairing, cleaning, laundry, e-commerce
  \item Grade: Secondary O-level 3/4
\end{itemize}

\textbf{Question}

A washing machine repair company has observed from its past records that the probability of successfully repairing a machine on the first visit is $0.7$. The probability that a customer books an appointment for a repair when their washing machine breaks down is $0.8$.

(a) What is the probability that a customer who needs a repair both books an appointment and has their washing machine successfully repaired on the first visit?

(b) If there are 100 customers whose washing machines break down, how many would you expect to have their washing machines successfully repaired on the first visit?

\textbf{Solution}

**(a)**

Let $A$ be the event that a customer books an appointment ($P(A) = 0.8$).
Let $B$ be the event that the repair is successful on the first visit ($P(B) = 0.7$).

Assuming these two events are independent, the probability that both happen is:

$
P(A \text{ and } B) = P(A) \times P(B) = 0.8 \times 0.7 = 0.56
$

**(b)**

For 100 customers, the expected number who will both book an appointment and have a successful repair on the first visit is

$
100 \times 0.56 = 56
$

**Final Answers:**

(a) $0.56$

(b) $56$ customers

\section*{Question 203}
\textbf{Metadata}

\begin{itemize}
  \item Question ID: O3-SPMulProb\_O3-SPFndPrCE\_GPT4.1\_Services\_01
  \item Primary KC: STATISTICS AND PROBABILITY | Multiplication | multiplication of probabilities
  \item Secondary KC: STATISTICS AND PROBABILITY | Finding probability | probability of simple combined events
  \item Topic: Services such as installation, maintenance, repairing, cleaning, laundry, e-commerce
  \item Grade: Secondary O-level 3/4
\end{itemize}

\textbf{Question}

A cleaning company offers services for both offices and homes. The probability that an office cleaning job is assigned to Team A is $\frac{2}{5}$, and the probability that a home cleaning job is assigned to Team A is $\frac{1}{4}$. On a particular day, the company receives one office cleaning job and one home cleaning job. 

What is the probability that both jobs are assigned to Team A?

\textbf{Solution}

Let $P(O)$ be the probability that the office job is assigned to Team A, which is $\frac{2}{5}$. Let $P(H)$ be the probability that the home job is assigned to Team A, which is $\frac{1}{4}$.

Assuming the assignment of teams for each job is independent, the probability that both jobs are assigned to Team A is:
$
P(\text{both jobs to Team A}) = P(O) \times P(H) = \frac{2}{5} \times \frac{1}{4} = \frac{2}{20} = \frac{1}{10}.
$

Therefore, the probability that both jobs are assigned to Team A is $\boxed{\frac{1}{10}}$.

\end{document}
