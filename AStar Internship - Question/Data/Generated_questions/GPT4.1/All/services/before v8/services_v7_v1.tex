\documentclass{article}
\usepackage[utf8]{inputenc}
\usepackage{amsmath}
\usepackage{amsfonts}
\usepackage{amssymb}
\usepackage{graphicx}
\usepackage{hyperref}
\title{All Services MWPs by GPT4.1 v7 v1}
\author{Tien Dung Doan}
\begin{document}
\maketitle
\section*{Question 1}
\textbf{Metadata}

\begin{itemize}
  \item Question ID: P3-WNAdd4d\_P1-WNCmp\_GPT4.1\_Services\_06
  \item Primary KC: WHOLE NUMBERS | Addition | adding whole numbers up to 4 digits
  \item Secondary KC: WHOLE NUMBERS | Comparison and ordering | comparing and ordering whole numbers
  \item Topic: Services such as installation, maintenance, repairing, cleaning, laundry, hotel, retail, e-commerce, streaming services, digital services etc.
  \item Grade: Primary 3
\end{itemize}

\textbf{Question}

A cleaning company received three orders for different cleaning services in one week. The first service was for cleaning an office building and cost \textdollar1530. The second service was for cleaning a house and cost \textdollar825. The third service was for cleaning a shop and cost \textdollar1205.

(a) What is the total amount the company earned from these three cleaning services?

(b) Arrange the amounts earned from each service in order, from the least to the greatest.

\textbf{Solution}

(a) To find the total amount earned, add the three amounts:

$1530 + 825 = 2355$

$2355 + 1205 = 3560$

So, the total amount the company earned is $\textdollar3560$.

(b) To arrange the amounts in order from the least to the greatest:

$825$ (house), $1205$ (shop), $1530$ (office building)

Therefore, the order is: $\textdollar825$, $\textdollar1205$, $\textdollar1530$.

\section*{Question 2}
\textbf{Metadata}

\begin{itemize}
  \item Question ID: P3-WNSub4d\_P1-WNCmp\_GPT4.1\_Services\_05
  \item Primary KC: WHOLE NUMBERS | Subtraction | subtracting whole numbers up to 4 digits
  \item Secondary KC: WHOLE NUMBERS | Comparison and ordering | comparing and ordering whole numbers
  \item Topic: Services such as installation, maintenance, repairing, cleaning, laundry, hotel, retail, e-commerce, streaming services, digital services etc.
  \item Grade: Primary 3
\end{itemize}

\textbf{Question}

A digital streaming company had $4,800$ subscribers at the start of the month. By the end of the month, the company lost $1,625$ subscribers who cancelled their subscriptions. At the same time, another similar company had $3,350$ subscribers. 

(a) How many subscribers does the first company have at the end of the month? 

(b) Which company has more subscribers at the end of the month, and by how many?

\textbf{Solution}

(a) Number of subscribers remaining at the end of the month for the first company:

$4,800 - 1,625 = 3,175$

So, the first company has $3,175$ subscribers at the end of the month.

(b) Comparing the two companies:

First company: $3,175$
Second company: $3,350$

$3,350 > 3,175$, so the second company has more subscribers.

Difference = $3,350 - 3,175 = 175$

Therefore, the second company has $175$ more subscribers than the first company at the end of the month.

\section*{Question 3}
\textbf{Metadata}

\begin{itemize}
  \item Question ID: P3-WNSub4d\_P1-WNAdd2nd\_GPT4.1\_Services\_04
  \item Primary KC: WHOLE NUMBERS | Subtraction | subtracting whole numbers up to 4 digits
  \item Secondary KC: WHOLE NUMBERS | Addition | adding whole numbers
  \item Topic: Services such as installation, maintenance, repairing, cleaning, laundry, hotel, retail, e-commerce, streaming services, digital services etc.
  \item Grade: Primary 3
\end{itemize}

\textbf{Question}

A hotel received $2750 in bookings on Monday. On Tuesday, the hotel received $1400 more in bookings than it did on Monday. After some guests cancelled, the hotel had to refund $1200 from Tuesday's bookings. How much money did the hotel have from bookings over the two days after the refund?

\textbf{Solution}

First, find out how much the hotel received in bookings on Tuesday:

$\textdollar2750 + \textdollar1400 = \textdollar4150$

Now, subtract the refund from Tuesday's bookings:

$\textdollar4150 - \textdollar1200 = \textdollar2950$

Finally, add Monday's bookings and the new Tuesday total:

$\textdollar2750 + \textdollar2950 = \textdollar5700$

\textbf{Answer:}

The hotel had \textdollar5700 from bookings over the two days after the refund.

\section*{Question 4}
\textbf{Metadata}

\begin{itemize}
  \item Question ID: P3-WNDivRmd3d\_P1-WNCmp\_GPT4.1\_Services\_04
  \item Primary KC: WHOLE NUMBERS | Division | dividing whole numbers up to 3 digits by 1 digit with remainder 
  \item Secondary KC: WHOLE NUMBERS | Comparison and ordering | comparing and ordering whole numbers
  \item Topic: Services such as installation, maintenance, repairing, cleaning, laundry, hotel, retail, e-commerce, streaming services, digital services etc.
  \item Grade: Primary 3
\end{itemize}

\textbf{Question}

Mrs Lim runs a laundry service. On Monday, she received $134$ shirts to wash. She divides the shirts equally among $4$ washing machines. How many shirts does each washing machine get, and how many shirts are left over? Compare the number of shirts each machine gets to the number of leftover shirts. Is the number of shirts each washing machine gets more, less or equal to the number of leftover shirts?

\textbf{Solution}

Divide $134$ by $4$:

$134 \div 4 = 33$ remainder $2$.

So, each washing machine gets $33$ shirts.

Number of shirts left over $= 2$.

Comparing $33$ and $2$, $33$ is more than $2$.

Therefore, the number of shirts each washing machine gets is more than the number of leftover shirts.

\section*{Question 5}
\textbf{Metadata}

\begin{itemize}
  \item Question ID: P3-WNDivRmd3d\_P1-WNAdd2nd\_GPT4.1\_Services\_04
  \item Primary KC: WHOLE NUMBERS | Division | dividing whole numbers up to 3 digits by 1 digit with remainder 
  \item Secondary KC: WHOLE NUMBERS | Addition | adding whole numbers
  \item Topic: Services such as installation, maintenance, repairing, cleaning, laundry, hotel, retail, e-commerce, streaming services, digital services etc.
  \item Grade: Primary 3
\end{itemize}

\textbf{Question}

A company has $275$ towels to send to several hotels for laundry service. They want to pack the towels equally into $4$ boxes. 

(a) How many towels can they put in each box? 

(b) How many towels will be left unpacked?

Later, the company receives $19$ more towels from the hotels. How many towels will they have in total after adding the ones left unpacked and the new towels received?

\textbf{Solution}

(a) To find how many towels can go in each box, divide $275$ by $4$:

\[ 275 \div 4 = 68 \text{ R } 3 \]

So, each box can contain $68$ towels.

(b) The remainder is $3$, so $3$ towels will be left unpacked.

After receiving $19$ more towels:

\[ 3 \text{ (leftover towels) } + 19 = 22 \]

So, the company will have $22$ towels in total after adding the ones left unpacked and the new towels received.

\section*{Question 6}
\textbf{Metadata}

\begin{itemize}
  \item Question ID: P3-WNDivRmd3d\_P1-WNSub2nd\_GPT4.1\_Services\_04
  \item Primary KC: WHOLE NUMBERS | Division | dividing whole numbers up to 3 digits by 1 digit with remainder 
  \item Secondary KC: WHOLE NUMBERS | Subtraction | subtracting whole numbers
  \item Topic: Services such as installation, maintenance, repairing, cleaning, laundry, hotel, retail, e-commerce, streaming services, digital services etc.
  \item Grade: Primary 3
\end{itemize}

\textbf{Question}

A laundry shop has 265 towels that need to be cleaned. The staff puts the towels equally into baskets, with each basket holding 6 towels. After filling as many baskets as possible, how many towels are not in any basket? If 4 towels are later removed because they are too dirty to be washed, how many towels are left without a basket?

\textbf{Solution}

First, divide $265$ by $6$ to see how many baskets can be filled and how many towels remain.

$265 \div 6 = 44$ remainder $1$ (since $6 \times 44 = 264$)

So, $44$ baskets are filled, and $1$ towel is left without a basket.

If $4$ towels are removed, the towels without a basket after filling the baskets are:

$1 - 4 = -3$

Since you cannot have a negative number of towels left, this means there are no towels left without a basket after removing the $4$ towels. In fact, $3$ of the towels that were going to be placed in the baskets will not be there because there are only $261$ towels remaining ($265 - 4 = 261$).

$261 \div 6 = 43$ remainder $3$ (since $6 \times 43 = 258$ and $261 - 258 = 3$)

So, after removing $4$ towels, $3$ towels are left without a basket.

\section*{Question 7}
\textbf{Metadata}

\begin{itemize}
  \item Question ID: P3-WNDivRmd3d\_P1-WNMul2nd\_GPT4.1\_Services\_04
  \item Primary KC: WHOLE NUMBERS | Division | dividing whole numbers up to 3 digits by 1 digit with remainder 
  \item Secondary KC: WHOLE NUMBERS | Multiplication | multiplying whole numbers
  \item Topic: Services such as installation, maintenance, repairing, cleaning, laundry, hotel, retail, e-commerce, streaming services, digital services etc.
  \item Grade: Primary 3
\end{itemize}

\textbf{Question}

A cleaning company has $238$ towels to wash after cleaning rooms in a hotel. They want to put an equal number of towels into $4$ washing machines. 

(a) How many towels will each washing machine have if the towels are divided equally?

(b) After the towels are divided, how many towels are left over?

(c) If each washing machine takes $3$ towels to be dried in one round, how many towels will be dried in total in one round for all the machines?

\textbf{Solution}

(a) To find out how many towels each washing machine gets, divide $238$ by $4$: 

$238 \div 4 = 59$ remainder $2$.

So, each washing machine will have $59$ towels.

(b) The remainder is $2$, so there will be $2$ towels left over.

(c) Each machine takes $3$ towels to be dried in one round. There are $4$ machines. So,

$3 \times 4 = 12$

So, $12$ towels will be dried in total in one round for all the machines.

\section*{Question 8}
\textbf{Metadata}

\begin{itemize}
  \item Question ID: P3-WNMul3d1d\_P1-WNCmp\_GPT4.1\_Services\_04
  \item Primary KC: WHOLE NUMBERS | Multiplication | multiplying whole numbers up to 3 digits by 1 digit
  \item Secondary KC: WHOLE NUMBERS | Comparison and ordering | comparing and ordering whole numbers
  \item Topic: Services such as installation, maintenance, repairing, cleaning, laundry, hotel, retail, e-commerce, streaming services, digital services etc.
  \item Grade: Primary 3
\end{itemize}

\textbf{Question}

A laundry shop washes $126$ shirts every day. Each worker can wash $4$ shirts in one hour. 

(a) How many hours does it take for one worker to wash all $126$ shirts?

(b) On another day, the laundry shop receives $148$ shirts. Compare the number of shirts washed on the two days. On which day did they wash more shirts?

\textbf{Solution}

(a) To find out how many hours one worker takes to wash all $126$ shirts, we divide $126$ by $4$.

$126 \div 4 = 31\ \text{remainder}\ 2.$

This means:
$4 \times 31 = 124$

So, $31$ hours to wash $124$ shirts, with $2$ shirts remaining.

It takes $31$ hours for one worker to wash $124$ shirts. For the $2$ remaining shirts, the worker needs $1$ more hour (as $1$ hour is required to wash up to $4$ shirts).

So, total hours needed $= 31 + 1 = 32$ hours.

\textbf{Answer:} It takes $32$ hours for one worker to wash all $126$ shirts.

(b) Compare $126$ shirts and $148$ shirts.

$148 > 126$

\textbf{Answer:} The laundry shop washed more shirts on the day they received $148$ shirts.

\section*{Question 9}
\textbf{Metadata}

\begin{itemize}
  \item Question ID: P3-WNMul3d1d\_P1-WNAdd2nd\_GPT4.1\_Services\_04
  \item Primary KC: WHOLE NUMBERS | Multiplication | multiplying whole numbers up to 3 digits by 1 digit
  \item Secondary KC: WHOLE NUMBERS | Addition | adding whole numbers
  \item Topic: Services such as installation, maintenance, repairing, cleaning, laundry, hotel, retail, e-commerce, streaming services, digital services etc.
  \item Grade: Primary 3
\end{itemize}

\textbf{Question}

A laundry shop charges \textdollar4 to wash one shirt. Amy brought 7 shirts to be washed. She also brought 3 trousers. Each trouser costs \textdollar6 to wash. How much did Amy pay in total for washing all her clothes?

\textbf{Solution}

Cost to wash 7 shirts $= 7 \times 4 = 28$ (\textdollar28) \\ 
Cost to wash 3 trousers $= 3 \times 6 = 18$ (\textdollar18) \\ 
Total amount paid $= 28 + 18 = 46$ \\ 
\boxed{\textdollar46}$ in total.

\section*{Question 10}
\textbf{Metadata}

\begin{itemize}
  \item Question ID: P3-WNMul3d1d\_P1-WNSub2nd\_GPT4.1\_Services\_04
  \item Primary KC: WHOLE NUMBERS | Multiplication | multiplying whole numbers up to 3 digits by 1 digit
  \item Secondary KC: WHOLE NUMBERS | Subtraction | subtracting whole numbers
  \item Topic: Services such as installation, maintenance, repairing, cleaning, laundry, hotel, retail, e-commerce, streaming services, digital services etc.
  \item Grade: Primary 3
\end{itemize}

\textbf{Question}

A laundry shop charges $\textdollar5$ to wash one shirt. Sarah brought $128$ shirts to the shop for washing. After collecting the washed shirts, she realised that $34$ shirts were not hers and returned them to the shop. What is the total amount that Sarah needs to pay for washing only her shirts?

\textbf{Solution}

First, find the number of shirts that belong to Sarah after returning the $34$ shirts.\newline Let $128 - 34 = 94$.\newline Sarah has $94$ shirts that need washing.\newline\newline Next, find the total amount Sarah needs to pay.\newline $94 \times 5 = 470$\newline\newline Therefore, Sarah needs to pay $\textdollar470$ for washing her shirts.

\section*{Question 11}
\textbf{Metadata}

\begin{itemize}
  \item Question ID: P3-WNDiv3d1d\_P1-WNCmp\_GPT4.1\_Services\_04
  \item Primary KC: WHOLE NUMBERS | Division | dividing whole numbers up to 3 digits by 1 digit
  \item Secondary KC: WHOLE NUMBERS | Comparison and ordering | comparing and ordering whole numbers
  \item Topic: Services such as installation, maintenance, repairing, cleaning, laundry, hotel, retail, e-commerce, streaming services, digital services etc.
  \item Grade: Primary 3
\end{itemize}

\textbf{Question}

A hotel has $216$ clean towels, and the staff wants to divide them equally among $4$ floors.

(a) How many towels will each floor get?

(b) If the fourth floor receives $60$ towels, does it have more or fewer towels than each of the other three floors? Arrange the number of towels on all four floors in order from least to greatest.

\textbf{Solution}

(a) To find out how many towels each floor gets, we divide $216$ by $4$:

\[
216 \div 4 = 54
\]

So, each floor should get $54$ towels.

(b) The fourth floor receives $60$ towels, which is different from the others. The other three floors each get $54$ towels, so we compare $54$ and $60$.

Since $60 > 54$, the fourth floor has more towels than each of the other three floors.

Arranging in order from least to greatest: $54, 54, 54, 60$.

\section*{Question 12}
\textbf{Metadata}

\begin{itemize}
  \item Question ID: P3-WNDiv3d1d\_P1-WNAdd2nd\_GPT4.1\_Services\_04
  \item Primary KC: WHOLE NUMBERS | Division | dividing whole numbers up to 3 digits by 1 digit
  \item Secondary KC: WHOLE NUMBERS | Addition | adding whole numbers
  \item Topic: Services such as installation, maintenance, repairing, cleaning, laundry, hotel, retail, e-commerce, streaming services, digital services etc.
  \item Grade: Primary 3
\end{itemize}

\textbf{Question}

A hotel laundry room washed $324$ towels over $6$ days. Each day, the washing machine washed the same number of towels. At the end of the week, the staff added $48$ more towels to be washed. 

How many towels did the laundry room wash in total over the week?

\textbf{Solution}

First, find out how many towels were washed each day:

$324 \div 6 = 54$

So, $54$ towels were washed each day. Over $6$ days:

$54 \times 6 = 324$

After $6$ days, $48$ more towels were added to be washed:

$324 + 48 = 372$

Therefore, the laundry room washed $372$ towels in total over the week.

\section*{Question 13}
\textbf{Metadata}

\begin{itemize}
  \item Question ID: P3-WNDiv3d1d\_P1-WNSub2nd\_GPT4.1\_Services\_04
  \item Primary KC: WHOLE NUMBERS | Division | dividing whole numbers up to 3 digits by 1 digit
  \item Secondary KC: WHOLE NUMBERS | Subtraction | subtracting whole numbers
  \item Topic: Services such as installation, maintenance, repairing, cleaning, laundry, hotel, retail, e-commerce, streaming services, digital services etc.
  \item Grade: Primary 3
\end{itemize}

\textbf{Question}

A laundry shop has $156$ clean towels that need to be distributed equally among $4$ storage shelves. 

After placing the towels equally on each shelf, $18$ towels are used by customers in one day. How many towels are left on the shelves after that?

\textbf{Solution}

First, divide $156$ towels equally among $4$ shelves: 

\[ 156 \div 4 = 39 \]

So, there are $39$ towels on each shelf. 

The total number of towels on all shelves immediately after the distribution is still $156$.

After $18$ towels are used by customers, subtract $18$ from $156$:

\[ 156 - 18 = 138 \]

\boxed{138}

There are $138$ towels left on the shelves after that.

\section*{Question 14}
\textbf{Metadata}

\begin{itemize}
  \item Question ID: P3-WNDiv3d1d\_P1-WNMul2nd\_GPT4.1\_Services\_04
  \item Primary KC: WHOLE NUMBERS | Division | dividing whole numbers up to 3 digits by 1 digit
  \item Secondary KC: WHOLE NUMBERS | Multiplication | multiplying whole numbers
  \item Topic: Services such as installation, maintenance, repairing, cleaning, laundry, hotel, retail, e-commerce, streaming services, digital services etc.
  \item Grade: Primary 3
\end{itemize}

\textbf{Question}

A laundry shop received 128 dirty shirts to wash in one day. The shop has 4 washing machines, and every machine can wash an equal number of shirts at a time. 

(a) How many shirts will each washing machine wash if the 128 shirts are divided equally among the 4 machines?

(b) If each machine takes 2 hours to wash all its shirts, how many hours in total do all 4 machines spend washing the shirts?

\textbf{Solution}

(a) To find out how many shirts each washing machine will wash, divide the total number of shirts by the number of machines:

$$128 \div 4 = 32$$

Each washing machine will wash $32$ shirts.

(b) Each washing machine spends $2$ hours washing $32$ shirts. There are $4$ machines, so the total number of hours spent by all machines is:

$$4 \times 2 = 8$$

All $4$ machines spend a total of $8$ hours washing the shirts.

\section*{Question 15}
\textbf{Metadata}

\begin{itemize}
  \item Question ID: P3-FrAddRl12\_P2-FrCmp\_GPT4.1\_Services\_04
  \item Primary KC: FRACTIONS | Addition | adding two related fractions within one whole with denominators of given fractions not exceeding 12
  \item Secondary KC: FRACTIONS | Comparison and ordering | comparing and ordering fractions
  \item Topic: Services such as installation, maintenance, repairing, cleaning, laundry, hotel, retail, e-commerce, streaming services, digital services etc.
  \item Grade: Primary 3
\end{itemize}

\textbf{Question}

The staff at a hotel spent $\frac{2}{6}$ of their time cleaning guest rooms and $\frac{3}{6}$ of their time maintaining the lobby yesterday. 

(a) What fraction of their total time did they spend on these two tasks altogether?

(b) Did the staff spend more, less or the same amount of time cleaning guest rooms as maintaining the lobby? Explain your answer.

\textbf{Solution}

(a) To find the total fraction of time spent on cleaning guest rooms and maintaining the lobby, we add the two fractions:

$\frac{2}{6} + \frac{3}{6} = \frac{2+3}{6} = \frac{5}{6}$

The staff spent $\frac{5}{6}$ of their time on these two tasks altogether.

(b) To compare $\frac{2}{6}$ and $\frac{3}{6}$:

$\frac{3}{6} > \frac{2}{6}$

This means the staff spent more time maintaining the lobby than cleaning guest rooms.

\section*{Question 16}
\textbf{Metadata}

\begin{itemize}
  \item Question ID: P3-FrAddRl12\_P3-FrSmp\_GPT4.1\_Services\_04
  \item Primary KC: FRACTIONS | Addition | adding two related fractions within one whole with denominators of given fractions not exceeding 12
  \item Secondary KC: FRACTIONS | Simplifying | expressing a fraction in its simplest form
  \item Topic: Services such as installation, maintenance, repairing, cleaning, laundry, hotel, retail, e-commerce, streaming services, digital services etc.
  \item Grade: Primary 3
\end{itemize}

\textbf{Question}

Aisha spent $\dfrac{4}{12}$ of an hour cleaning her room and $\dfrac{5}{12}$ of an hour helping her mother to fold laundry.

(a) What fraction of an hour did Aisha spend on both cleaning her room and folding laundry?

(b) Give your answer in its simplest form.

\textbf{Solution}

(a) First, add the fractions:

$\dfrac{4}{12} + \dfrac{5}{12} = \dfrac{9}{12}$

(b) Now, express $\dfrac{9}{12}$ in its simplest form. The highest common factor of $9$ and $12$ is $3$.

$\dfrac{9}{12} = \dfrac{9 \div 3}{12 \div 3} = \dfrac{3}{4}$

So, Aisha spent $\dfrac{3}{4}$ of an hour on both cleaning her room and folding laundry.

\section*{Question 17}
\textbf{Metadata}

\begin{itemize}
  \item Question ID: P3-FrSubRl12\_P2-FrCmp\_GPT4.1\_Services\_04
  \item Primary KC: FRACTIONS | Subtraction | subtracting two related fractions within one whole with denominators of given fractions not exceeding 12
  \item Secondary KC: FRACTIONS | Comparison and ordering | comparing and ordering fractions
  \item Topic: Services such as installation, maintenance, repairing, cleaning, laundry, hotel, retail, e-commerce, streaming services, digital services etc.
  \item Grade: Primary 3
\end{itemize}

\textbf{Question}

Olivia is cleaning her room. She spent $\frac{5}{8}$ of an hour tidying up her books and $\frac{3}{8}$ of an hour sweeping the floor. 

(a) How much longer did Olivia spend tidying her books than sweeping the floor?

(b) Arrange the following in order from the smallest to the largest: the time Olivia spent sweeping the floor, the time tidying her books, and half an hour ($\frac{1}{2}$ hour).

\textbf{Solution}

(a) To find how much longer Olivia spent tidying her books than sweeping the floor, subtract the two fractions:

Time difference $= \frac{5}{8} - \frac{3}{8} = \frac{2}{8} = \frac{1}{4}$ (hour)

Olivia spent $\frac{1}{4}$ hour longer tidying her books than sweeping the floor.

(b) Compare the fractions:
- Sweeping the floor: $\frac{3}{8}$
- Tidying books: $\frac{5}{8}$
- Half an hour: $\frac{1}{2}$

Let us compare $\frac{1}{2}$ and $\frac{3}{8}$:
$\frac{1}{2} = \frac{4}{8}$
So, $\frac{3}{8} < \frac{1}{2}$

Now compare $\frac{5}{8}$ to $\frac{1}{2}$:
$\frac{5}{8} > \frac{4}{8}$, so $\frac{5}{8} > \frac{1}{2}$

Arranging them in order from smallest to largest:
$\frac{3}{8}$ (sweeping the floor), $\frac{1}{2}$ (half an hour), $\frac{5}{8}$ (tidying books)

\section*{Question 18}
\textbf{Metadata}

\begin{itemize}
  \item Question ID: P3-FrSubRl12\_P2-FrAdd2nd\_GPT4.1\_Services\_04
  \item Primary KC: FRACTIONS | Subtraction | subtracting two related fractions within one whole with denominators of given fractions not exceeding 12
  \item Secondary KC: FRACTIONS | Addition | adding fractions
  \item Topic: Services such as installation, maintenance, repairing, cleaning, laundry, hotel, retail, e-commerce, streaming services, digital services etc.
  \item Grade: Primary 3
\end{itemize}

\textbf{Question}

A cleaning company uses a special disinfectant for cleaning homes. In the morning, the workers used $\frac{5}{8}$ of a bottle. In the afternoon, they used another $\frac{1}{4}$ of the same bottle. 

(a) What fraction of the bottle did they use altogether?

(b) After both cleaning sessions, what fraction of the bottle was left?

\textbf{Solution}

(a) Fraction of the bottle used altogether:

$\frac{5}{8} + \frac{1}{4}$

First, convert $\frac{1}{4}$ to an equivalent fraction with denominator $8$:

$\frac{1}{4} = \frac{2}{8}$

So, $\frac{5}{8} + \frac{2}{8} = \frac{7}{8}$

They used $\frac{7}{8}$ of the bottle in total.

(b) Fraction of the bottle left:

$1 - \frac{7}{8} = \frac{8}{8} - \frac{7}{8} = \frac{1}{8}$

There was $\frac{1}{8}$ of the bottle left after both sessions.

\section*{Question 19}
\textbf{Metadata}

\begin{itemize}
  \item Question ID: P3-FrSubRl12\_P3-FrSmp\_GPT4.1\_Services\_04
  \item Primary KC: FRACTIONS | Subtraction | subtracting two related fractions within one whole with denominators of given fractions not exceeding 12
  \item Secondary KC: FRACTIONS | Simplifying | expressing a fraction in its simplest form
  \item Topic: Services such as installation, maintenance, repairing, cleaning, laundry, hotel, retail, e-commerce, streaming services, digital services etc.
  \item Grade: Primary 3
\end{itemize}

\textbf{Question}

Aisha is helping her mother clean the windows in their house. She cleaned $\frac{7}{12}$ of the windows in the morning. After lunch, she realised that $\frac{5}{12}$ of the windows still needed to be cleaned.

What fraction of the windows did Aisha clean in the morning that is more than the windows left to clean after lunch? Give your answer in its simplest form.

\textbf{Solution}

Aisha cleaned $\frac{7}{12}$ of the windows.

There were $\frac{5}{12}$ of the windows left to clean after lunch.

To find out how much more she cleaned in the morning than what was left:

$\frac{7}{12} - \frac{5}{12} = \frac{7 - 5}{12} = \frac{2}{12}$

Now, simplify $\frac{2}{12}$ to its simplest form:

Divide numerator and denominator by 2:

$\frac{2 \div 2}{12 \div 2} = \frac{1}{6}$

So, Aisha cleaned $\frac{1}{6}$ more of the windows in the morning than the number left after lunch.

\section*{Question 20}
\textbf{Metadata}

\begin{itemize}
  \item Question ID: P4-WNMul4d1d\_P1-WNCmp\_GPT4.1\_Services\_04
  \item Primary KC: WHOLE NUMBERS | Multiplication | multiplying whole numbers up to 4 digits by 1 digit or up to 3 digits by 2 digits
  \item Secondary KC: WHOLE NUMBERS | Comparison and ordering | comparing and ordering whole numbers
  \item Topic: Services such as installation, maintenance, repairing, cleaning, laundry, hotel, retail, e-commerce, streaming services, digital services etc.
  \item Grade: Primary 4
\end{itemize}

\textbf{Question}

A laundry shop charges $\textdollar8$ to wash one bedsheet and $\textdollar12$ to wash one blanket. On Monday, they washed 125 bedsheets and 46 blankets for a hotel. \newline \newline \textbf{a)} How much did the laundry shop charge the hotel for washing all the bedsheets? \newline \textbf{b)} How much did the laundry shop charge for washing all the blankets? \newline \textbf{c)} Which service cost more, and by how much?

\textbf{Solution}

\textbf{a)}\newline \text{Total cost for washing bedsheets} = 125 \times \textdollar8 = \textdollar1000\newline\newline \textbf{b)}\newline \text{Total cost for washing blankets} = 46 \times \textdollar12 = \textdollar552\newline\newline \textbf{c)}\newline \text{Comparing the two amounts:}\newline \textdollar1000\ > \textdollar552,\ so washing the bedsheets cost more.\newline\text{The difference is } \textdollar1000 - \textdollar552 = \textdollar448.\newline\textbf{Therefore, washing the bedsheets cost \textdollar448 more than washing the blankets.}

\section*{Question 21}
\textbf{Metadata}

\begin{itemize}
  \item Question ID: P4-WNMul4d1d\_P1-WNAdd2nd\_GPT4.1\_Services\_04
  \item Primary KC: WHOLE NUMBERS | Multiplication | multiplying whole numbers up to 4 digits by 1 digit or up to 3 digits by 2 digits
  \item Secondary KC: WHOLE NUMBERS | Addition | adding whole numbers
  \item Topic: Services such as installation, maintenance, repairing, cleaning, laundry, hotel, retail, e-commerce, streaming services, digital services etc.
  \item Grade: Primary 4
\end{itemize}

\textbf{Question}

A cleaning company is hired to clean 125 hotel rooms. Each cleaning worker can clean 7 rooms in a day. The hotel manager wants to finish the cleaning in 2 days. 

(a) How many rooms can one worker clean in 2 days?

(b) How many workers are needed to clean all 125 rooms in 2 days?

(c) If the manager decides to hire 2 extra workers, how many rooms in total can all the workers clean in 2 days?

\textbf{Solution}

(a) Number of rooms one worker can clean in 2 days:

$7 \times 2 = 14$

One worker can clean 14 rooms in 2 days.

(b) Number of workers needed:

Total rooms to clean $= 125$
Rooms 1 worker can clean in 2 days $= 14$
So, number of workers needed $= \lceil 125 \div 14 \rceil = \lceil 8.93 \rceil = 9$

So, 9 workers are needed to clean all 125 rooms in 2 days.

(c) Number of rooms cleaned if 2 extra workers are hired:

Total workers $= 9 + 2 = 11$
Rooms each worker can clean in 2 days $= 14$
Total rooms $= 11 \times 14 = 154$

So, 11 workers can clean 154 rooms in 2 days.

\section*{Question 22}
\textbf{Metadata}

\begin{itemize}
  \item Question ID: P4-WNMul4d1d\_P1-WNSub2nd\_GPT4.1\_Services\_04
  \item Primary KC: WHOLE NUMBERS | Multiplication | multiplying whole numbers up to 4 digits by 1 digit or up to 3 digits by 2 digits
  \item Secondary KC: WHOLE NUMBERS | Subtraction | subtracting whole numbers
  \item Topic: Services such as installation, maintenance, repairing, cleaning, laundry, hotel, retail, e-commerce, streaming services, digital services etc.
  \item Grade: Primary 4
\end{itemize}

\textbf{Question}

A laundry shop charges $\textdollar15$ to wash one big blanket. Last weekend, Mrs Tan brought $48$ big blankets to the laundry shop to be washed for her hotel. After the washing, Mrs Tan realised that $7$ of the blankets were not cleaned properly and she asked for a refund for those $7$ blankets. How much did Mrs Tan pay in total after getting the refund?

\textbf{Solution}

First, we find the total cost to wash all $48$ blankets: 

$$48 \times 15 = 720$$

So, the initial amount Mrs Tan had to pay was $\textdollar720$.

Next, we calculate the refund for the $7$ blankets:

$$7 \times 15 = 105$$

So, Mrs Tan is refunded $\textdollar105$.

Now, subtract the refund from the initial payment to find out the total amount Mrs Tan paid:

$$720 - 105 = 615$$

Therefore, after getting the refund, Mrs Tan paid $\textdollar615$ in total.

\section*{Question 23}
\textbf{Metadata}

\begin{itemize}
  \item Question ID: P4-WNMul4d1d\_P4-WNRnd5d\_GPT4.1\_Services\_04
  \item Primary KC: WHOLE NUMBERS | Multiplication | multiplying whole numbers up to 4 digits by 1 digit or up to 3 digits by 2 digits
  \item Secondary KC: WHOLE NUMBERS | Rounding | rounding whole numbers up to 100000 to the nearest 10, 100 or 1000 
  \item Topic: Services such as installation, maintenance, repairing, cleaning, laundry, hotel, retail, e-commerce, streaming services, digital services etc.
  \item Grade: Primary 4
\end{itemize}

\textbf{Question}

A digital streaming service charges \textdollar368 per year for each subscription. A company wants to subscribe for 27 employees. 

(a) What is the total amount the company needs to pay for all 27 subscriptions in a year?

(b) Round your answer in part (a) to the nearest thousand.

\textbf{Solution}

(a) Total amount for 27 subscriptions:

$368 \times 27 = 9936$

The company needs to pay \textdollar9936 in total.

(b) Rounding \textdollar9936 to the nearest thousand:

The thousands digit is 9, and the number is 9936. The next digit (hundreds place) is 9, which is 5 or more. So, we round up.

Therefore, \textdollar9936 rounded to the nearest thousand is \textdollar10000.

\section*{Question 24}
\textbf{Metadata}

\begin{itemize}
  \item Question ID: P4-WNDiv4d1d\_P1-WNCmp\_GPT4.1\_Services\_04
  \item Primary KC: WHOLE NUMBERS | Division | dividing whole numbers up to 4 digits by 1 digit
  \item Secondary KC: WHOLE NUMBERS | Comparison and ordering | comparing and ordering whole numbers
  \item Topic: Services such as installation, maintenance, repairing, cleaning, laundry, hotel, retail, e-commerce, streaming services, digital services etc.
  \item Grade: Primary 4
\end{itemize}

\textbf{Question}

A cleaning company receives a contract to clean $2,496$ hotel rooms in a month. The company has $8$ cleaning teams. Each team is assigned an equal number of rooms to clean.\
\
(a) How many rooms does each team need to clean?\
\
(b) If Team A cleans $320$ rooms, Team B cleans $325$ rooms, Team C cleans $310$ rooms and Team D cleans the number of rooms you found in part (a), arrange the number of rooms cleaned by these four teams in order, from the least to the greatest.

\textbf{Solution}

(a) To find the number of rooms each team needs to clean, we divide the total number of rooms by the number of teams.\
\[
\frac{2,496}{8} = 312
\]\
So, each team needs to clean $312$ rooms.\
\
(b) The number of rooms cleaned are:\
Team A: $320$\
Team B: $325$\
Team C: $310$\
Team D: $312$\
\
Arrange these numbers in order from the least to the greatest:\
$310$, $312$, $320$, $325$\
\
So, the order is: $310 < 312 < 320 < 325$.

\section*{Question 25}
\textbf{Metadata}

\begin{itemize}
  \item Question ID: P4-WNDiv4d1d\_P1-WNAdd2nd\_GPT4.1\_Services\_04
  \item Primary KC: WHOLE NUMBERS | Division | dividing whole numbers up to 4 digits by 1 digit
  \item Secondary KC: WHOLE NUMBERS | Addition | adding whole numbers
  \item Topic: Services such as installation, maintenance, repairing, cleaning, laundry, hotel, retail, e-commerce, streaming services, digital services etc.
  \item Grade: Primary 4
\end{itemize}

\textbf{Question}

A cleaning company received a contract to clean 2,268 office chairs over several days. The manager plans to divide the work equally among 6 workers. After all the workers have finished their share, the manager also adds 35 more chairs that need to be cleaned. How many chairs will each worker clean after the extra chairs are included and the work is divided equally among the same 6 workers?

\textbf{Solution}

First, divide the initial number of chairs by the number of workers: 

\[
\frac{2,268}{6} = 378
\]

So, each worker will clean 378 chairs at first. 

Next, add the extra chairs to the total:

\[
2,268 + 35 = 2,303
\]

Now, divide the new total by 6 workers:

\[
\frac{2,303}{6} = 383\text{ R }5
\]

So, each worker will clean 383 chairs, and 5 chairs will be left. If the remaining 5 chairs are distributed, some workers may have to clean 1 more chair. But if only whole numbers are considered, each worker will clean 383 chairs after the extra chairs are included.

\section*{Question 26}
\textbf{Metadata}

\begin{itemize}
  \item Question ID: P4-WNDiv4d1d\_P1-WNSub2nd\_GPT4.1\_Services\_04
  \item Primary KC: WHOLE NUMBERS | Division | dividing whole numbers up to 4 digits by 1 digit
  \item Secondary KC: WHOLE NUMBERS | Subtraction | subtracting whole numbers
  \item Topic: Services such as installation, maintenance, repairing, cleaning, laundry, hotel, retail, e-commerce, streaming services, digital services etc.
  \item Grade: Primary 4
\end{itemize}

\textbf{Question}

Alvin is a manager at a laundry service. Today, he received 1,248 shirts that need to be cleaned. He has 6 washing machines. He wants to divide the shirts equally among the 6 washing machines. After washing, Alvin found that 32 shirts were still dirty and needed to be washed again. 

How many shirts did Alvin successfully clean the first time?

\textbf{Solution}

Step 1: Divide the total number of shirts by the number of washing machines: 

$1,248 \div 6 = 208$

So, each machine washes 208 shirts.

Step 2: Calculate the total number of shirts washed in one cycle:

$208 \times 6 = 1,248$

Step 3: Subtract the number of dirty shirts that need to be washed again:

$1,248 - 32 = 1,216$

Therefore, Alvin successfully cleaned $1,216$ shirts the first time.

\section*{Question 27}
\textbf{Metadata}

\begin{itemize}
  \item Question ID: P4-WNDiv4d1d\_P1-WNMul2nd\_GPT4.1\_Services\_04
  \item Primary KC: WHOLE NUMBERS | Division | dividing whole numbers up to 4 digits by 1 digit
  \item Secondary KC: WHOLE NUMBERS | Multiplication | multiplying whole numbers
  \item Topic: Services such as installation, maintenance, repairing, cleaning, laundry, hotel, retail, e-commerce, streaming services, digital services etc.
  \item Grade: Primary 4
\end{itemize}

\textbf{Question}

A cleaning company has \(2\,448\) chairs to clean in a large hotel. The company sends \(8\) workers, and each worker is assigned to clean an equal number of chairs.

(a) How many chairs does each worker have to clean?

(b) If each worker manages to clean \(12\) chairs in one hour, how many hours will it take for one worker to finish cleaning all of their assigned chairs?

\textbf{Solution}

(a) To find the number of chairs each worker has to clean, divide \(2\,448\) by \(8\):

\[
2\,448 \div 8 = 306
\]

Each worker has to clean \(306\) chairs.

(b) To find the number of hours needed for one worker to finish, divide the total chairs by the number of chairs cleaned per hour:

\[
306 \div 12 = 25.5
\]

It will take one worker \(25.5\) hours to finish cleaning all of their assigned chairs.

\section*{Question 28}
\textbf{Metadata}

\begin{itemize}
  \item Question ID: P4-WNDiv4d1d\_P4-WNRnd5d\_GPT4.1\_Services\_04
  \item Primary KC: WHOLE NUMBERS | Division | dividing whole numbers up to 4 digits by 1 digit
  \item Secondary KC: WHOLE NUMBERS | Rounding | rounding whole numbers up to 100000 to the nearest 10, 100 or 1000 
  \item Topic: Services such as installation, maintenance, repairing, cleaning, laundry, hotel, retail, e-commerce, streaming services, digital services etc.
  \item Grade: Primary 4
\end{itemize}

\textbf{Question}

A cleaning company was hired to clean 4,732 chairs in a large convention hall. They decided to divide the work equally among 8 workers. After dividing, the manager wants to know approximately how many chairs each worker has to clean. Round the answer to the nearest ten.

\textbf{Solution}

First, we divide the total number of chairs by the number of workers:  

$4,732 \div 8 = 591.5$  

Since the answer must be rounded to the nearest ten, we look at the digit in the ones place (which is 1 in 591.5). Since it is less than 5, we round down:

$591.5 \approx 590$

So, each worker has to clean about $590$ chairs (rounded to the nearest ten).

\section*{Question 29}
\textbf{Metadata}

\begin{itemize}
  \item Question ID: P4-FrRepMixIm\_P3-FrCnvEq\_GPT4.1\_Services\_04
  \item Primary KC: FRACTIONS | Representation and concept | mixed numbers and improper fractions
  \item Secondary KC: FRACTIONS | Conversion to equivalent fractions | Conversion to equivalent fractions (given either the denominator or the numerator)
  \item Topic: Services such as installation, maintenance, repairing, cleaning, laundry, hotel, retail, e-commerce, streaming services, digital services etc.
  \item Grade: Primary 4
\end{itemize}

\textbf{Question}

Mr. Tan cleaned hotel rooms for  $\dfrac{7}{4}$  hours on Monday. On Tuesday, he worked for $1\dfrac{1}{2}$ hours cleaning rooms. If he needs to report his total working hours as equivalent fractions with denominator $4$, how many total hours did Mr. Tan spend cleaning rooms over both days? Express your answer as a single improper fraction.

\textbf{Solution}

First, change the mixed number $1\dfrac{1}{2}$ to an improper fraction: 

$1\dfrac{1}{2} = \dfrac{2 \times 1 + 1}{2} = \dfrac{3}{2}$.

Next, we need to convert $\dfrac{3}{2}$ to an equivalent fraction with denominator 4:

$\dfrac{3}{2} = \dfrac{3 \times 2}{2 \times 2} = \dfrac{6}{4}$.

Mr. Tan cleaned $\dfrac{7}{4}$ hours on Monday and $\dfrac{6}{4}$ hours on Tuesday.

Total hours spent cleaning:

$\dfrac{7}{4} + \dfrac{6}{4} = \dfrac{13}{4}$.

He spent a total of $\dfrac{13}{4}$ hours cleaning rooms over both days, reported as an improper fraction with denominator $4$. 

\textbf{Final Answer:} $\dfrac{13}{4}$ hours.

\section*{Question 30}
\textbf{Metadata}

\begin{itemize}
  \item Question ID: P4-FrRepSet\_P3-FrCnvEq\_GPT4.1\_Services\_04
  \item Primary KC: FRACTIONS | Representation and concept | expressing a part of a set as a fraction
  \item Secondary KC: FRACTIONS | Conversion to equivalent fractions | Conversion to equivalent fractions (given either the denominator or the numerator)
  \item Topic: Services such as installation, maintenance, repairing, cleaning, laundry, hotel, retail, e-commerce, streaming services, digital services etc.
  \item Grade: Primary 4
\end{itemize}

\textbf{Question}

A laundry shop cleaned a bag of 24 shirts belonging to a hotel. The shop set aside 6 shirts that needed extra stain removal. 

(a) What fraction of the set of shirts needed extra stain removal? Express your answer in simplest form. 

(b) Write an equivalent fraction to your answer in (a) with a denominator of 12.

\textbf{Solution}

(a) The fraction of shirts that needed extra stain removal is $\dfrac{6}{24}$. 

We simplify $\dfrac{6}{24}$ by dividing both numerator and denominator by 6:

$\dfrac{6 \div 6}{24 \div 6} = \dfrac{1}{4}$

So, $\dfrac{1}{4}$ of the shirts needed extra stain removal.

(b) To write $\dfrac{1}{4}$ with a denominator of 12, we find a fraction that is equivalent to $\dfrac{1}{4}$ but has 12 as the denominator.

$\dfrac{1}{4} = \dfrac{1 \times 3}{4 \times 3} = \dfrac{3}{12}$

So, an equivalent fraction with a denominator of 12 is $\dfrac{3}{12}$.

\section*{Question 31}
\textbf{Metadata}

\begin{itemize}
  \item Question ID: P4-FrAddU12\_P2-FrCmp\_GPT4.1\_Services\_04
  \item Primary KC: FRACTIONS | Addition | adding unlike fractions with two different denominators not exceeding 12
  \item Secondary KC: FRACTIONS | Comparison and ordering | comparing and ordering fractions
  \item Topic: Services such as installation, maintenance, repairing, cleaning, laundry, hotel, retail, e-commerce, streaming services, digital services etc.
  \item Grade: Primary 4
\end{itemize}

\textbf{Question}

Mr. Lee's cleaning service cleaned two offices in the morning. For the first office, he finished \( \frac{2}{3} \) of the job before his break. For the second office, he completed \( \frac{3}{4} \) of the job before his break as well. 

(a) How much work did Mr. Lee complete in total before his break?

(b) Did Mr. Lee finish more than 1 whole office before his break? Show your working.

\textbf{Solution}

(a) First, add the fractions:

\[
\frac{2}{3} + \frac{3}{4}
\]

The least common denominator of 3 and 4 is 12.
\[
\frac{2}{3} = \frac{2 \times 4}{3 \times 4} = \frac{8}{12}
\]
\[
\frac{3}{4} = \frac{3 \times 3}{4 \times 3} = \frac{9}{12}
\]
\[
\frac{8}{12} + \frac{9}{12} = \frac{17}{12}
\]

So, Mr. Lee finished \( \frac{17}{12} \) of an office's work in total before his break.

(b) Now, compare \( \frac{17}{12} \) with 1:

\[
\frac{17}{12} > 1
\]

because \( 17 > 12 \).

Therefore, Mr. Lee finished more than 1 whole office before his break.

\section*{Question 32}
\textbf{Metadata}

\begin{itemize}
  \item Question ID: P4-FrAddU12\_P3-FrSmp\_GPT4.1\_Services\_04
  \item Primary KC: FRACTIONS | Addition | adding unlike fractions with two different denominators not exceeding 12
  \item Secondary KC: FRACTIONS | Simplifying | expressing a fraction in its simplest form
  \item Topic: Services such as installation, maintenance, repairing, cleaning, laundry, hotel, retail, e-commerce, streaming services, digital services etc.
  \item Grade: Primary 4
\end{itemize}

\textbf{Question}

Ginny works at a laundry service. She spent $\frac{3}{8}$ of her workday washing clothes and $\frac{5}{12}$ of her workday ironing clothes. What fraction of her workday did Ginny spend on these two tasks altogether? Express your answer in its simplest form.

\textbf{Solution}

To find the total fraction of the day Ginny spent washing and ironing, we need to add $\frac{3}{8}$ and $\frac{5}{12}$. 

First, find the lowest common multiple (LCM) of 8 and 12, which is 24.

\[
\frac{3}{8} = \frac{3 \times 3}{8 \times 3} = \frac{9}{24} \\
\frac{5}{12} = \frac{5 \times 2}{12 \times 2} = \frac{10}{24}
\]

Now add the fractions:
\[
\frac{9}{24} + \frac{10}{24} = \frac{19}{24}
\]

The fraction $\frac{19}{24}$ cannot be simplified further as 19 and 24 have no common factors other than 1.

**Answer:** Ginny spent $\frac{19}{24}$ of her workday washing and ironing, expressed in its simplest form.

\section*{Question 33}
\textbf{Metadata}

\begin{itemize}
  \item Question ID: P4-FrSubU12\_P2-FrCmp\_GPT4.1\_Services\_04
  \item Primary KC: FRACTIONS | Subtraction | subtracting unlike fractions with two different denominators not exceeding 12
  \item Secondary KC: FRACTIONS | Comparison and ordering | comparing and ordering fractions
  \item Topic: Services such as installation, maintenance, repairing, cleaning, laundry, hotel, retail, e-commerce, streaming services, digital services etc.
  \item Grade: Primary 4
\end{itemize}

\textbf{Question}

A laundry service offers different packages for cleaning bed sheets. Package A cleans $\dfrac{7}{12}$ of a basket of sheets in one hour, and Package B cleans $\dfrac{5}{8}$ of a basket in one hour. \newline \newline (a) How much more of a basket does Package A clean compared to Package B in one hour? \newline (b) Arrange the fractions $\dfrac{7}{12}$, $\dfrac{5}{8}$, and the answer from (a) in order from smallest to largest.

\textbf{Solution}

(a) To find how much more Package A cleans compared to Package B, subtract the fractions: \[ \dfrac{7}{12} - \dfrac{5}{8} \] \newline Find a common denominator for 12 and 8, which is 24. \newline \[ \dfrac{7}{12} = \dfrac{7 \times 2}{12 \times 2} = \dfrac{14}{24} \] \[ \dfrac{5}{8} = \dfrac{5 \times 3}{8 \times 3} = \dfrac{15}{24} \] \[ \dfrac{14}{24} - \dfrac{15}{24} = -\dfrac{1}{24} \] \newline Since the answer is negative, it means Package B actually cleans more. \newline So, Package B cleans $\dfrac{1}{24}$ of a basket more than Package A. \newline \newline (b) Arrange $\dfrac{7}{12}$, $\dfrac{5}{8}$, and $-\dfrac{1}{24}$ from smallest to largest: \newline Express all in twenty-fourths: \[ \dfrac{7}{12} = \dfrac{14}{24} \] \[ \dfrac{5}{8} = \dfrac{15}{24} \] \[ -\dfrac{1}{24} \, (\text{already in twenty-fourths}) \] \newline Therefore, in order from smallest to largest: \[ -\dfrac{1}{24}, \dfrac{14}{24} \, (\dfrac{7}{12}), \dfrac{15}{24} \, (\dfrac{5}{8}) \]

\section*{Question 34}
\textbf{Metadata}

\begin{itemize}
  \item Question ID: P4-FrSubU12\_P2-FrAdd2nd\_GPT4.1\_Services\_04
  \item Primary KC: FRACTIONS | Subtraction | subtracting unlike fractions with two different denominators not exceeding 12
  \item Secondary KC: FRACTIONS | Addition | adding fractions
  \item Topic: Services such as installation, maintenance, repairing, cleaning, laundry, hotel, retail, e-commerce, streaming services, digital services etc.
  \item Grade: Primary 4
\end{itemize}

\textbf{Question}

A cleaning company offers two types of cleaning packages. Package A allows customers to use $\frac{5}{6}$ hour per cleaning session. Package B allows customers to use $\frac{2}{3}$ hour per session. Last week, Mrs. Lee bought 1 session of Package A and 2 sessions of Package B. 

(a) What is the total number of hours Mrs. Lee can use for cleaning if she combines all her sessions?

(b) If she used $\frac{7}{4}$ hours in total, how many hours does she have left?

\textbf{Solution}

(a) 1 session of Package A is $\frac{5}{6}$ hour. 2 sessions of Package B: $2 \times \frac{2}{3} = \frac{4}{3}$ hours.
Now, add $\frac{5}{6}$ and $\frac{4}{3}$. To do this, express both with denominator 6:
$\frac{4}{3} = \frac{8}{6}$
$\frac{5}{6} + \frac{8}{6} = \frac{13}{6}$
So, Mrs. Lee can use $\frac{13}{6}$ hours in total.

(b) She used $\frac{7}{4}$ hours. Find how many hours she has left:
$\frac{13}{6} - \frac{7}{4}$
Change to common denominator (12):
$\frac{13}{6} = \frac{26}{12}$
$\frac{7}{4} = \frac{21}{12}$
Subtract:
$\frac{26}{12} - \frac{21}{12} = \frac{5}{12}$
Thus, Mrs. Lee has $\frac{5}{12}$ hour left.

\section*{Question 35}
\textbf{Metadata}

\begin{itemize}
  \item Question ID: P4-FrSubU12\_P3-FrSmp\_GPT4.1\_Services\_04
  \item Primary KC: FRACTIONS | Subtraction | subtracting unlike fractions with two different denominators not exceeding 12
  \item Secondary KC: FRACTIONS | Simplifying | expressing a fraction in its simplest form
  \item Topic: Services such as installation, maintenance, repairing, cleaning, laundry, hotel, retail, e-commerce, streaming services, digital services etc.
  \item Grade: Primary 4
\end{itemize}

\textbf{Question}

Sarah spent $\frac{5}{8}$ of an hour cleaning her room and $\frac{1}{3}$ of an hour doing laundry. How much more time did Sarah spend cleaning her room than doing laundry? Express your answer as a fraction in its simplest form.

\textbf{Solution}

To find out how much more time Sarah spent cleaning her room than doing laundry, subtract the time spent on laundry from the time spent cleaning her room:

\[
\frac{5}{8} - \frac{1}{3}
\]

We need to get a common denominator. The least common denominator for 8 and 3 is 24:

\[
\frac{5}{8} = \frac{5 \times 3}{8 \times 3} = \frac{15}{24}
\]

\[
\frac{1}{3} = \frac{1 \times 8}{3 \times 8} = \frac{8}{24}
\]

Now subtract:

\[
\frac{15}{24} - \frac{8}{24} = \frac{7}{24}
\]

So, Sarah spent $\frac{7}{24}$ of an hour more cleaning her room than doing laundry.

\section*{Question 36}
\textbf{Metadata}

\begin{itemize}
  \item Question ID: P4-DcAdd2d\_P4-DcCmp3d\_GPT4.1\_Services\_04
  \item Primary KC: DECIMALS | Addition | adding decimals (up to 2 decimal places)
  \item Secondary KC: DECIMALS | Comparison and ordering | comparing and ordering decimals up to 3 decimal places
  \item Topic: Services such as installation, maintenance, repairing, cleaning, laundry, hotel, retail, e-commerce, streaming services, digital services etc.
  \item Grade: Primary 4
\end{itemize}

\textbf{Question}

Sarah is comparing the prices of three different online movie streaming services. Service A charges $\textdollar8.99$ per month, Service B charges $\textdollar9.25$ per month, and Service C charges $\textdollar7.995$ per month.

(a) Arrange the prices of the three services in order from the lowest to the highest.

(b) If Sarah subscribes to Service B and Service C for one month each, how much does she need to pay in total?

\textbf{Solution}

(a) To compare the prices, we align the decimal places to 3 decimal places:

\begin{align*}
\text{Service A:} & \ \textdollar8.990 \\
\text{Service B:} & \ \textdollar9.250 \\
\text{Service C:} & \ \textdollar7.995
\end{align*}

Arranging the prices from lowest to highest:
Service C ($\textdollar7.995$), Service A ($\textdollar8.99$), Service B ($\textdollar9.25$).

(b) Add the prices of Service B and Service C:
$9.25 + 7.995 = 17.245$

Round to 2 decimal places: $17.25$

Sarah needs to pay $\textdollar17.25$ in total.

\section*{Question 37}
\textbf{Metadata}

\begin{itemize}
  \item Question ID: P4-DcAdd2d\_P4-DcCnv2Fr\_GPT4.1\_Services\_04
  \item Primary KC: DECIMALS | Addition | adding decimals (up to 2 decimal places)
  \item Secondary KC: DECIMALS | Conversion from decimals to fraction | expressing decimals as fractions
  \item Topic: Services such as installation, maintenance, repairing, cleaning, laundry, hotel, retail, e-commerce, streaming services, digital services etc.
  \item Grade: Primary 4
\end{itemize}

\textbf{Question}

Rachel hired a carpet cleaning service for her living room and bedroom. The cost to clean the living room was $38.75 and the cost to clean the bedroom was $24.60. 

(a) What was the total cost Rachel paid for cleaning both rooms? 

(b) Express the cost to clean the bedroom, $24.60, as a fraction in its simplest form.

\textbf{Solution}

(a) To find the total cost, add the amounts paid for each room:

$38.75 + 24.60 = 63.35$

Therefore, Rachel paid a total of $\textdollar63.35$ to clean both rooms.

(b) The cost to clean the bedroom is $24.60$. To express $24.60$ as a fraction:

$24.60 = 24 + 0.60$

$0.60$ as a fraction is $\frac{60}{100}$, which simplifies to $\frac{3}{5}$.

So,

$24.60 = 24\dfrac{3}{5}$

Therefore, $24.60$ as a fraction is $24\dfrac{3}{5}$. Alternatively, as an improper fraction:

$24.60 = \dfrac{2460}{100} = \dfrac{123}{5}$

So, $24.60 = \dfrac{123}{5}$.

\section*{Question 38}
\textbf{Metadata}

\begin{itemize}
  \item Question ID: P4-DcAdd2d\_P4-DcRnd3d\_GPT4.1\_Services\_04
  \item Primary KC: DECIMALS | Addition | adding decimals (up to 2 decimal places)
  \item Secondary KC: DECIMALS | Rounding | rounding decimals up to 3 decimal places to the nearest whole number, 1 decimal place and 2 decimal places 
  \item Topic: Services such as installation, maintenance, repairing, cleaning, laundry, hotel, retail, e-commerce, streaming services, digital services etc.
  \item Grade: Primary 4
\end{itemize}

\textbf{Question}

Alicia uses a cleaning service that charges $\textdollar12.55$ for cleaning the living room, $\textdollar9.80$ for cleaning the kitchen, and $\textdollar7.275$ for cleaning the bathroom. After the cleaners finish their work, Alicia wants to know:

(a) What is the total amount she needs to pay before rounding?

(b) Round the total amount to the nearest whole dollar.

(c) Round the total amount to the nearest $1$ decimal place.

(d) Round the total amount to the nearest $2$ decimal places.

\textbf{Solution}

(a) First, add up all the charges:

$\textdollar12.55 + \textdollar9.80 + \textdollar7.275 = \textdollar29.625$

So, the total amount Alicia needs to pay before rounding is $\textdollar29.625$.

(b) Rounded to the nearest whole dollar:

$29.625 \approx 30$

So, Alicia should pay $\textdollar30$ after rounding to the nearest whole dollar.

(c) Rounded to $1$ decimal place:

$29.625 \approx 29.6$

So, Alicia should pay $\textdollar29.6$ after rounding to the nearest $1$ decimal place.

(d) Rounded to $2$ decimal places:

$29.625 \approx 29.63$

So, Alicia should pay $\textdollar29.63$ after rounding to the nearest $2$ decimal places.

\section*{Question 39}
\textbf{Metadata}

\begin{itemize}
  \item Question ID: P4-DcSub2d\_P4-DcCmp3d\_GPT4.1\_Services\_04
  \item Primary KC: DECIMALS | Subtraction | subtracting decimals (up to 2 decimal places)
  \item Secondary KC: DECIMALS | Comparison and ordering | comparing and ordering decimals up to 3 decimal places
  \item Topic: Services such as installation, maintenance, repairing, cleaning, laundry, hotel, retail, e-commerce, streaming services, digital services etc.
  \item Grade: Primary 4
\end{itemize}

\textbf{Question}

A cleaning company charges \textdollar15.80 for a basic room cleaning and \textdollar15.295 for a curtain cleaning service. Mei Ling paid for both services. 

(a) Which service is more expensive, and by how much?

(b) If Mei Ling only paid for the more expensive service, how much money would she save compared to paying for both services?

\textbf{Solution}

(a) Compare the charges:

Room cleaning: \textdollar15.80
Curtain cleaning: \textdollar15.295

Comparing the two decimals (to 3 decimal places):
$15.800 \gt 15.295$

So, the room cleaning service is more expensive.

Amount difference:
\[
15.80 - 15.295 = 0.505
\]
So, the room cleaning is more expensive by \textdollar0.505.

(b) If Mei Ling only paid for the more expensive service, she would save the cost of the less expensive service:
\[
\text{Total paid for both} = 15.80 + 15.295 = 31.095 \\
\text{If only paid for room cleaning: } 31.095 - 15.80 = 15.295
\]
Alternatively, she would save the amount of the curtain cleaning service, which is \textdollar15.295.

\section*{Question 40}
\textbf{Metadata}

\begin{itemize}
  \item Question ID: P4-DcSub2d\_P4-DcCnv2Fr\_GPT4.1\_Services\_04
  \item Primary KC: DECIMALS | Subtraction | subtracting decimals (up to 2 decimal places)
  \item Secondary KC: DECIMALS | Conversion from decimals to fraction | expressing decimals as fractions
  \item Topic: Services such as installation, maintenance, repairing, cleaning, laundry, hotel, retail, e-commerce, streaming services, digital services etc.
  \item Grade: Primary 4
\end{itemize}

\textbf{Question}

Jane subscribes to a music streaming service for $\textdollar12.80$ a month. She found another similar service that only costs $\textdollar10.35$ a month. 

(a) How much less would Jane pay each month if she switches to the cheaper service? 

(b) Express the difference in cost from part (a) as a fraction in its simplest form.

\textbf{Solution}

(a) To find how much less Jane would pay each month, subtract $\textdollar10.35$ from $\textdollar12.80$:

\[
\textdollar12.80 - \textdollar10.35 = \textdollar2.45
\]

Jane would pay $\textdollar2.45$ less each month.

(b) Now, express $2.45$ as a fraction.

\[
2.45 = \frac{245}{100} = \frac{49}{20}
\]

So, the difference in cost is $\frac{49}{20}$ dollars.

\section*{Question 41}
\textbf{Metadata}

\begin{itemize}
  \item Question ID: P4-DcSub2d\_P4-DcRnd3d\_GPT4.1\_Services\_04
  \item Primary KC: DECIMALS | Subtraction | subtracting decimals (up to 2 decimal places)
  \item Secondary KC: DECIMALS | Rounding | rounding decimals up to 3 decimal places to the nearest whole number, 1 decimal place and 2 decimal places 
  \item Topic: Services such as installation, maintenance, repairing, cleaning, laundry, hotel, retail, e-commerce, streaming services, digital services etc.
  \item Grade: Primary 4
\end{itemize}

\textbf{Question}

Alice purchased an online movie streaming service for \textdollar42.80. A week later, she found a promotional offer that allowed her to get the same service for only \textdollar37.459. 

(a) How much more did Alice pay compared to the promotional price? 

(b) Round the promotional offer price to the nearest dollar, the nearest tenth (1 decimal place), and the nearest hundredth (2 decimal places).

\textbf{Solution}

(a) Amount Alice paid extra:

$42.80 - 37.459 = 5.341$

So, Alice paid \textdollar5.341 more than the promotional price.

(b) Rounding the promotional price $37.459$:
- To the nearest whole number: $37$
- To 1 decimal place: $37.5$
- To 2 decimal places: $37.46$

Final answers:
(a) \textdollar5.341
(b) Nearest dollar: \textdollar37; nearest 1 decimal place: \textdollar37.5; nearest 2 decimal places: \textdollar37.46

\section*{Question 42}
\textbf{Metadata}

\begin{itemize}
  \item Question ID: P4-DcSub2d\_P4-DcAdd2nd\_GPT4.1\_Services\_04
  \item Primary KC: DECIMALS | Subtraction | subtracting decimals (up to 2 decimal places)
  \item Secondary KC: DECIMALS | Addition | adding decimals
  \item Topic: Services such as installation, maintenance, repairing, cleaning, laundry, hotel, retail, e-commerce, streaming services, digital services etc.
  \item Grade: Primary 4
\end{itemize}

\textbf{Question}

Jenny hired a company to clean her house and wash her curtains. The company charged Jenny \textdollar55.80 for cleaning her house and \textdollar18.75 for washing her curtains. Jenny paid \textdollar80.00 in total. How much change should Jenny receive after paying for both services?

\textbf{Solution}

First, find the total amount Jenny needs to pay for both services:

\[
\textdollar55.80 + \textdollar18.75 = \textdollar74.55
\]

Next, subtract the amount to be paid from the amount Jenny gave to find her change:

\[
\textdollar80.00 - \textdollar74.55 = \textdollar5.45
\]

Therefore, Jenny should receive \textdollar5.45 in change.

\section*{Question 43}
\textbf{Metadata}

\begin{itemize}
  \item Question ID: P4-DcMul2d1d\_P4-DcCmp3d\_GPT4.1\_Services\_03
  \item Primary KC: DECIMALS | Multiplication | multiplying decimals (up to 2 decimal places) by a 1-digit whole number
  \item Secondary KC: DECIMALS | Comparison and ordering | comparing and ordering decimals up to 3 decimal places
  \item Topic: Services such as installation, maintenance, repairing, cleaning, laundry, hotel, retail, e-commerce, streaming services, digital services etc.
  \item Grade: Primary 4
\end{itemize}

\textbf{Question}

A laundromat charges $\textdollar 2.35$ for every kilogram of clothes washed. If Sarah brings a bag of clothes weighing $4$ kg, how much does she need to pay in total?

Later, her friend Julia also visits the laundromat with a bag weighing $3$ kg. Compare the total amount paid by Sarah and Julia. Who paid more and by how much?

Arrange the total amounts paid by Sarah and Julia in ascending order.

\textbf{Solution}

The amount Sarah needs to pay is:

$2.35 \times 4 = 9.40$

So, Sarah pays $\textdollar 9.40$.

The amount Julia needs to pay is:

$2.35 \times 3 = 7.05$

So, Julia pays $\textdollar 7.05$.

Comparing $9.40$ and $7.05$, Sarah paid more than Julia.

The difference is:

$9.40 - 7.05 = 2.35$

So Sarah paid $\textdollar 2.35$ more than Julia.

Arranging the total amounts paid in ascending order:

$7.05,\ 9.40$

Therefore, Julia paid less than Sarah.

\section*{Question 44}
\textbf{Metadata}

\begin{itemize}
  \item Question ID: P4-DcMul2d1d\_P4-DcCnv2Fr\_GPT4.1\_Services\_03
  \item Primary KC: DECIMALS | Multiplication | multiplying decimals (up to 2 decimal places) by a 1-digit whole number
  \item Secondary KC: DECIMALS | Conversion from decimals to fraction | expressing decimals as fractions
  \item Topic: Services such as installation, maintenance, repairing, cleaning, laundry, hotel, retail, e-commerce, streaming services, digital services etc.
  \item Grade: Primary 4
\end{itemize}

\textbf{Question}

A laundry service charges $\textdollar2.75$ to wash one shirt. Sarah wants to wash 6 shirts. 

(a) How much will Sarah need to pay in total to wash all 6 shirts? 

(b) Express the price to wash one shirt ($2.75) as a fraction in its simplest form.

\textbf{Solution}

(a) Cost to wash one shirt: $\textdollar2.75$

Number of shirts Sarah wants to wash: $6$

Total cost $= 2.75 \times 6 = 16.50$

So, Sarah will need to pay $\textdollar16.50$ in total.

(b) To express $2.75$ as a fraction:

$2.75 = 2$ whole and $0.75$ as a decimal.

$0.75 = \frac{75}{100} = \frac{3}{4}$ (in simplest form)

Therefore, $2.75 = 2\frac{3}{4} = \frac{11}{4}$.

So, $2.75$ as a fraction in its simplest form is $\frac{11}{4}$.

\section*{Question 45}
\textbf{Metadata}

\begin{itemize}
  \item Question ID: P4-DcMul2d1d\_P4-DcRnd3d\_GPT4.1\_Services\_03
  \item Primary KC: DECIMALS | Multiplication | multiplying decimals (up to 2 decimal places) by a 1-digit whole number
  \item Secondary KC: DECIMALS | Rounding | rounding decimals up to 3 decimal places to the nearest whole number, 1 decimal place and 2 decimal places 
  \item Topic: Services such as installation, maintenance, repairing, cleaning, laundry, hotel, retail, e-commerce, streaming services, digital services etc.
  \item Grade: Primary 4
\end{itemize}

\textbf{Question}

A laundry service charges $\textdollar 3.75$ to wash and dry one shirt. Oliver brought 5 shirts to the laundry. 

(a) How much did he have to pay for washing and drying all his shirts? 

(b) The laundry service records the total amount Oliver paid as a decimal number. Round this amount to the nearest whole number, then to 1 decimal place, and finally to 2 decimal places.

\textbf{Solution}

(a) The cost to wash and dry one shirt is $\textdollar 3.75$. Oliver brought 5 shirts, so the total cost is:

$3.75 \times 5 = 18.75$

Oliver paid $\textdollar 18.75$ for all his shirts.

(b) Rounding $18.75$:

- To the nearest whole number: $19$
- To 1 decimal place: $18.8$
- To 2 decimal places: $18.75$

So the answers are $19$ (whole number), $18.8$ (1 decimal place) and $18.75$ (2 decimal places).

\section*{Question 46}
\textbf{Metadata}

\begin{itemize}
  \item Question ID: P4-DcMul2d1d\_P4-DcAdd2nd\_GPT4.1\_Services\_03
  \item Primary KC: DECIMALS | Multiplication | multiplying decimals (up to 2 decimal places) by a 1-digit whole number
  \item Secondary KC: DECIMALS | Addition | adding decimals
  \item Topic: Services such as installation, maintenance, repairing, cleaning, laundry, hotel, retail, e-commerce, streaming services, digital services etc.
  \item Grade: Primary 4
\end{itemize}

\textbf{Question}

A laundry service charges $\textdollar2.50$ to clean one shirt. Mdm Lee sends 4 shirts to be cleaned. She also asks for 2 pairs of trousers to be cleaned, with each pair costing $\textdollar3.80$ to clean. How much does Mdm Lee have to pay in total for cleaning her clothes?

\textbf{Solution}

First, calculate the cost of cleaning 4 shirts: 

$4 \times 2.50 = 10.00$

Next, calculate the cost of cleaning 2 pairs of trousers: 

$2 \times 3.80 = 7.60$

Now, add the two amounts to find the total cost: 

$10.00 + 7.60 = 17.60$

Thus, Mdm Lee has to pay $\textdollar17.60$ in total for cleaning her clothes.

\section*{Question 47}
\textbf{Metadata}

\begin{itemize}
  \item Question ID: P4-DcMul2d1d\_P4-DcSub2nd\_GPT4.1\_Services\_03
  \item Primary KC: DECIMALS | Multiplication | multiplying decimals (up to 2 decimal places) by a 1-digit whole number
  \item Secondary KC: DECIMALS | Subtraction | subtracting decimals
  \item Topic: Services such as installation, maintenance, repairing, cleaning, laundry, hotel, retail, e-commerce, streaming services, digital services etc.
  \item Grade: Primary 4
\end{itemize}

\textbf{Question}

A laundry shop charges $\textdollar 2.75$ to wash one shirt. Daniel wants to wash 6 shirts. He pays the shop $\textdollar 20$ in total. After paying for the laundry service, how much change will Daniel receive?

\textbf{Solution}

First, we find the total cost to wash 6 shirts by multiplying the cost to wash one shirt by the number of shirts: 

$2.75 \times 6 = 16.50$

So, the total cost for 6 shirts is $\textdollar 16.50$.

Next, we subtract the total cost from the amount Daniel paid to find the change:

$20.00 - 16.50 = 3.50$

Therefore, Daniel will receive $\textdollar 3.50$ in change.

\section*{Question 48}
\textbf{Metadata}

\begin{itemize}
  \item Question ID: P4-DcDiv2d1d\_P4-DcCmp3d\_GPT4.1\_Services\_03
  \item Primary KC: DECIMALS | Division | dividing decimals (up to 2 decimal places) by a 1-digit whole number
  \item Secondary KC: DECIMALS | Comparison and ordering | comparing and ordering decimals up to 3 decimal places
  \item Topic: Services such as installation, maintenance, repairing, cleaning, laundry, hotel, retail, e-commerce, streaming services, digital services etc.
  \item Grade: Primary 4
\end{itemize}

\textbf{Question}

Siti runs a small laundry service. Last week, she received $36.45$ kg of dirty clothes from $5$ different customers. She wants to divide the total weight equally among the $5$ laundry machines so that each machine washes the same amount of clothes. 

(a) How many kilograms of clothes did each machine have to wash? Give your answer correct to $2$ decimal places.

After washing, she records the weight of clean clothes from each machine as $7.292$ kg, $7.289$ kg, $7.291$ kg, $7.290$ kg, and $7.293$ kg.

(b) Arrange the weights of clean clothes from smallest to largest.

\textbf{Solution}

(a) The total weight of clothes is $36.45$ kg.

To find out how many kilograms each machine washed:

$\frac{36.45}{5} = 7.29$

Each machine washed $7.29$ kg of clothes (correct to $2$ decimal places).

(b) The weights of clean clothes are: $7.292$ kg, $7.289$ kg, $7.291$ kg, $7.290$ kg, $7.293$ kg.

To arrange from smallest to largest:

Smallest: $7.289$ kg
Next: $7.290$ kg
Then: $7.291$ kg
Then: $7.292$ kg
Largest: $7.293$ kg

So the order is:
$7.289$ kg, $7.290$ kg, $7.291$ kg, $7.292$ kg, $7.293$ kg.

\section*{Question 49}
\textbf{Metadata}

\begin{itemize}
  \item Question ID: P4-DcDiv2d1d\_P4-DcCnv2Fr\_GPT4.1\_Services\_03
  \item Primary KC: DECIMALS | Division | dividing decimals (up to 2 decimal places) by a 1-digit whole number
  \item Secondary KC: DECIMALS | Conversion from decimals to fraction | expressing decimals as fractions
  \item Topic: Services such as installation, maintenance, repairing, cleaning, laundry, hotel, retail, e-commerce, streaming services, digital services etc.
  \item Grade: Primary 4
\end{itemize}

\textbf{Question}

A hotel charges guests for bottled water in their rooms. If 5 bottles of water cost \textdollar6.50 in total, how much does 1 bottle cost? Give your answer in decimals. Then, express the price of 1 bottle as a fraction in its simplest form.

\textbf{Solution}

First, divide the total cost by the number of bottles: 

\[
6.50 \div 5 = 1.30
\]

So, 1 bottle costs \textdollar1.30.

Now, express 1.30 as a fraction:

\[
1.30 = \frac{130}{100}
\]

Simplify the fraction by dividing both numerator and denominator by 10:

\[
\frac{130}{100} = \frac{13}{10}
\]

Thus, the price of 1 bottle is \textdollar1.30 or \(\frac{13}{10}\) dollars.

\section*{Question 50}
\textbf{Metadata}

\begin{itemize}
  \item Question ID: P4-DcDiv2d1d\_P4-DcRnd3d\_GPT4.1\_Services\_03
  \item Primary KC: DECIMALS | Division | dividing decimals (up to 2 decimal places) by a 1-digit whole number
  \item Secondary KC: DECIMALS | Rounding | rounding decimals up to 3 decimal places to the nearest whole number, 1 decimal place and 2 decimal places 
  \item Topic: Services such as installation, maintenance, repairing, cleaning, laundry, hotel, retail, e-commerce, streaming services, digital services etc.
  \item Grade: Primary 4
\end{itemize}

\textbf{Question}

A cleaning company charged a customer a total of $\textdollar36.75$ for cleaning her apartment over 5 days, with the cost divided equally between each day. 

(a) How much did the company charge per day?

(b) Round your answer in part (a) to the nearest dollar.

(c) Round your answer in part (a) to 2 decimal places.

\textbf{Solution}

(a) To find the cost per day, divide the total cost by the number of days:

$\displaystyle \frac{36.75}{5} = 7.35$

So, the company charged $\textdollar7.35$ per day.

(b) Rounding $7.35$ to the nearest dollar:

Since $0.35 < 0.5$, round down to $7$.

The cost per day to the nearest dollar is $\textdollar7$.

(c) Rounding $7.35$ to 2 decimal places:

$7.35$ already has 2 decimal places, so the answer remains $\textdollar7.35$.

\section*{Question 51}
\textbf{Metadata}

\begin{itemize}
  \item Question ID: P4-DcDiv2d1d\_P4-DcAdd2nd\_GPT4.1\_Services\_03
  \item Primary KC: DECIMALS | Division | dividing decimals (up to 2 decimal places) by a 1-digit whole number
  \item Secondary KC: DECIMALS | Addition | adding decimals
  \item Topic: Services such as installation, maintenance, repairing, cleaning, laundry, hotel, retail, e-commerce, streaming services, digital services etc.
  \item Grade: Primary 4
\end{itemize}

\textbf{Question}

A cleaning company received a payment of $\textdollar72.45$ for cleaning $9$ hotel rooms. The amount is to be divided equally among the $9$ rooms. Later, another job was completed for $\textdollar15.20$ and the company wants to add this amount to the previous total. How much did the cleaning company earn per room in total after both jobs are included?

\textbf{Solution}

First, find out how much was earned per room in the first job:

\[
72.45 \div 9 = 8.05
\]

So, each room contributed $\textdollar8.05$.

Next, add the payment from the second job:

\[
8.05 + 15.20 = 23.25
\]

The total amount earned after both jobs, per room, is $\textdollar23.25$.

\section*{Question 52}
\textbf{Metadata}

\begin{itemize}
  \item Question ID: P4-DcDiv2d1d\_P4-DcSub2nd\_GPT4.1\_Services\_03
  \item Primary KC: DECIMALS | Division | dividing decimals (up to 2 decimal places) by a 1-digit whole number
  \item Secondary KC: DECIMALS | Subtraction | subtracting decimals
  \item Topic: Services such as installation, maintenance, repairing, cleaning, laundry, hotel, retail, e-commerce, streaming services, digital services etc.
  \item Grade: Primary 4
\end{itemize}

\textbf{Question}

A digital music streaming service charges $\textdollar 14.88$ for a 6-month subscription. Alice paid for the 6 months in advance. However, after using the service for 4 months, she decided to cancel her subscription and got a refund for the unused months. 

(a) How much does Alice pay for each month of the service?

(b) How much is the refund that Alice received for the unused months?

\textbf{Solution}

(a) Alice paid $\textdollar 14.88$ for 6 months.

Amount paid per month $= \frac{14.88}{6} = 2.48$.

So, Alice paid $\textdollar 2.48$ per month.

(b) Alice used 4 months of the service, so the number of unused months $= 6 - 4 = 2$ months.

Refund for unused months $= 2 \times \textdollar 2.48 = \textdollar 4.96$.

Therefore, Alice received a refund of $\textdollar 4.96$.

\section*{Question 53}
\textbf{Metadata}

\begin{itemize}
  \item Question ID: P5-FrAddMix\_P2-FrCmp\_GPT4.1\_Services\_03
  \item Primary KC: FRACTIONS | Addition | adding mixed numbers
  \item Secondary KC: FRACTIONS | Comparison and ordering | comparing and ordering fractions
  \item Topic: Services such as installation, maintenance, repairing, cleaning, laundry, hotel, retail, e-commerce, streaming services, digital services etc.
  \item Grade: Primary 5
\end{itemize}

\textbf{Question}

Jia Min is comparing two different laundry services: CleanQuick and FreshWash. CleanQuick takes $2\frac{1}{4}$ hours to wash and dry her laundry, while FreshWash takes $1\frac{3}{8}$ hours to wash and $1\frac{5}{6}$ hours to dry.\
\
(a) How long does FreshWash take in total to wash and dry Jia Min's laundry?\
\
(b) Compare the total time taken by CleanQuick and FreshWash. Which service is faster, and by how much?\


\textbf{Solution}

(a)\
FreshWash's total time = $1\frac{3}{8} + 1\frac{5}{6}$ hours.\
\
Convert to improper fractions:\\
$1\frac{3}{8} = \frac{11}{8}$\\
$1\frac{5}{6} = \frac{11}{6}$\\
\
Find a common denominator for $\frac{11}{8}$ and $\frac{11}{6}$, which is $24$.\\
$\frac{11}{8} = \frac{11 \times 3}{8 \times 3} = \frac{33}{24}$\\
$\frac{11}{6} = \frac{11 \times 4}{6 \times 4} = \frac{44}{24}$\\
Now add: $\frac{33}{24} + \frac{44}{24} = \frac{77}{24}$\\
$\frac{77}{24}$ as a mixed number:\\
$24 \times 3 = 72$, so $\frac{77}{24} = 3\frac{5}{24}$ hours.\\
\
\textbf{FreshWash takes $3\frac{5}{24}$ hours to wash and dry Jia Min's laundry.}\\
\
(b)\
CleanQuick: $2\frac{1}{4}$ hours = $\frac{9}{4}$ hours\\
FInd a common denominator between $\frac{9}{4}$ and $\frac{77}{24}$, which is $24$.\\
$\frac{9}{4} = \frac{9 \times 6}{4 \times 6} = \frac{54}{24}$\\
$\frac{54}{24}$ (CleanQuick) vs $\frac{77}{24}$ (FreshWash)\\
\
$\frac{77}{24} - \frac{54}{24} = \frac{23}{24}$\\
\
$\frac{77}{24} > \frac{54}{24}$, so FreshWash takes longer.\\
\
\textbf{CleanQuick is faster than FreshWash by $\frac{23}{24}$ hours.}


\section*{Question 54}
\textbf{Metadata}

\begin{itemize}
  \item Question ID: P5-FrAddMix\_P3-FrSmp\_GPT4.1\_Services\_03
  \item Primary KC: FRACTIONS | Addition | adding mixed numbers
  \item Secondary KC: FRACTIONS | Simplifying | expressing a fraction in its simplest form
  \item Topic: Services such as installation, maintenance, repairing, cleaning, laundry, hotel, retail, e-commerce, streaming services, digital services etc.
  \item Grade: Primary 5
\end{itemize}

\textbf{Question}

Mrs Lim hires a cleaning service for her house. On Monday, the cleaners spent $2\frac{1}{4}$ hours cleaning the living room and $1\frac{2}{3}$ hours cleaning the kitchen. 

(a) How many hours did the cleaners spend in total cleaning both rooms? 

(b) Express your answer in part (a) in its simplest form.

\textbf{Solution}

Let us first add the two mixed numbers:

$2\frac{1}{4} + 1\frac{2}{3}$

Step 1: Convert each mixed number to an improper fraction.

$2\frac{1}{4} = 2 + \frac{1}{4} = \frac{8}{4} + \frac{1}{4} = \frac{9}{4}$

$1\frac{2}{3} = 1 + \frac{2}{3} = \frac{3}{3} + \frac{2}{3} = \frac{5}{3}$

Step 2: Add the fractions. First, find the common denominator:

LCM of 4 and 3 is 12.

$\frac{9}{4} = \frac{9 \times 3}{4 \times 3} = \frac{27}{12}$

$\frac{5}{3} = \frac{5 \times 4}{3 \times 4} = \frac{20}{12}$

So,

$\frac{27}{12} + \frac{20}{12} = \frac{47}{12}$

Step 3: Convert $\frac{47}{12}$ back to a mixed number.

$47 \div 12 = 3$ remainder $11$

So,

$\frac{47}{12} = 3\frac{11}{12}$

Step 4: Check if $\frac{11}{12}$ can be simplified. $11$ and $12$ have no common factors other than $1$, so it is in its simplest form.

Answer:

(a) The cleaners spent a total of $3\frac{11}{12}$ hours cleaning both rooms.

(b) $3\frac{11}{12}$ is already in its simplest form.

\section*{Question 55}
\textbf{Metadata}

\begin{itemize}
  \item Question ID: P5-FrAddMix\_P5-FrCnv2Dc\_GPT4.1\_Services\_03
  \item Primary KC: FRACTIONS | Addition | adding mixed numbers
  \item Secondary KC: FRACTIONS | Conversion to decimals | expressing fractions as decimals
  \item Topic: Services such as installation, maintenance, repairing, cleaning, laundry, hotel, retail, e-commerce, streaming services, digital services etc.
  \item Grade: Primary 5
\end{itemize}

\textbf{Question}

Aisha spent $1\dfrac{3}{4}$ hours cleaning a hotel room in the morning. In the afternoon, she spent another $2\dfrac{1}{2}$ hours cleaning a different room. \newline 
(a) How many hours did Aisha spend cleaning in total? \newline 
(b) Express your answer in part (a) as a decimal.

\textbf{Solution}

(a) Total time spent cleaning: \newline 
$1\dfrac{3}{4} + 2\dfrac{1}{2} = (1 + 2) + \left(\dfrac{3}{4} + \dfrac{1}{2}\right) = 3 + \left(\dfrac{3}{4} + \dfrac{2}{4}\right) = 3 + \dfrac{5}{4} = 3 + 1\dfrac{1}{4} = 4\dfrac{1}{4}$ hours. \newline 
(b) To express $4\dfrac{1}{4}$ as a decimal: \newline 
$\dfrac{1}{4} = 0.25$, so $4\dfrac{1}{4} = 4.25$. \newline 
\textbf{Answer:} Aisha spent $4\dfrac{1}{4}$ hours, or $4.25$ hours cleaning in total.

\section*{Question 56}
\textbf{Metadata}

\begin{itemize}
  \item Question ID: P5-FrSubMix\_P2-FrCmp\_GPT4.1\_Services\_03
  \item Primary KC: FRACTIONS | Subtraction | subtracting mixed numbers
  \item Secondary KC: FRACTIONS | Comparison and ordering | comparing and ordering fractions
  \item Topic: Services such as installation, maintenance, repairing, cleaning, laundry, hotel, retail, e-commerce, streaming services, digital services etc.
  \item Grade: Primary 5
\end{itemize}

\textbf{Question}

Mr Lim was cleaning two hotel rooms. He used $2\dfrac{3}{4}$ litres of cleaning solution in Room A and $1\dfrac{5}{8}$ litres in Room B. 

(a) How much more cleaning solution did Mr Lim use in Room A than in Room B?

(b) After cleaning both rooms, did Mr Lim use more or less than 4 litres of cleaning solution in total? Show your working.

\textbf{Solution}

(a) First, subtract the amount used in Room B from the amount used in Room A:

$2\dfrac{3}{4} - 1\dfrac{5}{8}$

Convert both numbers to improper fractions:

$2\dfrac{3}{4} = \dfrac{8 + 3}{4} = \dfrac{11}{4}$

$1\dfrac{5}{8} = \dfrac{8 + 5}{8} = \dfrac{13}{8}$

To subtract, use a common denominator (8):

$\dfrac{11}{4} = \dfrac{22}{8}$

So,

$\dfrac{22}{8} - \dfrac{13}{8} = \dfrac{9}{8} = 1\dfrac{1}{8}$

Mr Lim used $1\dfrac{1}{8}$ litres more cleaning solution in Room A.

(b) Add the amounts used in both rooms:

$2\dfrac{3}{4} + 1\dfrac{5}{8}$

Convert to improper fractions:

$2\dfrac{3}{4} = \dfrac{11}{4} = \dfrac{22}{8}$

$1\dfrac{5}{8} = \dfrac{13}{8}$

Add:

$\dfrac{22}{8} + \dfrac{13}{8} = \dfrac{35}{8} = 4\dfrac{3}{8}$

Compare $4\dfrac{3}{8}$ and 4:

$4\dfrac{3}{8}$ is more than 4.

So Mr Lim used more than 4 litres of cleaning solution in total.

\section*{Question 57}
\textbf{Metadata}

\begin{itemize}
  \item Question ID: P5-FrSubMix\_P2-FrAdd2nd\_GPT4.1\_Services\_03
  \item Primary KC: FRACTIONS | Subtraction | subtracting mixed numbers
  \item Secondary KC: FRACTIONS | Addition | adding fractions
  \item Topic: Services such as installation, maintenance, repairing, cleaning, laundry, hotel, retail, e-commerce, streaming services, digital services etc.
  \item Grade: Primary 5
\end{itemize}

\textbf{Question}

A washing machine repairman spent $4\dfrac{1}{2}$ hours repairing washing machines in the morning. In the afternoon, he repaired washing machines for another $2\dfrac{2}{3}$ hours. However, $1\dfrac{3}{4}$ hours of his afternoon work was spent cleaning up and not actual repairing. 

(a) What is the total number of hours he spent repairing washing machines (before removing the cleaning time)?

(b) How many hours did he actually spend repairing washing machines in total when the cleaning up time is not included?

\textbf{Solution}

(a) Total number of hours spent repairing (before removing the cleaning time):

$4\dfrac{1}{2} + 2\dfrac{2}{3}$

Convert to improper fractions:

$4\dfrac{1}{2} = \frac{9}{2}$

$2\dfrac{2}{3} = \frac{8}{3}$

Find a common denominator (6):

$\frac{9}{2} = \frac{27}{6}$
$\frac{8}{3} = \frac{16}{6}$

$\frac{27}{6} + \frac{16}{6} = \frac{43}{6}$

In mixed number, $\frac{43}{6} = 7\dfrac{1}{6}$

So, he spent $7\dfrac{1}{6}$ hours in total (before removing cleaning time).

(b) Actual hours spent repairing, after subtracting cleaning time:

Cleaning time: $1\dfrac{3}{4} = \frac{7}{4}$

Convert $7\dfrac{1}{6}$ to improper fraction:

$7\dfrac{1}{6} = \frac{43}{6}$

Find common denominator for subtraction. The LCM of 6 and 4 is 12.

$\frac{43}{6} = \frac{86}{12}$

$\frac{7}{4} = \frac{21}{12}$

$\frac{86}{12} - \frac{21}{12} = \frac{65}{12}$

$\frac{65}{12} = 5\dfrac{5}{12}$

So, the repairman actually spent $5\dfrac{5}{12}$ hours repairing washing machines in total when the cleaning time is not included.

\section*{Question 58}
\textbf{Metadata}

\begin{itemize}
  \item Question ID: P5-FrSubMix\_P3-FrSmp\_GPT4.1\_Services\_03
  \item Primary KC: FRACTIONS | Subtraction | subtracting mixed numbers
  \item Secondary KC: FRACTIONS | Simplifying | expressing a fraction in its simplest form
  \item Topic: Services such as installation, maintenance, repairing, cleaning, laundry, hotel, retail, e-commerce, streaming services, digital services etc.
  \item Grade: Primary 5
\end{itemize}

\textbf{Question}

A hotel cleaner filled a detergent tank with $4 \dfrac{2}{3}$ litres of cleaning solution at the start of the day. After cleaning all the guest rooms, there were $1 \dfrac{3}{4}$ litres of cleaning solution left in the tank. 

How much cleaning solution did the cleaner use? Express your answer in its simplest form.

\textbf{Solution}

The amount of cleaning solution used is:

\[
4 \dfrac{2}{3} - 1 \dfrac{3}{4}
\]

First, convert both mixed numbers to improper fractions:

\[
4 \dfrac{2}{3} = \frac{4 \times 3 + 2}{3} = \frac{14}{3}
\]
\[
1 \dfrac{3}{4} = \frac{1 \times 4 + 3}{4} = \frac{7}{4}
\]

Subtract the fractions (find common denominator):

The common denominator of $3$ and $4$ is $12$.

\[
\frac{14}{3} = \frac{56}{12}
\]
\[
\frac{7}{4} = \frac{21}{12}
\]

So,
\[
\frac{56}{12} - \frac{21}{12} = \frac{35}{12}
\]

Express $\frac{35}{12}$ as a mixed number:

\[
35 \div 12 = 2 \text{ remainder } 11
\]

So,
\[
\frac{35}{12} = 2 \dfrac{11}{12}
\]

Therefore, the cleaner used $2 \dfrac{11}{12}$ litres of cleaning solution.

\section*{Question 59}
\textbf{Metadata}

\begin{itemize}
  \item Question ID: P5-FrSubMix\_P5-FrCnv2Dc\_GPT4.1\_Services\_03
  \item Primary KC: FRACTIONS | Subtraction | subtracting mixed numbers
  \item Secondary KC: FRACTIONS | Conversion to decimals | expressing fractions as decimals
  \item Topic: Services such as installation, maintenance, repairing, cleaning, laundry, hotel, retail, e-commerce, streaming services, digital services etc.
  \item Grade: Primary 5
\end{itemize}

\textbf{Question}

A hotel offers a laundry service for its guests. Jasmine sent a bundle of clothes to be washed and they weighed $5\frac{3}{4}$ kg before cleaning. After cleaning and drying, the clothes now weigh $4\frac{2}{5}$ kg. 

(a) What is the difference in weight between the clothes before and after cleaning? Give your answer as a mixed number.

(b) Express your answer in part (a) as a decimal, correct to 2 decimal places.

\textbf{Solution}

(a) The difference in weight is $5\frac{3}{4} - 4\frac{2}{5}$ kg.

First, convert the mixed numbers to improper fractions:

$5\frac{3}{4} = \frac{20}{4} + \frac{3}{4} = \frac{23}{4}$

$4\frac{2}{5} = \frac{20}{5} + \frac{2}{5} = \frac{22}{5}$

Express both fractions with a common denominator (LCM of $4$ and $5$ is $20$):

$\frac{23}{4} = \frac{23 \times 5}{4 \times 5} = \frac{115}{20}$

$\frac{22}{5} = \frac{22 \times 4}{5 \times 4} = \frac{88}{20}$

Subtract:

$\frac{115}{20} - \frac{88}{20} = \frac{27}{20}$

$\frac{27}{20}$ as a mixed number is $1\frac{7}{20}$. So, the difference in weight is $1\frac{7}{20}$ kg.

(b) To express $1\frac{7}{20}$ as a decimal:

$\frac{7}{20} = 0.35$

So, $1 + 0.35 = 1.35$

Therefore, the difference in weight correct to 2 decimal places is $1.35$ kg.

\section*{Question 60}
\textbf{Metadata}

\begin{itemize}
  \item Question ID: P5-FrMulImN\_P2-FrCmp\_GPT4.1\_Services\_03
  \item Primary KC: FRACTIONS | Multiplication | multiplying a proper/improper fraction and a whole number
  \item Secondary KC: FRACTIONS | Comparison and ordering | comparing and ordering fractions
  \item Topic: Services such as installation, maintenance, repairing, cleaning, laundry, hotel, retail, e-commerce, streaming services, digital services etc.
  \item Grade: Primary 5
\end{itemize}

\textbf{Question}

A digital streaming service offers a family plan where each member can watch $\frac{3}{4}$ of the available movies each month. If there are 6 members in Annie's family, what is the total fraction of available movies watched by Annie's family in a month?

Last month, Annie compared the fraction of movies her family watched with her cousin's family, who watched $\frac{7}{5}$ of the available movies (as her cousin's family rewatched many movies). Order the total fraction of movies watched by Annie's family and her cousin's family from least to greatest.

\textbf{Solution}

First, find the total fraction of movies watched by Annie's family:

Each member watches $\frac{3}{4}$ of the movies.

There are 6 members:
\[
6 \times \frac{3}{4} = \frac{6 \times 3}{4} = \frac{18}{4} = \frac{9}{2}
\]

So, Annie's family watched $\frac{9}{2}$ of the movies (some movies may have been watched by more than one member).

Now, compare $\frac{9}{2}$ (Annie's family) and $\frac{7}{5}$ (cousin's family).

Convert both to decimals:
- $\frac{9}{2} = 4.5$
- $\frac{7}{5} = 1.4$

Order from least to greatest:
\[
\frac{7}{5},\ \frac{9}{2}
\]

So, Annie's family watched a greater fraction of the available movies than her cousin's family.

\section*{Question 61}
\textbf{Metadata}

\begin{itemize}
  \item Question ID: P5-FrMulImN\_P2-FrAdd2nd\_GPT4.1\_Services\_03
  \item Primary KC: FRACTIONS | Multiplication | multiplying a proper/improper fraction and a whole number
  \item Secondary KC: FRACTIONS | Addition | adding fractions
  \item Topic: Services such as installation, maintenance, repairing, cleaning, laundry, hotel, retail, e-commerce, streaming services, digital services etc.
  \item Grade: Primary 5
\end{itemize}

\textbf{Question}

Siti works for a laundry service. She needs to wash bedsheets for a hotel. For each room, she uses $\frac{3}{4}$ litre of detergent to wash the bedsheets. There are 8 rooms to wash bedsheets for today.<br><br>After washing the bedsheets for all the rooms, she finds that she still has $\frac{1}{2}$ litre of detergent left in her container. How much detergent did she have at first?

\textbf{Solution}

First, find the total amount of detergent Siti used for 8 rooms:<br><br>
$8 \times \frac{3}{4} = \frac{8 \times 3}{4} = \frac{24}{4} = 6$ litres<br><br>
Siti used 6 litres and has $\frac{1}{2}$ litre left. So, the total amount she had at first is:<br><br>
$6 + \frac{1}{2}$ litres<br><br>
Express 6 as a fraction with denominator 2:<br>
$6 = \frac{12}{2}$<br><br>
$\frac{12}{2} + \frac{1}{2} = \frac{13}{2}$ litres<br><br>
Therefore, Siti had $\frac{13}{2}$ litres of detergent at first, which is $6\frac{1}{2}$ litres.

\section*{Question 62}
\textbf{Metadata}

\begin{itemize}
  \item Question ID: P5-FrMulImN\_P2-FrSub2nd\_GPT4.1\_Services\_03
  \item Primary KC: FRACTIONS | Multiplication | multiplying a proper/improper fraction and a whole number
  \item Secondary KC: FRACTIONS | Subtraction | subtracting fractions
  \item Topic: Services such as installation, maintenance, repairing, cleaning, laundry, hotel, retail, e-commerce, streaming services, digital services etc.
  \item Grade: Primary 5
\end{itemize}

\textbf{Question}

A hotel uses $\dfrac{3}{4}$ of a bottle of cleaning solution to clean one guest room. On Monday, the hotel staff cleaned 6 rooms. On Tuesday, they used $\dfrac{5}{2}$ bottles of cleaning solution in total for cleaning. 

How many more bottles of cleaning solution did the hotel use on Monday than on Tuesday?

\textbf{Solution}

First, find how much cleaning solution was used on Monday:

They used $\dfrac{3}{4}$ bottle per room, for 6 rooms:

$\dfrac{3}{4} \times 6 = \dfrac{3 \times 6}{4} = \dfrac{18}{4} = \dfrac{9}{2}$ bottles$

On Tuesday, they used $\dfrac{5}{2}$ bottles.

Find the difference:

$\dfrac{9}{2} - \dfrac{5}{2} = \dfrac{4}{2} = 2$

So, the hotel used 2 more bottles of cleaning solution on Monday than on Tuesday.

\section*{Question 63}
\textbf{Metadata}

\begin{itemize}
  \item Question ID: P5-FrMulImN\_P3-FrSmp\_GPT4.1\_Services\_03
  \item Primary KC: FRACTIONS | Multiplication | multiplying a proper/improper fraction and a whole number
  \item Secondary KC: FRACTIONS | Simplifying | expressing a fraction in its simplest form
  \item Topic: Services such as installation, maintenance, repairing, cleaning, laundry, hotel, retail, e-commerce, streaming services, digital services etc.
  \item Grade: Primary 5
\end{itemize}

\textbf{Question}

A hotel serves fruit juice to its guests every morning. Each jug contains $\frac{7}{8}$ litres of juice. If the staff prepares 12 jugs of juice for breakfast, how many litres of juice do they have in total? Express your answer in its simplest fractional form.

\textbf{Solution}

To find out how many litres of juice there are in total, multiply the fraction of a litre in one jug by the number of jugs:

$\frac{7}{8} \times 12 = \frac{7 \times 12}{8}$

$= \frac{84}{8}$

Now, simplify $\frac{84}{8}$:

Divide the numerator and denominator by 4:

$\frac{84 \div 4}{8 \div 4} = \frac{21}{2}$

So, the hotel has $\frac{21}{2}$ litres of juice in total.

\section*{Question 64}
\textbf{Metadata}

\begin{itemize}
  \item Question ID: P5-FrMulImN\_P5-FrCnv2Dc\_GPT4.1\_Services\_03
  \item Primary KC: FRACTIONS | Multiplication | multiplying a proper/improper fraction and a whole number
  \item Secondary KC: FRACTIONS | Conversion to decimals | expressing fractions as decimals
  \item Topic: Services such as installation, maintenance, repairing, cleaning, laundry, hotel, retail, e-commerce, streaming services, digital services etc.
  \item Grade: Primary 5
\end{itemize}

\textbf{Question}

A cleaning company charges $\textdollar5$ to clean one small carpet. It usually takes $\frac{3}{5}$ of an hour to clean one small carpet. 

(a) How many hours does it take to clean 8 small carpets?

(b) Express the total time taken in part (a) as a decimal.

\textbf{Solution}

(a) Time taken to clean one carpet $= \frac{3}{5}$ hour.

Time taken to clean 8 carpets $= 8 \times \frac{3}{5} = \frac{24}{5}$ hours.

(b) To express $\frac{24}{5}$ as a decimal:

$\frac{24}{5} = 4.8$

So, the total time taken is $4.8$ hours.

\section*{Question 65}
\textbf{Metadata}

\begin{itemize}
  \item Question ID: P5-FrMulPIm\_P2-FrCmp\_GPT4.1\_Services\_03
  \item Primary KC: FRACTIONS | Multiplication | multiplying a proper fraction and a proper/improper fractions
  \item Secondary KC: FRACTIONS | Comparison and ordering | comparing and ordering fractions
  \item Topic: Services such as installation, maintenance, repairing, cleaning, laundry, hotel, retail, e-commerce, streaming services, digital services etc.
  \item Grade: Primary 5
\end{itemize}

\textbf{Question}

Lina is cleaning rooms in a hotel. She cleans $\frac{3}{5}$ of all the rooms in the morning. After lunch, she cleans another $\frac{7}{9}$ of what she cleaned in the morning.  

(a) What fraction of all the rooms did Lina clean after lunch?

(b) Compare and order the following fractions, from least to greatest: $\frac{3}{5}$ (the fraction cleaned in the morning), $\frac{7}{9}$ (the fraction of morning rooms cleaned after lunch), and the fraction of all rooms Lina cleaned after lunch.

\textbf{Solution}

(a) The fraction of all the rooms Lina cleaned after lunch is:

She cleaned $\frac{7}{9}$ \, \text{of what was cleaned in the morning}$, so
\[
\text{Fraction of all rooms cleaned after lunch} = \frac{7}{9} \times \frac{3}{5} = \frac{7 \times 3}{9 \times 5} = \frac{21}{45}.
\]
This fraction can be simplified by dividing both numerator and denominator by $3$:
\[
\frac{21}{45} = \frac{7}{15}.
\]
So, Lina cleaned $\frac{7}{15}$ of all the rooms after lunch.

(b) To compare and order $\frac{3}{5}, \frac{7}{9},$ and $\frac{7}{15}$:

First, convert each fraction to have a common denominator:
- The least common multiple of $5, 9, 15$ is $45$.

\[
\frac{3}{5} = \frac{3 \times 9}{5 \times 9} = \frac{27}{45}
\]
\[
\frac{7}{9} = \frac{7 \times 5}{9 \times 5} = \frac{35}{45}
\]
\[
\frac{7}{15} = \frac{7 \times 3}{15 \times 3} = \frac{21}{45}
\]
Now compare $\frac{21}{45}$, $\frac{27}{45}$, $\frac{35}{45}$.

\[
\frac{7}{15} < \frac{3}{5} < \frac{7}{9}.
\]
Thus, in order from least to greatest:
\[
\boxed{\frac{7}{15},\ \frac{3}{5},\ \frac{7}{9}}
\]

\section*{Question 66}
\textbf{Metadata}

\begin{itemize}
  \item Question ID: P5-FrMulPIm\_P2-FrAdd2nd\_GPT4.1\_Services\_03
  \item Primary KC: FRACTIONS | Multiplication | multiplying a proper fraction and a proper/improper fractions
  \item Secondary KC: FRACTIONS | Addition | adding fractions
  \item Topic: Services such as installation, maintenance, repairing, cleaning, laundry, hotel, retail, e-commerce, streaming services, digital services etc.
  \item Grade: Primary 5
\end{itemize}

\textbf{Question}

Mrs Lim is cleaning houses for a cleaning service. She cleans $\dfrac{3}{5}$ of a house each hour. On Monday, she worked for $\dfrac{7}{2}$ hours.

(a) How much of a house did she clean on Monday?

On Tuesday, she cleaned an additional $\dfrac{4}{5}$ of a house.

(b) How much of a house did Mrs Lim clean in total on Monday and Tuesday?

\textbf{Solution}

(a) Mrs Lim cleaned $\dfrac{3}{5}$ of a house each hour for $\dfrac{7}{2}$ hours.

So, the total fraction of a house cleaned on Monday is:
$\dfrac{3}{5} \times \dfrac{7}{2} = \dfrac{3 \times 7}{5 \times 2} = \dfrac{21}{10}$ of a house.

(b) On Tuesday, she cleaned $\dfrac{4}{5}$ of a house more. So, the total cleaned on both days is:

$\dfrac{21}{10} + \dfrac{4}{5}$

First, rewrite $\dfrac{4}{5}$ with denominator 10:
$\dfrac{4}{5} = \dfrac{8}{10}$

Now, add:
$\dfrac{21}{10} + \dfrac{8}{10} = \dfrac{29}{10}$

Therefore, Mrs Lim cleaned $\dfrac{29}{10}$ houses in total on Monday and Tuesday.

\section*{Question 67}
\textbf{Metadata}

\begin{itemize}
  \item Question ID: P5-FrMulPIm\_P2-FrSub2nd\_GPT4.1\_Services\_03
  \item Primary KC: FRACTIONS | Multiplication | multiplying a proper fraction and a proper/improper fractions
  \item Secondary KC: FRACTIONS | Subtraction | subtracting fractions
  \item Topic: Services such as installation, maintenance, repairing, cleaning, laundry, hotel, retail, e-commerce, streaming services, digital services etc.
  \item Grade: Primary 5
\end{itemize}

\textbf{Question}

A hotel offers a towel laundry service for its guests. On Monday, $\frac{5}{8}$ of all towels in the hotel were sent for laundry. Out of these towels, $\frac{3}{4}$ were white towels and the rest were coloured towels. After the laundry, a staff member noticed that $\frac{1}{6}$ of the total towels in the hotel were still dirty because they were not sent for laundry. What fraction of the total towels in the hotel were clean coloured towels after the laundry was done?

\textbf{Solution}

Let the total number of towels be 1 unit. The fraction of towels sent for laundry is $\frac{5}{8}$.\newline

Out of these, the fraction that were white towels: $\frac{3}{4} \times \frac{5}{8} = \frac{15}{32}$.\newline

Fraction of towels sent for laundry that were coloured towels: \newline
$\frac{5}{8} - \frac{15}{32} = \frac{20}{32} - \frac{15}{32} = \frac{5}{32}$.\newline

These $\frac{5}{32}$ represent the clean coloured towels after laundry.\newline

Therefore, the fraction of the total towels in the hotel that were clean coloured towels after the laundry was:\newline
$\boxed{\frac{5}{32}}$.

\section*{Question 68}
\textbf{Metadata}

\begin{itemize}
  \item Question ID: P5-FrMulPIm\_P3-FrSmp\_GPT4.1\_Services\_03
  \item Primary KC: FRACTIONS | Multiplication | multiplying a proper fraction and a proper/improper fractions
  \item Secondary KC: FRACTIONS | Simplifying | expressing a fraction in its simplest form
  \item Topic: Services such as installation, maintenance, repairing, cleaning, laundry, hotel, retail, e-commerce, streaming services, digital services etc.
  \item Grade: Primary 5
\end{itemize}

\textbf{Question}

A cleaning company charges $\textdollar80$ for a full house cleaning service. During a promotion, customers only need to pay $\frac{3}{4}$ of the usual price. If Mr. Tan uses an online discount coupon that further reduces the new price by $\frac{2}{5}$, how much does he need to pay for the cleaning service in the end? Give your answer in lowest terms.

\textbf{Solution}

First, calculate the promotional price:

\[
\text{Promotional price} = \frac{3}{4} \times \textdollar 80 = \textdollar \left(\frac{3}{4} \times 80\right) = \textdollar 60
\]

Next, calculate the price after using the coupon:

\[
\text{Discounted price} = \frac{2}{5} \times \textdollar 60 = \textdollar \left(\frac{2}{5} \times 60\right) = \textdollar 24
\]

So, the amount Mr. Tan still needs to pay:

\[
\textdollar 60 - \textdollar 24 = \textdollar 36
\]

Therefore, Mr. Tan needs to pay \(\boxed{\textdollar 36}\) for the cleaning service in the end.

\section*{Question 69}
\textbf{Metadata}

\begin{itemize}
  \item Question ID: P5-FrMulPIm\_P5-FrCnv2Dc\_GPT4.1\_Services\_03
  \item Primary KC: FRACTIONS | Multiplication | multiplying a proper fraction and a proper/improper fractions
  \item Secondary KC: FRACTIONS | Conversion to decimals | expressing fractions as decimals
  \item Topic: Services such as installation, maintenance, repairing, cleaning, laundry, hotel, retail, e-commerce, streaming services, digital services etc.
  \item Grade: Primary 5
\end{itemize}

\textbf{Question}

A hotel laundry charges its customers according to the weight of clothes to be washed. For every load, Jasmine sends $\frac{3}{4}$ kg of clothes to be cleaned. If the laundry cleans $\frac{5}{2}$ times as much weight in total during a sale week, how many kilograms of clothes does the laundry clean in total for Jasmine during the sale week? Express your answer as a decimal.

\textbf{Solution}

Jasmine sends $\frac{3}{4}$ kg for each load. During the sale week, the laundry cleans $\frac{5}{2}$ times as much:

$\frac{3}{4} \times \frac{5}{2} = \frac{3 \times 5}{4 \times 2} = \frac{15}{8}$ kg.

Now, to convert $\frac{15}{8}$ to a decimal, divide $15$ by $8$:

$\frac{15}{8} = 1.875$

Therefore, the laundry cleans $1.875$ kg of Jasmine's clothes in total during the sale week.

\section*{Question 70}
\textbf{Metadata}

\begin{itemize}
  \item Question ID: P5-FrMulImIm\_P2-FrCmp\_GPT4.1\_Services\_03
  \item Primary KC: FRACTIONS | Multiplication | multiplying two improper fractions
  \item Secondary KC: FRACTIONS | Comparison and ordering | comparing and ordering fractions
  \item Topic: Services such as installation, maintenance, repairing, cleaning, laundry, hotel, retail, e-commerce, streaming services, digital services etc.
  \item Grade: Primary 5
\end{itemize}

\textbf{Question}

A cleaning company offers a package where each room gets $\frac{7}{4}$ hours of cleaning service. Last week, Mrs. Tan booked the company to clean $\frac{5}{2}$ rooms in her house (because one room only needed half the usual service).

(a) What is the total number of hours the company spent cleaning Mrs. Tan's house?

(b) Another customer, Mr. Wong, booked cleaning for $\frac{3}{2}$ rooms at the same rate of $\frac{7}{4}$ hours per room. Who received more cleaning hours, Mrs. Tan or Mr. Wong? Arrange their total cleaning hours in order from least to greatest.

\textbf{Solution}

(a) The total number of cleaning hours for Mrs. Tan:

$\frac{7}{4} \times \frac{5}{2} = \frac{7 \times 5}{4 \times 2} = \frac{35}{8}$ hours.

(b) The total number of cleaning hours for Mr. Wong:

$\frac{7}{4} \times \frac{3}{2} = \frac{7 \times 3}{4 \times 2} = \frac{21}{8}$ hours.

Comparing the two totals:

$\frac{35}{8}$ and $\frac{21}{8}$

Since $35 > 21$, $\frac{35}{8} > \frac{21}{8}$.

Therefore, Mr. Wong received $\frac{21}{8}$ hours and Mrs. Tan received $\frac{35}{8}$ hours.

Arranging their total cleaning hours in order from least to greatest:

Mr. Wong: $\frac{21}{8}$, Mrs. Tan: $\frac{35}{8}$.

So, Mrs. Tan received more cleaning hours.

\section*{Question 71}
\textbf{Metadata}

\begin{itemize}
  \item Question ID: P5-FrMulImIm\_P2-FrAdd2nd\_GPT4.1\_Services\_03
  \item Primary KC: FRACTIONS | Multiplication | multiplying two improper fractions
  \item Secondary KC: FRACTIONS | Addition | adding fractions
  \item Topic: Services such as installation, maintenance, repairing, cleaning, laundry, hotel, retail, e-commerce, streaming services, digital services etc.
  \item Grade: Primary 5
\end{itemize}

\textbf{Question}

A hotel laundry service uses $\dfrac{9}{4}$ kilograms of detergent to wash all the bedsheets from one floor. If there are $\dfrac{5}{3}$ times as many towels as bedsheets, and each towel uses the same amount of detergent as each bedsheet, how much detergent is needed to wash all the towels from one floor? What is the total amount of detergent needed to wash all the bedsheets and towels from one floor?

\textbf{Solution}

To find the amount of detergent needed for the towels, multiply the amount of detergent used for bedsheets by $\dfrac{5}{3}$:

\[
\text{Detergent for towels} = \dfrac{9}{4} \times \dfrac{5}{3}
\]
Multiply the numerators and denominators:
\[
\dfrac{9 \times 5}{4 \times 3} = \dfrac{45}{12}
\]
Simplify $\dfrac{45}{12}$:
\[
\dfrac{45}{12} = \dfrac{15}{4}
\]
So, $\dfrac{15}{4}$ kilograms of detergent is needed for the towels.

Next, add the detergent needed for bedsheets and towels:
\[
\dfrac{9}{4} + \dfrac{15}{4} = \dfrac{24}{4} = 6
\]

\textbf{Answer:} The towels require $\dfrac{15}{4}$ kilograms of detergent. The total detergent needed for all the bedsheets and towels from one floor is $6$ kilograms.

\section*{Question 72}
\textbf{Metadata}

\begin{itemize}
  \item Question ID: P5-FrMulImIm\_P2-FrSub2nd\_GPT4.1\_Services\_03
  \item Primary KC: FRACTIONS | Multiplication | multiplying two improper fractions
  \item Secondary KC: FRACTIONS | Subtraction | subtracting fractions
  \item Topic: Services such as installation, maintenance, repairing, cleaning, laundry, hotel, retail, e-commerce, streaming services, digital services etc.
  \item Grade: Primary 5
\end{itemize}

\textbf{Question}

Siti is a manager at a laundry service. She received an order to clean \(\frac{7}{4}\) loads of bedsheets and \(\frac{9}{5}\) loads of curtains for a hotel. For each load, the laundry machine uses \(\frac{5}{3}\) litres of detergent. 

(a) How many litres of detergent are needed in total to clean both the bedsheets and the curtains?

(b) After cleaning the bedsheets and curtains, Siti found that \(\frac{8}{3}\) litres of detergent were still left. How many litres of detergent did she use for cleaning?

\textbf{Solution}

(a) Total loads to be cleaned: 
\[
\frac{7}{4} + \frac{9}{5}
\]
First, find a common denominator:
\[
\frac{7}{4} = \frac{35}{20}, \quad \frac{9}{5} = \frac{36}{20}
\]
So,
\[
\frac{35}{20} + \frac{36}{20} = \frac{71}{20}\text{ loads}
\]
Detergent needed per load is \( \frac{5}{3} \) litres. Total detergent:
\[
\frac{71}{20} \times \frac{5}{3} = \frac{71 \times 5}{20 \times 3} = \frac{355}{60} = \frac{71}{12}\text{ litres}
\]

(b) Detergent used = Total detergent needed $-$ Detergent left
\[
\frac{71}{12} - \frac{8}{3}
\]
Express \(\frac{8}{3}\) with denominator 12:
\[
\frac{8}{3} = \frac{32}{12}
\]
So,
\[
\frac{71}{12} - \frac{32}{12} = \frac{39}{12} = \frac{13}{4}\text{ litres}
\]

\textbf{Answers:}
(a) Siti needs \(\frac{71}{12}\) litres of detergent in total.

(b) She used \(\frac{13}{4}\) litres of detergent for cleaning.

\section*{Question 73}
\textbf{Metadata}

\begin{itemize}
  \item Question ID: P5-FrMulImIm\_P3-FrSmp\_GPT4.1\_Services\_03
  \item Primary KC: FRACTIONS | Multiplication | multiplying two improper fractions
  \item Secondary KC: FRACTIONS | Simplifying | expressing a fraction in its simplest form
  \item Topic: Services such as installation, maintenance, repairing, cleaning, laundry, hotel, retail, e-commerce, streaming services, digital services etc.
  \item Grade: Primary 5
\end{itemize}

\textbf{Question}

During the school holidays, Sally helped her parents by doing laundry for the entire family. Each time she used the washing machine, she was able to wash $\frac{9}{4}$ loads of clothes. On the first weekend, she washed $\frac{8}{3}$ times as many loads as usual because more relatives came to stay. 

How many loads of laundry did Sally wash on the first weekend in total? Give your answer as a fraction in its simplest form.

\textbf{Solution}

Sally usually washes $\frac{9}{4}$ loads each time. On the first weekend, she washed $\frac{8}{3}$ times as many loads. This means the total number of loads she washed is:

$\frac{9}{4} \times \frac{8}{3} = \frac{9 \times 8}{4 \times 3} = \frac{72}{12}$

Now, we simplify $\frac{72}{12}$:

$\frac{72}{12} = \frac{72 \div 12}{12 \div 12} = \frac{6}{1} = 6$

So, Sally washed $6$ loads of laundry on the first weekend.

\section*{Question 74}
\textbf{Metadata}

\begin{itemize}
  \item Question ID: P5-FrMulImIm\_P5-FrCnv2Dc\_GPT4.1\_Services\_03
  \item Primary KC: FRACTIONS | Multiplication | multiplying two improper fractions
  \item Secondary KC: FRACTIONS | Conversion to decimals | expressing fractions as decimals
  \item Topic: Services such as installation, maintenance, repairing, cleaning, laundry, hotel, retail, e-commerce, streaming services, digital services etc.
  \item Grade: Primary 5
\end{itemize}

\textbf{Question}

A digital movie streaming service allows users to watch movies for a certain number of days. Sarah watched movies for $\dfrac{9}{4}$ days in the first week and $\dfrac{7}{3}$ days in the second week. \textbf{a)} For how many days in total did she watch movies over the two weeks if the total is found by multiplying the two fractions together? \textbf{b)} Express your answer from part (a) as a decimal, rounded to 2 decimal places.

\textbf{Solution}

\textbf{a)}\quad$ Sarah watched $\dfrac{9}{4}$ days in the first week and $\dfrac{7}{3}$ days in the second week.\newline
To find the total number of days by multiplying, we calculate:

\[
\dfrac{9}{4} \times \dfrac{7}{3} = \dfrac{9 \times 7}{4 \times 3} = \dfrac{63}{12}
\]

\textbf{b)}\quad$ To express $\dfrac{63}{12}$ as a decimal:

\[
\dfrac{63}{12} = 5.25
\]

\textbf{Answer:} Sarah watched movies for $5.25$ days in total over the two weeks (rounded to 2 decimal places).

\section*{Question 75}
\textbf{Metadata}

\begin{itemize}
  \item Question ID: P5-FrMulMixN\_P2-FrCmp\_GPT4.1\_Services\_03
  \item Primary KC: FRACTIONS | Multiplication | multiplying a mixed number and a whole number
  \item Secondary KC: FRACTIONS | Comparison and ordering | comparing and ordering fractions
  \item Topic: Services such as installation, maintenance, repairing, cleaning, laundry, hotel, retail, e-commerce, streaming services, digital services etc.
  \item Grade: Primary 5
\end{itemize}

\textbf{Question}

A cleaning company offers special packages for cleaning hotel rooms. Each cleaning package covers $2\frac{2}{3}$ rooms. 

(a) If a hotel books 5 cleaning packages, how many rooms in total can be cleaned?

(b) After the packages are used, the company records the number of rooms cleaned by each staff member: $9\frac{1}{3}$, $10\frac{2}{3}$, and $8\frac{1}{2}$ rooms. Arrange the number of rooms cleaned in order from least to greatest.

\textbf{Solution}

(a) Total rooms cleaned $= 2\frac{2}{3} \times 5$. 

First, convert $2\frac{2}{3}$ to an improper fraction:
$$2\frac{2}{3} = 2 + \frac{2}{3} = \frac{6}{3} + \frac{2}{3} = \frac{8}{3}$$

Now, multiply by 5:
$$\frac{8}{3} \times 5 = \frac{8 \times 5}{3} = \frac{40}{3}$$

Convert $\frac{40}{3}$ to a mixed number:

$40 \div 3 = 13$ remainder $1$

So, $\frac{40}{3} = 13\frac{1}{3}$

Total rooms cleaned $= 13\frac{1}{3}$ rooms.

(b) The staff cleaned $9\frac{1}{3}$, $10\frac{2}{3}$, and $8\frac{1}{2}$ rooms. 

Let's convert all mixed numbers to improper fractions:

$9\frac{1}{3} = \frac{28}{3}$

$10\frac{2}{3} = \frac{32}{3}$

$8\frac{1}{2} = \frac{17}{2}$

To compare, find a common denominator. The least common multiple for 3 and 2 is 6.

$9\frac{1}{3} = \frac{28}{3} = \frac{28 \times 2}{3 \times 2} = \frac{56}{6}$

$10\frac{2}{3} = \frac{32}{3} = \frac{32 \times 2}{3 \times 2} = \frac{64}{6}$

$8\frac{1}{2} = \frac{17}{2} = \frac{17 \times 3}{2 \times 3} = \frac{51}{6}$

Now, arrange in order from least to greatest:

$\frac{51}{6}$ ($8\frac{1}{2}$ rooms), $\frac{56}{6}$ ($9\frac{1}{3}$ rooms), $\frac{64}{6}$ ($10\frac{2}{3}$ rooms)

So, the order is: $8\frac{1}{2}$, $9\frac{1}{3}$, $10\frac{2}{3}$.

\section*{Question 76}
\textbf{Metadata}

\begin{itemize}
  \item Question ID: P5-FrMulMixN\_P2-FrAdd2nd\_GPT4.1\_Services\_03
  \item Primary KC: FRACTIONS | Multiplication | multiplying a mixed number and a whole number
  \item Secondary KC: FRACTIONS | Addition | adding fractions
  \item Topic: Services such as installation, maintenance, repairing, cleaning, laundry, hotel, retail, e-commerce, streaming services, digital services etc.
  \item Grade: Primary 5
\end{itemize}

\textbf{Question}

A cleaning company charges for their cleaning service based on how long they clean. For each hour, they charge $\textdollar 12$. In one job, a team cleans a house for $2\dfrac{1}{2}$ hours. 

Afterwards, the customer asks them to clean an additional $\dfrac{3}{4}$ hour in another room. 

What is the total cost the customer has to pay for the cleaning service?

\textbf{Solution}

First, find out the cost for $2\dfrac{1}{2}$ hours.

$2\dfrac{1}{2} = \dfrac{5}{2}$

Multiply the fraction by $12$:

$\dfrac{5}{2} \times 12 = \dfrac{5 \times 12}{2} = \dfrac{60}{2} = 30$

So, the cost for the first part is $\textdollar 30$.

Next, find the cost for $\dfrac{3}{4}$ hour:

$\dfrac{3}{4} \times 12 = \dfrac{3 \times 12}{4} = \dfrac{36}{4} = 9$

So, the cost for the additional cleaning is $\textdollar 9$.

Finally, add both amounts together:

$30 + 9 = 39$

\boxed{\textdollar 39}$

Therefore, the total cost the customer has to pay is $\textdollar 39$.

\section*{Question 77}
\textbf{Metadata}

\begin{itemize}
  \item Question ID: P5-FrMulMixN\_P2-FrSub2nd\_GPT4.1\_Services\_03
  \item Primary KC: FRACTIONS | Multiplication | multiplying a mixed number and a whole number
  \item Secondary KC: FRACTIONS | Subtraction | subtracting fractions
  \item Topic: Services such as installation, maintenance, repairing, cleaning, laundry, hotel, retail, e-commerce, streaming services, digital services etc.
  \item Grade: Primary 5
\end{itemize}

\textbf{Question}

Sarah is helping her parents do laundry for their hotel business. In one day, she washes $3\dfrac{1}{2}$ loads of guest towels. Each load uses $4$ litres of detergent. 

(a) How many litres of detergent does Sarah use in total for washing the towels?

After completing the washing, Sarah realises that $2\dfrac{2}{3}$ litres of detergent is left in the bottle. 

(b) How much detergent did Sarah use, to the nearest fraction, if there were $16$ litres in the bottle at the start of the day?

\textbf{Solution}

(a) To find the total amount of detergent used:

$3\dfrac{1}{2} \times 4 = \left(\frac{7}{2}\right) \times 4 = \frac{7 \times 4}{2} = \frac{28}{2} = 14$

Sarah used $14$ litres of detergent in total for washing the towels.

(b) Sarah started with $16$ litres and finished with $2\dfrac{2}{3}$ litres left.

Amount of detergent used $= 16 - 2\dfrac{2}{3}$

Convert $2\dfrac{2}{3}$ to an improper fraction:
$2\dfrac{2}{3} = \frac{8}{3}$

Now, subtract:
$16 - \frac{8}{3} = \frac{48}{3} - \frac{8}{3} = \frac{40}{3}$

$\frac{40}{3} = 13\dfrac{1}{3}$ litres

Sarah used $13\dfrac{1}{3}$ litres of detergent, to the nearest fraction.

\textbf{Answers:}
(a) $14$ litres
(b) $13\dfrac{1}{3}$ litres

\section*{Question 78}
\textbf{Metadata}

\begin{itemize}
  \item Question ID: P5-FrMulMixN\_P3-FrSmp\_GPT4.1\_Services\_03
  \item Primary KC: FRACTIONS | Multiplication | multiplying a mixed number and a whole number
  \item Secondary KC: FRACTIONS | Simplifying | expressing a fraction in its simplest form
  \item Topic: Services such as installation, maintenance, repairing, cleaning, laundry, hotel, retail, e-commerce, streaming services, digital services etc.
  \item Grade: Primary 5
\end{itemize}

\textbf{Question}

A cleaning service charges $\textdollar 7$ for every $2\frac{1}{2}$ hours of work. If a customer hires the cleaning service for $4$ times that duration, how much does the cleaning service charge in total? Give your answer in its simplest fractional form (if necessary).

\textbf{Solution}

First, calculate the total hours of work: 

$2\frac{1}{2} \times 4 = \left(\frac{5}{2}\right) \times 4 = \frac{5 \times 4}{2} = \frac{20}{2} = 10$ hours.

The charge for $2\frac{1}{2}$ hours is $\textdollar 7$.

The charge per hour is:

$\dfrac{\textdollar 7}{2\frac{1}{2}} = \dfrac{\textdollar 7}{\frac{5}{2}} = \textdollar 7 \times \frac{2}{5} = \textdollar \frac{14}{5}$ per hour.

Total charge for $10$ hours:

$10 \times \textdollar \frac{14}{5} = \textdollar \frac{140}{5} = \textdollar 28$.

\textbf{Answer:} The cleaning service charges $\textdollar 28$ in total.

\section*{Question 79}
\textbf{Metadata}

\begin{itemize}
  \item Question ID: P5-FrMulMixN\_P5-FrCnv2Dc\_GPT4.1\_Services\_03
  \item Primary KC: FRACTIONS | Multiplication | multiplying a mixed number and a whole number
  \item Secondary KC: FRACTIONS | Conversion to decimals | expressing fractions as decimals
  \item Topic: Services such as installation, maintenance, repairing, cleaning, laundry, hotel, retail, e-commerce, streaming services, digital services etc.
  \item Grade: Primary 5
\end{itemize}

\textbf{Question}

A laundry shop charges $\textdollar 3.50$ to wash $1$ kg of clothes. Mei sends $2\frac{1}{2}$ kg of clothes for washing each week. How much does she have to pay for one week? Express your answer in decimal form.

\textbf{Solution}

First, convert the mixed number $2\frac{1}{2}$ into an improper fraction:

$2\frac{1}{2} = \frac{5}{2}$

Now, multiply the number of kilograms by the price per kilogram:

$\frac{5}{2} \times \textdollar 3.50 = \textdollar (5 \times 3.50 \div 2)$

$5 \times 3.50 = 17.50$

$17.50 \div 2 = 8.75$

Therefore, Mei has to pay $\textdollar 8.75$ for one week.

\section*{Question 80}
\textbf{Metadata}

\begin{itemize}
  \item Question ID: P5-DcMul3dK\_P4-DcCmp3d\_GPT4.1\_Services\_03
  \item Primary KC: DECIMALS | Multiplication | multiplying decimals (up to 3 decimal places) by 10, 100, 1000 and their multiples
  \item Secondary KC: DECIMALS | Comparison and ordering | comparing and ordering decimals up to 3 decimal places
  \item Topic: Services such as installation, maintenance, repairing, cleaning, laundry, hotel, retail, e-commerce, streaming services, digital services etc.
  \item Grade: Primary 5
\end{itemize}

\textbf{Question}

A hotel charges \textdollar2.375 per hour for high-speed internet access in its rooms. During a convention, a guest used the internet for 100 hours. Another guest used the internet for only 10 hours. 

(a) How much did each guest pay for their internet usage?

(b) Which guest paid more, and by how much?

(c) Arrange the following total amounts paid for different guests in ascending order: \textdollar23.750, \textdollar237.500, \textdollar2.375.

\textbf{Solution}

(a) For the guest who used 100 hours:

$2.375 \times 100 = 237.5$

So, the guest paid \textdollar237.500.

For the guest who used 10 hours:

$2.375 \times 10 = 23.75$

So, the guest paid \textdollar23.750.

(b) The guest who paid more is the one who used 100 hours. 

Difference:

$237.5 - 23.75 = 213.75$

He paid \textdollar213.750 more than the other guest.

(c) To arrange \textdollar23.750, \textdollar237.500, \textdollar2.375 in ascending order:

\textdollar2.375, \textdollar23.750, \textdollar237.500.

\section*{Question 81}
\textbf{Metadata}

\begin{itemize}
  \item Question ID: P5-DcMul3dK\_P4-DcCnv2Fr\_GPT4.1\_Services\_03
  \item Primary KC: DECIMALS | Multiplication | multiplying decimals (up to 3 decimal places) by 10, 100, 1000 and their multiples
  \item Secondary KC: DECIMALS | Conversion from decimals to fraction | expressing decimals as fractions
  \item Topic: Services such as installation, maintenance, repairing, cleaning, laundry, hotel, retail, e-commerce, streaming services, digital services etc.
  \item Grade: Primary 5
\end{itemize}

\textbf{Question}

A laundry shop charges \$3.25 to clean one shirt. A company sends 100 shirts to the laundry shop for cleaning for an upcoming event. 

(a) What is the total cost charged by the laundry shop to clean all 100 shirts?

(b) The manager wants to express the cost of cleaning one shirt as a fraction in its simplest form. Express $3.25 as a fraction in its simplest form.

\textbf{Solution}

(a) The total cost to clean all 100 shirts is:

\[
3.25 \times 100 = 325
\]

So, the laundry shop charges \textdollar325 for cleaning 100 shirts.

(b) To express $3.25 as a fraction:

\[
3.25 = 3 + 0.25 = \frac{3}{1} + \frac{25}{100}
\]

Simplify \( \frac{25}{100} = \frac{1}{4} \):

\[
3.25 = 3 + \frac{1}{4} = \frac{12}{4} + \frac{1}{4} = \frac{13}{4}
\]

So, $3.25 as a fraction in its simplest form is \( \frac{13}{4} \ ).

\section*{Question 82}
\textbf{Metadata}

\begin{itemize}
  \item Question ID: P5-DcMul3dK\_P4-DcRnd3d\_GPT4.1\_Services\_03
  \item Primary KC: DECIMALS | Multiplication | multiplying decimals (up to 3 decimal places) by 10, 100, 1000 and their multiples
  \item Secondary KC: DECIMALS | Rounding | rounding decimals up to 3 decimal places to the nearest whole number, 1 decimal place and 2 decimal places 
  \item Topic: Services such as installation, maintenance, repairing, cleaning, laundry, hotel, retail, e-commerce, streaming services, digital services etc.
  \item Grade: Primary 5
\end{itemize}

\textbf{Question}

Siti wants to hire a cleaning service for her home. The cleaning company charges $\$12.375$ per hour. She is planning to hire them for $7$ hours.\
\
(a) What is the total cost for $7$ hours of cleaning before rounding?\
\
(b) Round the total cost to the nearest whole number, to 1 decimal place, and to 2 decimal places.

\textbf{Solution}

(a) The total cost for $7$ hours is:\\
$\$12.375 \times 7 = \$86.625$\\
\
(b) Rounding the total cost:\\
- To the nearest whole number: $\$86.625$ rounds to $\$87$.\\
- To 1 decimal place: $\$86.625$ rounds to $\$86.6$.\\
- To 2 decimal places: $\$86.625$ rounds to $\$86.63$.

\section*{Question 83}
\textbf{Metadata}

\begin{itemize}
  \item Question ID: P5-DcMul3dK\_P4-DcAdd2nd\_GPT4.1\_Services\_03
  \item Primary KC: DECIMALS | Multiplication | multiplying decimals (up to 3 decimal places) by 10, 100, 1000 and their multiples
  \item Secondary KC: DECIMALS | Addition | adding decimals
  \item Topic: Services such as installation, maintenance, repairing, cleaning, laundry, hotel, retail, e-commerce, streaming services, digital services etc.
  \item Grade: Primary 5
\end{itemize}

\textbf{Question}

A cleaning company charges \$2.350 for each square metre of carpet cleaned. Mr Tan asked the company to clean 45 square metres of carpet in his office. After the cleaning, he also paid an extra \$15.60 for special stain removal. What is the total amount Mr Tan paid the cleaning company?

\textbf{Solution}

First, calculate the cost of cleaning 45 square metres of carpet:

\[
2.350 \times 45 = 105.750
\]

This means the cleaning cost is \$105.750.

Next, add the cost of special stain removal:

\[
105.750 + 15.60 = 121.350
\]

So, Mr Tan paid the cleaning company a total of \$121.350.

\section*{Question 84}
\textbf{Metadata}

\begin{itemize}
  \item Question ID: P5-DcMul3dK\_P4-DcSub2nd\_GPT4.1\_Services\_03
  \item Primary KC: DECIMALS | Multiplication | multiplying decimals (up to 3 decimal places) by 10, 100, 1000 and their multiples
  \item Secondary KC: DECIMALS | Subtraction | subtracting decimals
  \item Topic: Services such as installation, maintenance, repairing, cleaning, laundry, hotel, retail, e-commerce, streaming services, digital services etc.
  \item Grade: Primary 5
\end{itemize}

\textbf{Question}

A cleaning company charges $\textdollar1.275$ per square metre to clean carpets. Mr Tan wants to clean his living room, which has an area of $45$ square metres. He also has a small rug, which covers $3.5$ square metres, and he does not want it cleaned. 

(a) What is the total cost to clean only the living room area (excluding the rug)? 

(b) If Mr Tan paid $\textdollar70$ in cash, how much change should he get back after paying for the cleaning service?

\textbf{Solution}

(a) Area to clean $= 45 - 3.5 = 41.5$ square metres.

Cost per square metre $= \textdollar1.275$.

Total cost $= 41.5 \times 1.275 = 52.9125$.

So, total cost $= \textdollar52.9125$.

(b) Amount Mr Tan paid $= \textdollar70$.

Change he should get $= 70 - 52.9125 = 17.0875$.

So, the change is $\textdollar17.0875$. (To the nearest cent, $\textdollar17.09$)

\section*{Question 85}
\textbf{Metadata}

\begin{itemize}
  \item Question ID: P5-DcDiv3dK\_P4-DcCmp3d\_GPT4.1\_Services\_03
  \item Primary KC: DECIMALS | Division | dividing decimals (up to 3 decimal places) by 10, 100, 1000 and their multiples
  \item Secondary KC: DECIMALS | Comparison and ordering | comparing and ordering decimals up to 3 decimal places
  \item Topic: Services such as installation, maintenance, repairing, cleaning, laundry, hotel, retail, e-commerce, streaming services, digital services etc.
  \item Grade: Primary 5
\end{itemize}

\textbf{Question}

Jolene sent a parcel using an express delivery service. The total cost of delivery for a 2.375 kg parcel was \textdollar19. The delivery service charged the same price per 0.100 kg for all parcels. 

(a) Find the cost to deliver 0.100 kg of the parcel. Give your answer correct to 3 decimal places.

(b) The following week, Jolene sent three smaller parcels using the same service. Each parcel weighed 0.740 kg, 1.205 kg and 0.580 kg respectively. Arrange the delivery costs for these three parcels in order from the lowest to the highest. Give your answers correct to 3 decimal places.

\textbf{Solution}

(a) \text{To find the cost to deliver } 0.100 \text{ kg:}

\[
\text{Number of 0.100 kg units in } 2.375 \text{ kg} = \frac{2.375}{0.100} = 23.75
\]

\[
\text{Cost per 0.100 kg} = \frac{\textdollar19}{23.75} = \textdollar0.800
\]

So, the cost to deliver 0.100 kg is \boxed{\textdollar0.800}.

(b) \text{Find the cost for each parcel:}

\text{First parcel (0.740 kg):}
\[
\frac{0.740}{0.100} = 7.4\text{ units} \\
7.4 \times \textdollar0.800 = \textdollar5.920
\]

\text{Second parcel (1.205 kg):}
\[
\frac{1.205}{0.100} = 12.05\text{ units} \\
12.05 \times \textdollar0.800 = \textdollar9.640
\]

\text{Third parcel (0.580 kg):}
\[
\frac{0.580}{0.100} = 5.8\text{ units} \\
5.8 \times \textdollar0.800 = \textdollar4.640
\]

\text{Delivery costs (to 3 decimal places):}
\begin{align*}
0.740~\text{kg} &: \textdollar5.920 \\
1.205~\text{kg} &: \textdollar9.640 \\
0.580~\text{kg} &: \textdollar4.640 
\end{align*}

\text{Ordering from lowest to highest:}

\boxed{\textdollar4.640, \textdollar5.920, \textdollar9.640}

\text{(i.e., 0.580 kg, 0.740 kg, 1.205 kg)}.

\section*{Question 86}
\textbf{Metadata}

\begin{itemize}
  \item Question ID: P5-DcDiv3dK\_P4-DcCnv2Fr\_GPT4.1\_Services\_03
  \item Primary KC: DECIMALS | Division | dividing decimals (up to 3 decimal places) by 10, 100, 1000 and their multiples
  \item Secondary KC: DECIMALS | Conversion from decimals to fraction | expressing decimals as fractions
  \item Topic: Services such as installation, maintenance, repairing, cleaning, laundry, hotel, retail, e-commerce, streaming services, digital services etc.
  \item Grade: Primary 5
\end{itemize}

\textbf{Question}

A car wash service charges \textdollar25.800 for washing 100 cars in a day. 

(a) How much does it cost, on average, to wash 1 car? Give your answer as a decimal.

(b) Express your answer in (a) as a fraction in its simplest form.

\textbf{Solution}

(a) To find the average cost to wash 1 car, divide the total cost by the number of cars:

\[
\text{Average cost per car} = \frac{25.800}{100} = 0.258
\]

So, it costs \textdollar0.258 to wash 1 car.

(b) To express $0.258$ as a fraction:

\[
0.258 = \frac{258}{1000}
\]
Now, simplify $\frac{258}{1000}$ by dividing both numerator and denominator by 2:

\[
\frac{258}{1000} = \frac{129}{500}
\]

Therefore, $0.258$ as a fraction in its simplest form is $\frac{129}{500}$.

\section*{Question 87}
\textbf{Metadata}

\begin{itemize}
  \item Question ID: P5-DcDiv3dK\_P4-DcRnd3d\_GPT4.1\_Services\_03
  \item Primary KC: DECIMALS | Division | dividing decimals (up to 3 decimal places) by 10, 100, 1000 and their multiples
  \item Secondary KC: DECIMALS | Rounding | rounding decimals up to 3 decimal places to the nearest whole number, 1 decimal place and 2 decimal places 
  \item Topic: Services such as installation, maintenance, repairing, cleaning, laundry, hotel, retail, e-commerce, streaming services, digital services etc.
  \item Grade: Primary 5
\end{itemize}

\textbf{Question}

A cleaning company charges \textdollar145.680 for a deep cleaning service that lasts 10 hours. 

(a) How much does the company charge per hour? Give your answer in decimal form up to 3 decimal places.

(b) Round the hourly charge to the nearest cent (2 decimal places).

(c) If a customer needs only 1 hour of cleaning and the company rounds the charge to the nearest whole number, how much will the customer pay?

\textbf{Solution}

(a) To find the charge per hour:

\[
\text{Charge per hour} = \frac{145.680}{10} = 14.568
\]

(b) To the nearest cent (2 decimal places):

\[
14.568 \approx 14.57
\]

(c) Rounding \$14.568 to the nearest whole number:

\[
14.568 \approx 15
\]

So the customer will pay \textdollar15 for 1 hour of cleaning.

\section*{Question 88}
\textbf{Metadata}

\begin{itemize}
  \item Question ID: P5-DcDiv3dK\_P4-DcAdd2nd\_GPT4.1\_Services\_03
  \item Primary KC: DECIMALS | Division | dividing decimals (up to 3 decimal places) by 10, 100, 1000 and their multiples
  \item Secondary KC: DECIMALS | Addition | adding decimals
  \item Topic: Services such as installation, maintenance, repairing, cleaning, laundry, hotel, retail, e-commerce, streaming services, digital services etc.
  \item Grade: Primary 5
\end{itemize}

\textbf{Question}

A cleaning service company charges $\textdollar12.750$ to clean one room. The company is offering a special promotion: for every 10 rooms cleaned, you get a discount equal to the cost of cleaning one room divided by $10$, added to the total bill as a discount. 

(a) What is the amount of discount given for every 10 rooms cleaned?

(b) If a customer hires the company to clean $23$ rooms, how much is the total amount to be paid after applying the appropriate discount?


\textbf{Solution}

(a) The cost of cleaning one room is $12.750$. The discount for every 10 rooms is $\frac{12.750}{10} = 1.275$.

(b) For 23 rooms:

Cost before discount: $12.750 \times 23 = 293.250$

Number of discounts applied: For every 10 rooms, there is 1 discount. $23 \div 10 = 2$ complete sets of 10 rooms, so $2$ discounts.

Total discount: $1.275 \times 2 = 2.550$

Final amount to be paid: $293.250 - 2.550 = 290.700$

The customer should pay $\textdollar290.700$ after applying the discounts.

\section*{Question 89}
\textbf{Metadata}

\begin{itemize}
  \item Question ID: P5-DcDiv3dK\_P4-DcSub2nd\_GPT4.1\_Services\_03
  \item Primary KC: DECIMALS | Division | dividing decimals (up to 3 decimal places) by 10, 100, 1000 and their multiples
  \item Secondary KC: DECIMALS | Subtraction | subtracting decimals
  \item Topic: Services such as installation, maintenance, repairing, cleaning, laundry, hotel, retail, e-commerce, streaming services, digital services etc.
  \item Grade: Primary 5
\end{itemize}

\textbf{Question}

A hotel offers a digital movie streaming service that costs $ \textdollar 24.950 $ for 1,000 minutes of viewing. After watching some shows, Sarah realises that she has used only 375.250 minutes worth of streaming. 

(a) How much does 1 minute of streaming cost?

(b) What is the unused minute balance that Sarah has?

(c) How much is the value of her unused minutes?

\textbf{Solution}

(a) To find the cost of 1 minute of streaming, divide $ \textdollar 24.950 $ by $ 1,000 $:

\[
\text{Cost per minute} = \frac{24.950}{1,000} = 0.02495
\]
So, 1 minute of streaming costs $ \textdollar 0.02495 $.

(b) Sarah's unused minute balance is her total minutes minus the minutes she has already used:
\[
1,000 - 375.250 = 624.750 
\]
She has $ 624.750 $ unused minutes left.

(c) To find the value of her unused minutes, multiply the number of unused minutes by the cost per minute:
\[
624.750 \times 0.02495 = 15.5860125
\]
So, the value of her unused minutes is $ \textdollar 15.59 $ (rounded to 2 decimal places).

\section*{Question 90}
\textbf{Metadata}

\begin{itemize}
  \item Question ID: P5-PcRepWh\_P1-WNAdd2nd\_GPT4.1\_Services\_03
  \item Primary KC: PERCENTAGE | Representation and concept | expressing a part of a whole as a percentage
  \item Secondary KC: WHOLE NUMBERS | Addition | adding whole numbers
  \item Topic: Services such as installation, maintenance, repairing, cleaning, laundry, hotel, retail, e-commerce, streaming services, digital services etc.
  \item Grade: Primary 5
\end{itemize}

\textbf{Question}

A cleaning company charges a fee for cleaning hospital rooms. Last month, they cleaned 80 rooms. Out of these, 24 rooms were cleaned more than once. What percentage of the rooms cleaned last month were cleaned more than once? Give your answer as a percentage.

\textbf{Solution}

First, we know that the number of hospital rooms cleaned more than once is $24$ out of a total of $80$ rooms cleaned in the month.

To find the percentage, we express $24$ as a percentage of $80$:

\[
\text{Percentage} = \left( \frac{24}{80} \right) \times 100\%
\]

Calculate $\frac{24}{80}$:
\[
\frac{24}{80} = \frac{3}{10} = 0.3
\]

Then multiply by $100\%$:
\[
0.3 \times 100\% = 30\%
\]

Therefore, $30\%$ of the rooms cleaned last month were cleaned more than once.

\section*{Question 91}
\textbf{Metadata}

\begin{itemize}
  \item Question ID: P5-PcRepWh\_P1-WNSub2nd\_GPT4.1\_Services\_03
  \item Primary KC: PERCENTAGE | Representation and concept | expressing a part of a whole as a percentage
  \item Secondary KC: WHOLE NUMBERS | Subtraction | subtracting whole numbers
  \item Topic: Services such as installation, maintenance, repairing, cleaning, laundry, hotel, retail, e-commerce, streaming services, digital services etc.
  \item Grade: Primary 5
\end{itemize}

\textbf{Question}

A cleaning company had 120 customers last month. This month, they lost 15 customers. What percentage of their customers did the cleaning company lose this month?

\textbf{Solution}

Number of customers lost = $15$

Total customers last month = $120$

Percentage of customers lost $= \dfrac{15}{120} \times 100\%$

$= \dfrac{1}{8} \times 100\%$

$= 12.5\%$

\textbf{The cleaning company lost $12.5\%$ of their customers this month.}

\section*{Question 92}
\textbf{Metadata}

\begin{itemize}
  \item Question ID: P5-PcRepWh\_P1-WNMul2nd\_GPT4.1\_Services\_03
  \item Primary KC: PERCENTAGE | Representation and concept | expressing a part of a whole as a percentage
  \item Secondary KC: WHOLE NUMBERS | Multiplication | multiplying whole numbers
  \item Topic: Services such as installation, maintenance, repairing, cleaning, laundry, hotel, retail, e-commerce, streaming services, digital services etc.
  \item Grade: Primary 5
\end{itemize}

\textbf{Question}

A hotel is running a promotion where 20\% of its rooms are reserved for guests attending a special event. If there are 8 floors in the hotel and each floor has 25 rooms, how many rooms in total are reserved for the event?

\textbf{Solution}

First, find the total number of rooms in the hotel:

$8 \times 25 = 200$

So, the hotel has 200 rooms in total.

Next, find how many rooms are reserved for the event:

$20\%$ of $200 = \frac{20}{100} \times 200 = 40$

\textbf{Answer}: 40 rooms are reserved for the event.

\section*{Question 93}
\textbf{Metadata}

\begin{itemize}
  \item Question ID: P5-PcRepWh\_P1-WNDiv2nd\_GPT4.1\_Services\_03
  \item Primary KC: PERCENTAGE | Representation and concept | expressing a part of a whole as a percentage
  \item Secondary KC: WHOLE NUMBERS | Division | dividing whole numbers
  \item Topic: Services such as installation, maintenance, repairing, cleaning, laundry, hotel, retail, e-commerce, streaming services, digital services etc.
  \item Grade: Primary 5
\end{itemize}

\textbf{Question}

A cleaning company is hired to clean 80 hotel rooms in one day. After finishing their work, the company reports that 60\% of the rooms have been cleaned by lunchtime. How many hotel rooms did the company clean by lunchtime? If these cleaned rooms are to be split equally among 4 supervisors to check, how many rooms does each supervisor need to check?

\textbf{Solution}

The total number of rooms is $80$. To find out how many rooms were cleaned by lunchtime, calculate $60\%$ of $80$:

$60\%$ of $80 = \frac{60}{100} \times 80 = 0.6 \times 80 = 48$

So, the company cleaned $48$ rooms by lunchtime.

To find out how many rooms each supervisor checks, divide $48$ by $4$:

$\frac{48}{4} = 12$

Each supervisor needs to check $12$ rooms.

\section*{Question 94}
\textbf{Metadata}

\begin{itemize}
  \item Question ID: P5-RtFndR\_P2-DcCnvN2D\_GPT4.1\_Services\_03
  \item Primary KC: RATE | Finding rate | finding rate given total amount and number of units
  \item Secondary KC: DECIMALS | Conversion to larger units | converting an amount or measurement from a smaller unit to a larger unit in decimal form
  \item Topic: Services such as installation, maintenance, repairing, cleaning, laundry, hotel, retail, e-commerce, streaming services, digital services etc.
  \item Grade: Primary 5
\end{itemize}

\textbf{Question}

A car wash company earned a total of $\textdollar324$ from washing $27$ cars in one day. The company records the average earnings per car in dollars. However, at the end of the week, the manager wants to know the average earnings per $10$ cars in decimal form. What is the average amount earned:

(a) for washing $1$ car?

(b) for washing $10$ cars? Give your answer in decimal form with the correct unit.

\textbf{Solution}

(a) First, find the average amount earned for washing $1$ car:

\[
\text{Average per 1 car} = \frac{\textdollar324}{27} = \textdollar12
\]

(b) To find the average amount earned for washing $10$ cars, convert the amount to a larger unit:

\[
\text{Average per 10 cars} = \text{Average per car} \times 10 = \textdollar12 \times 10 = \textdollar120
\]

Therefore,

(a) The average amount earned for washing $1$ car is $\textdollar12$.

(b) The average amount earned for washing $10$ cars is $\textdollar120$. (in decimal form: $\textdollar120.00$)


\section*{Question 95}
\textbf{Metadata}

\begin{itemize}
  \item Question ID: P5-RtFndR\_P2-DcCnvD2N\_GPT4.1\_Services\_03
  \item Primary KC: RATE | Finding rate | finding rate given total amount and number of units
  \item Secondary KC: DECIMALS | Conversion to smaller units | converting an amount or measurement from a larger unit in decimal form to a smaller unit
  \item Topic: Services such as installation, maintenance, repairing, cleaning, laundry, hotel, retail, e-commerce, streaming services, digital services etc.
  \item Grade: Primary 5
\end{itemize}

\textbf{Question}

Anna works at a laundry service that charges by weight. She cleaned \(38.4\) kilograms of clothes for a customer. The total bill was \textdollar76.80. What is the cost per gram of laundry cleaned? (Note: \(1\) kilogram is \(1000\) grams.)

\textbf{Solution}

First, find the rate per kilogram:

\[
\text{Rate per kilogram} = \frac{\textdollar76.80}{38.4} = \textdollar2.00\text{ per kilogram}
\]

Next, convert the rate to cost per gram. Since \(1\) kilogram = \(1000\) grams:

\[
\text{Rate per gram} = \frac{\textdollar2.00}{1000} = \textdollar0.002\text{ per gram}
\]

So, the cost per gram of laundry cleaned is \(\boxed{\textdollar0.002}\).

\section*{Question 96}
\textbf{Metadata}

\begin{itemize}
  \item Question ID: P5-RtFndT\_P2-DcCnvN2D\_GPT4.1\_Services\_03
  \item Primary KC: RATE | Finding total amount | finding total amount, given rate and number of units
  \item Secondary KC: DECIMALS | Conversion to larger units | converting an amount or measurement from a smaller unit to a larger unit in decimal form
  \item Topic: Services such as installation, maintenance, repairing, cleaning, laundry, hotel, retail, e-commerce, streaming services, digital services etc.
  \item Grade: Primary 5
\end{itemize}

\textbf{Question}

A cleaning company charges $\textdollar 0.75$ for cleaning 1 square metre of carpet. Mrs Tan hired the company to clean $4500$ square decimetres of carpet in her office. What is the total amount Mrs Tan needs to pay? (Note: $1$ square metre $= 100$ square decimetres)

\textbf{Solution}

First, convert $4500$ square decimetres to square metres.

Since $1$ square metre $= 100$ square decimetres:

$\dfrac{4500}{100} = 45$ square metres.

Next, find the total cost by multiplying by the rate:

Total amount $= 45 \times \textdollar 0.75 = \textdollar 33.75$.

\textbf{Mrs Tan needs to pay $\textdollar 33.75$.}

\section*{Question 97}
\textbf{Metadata}

\begin{itemize}
  \item Question ID: P5-RtFndT\_P2-DcCnvD2N\_GPT4.1\_Services\_03
  \item Primary KC: RATE | Finding total amount | finding total amount, given rate and number of units
  \item Secondary KC: DECIMALS | Conversion to smaller units | converting an amount or measurement from a larger unit in decimal form to a smaller unit
  \item Topic: Services such as installation, maintenance, repairing, cleaning, laundry, hotel, retail, e-commerce, streaming services, digital services etc.
  \item Grade: Primary 5
\end{itemize}

\textbf{Question}

Mr. Tan hires a cleaning company to clean his office. The company charges $\textdollar2.50$ per square metre. His office has an area of $36.75$ square metres. How much does Mr. Tan need to pay in total? If Mr. Tan wants to know how much he pays per square centimetre, convert his total payment to the amount per square centimetre. ($1$ square metre $= 10\,000$ square centimetres)

\textbf{Solution}

First, find the total amount Mr. Tan needs to pay:  

Amount to pay = Rate per square metre $\times$ Number of square metres  
$= \textdollar2.50 \times 36.75$  
$= \textdollar91.875$

Next, find the total area in square centimetres:
$36.75$ square metres $= 36.75 \times 10,000 = 367,500$ square centimetres

Now, find the amount paid per square centimetre:

Amount per square centimetre $= \dfrac{\textdollar91.875}{367,500}$
$= \textdollar0.00025$

So, Mr. Tan needs to pay \textdollar91.88 ($\textdollar91.875$ rounded to the nearest cent) in total, and he pays \textdollar0.00025 per square centimetre.

\section*{Question 98}
\textbf{Metadata}

\begin{itemize}
  \item Question ID: P5-RtFndU\_P2-DcCnvN2D\_GPT4.1\_Services\_03
  \item Primary KC: RATE | Finding number of unit | finding number of units given rate and total amount
  \item Secondary KC: DECIMALS | Conversion to larger units | converting an amount or measurement from a smaller unit to a larger unit in decimal form
  \item Topic: Services such as installation, maintenance, repairing, cleaning, laundry, hotel, retail, e-commerce, streaming services, digital services etc.
  \item Grade: Primary 5
\end{itemize}

\textbf{Question}

A cleaning company charges $\textdollar0.75$ for each square metre of carpet cleaned. Mrs Lim has a carpet that measures $1850$ square decimetres. She wants to get her carpet cleaned and wants to know how many square metres of carpet she has, and how much she needs to pay in total.

(a) Convert $1850$ square decimetres to square metres in decimal form.

(b) How much does Mrs Lim need to pay the cleaning company to clean her carpet?

\textbf{Solution}

(a) To convert $1850$ square decimetres to square metres:

$1$ square metre $= 100$ square decimetres

So, number of square metres $= 1850 \div 100 = 18.5$ square metres

(b) The cleaning company charges $\textdollar0.75$ per square metre.

Total cost $= 18.5 \times 0.75 = 13.875$

So, Mrs Lim needs to pay $\textdollar13.88$ (rounded to the nearest cent).

\section*{Question 99}
\textbf{Metadata}

\begin{itemize}
  \item Question ID: P5-RtFndU\_P2-DcCnvD2N\_GPT4.1\_Services\_03
  \item Primary KC: RATE | Finding number of unit | finding number of units given rate and total amount
  \item Secondary KC: DECIMALS | Conversion to smaller units | converting an amount or measurement from a larger unit in decimal form to a smaller unit
  \item Topic: Services such as installation, maintenance, repairing, cleaning, laundry, hotel, retail, e-commerce, streaming services, digital services etc.
  \item Grade: Primary 5
\end{itemize}

\textbf{Question}

A laundry shop charges \textdollar2.40 per kilogram to wash clothes. Mrs Tan brought some clothes to the shop, and the total wash cost was \textdollar16.80.

(a) How many kilograms of clothes did Mrs Tan send for washing?

(b) If Mrs Tan wants to know how many grams of clothes she sent, convert your answer in (a) to grams. (Note: $1$ kilogram $= 1000$ grams.)

\textbf{Solution}

(a) The price per kilogram is \textdollar2.40. The total cost was \textdollar16.80.

Number of kilograms $= \dfrac{\textdollar16.80}{\textdollar2.40} = 7$

So, Mrs Tan sent $7$ kilograms of clothes.

(b) To find the number of grams, we multiply the number of kilograms by $1000$:

Number of grams $= 7 \times 1000 = 7000$

So, Mrs Tan sent $7000$ grams of clothes.

\section*{Question 100}
\textbf{Metadata}

\begin{itemize}
  \item Question ID: P6-FrDivPN\_P2-FrCmp\_GPT4.1\_Services\_08
  \item Primary KC: FRACTIONS | Division | dividing a proper fraction by a whole number
  \item Secondary KC: FRACTIONS | Comparison and ordering | comparing and ordering fractions
  \item Topic: Services such as installation, maintenance, repairing, cleaning, laundry, hotel, retail, e-commerce, streaming services, digital services etc.
  \item Grade: Primary 6
\end{itemize}

\textbf{Question}

A cleaning company has a bottle containing $\frac{3}{4}$ litre of cleaning solution. The manager wants to divide the cleaning solution equally among 3 workers for their tasks today.

(a) How much cleaning solution will each worker receive?

(b) One worker compares the amount of cleaning solution he received to another worker who was given $\frac{2}{9}$ litre. Who received more cleaning solution? Arrange the two amounts in order from greatest to least.

\textbf{Solution}

(a) To find how much each worker receives, divide $\frac{3}{4}$ litre by 3:

\[
\frac{3}{4} \div 3 = \frac{3}{4} \times \frac{1}{3} = \frac{3\times 1}{4\times 3} = \frac{3}{12} = \frac{1}{4}
\]

So, each worker receives $\frac{1}{4}$ litre of cleaning solution.

(b) Now, compare $\frac{1}{4}$ and $\frac{2}{9}$:

To compare, find a common denominator. The lowest common multiple of 4 and 9 is 36.

\[
\frac{1}{4} = \frac{1 \times 9}{4 \times 9} = \frac{9}{36}
\]
\[
\frac{2}{9} = \frac{2 \times 4}{9 \times 4} = \frac{8}{36}
\]

So, $\frac{1}{4} (=\frac{9}{36})$ is greater than $\frac{2}{9} (=\frac{8}{36})$.

Arranged from greatest to least:

$\frac{1}{4}$, $\frac{2}{9}$

Therefore, the worker who received $\frac{1}{4}$ litre of cleaning solution got more than the worker who received $\frac{2}{9}$ litre.

\section*{Question 101}
\textbf{Metadata}

\begin{itemize}
  \item Question ID: P6-FrDivPN\_P2-FrAdd2nd\_GPT4.1\_Services\_03
  \item Primary KC: FRACTIONS | Division | dividing a proper fraction by a whole number
  \item Secondary KC: FRACTIONS | Addition | adding fractions
  \item Topic: Services such as installation, maintenance, repairing, cleaning, laundry, hotel, retail, e-commerce, streaming services, digital services etc.
  \item Grade: Primary 6
\end{itemize}

\textbf{Question}

A cleaning company receives an order to clean 3 identical office rooms. It takes $\frac{2}{3}$ hours to clean one room. The company has 4 workers who will share the work equally.\
\
(a) How many hours of cleaning does each worker need to do if the total cleaning time for all rooms is divided equally among the 4 workers?\
\
(b) If one worker cleans his share and then helps another worker by doing an additional $\frac{1}{4}$ hour of cleaning, what is the total amount of time this worker spends cleaning?

\textbf{Solution}

(a) First, find the total cleaning time for the 3 rooms: \\
$\frac{2}{3} \times 3 = 2$ hours\\
So, total cleaning time is 2 hours.\\
Each worker's share: \\ $2 \div 4 = \frac{2}{4} = \frac{1}{2}$ hour\\
\
\textbf{Each worker needs to clean for $\frac{1}{2}$ hour.}\\
\
(b) If one worker does his share ($\frac{1}{2}$ hour) and helps for an additional $\frac{1}{4}$ hour,\\
Total cleaning time = $\frac{1}{2} + \frac{1}{4}$\\
Rewrite $\frac{1}{2}$ as $\frac{2}{4}$ to add: $\frac{2}{4} + \frac{1}{4} = \frac{3}{4}$\\
\textbf{The worker spends a total of $\frac{3}{4}$ hour cleaning.}

\section*{Question 102}
\textbf{Metadata}

\begin{itemize}
  \item Question ID: P6-FrDivPN\_P2-FrSub2nd\_GPT4.1\_Services\_03
  \item Primary KC: FRACTIONS | Division | dividing a proper fraction by a whole number
  \item Secondary KC: FRACTIONS | Subtraction | subtracting fractions
  \item Topic: Services such as installation, maintenance, repairing, cleaning, laundry, hotel, retail, e-commerce, streaming services, digital services etc.
  \item Grade: Primary 6
\end{itemize}

\textbf{Question}

A hotel offers a laundry service to its guests. On Monday, the housekeeping staff collected $\frac{3}{4}$ kg of clothes from one guest to wash. The staff needs to divide these clothes evenly into 2 washing machines so that both have the same amount of clothes. 

(a) How many kilograms of clothes will go into each washing machine?

Later that day, another guest brings $\frac{1}{8}$ kg of clothes. After washing, the staff wants to subtract this amount from the original $\frac{3}{4}$ kg collected earlier to find out how many kilograms were initially more than the new collection. 

(b) How many more kilograms of clothes did the first guest give compared to the second guest?

\textbf{Solution}

(a) To find how many kilograms go into each washing machine, divide $\frac{3}{4}$ kg by 2:
\[
\frac{3}{4} \div 2 = \frac{3}{4} \times \frac{1}{2} = \frac{3 \times 1}{4 \times 2} = \frac{3}{8}
\]
So, each washing machine will have $\frac{3}{8}$ kg of clothes.

(b) To find how many more kilograms the first guest gave, subtract $\frac{1}{8}$ kg from $\frac{3}{4}$ kg:
\[
\frac{3}{4} - \frac{1}{8}
\]
Convert $\frac{3}{4}$ to eighths:
\[
\frac{3}{4} = \frac{6}{8}
\]
Now subtract:
\[
\frac{6}{8} - \frac{1}{8} = \frac{5}{8}
\]
So, the first guest gave $\frac{5}{8}$ kg more clothes than the second guest.

\section*{Question 103}
\textbf{Metadata}

\begin{itemize}
  \item Question ID: P6-FrDivPN\_P3-FrSmp\_GPT4.1\_Services\_08
  \item Primary KC: FRACTIONS | Division | dividing a proper fraction by a whole number
  \item Secondary KC: FRACTIONS | Simplifying | expressing a fraction in its simplest form
  \item Topic: Services such as installation, maintenance, repairing, cleaning, laundry, hotel, retail, e-commerce, streaming services, digital services etc.
  \item Grade: Primary 6
\end{itemize}

\textbf{Question}

A laundry shop charges by the basket for washing clothes. Mrs. Tan brought in $\frac{3}{4}$ basket of clothes and asked to split the load equally between 5 washing machines. 

(a) What fraction of a basket of clothes did Mrs. Tan put into each washing machine?

(b) Give your answer in its simplest form.

\textbf{Solution}

(a) To find the fraction of a basket in each machine, divide $\frac{3}{4}$ by $5$:

\[
\frac{3}{4} \div 5 = \frac{3}{4} \times \frac{1}{5} = \frac{3 \times 1}{4 \times 5} = \frac{3}{20}
\]

(b) The answer $\frac{3}{20}$ is already in its simplest form.

\textbf{Final Answer:} Each washing machine had $\frac{3}{20}$ of a basket of clothes, in its simplest form.

\section*{Question 104}
\textbf{Metadata}

\begin{itemize}
  \item Question ID: P6-FrDivPN\_P5-FrCnv2Dc\_GPT4.1\_Services\_08
  \item Primary KC: FRACTIONS | Division | dividing a proper fraction by a whole number
  \item Secondary KC: FRACTIONS | Conversion to decimals | expressing fractions as decimals
  \item Topic: Services such as installation, maintenance, repairing, cleaning, laundry, hotel, retail, e-commerce, streaming services, digital services etc.
  \item Grade: Primary 6
\end{itemize}

\textbf{Question}

A cleaning company is paid to clean a large office and will split the total cleaning work evenly among its staff. The total amount of cleaning required is $\frac{3}{4}$ of the office, and there are 5 staff members working.

(a) What fraction of the office will each staff member clean?

(b) Express the fraction of the office each staff member will clean as a decimal.

\textbf{Solution}

Let the total work be $\frac{3}{4}$ of the office, to be divided by 5 staff members.

(a) Fraction of the office each staff member cleans:

$\frac{3}{4} \div 5 = \frac{3}{4} \times \frac{1}{5} = \frac{3}{20}$

(b) Express $\frac{3}{20}$ as a decimal:

$\frac{3}{20} = \frac{3 \div 1}{20 \div 1} = 0.15$

So each staff member will clean $\frac{3}{20}$ (or $0.15$) of the office.

\section*{Question 105}
\textbf{Metadata}

\begin{itemize}
  \item Question ID: P6-FrDivPN\_P5-FrMul2nd\_GPT4.1\_Services\_03
  \item Primary KC: FRACTIONS | Division | dividing a proper fraction by a whole number
  \item Secondary KC: FRACTIONS | Multiplication | fraction multiplication
  \item Topic: Services such as installation, maintenance, repairing, cleaning, laundry, hotel, retail, e-commerce, streaming services, digital services etc.
  \item Grade: Primary 6
\end{itemize}

\textbf{Question}

A cleaning company received an order to clean 3 rooms in a hotel. The total amount of cleaning solution they have is $\frac{5}{6}$ litres, which needs to be divided equally among all 3 rooms.\
\
(a) How many litres of cleaning solution will be used in each room?\
\
(b) If each room requires $\frac{3}{4}$ hour to clean, how many total hours will the company spend cleaning all 3 rooms?

\textbf{Solution}

(a) Amount of cleaning solution per room = $\frac{5}{6} \div 3 = \frac{5}{6} \times \frac{1}{3} = \frac{5}{18}$ litres.\
\
(b) Time to clean one room = $\frac{3}{4}$ hour.\
\
Total time for 3 rooms $= 3 \times \frac{3}{4} = \frac{9}{4}$ hours $= 2\frac{1}{4}$ hours.\
\
\textbf{Answers:}\
(a) $\frac{5}{18}$ litres per room.\
(b) $2\frac{1}{4}$ hours in total.

\section*{Question 106}
\textbf{Metadata}

\begin{itemize}
  \item Question ID: P6-FrDivPP\_P2-FrCmp\_GPT4.1\_Services\_08
  \item Primary KC: FRACTIONS | Division | dividing a whole number/proper fraction by a proper fraction
  \item Secondary KC: FRACTIONS | Comparison and ordering | comparing and ordering fractions
  \item Topic: Services such as installation, maintenance, repairing, cleaning, laundry, hotel, retail, e-commerce, streaming services, digital services etc.
  \item Grade: Primary 6
\end{itemize}

\textbf{Question}

A hotel uses $\dfrac{3}{4}$ litres of cleaning solution to wash each room. On a certain day, the cleaners have $6$ litres of cleaning solution. 

(a) How many rooms can be washed with $6$ litres of cleaning solution?

(b) If one day they used $\dfrac{5}{6}$ litres of solution per room instead, could they clean more, the same, or fewer rooms with $6$ litres? Show your comparison clearly.

\textbf{Solution}

(a) Number of rooms that can be washed $= \dfrac{6}{\dfrac{3}{4}}$

To divide a whole number by a fraction, multiply by the reciprocal:

$= 6 \times \dfrac{4}{3}$
$= \dfrac{24}{3}$
$= 8$

So, $8$ rooms can be washed with $6$ litres of cleaning solution.

(b) If they use $\dfrac{5}{6}$ litres per room:

Number of rooms that can be washed $= \dfrac{6}{\dfrac{5}{6}}$
$= 6 \times \dfrac{6}{5}$
$= \dfrac{36}{5}$
$= 7.2$

They can clean $7$ full rooms with a bit of solution left over (not enough for the $8^{th}$ room).

Comparing $8$ rooms (with $\dfrac{3}{4}$ litre per room) and $7.2$ rooms (with $\dfrac{5}{6}$ litre per room):

Since $8 > 7.2$, the cleaners could wash more rooms when each room uses $\dfrac{3}{4}$ litre of solution compared to $\dfrac{5}{6}$ litre per room.

\section*{Question 107}
\textbf{Metadata}

\begin{itemize}
  \item Question ID: P6-FrDivPP\_P2-FrAdd2nd\_GPT4.1\_Services\_03
  \item Primary KC: FRACTIONS | Division | dividing a whole number/proper fraction by a proper fraction
  \item Secondary KC: FRACTIONS | Addition | adding fractions
  \item Topic: Services such as installation, maintenance, repairing, cleaning, laundry, hotel, retail, e-commerce, streaming services, digital services etc.
  \item Grade: Primary 6
\end{itemize}

\textbf{Question}

A cleaning company is hired to clean an office block. Each cleaning team can finish $\frac{2}{5}$ of a floor in one hour. If there are $3$ teams working together, how many hours will it take to clean $4$ floors in total? After they finish the $4$ floors, the owner requests that they clean an additional $\frac{3}{5}$ of a floor. How many total hours do the teams need to clean all the floors, including the additional part? Express your answer in the simplest form.

\textbf{Solution}

First, we find how much of a floor all three teams can clean in one hour:

One team's rate per hour: $\frac{2}{5}$ of a floor.
So, three teams' rate per hour: $3 \times \frac{2}{5} = \frac{6}{5}$ floors per hour.

To clean $4$ floors:
Number of hours required: $4 \div \frac{6}{5} = 4 \times \frac{5}{6} = \frac{20}{6} = \frac{10}{3}$ hours.

Next, for the additional $\frac{3}{5}$ of a floor:
Number of hours required: $\frac{3}{5} \div \frac{6}{5} = \frac{3}{5} \times \frac{5}{6} = \frac{3}{6} = \frac{1}{2}$ hour.

Total hours:
$\frac{10}{3} + \frac{1}{2}$
Find a common denominator:
LCD of $3$ and $2$ is $6$.
$\frac{10}{3} = \frac{20}{6}$
$\frac{1}{2} = \frac{3}{6}$
Total: $\frac{20}{6} + \frac{3}{6} = \frac{23}{6}$ hours

Therefore, the teams need $\boxed{\frac{23}{6}}$ hours in total to clean all the floors, including the additional part.

\section*{Question 108}
\textbf{Metadata}

\begin{itemize}
  \item Question ID: P6-FrDivPP\_P2-FrSub2nd\_GPT4.1\_Services\_03
  \item Primary KC: FRACTIONS | Division | dividing a whole number/proper fraction by a proper fraction
  \item Secondary KC: FRACTIONS | Subtraction | subtracting fractions
  \item Topic: Services such as installation, maintenance, repairing, cleaning, laundry, hotel, retail, e-commerce, streaming services, digital services etc.
  \item Grade: Primary 6
\end{itemize}

\textbf{Question}

A hotel manager has $\dfrac{5}{6}$ of a box of detergent left. Each load of laundry for the hotel's cleaning service uses $\dfrac{1}{8}$ of a box of detergent.\
\
(a) How many loads of laundry can the cleaning service complete with the detergent left?\
\
After doing 3 loads, the cleaning staff accidentally spilled $\dfrac{1}{12}$ of a box of detergent.\
\
(b) How much detergent is left now? \
\
(c) How many more loads of laundry can the cleaning service complete with the detergent that is left after the spill?

\textbf{Solution}

(a) To find the number of loads that can be completed: \
\[ \frac{5}{6} \div \frac{1}{8} = \frac{5}{6} \times \frac{8}{1} = \frac{40}{6} = \frac{20}{3} = 6\dfrac{2}{3} \] \
This means the cleaning service can complete $6$ full loads and start on the 7th load.\
\
(b) Detergent used for 3 loads: \
\[ 3 \times \frac{1}{8} = \frac{3}{8} \] \
Detergent left after 3 loads: \
\[ \frac{5}{6} - \frac{3}{8} = \frac{20}{24} - \frac{9}{24} = \frac{11}{24} \] \
After spilling $\dfrac{1}{12}$ of a box: \
\[ \frac{11}{24} - \frac{1}{12} = \frac{11}{24} - \frac{2}{24} = \frac{9}{24} = \frac{3}{8} \] \
So, $\frac{3}{8}$ of a box of detergent is left.\
\
(c) Number of more loads that can be completed: \
\[ \frac{3}{8} \div \frac{1}{8} = \frac{3}{8} \times \frac{8}{1} = 3 \] \
The cleaning service can complete 3 more loads of laundry with the detergent left after the spill.

\section*{Question 109}
\textbf{Metadata}

\begin{itemize}
  \item Question ID: P6-FrDivPP\_P3-FrSmp\_GPT4.1\_Services\_08
  \item Primary KC: FRACTIONS | Division | dividing a whole number/proper fraction by a proper fraction
  \item Secondary KC: FRACTIONS | Simplifying | expressing a fraction in its simplest form
  \item Topic: Services such as installation, maintenance, repairing, cleaning, laundry, hotel, retail, e-commerce, streaming services, digital services etc.
  \item Grade: Primary 6
\end{itemize}

\textbf{Question}

A cleaning company has a task to clean $\frac{3}{4}$ of an office building each day. If the company assigns $\frac{1}{8}$ of the building to each of their cleaning staff every day, how many cleaning staff are needed to clean $\frac{3}{4}$ of the building in one day? Give your answer in its simplest form.

\textbf{Solution}

We need to find out how many cleaning staff are needed to clean $\frac{3}{4}$ of the building if each one cleans $\frac{1}{8}$.

To find this, we divide $\frac{3}{4}$ by $\frac{1}{8}$:

\[
\frac{3}{4} \div \frac{1}{8} = \frac{3}{4} \times \frac{8}{1} = \frac{3 \times 8}{4 \times 1} = \frac{24}{4} = 6
\]

Therefore, $6$ cleaning staff are needed.

\section*{Question 110}
\textbf{Metadata}

\begin{itemize}
  \item Question ID: P6-FrDivPP\_P5-FrCnv2Dc\_GPT4.1\_Services\_08
  \item Primary KC: FRACTIONS | Division | dividing a whole number/proper fraction by a proper fraction
  \item Secondary KC: FRACTIONS | Conversion to decimals | expressing fractions as decimals
  \item Topic: Services such as installation, maintenance, repairing, cleaning, laundry, hotel, retail, e-commerce, streaming services, digital services etc.
  \item Grade: Primary 6
\end{itemize}

\textbf{Question}

A cleaning company was hired to clean 3 buildings. Each building requires $\dfrac{3}{4}$ of a day to be cleaned. The company has a team that works for $1$ day each time. 

(a) How many buildings can the team clean in $1$ day? 

(b) Express your answer in part (a) as a decimal.

\textbf{Solution}

(a) To find out how many buildings the team can clean in $1$ day, divide $1$ day by the time needed to clean $1$ building:

\[
\text{Number of buildings} = 1 \div \dfrac{3}{4} = 1 \times \dfrac{4}{3} = \dfrac{4}{3}
\]

So, the team can clean $\dfrac{4}{3}$ buildings in $1$ day.

(b) Express $\dfrac{4}{3}$ as a decimal:

\[
\dfrac{4}{3} = 1.33\ldots
\]

To 2 decimal places, this is $1.33$.

\textbf{Answer:} (a) The team can clean $\dfrac{4}{3}$ buildings in $1$ day. (b) This is $1.33$ buildings (2 decimal places).

\section*{Question 111}
\textbf{Metadata}

\begin{itemize}
  \item Question ID: P6-FrDivPP\_P5-FrMul2nd\_GPT4.1\_Services\_03
  \item Primary KC: FRACTIONS | Division | dividing a whole number/proper fraction by a proper fraction
  \item Secondary KC: FRACTIONS | Multiplication | fraction multiplication
  \item Topic: Services such as installation, maintenance, repairing, cleaning, laundry, hotel, retail, e-commerce, streaming services, digital services etc.
  \item Grade: Primary 6
\end{itemize}

\textbf{Question}

A cleaning company offers carpet-cleaning services for offices. Each worker can clean $\frac{3}{5}$ of an office floor in 1 hour. The company is hired to clean a large office floor that is 4 times as large as a regular office floor. If a team of 2 workers is assigned to work together without any breaks, how many hours will it take for them to finish cleaning the entire large office floor?

Express your answer as a fraction.

\textbf{Solution}

First, let us determine how much of a regular office floor 2 workers can clean in 1 hour:

Each worker cleans $\frac{3}{5}$ in 1 hour, so 2 workers clean:
$$
2 \times \frac{3}{5} = \frac{6}{5}
$$
This means 2 workers can clean all of a regular office floor and a bit more in 1 hour.

The large office floor is 4 times as large as a regular office floor.

Let the size of a regular office floor be 1 unit. The large office is $4$ units.

To find the number of hours needed, divide the total to be cleaned by what they can clean in 1 hour:
$$
\text{Number of hours} = \frac{4}{\frac{6}{5}} = 4 \times \frac{5}{6} = \frac{20}{6} = \frac{10}{3}
$$

So, it will take $\boxed{\frac{10}{3}}$ hours for the team of 2 workers to clean the entire large office floor.

\section*{Question 112}
\textbf{Metadata}

\begin{itemize}
  \item Question ID: P6-PcFndWN\_P1-WNAdd2nd\_GPT4.1\_Services\_03
  \item Primary KC: PERCENTAGE | Finding the whole | finding the whole given a part and the percentage
  \item Secondary KC: WHOLE NUMBERS | Addition | adding whole numbers
  \item Topic: Services such as installation, maintenance, repairing, cleaning, laundry, hotel, retail, e-commerce, streaming services, digital services etc.
  \item Grade: Primary 6
\end{itemize}

\textbf{Question}

A cleaning company charged Mei Ling \textdollar120 for cleaning her apartment. This amount was 30\% of her total cleaning expenses for the month. Later, Mei Ling paid \textdollar75 for laundry services in the same month. 

What was the total amount Mei Ling spent on cleaning and laundry for the month?

\textbf{Solution}

Let the total cleaning expenses be $x$.

We are told that 30\% of her cleaning expenses is \textdollar120, so:

$30\% \times x = 120$

$0.3x = 120$

$x = \frac{120}{0.3} = 400$

Mei Ling's total cleaning expenses was \textdollar400. 

Adding the laundry expenses:

\textdollar400 + \textdollar75 = \textdollar475

Thus, the total amount Mei Ling spent on cleaning and laundry for the month was \textdollar475.

\section*{Question 113}
\textbf{Metadata}

\begin{itemize}
  \item Question ID: P6-PcFndWN\_P1-WNSub2nd\_GPT4.1\_Services\_03
  \item Primary KC: PERCENTAGE | Finding the whole | finding the whole given a part and the percentage
  \item Secondary KC: WHOLE NUMBERS | Subtraction | subtracting whole numbers
  \item Topic: Services such as installation, maintenance, repairing, cleaning, laundry, hotel, retail, e-commerce, streaming services, digital services etc.
  \item Grade: Primary 6
\end{itemize}

\textbf{Question}

A cleaning company charges a customer for cleaning an office. After finishing the job, the company gave the customer a bill showing that the amount paid, $\textdollar150$, was only 60\% of the total amount because the customer used a discount voucher. Later, the customer realised that she should subtract the discount amount she received from the total amount before claiming the cost from her company. How much was the actual discount amount that the customer received?

\textbf{Solution}

Let the total original amount be $x$. The paid amount is 60\% of the total, so:

\[
0.6x = 150
\]

To find the whole (original price):
\[
x = \frac{150}{0.6} = 250
\]

The amount of discount is:
\[
250 - 150 = 100
\]

Therefore, the actual discount amount the customer received is $\textdollar100$.

\section*{Question 114}
\textbf{Metadata}

\begin{itemize}
  \item Question ID: P6-PcFndWN\_P1-WNMul2nd\_GPT4.1\_Services\_03
  \item Primary KC: PERCENTAGE | Finding the whole | finding the whole given a part and the percentage
  \item Secondary KC: WHOLE NUMBERS | Multiplication | multiplying whole numbers
  \item Topic: Services such as installation, maintenance, repairing, cleaning, laundry, hotel, retail, e-commerce, streaming services, digital services etc.
  \item Grade: Primary 6
\end{itemize}

\textbf{Question}

A streaming service offers a monthly family plan. Last month, 25\% of all subscribers used the family plan. This means 1,200 subscribers were on the family plan. 

(a) How many total subscribers did the service have last month?

(b) If each subscriber on the family plan pays $\textdollar12 every month, how much money did the streaming service collect from all family plan subscribers last month?

\textbf{Solution}

(a) Let the total number of subscribers be $x$.

We know that 25\% of $x$ equals 1,200.

$25\% \times x = 1,200$

$\dfrac{25}{100} \times x = 1,200$

$\dfrac{1}{4} \times x = 1,200$

$x = 1,200 \times 4$

$x = 4,800$

So, the streaming service had 4,800 subscribers last month.

(b) Each family plan subscriber pays $\textdollar12$ per month. There are 1,200 subscribers on the family plan.

Total amount collected $= 1,200 \times 12$

$= 14,400$

So, the streaming service collected $\textdollar14,400$ from family plan subscribers last month.

\section*{Question 115}
\textbf{Metadata}

\begin{itemize}
  \item Question ID: P6-PcFndWN\_P1-WNDiv2nd\_GPT4.1\_Services\_03
  \item Primary KC: PERCENTAGE | Finding the whole | finding the whole given a part and the percentage
  \item Secondary KC: WHOLE NUMBERS | Division | dividing whole numbers
  \item Topic: Services such as installation, maintenance, repairing, cleaning, laundry, hotel, retail, e-commerce, streaming services, digital services etc.
  \item Grade: Primary 6
\end{itemize}

\textbf{Question}

A cleaning company charges 20\% of the total contract amount as a deposit before starting the job. If a customer paid a deposit of \textdollar 180, how much is the total contract amount? After the cleaning is completed, the customer pays the remaining balance by dividing the amount equally over 4 monthly payments. How much does the customer pay each month?

\textbf{Solution}

Let the total contract amount be $x$.\newline
Given that the deposit is 20\% of the total amount,\newline
$\displaystyle 20\% \times x = 180$\newline
$\displaystyle \frac{20}{100} \times x = 180$\newline
$\displaystyle \frac{1}{5} \times x = 180$\newline
$\displaystyle x = 180 \times 5 = 900$\newline
\textbf{Total contract amount:} \textdollar 900.\newline
\newline
After paying the deposit of \textdollar 180, the remaining balance is:\newline
$900-180 = 720$\newline
The customer pays this balance equally over 4 months.\newline
$\displaystyle \frac{720}{4} = 180$\newline
\textbf{Monthly payment:} \textdollar 180 each month.

\section*{Question 116}
\textbf{Metadata}

\begin{itemize}
  \item Question ID: P6-PcFndChg\_P1-WNAdd2nd\_GPT4.1\_Services\_03
  \item Primary KC: PERCENTAGE | Finding change | finding percentage increase/decrease
  \item Secondary KC: WHOLE NUMBERS | Addition | adding whole numbers
  \item Topic: Services such as installation, maintenance, repairing, cleaning, laundry, hotel, retail, e-commerce, streaming services, digital services etc.
  \item Grade: Primary 6
\end{itemize}

\textbf{Question}

A steam cleaning company charges \textdollar40 to clean a sofa and \textdollar55 to clean a mattress. Last year, the company charged \textdollar80 to clean 1 sofa and 1 mattress together as a package, but this year, the package costs the total of the individual cleaning prices. 

(a) What is the total price to clean 1 sofa and 1 mattress this year?

(b) By what percentage did the package price increase from last year to this year?

\textbf{Solution}

(a) The total price to clean 1 sofa and 1 mattress this year is:

$\textdollar40 + \textdollar55 = \textdollar95$

(b) The percentage increase in the package price from last year ($\textdollar80$) to this year ($\textdollar95$) is:

Percentage increase $= \dfrac{\text{New price} - \text{Old price}}{\text{Old price}} \times 100\%$

$= \dfrac{95 - 80}{80} \times 100\%$

$= \dfrac{15}{80} \times 100\%$

$= 0.1875 \times 100\%$

$= 18.75\%$

\textbf{Final answers:}

(a) \textdollar95

(b) $18.75\%$ increase


\section*{Question 117}
\textbf{Metadata}

\begin{itemize}
  \item Question ID: P6-PcFndChg\_P1-WNSub2nd\_GPT4.1\_Services\_03
  \item Primary KC: PERCENTAGE | Finding change | finding percentage increase/decrease
  \item Secondary KC: WHOLE NUMBERS | Subtraction | subtracting whole numbers
  \item Topic: Services such as installation, maintenance, repairing, cleaning, laundry, hotel, retail, e-commerce, streaming services, digital services etc.
  \item Grade: Primary 6
\end{itemize}

\textbf{Question}

A mobile phone repair shop charged \textdollar120 for repairing a phone last year. This year, the price for the same repair increased to \textdollar150.

(a) By how much did the repair cost increase?

(b) What is the percentage increase in the repair cost from last year to this year?

\textbf{Solution}

(a) The increase in repair cost is $150 - 120 = 30$.

So, the repair cost increased by \textdollar30.

(b) To find the percentage increase:

Percentage increase $= \dfrac{\text{Increase}}{\text{Original amount}} \times 100\%$

$= \dfrac{30}{120} \times 100\%$

$= 0.25 \times 100\%$

$= 25\%$

The percentage increase in the repair cost is $25\%$.

\section*{Question 118}
\textbf{Metadata}

\begin{itemize}
  \item Question ID: P6-PcFndChg\_P1-WNMul2nd\_GPT4.1\_Services\_03
  \item Primary KC: PERCENTAGE | Finding change | finding percentage increase/decrease
  \item Secondary KC: WHOLE NUMBERS | Multiplication | multiplying whole numbers
  \item Topic: Services such as installation, maintenance, repairing, cleaning, laundry, hotel, retail, e-commerce, streaming services, digital services etc.
  \item Grade: Primary 6
\end{itemize}

\textbf{Question}

A cleaning company charges \textdollar80 to clean one office room. Last month, they cleaned 25 rooms. This month, they cleaned 30 rooms. 

(a) Find the total amount the company earned from cleaning the rooms last month.

(b) Find the total amount the company earned from cleaning the rooms this month.

(c) What is the percentage increase in the number of rooms cleaned from last month to this month?

\textbf{Solution}

(a) Last month: Number of rooms cleaned $= 25$.

Amount earned $= 25 \times 80 = 2000$.

So, the company earned $\textdollar2000$ last month.

(b) This month: Number of rooms cleaned $= 30$.

Amount earned $= 30 \times 80 = 2400$.

So, the company earned $\textdollar2400$ this month.

(c) The increase in number of rooms cleaned $= 30 - 25 = 5$.

Percentage increase $= \dfrac{5}{25} \times 100\% = 20\%$.

So, the percentage increase in the number of rooms cleaned is $20\%$.

\section*{Question 119}
\textbf{Metadata}

\begin{itemize}
  \item Question ID: P6-PcFndChg\_P1-WNDiv2nd\_GPT4.1\_Services\_03
  \item Primary KC: PERCENTAGE | Finding change | finding percentage increase/decrease
  \item Secondary KC: WHOLE NUMBERS | Division | dividing whole numbers
  \item Topic: Services such as installation, maintenance, repairing, cleaning, laundry, hotel, retail, e-commerce, streaming services, digital services etc.
  \item Grade: Primary 6
\end{itemize}

\textbf{Question}

A streaming service has 900 subscribers in January. By June, the number of subscribers increased to 1,170. The company wants to reward customer support teams based on the increase in subscribers. If the increase in the number of subscribers is to be divided equally among 5 support teams, how many additional subscribers will each team be credited for? What is the percentage increase in the number of subscribers from January to June?

\textbf{Solution}

First, we find the increase in the number of subscribers:

\[
\text{Increase} = 1,170 - 900 = 270
\]

Divide this increase equally among 5 support teams:

\[
\text{Subscribers per team} = \frac{270}{5} = 54
\]

Next, we calculate the percentage increase:

\[
\text{Percentage increase} = \frac{\text{Increase}}{\text{Original number}} \times 100\% = \frac{270}{900} \times 100\% = 30\% 
\]

\textbf{Answers:} Each support team will be credited for 54 additional subscribers, and the percentage increase in the number of subscribers from January to June is $30\%$. 

\section*{Question 120}
\textbf{Metadata}

\begin{itemize}
  \item Question ID: P6-RoFndDvqWN\_P1-WNAdd2nd\_GPT4.1\_Services\_01
  \item Primary KC: RATIO | Finding divided quantities | dividing a given quantity in a given ratio
  \item Secondary KC: WHOLE NUMBERS | Addition | adding whole numbers
  \item Topic: Services such as installation, maintenance, repairing, cleaning, laundry, hotel, retail, e-commerce, streaming services, digital services etc.
  \item Grade: Primary 6
\end{itemize}

\textbf{Question}

A hotel is distributing \textdollar1200 among its housekeeping, reception, and maintenance departments as a reward for good service. The money is to be divided among the three departments in the ratio $5:3:2$. After the housekeeping department receives its share, the hotel decides to add an extra \textdollar200 to their amount as a bonus. 

What is the final amount that the housekeeping department receives?

\textbf{Solution}

Step 1: Find the total number of shares.

$5 + 3 + 2 = 10$ shares

Step 2: Find the value of one share.

One share $= \dfrac{\textdollar1200}{10} = \textdollar120$

Step 3: Calculate the amount for the housekeeping department (5 shares).

$5 \times \textdollar120 = \textdollar600$

Step 4: Add the bonus.

$\textdollar600 + \textdollar200 = \textdollar800$

\boxed{\textdollar800}$ is the final amount that the housekeeping department receives.

\section*{Question 121}
\textbf{Metadata}

\begin{itemize}
  \item Question ID: P6-RoFndDvqWN\_P1-WNSub2nd\_GPT4.1\_Services\_01
  \item Primary KC: RATIO | Finding divided quantities | dividing a given quantity in a given ratio
  \item Secondary KC: WHOLE NUMBERS | Subtraction | subtracting whole numbers
  \item Topic: Services such as installation, maintenance, repairing, cleaning, laundry, hotel, retail, e-commerce, streaming services, digital services etc.
  \item Grade: Primary 6
\end{itemize}

\textbf{Question}

A cleaning company received a payment of $\textdollar540$ for cleaning two office buildings. The total amount was to be divided between two teams, Team A and Team B, in the ratio $5:4$. However, Team A had to pay $\textdollar30$ for cleaning supplies out of their share, while Team B did not have any extra expenses. 

(a) How much money did each team receive before any expenses?

(b) After Team A pays for the cleaning supplies, how much money does Team A have left?

\textbf{Solution}

(a) The total payment is $\textdollar540$, to be divided in the ratio $5:4$.

Total parts = $5 + 4 = 9$

Each part is $\frac{\textdollar540}{9} = \textdollar60$

Team A's share: $5 \times \textdollar60 = \textdollar300$

Team B's share: $4 \times \textdollar60 = \textdollar240$

So, Team A receives $\textdollar300$ and Team B receives $\textdollar240$.

(b) Team A pays $\textdollar30$ for cleaning supplies.

Amount left for Team A: $\textdollar300 - \textdollar30 = \textdollar270$

\textbf{Answer:}

(a) Team A receives $\textdollar300$ and Team B receives $\textdollar240$ before any expenses.

(b) After paying for the cleaning supplies, Team A has $\textdollar270$ left.

\section*{Question 122}
\textbf{Metadata}

\begin{itemize}
  \item Question ID: P6-RoFndRoWN\_P1-WNAdd2nd\_GPT4.1\_Services\_03
  \item Primary KC: RATIO | Finding ratio | finding the ratio of two or three given whole numbers
  \item Secondary KC: WHOLE NUMBERS | Addition | adding whole numbers
  \item Topic: Services such as installation, maintenance, repairing, cleaning, laundry, hotel, retail, e-commerce, streaming services, digital services etc.
  \item Grade: Primary 6
\end{itemize}

\textbf{Question}

A cleaning company received three contracts on Monday for hotel room cleaning. Contract A required cleaning $18$ rooms, Contract B required cleaning $27$ rooms, and Contract C required cleaning $15$ rooms. 

(a) Find the ratio of the number of rooms cleaned for Contract A to Contract B to Contract C. 

(b) What is the total number of rooms cleaned on that day?

\textbf{Solution}

(a) The number of rooms for Contract A, B, and C are $18$, $27$, and $15$ respectively. 

To find the ratio:
$18:27:15$
Divide each number by their highest common factor, which is $3$.

$\dfrac{18}{3} : \dfrac{27}{3} : \dfrac{15}{3} = 6:9:5$

So, the ratio is $6:9:5$.

(b) To find the total number of rooms cleaned:
$18 + 27 + 15 = 60$

So, the total number of rooms cleaned is $60$.

\section*{Question 123}
\textbf{Metadata}

\begin{itemize}
  \item Question ID: P6-RoFndRoWN\_P1-WNSub2nd\_GPT4.1\_Services\_03
  \item Primary KC: RATIO | Finding ratio | finding the ratio of two or three given whole numbers
  \item Secondary KC: WHOLE NUMBERS | Subtraction | subtracting whole numbers
  \item Topic: Services such as installation, maintenance, repairing, cleaning, laundry, hotel, retail, e-commerce, streaming services, digital services etc.
  \item Grade: Primary 6
\end{itemize}

\textbf{Question}

A cleaning company had 60 cleaning cloths. After completing several jobs, the workers found that 18 cloths were too worn out to be used further. The remaining cloths were either blue or white. There were 24 blue cloths. Find the ratio of blue cloths to white cloths among the remaining cloths.

\textbf{Solution}

First, find the number of remaining cloths after subtracting the worn-out ones:

$60 - 18 = 42$ (remaining cloths)

Next, since there are 24 blue cloths, the rest must be white:

$42 - 24 = 18$ (white cloths)

So, the ratio of blue cloths to white cloths is:

$24:18$

This ratio can be simplified by dividing both numbers by 6:

$24 \div 6 = 4$

$18 \div 6 = 3$

Therefore, the simplest ratio of blue cloths to white cloths is $4:3$. 

\section*{Question 124}
\textbf{Metadata}

\begin{itemize}
  \item Question ID: P6-RoFndRoWN\_P1-WNMul2nd\_GPT4.1\_Services\_03
  \item Primary KC: RATIO | Finding ratio | finding the ratio of two or three given whole numbers
  \item Secondary KC: WHOLE NUMBERS | Multiplication | multiplying whole numbers
  \item Topic: Services such as installation, maintenance, repairing, cleaning, laundry, hotel, retail, e-commerce, streaming services, digital services etc.
  \item Grade: Primary 6
\end{itemize}

\textbf{Question}

A cleaning company offers three different packages: Basic, Standard, and Premium. In one week, the company cleaned 18 houses using the Basic package, 36 houses using the Standard package, and 54 houses using the Premium package. 

(a) What is the ratio of the number of houses cleaned using the Basic package to the number of houses cleaned using the Standard package to the number of houses cleaned using the Premium package?

Each cleaning for the Premium package costs $\textdollar65$.

(b) What is the total amount collected from all the houses cleaned using the Premium package that week?

\textbf{Solution}

(a) The numbers of houses cleaned are 18 (Basic), 36 (Standard), and 54 (Premium).

To find the ratio, we write:
$18:36:54$

Divide each term by the greatest common divisor, which is 18:

$\frac{18}{18} : \frac{36}{18} : \frac{54}{18} = 1 : 2 : 3$

So, the ratio is $1:2:3$.

(b) Number of houses cleaned using Premium package = $54$

Each Premium cleaning costs $\textdollar65$.

Total amount collected:

$54 \times 65 = 3510$

So, the total amount collected from Premium package is $\textdollar3510$.

\section*{Question 125}
\textbf{Metadata}

\begin{itemize}
  \item Question ID: P6-RoFndRoWN\_P1-WNDiv2nd\_GPT4.1\_Services\_03
  \item Primary KC: RATIO | Finding ratio | finding the ratio of two or three given whole numbers
  \item Secondary KC: WHOLE NUMBERS | Division | dividing whole numbers
  \item Topic: Services such as installation, maintenance, repairing, cleaning, laundry, hotel, retail, e-commerce, streaming services, digital services etc.
  \item Grade: Primary 6
\end{itemize}

\textbf{Question}

A cleaning company was hired to clean 3 floors of an office building. The company assigned 18 workers to clean the 1st floor, 12 workers to clean the 2nd floor, and 6 workers to clean the 3rd floor.

(a) Find the ratio of the number of workers cleaning the 1st floor to the number of workers cleaning the 2nd floor to the number of workers cleaning the 3rd floor.

(b) If the company wants to distribute a total bonus of \textdollar360 equally among all the workers, how much bonus will each worker receive?

\textbf{Solution}

(a) The number of workers on the 1st floor is $18$, on the 2nd floor is $12$, and on the 3rd floor is $6$.

The ratio is $18 : 12 : 6$.

To simplify, divide each number by $6$ (the highest common factor):
\[
18 \div 6 = 3\\
12 \div 6 = 2\\
6 \div 6 = 1
\]
So, the simplest ratio is $3 : 2 : 1$.

(b) Total number of workers = $18 + 12 + 6 = 36$ workers.

Each worker receives:
\[
\text{Amount per worker} = \frac{\textdollar360}{36} = \textdollar10
\]

\textbf{Each worker will receive \textdollar10.}

\section*{Question 126}
\textbf{Metadata}

\begin{itemize}
  \item Question ID: P6-RoFndRoWN\_P6-RoSmpWN\_GPT4.1\_Services\_08
  \item Primary KC: RATIO | Finding ratio | finding the ratio of two or three given whole numbers
  \item Secondary KC: RATIO | Simplifying | expressing a ratio in its simplest form
  \item Topic: Services such as installation, maintenance, repairing, cleaning, laundry, hotel, retail, e-commerce, streaming services, digital services etc.
  \item Grade: Primary 6
\end{itemize}

\textbf{Question}

Anna manages a laundry shop. One morning, she received 18 shirts, 12 pairs of trousers, and 9 jackets to be washed.

(a) Find the ratio of the number of shirts to the number of trousers to the number of jackets.

(b) Express this ratio in its simplest form.

\textbf{Solution}

(a) The ratio of the number of shirts to the number of trousers to the number of jackets is $18 : 12 : 9$.

(b) To express this ratio in its simplest form, find the highest common factor (HCF) of 18, 12, and 9.

The HCF of 18, 12, and 9 is $3$.

$18 \div 3 = 6$

$12 \div 3 = 4$

$9 \div 3 = 3$

Therefore, the simplest form of the ratio is $6 : 4 : 3$.

\section*{Question 127}
\textbf{Metadata}

\begin{itemize}
  \item Question ID: P6-RoFndTmWN\_P1-WNAdd2nd\_GPT4.1\_Services\_01
  \item Primary KC: RATIO | Finding a missing term | finding the missing term in a pair of equivalent ratios
  \item Secondary KC: WHOLE NUMBERS | Addition | adding whole numbers
  \item Topic: Services such as installation, maintenance, repairing, cleaning, laundry, hotel, retail, e-commerce, streaming services, digital services etc.
  \item Grade: Primary 6
\end{itemize}

\textbf{Question}

A cleaning company provides two different service packages. In the Standard Package, the ratio of the number of hours spent cleaning to the number of workers is $3:2$. In a certain week, the Standard Package was booked for a total of $12$ hours by adding together two customer bookings. If the company assigns $x$ workers for these $12$ hours so that the same ratio is maintained, how many workers, $x$, are needed?

\textbf{Solution}

The ratio for the Standard Package is $3:2$ (hours : workers). This means for every $3$ hours of cleaning, $2$ workers are assigned.\
\
First, find the equivalent number of workers for $12$ hours of cleaning.\
\
Set up the proportion:\\
\[\frac{3}{2} = \frac{12}{x}\]
\
Cross-multiply to solve for $x$:\\
$3x = 2 \times 12$\\
$3x = 24$\\
$x = \frac{24}{3}$\\
$x = 8$\\
\
Therefore, $\boxed{8}$ workers are needed for $12$ hours of cleaning to maintain the same ratio.

\section*{Question 128}
\textbf{Metadata}

\begin{itemize}
  \item Question ID: P6-RoFndTmWN\_P1-WNSub2nd\_GPT4.1\_Services\_01
  \item Primary KC: RATIO | Finding a missing term | finding the missing term in a pair of equivalent ratios
  \item Secondary KC: WHOLE NUMBERS | Subtraction | subtracting whole numbers
  \item Topic: Services such as installation, maintenance, repairing, cleaning, laundry, hotel, retail, e-commerce, streaming services, digital services etc.
  \item Grade: Primary 6
\end{itemize}

\textbf{Question}

A laundry service uses a specific ratio of detergent to water for washing clothes. For every $3$ parts of detergent, they mix $15$ parts of water. During a busy day, the service used $24$ parts of detergent. 

However, $7$ parts of water were accidentally spilled before mixing. How much water was finally mixed with the $24$ parts of detergent to maintain the same ratio after accounting for the spillage?

\textbf{Solution}

First, we find the total amount of water needed to keep the same ratio when $24$ parts of detergent are used.

The original ratio of detergent to water is $3:15$. This can be simplified to $1:5$.

So, for every $1$ part of detergent, $5$ parts of water are needed.

If $24$ parts of detergent are used:
\[
\text{Water needed} = 24 \times 5 = 120 \text{ parts}
\]

However, $7$ parts of water were spilled.

So, the amount of water that should be mixed after the spillage is:
\[
120 - 7 = 113 \text{ parts}
\]

\boxed{113} parts of water were finally mixed with $24$ parts of detergent to maintain the same ratio.

\section*{Question 129}
\textbf{Metadata}

\begin{itemize}
  \item Question ID: P6-AgRepLrEx\_P6-AgSmpLrEx\_GPT4.1\_Services\_08
  \item Primary KC: ALGEBRA | Representation and concept | translation of real-world situations into linear algebraic expressions
  \item Secondary KC: ALGEBRA | Simplifying | simplifying linear expressions
  \item Topic: Services such as installation, maintenance, repairing, cleaning, laundry, hotel, retail, e-commerce, streaming services, digital services etc.
  \item Grade: Primary 6
\end{itemize}

\textbf{Question}

A company charges a basic fee of $\textdollar15$ to clean a house, plus $\textdollar8$ for each room cleaned. If $n$ is the number of rooms to be cleaned, write an algebraic expression for the total cost of cleaning the house. Then, simplify the expression if possible.

\textbf{Solution}

Let $n$ be the number of rooms to be cleaned.

The total cost for cleaning the house is given by:

\[
\text{Total cost} = \text{basic fee} + (\text{cost per room} \times \text{number of rooms})
\]

Substituting the given values:
\[
\text{Total cost} = \textdollar15 + (\textdollar8 \times n)
\]

This gives the algebraic expression:
\[
\textdollar15 + \textdollar8n
\]

This expression is already in its simplest form.

\section*{Question 130}
\textbf{Metadata}

\begin{itemize}
  \item Question ID: P6-AgRepLrEx\_P6-AgEvlLrEx\_GPT4.1\_Services\_08
  \item Primary KC: ALGEBRA | Representation and concept | translation of real-world situations into linear algebraic expressions
  \item Secondary KC: ALGEBRA | Evaluation | evaluating linear expressions by substitution
  \item Topic: Services such as installation, maintenance, repairing, cleaning, laundry, hotel, retail, e-commerce, streaming services, digital services etc.
  \item Grade: Primary 6
\end{itemize}

\textbf{Question}

A cleaning company charges a fixed fee of $\textdollar15$ for each house cleaning, plus $\textdollar8$ for every hour they spend cleaning. 

(a) Write a linear algebraic expression to represent the total cost, $C$, for cleaning a house for $h$ hours.

(b) If a customer needs their house cleaned for $3$ hours, how much will the cleaning company charge?

\textbf{Solution}

(a) The total cost, $C$, consists of a fixed fee of $\textdollar15$ and an hourly fee of $\textdollar8$ for each hour. So,

$$C = 15 + 8h$$

where $h$ is the number of hours.

(b) Substituting $h = 3$ into the expression:

$$C = 15 + 8 \times 3$$
$$C = 15 + 24$$
$$C = 39$$

The cleaning company will charge $\textdollar39$ if the cleaning takes $3$ hours.

\section*{Question 131}
\textbf{Metadata}

\begin{itemize}
  \item Question ID: P6-AgSlvLrN\_P6-AgRepLrEx\_GPT4.1\_Services\_08
  \item Primary KC: ALGEBRA | Solving simple linear equations | solving linear equations involving whole number coefficient and one variable only
  \item Secondary KC: ALGEBRA | Representation and concept | translation of real-world situations into linear algebraic expressions
  \item Topic: Services such as installation, maintenance, repairing, cleaning, laundry, hotel, retail, e-commerce, streaming services, digital services etc.
  \item Grade: Primary 6
\end{itemize}

\textbf{Question}

A carpet cleaning company charges a fixed fee of $\textdollar 30$ plus $\textdollar 8$ for each room cleaned. Mr Lim paid a total of $\textdollar 94$ to have his house cleaned. 

How many rooms did Mr Lim get cleaned? Let $x$ represent the number of rooms Mr Lim got cleaned. Write an equation to represent this situation and solve for $x$.

\textbf{Solution}

Let $x$ be the number of rooms cleaned.

The total amount paid is:

$\textdollar 30$ (fixed fee) $+ \textdollar 8 \times x$ (for $x$ rooms) $= \textdollar 94$

Write the equation:

$30 + 8x = 94$

Subtract $30$ from both sides:

$8x = 94 - 30$

$8x = 64$

Divide both sides by $8$:

$x = \dfrac{64}{8}$

$x = 8$

Therefore, Mr Lim got $8$ rooms cleaned.

\section*{Question 132}
\textbf{Metadata}

\begin{itemize}
  \item Question ID: O1-FDSub\_O1-FDAdd\_GPT4.1\_Services\_04
  \item Primary KC: FRACTIONS AND DECIMALS | Subtraction | Subtraction involving one fraction and one decimal number
  \item Secondary KC: FRACTIONS AND DECIMALS | Addition | Addition involving one fraction and one decimal number
  \item Topic: Services such as installation, maintenance, repairing, cleaning, laundry, hotel, retail, e-commerce, streaming services, digital services etc.
  \item Grade: Secondary O-level 1
\end{itemize}

\textbf{Question}

A hotel offers a weekend room cleaning service. For one weekend, a guest used the minibar in the room and was charged $0.75 for snacks. The guest also requested additional laundry service, which cost $\textdollar0.5$ (represented as the fraction $\frac{1}{2}$ dollar). At the end of the stay, the hotel deducted a $0.25$ discount (as a decimal) from the total additional service charges. 

What is the total amount the guest had to pay for the additional services after the discount? Show your working clearly.

\textbf{Solution}

First, we add the cost of the minibar snacks and the laundry service:

\[
0.75 + \frac{1}{2}
\]

Converting $\frac{1}{2}$ to decimal gives $0.5$.

\[
0.75 + 0.5 = 1.25
\]

Next, the hotel deducts a $0.25$ discount:

\[
1.25 - 0.25 = 1.00
\]

Therefore, the total amount the guest had to pay for the additional services after the discount is $\textdollar1.00$.

\section*{Question 133}
\textbf{Metadata}

\begin{itemize}
  \item Question ID: O1-FDMul\_O1-FDAdd\_GPT4.1\_Services\_04
  \item Primary KC: FRACTIONS AND DECIMALS | Multiplication | Multiplication involving one fraction and one decimal number
  \item Secondary KC: FRACTIONS AND DECIMALS | Addition | Addition involving one fraction and one decimal number
  \item Topic: Services such as installation, maintenance, repairing, cleaning, laundry, hotel, retail, e-commerce, streaming services, digital services etc.
  \item Grade: Secondary O-level 1
\end{itemize}

\textbf{Question}

A cleaning company charges $\textdollar2.50$ per square metre to clean carpets. After cleaning, they offer to polish $\frac{1}{4}$ of the total area cleaned for an extra $0.80$ per square metre. If a customer has a room with $36$ square metres of carpet, what is the total cost for cleaning and then polishing $\frac{1}{4}$ of the area? Give your answer in dollars, using two decimal places where necessary.

\textbf{Solution}

First, find the total area cleaned: $36$ m$^2$.

Cleaning cost:
$36 \times \textdollar2.50 = \textdollar90$

Area to be polished:
$\frac{1}{4} \times 36 = 9$ m$^2$

Polishing cost:
$9 \times \textdollar0.80 = \textdollar7.20$

Total cost:
$\textdollar90 + \textdollar7.20 = \textdollar97.20$

\boxed{\textdollar97.20}$ is the total cost for cleaning and polishing.

\section*{Question 134}
\textbf{Metadata}

\begin{itemize}
  \item Question ID: O1-FDMul\_O1-FDSub\_GPT4.1\_Services\_04
  \item Primary KC: FRACTIONS AND DECIMALS | Multiplication | Multiplication involving one fraction and one decimal number
  \item Secondary KC: FRACTIONS AND DECIMALS | Subtraction | Subtraction involving one fraction and one decimal number
  \item Topic: Services such as installation, maintenance, repairing, cleaning, laundry, hotel, retail, e-commerce, streaming services, digital services etc.
  \item Grade: Secondary O-level 1
\end{itemize}

\textbf{Question}

Kevin is working at a digital music store. He earns a commission based on the number of albums he sells online each week. Last week, he sold $0.8$ times as many albums as $rac{3}{5}$ of his usual weekly sales. If his usual weekly sales are $125$ albums, how many albums did he sell last week? 

After selling these albums, Kevin gave away $0.2$ albums (in decimal) to a friend and returned $rac{1}{4}$ album due to a customer complaint. How many more albums did Kevin have after giving away and returning the albums, compared to $0.2$ albums subtracted from his total sales?

Give your answer as a decimal.

\textbf{Solution}

Kevin's usual weekly sales = $125$ albums. 
He sold $0.8 \times \frac{3}{5}$ of $125$ albums last week.

First, find $\frac{3}{5}$ of $125$:

$\frac{3}{5} \times 125 = \frac{3 \times 125}{5} = \frac{375}{5} = 75$.

Next, find $0.8$ times $75$:

$0.8 \times 75 = 60$.

So, Kevin sold $60$ albums last week.

He gave away $0.2$ albums and returned $\frac{1}{4}$ album. The total number he lost = $0.2 + \frac{1}{4}$.

Convert $\frac{1}{4}$ to decimal: $\frac{1}{4} = 0.25$.
Add $0.2 + 0.25 = 0.45$.

Albums Kevin had after giving away and returning = 
$60 - 0.45 = 59.55$.

Compared to only giving away $0.2$ albums:
$60 - 0.2 = 59.8$.

Difference:
$59.8 - 59.55 = 0.25$.

\textbf{Final Answer:} He had $0.25$ more albums after only giving away $0.2$ albums than after both giving away $0.2$ and returning $\frac{1}{4}$ album.

\section*{Question 135}
\textbf{Metadata}

\begin{itemize}
  \item Question ID: O1-FDDiv\_O1-FDAdd\_GPT4.1\_Services\_04
  \item Primary KC: FRACTIONS AND DECIMALS | Division | Division involving one fraction and one decimal number
  \item Secondary KC: FRACTIONS AND DECIMALS | Addition | Addition involving one fraction and one decimal number
  \item Topic: Services such as installation, maintenance, repairing, cleaning, laundry, hotel, retail, e-commerce, streaming services, digital services etc.
  \item Grade: Secondary O-level 1
\end{itemize}

\textbf{Question}

A cleaning company uses a special disinfectant to clean hotel rooms. For each room, the supervisor estimates that $0.75$ litres of disinfectant is needed. The company has $\dfrac{7}{2}$ litres of disinfectant left in stock.\
(a) How many hotel rooms can be fully cleaned with the remaining disinfectant?\
(b) After cleaning those rooms, the company decides to order an additional $1.2$ litres of disinfectant. What is the new total amount of disinfectant (in litres) the company will have after the order arrives?

\textbf{Solution}

(a) The number of hotel rooms that can be fully cleaned is calculated by dividing the total amount of disinfectant by the amount needed per room:\\
\\
$\displaystyle \text{Number of rooms} = \dfrac{\dfrac{7}{2}}{0.75}$\\
First, express $0.75$ as a fraction: $0.75 = \dfrac{3}{4}$\\
So,\\
$\displaystyle \dfrac{7}{2} \div \dfrac{3}{4} = \dfrac{7}{2} \times \dfrac{4}{3} = \dfrac{7 \times 4}{2 \times 3} = \dfrac{28}{6} = \dfrac{14}{3} \approx 4.666...$\\
Therefore, the company can fully clean $4$ hotel rooms (since partial rooms cannot be fully cleaned).\\
\\
(b) Amount of disinfectant left after cleaning $4$ rooms:\\
$\displaystyle \text{Used disinfectant} = 4 \times 0.75 = 3.00$ litres\\
$\displaystyle \text{Disinfectant remaining} = \dfrac{7}{2} - 3.00 = 3.5 - 3.0 = 0.5$ litres\\
After ordering an additional $1.2$ litres:\\
$\displaystyle \text{New total amount} = 0.5 + 1.2 = 1.7$ litres\\
\\
\boxed{\text{The new total amount of disinfectant the company will have is } 1.7 \text{ litres.}}

\section*{Question 136}
\textbf{Metadata}

\begin{itemize}
  \item Question ID: O1-FDDiv\_O1-FDSub\_GPT4.1\_Services\_04
  \item Primary KC: FRACTIONS AND DECIMALS | Division | Division involving one fraction and one decimal number
  \item Secondary KC: FRACTIONS AND DECIMALS | Subtraction | Subtraction involving one fraction and one decimal number
  \item Topic: Services such as installation, maintenance, repairing, cleaning, laundry, hotel, retail, e-commerce, streaming services, digital services etc.
  \item Grade: Secondary O-level 1
\end{itemize}

\textbf{Question}

A cleaning company charges $\textdollar18$ to clean one room. One week, Mrs. Tan paid $\frac{3}{4}$ of the regular price to have her living room cleaned, as there was a special discount. After paying for the discounted living room cleaning, Mrs. Tan bought an air freshener that costs $\textdollar4.20$ and used it in her living room. How much more did Mrs. Tan spend on cleaning and freshening her living room compared to what she would have spent on a standard cleaning at the regular price?

\textbf{Solution}

First, find the discounted price Mrs. Tan paid for the cleaning:

$\text{Discounted cleaning price} = \frac{3}{4} \times 18 = \frac{3 \times 18}{4} = \frac{54}{4} = 13.50$

Add the cost of the air freshener:

$\text{Total spent} = 13.50 + 4.20 = 17.70$

Now, find the difference between the total spent and the regular cleaning price:

$\text{Difference} = 17.70 - 18 = -0.30$

So Mrs. Tan spent $\textdollar0.30$ less than the regular cleaning price. 

Answer: Mrs. Tan spent $\textdollar0.30$ less than the regular cleaning price for cleaning and freshening her living room.

\section*{Question 137}
\textbf{Metadata}

\begin{itemize}
  \item Question ID: O1-FDDiv\_O1-FDMul\_GPT4.1\_Services\_04
  \item Primary KC: FRACTIONS AND DECIMALS | Division | Division involving one fraction and one decimal number
  \item Secondary KC: FRACTIONS AND DECIMALS | Multiplication | Multiplication involving one fraction and one decimal number
  \item Topic: Services such as installation, maintenance, repairing, cleaning, laundry, hotel, retail, e-commerce, streaming services, digital services etc.
  \item Grade: Secondary O-level 1
\end{itemize}

\textbf{Question}

Laundry Express charges $\textdollar2.40$ to wash $\frac{3}{5}$ of a kilogram of clothes. 

(a) Mandy wants to wash a total of $2.4$ kilograms of clothes. How much would Laundry Express charge her for washing the $2.4$ kilograms?

(b) If Mandy decides to wash $1.5$ times the amount of clothes she planned in part (a), how much would she need to pay?

\textbf{Solution}

(a) First, we need to find the charge for $1$ kilogram. We know the charge for $\frac{3}{5}$ kg is $\textdollar2.40$.

The charge for $1$ kg:

\[
\text{Charge for } 1 \text{ kg} = \frac{2.40}{\frac{3}{5}} = 2.40 \times \frac{5}{3} = 4.00
\]

So, Laundry Express charges $\textdollar4.00$ per kilogram.

For $2.4$ kg of clothes:

\[
\text{Total Cost} = 2.4 \times 4.00 = \textdollar9.60
\]

(b) Mandy now wants to wash $1.5$ times $2.4$ kg:

\[
1.5 \times 2.4 = 3.6 \text{ kg}
\]

Cost for $3.6$ kg:

\[
3.6 \times 4.00 = \textdollar14.40
\]

\textbf{Answers:}

(a) Laundry Express would charge Mandy $\textdollar9.60$ for washing $2.4$ kg of clothes.

(b) Mandy would need to pay $\textdollar14.40$ if she washes $3.6$ kg of clothes.


\section*{Question 138}
\textbf{Metadata}

\begin{itemize}
  \item Question ID: O1-RoRepFr\_P2-FrAdd2nd\_GPT4.1\_Services\_03
  \item Primary KC: RATIO | Representation and concept | ratios involving fractions
  \item Secondary KC: FRACTIONS | Addition | adding fractions
  \item Topic: Services such as installation, maintenance, repairing, cleaning, laundry, hotel, retail, e-commerce, streaming services, digital services etc.
  \item Grade: Secondary O-level 1
\end{itemize}

\textbf{Question}

A cleaning company provides two types of cleaning services: basic cleaning and deep cleaning. On Monday, the ratio of the number of hours spent on basic cleaning to the number of hours spent on deep cleaning was $\frac{3}{5} : \frac{2}{5}$. If the team worked a total of $9\frac{1}{2}$ hours on Monday, how many hours did they spend on deep cleaning? Show your working clearly.

\textbf{Solution}

Let the number of hours spent on basic cleaning be $\frac{3}{5}x$ and the number of hours spent on deep cleaning be $\frac{2}{5}x$. 

Since the total number of hours worked is $9\frac{1}{2}$,
\[
\frac{3}{5}x + \frac{2}{5}x = 9\frac{1}{2}
\]
\[
\left(\frac{3}{5} + \frac{2}{5}\right)x = 9\frac{1}{2}
\]
\[
1x = 9\frac{1}{2}
\]
So,
\[
x = 9\frac{1}{2} = \frac{19}{2}
\]

The number of hours spent on deep cleaning is $\frac{2}{5}x$:
\[
\frac{2}{5} \times \frac{19}{2} = \frac{19}{5} = 3\frac{4}{5}
\]

\textbf{Answer: The team spent $3\frac{4}{5}$ hours on deep cleaning.}

\section*{Question 139}
\textbf{Metadata}

\begin{itemize}
  \item Question ID: O1-RoRepFr\_P2-FrSub2nd\_GPT4.1\_Services\_03
  \item Primary KC: RATIO | Representation and concept | ratios involving fractions
  \item Secondary KC: FRACTIONS | Subtraction | subtracting fractions
  \item Topic: Services such as installation, maintenance, repairing, cleaning, laundry, hotel, retail, e-commerce, streaming services, digital services etc.
  \item Grade: Secondary O-level 1
\end{itemize}

\textbf{Question}

A cleaning company provides weekend cleaning services for offices. The ratio of offices cleaned using environmentally-friendly cleaning agents to those cleaned with regular agents is $\frac{2}{3}:1$. Last weekend, the company cleaned a total of $24$ offices. After completing the cleaning, $\frac{1}{6}$ of the offices cleaned with environmentally-friendly agents had to be re-cleaned due to client complaints.

(a) How many offices were cleaned using environmentally-friendly agents?

(b) After subtracting the number of offices that needed re-cleaning, how many offices remained that were cleaned with environmentally-friendly agents and did not require re-cleaning?

\textbf{Solution}

(a) Let the number of offices cleaned with environmentally-friendly agents be $x$ and those cleaned with regular agents be $y$.

We are told the total number of offices is $24$, so
$$
x + y = 24
$$

The ratio is $\frac{2}{3}$ to $1$, which means:
$$
\frac{x}{y} = \frac{2}{3}
$$

Cross-multiply:
$$
3x = 2y
$$

So $y = \frac{3x}{2}$.

Substitute into the total:
$$
x + \frac{3x}{2} = 24
$$
$$
\frac{2x + 3x}{2} = 24
$$
$$
\frac{5x}{2} = 24
$$
$$
5x = 48
$$
$$
x = 9.6
$$

However, since we are dealing with whole offices, let's adjust the ratio. Rewrite the ratio in whole numbers:
$$
\frac{2}{3}:1 = 2:3
$$

Let $2$ parts be for environmentally-friendly agents and $3$ parts for regular agents. Total parts $= 2 + 3 = 5$.

One part $= \frac{24}{5} = 4.8$

Number of offices cleaned with environmentally-friendly agents:
$$
2 \times 4.8 = 9.6
$$

Since we rarely clean a fraction of an office, let's reinterpret the ratio. Assume the problem meant $2:3$ instead of involving fractions as parts checked above, or else it would not be possible to have a realistic answer with whole numbers. But to explicitly use the given $\frac{2}{3}:1$ ratio:
Let $y$ be the number of offices cleaned with regular agents.
Then, environmentally-friendly agents: $x = \frac{2}{3} y$.
Total: $x + y = 24$
So:
$$
\frac{2}{3}y + y = 24
$$
$$
\frac{5}{3}y = 24
$$
$$
y = 24 \times \frac{3}{5} = 14.4
$$
So $x = \frac{2}{3} \times 14.4 = 9.6$

Still, this gives fractions for the number of offices. For the sake of working with the knowledge component as required, we proceed:

**Answer to (a):**
The number of offices cleaned using environmentally-friendly agents is $9.6$.

(b) The number of offices that needed re-cleaning:
$$
\frac{1}{6} \times 9.6 = 1.6
$$

The number of offices that did not require re-cleaning:
$$
9.6 - 1.6 = 8
$$

**Final Answers:**
(a) $9.6$ offices were cleaned using environmentally-friendly agents.

(b) $8$ offices remained that were cleaned with environmentally-friendly agents and did not require re-cleaning.

\section*{Question 140}
\textbf{Metadata}

\begin{itemize}
  \item Question ID: O1-RoRepFr\_P5-FrMul2nd\_GPT4.1\_Services\_03
  \item Primary KC: RATIO | Representation and concept | ratios involving fractions
  \item Secondary KC: FRACTIONS | Multiplication | fraction multiplication
  \item Topic: Services such as installation, maintenance, repairing, cleaning, laundry, hotel, retail, e-commerce, streaming services, digital services etc.
  \item Grade: Secondary O-level 1
\end{itemize}

\textbf{Question}

A cleaning company uses two types of cleaning solutions to disinfect hotel rooms. The company mixes Solution A and Solution B in the ratio $\dfrac{2}{3} : 1$ by volume. If the company needs a total of $4\dfrac{1}{2}$ litres of cleaning mixture to clean a suite and Solution A is sold in bottles of $\dfrac{3}{4}$ litres each, how many bottles of Solution A are needed to prepare enough mixture for the suite?

\textbf{Solution}

Let the amount of Solution A needed be $a$ litres and Solution B be $b$ litres. 

Given the ratio is $\dfrac{2}{3} : 1$, we represent the parts as $\dfrac{2}{3}x$ for Solution A and $1x$ for Solution B, for some $x$. 

Total mixture = $\dfrac{2}{3}x + 1x = \dfrac{5}{3}x$.

Set this equal to the total required: $\dfrac{5}{3}x = 4\dfrac{1}{2} = \dfrac{9}{2}$.

Solve for $x$:
$$
\dfrac{5}{3}x = \dfrac{9}{2}\\
x = \dfrac{9}{2} \times \dfrac{3}{5} = \dfrac{27}{10}
$$

Amount of Solution A needed:
$$
\text{Solution A} = \dfrac{2}{3}x = \dfrac{2}{3} \times \dfrac{27}{10} = \dfrac{54}{30} = \dfrac{9}{5} \,\text{litres}
$$

Each bottle contains $\dfrac{3}{4}$ litres, so the number of bottles needed is:
$$
\dfrac{9}{5} \div \dfrac{3}{4} = \dfrac{9}{5} \times \dfrac{4}{3} = \dfrac{36}{15} = \dfrac{12}{5} = 2.4
$$
The company needs $3$ bottles of Solution A (since we cannot buy a fraction of a bottle and must round up).

\textbf{Answer:} The company needs $3$ bottles of Solution A to prepare enough mixture for the suite.

\section*{Question 141}
\textbf{Metadata}

\begin{itemize}
  \item Question ID: O1-RoRepFr\_P6-FrDiv2nd\_GPT4.1\_Services\_03
  \item Primary KC: RATIO | Representation and concept | ratios involving fractions
  \item Secondary KC: FRACTIONS | Division | fraction division
  \item Topic: Services such as installation, maintenance, repairing, cleaning, laundry, hotel, retail, e-commerce, streaming services, digital services etc.
  \item Grade: Secondary O-level 1
\end{itemize}

\textbf{Question}

A hotel is changing the cleaning liquid used by its housekeeping staff. The staff currently use a mixture where the ratio of cleaning liquid to water is $\frac{3}{4}:1$. For a batch, the housekeeping staff need $12\ \text{litres}$ of water. 

(a) How many litres of cleaning liquid do they need?

(b) If one container of cleaning liquid holds $2\ \text{litres}$, how many containers are needed to prepare this mixture for six rooms, if the amount of mixture used is the same for each room?

\textbf{Solution}

(a) The ratio of cleaning liquid to water is $\frac{3}{4}:1$, meaning for every $1$ litre of water, they use $\frac{3}{4}$ litre of cleaning liquid. 

For $12$ litres of water, the litres of cleaning liquid needed is:
\[
12 \times \frac{3}{4} = 9
\]

So, they need $9$ litres of cleaning liquid.

(b) Total amount of mixture needed for one batch (for one room) = $12$ litres of water $+ 9$ litres of cleaning liquid $= 21$ litres.

For six rooms: $21 \times 6 = 126$ litres$ $.

Total cleaning liquid needed for six rooms: $9 \times 6 = 54$ litres.

Each container has $2$ litres, so the number of containers required is:
\[
\frac{54}{2} = 27
\]

So, they need $27$ containers of cleaning liquid to prepare the mixture for six rooms.

\section*{Question 142}
\textbf{Metadata}

\begin{itemize}
  \item Question ID: O1-RoRepFr\_O1-RoSmpFr\_GPT4.1\_Services\_04
  \item Primary KC: RATIO | Representation and concept | ratios involving fractions
  \item Secondary KC: RATIO | Simplifying | converting a ratio involving fractions to its simplest form
  \item Topic: Services such as installation, maintenance, repairing, cleaning, laundry, hotel, retail, e-commerce, streaming services, digital services etc.
  \item Grade: Secondary O-level 1
\end{itemize}

\textbf{Question}

A hotel offers two types of bath towel sizes for its guests: large and small. The ratio of the number of large towels to the number of small towels used in one week is $\frac{3}{4} : \frac{5}{6}$. 

(a) Express the ratio of large towels to small towels in its simplest form.

(b) If there were 48 large towels used that week, how many small towels were used in the same week?

\textbf{Solution}

(a) The given ratio is $\frac{3}{4} : \frac{5}{6}$. 

To simplify, write as a fraction:
\[
\frac{\frac{3}{4}}{\frac{5}{6}} = \frac{3}{4} \times \frac{6}{5} = \frac{3 \times 6}{4 \times 5} = \frac{18}{20}
\]

Now, simplify $\frac{18}{20}$ by dividing numerator and denominator by 2:
\[
\frac{18}{20} = \frac{9}{10}
\]

Therefore, the simplest form of the ratio is $9:10$.

(b) Let the number of small towels be $x$.

Since $9:10 = 48:x$, set up the proportion:
\[
\frac{9}{10} = \frac{48}{x}
\]

Cross-multiply:
\[
9x = 480
\]
\[
x = \frac{480}{9} = 53.33
\]

Since the number of towels must be whole, check the context: The original ratio $\frac{3}{4} : \frac{5}{6}$ means for every $\frac{3}{4}$ large there are $\frac{5}{6}$ small towels, and $48$ is a multiple of $\frac{3}{4}$.

Let $k$ be the multiplication constant:
\[
Large = \frac{3}{4}k = 48 \implies k = 48 \div \frac{3}{4} = 48 \times \frac{4}{3} = 64
\]

So, number of small towels:
\[
\frac{5}{6} \times 64 = \frac{320}{6} = 53.33
\]

However, since the fraction answers do not give whole numbers, clarify that the answer is $53\frac{1}{3}$ towels, but since towels can't be cut, the exact ratio represents an average or a scaled value. In most cases, state the answer as 53 towels if rounding down is required.

Final Answer:

(a) $9:10$

(b) $53$ (to the nearest whole towel)

\section*{Question 143}
\textbf{Metadata}

\begin{itemize}
  \item Question ID: O1-RoRepDc\_P4-DcAdd2nd\_GPT4.1\_Services\_03
  \item Primary KC: RATIO | Representation and concept | ratios involving decimals
  \item Secondary KC: DECIMALS | Addition | adding decimals
  \item Topic: Services such as installation, maintenance, repairing, cleaning, laundry, hotel, retail, e-commerce, streaming services, digital services etc.
  \item Grade: Secondary O-level 1
\end{itemize}

\textbf{Question}

A cleaning company uses two different detergents to clean hotel rooms. For every $2.5$ litres of Detergent A used, they use $3.75$ litres of Detergent B. Over the weekend, the staff used a total of $12.5$ litres of Detergent A and $18.75$ litres of Detergent B. 

(a) Find the ratio of the amount of Detergent A used to the amount of Detergent B used over the weekend, in its simplest form.

(b) Find the total amount of detergent (in litres) the company used over the weekend.

\textbf{Solution}

(a) The amount of Detergent A used : amount of Detergent B used = $12.5 : 18.75$.

To simplify, divide both terms by $12.5$:
\[
\frac{12.5}{12.5} : \frac{18.75}{12.5} = 1 : 1.5
\]
Express $1 : 1.5$ as a ratio in whole numbers by multiplying both terms by 2:
\[
1 \times 2 : 1.5 \times 2 = 2 : 3
\]
So, the ratio is $2 : 3$.

(b) The total amount of detergent used $= 12.5 + 18.75 = 31.25$ litres.

\textbf{Answers:}
(a) $2 : 3$

(b) $31.25$ litres.

\section*{Question 144}
\textbf{Metadata}

\begin{itemize}
  \item Question ID: O1-RoRepDc\_P4-DcSub2nd\_GPT4.1\_Services\_03
  \item Primary KC: RATIO | Representation and concept | ratios involving decimals
  \item Secondary KC: DECIMALS | Subtraction | subtracting decimals
  \item Topic: Services such as installation, maintenance, repairing, cleaning, laundry, hotel, retail, e-commerce, streaming services, digital services etc.
  \item Grade: Secondary O-level 1
\end{itemize}

\textbf{Question}

A cleaning company offers a special promotion for office cleaning services. The time spent cleaning each of three rooms (Room A, Room B, and Room C) is in the ratio of $1.5 : 2.5 : 3$. If the total cleaning time for all three rooms is $14.8$ hours, how many hours were spent cleaning Room B? After finishing Room B, the supervisor notes that $0.7$ hours were spent waiting for equipment, so the actual cleaning time spent in Room B was less than recorded. What is the actual cleaning time used for Room B after subtracting the waiting time?

\textbf{Solution}

Let the cleaning time for Room A be $1.5x$, for Room B be $2.5x$, and for Room C be $3x$.

\[
1.5x + 2.5x + 3x = 14.8
\]
\[
(1.5 + 2.5 + 3)x = 14.8
\]
\[
7x = 14.8
\]
\[
x = \frac{14.8}{7} = 2.1142857\ \, (rounded \ to \ 4 \ decimal \ places)
\]

Time spent on Room B:
\[
2.5x = 2.5 \times 2.1143 = 5.2857 \text{ hours } (rounded \ to \ 4 \ decimal \ places)
\]

After subtracting the $0.7$ hours spent waiting:
\[
5.2857 - 0.7 = 4.5857 \text{ hours } (rounded \ to \ 4 \ decimal \ places)
\]

\textbf{Answer:} The recorded cleaning time for Room B was $5.29$ hours (rounded to $2$ decimal places), and the actual cleaning time used for Room B after subtracting the waiting time was $4.59$ hours.

\section*{Question 145}
\textbf{Metadata}

\begin{itemize}
  \item Question ID: O1-RoRepDc\_O1-RoSmpDc\_GPT4.1\_Services\_04
  \item Primary KC: RATIO | Representation and concept | ratios involving decimals
  \item Secondary KC: RATIO | Simplifying | converting a ratio involving decimals to its simplest form
  \item Topic: Services such as installation, maintenance, repairing, cleaning, laundry, hotel, retail, e-commerce, streaming services, digital services etc.
  \item Grade: Secondary O-level 1
\end{itemize}

\textbf{Question}

In a hotel laundry service, the staff washed $6.5$ kg of towels and $13$ kg of bedsheets in one load. 

(a) Write the ratio of the mass of towels to the mass of bedsheets washed.

(b) Express this ratio in its simplest form.

\textbf{Solution}

(a) The mass of towels to bedsheets is $6.5 : 13$.

(b) To simplify $6.5 : 13$, divide both terms by $6.5$:

\[
\frac{6.5}{6.5} : \frac{13}{6.5} = 1 : 2
\]

So, the simplified ratio is $1 : 2$.

\section*{Question 146}
\textbf{Metadata}

\begin{itemize}
  \item Question ID: O1-PcRep2q\_O1-PcCnv2Fr\_GPT4.1\_Services\_04
  \item Primary KC: PERCENTAGE | Representation and concept | comparing two quantities by percentage
  \item Secondary KC: PERCENTAGE | Conversion to fraction | expressing percentage as a fraction
  \item Topic: Services such as installation, maintenance, repairing, cleaning, laundry, hotel, retail, e-commerce, streaming services, digital services etc.
  \item Grade: Secondary O-level 1
\end{itemize}

\textbf{Question}

A digital streaming service had 320,000 subscribers last year. This year, the number of subscribers increased by 25\%. 

(a) By what percentage is this year's subscriber count higher than last year's subscriber count?

(b) Express this 25\% increase as a fraction in its simplest form.

\textbf{Solution}

(a) The percentage increase is already directly given as 25\%. 

The number of subscribers this year $= 320,000 + 25\% \text{ of } 320,000 = 320,000 + \frac{25}{100} \times 320,000 = 320,000 + 80,000 = 400,000$. 

So, this year's subscriber count is 25\% higher than last year's subscriber count.

(b) To express 25\% as a fraction:

$25\% = \frac{25}{100} = \frac{1}{4}$

So, a 25\% increase is the same as an increase by $\frac{1}{4}$ of last year's subscribers.

\section*{Question 147}
\textbf{Metadata}

\begin{itemize}
  \item Question ID: O1-PcRep2q\_O1-PcCnv2Dc\_GPT4.1\_Services\_04
  \item Primary KC: PERCENTAGE | Representation and concept | comparing two quantities by percentage
  \item Secondary KC: PERCENTAGE | Conversion to decimals | expressing percentage as a decimal
  \item Topic: Services such as installation, maintenance, repairing, cleaning, laundry, hotel, retail, e-commerce, streaming services, digital services etc.
  \item Grade: Secondary O-level 1
\end{itemize}

\textbf{Question}

An online streaming service has 1,200 subscribers in January. By June, the number of subscribers increased to 1,350.

(a) Express the increase in the number of subscribers from January to June as a percentage of the number of subscribers in January.

(b) Express your answer in (a) as a decimal.

\textbf{Solution}

(a) The increase in number of subscribers from January to June is $1,350 - 1,200 = 150$.

The percentage increase is:
\[
\frac{150}{1,200} \times 100\% = 12.5\%
\]

(b) Expressing $12.5\%$ as a decimal:
\[
12.5\% = \frac{12.5}{100} = 0.125
\]
So, the answer in decimal form is $0.125$.

\section*{Question 148}
\textbf{Metadata}

\begin{itemize}
  \item Question ID: O1-PcFndRslt\_P1-WNAdd2nd\_GPT4.1\_Services\_03
  \item Primary KC: PERCENTAGE | Finding result after change | increasing/decreasing a quantity by a given percentage
  \item Secondary KC: WHOLE NUMBERS | Addition | adding whole numbers
  \item Topic: Services such as installation, maintenance, repairing, cleaning, laundry, hotel, retail, e-commerce, streaming services, digital services etc.
  \item Grade: Secondary O-level 1
\end{itemize}

\textbf{Question}

A laundry service charges $\textdollar25$ to wash a set of clothes. During the school holiday promotion, they increase the charge by $12\%$. On the same day, a customer brings in 3 sets of clothes to be washed. 

What is the total amount the customer has to pay for all the sets of clothes after the price increase?

\textbf{Solution}

First, find the increased price for one set of clothes.

Increase = $12\%$ of $\textdollar25 = \frac{12}{100} \times 25 = \textdollar3$.

New price per set = $\textdollar25 + \textdollar3 = \textdollar28$.

The customer brings 3 sets of clothes. Total amount paid = $3 \times \textdollar28 = \textdollar84$.

Hence, the customer has to pay $\textdollar84$ in total.

\section*{Question 149}
\textbf{Metadata}

\begin{itemize}
  \item Question ID: O1-PcFndRslt\_P1-WNSub2nd\_GPT4.1\_Services\_03
  \item Primary KC: PERCENTAGE | Finding result after change | increasing/decreasing a quantity by a given percentage
  \item Secondary KC: WHOLE NUMBERS | Subtraction | subtracting whole numbers
  \item Topic: Services such as installation, maintenance, repairing, cleaning, laundry, hotel, retail, e-commerce, streaming services, digital services etc.
  \item Grade: Secondary O-level 1
\end{itemize}

\textbf{Question}

A company offers an annual subscription for a digital streaming service at \textdollar120 per year. This year, they decide to increase the subscription fee by 15\%. 

(a) What is the new annual subscription fee after the increase? 

(b) If a customer decides not to renew and cancels the subscription, how much less does the company earn compared to if the customer continued at the new rate?

\textbf{Solution}

(a) To find the new subscription fee after a 15\% increase:

Increase = 15\% of \textdollar120 = $\frac{15}{100} \times 120 = 18$

New subscription fee = Original fee $+ \text{Increase} = 120 + 18 = \textdollar138$

(b) If a customer cancels the subscription, the company earns \textdollar0 instead of \textdollar138:

Difference = \textdollar138 - \textdollar0 = \textdollar138

The company earns \textdollar138 less if the customer cancels the subscription.

\section*{Question 150}
\textbf{Metadata}

\begin{itemize}
  \item Question ID: O1-PcFndRslt\_P1-WNMul2nd\_GPT4.1\_Services\_03
  \item Primary KC: PERCENTAGE | Finding result after change | increasing/decreasing a quantity by a given percentage
  \item Secondary KC: WHOLE NUMBERS | Multiplication | multiplying whole numbers
  \item Topic: Services such as installation, maintenance, repairing, cleaning, laundry, hotel, retail, e-commerce, streaming services, digital services etc.
  \item Grade: Secondary O-level 1
\end{itemize}

\textbf{Question}

A laundry service charges \textdollar5 to wash one shirt. During a promotion, they increased their price by 20\%. After the price increase, Sarah sent 8 shirts for washing. How much did Sarah pay in total for washing her 8 shirts during the promotion?

\textbf{Solution}

First, we need to find the new price for washing one shirt after a 20\% increase.\newline

Original price of washing 1 shirt $= \textdollar5$\newline
Increase $= 20\% \times \textdollar5 = \frac{20}{100} \times 5 = \textdollar1$\newline

New price per shirt $= \textdollar5 + \textdollar1 = \textdollar6$\newline

Sarah washed 8 shirts, so:\newline
Total amount paid $= 8 \times \textdollar6 = \textdollar48$\newline

\textbf{Sarah paid \textdollar48 in total for washing her 8 shirts during the promotion.}

\section*{Question 151}
\textbf{Metadata}

\begin{itemize}
  \item Question ID: O1-PcFndRslt\_P1-WNDiv2nd\_GPT4.1\_Services\_03
  \item Primary KC: PERCENTAGE | Finding result after change | increasing/decreasing a quantity by a given percentage
  \item Secondary KC: WHOLE NUMBERS | Division | dividing whole numbers
  \item Topic: Services such as installation, maintenance, repairing, cleaning, laundry, hotel, retail, e-commerce, streaming services, digital services etc.
  \item Grade: Secondary O-level 1
\end{itemize}

\textbf{Question}

A digital streaming service increased its monthly subscription price by $15\%$. The service previously charged $\textdollar80$ per month. After the price increase, a school wants to sign up for a group account for 12 teachers.\
\
(a) What is the new monthly subscription price after the increase? \\
(b) How much does each teacher need to pay per month if the cost is divided equally among them?

\textbf{Solution}

(a) The monthly price is increased by $15\%$.\\
Amount of increase $= 15\% \times \textdollar80 = \frac{15}{100} \times 80 = 12$ \\
New monthly subscription price $= \textdollar80 + \textdollar12 = \textdollar92$\\
\\
(b) Amount each teacher pays per month $= \frac{\textdollar92}{12} = \textdollar7.67$\\
\\
Therefore, the new monthly subscription price is $\textdollar92$ and each teacher pays $\textdollar7.67$ per month.

\section*{Question 152}
\textbf{Metadata}

\begin{itemize}
  \item Question ID: O1-PcRepRvs\_O1-PcCnv2Fr\_GPT4.1\_Services\_04
  \item Primary KC: PERCENTAGE | Representation and concept | reverse percentages
  \item Secondary KC: PERCENTAGE | Conversion to fraction | expressing percentage as a fraction
  \item Topic: Services such as installation, maintenance, repairing, cleaning, laundry, hotel, retail, e-commerce, streaming services, digital services etc.
  \item Grade: Secondary O-level 1
\end{itemize}

\textbf{Question}

A hotel is having a special discount on its standard room rate. After offering a discount of $20\%$, a guest paid \textdollar160$ for a night's stay. What was the original price of the room before the discount?

Express the discount percentage as a fraction in your working.

\textbf{Solution}

Let the original price of the room be $x$ dollars.

The guest paid \textdollar160$ after a $20\%$ discount, so the guest paid $100\% - 20\% = 80\%$ of the original price.

First, express $20\%$ as a fraction:

$20\% = \dfrac{20}{100} = \dfrac{1}{5}$

Similarly, $80\% = 1 - \dfrac{1}{5} = \dfrac{4}{5}$

So, \dfrac{4}{5}x = 160$

To find $x$, divide both sides by $\dfrac{4}{5}$:

$x = 160 \div \dfrac{4}{5} = 160 \times \dfrac{5}{4} = 40 \times 5 = 200$

\textbf{Therefore, the original price of the room before the discount was \textdollar200$.}

\section*{Question 153}
\textbf{Metadata}

\begin{itemize}
  \item Question ID: O1-PcRepRvs\_O1-PcCnv2Dc\_GPT4.1\_Services\_04
  \item Primary KC: PERCENTAGE | Representation and concept | reverse percentages
  \item Secondary KC: PERCENTAGE | Conversion to decimals | expressing percentage as a decimal
  \item Topic: Services such as installation, maintenance, repairing, cleaning, laundry, hotel, retail, e-commerce, streaming services, digital services etc.
  \item Grade: Secondary O-level 1
\end{itemize}

\textbf{Question}

A digital streaming service announced a 20\% discount on its annual subscription price during a special promotion. After the discount, the subscription costs \textdollar79.20. What was the original price of the annual subscription before the discount? Express your answer clearly, showing how to convert the percentage to a decimal when calculating.

\textbf{Solution}

Let the original price of the subscription be $x$ dollars.

The price after a 20\% discount is 80\% of the original price:
\[ 80\% = 0.8 \text{ (converted from percentage to decimal)} \]

So,
\[ 0.8x = 79.20 \]
\[ x = \frac{79.20}{0.8} \]
\[ x = 99 \]

Therefore, the original price of the annual subscription was \textdollar99.

\section*{Question 154}
\textbf{Metadata}

\begin{itemize}
  \item Question ID: O1-AgRepExSq\_O1-AgEvlEx\_GPT4.1\_Services\_04
  \item Primary KC: ALGEBRA | Representation and concept | translation of simple real-world situations into quadratic algebraic expressions
  \item Secondary KC: ALGEBRA | Evaluation | evaluation of algebraic expressions and formulae
  \item Topic: Services such as installation, maintenance, repairing, cleaning, laundry, hotel, retail, e-commerce, streaming services, digital services etc.
  \item Grade: Secondary O-level 1
\end{itemize}

\textbf{Question}

A cleaning company offers a package for office cleaning. The total cost, $C$, in \textdollar is given by the formula $C = 5x^2 + 20x + 100$, where $x$ is the number of cleaning visits in a month.

(a) Write down a quadratic algebraic expression that represents the total cost for $x$ visits.

(b) If a company requires 3 visits per month, calculate the total amount it would need to pay the cleaning company.

\textbf{Solution}

(a) The quadratic algebraic expression that represents the total cost for $x$ visits is $C = 5x^2 + 20x + 100$.

(b) Substitute $x = 3$ into the expression:
\begin{align*}
C &= 5(3)^2 + 20(3) + 100 \\
  &= 5(9) + 60 + 100 \\
  &= 45 + 60 + 100 \\
  &= 205
\end{align*}
The total amount the company needs to pay is \textdollar205.

\section*{Question 155}
\textbf{Metadata}

\begin{itemize}
  \item Question ID: O1-AgRepnth\_O1-AgEvlEx\_GPT4.1\_Services\_04
  \item Primary KC: ALGEBRA | Representation and concept | recognising and representing patterns/relationships by finding an algebraic expression for the nth term
  \item Secondary KC: ALGEBRA | Evaluation | evaluation of algebraic expressions and formulae
  \item Topic: Services such as installation, maintenance, repairing, cleaning, laundry, hotel, retail, e-commerce, streaming services, digital services etc.
  \item Grade: Secondary O-level 1
\end{itemize}

\textbf{Question}

A streaming service company offers two types of subscription packages: Basic and Premium. The Basic package charges a flat rate of $\textdollar15$ per month, while the Premium package charges a joining fee of $\textdollar20$ plus $\textdollar12$ for each month. Let $n$ represent the number of months a customer subscribes.

(a) Write an algebraic expression that represents the total cost, $C$, of subscribing to the Basic package for $n$ months.

(b) Write an algebraic expression for the total cost, $P$, of subscribing to the Premium package for $n$ months.

(c) Find the total amount paid for each package if a customer uses the service for 6 months.


\textbf{Solution}

(a) The total cost for the Basic package for $n$ months is:
\[
C = 15n
\]

(b) The total cost for the Premium package for $n$ months is:
\[
P = 20 + 12n
\]

(c) When $n = 6$:
\[
C = 15 \times 6 = 90
\]
So, the total cost for the Basic package is $\textdollar90$.

\[
P = 20 + 12 \times 6 = 20 + 72 = 92
\]
So, the total cost for the Premium package is $\textdollar92$.

\section*{Question 156}
\textbf{Metadata}

\begin{itemize}
  \item Question ID: O1-AgSlvFrLr\_O1-AgRepEq\_GPT4.1\_Services\_04
  \item Primary KC: ALGEBRA | Solving | solving simple fractional equations that can be reduced to linear equations
  \item Secondary KC: ALGEBRA | Representation and concept | translation of simple real-world situations to equations
  \item Topic: Services such as installation, maintenance, repairing, cleaning, laundry, hotel, retail, e-commerce, streaming services, digital services etc.
  \item Grade: Secondary O-level 1
\end{itemize}

\textbf{Question}

A digital music streaming service offers a monthly subscription plan. Cindy pays $\textdollar x$ per month for her subscription. Last month, she received a special promotion that gave her one-third off the usual monthly price, and she only paid $\textdollar 10$ that month. Write an equation to represent this situation, and find the usual monthly subscription price, $x$.

\textbf{Solution}

Let the usual monthly price be $x$.

Cindy received one-third off the usual price, so she paid $\dfrac{2}{3}$ of the usual price.

So, $\dfrac{2}{3}x = 10$.

To solve for $x$, multiply both sides by $\dfrac{3}{2}$:

$\begin{align*}
\dfrac{2}{3}x &= 10 \\
\Rightarrow x &= 10 \times \dfrac{3}{2} \\
&= 15
\end{align*}$

Therefore, the usual monthly subscription price is $\textdollar 15$.

\section*{Question 157}
\textbf{Metadata}

\begin{itemize}
  \item Question ID: O2-RoRepDP\_P1-WNMul2nd\_GPT4.1\_Services\_03
  \item Primary KC: RATIO | Representation and concept | direct proportion
  \item Secondary KC: WHOLE NUMBERS | Multiplication | multiplying whole numbers
  \item Topic: Services such as installation, maintenance, repairing, cleaning, laundry, hotel, retail, e-commerce, streaming services, digital services etc.
  \item Grade: Secondary O-level 2
\end{itemize}

\textbf{Question}

A cleaning company charges its customers based on the area cleaned. If the cost to clean 50 square metres is $\textdollar250$, how much would it cost to clean an office that is 84 square metres? Assume the cost is directly proportional to the area cleaned.

\textbf{Solution}

The cost is directly proportional to the area cleaned. 

Let $x$ be the cost to clean 84 square metres.

First, find the cost per square metre:

\[
\text{Cost per square metre} = \frac{\textdollar250}{50} = \textdollar5\text{ per square metre}
\]

Now, multiply by the total area to be cleaned:

\[
x = 84 \times \textdollar5 = \textdollar420
\]

Thus, the cost to clean 84 square metres is $\textdollar420$.

\section*{Question 158}
\textbf{Metadata}

\begin{itemize}
  \item Question ID: O2-RoRepDP\_P1-WNDiv2nd\_GPT4.1\_Services\_03
  \item Primary KC: RATIO | Representation and concept | direct proportion
  \item Secondary KC: WHOLE NUMBERS | Division | dividing whole numbers
  \item Topic: Services such as installation, maintenance, repairing, cleaning, laundry, hotel, retail, e-commerce, streaming services, digital services etc.
  \item Grade: Secondary O-level 2
\end{itemize}

\textbf{Question}

A cleaning company charges its clients based on the number of hours worked. The cost is directly proportional to the number of hours the cleaners work. If the company charges \textdollar120 for 8 hours of cleaning,\text{ how much would it charge for } 15 \text{ hours of cleaning?} 

After cleaning, the total bill is to be equally shared among 5 households that booked the service together. 

How much does each household need to pay?

\textbf{Solution}

Let the cost be $C$ and the number of hours be $h$.

Since the cost is directly proportional to the number of hours,

\[
C = k \times h
\]

where $k$ is the constant of proportionality.

Given for 8 hours:

\[
120 = k \times 8
\Rightarrow k = \frac{120}{8} = 15
\]

Cost for 15 hours:

\[
C = 15 \times 15 = \textdollar225
\]

This total bill of \textdollar225 is to be shared equally among 5 households.

Amount each household pays:
\[
\frac{225}{5} = \textdollar45
\]

\textbf{Final Answer:}

The company would charge \textdollar225 for 15 hours of cleaning, and each household would need to pay \textdollar45.

\section*{Question 159}
\textbf{Metadata}

\begin{itemize}
  \item Question ID: O2-RoRepIvP\_P1-WNMul2nd\_GPT4.1\_Services\_03
  \item Primary KC: RATIO | Representation and concept | inverse proportion
  \item Secondary KC: WHOLE NUMBERS | Multiplication | multiplying whole numbers
  \item Topic: Services such as installation, maintenance, repairing, cleaning, laundry, hotel, retail, e-commerce, streaming services, digital services etc.
  \item Grade: Secondary O-level 2
\end{itemize}

\textbf{Question}

A hotel hires a certain number of cleaning staff to prepare rooms before guests arrive. If 6 staff members can clean all the rooms in 4 hours, how many hours would it take for 8 staff members to clean the same number of rooms, assuming they work at the same rate? Then, if each staff member earns $\textdollar12$ per hour, calculate the total amount the hotel pays for the cleaning service when 8 staff members do the job.

\textbf{Solution}

Let the number of hours required by 8 staff members be $h$.

Since the number of staff and the time taken are in inverse proportion (more staff means less time), we use:
\[
6 \times 4 = 8 \times h
\]
\[
24 = 8h
\]
\[
h = \frac{24}{8} = 3
\]
So, 8 staff members will take 3 hours.

Each staff member earns $\textdollar12$ per hour, so the total amount paid is:
\[
8 \times 3 \times 12 = 24 \times 12 = 288
\]
Therefore, the hotel pays $\textdollar288$ for the cleaning service when 8 staff members do the job.

\section*{Question 160}
\textbf{Metadata}

\begin{itemize}
  \item Question ID: O2-RoRepIvP\_P1-WNDiv2nd\_GPT4.1\_Services\_03
  \item Primary KC: RATIO | Representation and concept | inverse proportion
  \item Secondary KC: WHOLE NUMBERS | Division | dividing whole numbers
  \item Topic: Services such as installation, maintenance, repairing, cleaning, laundry, hotel, retail, e-commerce, streaming services, digital services etc.
  \item Grade: Secondary O-level 2
\end{itemize}

\textbf{Question}

A cleaning company has a contract to clean \( 24 \) identical hotel rooms. When \( 8 \) cleaners are assigned to the job, they can finish cleaning all the rooms in \( 6 \) hours. The manager decides to assign a different number of cleaners such that the cleaning can be finished in only \( 4 \) hours. 

(a) Assuming each cleaner works at the same rate, how many cleaners need to be assigned for the rooms to be cleaned in \( 4 \) hours?

(b) If the \( 24 \) rooms are divided equally among the new number of cleaners from part (a), how many rooms will each cleaner be responsible for?

\textbf{Solution}

(a) Let the number of cleaners required to clean all rooms in \( 4 \) hours be \( x \).

Since the total amount of work (cleaning \( 24 \) rooms) does not change and the rate of working is constant, the situation is an example of inverse proportion:

\[
\text{Number of cleaners} \times \text{Time taken} = \text{Constant}
\]

Set up the relation:
\[
8 \times 6 = x \times 4
\]
\[
48 = 4x
\]
\[
x = \frac{48}{4} = 12
\]

\textbf{Answer to (a):} \( 12 \) cleaners are needed to finish the job in \( 4 \) hours.

(b) If \( 24 \) rooms are divided equally among \( 12 \) cleaners,
\[
\text{Rooms per cleaner} = \frac{24}{12} = 2
\]

\textbf{Answer to (b):} Each cleaner will be responsible for \( 2 \) rooms.

\section*{Question 161}
\textbf{Metadata}

\begin{itemize}
  \item Question ID: O2-AgSlvIneq\_O2-AgRepIneq\_GPT4.1\_Services\_03
  \item Primary KC: ALGEBRA | Solving | solving simple linear inequalities with one variable
  \item Secondary KC: ALGEBRA | Representation and concept | translation of simple real-world situations to simple linear inequalities with one variable
  \item Topic: Services such as installation, maintenance, repairing, cleaning, laundry, hotel, retail, e-commerce, streaming services, digital services etc.
  \item Grade: Secondary O-level 2
\end{itemize}

\textbf{Question}

Sarah is considering subscribing to a music streaming service that charges a one-time registration fee of $\textdollar12$ and a monthly subscription fee of $\textdollarx$ per month. She wants to sign up for the service for 6 months and her total expenses should not exceed $\textdollar60$. 

(a) Formulate a linear inequality in $x$ to represent this situation. 

(b) Solve the inequality to find the maximum possible value of $x$ that Sarah can afford to pay per month.

\textbf{Solution}

(a) The total cost for 6 months is the registration fee plus 6 times the monthly fee:
\[
\text{Total cost} = 12 + 6x
\]
She wants this amount to be at most $\textdollar60$:
\[
12 + 6x \leq 60
\]

(b) To find the maximum possible value of $x$,
\[
12 + 6x \leq 60 \\
6x \leq 60 - 12 \\
6x \leq 48 \\
x \leq 8
\]

Therefore, the maximum monthly subscription fee that Sarah can afford is $\textdollar8$.

\section*{Question 162}
\textbf{Metadata}

\begin{itemize}
  \item Question ID: O2-AgSlvSq1v\_O1-AgRepEq\_GPT4.1\_Services\_03
  \item Primary KC: ALGEBRA | Solving | solving quadratic equations in one variable
  \item Secondary KC: ALGEBRA | Representation and concept | translation of simple real-world situations to equations
  \item Topic: Services such as installation, maintenance, repairing, cleaning, laundry, hotel, retail, e-commerce, streaming services, digital services etc.
  \item Grade: Secondary O-level 2
\end{itemize}

\textbf{Question}

A hotel offers a weekend package where each room rented out earns the hotel \textdollar120 per night. However, the hotel incurs a fixed overhead cost of \textdollar2000 for the weekend and spends \textdollar10 for cleaning per room every night. After a certain number of rooms are rented each night for two nights, the total profit made by the hotel is zero. 

Let $x$ be the number of rooms rented each night for the weekend. 

(a) Write down an equation in $x$ to represent the above situation.  
(b) Solve the equation to find the number of rooms rented each night that results in zero profit for the hotel.

\textbf{Solution}

(a) The total revenue for 2 nights is $2 \times 120x = 240x$. 

The total cleaning cost for 2 nights is $2 \times 10x = 20x$. 

Total cost = Overhead + Cleaning = $2000 + 20x$. 

Profit = Revenue - Total cost = $240x - (2000 + 20x) = 240x - 2000 - 20x = 220x - 2000$. 

Given that the profit is zero:

$220x - 2000 = 0$

(b) $220x = 2000$

$x = \frac{2000}{220} = \frac{1000}{110} = \frac{100}{11} \approx 9.09$

Since the number of rooms must be a whole number, the hotel needs to rent 10 rooms each night to make a slight profit, or exactly 9 rooms to have a small loss. But for zero profit, 

\boxed{x = 9.09} (which means renting 10 rooms would ensure at least no loss)

\section*{Question 163}
\textbf{Metadata}

\begin{itemize}
  \item Question ID: O2-AgSlvLr2v\_O1-AgRepEq\_GPT4.1\_Services\_03
  \item Primary KC: ALGEBRA | Solving | solving linear equations in two variables
  \item Secondary KC: ALGEBRA | Representation and concept | translation of simple real-world situations to equations
  \item Topic: Services such as installation, maintenance, repairing, cleaning, laundry, hotel, retail, e-commerce, streaming services, digital services etc.
  \item Grade: Secondary O-level 2
\end{itemize}

\textbf{Question}

A company offers a basic cleaning service for offices that charges a fixed fee per session and an additional fee for each square metre cleaned. On Monday, they cleaned an office of $50$ square metres and charged \textdollar 130. On Tuesday, they cleaned another office of $80$ square metres and charged \textdollar 190. 

Let $x$ be the fixed fee per cleaning session (in \textdollar) and $y$ be the fee charged per square metre cleaned (in \textdollar).

(a) Write down two equations to represent the charges on Monday and Tuesday.

(b) Solve these equations to find the fixed fee per session and the fee per square metre cleaned.

\textbf{Solution}

(a)
Let $x$ be the fixed fee per session and $y$ be the fee per square metre. 

Monday: $x + 50y = 130$

Tuesday: $x + 80y = 190$

(b)
Subtract the first equation from the second:

$(x + 80y) - (x + 50y) = 190 - 130$

$x + 80y - x - 50y = 60$

$30y = 60$

$y = 2$

Substitute $y = 2$ into $x + 50y = 130$:

$x + 50(2) = 130$

$x + 100 = 130$

$x = 30$

Therefore, the fixed fee per session is \textdollar30, and the fee per square metre cleaned is \textdollar2.

\section*{Question 164}
\textbf{Metadata}

\begin{itemize}
  \item Question ID: O2-SPFndmdn\_O2-SPFndmode\_GPT4.1\_Services\_03
  \item Primary KC: STATISTICS AND PROBABILITY | Finding median | Finding median for a set of data
  \item Secondary KC: STATISTICS AND PROBABILITY | Finding mode | Finding mode for a set of data
  \item Topic: Services such as installation, maintenance, repairing, cleaning, laundry, hotel, retail, e-commerce, streaming services, digital services etc.
  \item Grade: Secondary O-level 2
\end{itemize}

\textbf{Question}

A laundry shop recorded the number of shirts washed for 7 consecutive days: $22$, $30$, $25$, $22$, $34$, $22$, $30$.

(a) What is the median number of shirts washed in a day?

(b) What is the mode of the number of shirts washed per day?

\textbf{Solution}

(a) To find the median, first list the data in order:
$22$, $22$, $22$, $25$, $30$, $30$, $34$

Since there are $7$ values, the median is the $4^{\text{th}}$ value:
\[
\boxed{25}
\]

(b) The mode is the number that appears the most in the data.

$22$ appears $3$ times, which is more frequent than the other numbers.

So, the mode is:
\[
\boxed{22}
\]

\section*{Question 165}
\textbf{Metadata}

\begin{itemize}
  \item Question ID: O2-SPFndmdn\_O3-SPFndPctl\_GPT4.1\_Services\_03
  \item Primary KC: STATISTICS AND PROBABILITY | Finding median | Finding median for a set of data
  \item Secondary KC: STATISTICS AND PROBABILITY | Finding percentiles | finding percentiles for a set of data
  \item Topic: Services such as installation, maintenance, repairing, cleaning, laundry, hotel, retail, e-commerce, streaming services, digital services etc.
  \item Grade: Secondary O-level 2
\end{itemize}

\textbf{Question}

A digital services company tracks the monthly subscription amounts (in \textdollar) for 11 of its customers: 12, 18, 20, 15, 22, 17, 25, 19, 21, 16, 24.

(a) Find the median subscription amount.

(b) Find the 75th percentile (third quartile, Q3) of the subscription amounts.

\textbf{Solution}

(a) First, arrange the subscription amounts in ascending order: 12, 15, 16, 17, 18, 19, 20, 21, 22, 24, 25.

Since there are 11 amounts, the median is the 6th value:

$\text{Median} = 19$

(b) To find the 75th percentile (Q3):

Position of 75th percentile $= \frac{75}{100} \times (n+1) = 0.75 \times (11+1) = 0.75 \times 12 = 9$

So, Q3 is the value at the 9th position in the ordered list:

$\text{Q}_3 = 22$

Therefore, the 75th percentile is $\textdollar22$.

\section*{Question 166}
\textbf{Metadata}

\begin{itemize}
  \item Question ID: O2-SPFndmdn\_O3-SPFndrng\_GPT4.1\_Services\_03
  \item Primary KC: STATISTICS AND PROBABILITY | Finding median | Finding median for a set of data
  \item Secondary KC: STATISTICS AND PROBABILITY | Finding range | finding range as measures of spread for a set of data 
  \item Topic: Services such as installation, maintenance, repairing, cleaning, laundry, hotel, retail, e-commerce, streaming services, digital services etc.
  \item Grade: Secondary O-level 2
\end{itemize}

\textbf{Question}

A cleaning company completed 7 cleaning jobs in one week. The amounts (in \textdollar) charged for each job were as follows: $48$, $52$, $60$, $55$, $62$, $56$, and $51$.

(a) What is the median amount charged?

(b) What is the range of the amounts charged?

\textbf{Solution}

(a) First, arrange the amounts in ascending order:
$48$, $51$, $52$, $55$, $56$, $60$, $62$

Since there are $7$ values (an odd number), the median is the $4$th value:
Median = $55$

(b) The range is the difference between the highest and lowest amounts:
\[
\text{Range} = 62 - 48 = 14
\]

So, the median amount charged is $\textdollar 55$ and the range is $14$.

\section*{Question 167}
\textbf{Metadata}

\begin{itemize}
  \item Question ID: O2-SPFndmean\_O3-BPRepSN\_GPT4.1\_Services\_03
  \item Primary KC: STATISTICS AND PROBABILITY | Finding mean deviation | calculation of the mean for a set of data
  \item Secondary KC: BASE AND POWER | Representation and concept  | use of standard form Ax10^n , where n is an integer, and 1<= A<= 10
  \item Topic: Services such as installation, maintenance, repairing, cleaning, laundry, hotel, retail, e-commerce, streaming services, digital services etc.
  \item Grade: Secondary O-level 2
\end{itemize}

\textbf{Question}

A digital laundry service processed the following number of orders per day over 5 days: $3.2 \times 10^2$, $2.9 \times 10^2$, $3.6 \times 10^2$, $3.1 \times 10^2$, and $2.8 \times 10^2$. All values are represented in standard form and indicate the number of laundry orders completed each day.

(a) Find the mean number of orders per day, leaving your answer in standard form.

(b) Calculate the mean deviation of the number of orders per day, giving your answer in standard form.

\textbf{Solution}

(a) To find the mean:

\[
\text{Sum} = (3.2 \times 10^2) + (2.9 \times 10^2) + (3.6 \times 10^2) + (3.1 \times 10^2) + (2.8 \times 10^2)
\]
\[
= (3.2 + 2.9 + 3.6 + 3.1 + 2.8) \times 10^2 = 15.6 \times 10^2
\]
\[
\text{Mean} = \frac{15.6 \times 10^2}{5} = 3.12 \times 10^2
\]

(b) Mean deviation:

Find the deviations (absolute differences from the mean):
\[
|3.2 \times 10^2 - 3.12 \times 10^2| = 0.08 \times 10^2 = 8
\]
\[
|2.9 \times 10^2 - 3.12 \times 10^2| = 0.22 \times 10^2 = 22
\]
\[
|3.6 \times 10^2 - 3.12 \times 10^2| = 0.48 \times 10^2 = 48
\]
\[
|3.1 \times 10^2 - 3.12 \times 10^2| = 0.02 \times 10^2 = 2
\]
\[
|2.8 \times 10^2 - 3.12 \times 10^2| = 0.32 \times 10^2 = 32
\]
Sum of absolute deviations: $8 + 22 + 48 + 2 + 32 = 112$

Mean deviation:
\[
\frac{112}{5} = 22.4 = 2.24 \times 10^1
\]

Final answers:
\[
\text{(a) Mean} = 3.12 \times 10^2
\]
\[
\text{(b) Mean deviation} = 2.24 \times 10^1
\]

\section*{Question 168}
\textbf{Metadata}

\begin{itemize}
  \item Question ID: O3-BPOpr\_O3-BPRepPosI\_GPT4.1\_Services\_03
  \item Primary KC: BASE AND POWER | Operations | laws of indices
  \item Secondary KC: BASE AND POWER | Representation and concept  | positive indices that is not 1
  \item Topic: Services such as installation, maintenance, repairing, cleaning, laundry, hotel, retail, e-commerce, streaming services, digital services etc.
  \item Grade: Secondary O-level 3/4
\end{itemize}

\textbf{Question}

A company provides a digital storage service where each user receives $2^3$ GB of storage space in the basic plan. For each upgrade, the storage space is multiplied by $3^2$. If Emily upgrades her plan twice, what is the total amount of storage space (in GB) she will have after the second upgrade? Express your answer as a single power of a positive integer.

\textbf{Solution}

Emily starts with $2^3$ GB. Each upgrade multiplies the storage by $3^2$, and she upgrades twice. Thus, after two upgrades, her total storage is:

\[
2^3 \times (3^2)^2 = 2^3 \times 3^{2\times2} = 2^3 \times 3^4
\]

So, Emily will have $2^3 \times 3^4$ GB of storage after the second upgrade.

\section*{Question 169}
\textbf{Metadata}

\begin{itemize}
  \item Question ID: O3-BPOpr\_O3-BPRepNegI\_GPT4.1\_Services\_03
  \item Primary KC: BASE AND POWER | Operations | laws of indices
  \item Secondary KC: BASE AND POWER | Representation and concept  | negative indices
  \item Topic: Services such as installation, maintenance, repairing, cleaning, laundry, hotel, retail, e-commerce, streaming services, digital services etc.
  \item Grade: Secondary O-level 3/4
\end{itemize}

\textbf{Question}

A digital streaming service offers a special promotion: If a user subscribes for $n$ consecutive months, the total cost (in dollars) is given by $20 \times 2^{n-2}$. However, if the user cancels and re-subscribes later, the original price multiplies by $2^{-k}$, where $k$ is the number of months the subscription was inactive.

(a) If Melanie subscribed for 4 consecutive months, what was her total cost?

(b) Suppose Melanie cancels her subscription for 2 months, and then re-subscribes for 1 month. What is the cost of her new subscription for that month, using the promotion formula and the given rule for inactivity?

(c) Find an expression, in terms of $n$ and $k$, for the cost of a new subscription after $k$ months of inactivity, for $n$ months.

\textbf{Solution}

(a) Melanie subscribed for $n = 4$ months. The total cost is:

\[
20 \times 2^{4-2} = 20 \times 2^2 = 20 \times 4 = \textdollar80
\]

(b) After cancelling, she was inactive for $k = 2$ months. Then she subscribes for $n = 1$ month. Her cost is:
\[
\text{Promotion cost for 1 month: } 20 \times 2^{1-2} = 20 \times 2^{-1} = 20 \times \frac{1}{2} = \textdollar10
\]
\[
\text{Applying the inactivity factor:}
\]
\[
\text{Total cost} = 10 \times 2^{-2} = 10 \times \frac{1}{4} = \textdollar2.50
\]

(c) The general expression:
\[
\text{Original cost for $n$ months: } 20 \times 2^{n-2}
\]
\[
\text{Cost after $k$ months of inactivity: } 20 \times 2^{n-2} \times 2^{-k} = 20 \times 2^{n-2-k}
\]

So, the cost is $20 \times 2^{n-2-k}$ dollars.

\section*{Question 170}
\textbf{Metadata}

\begin{itemize}
  \item Question ID: O3-BPOpr\_O3-BPRepFrI\_GPT4.1\_Services\_03
  \item Primary KC: BASE AND POWER | Operations | laws of indices
  \item Secondary KC: BASE AND POWER | Representation and concept  | fractional indices
  \item Topic: Services such as installation, maintenance, repairing, cleaning, laundry, hotel, retail, e-commerce, streaming services, digital services etc.
  \item Grade: Secondary O-level 3/4
\end{itemize}

\textbf{Question}

A laundry service offers a subscription where the cost for cleaning clothes is calculated using a special pricing formula. The number of clothes that can be cleaned per month is $x$, and the monthly cost in \textdollar is given by $C = 5x^{1/2}$. 

(a) If a customer chooses to have 36 clothes cleaned in a month, what will be the monthly cost?

(b) Another customer wants to know how many clothes she can have cleaned if she is willing to spend exactly \textdollar 20 in a month. How many clothes can she afford to have cleaned?

\textbf{Solution}

(a) Given $x = 36$, substitute into the formula:

\[
C = 5x^{1/2} = 5 \times 36^{1/2}
\]
\[
36^{1/2} = 6
\]
\[
C = 5 \times 6 = \textdollar 30
\]

The monthly cost is \textdollar 30.

(b) Set $C = \textdollar 20$ and solve for $x$:
\[
20 = 5x^{1/2}
\]
Divide both sides by 5:
\[
x^{1/2} = 4
\]
Rewrite $x^{1/2}$ as $\sqrt{x} = 4$:
\[
x = 4^2 = 16
\]

She can afford to have 16 clothes cleaned per month.

\section*{Question 171}
\textbf{Metadata}

\begin{itemize}
  \item Question ID: O3-STOprUn\_O3-STOprIns\_GPT4.1\_Services\_03
  \item Primary KC: SET | Set operations | union of two sets
  \item Secondary KC: SET | Set operations | intersection of two sets
  \item Topic: Services such as installation, maintenance, repairing, cleaning, laundry, hotel, retail, e-commerce, streaming services, digital services etc.
  \item Grade: Secondary O-level 3/4
\end{itemize}

\textbf{Question}

A hotel offers two types of digital entertainment services to its guests: streaming movies and online games. Last month, $62$ guests subscribed to either the streaming movies service, the online games service, or both. $38$ guests subscribed to the streaming movies service, $31$ guests subscribed to the online games service, and $12$ guests subscribed to both services. 

(a) How many guests subscribed to at least one of the two services?

(b) How many guests subscribed to only one service?

\textbf{Solution}

(a) Let $A$ be the set of guests who subscribed to the streaming movies service, and $B$ be the set of guests who subscribed to the online games service. The number of guests who subscribed to at least one service is given by the union:

$|A \cup B| = |A| + |B| - |A \cap B|$

Given $|A| = 38$, $|B| = 31$, and $|A \cap B| = 12$:

$|A \cup B| = 38 + 31 - 12 = 57$

So, $57$ guests subscribed to at least one of the two services.

(b) To find the number of guests who subscribed to only one service, we subtract those who subscribed to both from the total in each service and add these two results:

Number who only subscribed to streaming movies: $|A| - |A \cap B| = 38 - 12 = 26$

Number who only subscribed to online games: $|B| - |A \cap B| = 31 - 12 = 19$

Total who subscribed to only one service: $26 + 19 = 45$

So, $45$ guests subscribed to only one service.

\section*{Question 172}
\textbf{Metadata}

\begin{itemize}
  \item Question ID: O3-MXMulSM\_O3-MXAdd\_GPT4.1\_Services\_03
  \item Primary KC: MATRICES | Multiplication | product of a scalar quantity and a matrix
  \item Secondary KC: MATRICES | Addition | addition of matrices
  \item Topic: Services such as installation, maintenance, repairing, cleaning, laundry, hotel, retail, e-commerce, streaming services, digital services etc.
  \item Grade: Secondary O-level 3/4
\end{itemize}

\textbf{Question}

A cleaning company offers two packages for office buildings. The first package is represented by the matrix $A = \begin{bmatrix} 5 \\ 3 \end{bmatrix}$, where the first entry represents the number of small offices cleaned, and the second entry represents the number of large offices cleaned per day. 

The company decides to increase its daily cleaning operations by $4$ times for a special event week. In addition, they introduce an extra team represented by the matrix $B = \begin{bmatrix} 2 \\ 1 \end{bmatrix}$, who will clean the same types of offices each day.

(a) Find the matrix that represents the total number of each office type cleaned per day during the special event week.

(b) What is the total number of large offices cleaned per day during the special event week?

\textbf{Solution}

(a) First, we multiply matrix $A$ by the scalar $4$:

$4A = 4 \times \begin{bmatrix} 5 \\ 3 \end{bmatrix} = \begin{bmatrix} 4 \times 5 \\ 4 \times 3 \end{bmatrix} = \begin{bmatrix} 20 \\ 12 \end{bmatrix}$

Now, add the extra team's matrix $B$:

$Total = 4A + B = \begin{bmatrix} 20 \\ 12 \end{bmatrix} + \begin{bmatrix} 2 \\ 1 \end{bmatrix} = \begin{bmatrix} 20+2 \\ 12+1 \end{bmatrix} = \begin{bmatrix} 22 \\ 13 \end{bmatrix}$

So, the matrix representing the total number of offices cleaned per day during the special event week is $\begin{bmatrix} 22 \\ 13 \end{bmatrix}$.

(b) The total number of large offices cleaned per day during the special event week is the second entry of the matrix, which is $13$.

\section*{Question 173}
\textbf{Metadata}

\begin{itemize}
  \item Question ID: O3-MXMulSM\_O3-MXSub\_GPT4.1\_Services\_03
  \item Primary KC: MATRICES | Multiplication | product of a scalar quantity and a matrix
  \item Secondary KC: MATRICES | Subtraction | subtraction of matrices
  \item Topic: Services such as installation, maintenance, repairing, cleaning, laundry, hotel, retail, e-commerce, streaming services, digital services etc.
  \item Grade: Secondary O-level 3/4
\end{itemize}

\textbf{Question}

A cleaning service company uses a matrix to track the number of different cleaning supplies used at two hotel branches in one week. The supplies are represented as follows:

$A = \begin{bmatrix}12 & 18 \\ 8 & 9\end{bmatrix}$

The first column represents Branch 1 and the second column represents Branch 2. The first row counts bottles of cleaning spray, and the second row counts mops. Due to an upcoming event, management decides to double the order of each supply for the following week. However, Branch 2 informs the company to subtract last week's usage matrix, $B = \begin{bmatrix}0 & 5 \\ 0 & 3\end{bmatrix}$, from the doubled order for them, as they still have extra supplies.

Calculate the final matrix showing the number of supplies to be sent to each branch for the coming week.

\textbf{Solution}

First, double the order using scalar multiplication:

$2A = 2 \times \begin{bmatrix}12 & 18 \\ 8 & 9\end{bmatrix} = \begin{bmatrix}24 & 36 \\ 16 & 18\end{bmatrix}$

Next, subtract matrix $B$ from the doubled order:

$2A - B = \begin{bmatrix}24 & 36 \\ 16 & 18\end{bmatrix} - \begin{bmatrix}0 & 5 \\ 0 & 3\end{bmatrix} = \begin{bmatrix}24 - 0 & 36 - 5 \\ 16 - 0 & 18 - 3\end{bmatrix} = \begin{bmatrix}24 & 31 \\ 16 & 15\end{bmatrix}$

Thus, the number of cleaning supplies to be sent next week is:
- Branch 1: $24$ bottles of cleaning spray, $16$ mops
- Branch 2: $31$ bottles of cleaning spray, $15$ mops

\section*{Question 174}
\textbf{Metadata}

\begin{itemize}
  \item Question ID: O3-MXMulSM\_O3-MXMul\_GPT4.1\_Services\_03
  \item Primary KC: MATRICES | Multiplication | product of a scalar quantity and a matrix
  \item Secondary KC: MATRICES | Multiplication | multiplication of matrices
  \item Topic: Services such as installation, maintenance, repairing, cleaning, laundry, hotel, retail, e-commerce, streaming services, digital services etc.
  \item Grade: Secondary O-level 3/4
\end{itemize}

\textbf{Question}

A digital marketing company tracks the monthly costs of providing streaming services using a matrix. The matrix $A$ shows the costs (in \textdollar) for three services—video, music, and e-books—over two different months:

$A = \begin{bmatrix} 120 & 130 \\ 90 & 110 \\ 60 & 80 \end{bmatrix}$

Last month, the prices of delivering each service increased by a factor of $2$. The company also wanted to know the total cost for each service for both months combined. 

(a) Find the new cost matrix $B$ after the price increase using scalar multiplication.

(b) Find the total cost for each service for both months combined by multiplying the new cost matrix $B$ by the following column matrix:

$C = \begin{bmatrix} 1 \\ 1 \end{bmatrix}$

What is the resulting matrix?


\textbf{Solution}

(a) To find the new cost matrix $B$, multiply each element of $A$ by $2$:

$B = 2 \times A = 2 \times \begin{bmatrix} 120 & 130 \\ 90 & 110 \\ 60 & 80 \end{bmatrix} = \begin{bmatrix} 240 & 260 \\ 180 & 220 \\ 120 & 160 \end{bmatrix}$

(b) To find the total cost for each service over both months, multiply $B$ by the column matrix $C$:

$\begin{bmatrix} 240 & 260 \\ 180 & 220 \\ 120 & 160 \end{bmatrix} \times \begin{bmatrix} 1 \\ 1 \end{bmatrix} = \begin{bmatrix} 240 \times 1 + 260 \times 1 \\ 180 \times 1 + 220 \times 1 \\ 120 \times 1 + 160 \times 1 \end{bmatrix} = \begin{bmatrix} 500 \\ 400 \\ 280 \end{bmatrix}$

So the resulting matrix is $\begin{bmatrix} 500 \\ 400 \\ 280 \end{bmatrix}$, meaning the total costs for video, music and e-books respectively are \textdollar500, \textdollar400, and \textdollar280.

\section*{Question 175}
\textbf{Metadata}

\begin{itemize}
  \item Question ID: O3-MXSub\_O3-MXAdd\_GPT4.1\_Services\_03
  \item Primary KC: MATRICES | Subtraction | subtraction of matrices
  \item Secondary KC: MATRICES | Addition | addition of matrices
  \item Topic: Services such as installation, maintenance, repairing, cleaning, laundry, hotel, retail, e-commerce, streaming services, digital services etc.
  \item Grade: Secondary O-level 3/4
\end{itemize}

\textbf{Question}

A hotel tracks the number of maintenance and cleaning requests made by guests on two consecutive days using matrices. 

On Monday, the number of requests from Block A and Block B are represented by the matrix: 
\[ \begin{bmatrix} 8 & 6\\ 5 & 7 \end{bmatrix} \]
where the first row is maintenance and the second row is cleaning. The first column is Block A and the second column is Block B.

On Tuesday, the requests are recorded as: 
\[ \begin{bmatrix} 10 & 7\\ 4 & 9 \end{bmatrix} \]

(a) Find the total number of maintenance and cleaning requests for the two days in each block by adding the matrices.

(b) Find the difference in the number of requests between Tuesday and Monday in each category and block by subtracting the Monday matrix from the Tuesday matrix.

\textbf{Solution}

(a) To find the total number of requests for the two days, add the matrices:

\[
\begin{bmatrix} 8 & 6\\ 5 & 7 \end{bmatrix} + \begin{bmatrix} 10 & 7\\ 4 & 9 \end{bmatrix} = \begin{bmatrix} 8+10 & 6+7\\ 5+4 & 7+9 \end{bmatrix} = \begin{bmatrix} 18 & 13\\ 9 & 16 \end{bmatrix}
\]

So, in Block A, there were $18$ maintenance and $9$ cleaning requests over two days. In Block B, there were $13$ maintenance and $16$ cleaning requests.

(b) To find the difference in requests between Tuesday and Monday, subtract the Monday matrix from the Tuesday matrix:

\[
\begin{bmatrix} 10 & 7\\ 4 & 9 \end{bmatrix} - \begin{bmatrix} 8 & 6\\ 5 & 7 \end{bmatrix} = \begin{bmatrix} 10-8 & 7-6\\ 4-5 & 9-7 \end{bmatrix} = \begin{bmatrix} 2 & 1\\ -1 & 2 \end{bmatrix}
\]

This means Block A had $2$ more maintenance requests and $1$ fewer cleaning requests on Tuesday compared to Monday, while Block B had $1$ more maintenance request and $2$ more cleaning requests on Tuesday.

\section*{Question 176}
\textbf{Metadata}

\begin{itemize}
  \item Question ID: O3-MXMul\_O3-MXAdd\_GPT4.1\_Services\_03
  \item Primary KC: MATRICES | Multiplication | multiplication of matrices
  \item Secondary KC: MATRICES | Addition | addition of matrices
  \item Topic: Services such as installation, maintenance, repairing, cleaning, laundry, hotel, retail, e-commerce, streaming services, digital services etc.
  \item Grade: Secondary O-level 3/4
\end{itemize}

\textbf{Question}

A cleaning company uses matrices to track and update their service records. Matrix $A$ shows the number of standard and deep cleaning jobs completed over two weeks:

$$A = \begin{pmatrix} 7 & 5 \\ 4 & 6 \end{pmatrix}$$

where the rows represent Week 1 and Week 2, and the columns represent Standard and Deep cleaning services respectively.

Matrix $B$ shows the charges (in \textdollar) for each service type:

$$B = \begin{pmatrix} 30 & 50 \\ 25 & 45 \end{pmatrix}$$

where the rows represent Standard and Deep cleaning jobs, and the columns represent Residential and Commercial customers respectively.

At the end of Week 2, an extra promotional job was added, represented by matrix $C$:

$$C = \begin{pmatrix} 0 & 0 \\ 1 & 1 \end{pmatrix}$$

(a) Find the new service record matrix by adding $A$ and $C$.

(b) Find the total revenue matrix for each week and customer type by multiplying the updated service record matrix by $B$.


\textbf{Solution}

(a) Add matrices $A$ and $C$:
\begin{align*}
A + C &= \begin{pmatrix} 7 & 5 \\ 4 & 6 \end{pmatrix} + \begin{pmatrix} 0 & 0 \\ 1 & 1 \end{pmatrix} \\
&= \begin{pmatrix} 7+0 & 5+0 \\ 4+1 & 6+1 \end{pmatrix} \\
&= \begin{pmatrix} 7 & 5 \\ 5 & 7 \end{pmatrix}
\end{align*}

(b) Multiply the updated service record matrix $(A + C)$ by $B$.

Let $D = (A + C) B$.

First, recall $A + C$ shape: $2 \times 2$, and $B$ shape: $2 \times 2$

\begin{align*}
D &= \begin{pmatrix} 7 & 5 \\ 5 & 7 \end{pmatrix} \times \begin{pmatrix} 30 & 50 \\ 25 & 45 \end{pmatrix} 
begin{pmatrix}
7 \times 30 + 5 \times 25 & 7 \times 50 + 5 \times 45 \\
5 \times 30 + 7 \times 25 & 5 \times 50 + 7 \times 45 
end{pmatrix} \\
&= \begin{pmatrix}
210 + 125 & 350 + 225 \\
150 + 175 & 250 + 315
\end{pmatrix} 
= \begin{pmatrix}
335 & 575 \\
325 & 565
\end{pmatrix}
\end{align*}

So,
- First row: Total revenue for Week 1 (Residential, Commercial): \textdollar335, \textdollar575
- Second row: Total revenue for Week 2 (Residential, Commercial): \textdollar325, \textdollar565

\textbf{Final Answers:}

(a) The new service record matrix is $\begin{pmatrix} 7 & 5 \\ 5 & 7 \end{pmatrix}$.

(b) The total revenue matrix is $\begin{pmatrix} \textdollar335 & \textdollar575 \\ \textdollar325 & \textdollar565 \end{pmatrix}$, where rows correspond to Week 1 and 2, and columns correspond to Residential and Commercial customers.

\section*{Question 177}
\textbf{Metadata}

\begin{itemize}
  \item Question ID: O3-MXMul\_O3-MXSub\_GPT4.1\_Services\_03
  \item Primary KC: MATRICES | Multiplication | multiplication of matrices
  \item Secondary KC: MATRICES | Subtraction | subtraction of matrices
  \item Topic: Services such as installation, maintenance, repairing, cleaning, laundry, hotel, retail, e-commerce, streaming services, digital services etc.
  \item Grade: Secondary O-level 3/4
\end{itemize}

\textbf{Question}

A laundry company records the number of shirts, pants, and jackets cleaned for two hotels, Hotel A and Hotel B, over two weeks. 

The data for Week 1 is shown in matrix $P$ and for Week 2 in matrix $Q$ (entries represent the number of items cleaned).

$\displaystyle P = \begin{bmatrix} 30 & 45 & 25 \\ 40 & 35 & 20 \end{bmatrix}$  \hspace{1cm}  $\displaystyle Q = \begin{bmatrix} 32 & 50 & 30 \\ 38 & 40 & 22 \end{bmatrix}$

Each shirt cleaning costs $\textdollar2$, each pant cleaning costs $\textdollar3$, and each jacket cleaning costs $\textdollar5$. Represent the cost per item in the row matrix $R$:

$R = \begin{bmatrix} 2 & 3 & 5 \end{bmatrix}$

(a) Find the total income from all items cleaned in Week 1 and Week 2 for both hotels by multiplying the respective matrices by $R^T$.

(b) Next, find the difference in the number of each type of item cleaned between Week 2 and Week 1 for both hotels as a matrix.

(c) Hence, calculate the change in total income from Week 1 to Week 2 for each hotel.


\textbf{Solution}

(a) First, multiply the matrices for Week 1 and Week 2 by $R^T$.

$P \times R^T = \begin{bmatrix} 30 & 45 & 25 \\ 40 & 35 & 20 \end{bmatrix} \begin{bmatrix} 2 \\ 3 \\ 5 \end{bmatrix}$

$= \begin{bmatrix} 30\times2 + 45\times3 + 25\times5 \\
40\times2 + 35\times3 + 20\times5 \end{bmatrix}$

$= \begin{bmatrix} 60 + 135 + 125 \\
80 + 105 + 100 \end{bmatrix}$

$= \begin{bmatrix} 320 \\
285 \end{bmatrix}$

So, the total income for Week 1: \textdollar320 (Hotel A), \textdollar285 (Hotel B).

Similarly, $Q \times R^T = \begin{bmatrix} 32 & 50 & 30 \\ 38 & 40 & 22 \end{bmatrix} \begin{bmatrix} 2 \\ 3 \\ 5 \end{bmatrix}$

$= \begin{bmatrix} 32\times2 + 50\times3 + 30\times5 \\
38\times2 + 40\times3 + 22\times5 \end{bmatrix}$

$= \begin{bmatrix} 64 + 150 + 150 \\
76 + 120 + 110 \end{bmatrix}$

$= \begin{bmatrix} 364 \\
306 \end{bmatrix}$

So, the total income for Week 2: \textdollar364 (Hotel A), \textdollar306 (Hotel B).

(b) Find the difference in the number of items cleaned:

$Q - P = \begin{bmatrix} 32 - 30 & 50 - 45 & 30 - 25 \\ 38 - 40 & 40 - 35 & 22 - 20 \end{bmatrix} = \begin{bmatrix} 2 & 5 & 5 \\ -2 & 5 & 2 \end{bmatrix}$

(c) To find the change in total income, multiply this matrix by $R^T$:

$(Q - P) \times R^T = \begin{bmatrix} 2 & 5 & 5 \\ -2 & 5 & 2 \end{bmatrix} \begin{bmatrix} 2 \\ 3 \\ 5 \end{bmatrix}$

$= \begin{bmatrix} 2\times2 + 5\times3 + 5\times5 \\
-2\times2 + 5\times3 + 2\times5 \end{bmatrix}$

$= \begin{bmatrix} 4 + 15 + 25 \\
-4 + 15 + 10 \end{bmatrix}$

$= \begin{bmatrix} 44 \\
21 \end{bmatrix}$

So, the change in total income from Week 1 to Week 2 is an increase of \textdollar44 for Hotel A and \textdollar21 for Hotel B.

\section*{Question 178}
\textbf{Metadata}

\begin{itemize}
  \item Question ID: O3-SPFndQtl\_O3-SPFndIQR\_GPT4.1\_Services\_03
  \item Primary KC: STATISTICS AND PROBABILITY | Finding quartiles | finding quartiles for a set of data
  \item Secondary KC: STATISTICS AND PROBABILITY | Finding range | finding interquartile range as measures of spread for a set of data 
  \item Topic: Services such as installation, maintenance, repairing, cleaning, laundry, hotel, retail, e-commerce, streaming services, digital services etc.
  \item Grade: Secondary O-level 3/4
\end{itemize}

\textbf{Question}

A hotel records the nightly rates charged for 9 of its rooms on a particular weekend as follows:\\

$\textdollar 110, \textdollar 120, \textdollar 150, \textdollar 155, \textdollar 160, \textdollar 170, \textdollar 180, \textdollar 185, \textdollar 200$\\

(a) Find the lower quartile ($Q_1$), median ($Q_2$), and upper quartile ($Q_3$) of the room rates.\\

(b) Find the interquartile range (IQR) of the room rates.

\textbf{Solution}

(a) Arrange the data in order (already arranged):\\
$110, 120, 150, 155, 160, 170, 180, 185, 200$\\

Number of data points $n = 9$\\
\\
Median ($Q_2$) is the $\frac{n+1}{2} = \frac{10}{2} = 5$th value:\\
$Q_2 = 160$\\
\\
Lower half (excluding median): $110, 120, 150, 155$\\
Upper half (excluding median): $170, 180, 185, 200$\\

Lower quartile ($Q_1$): median of $110, 120, 150, 155$\\
There are 4 values, so $Q_1$ is the average of the 2nd and 3rd values.\\
$Q_1 = \frac{120 + 150}{2} = 135$\\

Upper quartile ($Q_3$): median of $170, 180, 185, 200$\\
$Q_3 = \frac{180 + 185}{2} = 182.5$\\

(b) Interquartile range (IQR)\\
$IQR = Q_3 - Q_1 = 182.5 - 135 = 47.5$\\

Therefore,\\
$Q_1 = \textdollar 135$\\
$Q_2 = \textdollar 160$\\
$Q_3 = \textdollar 182.5$\\
$IQR = \textdollar 47.5$.

\section*{Question 179}
\textbf{Metadata}

\begin{itemize}
  \item Question ID: O3-SPFndstd\_O2-SPFndmean\_GPT4.1\_Services\_03
  \item Primary KC: STATISTICS AND PROBABILITY | Finding standard deviation | calculation of the standard deviation for a set of data
  \item Secondary KC: STATISTICS AND PROBABILITY | Finding mean deviation | calculation of the mean for a set of data
  \item Topic: Services such as installation, maintenance, repairing, cleaning, laundry, hotel, retail, e-commerce, streaming services, digital services etc.
  \item Grade: Secondary O-level 3/4
\end{itemize}

\textbf{Question}

A manager at a laundry service company recorded the number of shirts cleaned per day over a week (7 days). The data recorded were: $24$, $30$, $28$, $22$, $26$, $30$, $20$.

(a) Find the mean number of shirts cleaned per day.

(b) Calculate the standard deviation of the number of shirts cleaned per day. Give your answer correct to $2$ decimal places.

\textbf{Solution}

(a) Mean number of shirts cleaned per day:

\[
\text{Mean} = \frac{24 + 30 + 28 + 22 + 26 + 30 + 20}{7} = \frac{180}{7} \approx 25.71
\]

(b) Standard deviation calculation:
First, calculate the deviation of each value from the mean and square it:
\begin{align*}
24:~ &(24 - 25.71)^2 = (-1.71)^2 = 2.9241 \\
30:~ &(30 - 25.71)^2 = (4.29)^2 = 18.4041 \\
28:~ &(28 - 25.71)^2 = (2.29)^2 = 5.2441 \\
22:~ &(22 - 25.71)^2 = (-3.71)^2 = 13.7641 \\
26:~ &(26 - 25.71)^2 = (0.29)^2 = 0.0841 \\
30:~ &(30 - 25.71)^2 = (4.29)^2 = 18.4041 \\
20:~ &(20 - 25.71)^2 = (-5.71)^2 = 32.6241 \\
\end{align*}

Sum of squared deviations:
\[
2.9241 + 18.4041 + 5.2441 + 13.7641 + 0.0841 + 18.4041 + 32.6241 = 91.4487
\]

Variance:
\[
\text{Variance} = \frac{91.4487}{7} = 13.0641
\]

Standard deviation:
\[
\text{Standard deviation} = \sqrt{13.0641} \approx 3.61
\]

\textbf{Final answers:}
\begin{itemize}
  \item (a) Mean = $25.71$
  \item (b) Standard deviation $\approx 3.61$ (correct to $2$ decimal places)
\end{itemize}


\section*{Question 180}
\textbf{Metadata}

\begin{itemize}
  \item Question ID: O3-SPAddProb\_O2-SPRepPrSE\_GPT4.1\_Services\_03
  \item Primary KC: STATISTICS AND PROBABILITY | Addition | addition of probabilities
  \item Secondary KC: STATISTICS AND PROBABILITY | Representation and concept | probability of single events
  \item Topic: Services such as installation, maintenance, repairing, cleaning, laundry, hotel, retail, e-commerce, streaming services, digital services etc.
  \item Grade: Secondary O-level 3/4
\end{itemize}

\textbf{Question}

A hotel offers a cleaning service that can be booked in advance or on the day of stay. On a particular day, $40\%$ of the guests booked the cleaning service in advance, and $25\%$ of the guests booked it on the day of their stay. Some guests did not book the service at all.

If a guest is chosen at random, what is the probability that the guest booked the cleaning service either in advance or on the day of their stay?

Express your answer as a decimal.

\textbf{Solution}

Let $P(\text{Advance}) = 0.40$ and $P(\text{On the day}) = 0.25$.

Assume that no guest booked both in advance and on the day (the events are mutually exclusive).

The probability that a randomly chosen guest booked the cleaning service either in advance or on the day is:

\[
P(\text{Advance or On the day}) = P(\text{Advance}) + P(\text{On the day})
\]
\[
= 0.40 + 0.25
\]
\[
= 0.65
\]

Therefore, the probability is $0.65$.

\section*{Question 181}
\textbf{Metadata}

\begin{itemize}
  \item Question ID: O3-SPAddProb\_O3-SPFndPrCE\_GPT4.1\_Services\_03
  \item Primary KC: STATISTICS AND PROBABILITY | Addition | addition of probabilities
  \item Secondary KC: STATISTICS AND PROBABILITY | Finding probability | probability of simple combined events
  \item Topic: Services such as installation, maintenance, repairing, cleaning, laundry, hotel, retail, e-commerce, streaming services, digital services etc.
  \item Grade: Secondary O-level 3/4
\end{itemize}

\textbf{Question}

A hotel offers two types of digital services to its guests: movie streaming and online gaming. The probability that a randomly chosen guest uses the movie streaming service is $0.4$, and the probability that a guest uses the online gaming service is $0.3$. Assume that no guest uses both services at the same time. 

What is the probability that a randomly chosen guest uses either the movie streaming service or the online gaming service?

\textbf{Solution}

Let $P(M)$ be the probability that a guest uses the movie streaming service and $P(G)$ the probability that a guest uses the online gaming service. Since no guest uses both services at the same time (the events are mutually exclusive), we simply add the probabilities:

\[
P(M \text{ or } G) = P(M) + P(G) = 0.4 + 0.3 = 0.7
\]

Therefore, the probability that a randomly chosen guest uses either service is $0.7$ (or $70\%$).

\section*{Question 182}
\textbf{Metadata}

\begin{itemize}
  \item Question ID: O3-SPMulProb\_O2-SPRepPrSE\_GPT4.1\_Services\_03
  \item Primary KC: STATISTICS AND PROBABILITY | Multiplication | multiplication of probabilities
  \item Secondary KC: STATISTICS AND PROBABILITY | Representation and concept | probability of single events
  \item Topic: Services such as installation, maintenance, repairing, cleaning, laundry, hotel, retail, e-commerce, streaming services, digital services etc.
  \item Grade: Secondary O-level 3/4
\end{itemize}

\textbf{Question}

A hotel offers an online booking service. From their previous records, the probability that a randomly chosen guest will book a room using the hotel’s official website is $0.4$. The probability that a guest who books online will also request a room cleaning service during their stay is $0.6$.

(a) What is the probability that a randomly chosen guest will both book a room online and request a room cleaning service?

(b) What is the probability that a randomly chosen guest will book a room online or request a room cleaning service, assuming that $75\%$ of all guests request cleaning regardless of how they book?

\textbf{Solution}

Let $A$ be the event 'guest books a room online' and $B$ be the event 'guest requests a room cleaning service'.

Given:
$P(A) = 0.4$

Given that the guest books online, $P(B|A) = 0.6$

(a) The probability that a guest books online and requests cleaning is given by the multiplication of probabilities:
\[
P(A \cap B) = P(A) \times P(B|A) = 0.4 \times 0.6 = 0.24
\]
So, the probability is $0.24$.

(b) Given $75\%$ of all guests request cleaning, $P(B) = 0.75$.
We want $P(A \cup B) = P(A) + P(B) - P(A \cap B)$.
Substitute:
\[
P(A \cup B) = 0.4 + 0.75 - 0.24 = 1.15 - 0.24 = 0.91
\]
So, the probability that a randomly chosen guest will either book online or request a room cleaning service is $0.91$.

\section*{Question 183}
\textbf{Metadata}

\begin{itemize}
  \item Question ID: O3-SPMulProb\_O3-SPFndPrCE\_GPT4.1\_Services\_03
  \item Primary KC: STATISTICS AND PROBABILITY | Multiplication | multiplication of probabilities
  \item Secondary KC: STATISTICS AND PROBABILITY | Finding probability | probability of simple combined events
  \item Topic: Services such as installation, maintenance, repairing, cleaning, laundry, hotel, retail, e-commerce, streaming services, digital services etc.
  \item Grade: Secondary O-level 3/4
\end{itemize}

\textbf{Question}

A hotel employs a maintenance team to ensure that all air-conditioners in guest rooms function properly. Statistics show that on any given day, the probability that a randomly selected air-conditioner in the hotel is working is $0.97$. The probability that the room’s digital safe is working is $0.95$ on the same day. If a guest is randomly assigned a room, what is the probability that both the air-conditioner and the digital safe in their room are working on that day, assuming these two events are independent?

\textbf{Solution}

Let $P(A)$ be the probability that the air-conditioner is working, $P(A) = 0.97$. Let $P(S)$ be the probability that the safe is working, $P(S) = 0.95$. Since the events are independent, the probability that both are working is 

$$P(A \text{ and } S) = P(A) \times P(S) = 0.97 \times 0.95 = 0.9215$$

Therefore, the probability that both the air-conditioner and the safe are working is $0.9215$ or $92.15\%$. 

\end{document}
