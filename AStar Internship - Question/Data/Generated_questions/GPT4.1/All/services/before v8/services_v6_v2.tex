\documentclass{article}
\usepackage[utf8]{inputenc}
\usepackage{amsmath}
\usepackage{amsfonts}
\usepackage{amssymb}
\usepackage{graphicx}
\usepackage{hyperref}
\title{All Services MVPs by GPT4.1}
\author{Tien Dung Doan}
\begin{document}
\maketitle
\section*{Question 1}
\textbf{Metadata}

\begin{itemize}
  \item Question ID: P3-WNAdd4d\_P1-WNCmp\_GPT4.1\_Services\_05
  \item Primary KC: WHOLE NUMBERS | Addition | adding whole numbers up to 4 digits
  \item Secondary KC: WHOLE NUMBERS | Comparison and ordering | comparing and ordering whole numbers
  \item Topic: Services such as installation, maintenance, repairing, cleaning, laundry, e-commerce
  \item Grade: Primary 3
\end{itemize}

\textbf{Question}

A laundry shop washed $2,345$ shirts on Monday and $1,873$ shirts on Tuesday. 

(a) How many shirts did the shop wash on both days altogether?

(b) Another laundry shop washed $4,298$ shirts in two days. Which shop washed more shirts in two days, and by how many shirts?

\textbf{Solution}

(a) To find the total number of shirts washed on both days, add $2,345$ and $1,873$:

$2,345 + 1,873 = 4,218$

So, the shop washed $4,218$ shirts in total on both days.

(b) The other shop washed $4,298$ shirts in two days. Compare $4,218$ and $4,298$:

$4,298 > 4,218$

Subtract to find the difference:

$4,298 - 4,218 = 80$

The other shop washed $80$ more shirts than the first shop.

\section*{Question 2}
\textbf{Metadata}

\begin{itemize}
  \item Question ID: P3-WNAdd4d\_P1-WNAdd2nd\_GPT4.1\_Services\_04
  \item Primary KC: WHOLE NUMBERS | Addition | adding whole numbers up to 4 digits
  \item Secondary KC: WHOLE NUMBERS | Addition | adding whole numbers
  \item Topic: Services such as installation, maintenance, repairing, cleaning, laundry, e-commerce
  \item Grade: Primary 3
\end{itemize}

\textbf{Question}

Mrs Tan runs a small cleaning service. On Monday, her team cleaned 1,250 apartments. On Tuesday, they cleaned 2,346 apartments. On Wednesday, they cleaned 1,578 apartments. On Thursday, they cleaned another 2,424 apartments. What is the total number of apartments her team cleaned from Monday to Thursday?

\textbf{Solution}

To find the total number of apartments cleaned, we need to add the numbers for each day:

$1,250 + 2,346 = 3,596$

$3,596 + 1,578 = 5,174$

$5,174 + 2,424 = 7,598$

Therefore, Mrs Tan's team cleaned a total of 7,598 apartments from Monday to Thursday.

\section*{Question 3}
\textbf{Metadata}

\begin{itemize}
  \item Question ID: P3-WNSub4d\_P1-WNCmp\_GPT4.1\_Services\_04
  \item Primary KC: WHOLE NUMBERS | Subtraction | subtracting whole numbers up to 4 digits
  \item Secondary KC: WHOLE NUMBERS | Comparison and ordering | comparing and ordering whole numbers
  \item Topic: Services such as installation, maintenance, repairing, cleaning, laundry, e-commerce
  \item Grade: Primary 3
\end{itemize}

\textbf{Question}

Mr. Tan runs a cleaning service. In May, he cleaned $3,256$ homes. In June, he cleaned $2,789$ homes. 

(a) How many fewer homes did Mr. Tan clean in June than in May?

(b) If Mr. Tan wants to compare the number of homes cleaned in May and June, which month had more homes cleaned, and by how many?

Arrange the months in order from the month with fewer homes cleaned to the month with more homes cleaned.

\textbf{Solution}

(a) Number of homes cleaned in May: $3,256$

Number of homes cleaned in June: $2,789$

Number of fewer homes cleaned in June = $3,256 - 2,789 = 467$

So, Mr. Tan cleaned $467$ fewer homes in June than in May.

(b) May had more homes cleaned than June. 

$3,256$ (May) is greater than $2,789$ (June).

May had $467$ more homes cleaned than June (as calculated above).

Arranged in order:

June ($2,789$ homes) < May ($3,256$ homes)

So, June comes first, followed by May.

\section*{Question 4}
\textbf{Metadata}

\begin{itemize}
  \item Question ID: P3-WNSub4d\_P1-WNAdd2nd\_GPT4.1\_Services\_03
  \item Primary KC: WHOLE NUMBERS | Subtraction | subtracting whole numbers up to 4 digits
  \item Secondary KC: WHOLE NUMBERS | Addition | adding whole numbers
  \item Topic: Services such as installation, maintenance, repairing, cleaning, laundry, e-commerce
  \item Grade: Primary 3
\end{itemize}

\textbf{Question}

Li Wei is running a cleaning service. On Monday, he cleaned $1358$ offices. On Tuesday, he cleaned $247$ more offices than he did on Monday. How many offices did Li Wei clean on both days altogether? If $1095$ offices still need to be cleaned after Tuesday, how many offices were there at first?

\textbf{Solution}

First, find out how many offices Li Wei cleaned on Tuesday: 

$1358 + 247 = 1605$

Next, find out the total number of offices he cleaned on both days: 

$1358 + 1605 = 2963$

If $1095$ offices still need to be cleaned after Tuesday, the total number of offices at first was: 

$2963 + 1095 = 4058$

So, Li Wei cleaned $2963$ offices in total on both days, and there were $4058$ offices at first.

\section*{Question 5}
\textbf{Metadata}

\begin{itemize}
  \item Question ID: P3-WNSub4d\_P1-WNSub2nd\_GPT4.1\_Services\_03
  \item Primary KC: WHOLE NUMBERS | Subtraction | subtracting whole numbers up to 4 digits
  \item Secondary KC: WHOLE NUMBERS | Subtraction | subtracting whole numbers
  \item Topic: Services such as installation, maintenance, repairing, cleaning, laundry, e-commerce
  \item Grade: Primary 3
\end{itemize}

\textbf{Question}

Mrs. Lim runs a cleaning service company. On Monday, her team cleaned $2\,430$ offices. By the end of Tuesday, the total number of offices cleaned increased to $4\,895$. On Wednesday, Mrs. Lim realised that $1\,520$ more offices needed cleaning to reach her team's weekly target. 

How many offices did Mrs. Lim's team clean on Tuesday and what is the team's weekly target?

\textbf{Solution}

First, subtract the number of offices cleaned by the end of Monday from the number cleaned by the end of Tuesday to find the number of offices cleaned on Tuesday:

$4,895 - 2,430 = 2,465$

So, Mrs. Lim's team cleaned $2,465$ offices on Tuesday.

Next, add the number of offices cleaned by the end of Tuesday to the number of offices still needed to reach the target:

$4,895 + 1,520 = 6,415$

The team's weekly target is $6,415$ offices.

\section*{Question 6}
\textbf{Metadata}

\begin{itemize}
  \item Question ID: P3-WNDivRmd3d\_P1-WNCmp\_GPT4.1\_Services\_03
  \item Primary KC: WHOLE NUMBERS | Division | dividing whole numbers up to 3 digits by 1 digit with remainder 
  \item Secondary KC: WHOLE NUMBERS | Comparison and ordering | comparing and ordering whole numbers
  \item Topic: Services such as installation, maintenance, repairing, cleaning, laundry, e-commerce
  \item Grade: Primary 3
\end{itemize}

\textbf{Question}

A laundry shop needs to deliver $252$ washed shirts to several customers. The delivery worker can fit $4$ shirts in one delivery bag. 

(a) How many full bags can the worker prepare, and how many shirts will be left over?

(b) If the shop receives another order, bringing the total number of washed shirts to $260$, is this new total more or less than $252$? Arrange $252$ and $260$ in order from least to greatest.

\textbf{Solution}

(a) To find out how many full bags the worker can prepare and how many shirts are left over, divide $252$ by $4$:

$252 \div 4 = 63$ remainder $0$

So, there are $63$ full bags, and $0$ shirts are left over.

(b) $260$ is more than $252$. Arranging the numbers from least to greatest: $252, 260$.

\section*{Question 7}
\textbf{Metadata}

\begin{itemize}
  \item Question ID: P3-WNDivRmd3d\_P1-WNAdd2nd\_GPT4.1\_Services\_03
  \item Primary KC: WHOLE NUMBERS | Division | dividing whole numbers up to 3 digits by 1 digit with remainder 
  \item Secondary KC: WHOLE NUMBERS | Addition | adding whole numbers
  \item Topic: Services such as installation, maintenance, repairing, cleaning, laundry, e-commerce
  \item Grade: Primary 3
\end{itemize}

\textbf{Question}

A laundry shop received $247$ towels to wash in one day. The machines in the shop can wash $6$ towels at a time. 

(a) How many loads of laundry can be done with $247$ towels?

(b) If another $53$ towels are brought in later, how many towels does the shop have to wash in total? How many loads will there be now if each machine still washes $6$ towels at a time?

\textbf{Solution}

Part (a):

Divide $247$ towels by $6$ towels per load:

$247 \div 6 = 41$ remainder $1$

So, $41$ full loads and $1$ towel left for another load. Therefore, the shop needs $42$ loads in total.

Part (b):

$247 + 53 = 300$

Total towels to wash: $300$

Now, $300 \div 6 = 50$

So, the shop will need $50$ loads in total.

\section*{Question 8}
\textbf{Metadata}

\begin{itemize}
  \item Question ID: P3-WNDivRmd3d\_P1-WNSub2nd\_GPT4.1\_Services\_03
  \item Primary KC: WHOLE NUMBERS | Division | dividing whole numbers up to 3 digits by 1 digit with remainder 
  \item Secondary KC: WHOLE NUMBERS | Subtraction | subtracting whole numbers
  \item Topic: Services such as installation, maintenance, repairing, cleaning, laundry, e-commerce
  \item Grade: Primary 3
\end{itemize}

\textbf{Question}

A cleaning company received $233$ requests in a week to clean apartments. Each cleaning team can handle $6$ requests a day. 

(a) How many complete teams are needed to handle all the requests if each team works on the same number of requests? 

(b) After assigning requests equally to the teams, how many requests will be left unassigned?

(c) If the company wants to handle the leftover requests, how many fewer requests does it need so that all requests can be distributed evenly among the teams?

\textbf{Solution}

Let us answer each part step by step.

(a) To find out how many complete teams are needed, we divide the total number of requests by the number of requests each team handles:

$$ 233 \div 6 = 38 \text{ remainder } 5 $$

So, $38$ teams will have $6$ requests each, and $5$ requests will be left over.

(b) Number of requests left unassigned:

The remainder is $5$, so $5$ requests are left after dividing $233$ by $6$.

(c) To have all requests distributed evenly, the number of requests should be a multiple of $6$.

The nearest smaller multiple of $6$ is:

$$ 6 \times 38 = 228 $$

So, the company needs to have:

$$ 233 - 228 = 5 $$

fewer requests for even distribution.

**Answers:**

(a) $38$ complete teams are needed.

(b) $5$ requests will be left unassigned.

(c) The company needs $5$ fewer requests to distribute all evenly among the teams.

\section*{Question 9}
\textbf{Metadata}

\begin{itemize}
  \item Question ID: P3-WNDivRmd3d\_P1-WNMul2nd\_GPT4.1\_Services\_03
  \item Primary KC: WHOLE NUMBERS | Division | dividing whole numbers up to 3 digits by 1 digit with remainder 
  \item Secondary KC: WHOLE NUMBERS | Multiplication | multiplying whole numbers
  \item Topic: Services such as installation, maintenance, repairing, cleaning, laundry, e-commerce
  \item Grade: Primary 3
\end{itemize}

\textbf{Question}

A laundry shop receives $247$ dirty uniforms to clean in one day. Each washing machine in the shop can wash $6$ uniforms at a time.

(a) How many full loads of uniforms can the shop wash if each load must have exactly $6$ uniforms?

(b) How many uniforms will be left for the last, not full, load?

(c) If it takes $5$ minutes to wash each load, what is the total washing time (in minutes) for all the full loads?

\textbf{Solution}

(a) To find the number of full loads, divide $247$ by $6$:

$247 \div 6 = 41$ remainder $1$

So, there are $41$ full loads.

(b) The remainder from the division is $1$. So, $1$ uniform will be left for the last, not full, load.

(c) Each full load takes $5$ minutes.
Total time $= 41 \times 5 = 205$ minutes.

Final answers:
(a) $41$ full loads
(b) $1$ uniform left
(c) $205$ minutes for washing all the full loads.

\section*{Question 10}
\textbf{Metadata}

\begin{itemize}
  \item Question ID: P3-WNMul3d1d\_P1-WNCmp\_GPT4.1\_Services\_03
  \item Primary KC: WHOLE NUMBERS | Multiplication | multiplying whole numbers up to 3 digits by 1 digit
  \item Secondary KC: WHOLE NUMBERS | Comparison and ordering | comparing and ordering whole numbers
  \item Topic: Services such as installation, maintenance, repairing, cleaning, laundry, e-commerce
  \item Grade: Primary 3
\end{itemize}

\textbf{Question}

A laundry shop has 3 different types of washing machines. Type A can wash $128$ shirts at a time, Type B can wash $85$ shirts at a time, and Type C can wash $64$ shirts at a time. Each day, the laundry shop uses each type of machine $4$ times. 

(a) How many shirts does the laundry shop wash in total with each type of machine in one day?

(b) Arrange the total numbers of shirts washed by each type of machine in order from least to greatest.

\textbf{Solution}

(a) For Type A: $128 \times 4 = 512$

For Type B: $85 \times 4 = 340$

For Type C: $64 \times 4 = 256$

(b) The totals are: $512$, $340$, $256$. 
Arranged from least to greatest: $256$, $340$, $512$.

\section*{Question 11}
\textbf{Metadata}

\begin{itemize}
  \item Question ID: P3-WNMul3d1d\_P1-WNAdd2nd\_GPT4.1\_Services\_03
  \item Primary KC: WHOLE NUMBERS | Multiplication | multiplying whole numbers up to 3 digits by 1 digit
  \item Secondary KC: WHOLE NUMBERS | Addition | adding whole numbers
  \item Topic: Services such as installation, maintenance, repairing, cleaning, laundry, e-commerce
  \item Grade: Primary 3
\end{itemize}

\textbf{Question}

A cleaning company was hired to clean 7 office buildings on Monday. Each building needs 224 chairs wiped. If a worker can clean all the chairs in one building in a day, and the company sends 3 workers, each to a different building, how many chairs in total do the workers clean on Monday? If the company cleaned 1,120 chairs in total the next day, what is the combined number of chairs cleaned over both days?

\textbf{Solution}

On Monday, each of the 3 workers cleans 224 chairs in one building. The total number of chairs cleaned by 3 workers:

$224 \times 3 = 672$ chairs cleaned on Monday.

On the next day, 1,120 chairs were cleaned.

Combined number of chairs cleaned over both days:

$672 + 1,120 = 1,792$

So, the workers cleaned a total of 1,792 chairs over both days.

\section*{Question 12}
\textbf{Metadata}

\begin{itemize}
  \item Question ID: P3-WNMul3d1d\_P1-WNSub2nd\_GPT4.1\_Services\_03
  \item Primary KC: WHOLE NUMBERS | Multiplication | multiplying whole numbers up to 3 digits by 1 digit
  \item Secondary KC: WHOLE NUMBERS | Subtraction | subtracting whole numbers
  \item Topic: Services such as installation, maintenance, repairing, cleaning, laundry, e-commerce
  \item Grade: Primary 3
\end{itemize}

\textbf{Question}

A laundry shop charges $5$\textdollar{} to wash one shirt. Ben brought $8$ shirts to the shop. After the shirts were washed, Ben used a discount voucher worth $10$\textdollar{} to pay for part of the laundry bill. 

How much does Ben need to pay after using the discount voucher?

\textbf{Solution}

First, find the total cost to wash $8$ shirts:

$8 \times 5 = 40$

So, washing $8$ shirts costs $40$\textdollar{}.

Next, subtract the value of the discount voucher:

$40 - 10 = 30$

Therefore, Ben needs to pay $30$\textdollar{} after using the discount voucher.

\section*{Question 13}
\textbf{Metadata}

\begin{itemize}
  \item Question ID: P3-WNMul3d1d\_P1-WNMul2nd\_GPT4.1\_Services\_03
  \item Primary KC: WHOLE NUMBERS | Multiplication | multiplying whole numbers up to 3 digits by 1 digit
  \item Secondary KC: WHOLE NUMBERS | Multiplication | multiplying whole numbers
  \item Topic: Services such as installation, maintenance, repairing, cleaning, laundry, e-commerce
  \item Grade: Primary 3
\end{itemize}

\textbf{Question}

A laundry shop charges $7$\textdollar$ to wash one large blanket. Mrs Tan has $4$ blankets and $3$ bed covers to wash. Each bed cover costs $5$\textdollar$ to wash. How much does Mrs Tan need to pay in total for washing all her blankets and bed covers?

\textbf{Solution}

First, find the cost of washing the $4$ blankets: $4 \times 7 = 28\textdollar$.

Next, find the cost of washing the $3$ bed covers: $3 \times 5 = 15\textdollar$.

Add both amounts to get the total cost: $28\textdollar + 15\textdollar = 43\textdollar$.

So, Mrs Tan needs to pay $43\textdollar$ in total.

\section*{Question 14}
\textbf{Metadata}

\begin{itemize}
  \item Question ID: P3-WNDiv3d1d\_P1-WNCmp\_GPT4.1\_Services\_03
  \item Primary KC: WHOLE NUMBERS | Division | dividing whole numbers up to 3 digits by 1 digit
  \item Secondary KC: WHOLE NUMBERS | Comparison and ordering | comparing and ordering whole numbers
  \item Topic: Services such as installation, maintenance, repairing, cleaning, laundry, e-commerce
  \item Grade: Primary 3
\end{itemize}

\textbf{Question}

A cleaning company received $246$ requests for cleaning homes over a week. The company has $3$ teams of cleaners. If each team gets an equal number of requests, how many requests does each team receive? 

After dividing the requests equally, which team number (1, 2, or 3) would have the most requests if one extra request was given to Team 1?

\textbf{Solution}

First, divide $246$ by $3$ to find out how many requests each team receives:

$$
246 \div 3 = 82
$$
So, each team receives $82$ requests.

If Team 1 is given one extra request, Team 1 will have:

$$
82 + 1 = 83
$$
requests.

Now, compare the number of requests for each team:
- Team 1: $83$
- Team 2: $82$
- Team 3: $82$

Team 1 now has the most requests.

\section*{Question 15}
\textbf{Metadata}

\begin{itemize}
  \item Question ID: P3-WNDiv3d1d\_P1-WNAdd2nd\_GPT4.1\_Services\_03
  \item Primary KC: WHOLE NUMBERS | Division | dividing whole numbers up to 3 digits by 1 digit
  \item Secondary KC: WHOLE NUMBERS | Addition | adding whole numbers
  \item Topic: Services such as installation, maintenance, repairing, cleaning, laundry, e-commerce
  \item Grade: Primary 3
\end{itemize}

\textbf{Question}

Mr. Tan has \(216\) shirts that need to be washed at the laundry service. The washing machines at the laundry can each wash \(6\) shirts at a time. 

(a) How many washing machines will Mr. Tan need to wash all \(216\) shirts at the same time?

Later, Mr. Tan wants to add \(48\) more shirts that he found at home to the laundry. 

(b) What is the new total number of shirts to be washed?

\textbf{Solution}

(a) To find the number of washing machines needed:

Each machine can wash \(6\) shirts at a time.

Number of machines needed: \(\frac{216}{6} = 36\) machines

(b) Add the \(48\) more shirts to the original \(216\) shirts:

Total shirts = \(216 + 48 = 264\) shirts

So, the new total number of shirts to be washed is \(264\).

\section*{Question 16}
\textbf{Metadata}

\begin{itemize}
  \item Question ID: P3-WNDiv3d1d\_P1-WNSub2nd\_GPT4.1\_Services\_03
  \item Primary KC: WHOLE NUMBERS | Division | dividing whole numbers up to 3 digits by 1 digit
  \item Secondary KC: WHOLE NUMBERS | Subtraction | subtracting whole numbers
  \item Topic: Services such as installation, maintenance, repairing, cleaning, laundry, e-commerce
  \item Grade: Primary 3
\end{itemize}

\textbf{Question}

Sam operates a laundry service and received $156$ pieces of clothing to wash in one day. He can fit $4$ pieces of clothing into one washing machine at a time. After washing as many loads as possible, he found there were some pieces left unwashed. 

(a) How many full washing machine loads did Sam complete? 

(b) How many pieces of clothing were left unwashed after all the full loads?

\textbf{Solution}

To find the number of full washing machine loads, we divide $156$ by $4$. 

$156 \div 4 = 39$ loads, with no remainder. 

That means Sam washed all $156$ pieces in full loads, and no clothing was left over.

(a) Number of full loads: $39$

(b) Number of pieces left unwashed: $156 - (4 \times 39) = 156 - 156 = 0$

So, (a) $39$ full loads, (b) $0$ pieces left unwashed.

\section*{Question 17}
\textbf{Metadata}

\begin{itemize}
  \item Question ID: P3-WNDiv3d1d\_P1-WNMul2nd\_GPT4.1\_Services\_03
  \item Primary KC: WHOLE NUMBERS | Division | dividing whole numbers up to 3 digits by 1 digit
  \item Secondary KC: WHOLE NUMBERS | Multiplication | multiplying whole numbers
  \item Topic: Services such as installation, maintenance, repairing, cleaning, laundry, e-commerce
  \item Grade: Primary 3
\end{itemize}

\textbf{Question}

A cleaning company received an order to clean $162$ windows in an office building. The company sent $3$ workers to do the job. Each worker cleaned the same number of windows. 

(a) How many windows did each worker clean?

(b) If each worker can clean $9$ windows in $1$ hour, how many hours did each worker take to finish their share?

\textbf{Solution}

(a) Total number of windows $= 162$

Number of workers $= 3$

Number of windows cleaned by each worker $= 162 \div 3 = 54$

Each worker cleaned $54$ windows.

(b) Each worker can clean $9$ windows in $1$ hour.

Number of windows each worker cleaned $= 54$

Time taken by each worker $= 54 \div 9 = 6$

Each worker took $6$ hours to finish their share.

\section*{Question 18}
\textbf{Metadata}

\begin{itemize}
  \item Question ID: P3-FrAddRl12\_P2-FrCmp\_GPT4.1\_Services\_03
  \item Primary KC: FRACTIONS | Addition | adding two related fractions within one whole with denominators of given fractions not exceeding 12
  \item Secondary KC: FRACTIONS | Comparison and ordering | comparing and ordering fractions
  \item Topic: Services such as installation, maintenance, repairing, cleaning, laundry, e-commerce
  \item Grade: Primary 3
\end{itemize}

\textbf{Question}

Amanda is cleaning her room. She spends $\frac{1}{4}$ of an hour dusting her shelves and $\frac{1}{2}$ of an hour sweeping the floor. 

(a) How much time, in total, did Amanda spend on cleaning both the shelves and the floor? 

(b) After that, Amanda compares this total cleaning time with the time she spent arranging her books, which took $\frac{3}{4}$ of an hour. Did Amanda spend more time cleaning the shelves and floor or arranging her books? Arrange the three activities from the shortest to the longest time spent.

\textbf{Solution}

(a) Time dusting shelves: $\frac{1}{4}$ hour 
Time sweeping floor: $\frac{1}{2}$ hour 
To add $\frac{1}{4}$ and $\frac{1}{2}$, rewrite $\frac{1}{2}$ with denominator $4$:

$\frac{1}{2} = \frac{2}{4}$

Total time spent cleaning = $\frac{1}{4} + \frac{2}{4} = \frac{3}{4}$ hour

(b) Arranging books took $\frac{3}{4}$ hour. The total cleaning time is also $\frac{3}{4}$ hour. 

Comparing the times:
- Dusting shelves: $\frac{1}{4}$ hour (shortest)
- Sweeping floor: $\frac{1}{2}$ hour (next)
- Cleaning both shelves and floor: $\frac{3}{4}$ hour
- Arranging books: $\frac{3}{4}$ hour

Arranging the main activities:
1. Dusting shelves ($\frac{1}{4}$ hour)
2. Sweeping floor ($\frac{1}{2}$ hour)
3. Cleaning shelves and floor OR arranging books ($\frac{3}{4}$ hour) (both take the same amount of time)

Amanda spent the same amount of time cleaning both the shelves and floor as she did arranging her books.

\section*{Question 19}
\textbf{Metadata}

\begin{itemize}
  \item Question ID: P3-FrAddRl12\_P2-FrAdd2nd\_GPT4.1\_Services\_03
  \item Primary KC: FRACTIONS | Addition | adding two related fractions within one whole with denominators of given fractions not exceeding 12
  \item Secondary KC: FRACTIONS | Addition | adding fractions
  \item Topic: Services such as installation, maintenance, repairing, cleaning, laundry, e-commerce
  \item Grade: Primary 3
\end{itemize}

\textbf{Question}

A washing machine repairman cleans the washing machine drum for $\frac{2}{8}$ of an hour and then checks the filter for $\frac{3}{8}$ of an hour. 

How much of an hour does he spend on both tasks altogether?

\textbf{Solution}

Time spent cleaning the drum: $\frac{2}{8}$ hour.

Time spent checking the filter: $\frac{3}{8}$ hour.

Since the denominators are the same, we can add the numerators:

$\frac{2}{8} + \frac{3}{8} = \frac{2+3}{8} = \frac{5}{8}$ hour.

The repairman spends $\frac{5}{8}$ hour on both tasks altogether.

\section*{Question 20}
\textbf{Metadata}

\begin{itemize}
  \item Question ID: P3-FrAddRl12\_P3-FrSmp\_GPT4.1\_Services\_03
  \item Primary KC: FRACTIONS | Addition | adding two related fractions within one whole with denominators of given fractions not exceeding 12
  \item Secondary KC: FRACTIONS | Simplifying | expressing a fraction in its simplest form
  \item Topic: Services such as installation, maintenance, repairing, cleaning, laundry, e-commerce
  \item Grade: Primary 3
\end{itemize}

\textbf{Question}

A washing machine repairman spent $\frac{3}{12}$ of an hour checking the machine and $\frac{5}{12}$ of an hour repairing it at Mei Ling's house. What fraction of an hour did the repairman spend at Mei Ling's house altogether? Express your answer in its simplest form.

\textbf{Solution}

To find the total time the repairman spent, add the two fractions:

$\frac{3}{12} + \frac{5}{12} = \frac{3+5}{12} = \frac{8}{12}$.

Now, express $\frac{8}{12}$ in its simplest form:

The greatest common factor of 8 and 12 is 4.

$\frac{8 \div 4}{12 \div 4} = \frac{2}{3}$

So, the repairman spent $\frac{2}{3}$ of an hour at Mei Ling's house altogether.

\section*{Question 21}
\textbf{Metadata}

\begin{itemize}
  \item Question ID: P3-FrSubRl12\_P2-FrCmp\_GPT4.1\_Services\_03
  \item Primary KC: FRACTIONS | Subtraction | subtracting two related fractions within one whole with denominators of given fractions not exceeding 12
  \item Secondary KC: FRACTIONS | Comparison and ordering | comparing and ordering fractions
  \item Topic: Services such as installation, maintenance, repairing, cleaning, laundry, e-commerce
  \item Grade: Primary 3
\end{itemize}

\textbf{Question}

David is cleaning his room. He cleaned $\frac{3}{4}$ of his room in the morning and $\frac{5}{12}$ of his room in the afternoon.\
\
(a) What fraction of his room did David clean in the morning than in the afternoon?\
\
(b) Write the fractions $\frac{3}{4}$ and $\frac{5}{12}$ in order, from the least to the greatest.

\textbf{Solution}

(a) To find how much more David cleaned in the morning than in the afternoon, subtract the fraction cleaned in the afternoon from the fraction cleaned in the morning:\\
$\frac{3}{4} - \frac{5}{12}$\\
First, find a common denominator. The denominators are 4 and 12. The least common denominator is 12.\\
$\frac{3}{4} = \frac{9}{12}$\\
$\frac{5}{12}$ stays the same.\\
Now subtract:\\
$\frac{9}{12} - \frac{5}{12} = \frac{4}{12} = \frac{1}{3}$\\
So, David cleaned $\frac{1}{3}$ more of his room in the morning than in the afternoon.\\
\\
(b) To order $\frac{3}{4}$ and $\frac{5}{12}$ from least to greatest, make the denominators the same.\\
$\frac{3}{4} = \frac{9}{12}$\\
Now compare $\frac{9}{12}$ and $\frac{5}{12}$: $\frac{5}{12} < \frac{9}{12}$\\
So, $\frac{5}{12}$, $\frac{3}{4}$ (from least to greatest).

\section*{Question 22}
\textbf{Metadata}

\begin{itemize}
  \item Question ID: P3-FrSubRl12\_P2-FrAdd2nd\_GPT4.1\_Services\_03
  \item Primary KC: FRACTIONS | Subtraction | subtracting two related fractions within one whole with denominators of given fractions not exceeding 12
  \item Secondary KC: FRACTIONS | Addition | adding fractions
  \item Topic: Services such as installation, maintenance, repairing, cleaning, laundry, e-commerce
  \item Grade: Primary 3
\end{itemize}

\textbf{Question}

Siti helps her mother with the laundry. On Monday, Siti spent $\frac{5}{8}$ of an hour washing clothes. After that, she spent $\frac{1}{8}$ of an hour folding the clean laundry. How much longer did Siti spend washing clothes than folding them? Then, Siti decided to help a bit more, so she spent another $\frac{1}{4}$ of an hour ironing the clothes. What is the total amount of time Siti spent on all the laundry tasks?

\textbf{Solution}

First, find out how much longer Siti spent washing than folding:

$\frac{5}{8} - \frac{1}{8} = \frac{4}{8} = \frac{1}{2}$ (hour)

So, Siti spent $\frac{1}{2}$ hour longer washing than folding.

Next, add all the time spent on laundry:

Washing: $\frac{5}{8}$ hour  
Folding: $\frac{1}{8}$ hour  
Ironing: $\frac{1}{4}$ hour

First, add folding and ironing. Convert $\frac{1}{4}$ to eighths:  
$\frac{1}{4} = \frac{2}{8}$

So, $\frac{1}{8} + \frac{2}{8} = \frac{3}{8}$

Now add washing:
$\frac{5}{8} + \frac{3}{8} = \frac{8}{8} = 1$

Siti spent a total of 1 hour on all the laundry tasks.

\section*{Question 23}
\textbf{Metadata}

\begin{itemize}
  \item Question ID: P3-FrSubRl12\_P2-FrSub2nd\_GPT4.1\_Services\_03
  \item Primary KC: FRACTIONS | Subtraction | subtracting two related fractions within one whole with denominators of given fractions not exceeding 12
  \item Secondary KC: FRACTIONS | Subtraction | subtracting fractions
  \item Topic: Services such as installation, maintenance, repairing, cleaning, laundry, e-commerce
  \item Grade: Primary 3
\end{itemize}

\textbf{Question}

A washing machine repairman finished fixing \(\frac{5}{8}\) of the washing machines in a laundromat by lunchtime. After lunch, he repaired \(\frac{3}{8}\) more. 

How many more washing machines did he repair after lunch compared to before lunch?

\textbf{Solution}

To find out how many more washing machines the repairman fixed after lunch compared to before lunch, subtract the number fixed before lunch from the number fixed after lunch:

\[ \frac{5}{8} - \frac{3}{8} = \frac{2}{8} \]

\[ \frac{2}{8} = \frac{1}{4} \]

So, the repairman fixed \(\frac{1}{4}\) more of the washing machines before lunch compared to after lunch.

\section*{Question 24}
\textbf{Metadata}

\begin{itemize}
  \item Question ID: P3-FrSubRl12\_P3-FrSmp\_GPT4.1\_Services\_03
  \item Primary KC: FRACTIONS | Subtraction | subtracting two related fractions within one whole with denominators of given fractions not exceeding 12
  \item Secondary KC: FRACTIONS | Simplifying | expressing a fraction in its simplest form
  \item Topic: Services such as installation, maintenance, repairing, cleaning, laundry, e-commerce
  \item Grade: Primary 3
\end{itemize}

\textbf{Question}

Mina is helping her mother to do the laundry. Out of the clothes in the laundry basket, $\frac{9}{12}$ of them are shirts and the rest are trousers. What fraction of the clothes are trousers? Express your answer in its simplest form.

\textbf{Solution}

The shirts make up $\frac{9}{12}$ of the clothes. Since all the clothes in the basket make 1 whole, the fraction that is trousers is:

$1 - \frac{9}{12} = \frac{12}{12} - \frac{9}{12} = \frac{3}{12}$

Now, simplify $\frac{3}{12}$:

$\frac{3}{12} = \frac{1}{4}$.

So, $\frac{1}{4}$ of the clothes are trousers.

\section*{Question 25}
\textbf{Metadata}

\begin{itemize}
  \item Question ID: P4-WNMul4d1d\_P1-WNCmp\_GPT4.1\_Services\_03
  \item Primary KC: WHOLE NUMBERS | Multiplication | multiplying whole numbers up to 4 digits by 1 digit or up to 3 digits by 2 digits
  \item Secondary KC: WHOLE NUMBERS | Comparison and ordering | comparing and ordering whole numbers
  \item Topic: Services such as installation, maintenance, repairing, cleaning, laundry, e-commerce
  \item Grade: Primary 4
\end{itemize}

\textbf{Question}

A cleaning company was hired to clean two different office buildings. Each floor in Building A requires 125 cleaners, and there are 8 floors. Each floor in Building B requires 98 cleaners, and there are 10 floors. 

(a) Calculate the total number of cleaners needed for each building.

(b) Which building needs more cleaners, and by how many?

\textbf{Solution}

(a) For Building A: $125 \times 8 = 1000$ cleaners.

For Building B: $98 \times 10 = 980$ cleaners.

(b) Building A needs $1000$ cleaners, which is more than Building B.

Difference: $1000 - 980 = 20$ cleaners.

Therefore, Building A needs more cleaners, by 20.

\section*{Question 26}
\textbf{Metadata}

\begin{itemize}
  \item Question ID: P4-WNMul4d1d\_P1-WNAdd2nd\_GPT4.1\_Services\_03
  \item Primary KC: WHOLE NUMBERS | Multiplication | multiplying whole numbers up to 4 digits by 1 digit or up to 3 digits by 2 digits
  \item Secondary KC: WHOLE NUMBERS | Addition | adding whole numbers
  \item Topic: Services such as installation, maintenance, repairing, cleaning, laundry, e-commerce
  \item Grade: Primary 4
\end{itemize}

\textbf{Question}

A cleaning company was hired to clean 135 houses, and each house cleaning costs $27$. On another day, the company cleaned 89 offices, and the cleaning cost for each office was $45$. What is the total amount of money the company earned from cleaning all the houses and offices?

\textbf{Solution}

First, find the amount earned from cleaning the houses:

$135 \times 27 = 3645$

So, the company earned $\textdollar3645$ from the houses.

Next, find the amount earned from cleaning the offices:

$89 \times 45 = 4005$

So, the company earned $\textdollar4005$ from the offices.

Now, add the two amounts to find the total earned:

$3645 + 4005 = 7650$

Therefore, the company earned a total of $\textdollar7650$ from cleaning all the houses and offices.

\section*{Question 27}
\textbf{Metadata}

\begin{itemize}
  \item Question ID: P4-WNMul4d1d\_P1-WNSub2nd\_GPT4.1\_Services\_03
  \item Primary KC: WHOLE NUMBERS | Multiplication | multiplying whole numbers up to 4 digits by 1 digit or up to 3 digits by 2 digits
  \item Secondary KC: WHOLE NUMBERS | Subtraction | subtracting whole numbers
  \item Topic: Services such as installation, maintenance, repairing, cleaning, laundry, e-commerce
  \item Grade: Primary 4
\end{itemize}

\textbf{Question}

A laundry service washes $128$ shirts each day for $6$ days in a week. Due to a public holiday, on one of the days, $75$ fewer shirts were washed than usual. How many shirts did the laundry service wash in total during that week?

\textbf{Solution}

First, calculate the usual number of shirts washed in a day: $\quad 128$ shirts.\newline
Number of shirts washed on $5$ normal days: $128 \times 5 = 640$.\newline
On the day with fewer shirts, number of shirts washed:
$128 - 75 = 53$ shirts.\newline
Total number of shirts washed for the week:
$640 + 53 = 693$ shirts.\newline
\boxed{693}$ shirts were washed in total during that week.

\section*{Question 28}
\textbf{Metadata}

\begin{itemize}
  \item Question ID: P4-WNMul4d1d\_P1-WNMul2nd\_GPT4.1\_Services\_03
  \item Primary KC: WHOLE NUMBERS | Multiplication | multiplying whole numbers up to 4 digits by 1 digit or up to 3 digits by 2 digits
  \item Secondary KC: WHOLE NUMBERS | Multiplication | multiplying whole numbers
  \item Topic: Services such as installation, maintenance, repairing, cleaning, laundry, e-commerce
  \item Grade: Primary 4
\end{itemize}

\textbf{Question}

A laundry service company washes uniforms for a hotel. Each day, the company washes $128$ uniforms for the hotel staff. Each staff uniform costs $4\textdollar$ to wash. If the company works $6$ days a week for the hotel, how much money does the company earn from washing the uniforms for one week?

\textbf{Solution}

First, calculate the total number of uniforms washed in a week:

$128$ uniforms per day $\times$ $6$ days $= 768$ uniforms.

Next, find out the total amount earned:

$768$ uniforms $\times$ $4\textdollar$ per uniform $= 3072\textdollar$.

So, the company earns $3072\textdollar$ from washing the hotel's uniforms for one week.

\section*{Question 29}
\textbf{Metadata}

\begin{itemize}
  \item Question ID: P4-WNMul4d1d\_P4-WNRnd5d\_GPT4.1\_Services\_03
  \item Primary KC: WHOLE NUMBERS | Multiplication | multiplying whole numbers up to 4 digits by 1 digit or up to 3 digits by 2 digits
  \item Secondary KC: WHOLE NUMBERS | Rounding | rounding whole numbers up to 100000 to the nearest 10, 100 or 1000 
  \item Topic: Services such as installation, maintenance, repairing, cleaning, laundry, e-commerce
  \item Grade: Primary 4
\end{itemize}

\textbf{Question}

A cleaning company charges \$178 for cleaning one office. Last month, the company cleaned 34 offices in total.

(a) How much money did the company collect for cleaning all 34 offices?

(b) Round off the total amount collected to the nearest hundred.

\textbf{Solution}

(a) The total amount collected is $178 \times 34 = 6,052$.

So, the cleaning company collected \$6,052 for cleaning all 34 offices.

(b) Rounding \$6,052 to the nearest hundred:

The digit in the tens place is 5 (in 6,052), so we round up.

\$6,052$ rounded to the nearest hundred is \$6,100.

Final answers:
(a) \$6,052
(b) \$6,100

\section*{Question 30}
\textbf{Metadata}

\begin{itemize}
  \item Question ID: P4-WNDiv4d1d\_P1-WNCmp\_GPT4.1\_Services\_03
  \item Primary KC: WHOLE NUMBERS | Division | dividing whole numbers up to 4 digits by 1 digit
  \item Secondary KC: WHOLE NUMBERS | Comparison and ordering | comparing and ordering whole numbers
  \item Topic: Services such as installation, maintenance, repairing, cleaning, laundry, e-commerce
  \item Grade: Primary 4
\end{itemize}

\textbf{Question}

A cleaning company received $2,856$ towels to wash for several hotels. The towels are to be packed equally into $6$ laundry bags before being delivered back to the hotels. After dividing the towels, the manager needs to arrange the laundry bags from the one with the least towels to the one with the most towels, if some bags accidentally get extra or fewer towels. 

(a) How many towels should each laundry bag have if the towels are divided equally?  
(b) If two bags end up with $472$ towels each and the other four bags have $476$ towels each, list the number of towels in each bag in increasing order.

\textbf{Solution}

(a) Divide $2,856$ by $6$ to find out how many towels each bag should have:

$2,856 \div 6 = 476$

So, each laundry bag should have $476$ towels.

(b) The towels in the bags are $472, 472, 476, 476, 476, 476$.

Arranged in increasing order: $472, 472, 476, 476, 476, 476$.

\section*{Question 31}
\textbf{Metadata}

\begin{itemize}
  \item Question ID: P4-WNDiv4d1d\_P1-WNAdd2nd\_GPT4.1\_Services\_03
  \item Primary KC: WHOLE NUMBERS | Division | dividing whole numbers up to 4 digits by 1 digit
  \item Secondary KC: WHOLE NUMBERS | Addition | adding whole numbers
  \item Topic: Services such as installation, maintenance, repairing, cleaning, laundry, e-commerce
  \item Grade: Primary 4
\end{itemize}

\textbf{Question}

A laundry company washed $2,376$ towels in one week. They divided the towels evenly among $4$ delivery vans to be distributed to various hotels. How many towels did each van get? If each van then received an extra $28$ towels for a special order, how many towels did each van deliver in total?

\textbf{Solution}

First, divide the total number of towels by the number of vans: 

$2,376 \div 4 = 594$

Each van got $594$ towels. 

Next, add the extra towels for each van:

$594 + 28 = 622$

Each van delivered $622$ towels in total.

\section*{Question 32}
\textbf{Metadata}

\begin{itemize}
  \item Question ID: P4-WNDiv4d1d\_P1-WNSub2nd\_GPT4.1\_Services\_03
  \item Primary KC: WHOLE NUMBERS | Division | dividing whole numbers up to 4 digits by 1 digit
  \item Secondary KC: WHOLE NUMBERS | Subtraction | subtracting whole numbers
  \item Topic: Services such as installation, maintenance, repairing, cleaning, laundry, e-commerce
  \item Grade: Primary 4
\end{itemize}

\textbf{Question}

A laundry company washed $3,684$ shirts for different customers in a day. The manager wants to pack the shirts equally into $6$ delivery bags. After packing, $78$ shirts were found to be damaged and removed from the delivery. How many shirts will be delivered in total?

\textbf{Solution}

First, divide $3,684$ shirts equally into $6$ delivery bags:

$3,684 \div 6 = 614$

So, there are $614$ shirts in each bag and a total of $3,684$ shirts packed. After packing, $78$ shirts were removed because they were damaged.

Next, subtract the damaged shirts from the total delivered:

$3,684 - 78 = 3,606$

So, $3,606$ shirts will be delivered in total.

\section*{Question 33}
\textbf{Metadata}

\begin{itemize}
  \item Question ID: P4-WNDiv4d1d\_P1-WNMul2nd\_GPT4.1\_Services\_03
  \item Primary KC: WHOLE NUMBERS | Division | dividing whole numbers up to 4 digits by 1 digit
  \item Secondary KC: WHOLE NUMBERS | Multiplication | multiplying whole numbers
  \item Topic: Services such as installation, maintenance, repairing, cleaning, laundry, e-commerce
  \item Grade: Primary 4
\end{itemize}

\textbf{Question}

A cleaning company received $2,448$ towels from a hotel to wash. The towels are to be packed equally into boxes, with each box holding $6$ towels. 

(a) How many boxes are needed to pack all the towels? 

(b) After packing, the company sent the boxes to 4 hotels so that each hotel received the same number of boxes. How many boxes did each hotel receive?

\textbf{Solution}

(a) Total number of boxes needed: 

$$ \frac{2448}{6} = 408 $$

So, $408$ boxes are needed.

(b) Number of boxes each hotel received: 

$$ \frac{408}{4} = 102 $$

So, each hotel received $102$ boxes.

\section*{Question 34}
\textbf{Metadata}

\begin{itemize}
  \item Question ID: P4-WNDiv4d1d\_P4-WNRnd5d\_GPT4.1\_Services\_03
  \item Primary KC: WHOLE NUMBERS | Division | dividing whole numbers up to 4 digits by 1 digit
  \item Secondary KC: WHOLE NUMBERS | Rounding | rounding whole numbers up to 100000 to the nearest 10, 100 or 1000 
  \item Topic: Services such as installation, maintenance, repairing, cleaning, laundry, e-commerce
  \item Grade: Primary 4
\end{itemize}

\textbf{Question}

A cleaning company received a contract to clean $7,845$ chairs in a large auditorium. The manager wants to divide the cleaning work equally among $8$ cleaning staff. 

(a) How many chairs will each staff member be responsible for cleaning?

(b) Round the number of chairs each staff member has to clean to the nearest $10$.

\textbf{Solution}

(a) To find out how many chairs each staff member will clean, divide $7,845$ by $8$:

$7,845 \div 8 = 980$ remainder $5$

So, each staff member will be responsible for cleaning $980$ chairs, and there will be $5$ chairs left.

(b) To round $980$ to the nearest $10$:

The digit in the ones place is $0$. Since it is $0$, we keep $980$ as it is.

So, rounded to the nearest $10$, each staff member cleans $980$ chairs.

\section*{Question 35}
\textbf{Metadata}

\begin{itemize}
  \item Question ID: P4-FrRepMixIm\_P3-FrCnvEq\_GPT4.1\_Services\_03
  \item Primary KC: FRACTIONS | Representation and concept | mixed numbers and improper fractions
  \item Secondary KC: FRACTIONS | Conversion to equivalent fractions | Conversion to equivalent fractions (given either the denominator or the numerator)
  \item Topic: Services such as installation, maintenance, repairing, cleaning, laundry, e-commerce
  \item Grade: Primary 4
\end{itemize}

\textbf{Question}

A cleaning company sent Suria to clean 1 \(\dfrac{3}{4}\) floors of a shopping centre. 

(a) Express the total number of floors Suria cleaned as an improper fraction.

(b) Each floor is divided into parts of size \(\dfrac{1}{8}\) of a floor. How many eighth-floors did Suria clean in total? (Express your answer as an equivalent fraction with denominator 8.)

\textbf{Solution}

(a) To express \(1\dfrac{3}{4}\) as an improper fraction:

\[
1\dfrac{3}{4} = \frac{4 \times 1 + 3}{4} = \frac{7}{4}
\]

(b) To convert \(\frac{7}{4}\) into an equivalent fraction with denominator 8:

\[
\frac{7}{4} = \frac{7 \times 2}{4 \times 2} = \frac{14}{8}
\]

So Suria cleaned \(\frac{14}{8}\) (fourteen eighth-floors) in total.

\section*{Question 36}
\textbf{Metadata}

\begin{itemize}
  \item Question ID: P4-FrRepSet\_P3-FrCnvEq\_GPT4.1\_Services\_03
  \item Primary KC: FRACTIONS | Representation and concept | fraction as part of a set 
  \item Secondary KC: FRACTIONS | Conversion to equivalent fractions | Conversion to equivalent fractions (given either the denominator or the numerator)
  \item Topic: Services such as installation, maintenance, repairing, cleaning, laundry, e-commerce
  \item Grade: Primary 4
\end{itemize}

\textbf{Question}

Rachel runs a laundry service where she received a batch of $24$ towels to clean. After finishing, she found that $\frac{1}{3}$ of the towels were still stained and had to be washed again. What fraction of the towels were still stained? Express this fraction as an equivalent fraction with a denominator of $12$.

\textbf{Solution}

Rachel found $\frac{1}{3}$ of $24$ towels were still stained.\newline
Number of stained towels $= \frac{1}{3} \times 24 = 8$.\newline
The fraction of towels still stained is $\frac{8}{24}$.\newline
To find its equivalent fraction with denominator $12$, divide numerator and denominator by $2$: $\frac{8}{24} = \frac{4}{12}$.\newline
Therefore, the equivalent fraction is $\frac{4}{12}$.

\section*{Question 37}
\textbf{Metadata}

\begin{itemize}
  \item Question ID: P4-FrAddU12\_P2-FrCmp\_GPT4.1\_Services\_03
  \item Primary KC: FRACTIONS | Addition | adding unlike fractions with two different denominators not exceeding 12
  \item Secondary KC: FRACTIONS | Comparison and ordering | comparing and ordering fractions
  \item Topic: Services such as installation, maintenance, repairing, cleaning, laundry, e-commerce
  \item Grade: Primary 4
\end{itemize}

\textbf{Question}

A cleaning company offers window cleaning and carpet cleaning services. Sarah cleaned $\frac{3}{8}$ of her windows and $\frac{1}{6}$ of her carpets on Saturday. 

(a) What fraction of her windows and carpets did Sarah clean in total on Saturday? Give your answer in its simplest form.

(b) On Sunday, Sarah cleaned another $\frac{1}{4}$ of her windows. Compare the amount of cleaning Sarah did on Saturday and Sunday. On which day did she clean more, and by how much?

\textbf{Solution}

(a) To find the total fraction of windows and carpets cleaned on Saturday, add $\frac{3}{8}$ and $\frac{1}{6}$.

Find the lowest common denominator (LCD) of $8$ and $6$. The LCD is $24$.

$\frac{3}{8} = \frac{3 \times 3}{8 \times 3} = \frac{9}{24}$

$\frac{1}{6} = \frac{1 \times 4}{6 \times 4} = \frac{4}{24}$

$\frac{9}{24} + \frac{4}{24} = \frac{13}{24}$

So, Sarah cleaned $\frac{13}{24}$ of her windows and carpets on Saturday.

(b) On Sunday, she cleaned $\frac{1}{4}$ of her windows.

To compare $\frac{13}{24}$ (Saturday) and $\frac{1}{4}$ (Sunday), convert $\frac{1}{4}$ to $\frac{6}{24}$:

$\frac{1}{4} = \frac{1 \times 6}{4 \times 6} = \frac{6}{24}$

$\frac{13}{24} > \frac{6}{24}$

$\frac{13}{24} - \frac{6}{24} = \frac{7}{24}$

Sarah cleaned more on Saturday, by $\frac{7}{24}$.

\section*{Question 38}
\textbf{Metadata}

\begin{itemize}
  \item Question ID: P4-FrAddU12\_P2-FrAdd2nd\_GPT4.1\_Services\_03
  \item Primary KC: FRACTIONS | Addition | adding unlike fractions with two different denominators not exceeding 12
  \item Secondary KC: FRACTIONS | Addition | adding fractions
  \item Topic: Services such as installation, maintenance, repairing, cleaning, laundry, e-commerce
  \item Grade: Primary 4
\end{itemize}

\textbf{Question}

A washing machine repairman spends $\frac{2}{3}$ hour installing a new part and $\frac{3}{4}$ hour cleaning the machine at Mrs. Lee’s house. How much time, in hours, did he spend on both tasks altogether?

\textbf{Solution}

To find the total time spent, we add $\frac{2}{3}$ hour and $\frac{3}{4}$ hour.

Find a common denominator for $3$ and $4$, which is $12$.

$\frac{2}{3} = \frac{2 \times 4}{3 \times 4} = \frac{8}{12}$

$\frac{3}{4} = \frac{3 \times 3}{4 \times 3} = \frac{9}{12}$

Add the two fractions:

$\frac{8}{12} + \frac{9}{12} = \frac{17}{12}$

$\frac{17}{12}$ hours can also be written as $1\frac{5}{12}$ hours.

**Answer:** He spent $\frac{17}{12}$ hours, or $1\frac{5}{12}$ hours, on both tasks altogether.

\section*{Question 39}
\textbf{Metadata}

\begin{itemize}
  \item Question ID: P4-FrAddU12\_P3-FrSmp\_GPT4.1\_Services\_03
  \item Primary KC: FRACTIONS | Addition | adding unlike fractions with two different denominators not exceeding 12
  \item Secondary KC: FRACTIONS | Simplifying | expressing a fraction in its simplest form
  \item Topic: Services such as installation, maintenance, repairing, cleaning, laundry, e-commerce
  \item Grade: Primary 4
\end{itemize}

\textbf{Question}

A cleaning company offers to clean the windows and carpets of an office. In one morning, the cleaners cleaned $\frac{3}{8}$ of the windows and $\frac{1}{6}$ of the carpets. What fraction of the total cleaning work was done in the morning? Express your answer in its simplest form.

\textbf{Solution}

To find the total fraction of work done, add $\frac{3}{8}$ and $\frac{1}{6}$.

First, find a common denominator for 8 and 6. The lowest common multiple is 24.

$\frac{3}{8} = \frac{3 \times 3}{8 \times 3} = \frac{9}{24}$
$\frac{1}{6} = \frac{1 \times 4}{6 \times 4} = \frac{4}{24}$

Now, add the fractions:
$\frac{9}{24} + \frac{4}{24} = \frac{13}{24}$

$\frac{13}{24}$ is already in its simplest form, since 13 and 24 have no common factors besides 1.

Therefore, $\frac{13}{24}$ of the total cleaning work was done in the morning.

\section*{Question 40}
\textbf{Metadata}

\begin{itemize}
  \item Question ID: P4-FrSubU12\_P2-FrCmp\_GPT4.1\_Services\_03
  \item Primary KC: FRACTIONS | Subtraction | subtracting unlike fractions with two different denominators not exceeding 12
  \item Secondary KC: FRACTIONS | Comparison and ordering | comparing and ordering fractions
  \item Topic: Services such as installation, maintenance, repairing, cleaning, laundry, e-commerce
  \item Grade: Primary 4
\end{itemize}

\textbf{Question}

A cleaning company completed $\frac{5}{6}$ of a spring-cleaning job in the morning. In the afternoon, the company finished $\frac{1}{4}$ of the same job. 

(a) How much more of the job did the company complete in the morning than in the afternoon? 

(b) After both sessions, which fraction is greater: the part completed in the morning or the part completed in the afternoon? Arrange the two fractions in order from least to greatest.

\textbf{Solution}

(a) To find out how much more of the job the company did in the morning than in the afternoon, subtract the fraction completed in the afternoon from that in the morning:

$\frac{5}{6} - \frac{1}{4}$

Find a common denominator. The lowest common denominator of 6 and 4 is 12.

$\frac{5}{6} = \frac{5 \times 2}{6 \times 2} = \frac{10}{12}$
$\frac{1}{4} = \frac{1 \times 3}{4 \times 3} = \frac{3}{12}$

Now, subtract:

$\frac{10}{12} - \frac{3}{12} = \frac{7}{12}$

The company completed $\frac{7}{12}$ more of the job in the morning than in the afternoon.

(b) To compare $\frac{1}{4}$ and $\frac{5}{6}$, convert to a common denominator as above:

$\frac{1}{4} = \frac{3}{12}$
$\frac{5}{6} = \frac{10}{12}$

$\frac{3}{12} < \frac{10}{12}$, so $\frac{1}{4} < \frac{5}{6}$.

Arranged in order from least to greatest:

$\frac{1}{4}, \frac{5}{6}$

Conclusion: The company completed a greater fraction of the job in the morning.

\section*{Question 41}
\textbf{Metadata}

\begin{itemize}
  \item Question ID: P4-FrSubU12\_P2-FrAdd2nd\_GPT4.1\_Services\_03
  \item Primary KC: FRACTIONS | Subtraction | subtracting unlike fractions with two different denominators not exceeding 12
  \item Secondary KC: FRACTIONS | Addition | adding fractions
  \item Topic: Services such as installation, maintenance, repairing, cleaning, laundry, e-commerce
  \item Grade: Primary 4
\end{itemize}

\textbf{Question}

A laundry service used $\frac{5}{6}$ of a bottle of washing liquid in the morning and $\frac{1}{4}$ of a bottle in the afternoon. After both washing sessions, they found that there was still $\frac{1}{2}$ of the bottle left. 

(a) What fraction of the bottle was used in total? 

(b) How much more washing liquid was used in the morning than in the afternoon?

\textbf{Solution}

(a) To find the total amount of washing liquid used, add the fractions for the morning and afternoon:

$\frac{5}{6} + \frac{1}{4}$

Find a common denominator for 6 and 4, which is 12:

$\frac{5}{6} = \frac{10}{12}$

$\frac{1}{4} = \frac{3}{12}$

So, $\frac{10}{12} + \frac{3}{12} = \frac{13}{12}$

But the question states there was $\frac{1}{2}$ of the bottle left, so the total used should be:

$1 - \frac{1}{2} = \frac{2}{2} - \frac{1}{2} = \frac{1}{2}$

But based on the sum $\frac{13}{12}$, it seems the bottle was larger or the values refer to portions of several bottles used. Let's revise the word problem so that the used fractions do not exceed 1 in total and are realistic for a single bottle:

Let me correct the arithmetic and the word problem:

Suppose: Morning: $\frac{2}{3}$, Afternoon: $\frac{1}{4}$.

Now continue:

$\frac{2}{3} + \frac{1}{4} = \frac{8}{12} + \frac{3}{12} = \frac{11}{12}$

(b) To find how much more washing liquid was used in the morning than in the afternoon:

$\frac{2}{3} - \frac{1}{4}$

$\frac{8}{12} - \frac{3}{12} = \frac{5}{12}$

**Final Answers:**

(a) The total fraction of the bottle used was $\frac{11}{12}$.

(b) The laundry service used $\frac{5}{12}$ more of the bottle in the morning than in the afternoon.

\section*{Question 42}
\textbf{Metadata}

\begin{itemize}
  \item Question ID: P4-FrSubU12\_P2-FrSub2nd\_GPT4.1\_Services\_03
  \item Primary KC: FRACTIONS | Subtraction | subtracting unlike fractions with two different denominators not exceeding 12
  \item Secondary KC: FRACTIONS | Subtraction | subtracting fractions
  \item Topic: Services such as installation, maintenance, repairing, cleaning, laundry, e-commerce
  \item Grade: Primary 4
\end{itemize}

\textbf{Question}

A washing service company uses two bottles of detergent for cleaning. In the morning, they used $\frac{5}{6}$ of a bottle to clean a set of uniforms. In the afternoon, they used $\frac{1}{4}$ of a bottle to clean curtains for another customer. How much more detergent did they use in the morning than in the afternoon?

\textbf{Solution}

To find how much more detergent was used in the morning than in the afternoon, subtract the afternoon fraction from the morning fraction: 

\[
\frac{5}{6} - \frac{1}{4}
\]

First, find a common denominator. The lowest common denominator of $6$ and $4$ is $12$.

Rewrite each fraction with denominator $12$:

\[
\frac{5}{6} = \frac{5 \times 2}{6 \times 2} = \frac{10}{12}
\]
\[
\frac{1}{4} = \frac{1 \times 3}{4 \times 3} = \frac{3}{12}
\]

Now, subtract:

\[
\frac{10}{12} - \frac{3}{12} = \frac{7}{12}
\]

So, the company used $\frac{7}{12}$ of a bottle more detergent in the morning than in the afternoon.

\section*{Question 43}
\textbf{Metadata}

\begin{itemize}
  \item Question ID: P4-FrSubU12\_P3-FrSmp\_GPT4.1\_Services\_03
  \item Primary KC: FRACTIONS | Subtraction | subtracting unlike fractions with two different denominators not exceeding 12
  \item Secondary KC: FRACTIONS | Simplifying | expressing a fraction in its simplest form
  \item Topic: Services such as installation, maintenance, repairing, cleaning, laundry, e-commerce
  \item Grade: Primary 4
\end{itemize}

\textbf{Question}

Aisha is cleaning her house. She spent $\frac{5}{6}$ of an hour cleaning the living room, and $\frac{1}{4}$ of an hour cleaning the bathroom. How much longer did she spend cleaning the living room than the bathroom? Give your answer in its simplest form.

\textbf{Solution}

To find how much longer Aisha spent on the living room, subtract the time she spent on the bathroom from the time she spent on the living room:

$\frac{5}{6} - \frac{1}{4}$

First, find the lowest common denominator (LCD) of $6$ and $4$, which is $12$.

Rewrite the fractions with denominator $12$:

$\frac{5}{6} = \frac{5 \times 2}{6 \times 2} = \frac{10}{12}$

$\frac{1}{4} = \frac{1 \times 3}{4 \times 3} = \frac{3}{12}$

Now subtract:

$\frac{10}{12} - \frac{3}{12} = \frac{7}{12}$

Therefore, Aisha spent $\frac{7}{12}$ of an hour longer cleaning the living room than the bathroom. The answer is $\frac{7}{12}$ hour.

\section*{Question 44}
\textbf{Metadata}

\begin{itemize}
  \item Question ID: P4-DcAdd2d\_P4-DcCmp3d\_GPT4.1\_Services\_03
  \item Primary KC: DECIMALS | Addition | adding decimals (up to 2 decimal places)
  \item Secondary KC: DECIMALS | Comparison and ordering | comparing and ordering decimals up to 3 decimal places
  \item Topic: Services such as installation, maintenance, repairing, cleaning, laundry, e-commerce
  \item Grade: Primary 4
\end{itemize}

\textbf{Question}

Maya’s family is comparing prices from three cleaning services to clean their house. The prices quoted per hour are $18.50$, $18.74$, and $18.615$. Maya wants to hire the service that offers the second lowest rate. She also plans to book that service for $2.75$ hours. What will be the total amount Maya has to pay?

\textbf{Solution}

First, compare the three prices per hour: $18.50$, $18.74$, and $18.615$.

Arrange them in order from lowest to highest:

$18.50 < 18.615 < 18.74$

The second lowest rate is $18.615$.

Now, calculate the total amount for $2.75$ hours at $18.615$ per hour:

$18.615 \times 2.75 = 51.19125$

Rounded to the nearest cent (since dollars and cents are used):

$51.19\textdollar$

So Maya will have to pay $51.19\textdollar$.

\section*{Question 45}
\textbf{Metadata}

\begin{itemize}
  \item Question ID: P4-DcAdd2d\_P4-DcCnv2Fr\_GPT4.1\_Services\_03
  \item Primary KC: DECIMALS | Addition | adding decimals (up to 2 decimal places)
  \item Secondary KC: DECIMALS | Conversion from decimals to fraction | expressing decimals as fractions
  \item Topic: Services such as installation, maintenance, repairing, cleaning, laundry, e-commerce
  \item Grade: Primary 4
\end{itemize}

\textbf{Question}

Mrs Lim hired a cleaning service to clean her living room and kitchen. The cleaning of the living room cost $25.75$ and the cleaning of the kitchen cost $17.40$. 

(a) What is the total amount Mrs Lim paid for the cleaning service?

(b) Express the total amount paid (from part a) as a fraction in its simplest form.

\textbf{Solution}

(a) $25.75 + 17.40 = 43.15$

The total amount Mrs Lim paid is $43.15$.

(b) $43.15$ can be written as $43 + 0.15$.

$0.15 = \frac{15}{100}$.

So, $43.15 = 43\frac{15}{100}$.

To simplify $\frac{15}{100}$:
- Find the highest common factor of $15$ and $100$, which is $5$.
- $\frac{15}{100} = \frac{3}{20}$

Therefore, the total amount paid as a fraction is $43\frac{3}{20}$.

\section*{Question 46}
\textbf{Metadata}

\begin{itemize}
  \item Question ID: P4-DcAdd2d\_P4-DcRnd3d\_GPT4.1\_Services\_03
  \item Primary KC: DECIMALS | Addition | adding decimals (up to 2 decimal places)
  \item Secondary KC: DECIMALS | Rounding | rounding decimals up to 3 decimal places to the nearest whole number, 1 decimal place and 2 decimal places 
  \item Topic: Services such as installation, maintenance, repairing, cleaning, laundry, e-commerce
  \item Grade: Primary 4
\end{itemize}

\textbf{Question}

A cleaning company charges $12.35$ per room to clean carpets and an extra $8.45$ per room for deep-cleaning services. Mrs. Lim hires the company to clean and deep-clean 2 rooms in her home. 

(a) What is the total cost for cleaning and deep-cleaning 1 room? 

(b) What is the total amount Mrs. Lim has to pay to clean and deep-clean 2 rooms? 

(c) Round the total amount found in (b) to the nearest whole number, to 1 decimal place, and to 2 decimal places.

\textbf{Solution}

Part (a): 
\[ \text{Total cost for 1 room} = \$12.35 + \$8.45 = \$20.80 \]

Part (b): 
\[ \text{Total for 2 rooms} = 2 \times \$20.80 = \$41.60 \]

Part (c): 
\[ \text{Nearest whole number: } \$41.60 \approx 42\textdollar \]
\[ \text{1 decimal place: } \$41.60 = 41.6\textdollar \]
\[ \text{2 decimal places: } \$41.60 = 41.60\textdollar \]

\section*{Question 47}
\textbf{Metadata}

\begin{itemize}
  \item Question ID: P4-DcAdd2d\_P4-DcAdd2nd\_GPT4.1\_Services\_03
  \item Primary KC: DECIMALS | Addition | adding decimals (up to 2 decimal places)
  \item Secondary KC: DECIMALS | Addition | adding decimals
  \item Topic: Services such as installation, maintenance, repairing, cleaning, laundry, e-commerce
  \item Grade: Primary 4
\end{itemize}

\textbf{Question}

Mrs Tan hired a cleaning service for her home. She was charged $42.75$ for the cleaning of her living room and $37.30$ for the cleaning of her kitchen. Later, she decided to add $15.60$ to clean her balcony as well. How much did Mrs Tan pay in total for all the cleaning services?

\textbf{Solution}

First, we add the charges for the living room and the kitchen:

$42.75 + 37.30 = 80.05$

Next, we add the cost for cleaning the balcony:

$80.05 + 15.60 = 95.65$

So, Mrs Tan paid a total of $95.65$ for all the cleaning services.

\section*{Question 48}
\textbf{Metadata}

\begin{itemize}
  \item Question ID: P4-DcSub2d\_P4-DcCmp3d\_GPT4.1\_Services\_03
  \item Primary KC: DECIMALS | Subtraction | subtracting decimals (up to 2 decimal places)
  \item Secondary KC: DECIMALS | Comparison and ordering | comparing and ordering decimals up to 3 decimal places
  \item Topic: Services such as installation, maintenance, repairing, cleaning, laundry, e-commerce
  \item Grade: Primary 4
\end{itemize}

\textbf{Question}

Amanda hired a cleaning service to clean her house. The total cost of cleaning was $72.80$. She used an online coupon to get a discount of $5.95$ from the total amount. 

After paying, she compared the prices of the cleaning service offered by four different companies for similar jobs:

\begin{itemize}
    \item SparkleClean: $66.900$
    \item FreshHome: $67.850$
    \item ShinePro: $66.500$
    \item Amanda's final price after discount
\end{itemize}

(a) How much did Amanda pay after using her coupon?

(b) Arrange all the prices (including Amanda's final price) from the lowest to the highest.

\textbf{Solution}

(a) Amanda paid:

$72.80 - 5.95 = 66.85$

So Amanda paid $66.85$ after the discount.

(b) List of prices:

\begin{itemize}
    \item SparkleClean: $66.900$
    \item FreshHome: $67.850$
    \item ShinePro: $66.500$
    \item Amanda: $66.850$
\end{itemize}

To arrange these from the lowest to the highest:

- ShinePro: $66.500$
- Amanda: $66.850$
- SparkleClean: $66.900$
- FreshHome: $67.850$

\section*{Question 49}
\textbf{Metadata}

\begin{itemize}
  \item Question ID: P4-DcSub2d\_P4-DcCnv2Fr\_GPT4.1\_Services\_03
  \item Primary KC: DECIMALS | Subtraction | subtracting decimals (up to 2 decimal places)
  \item Secondary KC: DECIMALS | Conversion from decimals to fraction | expressing decimals as fractions
  \item Topic: Services such as installation, maintenance, repairing, cleaning, laundry, e-commerce
  \item Grade: Primary 4
\end{itemize}

\textbf{Question}

Jenny hired a cleaning service to clean her house. The cleaning service charged Jenny $35.60$. After checking the invoice, Jenny noticed that she was supposed to get a discount of $12.25$. 

(a) How much did Jenny actually need to pay after the discount?

(b) Express the amount Jenny needed to pay as a fraction in simplest form.

\textbf{Solution}

(a) Amount Jenny needed to pay after the discount:

$35.60 - 12.25 = 23.35$

Jenny needed to pay $23.35$ after the discount.

(b) To express $23.35$ as a fraction:

$23.35 = 23 + 0.35$

$0.35 = \frac{35}{100} = \frac{7}{20}$

So, $23.35 = 23\frac{7}{20}$ or as an improper fraction:

$23\frac{7}{20} = \frac{23 \times 20 + 7}{20} = \frac{460 + 7}{20} = \frac{467}{20}$

Therefore, the amount Jenny needed to pay as a fraction in simplest form is $\frac{467}{20}$.

\section*{Question 50}
\textbf{Metadata}

\begin{itemize}
  \item Question ID: P4-DcSub2d\_P4-DcRnd3d\_GPT4.1\_Services\_03
  \item Primary KC: DECIMALS | Subtraction | subtracting decimals (up to 2 decimal places)
  \item Secondary KC: DECIMALS | Rounding | rounding decimals up to 3 decimal places to the nearest whole number, 1 decimal place and 2 decimal places 
  \item Topic: Services such as installation, maintenance, repairing, cleaning, laundry, e-commerce
  \item Grade: Primary 4
\end{itemize}

\textbf{Question}

Mrs Tan received two bills for air-conditioner cleaning services. The first bill was for $48.75$ and the second bill was for $32.89$. She decided to compare the two bills by subtracting the lower amount from the higher amount. After finding the difference, round the difference (1) to the nearest dollar, (2) to 1 decimal place, and (3) to 2 decimal places. What are the rounded values for the difference?

\textbf{Solution}

First, find the difference:

$48.75 - 32.89 = 15.86$

Now, round $15.86$ to the required forms:

(1) To the nearest whole number (dollar):

$15.86$ rounded to the nearest dollar is $16$.

(2) To 1 decimal place:

$15.86$ to 1 decimal place is $15.9$.

(3) To 2 decimal places:

$15.86$ to 2 decimal places remains $15.86$.

So, the answers are $16$, $15.9$, and $15.86$.

\section*{Question 51}
\textbf{Metadata}

\begin{itemize}
  \item Question ID: P4-DcSub2d\_P4-DcAdd2nd\_GPT4.1\_Services\_03
  \item Primary KC: DECIMALS | Subtraction | subtracting decimals (up to 2 decimal places)
  \item Secondary KC: DECIMALS | Addition | adding decimals
  \item Topic: Services such as installation, maintenance, repairing, cleaning, laundry, e-commerce
  \item Grade: Primary 4
\end{itemize}

\textbf{Question}

Mrs Tan hired a cleaning service for her house. She spent $19.75 for kitchen cleaning and $15.90 for window cleaning. After getting a discount, she paid $31.20 in total for these 2 services. How much money did Mrs Tan save with the discount?

\textbf{Solution}

First, add the original cost of the 2 services: 

$19.75 + 15.90 = 35.65$

Next, subtract the discounted price from the original total to find how much Mrs Tan saved:

$35.65 - 31.20 = 4.45$

So, Mrs Tan saved $4.45.

\section*{Question 52}
\textbf{Metadata}

\begin{itemize}
  \item Question ID: P4-DcSub2d\_P4-DcSub2nd\_GPT4.1\_Services\_03
  \item Primary KC: DECIMALS | Subtraction | subtracting decimals (up to 2 decimal places)
  \item Secondary KC: DECIMALS | Subtraction | subtracting decimals
  \item Topic: Services such as installation, maintenance, repairing, cleaning, laundry, e-commerce
  \item Grade: Primary 4
\end{itemize}

\textbf{Question}

Mrs Tan sent her curtains to a laundry shop. The total bill was $24.80$. She used a discount coupon worth $6.55$ to reduce her bill. How much did Mrs Tan pay after using the coupon?

\textbf{Solution}

To find how much Mrs Tan paid after using the coupon, subtract the value of the coupon from the total bill:

$24.80 - 6.55 = 18.25$

So, Mrs Tan paid $18.25$ after using the coupon.

\section*{Question 53}
\textbf{Metadata}

\begin{itemize}
  \item Question ID: P4-DcMul2d1d\_P4-DcCmp3d\_GPT4.1\_Services\_02
  \item Primary KC: DECIMALS | Multiplication | multiplying decimals (up to 2 decimal places) by a 1-digit whole number
  \item Secondary KC: DECIMALS | Comparison and ordering | comparing and ordering decimals up to 3 decimal places
  \item Topic: Services such as installation, maintenance, repairing, cleaning, laundry, e-commerce
  \item Grade: Primary 4
\end{itemize}

\textbf{Question}

A laundry shop charges \$2.35 to clean one shirt. Mr. Lee sends 7 shirts to be cleaned. 

After cleaning, the shop records the weights of the cleaned shirts: $0.946\ \text{kg}$, $0.972\ \text{kg}$, $0.959\ \text{kg}$, $0.981\ \text{kg}$, $0.950\ \text{kg}$, $0.966\ \text{kg}$, and $0.955\ \text{kg}$. 

(a) How much does Mr. Lee need to pay to clean all 7 shirts? 

(b) List the weights of the shirts in order from lightest to heaviest.

\textbf{Solution}

(a) To find the total cost, multiply the cost of cleaning 1 shirt by the number of shirts:

$2.35 \times 7 = 16.45$

So, Mr. Lee needs to pay \$16.45.

(b) To order the weights from lightest to heaviest, compare the decimal numbers:

$0.946$ kg, $0.950$ kg, $0.955$ kg, $0.959$ kg, $0.966$ kg, $0.972$ kg, $0.981$ kg

Order: 

$0.946$ kg, $0.950$ kg, $0.955$ kg, $0.959$ kg, $0.966$ kg, $0.972$ kg, $0.981$ kg

\section*{Question 54}
\textbf{Metadata}

\begin{itemize}
  \item Question ID: P4-DcMul2d1d\_P4-DcCnv2Fr\_GPT4.1\_Services\_02
  \item Primary KC: DECIMALS | Multiplication | multiplying decimals (up to 2 decimal places) by a 1-digit whole number
  \item Secondary KC: DECIMALS | Conversion from decimals to fraction | expressing decimals as fractions
  \item Topic: Services such as installation, maintenance, repairing, cleaning, laundry, e-commerce
  \item Grade: Primary 4
\end{itemize}

\textbf{Question}

A laundry shop charges $2.75$ dollars to wash one shirt. Mark wants to wash $4$ shirts. 

(a) How much will Mark pay in total? 

(b) Express the amount Mark pays for washing one shirt as a fraction in its simplest form.

\textbf{Solution}

(a) To find out how much Mark will pay in total, multiply the cost of washing one shirt by the number of shirts:

\[
2.75 \times 4 = 11.00
\]

So, Mark will pay \textdollar 11 in total.

(b) To express $2.75$ as a fraction:

\[
2.75 = 2 + 0.75 = 2 + \frac{75}{100}
\]

Simplify $\frac{75}{100}$ by dividing numerator and denominator by 25:

\[
\frac{75}{100} = \frac{3}{4}
\]

So, $2.75 = 2\frac{3}{4}$ as a mixed number, or $\frac{11}{4}$ as an improper fraction.

Final answers:

(a) \textdollar 11

(b) $2.75 = \frac{11}{4}$

\section*{Question 55}
\textbf{Metadata}

\begin{itemize}
  \item Question ID: P4-DcMul2d1d\_P4-DcRnd3d\_GPT4.1\_Services\_02
  \item Primary KC: DECIMALS | Multiplication | multiplying decimals (up to 2 decimal places) by a 1-digit whole number
  \item Secondary KC: DECIMALS | Rounding | rounding decimals up to 3 decimal places to the nearest whole number, 1 decimal place and 2 decimal places 
  \item Topic: Services such as installation, maintenance, repairing, cleaning, laundry, e-commerce
  \item Grade: Primary 4
\end{itemize}

\textbf{Question}

A cleaning company charges $8.75$ dollars per hour to clean an office. Mrs. Tan booked the company for $4$ hours. 

(a) How much will Mrs. Tan have to pay in total for $4$ hours of cleaning?

(b) Round the total amount Mrs. Tan has to pay to the nearest dollar, to 1 decimal place, and to 2 decimal places.

\textbf{Solution}

(a) The cost per hour is $8.75$ dollars. For $4$ hours:

$8.75 \times 4 = 35.00$

So, Mrs. Tan has to pay $35\textdollar$ in total.

(b) Rounding $35.00$ to:
- The nearest dollar: $35\textdollar$
- 1 decimal place: $35.0\textdollar$
- 2 decimal places: $35.00\textdollar$

\section*{Question 56}
\textbf{Metadata}

\begin{itemize}
  \item Question ID: P4-DcMul2d1d\_P4-DcAdd2nd\_GPT4.1\_Services\_02
  \item Primary KC: DECIMALS | Multiplication | multiplying decimals (up to 2 decimal places) by a 1-digit whole number
  \item Secondary KC: DECIMALS | Addition | adding decimals
  \item Topic: Services such as installation, maintenance, repairing, cleaning, laundry, e-commerce
  \item Grade: Primary 4
\end{itemize}

\textbf{Question}

A cleaning company charges $3.75$ for cleaning one window. Mrs. Lee wants all $4$ windows in her living room cleaned. In addition, she wants to get her balcony cleaned, which costs $5.50$. 

How much will Mrs. Lee pay in total for cleaning her living room windows and balcony?

\textbf{Solution}

First, find the cost of cleaning $4$ windows:

$3.75 \times 4 = 15.00$

So, the cost for cleaning the windows is $15.00$.

Next, add the cost for cleaning the balcony:

$15.00 + 5.50 = 20.50$

Therefore, Mrs. Lee will pay $20.50$ in total for the cleaning services.

\section*{Question 57}
\textbf{Metadata}

\begin{itemize}
  \item Question ID: P4-DcMul2d1d\_P4-DcSub2nd\_GPT4.1\_Services\_02
  \item Primary KC: DECIMALS | Multiplication | multiplying decimals (up to 2 decimal places) by a 1-digit whole number
  \item Secondary KC: DECIMALS | Subtraction | subtracting decimals
  \item Topic: Services such as installation, maintenance, repairing, cleaning, laundry, e-commerce
  \item Grade: Primary 4
\end{itemize}

\textbf{Question}

Amira runs a laundry service. She charges \$2.35 to wash one shirt. Yesterday, she washed 6 shirts for a customer. The customer had a \$20 note and paid for the washing. After paying, how much money did the customer have left?

\textbf{Solution}

First, calculate the total cost to wash 6 shirts:

$2.35 \times 6 = 14.10$

So, it costs \$14.10 to wash 6 shirts.

Next, subtract this amount from the \$20 the customer had:

$20.00 - 14.10 = 5.90$

The customer had \$5.90 left after paying.

\section*{Question 58}
\textbf{Metadata}

\begin{itemize}
  \item Question ID: P4-DcDiv2d1d\_P4-DcCmp3d\_GPT4.1\_Services\_02
  \item Primary KC: DECIMALS | Division | dividing decimals (up to 2 decimal places) by a 1-digit whole number
  \item Secondary KC: DECIMALS | Comparison and ordering | comparing and ordering decimals up to 3 decimal places
  \item Topic: Services such as installation, maintenance, repairing, cleaning, laundry, e-commerce
  \item Grade: Primary 4
\end{itemize}

\textbf{Question}

A cleaning company charges $\textdollar21.60$ for a service that lasts 3 hours. During a busy week, they provided 5 cleaning services, each with total durations of $2.35$ hours, $2.9$ hours, $3.05$ hours, $3.5$ hours, and $2.85$ hours respectively.\
(a) How much does the company charge for 1 hour of service?\
(b) Arrange the five service durations in order from the shortest to the longest.\
(c) Which service took longer: the one on Monday ($2.9$ hours) or the one on Friday ($3.05$ hours)?

\textbf{Solution}

(a) To find out how much the company charges for 1 hour of service, divide $\textdollar21.60$ by 3: 

$\textdollar21.60 \div 3 = \textdollar7.20$ per hour.

(b) Comparing $2.35$, $2.9$, $3.05$, $3.5$, and $2.85$, we arrange from smallest to largest:

$2.35 < 2.85 < 2.9 < 3.05 < 3.5$

So, the order is $2.35$, $2.85$, $2.9$, $3.05$, $3.5$ hours.

(c) $2.9$ hours (Monday) vs $3.05$ hours (Friday):

$3.05 > 2.9$

So, the service on Friday ($3.05$ hours) took longer than the one on Monday ($2.9$ hours).

\section*{Question 59}
\textbf{Metadata}

\begin{itemize}
  \item Question ID: P4-DcDiv2d1d\_P4-DcCnv2Fr\_GPT4.1\_Services\_02
  \item Primary KC: DECIMALS | Division | dividing decimals (up to 2 decimal places) by a 1-digit whole number
  \item Secondary KC: DECIMALS | Conversion from decimals to fraction | expressing decimals as fractions
  \item Topic: Services such as installation, maintenance, repairing, cleaning, laundry, e-commerce
  \item Grade: Primary 4
\end{itemize}

\textbf{Question}

Maya runs a laundry service. She received $18.60 from a customer for washing and ironing a number of shirts. If Maya charges the same amount for each shirt, and the total amount received is divided equally among 6 shirts, how much did Maya charge for each shirt? Express your answer as a decimal and then as a fraction in simplest form.

\textbf{Solution}

Total amount received is $18.60$.

Number of shirts = $6$

Amount charged per shirt = $\frac{18.60}{6} = 3.10$

So, Maya charged $3.10$ per shirt.

Now, to express $3.10$ as a fraction:

$3.10 = 3 + 0.10$

$0.10 = \frac{10}{100} = \frac{1}{10}$

So, $3.10 = 3\frac{1}{10}$

In improper fraction, $3.10 = \frac{31}{10}$.

Therefore, Maya charged $3.10$ per shirt or $\frac{31}{10}$ in fraction.

\section*{Question 60}
\textbf{Metadata}

\begin{itemize}
  \item Question ID: P4-DcDiv2d1d\_P4-DcRnd3d\_GPT4.1\_Services\_02
  \item Primary KC: DECIMALS | Division | dividing decimals (up to 2 decimal places) by a 1-digit whole number
  \item Secondary KC: DECIMALS | Rounding | rounding decimals up to 3 decimal places to the nearest whole number, 1 decimal place and 2 decimal places 
  \item Topic: Services such as installation, maintenance, repairing, cleaning, laundry, e-commerce
  \item Grade: Primary 4
\end{itemize}

\textbf{Question}

A cleaning company charges $\textdollar 63.75$ to clean 5 identical apartments. The cost is shared equally among the 5 apartment owners. 

(a) What is the cost for each apartment owner to clean their apartment? 

(b) Round the answer in part (a) to the nearest whole number of dollars.

\textbf{Solution}

Let the total cost for cleaning be $\textdollar 63.75$ and the number of apartments be $5$.

(a) The cost for each apartment owner is: 

\[
\frac{63.75}{5} = 12.75
\]

So, each apartment owner pays $\textdollar 12.75$.

(b) Rounding $12.75$ to the nearest whole number:
\[
\text{Since } 12.75 \geq 12.5, \text{we round up to } 13.
\]
So, the cost rounded to the nearest whole number is $\textdollar 13$.

\section*{Question 61}
\textbf{Metadata}

\begin{itemize}
  \item Question ID: P4-DcDiv2d1d\_P4-DcAdd2nd\_GPT4.1\_Services\_02
  \item Primary KC: DECIMALS | Division | dividing decimals (up to 2 decimal places) by a 1-digit whole number
  \item Secondary KC: DECIMALS | Addition | adding decimals
  \item Topic: Services such as installation, maintenance, repairing, cleaning, laundry, e-commerce
  \item Grade: Primary 4
\end{itemize}

\textbf{Question}

Anna wants to wash her curtains at a laundry service. She has 3 different sets of curtains that weigh $2.85\,\text{kg}$, $1.6\,\text{kg}$, and $3.55\,\text{kg}$ respectively. The laundry shop charges $\textdollar4.50$ for every $2$ kg of curtains. Anna asks to divide the total weight of her curtains equally among $3$ machines. How much weight of curtains will each machine have? Also, what is the total cost Anna needs to pay to wash all her curtains?

\textbf{Solution}

First, add the weights of all the curtains: 

\[
2.85 + 1.6 + 3.55 = 8.00\,\text{kg}
\]

Next, divide the total weight by $3$ to find the weight for each machine: 

\[
\frac{8.00}{3} = 2.67\,\text{kg}
\]

Each machine will have $2.67\,\text{kg}$ of curtains.

To find the cost:
- Every $2$ kg costs $\textdollar4.50$.
- Anna has $8.00\,\text{kg}$ in total.
- Number of $2$ kg portions in $8.00\,\text{kg}$:

\[
\frac{8.00}{2} = 4
\]

Total cost: 

\[
4 \times \textdollar4.50 = \textdollar18
\]

Anna needs to pay $\textdollar18$ to wash all her curtains.

\section*{Question 62}
\textbf{Metadata}

\begin{itemize}
  \item Question ID: P4-DcDiv2d1d\_P4-DcSub2nd\_GPT4.1\_Services\_02
  \item Primary KC: DECIMALS | Division | dividing decimals (up to 2 decimal places) by a 1-digit whole number
  \item Secondary KC: DECIMALS | Subtraction | subtracting decimals
  \item Topic: Services such as installation, maintenance, repairing, cleaning, laundry, e-commerce
  \item Grade: Primary 4
\end{itemize}

\textbf{Question}

A cleaning company charged Mrs Lim $28.80$ for cleaning her house. She also used $6.40$ for laundry services. If Mrs Lim wants to split the total cost of both services equally among her 4 family members, how much does each person have to pay?

\textbf{Solution}

First, find the total cost Mrs Lim spent on both services:

$28.80 + 6.40 = 35.20$.

Next, divide the total cost by 4 to find out how much each person pays:

$35.20 \div 4 = 8.80$.

So, each person has to pay $\textdollar8.80$.

\section*{Question 63}
\textbf{Metadata}

\begin{itemize}
  \item Question ID: P5-FrAddMix\_P2-FrCmp\_GPT4.1\_Services\_02
  \item Primary KC: FRACTIONS | Addition | adding mixed numbers
  \item Secondary KC: FRACTIONS | Comparison and ordering | comparing and ordering fractions
  \item Topic: Services such as installation, maintenance, repairing, cleaning, laundry, e-commerce
  \item Grade: Primary 5
\end{itemize}

\textbf{Question}

Peter is a technician who installs air-conditioners. On Monday, he spent $2\frac{1}{3}$ hours installing an air-conditioner at Mr. Tan's house and $1\frac{3}{4}$ hours repairing another unit at Mrs. Lee's apartment. On Tuesday, he spent $3\frac{1}{8}$ hours maintaining air-conditioners at a school. 

(a) How much time did Peter spend on installations and repairs on Monday?

(b) On which day did Peter spend more time working? Arrange the total hours he worked each day in ascending order.

\textbf{Solution}

(a) Total time spent on Monday: $2\frac{1}{3} + 1\frac{3}{4}$

Convert to improper fractions:
$2\frac{1}{3} = \frac{7}{3}$
$1\frac{3}{4} = \frac{7}{4}$

Find the common denominator (12):
$\frac{7}{3} = \frac{28}{12}$
$\frac{7}{4} = \frac{21}{12}$

Add: $\frac{28}{12} + \frac{21}{12} = \frac{49}{12}$

Convert back to mixed number:
$\frac{49}{12} = 4\frac{1}{12}$ hours

(b) Time spent on Tuesday: $3\frac{1}{8} = \frac{25}{8}$

Convert $\frac{49}{12}$ and $\frac{25}{8}$ to decimals to compare:
$\frac{49}{12} \approx 4.08$ hours 
$\frac{25}{8} = 3.125$ hours

So, Peter spent more time working on Monday.

Arranging the total hours in ascending order:
Tuesday ($3\frac{1}{8}$ hours), Monday ($4\frac{1}{12}$ hours)

\section*{Question 64}
\textbf{Metadata}

\begin{itemize}
  \item Question ID: P5-FrAddMix\_P2-FrAdd2nd\_GPT4.1\_Services\_02
  \item Primary KC: FRACTIONS | Addition | adding mixed numbers
  \item Secondary KC: FRACTIONS | Addition | adding fractions
  \item Topic: Services such as installation, maintenance, repairing, cleaning, laundry, e-commerce
  \item Grade: Primary 5
\end{itemize}

\textbf{Question}

Mrs Lim hired a cleaning service to clean her house. The workers spent $2\frac{1}{4}$ hours cleaning the kitchen and $1\frac{2}{3}$ hours cleaning the living room. After finishing both, they took an additional $\frac{5}{6}$ hour to tidy up the bedrooms. 

How many hours did the workers spend cleaning the entire house?

\textbf{Solution}

First, add the two mixed numbers:

$2\frac{1}{4} + 1\frac{2}{3}$

Convert to improper fractions:
$2\frac{1}{4} = \frac{9}{4}$
$1\frac{2}{3} = \frac{5}{3}$

Find common denominator (12):
$\frac{9}{4} = \frac{27}{12}$
$\frac{5}{3} = \frac{20}{12}$

Add:
$\frac{27}{12} + \frac{20}{12} = \frac{47}{12}$

Now, add $\frac{5}{6}$ (bedrooms):
Convert $\frac{5}{6}$ to twelfths: $\frac{5}{6} = \frac{10}{12}$

$\frac{47}{12} + \frac{10}{12} = \frac{57}{12}$

Simplify $\frac{57}{12}$:
$57 \div 12 = 4$ remainder $9$
So, $\frac{57}{12} = 4\frac{9}{12} = 4\frac{3}{4}$

The workers spent $4\frac{3}{4}$ hours cleaning the entire house.

\section*{Question 65}
\textbf{Metadata}

\begin{itemize}
  \item Question ID: P5-FrAddMix\_P3-FrSmp\_GPT4.1\_Services\_02
  \item Primary KC: FRACTIONS | Addition | adding mixed numbers
  \item Secondary KC: FRACTIONS | Simplifying | expressing a fraction in its simplest form
  \item Topic: Services such as installation, maintenance, repairing, cleaning, laundry, e-commerce
  \item Grade: Primary 5
\end{itemize}

\textbf{Question}

Maya is helping her mother do laundry at home. She spends $1\frac{1}{4}$ hours washing clothes and $2\frac{2}{3}$ hours ironing them. 

What is the total time, in hours, Maya spends on laundry? Express your answer as a mixed number in its simplest form.

\textbf{Solution}

To find the total time Maya spends, add the mixed numbers:

$1\frac{1}{4} + 2\frac{2}{3}$

Convert both to improper fractions:

$1\frac{1}{4} = \frac{5}{4}$
$2\frac{2}{3} = \frac{8}{3}$

Next, find a common denominator. The least common multiple of 4 and 3 is 12.

$\frac{5}{4} = \frac{15}{12}$
$\frac{8}{3} = \frac{32}{12}$

Add the fractions:

$\frac{15}{12} + \frac{32}{12} = \frac{47}{12}$

Convert $\frac{47}{12}$ to a mixed number:

$47 \div 12 = 3$ remainder $11$

So, $\frac{47}{12} = 3\frac{11}{12}$

Maya spends a total of $3\frac{11}{12}$ hours on laundry.

\section*{Question 66}
\textbf{Metadata}

\begin{itemize}
  \item Question ID: P5-FrAddMix\_P5-FrCnv2Dc\_GPT4.1\_Services\_02
  \item Primary KC: FRACTIONS | Addition | adding mixed numbers
  \item Secondary KC: FRACTIONS | Conversion to decimals | expressing fractions as decimals
  \item Topic: Services such as installation, maintenance, repairing, cleaning, laundry, e-commerce
  \item Grade: Primary 5
\end{itemize}

\textbf{Question}

A washing machine repairman spent $2\frac{1}{2}$ hours fixing one washing machine and $1\frac{3}{4}$ hours servicing another on the same day. 

(a) How many hours did he spend working in total? Give your answer as a mixed number.

(b) Express the total time in part (a) as a decimal, correct to 2 decimal places.

\textbf{Solution}

(a) First, add the mixed numbers:

$2\frac{1}{2} + 1\frac{3}{4}$

Convert to improper fractions:
$2\frac{1}{2} = \frac{5}{2}$

$1\frac{3}{4} = \frac{7}{4}$

Find a common denominator (4):

$\frac{5}{2} = \frac{10}{4}$

So, $\frac{10}{4} + \frac{7}{4} = \frac{17}{4}$

Convert $\frac{17}{4}$ back to a mixed number:

$17 \div 4 = 4$ remainder $1$, so $4\frac{1}{4}$

He spent $4\frac{1}{4}$ hours in total.

(b) Express $4\frac{1}{4}$ as a decimal:

$1\div 4 = 0.25$, so $4\frac{1}{4} = 4.25$

Final answer: $4.25$ hours.

\section*{Question 67}
\textbf{Metadata}

\begin{itemize}
  \item Question ID: P5-FrSubMix\_P2-FrCmp\_GPT4.1\_Services\_02
  \item Primary KC: FRACTIONS | Subtraction | subtracting mixed numbers
  \item Secondary KC: FRACTIONS | Comparison and ordering | comparing and ordering fractions
  \item Topic: Services such as installation, maintenance, repairing, cleaning, laundry, e-commerce
  \item Grade: Primary 5
\end{itemize}

\textbf{Question}

Mrs Lim is hiring a cleaning company to clean her house. The company spent $3\dfrac{2}{3}$ hours cleaning the bedrooms and $2\dfrac{5}{6}$ hours cleaning the living room. 

(a) How much longer did they spend cleaning the bedrooms than the living room? 

(b) After comparing, did the company spend more time cleaning the living room or the bedrooms, and by how much?

\textbf{Solution}

(a) To find how much longer was spent on the bedrooms, subtract the time spent on the living room from the time spent on the bedrooms:

$$3\dfrac{2}{3} - 2\dfrac{5}{6}$$

First, convert the mixed numbers to improper fractions:

$3\dfrac{2}{3} = \dfrac{11}{3}$

$2\dfrac{5}{6} = \dfrac{17}{6}$

Rewrite $\dfrac{11}{3}$ with denominator 6:

$\dfrac{11}{3} = \dfrac{22}{6}$

Subtract:

$\dfrac{22}{6} - \dfrac{17}{6} = \dfrac{5}{6}$

So, they spent $\dfrac{5}{6}$ hour longer cleaning the bedrooms.

(b) Since $3\dfrac{2}{3} > 2\dfrac{5}{6}$, the company spent more time cleaning the bedrooms. The difference, as calculated, is $\dfrac{5}{6}$ hour.

\section*{Question 68}
\textbf{Metadata}

\begin{itemize}
  \item Question ID: P5-FrSubMix\_P2-FrAdd2nd\_GPT4.1\_Services\_02
  \item Primary KC: FRACTIONS | Subtraction | subtracting mixed numbers
  \item Secondary KC: FRACTIONS | Addition | adding fractions
  \item Topic: Services such as installation, maintenance, repairing, cleaning, laundry, e-commerce
  \item Grade: Primary 5
\end{itemize}

\textbf{Question}

Mrs Tan spent $5\dfrac{1}{2}$ hours over two days cleaning her house. On the first day, she spent $2\dfrac{3}{4}$ hours cleaning the living room and the kitchen. On the second day, she spent $1\dfrac{2}{3}$ hours cleaning the bedrooms. 

(a) What was the total number of hours Mrs Tan spent cleaning on the first and second day altogether?

(b) How much more time did she spend cleaning on both days than on just the first day?

\textbf{Solution}

(a) Total number of hours spent cleaning on the first and second day is:

\[
2\dfrac{3}{4} + 1\dfrac{2}{3} = \frac{11}{4} + \frac{5}{3}
\]
To add these, find a common denominator (12):

\[
\frac{11}{4} = \frac{33}{12},\quad \frac{5}{3} = \frac{20}{12}
\]
\[
\frac{33}{12} + \frac{20}{12} = \frac{53}{12} = 4\dfrac{5}{12}
\]

So, Mrs Tan spent $4\dfrac{5}{12}$ hours cleaning on both days altogether.

(b) The total time spent cleaning over two days was $5\dfrac{1}{2}$ hours. She spent $2\dfrac{3}{4}$ hours on the first day.

\[
5\dfrac{1}{2} - 2\dfrac{3}{4} = \frac{11}{2} - \frac{11}{4}
\]
Find a common denominator (4):
\[
\frac{11}{2} = \frac{22}{4},\quad \frac{11}{4}
\]
\[
\frac{22}{4} - \frac{11}{4} = \frac{11}{4} = 2\dfrac{3}{4}
\]

She spent $2\dfrac{3}{4}$ hours more cleaning over two days than just on the first day.

\section*{Question 69}
\textbf{Metadata}

\begin{itemize}
  \item Question ID: P5-FrSubMix\_P2-FrSub2nd\_GPT4.1\_Services\_02
  \item Primary KC: FRACTIONS | Subtraction | subtracting mixed numbers
  \item Secondary KC: FRACTIONS | Subtraction | subtracting fractions
  \item Topic: Services such as installation, maintenance, repairing, cleaning, laundry, e-commerce
  \item Grade: Primary 5
\end{itemize}

\textbf{Question}

A washing machine repairman was fixing two washing machines in a day. For the first washing machine, he spent $2 \frac{1}{2}$ hours. For the second washing machine, he spent $1 \frac{3}{4}$ hours. 

(a) How many more hours did the repairman spend on the first washing machine than on the second washing machine? Give your answer in hours and fractions of an hour.

(b) If he took a $\frac{1}{2}$-hour break between the two jobs, what is the difference in total time spent working (excluding the break) and the total time including the break?

\textbf{Solution}

(a) Time spent on the first washing machine: $2 \frac{1}{2} = \frac{5}{2}$ hours.

Time spent on the second washing machine: $1 \frac{3}{4} = \frac{7}{4}$ hours.

To find the difference: $\frac{5}{2} - \frac{7}{4}$

First, convert $\frac{5}{2}$ to quarters: $\frac{5}{2} = \frac{10}{4}$.

Now subtract:
$\frac{10}{4} - \frac{7}{4} = \frac{3}{4}$

So, the repairman spent $\frac{3}{4}$ hour more on the first washing machine than the second.

(b) Total time working (excluding break): $2 \frac{1}{2} + 1 \frac{3}{4}$

$2 \frac{1}{2} = \frac{5}{2}$, $1 \frac{3}{4} = \frac{7}{4}$

Convert $\frac{5}{2}$ to quarters: $\frac{5}{2} = \frac{10}{4}$

Add: $\frac{10}{4} + \frac{7}{4} = \frac{17}{4} = 4 \frac{1}{4}$ hours

Total time including the $\frac{1}{2}$-hour break: $4 \frac{1}{4} + \frac{1}{2}$

$4 \frac{1}{4} = \frac{17}{4}$, $\frac{1}{2} = \frac{2}{4}$

Sum: $\frac{17}{4} + \frac{2}{4} = \frac{19}{4} = 4 \frac{3}{4}$ hours

Difference between total time including break and total time excluding break:
$4 \frac{3}{4} - 4 \frac{1}{4} = \frac{19}{4} - \frac{17}{4} = \frac{2}{4} = \frac{1}{2}$ hour

So, the difference is $\frac{1}{2}$ hour, which is the length of the break.

\section*{Question 70}
\textbf{Metadata}

\begin{itemize}
  \item Question ID: P5-FrSubMix\_P3-FrSmp\_GPT4.1\_Services\_02
  \item Primary KC: FRACTIONS | Subtraction | subtracting mixed numbers
  \item Secondary KC: FRACTIONS | Simplifying | expressing a fraction in its simplest form
  \item Topic: Services such as installation, maintenance, repairing, cleaning, laundry, e-commerce
  \item Grade: Primary 5
\end{itemize}

\textbf{Question}

Mdm Wong hired a cleaning service to clean her two rooms. The cleaners spent $2\dfrac{3}{4}$ hours on the first room and $1\dfrac{2}{5}$ hours on the second room. How much longer did they spend cleaning the first room than the second room? Give your answer in its simplest form.

\textbf{Solution}

First, rewrite the mixed numbers as improper fractions:

$2\dfrac{3}{4} = \frac{2 \times 4 + 3}{4} = \frac{8 + 3}{4} = \frac{11}{4}$

$1\dfrac{2}{5} = \frac{1 \times 5 + 2}{5} = \frac{5 + 2}{5} = \frac{7}{5}$

To subtract, we need a common denominator. The LCM of $4$ and $5$ is $20$.

$\frac{11}{4} = \frac{11 \times 5}{4 \times 5} = \frac{55}{20}$

$\frac{7}{5} = \frac{7 \times 4}{5 \times 4} = \frac{28}{20}$

Now subtract:

$\frac{55}{20} - \frac{28}{20} = \frac{27}{20}$

Express $\frac{27}{20}$ as a mixed number:

$\frac{27}{20} = 1\dfrac{7}{20}$

Since $7$ and $20$ have no common factors other than $1$, $\frac{7}{20}$ is already in its simplest form.

\textbf{Answer:} The cleaners spent $1\dfrac{7}{20}$ hours longer cleaning the first room than the second room.

\section*{Question 71}
\textbf{Metadata}

\begin{itemize}
  \item Question ID: P5-FrSubMix\_P5-FrCnv2Dc\_GPT4.1\_Services\_02
  \item Primary KC: FRACTIONS | Subtraction | subtracting mixed numbers
  \item Secondary KC: FRACTIONS | Conversion to decimals | expressing fractions as decimals
  \item Topic: Services such as installation, maintenance, repairing, cleaning, laundry, e-commerce
  \item Grade: Primary 5
\end{itemize}

\textbf{Question}

A laundry service cleaned $7\frac{3}{4}$ kg of clothes on Monday and $5\frac{2}{5}$ kg of clothes on Tuesday. 

(a) How many more kilograms of clothes did the service clean on Monday than on Tuesday?

(b) Express your answer in part (a) as a decimal, correct to 2 decimal places.

\textbf{Solution}

First, subtract the mixed numbers:

Monday: $7\frac{3}{4} = 7 + \frac{3}{4} = 7.75$

Tuesday: $5\frac{2}{5} = 5 + \frac{2}{5} = 5 + 0.4 = 5.4$

(a) $7\frac{3}{4} - 5\frac{2}{5}$

Let’s subtract using fractions:

Convert both mixed numbers to improper fractions:
$7\frac{3}{4} = \frac{7 \times 4 + 3}{4} = \frac{28 + 3}{4} = \frac{31}{4}$

$5\frac{2}{5} = \frac{5 \times 5 + 2}{5} = \frac{25 + 2}{5} = \frac{27}{5}$

Find common denominator (20):

$\frac{31}{4} = \frac{31 \times 5}{4 \times 5} = \frac{155}{20}$

$\frac{27}{5} = \frac{27 \times 4}{5 \times 4} = \frac{108}{20}$

Subtract:
$\frac{155}{20} - \frac{108}{20} = \frac{47}{20}$ kg

(b) Express $\frac{47}{20}$ as a decimal:

$\frac{47}{20} = 2.35$

So, the answer to part (a) is $\frac{47}{20}$ kg, and to part (b): $2.35$ kg.

\section*{Question 72}
\textbf{Metadata}

\begin{itemize}
  \item Question ID: P5-FrMulImN\_P2-FrCmp\_GPT4.1\_Services\_02
  \item Primary KC: FRACTIONS | Multiplication | multiplying a proper/improper fraction and a whole number
  \item Secondary KC: FRACTIONS | Comparison and ordering | comparing and ordering fractions
  \item Topic: Services such as installation, maintenance, repairing, cleaning, laundry, e-commerce
  \item Grade: Primary 5
\end{itemize}

\textbf{Question}

A cleaning company charges $\frac{3}{4}$ of the usual price for each chair if more than 10 chairs are cleaned. The usual price for cleaning one chair is $8\textdollar$. 

Mr. Tan wants 16 chairs cleaned. 

1. What is the total amount, in dollars, Mr. Tan has to pay for cleaning the 16 chairs at the discounted rate?

2. If another company charges $\frac{5}{6}$ of the usual price per chair for cleaning any number of chairs, which company's rate per chair is cheaper? Arrange the two rates per chair from the lowest to the highest.

\textbf{Solution}

1. The discounted price per chair is $\frac{3}{4} \times 8 = 6\textdollar$.

So, for 16 chairs: $16 \times 6\textdollar = 96\textdollar$.

Mr. Tan has to pay \( 96\textdollar \).

2. The other company's price per chair is $\frac{5}{6} \times 8 = \frac{40}{6} = 6 \frac{2}{3}\textdollar$.

Compare $6\textdollar$ and $6\frac{2}{3}\textdollar$. Since $6 < 6\frac{2}{3}$, the first company's rate is cheaper.

Arranged from lowest to highest:

$6\textdollar$ (First company), $6 \frac{2}{3}\textdollar$ (Second company).

\section*{Question 73}
\textbf{Metadata}

\begin{itemize}
  \item Question ID: P5-FrMulImN\_P2-FrAdd2nd\_GPT4.1\_Services\_02
  \item Primary KC: FRACTIONS | Multiplication | multiplying a proper/improper fraction and a whole number
  \item Secondary KC: FRACTIONS | Addition | adding fractions
  \item Topic: Services such as installation, maintenance, repairing, cleaning, laundry, e-commerce
  \item Grade: Primary 5
\end{itemize}

\textbf{Question}

Ali is helping his mother do the laundry. Each load of laundry uses \( \frac{2}{5} \) cups of detergent. On Monday, he does 3 loads in the morning and 2 more loads in the evening. How many cups of detergent does Ali use in total on Monday?

\textbf{Solution}

First, find the total number of loads Ali did: \( 3 + 2 = 5 \) loads. Each load uses \( \frac{2}{5} \) cups, so total detergent used is \( 5 \times \frac{2}{5} = \frac{10}{5} = 2 \) cups. 

Alternatively, calculate the detergent for each session and add: 
- Morning: \( 3 \times \frac{2}{5} = \frac{6}{5} \) cups
- Evening: \( 2 \times \frac{2}{5} = \frac{4}{5} \) cups

Total: \( \frac{6}{5} + \frac{4}{5} = \frac{10}{5} = 2 \) cups. 

Ali used a total of 2 cups of detergent on Monday.

\section*{Question 74}
\textbf{Metadata}

\begin{itemize}
  \item Question ID: P5-FrMulImN\_P2-FrSub2nd\_GPT4.1\_Services\_02
  \item Primary KC: FRACTIONS | Multiplication | multiplying a proper/improper fraction and a whole number
  \item Secondary KC: FRACTIONS | Subtraction | subtracting fractions
  \item Topic: Services such as installation, maintenance, repairing, cleaning, laundry, e-commerce
  \item Grade: Primary 5
\end{itemize}

\textbf{Question}

A cleaning company offers cleaning services for offices. They charge $\textdollar 120$ for cleaning one office. Last month, they cleaned $\frac{5}{4}$ times as many offices as they did in January. If they cleaned $32$ offices in January:

(a) How many offices did they clean last month?  
(b) If $\frac{3}{8}$ of the offices cleaned last month needed extra maintenance, how many offices did NOT need extra maintenance last month?

\textbf{Solution}

Let the number of offices cleaned last month be $x$.

(a) Number of offices cleaned last month:
$$
x = 32 \times \frac{5}{4} = 32 \times 1.25 = 40
$$
So, they cleaned $40$ offices last month.

(b) Number of offices that needed extra maintenance:
$$
\text{Number needing extra maintenance} = 40 \times \frac{3}{8} = 15
$$
Number of offices that did NOT need extra maintenance:
$$
40 - 15 = 25
$$
Therefore, $25$ offices did not need extra maintenance last month.

\section*{Question 75}
\textbf{Metadata}

\begin{itemize}
  \item Question ID: P5-FrMulImN\_P3-FrSmp\_GPT4.1\_Services\_02
  \item Primary KC: FRACTIONS | Multiplication | multiplying a proper/improper fraction and a whole number
  \item Secondary KC: FRACTIONS | Simplifying | expressing a fraction in its simplest form
  \item Topic: Services such as installation, maintenance, repairing, cleaning, laundry, e-commerce
  \item Grade: Primary 5
\end{itemize}

\textbf{Question}

A cleaning company charges $\frac{3}{4}$ hour for each room it cleans. If the company is hired to clean 6 rooms in an office building, 
\begin{enumerate}
  \item How many hours in total will the company spend cleaning all 6 rooms? 
  \item Express your answer in its simplest form.
\end{enumerate}


\textbf{Solution}

\begin{enumerate}
  \item To find the total time taken, multiply the time taken for one room by the number of rooms:

  $\frac{3}{4} \times 6 = \frac{18}{4} = \frac{18}{4}$ hours.

  \item To express $\frac{18}{4}$ in its simplest form, divide both numerator and denominator by 2:

    $\frac{18}{4} = \frac{9}{2}$

    $\frac{9}{2}$ hours can also be written as $4\frac{1}{2}$ hours.
\end{enumerate}

So, the company will spend a total of $\frac{9}{2}$ hours, or $4\frac{1}{2}$ hours, cleaning all 6 rooms.

\section*{Question 76}
\textbf{Metadata}

\begin{itemize}
  \item Question ID: P5-FrMulImN\_P5-FrCnv2Dc\_GPT4.1\_Services\_02
  \item Primary KC: FRACTIONS | Multiplication | multiplying a proper/improper fraction and a whole number
  \item Secondary KC: FRACTIONS | Conversion to decimals | expressing fractions as decimals
  \item Topic: Services such as installation, maintenance, repairing, cleaning, laundry, e-commerce
  \item Grade: Primary 5
\end{itemize}

\textbf{Question}

A window cleaning company charges \$6 for every $\frac{3}{5}$ window cleaned. They have a contract to clean 18 windows in an office building. 

(a) How much will the company earn for cleaning $\frac{3}{5}$ of the 18 windows?

(b) Express $\frac{3}{5}$ of 18 as a decimal.

\textbf{Solution}

(a) To find how many windows are cleaned in $\frac{3}{5}$ of 18 windows:

$\frac{3}{5} \times 18 = \frac{3 \times 18}{5} = \frac{54}{5} = 10.8$

So, $\frac{3}{5}$ of the windows is 10.8 windows.

The company earns \$6 for every $\frac{3}{5}$ window cleaned. For 10.8 windows:

Total earnings $= 10.8 \div \frac{3}{5} \times 6$ (Since 6 dollars is for every $\frac{3}{5}$ window)

But since \$6 is for every $\frac{3}{5}$ window, for $10.8$ windows:

Number of $\frac{3}{5}$ window groups in $10.8$ windows: $10.8 \div \frac{3}{5} = 10.8 \times \frac{5}{3} = 18$

So, the company earns $18 \times 6 = \textdollar 108$

(b) $\frac{3}{5}$ as a decimal is $0.6$. 

So, $\frac{3}{5} \times 18 = 0.6 \times 18 = 10.8$.

$\boxed{\text{(a) \textdollar 108,  (b) 10.8}}$

\section*{Question 77}
\textbf{Metadata}

\begin{itemize}
  \item Question ID: P5-FrMulImN\_P5-FrMul2nd\_GPT4.1\_Services\_02
  \item Primary KC: FRACTIONS | Multiplication | multiplying a proper/improper fraction and a whole number
  \item Secondary KC: FRACTIONS | Multiplication | fraction multiplication
  \item Topic: Services such as installation, maintenance, repairing, cleaning, laundry, e-commerce
  \item Grade: Primary 5
\end{itemize}

\textbf{Question}

Ben runs a cleaning service. For each house he cleans, he uses $\frac{2}{3}$ litres of floor cleaner. If he cleans 5 houses in a day, how many litres of cleaner does he use in total?

The next day, Ben needs to clean 3 offices, and each office is twice as large as a house. He uses $\frac{2}{3}$ litres of cleaner per house-sized area for each office. How many litres of cleaner does he use to clean all 3 offices?

\textbf{Solution}

For the 5 houses:
He uses $\frac{2}{3}$ litres per house.
Total cleaner used $= 5 \times \frac{2}{3} = \frac{10}{3} = 3\frac{1}{3}$ litres.

For the offices:
Each office is twice as large as a house, so each office requires $2 \times \frac{2}{3} = \frac{4}{3}$ litres.
For 3 offices: $3 \times \frac{4}{3} = \frac{12}{3} = 4$ litres.

Therefore, 
- Total cleaner used for 5 houses $= 3\frac{1}{3}$ litres,
- Total cleaner used for 3 offices $= 4$ litres.

\section*{Question 78}
\textbf{Metadata}

\begin{itemize}
  \item Question ID: P5-FrMulPIm\_P2-FrCmp\_GPT4.1\_Services\_02
  \item Primary KC: FRACTIONS | Multiplication | multiplying a proper fraction and a proper/improper fractions
  \item Secondary KC: FRACTIONS | Comparison and ordering | comparing and ordering fractions
  \item Topic: Services such as installation, maintenance, repairing, cleaning, laundry, e-commerce
  \item Grade: Primary 5
\end{itemize}

\textbf{Question}

A laundry shop is cleaning curtains for two customers. For the first customer, they have to clean $\frac{2}{3}$ of all the curtains in a large hall. Each curtain takes $\frac{3}{4}$ of an hour to clean. For the second customer, they have to clean $\frac{7}{8}$ of the curtains in a meeting room, and each curtain takes $\frac{1}{2}$ of an hour to clean.

Which customer’s cleaning will take more total time if both halls have the same number of curtains, and how do the total times compare?

\textbf{Solution}

Let’s assume there are $12$ curtains in each hall (any common multiple will do; the exact number is not critical for comparison as the same number is in both).

First customer:
- Number of curtains cleaned: $12 \times \frac{2}{3} = 8$
- Total time: $8$ curtains $\times \frac{3}{4}$ hour per curtain $= 8 \times \frac{3}{4} = 6$ hours

Second customer:
- Number of curtains cleaned: $12 \times \frac{7}{8} = 10.5$ curtains
- Total time: $10.5$ curtains $\times \frac{1}{2}$ hour per curtain $ = 10.5 \times \frac{1}{2} = 5.25$ hours

Now, compare $6$ hours and $5.25$ hours.

$6 > 5.25$, so the first customer’s cleaning takes more total time.

**Conclusion:** The first customer’s cleaning will take more total time than the second customer’s cleaning.

\section*{Question 79}
\textbf{Metadata}

\begin{itemize}
  \item Question ID: P5-FrMulPIm\_P2-FrAdd2nd\_GPT4.1\_Services\_02
  \item Primary KC: FRACTIONS | Multiplication | multiplying a proper fraction and a proper/improper fractions
  \item Secondary KC: FRACTIONS | Addition | adding fractions
  \item Topic: Services such as installation, maintenance, repairing, cleaning, laundry, e-commerce
  \item Grade: Primary 5
\end{itemize}

\textbf{Question}

A washing service has 3 types of laundry machines. On a weekend, $\frac{3}{5}$ of the machines are used for washing bed sheets, and each of these machines spends $\frac{2}{3}$ of the day in use. The rest of the machines are used for washing clothes. After their use, an extra $\frac{1}{4}$ of the machines used for washing bed sheets are added to wash clothes. 

What fraction of all the laundry machines are used for washing clothes after accounting for the extra machines from the bed sheet washers?

\textbf{Solution}

Let the total number of machines be 1 (as a whole).

First, $\frac{3}{5}$ of the machines are used for washing bed sheets.

Each of these machines spends $\frac{2}{3}$ of the day in use:

$\frac{3}{5} \times \frac{2}{3} = \frac{2}{5}$

So, $\frac{2}{5}$ of all machines are used for washing bed sheets during the day.

The rest of the machines are used for washing clothes.

Fraction of machines not used for bed sheets:$
1 - \frac{3}{5} = \frac{2}{5}$

After their use, $\frac{1}{4}$ of the machines **used for washing bed sheets** are added to wash clothes:

$\frac{1}{4} \times \frac{3}{5} = \frac{3}{20}$

Total fraction of machines used for clothes:
$\frac{2}{5} + \frac{3}{20} = \frac{8}{20} + \frac{3}{20} = \frac{11}{20}$

**Final Answer:** $\boxed{\frac{11}{20}}$ of all the laundry machines are used for washing clothes after accounting for the extra machines from bed sheet washing.

\section*{Question 80}
\textbf{Metadata}

\begin{itemize}
  \item Question ID: P5-FrMulPIm\_P2-FrSub2nd\_GPT4.1\_Services\_02
  \item Primary KC: FRACTIONS | Multiplication | multiplying a proper fraction and a proper/improper fractions
  \item Secondary KC: FRACTIONS | Subtraction | subtracting fractions
  \item Topic: Services such as installation, maintenance, repairing, cleaning, laundry, e-commerce
  \item Grade: Primary 5
\end{itemize}

\textbf{Question}

A laundry service cleans $\frac{5}{8}$ of all the clothes they receive in the morning and plans to clean the rest in the afternoon. Out of what they clean in the morning, $\frac{3}{5}$ are uniforms, and the remaining are casual clothes. If the laundry receives $120$ pieces of clothing in total, 

(a) How many uniforms are cleaned in the morning?

(b) What fraction of all the clothes received are casual clothes cleaned in the morning?

(c) What fraction of all the clothes are left to be cleaned in the afternoon?

\textbf{Solution}

Let the total number of clothes be $120$.

(a) The number of clothes cleaned in the morning is $\frac{5}{8} \times 120 = 75$.

The number of uniforms among these is $\frac{3}{5} \times 75 = 45$.

**Answer to (a):** $45$ uniforms are cleaned in the morning.

(b) The number of casual clothes cleaned in the morning is $75 - 45 = 30$.

The fraction of all the clothes received that are casual clothes cleaned in the morning:

$\frac{30}{120} = \frac{1}{4}$.

**Answer to (b):** $\frac{1}{4}$ of all the clothes are casual clothes cleaned in the morning.

(c) Fraction of clothes left to be cleaned in the afternoon:
The fraction cleaned in the morning is $\frac{5}{8}$, so the fraction left is $1 - \frac{5}{8} = \frac{3}{8}$.

**Answer to (c):** $\frac{3}{8}$ of all the clothes are left to be cleaned in the afternoon.

\section*{Question 81}
\textbf{Metadata}

\begin{itemize}
  \item Question ID: P5-FrMulPIm\_P3-FrSmp\_GPT4.1\_Services\_02
  \item Primary KC: FRACTIONS | Multiplication | multiplying a proper fraction and a proper/improper fractions
  \item Secondary KC: FRACTIONS | Simplifying | expressing a fraction in its simplest form
  \item Topic: Services such as installation, maintenance, repairing, cleaning, laundry, e-commerce
  \item Grade: Primary 5
\end{itemize}

\textbf{Question}

A laundry shop uses $\dfrac{3}{5}$ of a bottle of detergent to wash one batch of clothes. For large orders, they wash $\dfrac{7}{4}$ batches at a time using one big washing machine. 

How many bottles of detergent does the shop use in total for one large wash? Give your answer as a fraction in its simplest form.

\textbf{Solution}

To find out how many bottles of detergent are used, multiply the amount of detergent for one batch by the number of batches:

$\dfrac{3}{5} \times \dfrac{7}{4} = \dfrac{3 \times 7}{5 \times 4} = \dfrac{21}{20}$

$\dfrac{21}{20}$ is already in its simplest form. So the laundry shop uses $\boxed{\dfrac{21}{20}}$ bottles of detergent for one large wash.

\section*{Question 82}
\textbf{Metadata}

\begin{itemize}
  \item Question ID: P5-FrMulPIm\_P5-FrCnv2Dc\_GPT4.1\_Services\_02
  \item Primary KC: FRACTIONS | Multiplication | multiplying a proper fraction and a proper/improper fractions
  \item Secondary KC: FRACTIONS | Conversion to decimals | expressing fractions as decimals
  \item Topic: Services such as installation, maintenance, repairing, cleaning, laundry, e-commerce
  \item Grade: Primary 5
\end{itemize}

\textbf{Question}

A cleaning company uses $\frac{3}{5}$ of a bottle of detergent to clean one office. If they cleaned $\frac{7}{2}$ offices in a day, how many bottles of detergent did they use that day? Express your answer as a decimal.

\textbf{Solution}

First, find the total amount of detergent bottles used:

$\frac{3}{5} \times \frac{7}{2} = \frac{3 \times 7}{5 \times 2} = \frac{21}{10}$

Now, express $\frac{21}{10}$ as a decimal:

$\frac{21}{10} = 2.1$

So, the company used $2.1$ bottles of detergent that day.

\section*{Question 83}
\textbf{Metadata}

\begin{itemize}
  \item Question ID: P5-FrMulPIm\_P5-FrMul2nd\_GPT4.1\_Services\_02
  \item Primary KC: FRACTIONS | Multiplication | multiplying a proper fraction and a proper/improper fractions
  \item Secondary KC: FRACTIONS | Multiplication | fraction multiplication
  \item Topic: Services such as installation, maintenance, repairing, cleaning, laundry, e-commerce
  \item Grade: Primary 5
\end{itemize}

\textbf{Question}

A cleaning company uses $\frac{3}{4}$ of a bottle of detergent to clean one office room. Last week, they cleaned $\frac{5}{2}$ times as many rooms as they usually do in a week. How many bottles of detergent did they use last week in total?

\textbf{Solution}

The number of bottles of detergent used can be found by multiplying the amount used for one room by the number of rooms cleaned: 

Amount used per room $= \frac{3}{4}$ bottle

Number of times rooms cleaned $= \frac{5}{2}$

Total bottles of detergent used $= \frac{3}{4} \times \frac{5}{2}$

$= \frac{3 \times 5}{4 \times 2}$

$= \frac{15}{8}$

So, the company used $\frac{15}{8}$ bottles of detergent last week, 
or $1\frac{7}{8}$ bottles.

\section*{Question 84}
\textbf{Metadata}

\begin{itemize}
  \item Question ID: P5-FrMulImIm\_P2-FrCmp\_GPT4.1\_Services\_02
  \item Primary KC: FRACTIONS | Multiplication | multiplying two improper fractions
  \item Secondary KC: FRACTIONS | Comparison and ordering | comparing and ordering fractions
  \item Topic: Services such as installation, maintenance, repairing, cleaning, laundry, e-commerce
  \item Grade: Primary 5
\end{itemize}

\textbf{Question}

At a laundry shop, Sarah uses $\frac{9}{4}$ kg of washing powder for each load of bed sheets and $\frac{7}{3}$ kg for each load of curtains. 

(a) How much washing powder does she use in total if she washes one load of bed sheets and one load of curtains together? Express your answer as an improper fraction.

(b) Sarah wants to know which type of load uses more washing powder per load. Compare $\frac{9}{4}$ and $\frac{7}{3}$, and state which one is greater.

\textbf{Solution}

(a) Total washing powder used: 

$\frac{9}{4} + \frac{7}{3} = \frac{9 \times 3}{4 \times 3} + \frac{7 \times 4}{3 \times 4} = \frac{27}{12} + \frac{28}{12} = \frac{55}{12}$ kg

Sarah uses $\frac{55}{12}$ kg of washing powder in total.

(b) To compare $\frac{9}{4}$ and $\frac{7}{3}$, find a common denominator:

$\frac{9}{4} = \frac{9 \times 3}{4 \times 3} = \frac{27}{12}$
$\frac{7}{3} = \frac{7 \times 4}{3 \times 4} = \frac{28}{12}$

Since $28 > 27$, $\frac{7}{3}$ is greater than $\frac{9}{4}$.

So, washing curtains uses more washing powder per load than washing bed sheets.

\section*{Question 85}
\textbf{Metadata}

\begin{itemize}
  \item Question ID: P5-FrMulImIm\_P2-FrAdd2nd\_GPT4.1\_Services\_02
  \item Primary KC: FRACTIONS | Multiplication | multiplying two improper fractions
  \item Secondary KC: FRACTIONS | Addition | adding fractions
  \item Topic: Services such as installation, maintenance, repairing, cleaning, laundry, e-commerce
  \item Grade: Primary 5
\end{itemize}

\textbf{Question}

A cleaning company has a contract to clean two offices each day. In the first office, the workers use $\frac{7}{4}$ litres of cleaning solution for every $\frac{5}{3}$ hours of work. In the second office, they use $2\frac{1}{4}$ litres of cleaning solution every day. 

(a) How many litres of cleaning solution do the workers use for $2\frac{1}{2}$ hours in the first office? (Give your answer as an improper fraction.) 

(b) What is the total amount of cleaning solution used for both offices on a day when they work $2\frac{1}{2}$ hours in the first office and clean the second office as usual? (Give your answer as an improper fraction.)

\textbf{Solution}

(a) Find the amount of cleaning solution used for $2\frac{1}{2}$ hours in the first office:

First, convert $2\frac{1}{2}$ to an improper fraction: $2\frac{1}{2} = \frac{5}{2}$.

Let $x$ be the amount of solution used:

$\frac{7}{4}$ litres is used for $\frac{5}{3}$ hours. For $\frac{5}{2}$ hours:

Set up the proportion:
$
\frac{7}{4} : \frac{5}{3} = x : \frac{5}{2}
$

So, $x = \frac{7}{4} \times \frac{5}{2} \div \frac{5}{3}$

$= \frac{7}{4} \times \frac{5}{2} \times \frac{3}{5}$

$= \frac{7 \times 5 \times 3}{4 \times 2 \times 5}$

$= \frac{105}{40}$

Simplify: $\frac{105}{40} = \frac{21}{8}$

So, $\frac{21}{8}$ litres are used for $2\frac{1}{2}$ hours.

(b) The second office uses $2\frac{1}{4} = \frac{9}{4}$ litres.

Total cleaning solution used:

$= \frac{21}{8} + \frac{9}{4}$

Convert $\frac{9}{4}$ to eighths: $\frac{9}{4} = \frac{18}{8}$

So,
$\frac{21}{8} + \frac{18}{8} = \frac{39}{8}$

Final answer: $\frac{39}{8}$ litres of cleaning solution are used in total.

\section*{Question 86}
\textbf{Metadata}

\begin{itemize}
  \item Question ID: P5-FrMulImIm\_P2-FrSub2nd\_GPT4.1\_Services\_02
  \item Primary KC: FRACTIONS | Multiplication | multiplying two improper fractions
  \item Secondary KC: FRACTIONS | Subtraction | subtracting fractions
  \item Topic: Services such as installation, maintenance, repairing, cleaning, laundry, e-commerce
  \item Grade: Primary 5
\end{itemize}

\textbf{Question}

A washing machine repair service charges a fee based on the time taken for the repair. For each repair, they use \( \frac{9}{4} \) hours per machine. On Monday, the technician repaired \( \frac{7}{3} \) machines before lunch. 

(a) How many hours did the technician spend repairing machines before lunch on Monday? 

After lunch, the technician spent \( \frac{21}{8} \) hours repairing more machines. 

(b) How many more hours did the technician spend repairing machines after lunch than before lunch?

\textbf{Solution}

Part (a): The total hours the technician spent before lunch can be found by multiplying the time per machine by the number of machines:

\[
\frac{9}{4} \times \frac{7}{3} = \frac{9 \times 7}{4 \times 3} = \frac{63}{12}
\]

Simplify \( \frac{63}{12} \) by dividing top and bottom by 3:
\[
\frac{63 \div 3}{12 \div 3} = \frac{21}{4}
\]

Therefore, the technician spent \( \frac{21}{4} \) hours repairing machines before lunch.

Part (b): To find how many more hours were spent after lunch than before lunch, subtract the two times:

\[
\frac{21}{8} - \frac{21}{4}
\]

To subtract, express both fractions with a common denominator (8):
\[
\frac{21}{4} = \frac{21 \times 2}{4 \times 2} = \frac{42}{8}
\]

So,
\[
\frac{21}{8} - \frac{42}{8} = \frac{21 - 42}{8} = \frac{-21}{8}
\]

This result means the technician actually spent \( 2\frac{5}{8} \) hours **less** after lunch than before lunch, since the answer is negative. To answer the original question (how many more hours did the technician spend after lunch), take the absolute value:

\[
|\frac{-21}{8}| = \frac{21}{8} = 2\frac{5}{8}
\]

So, the technician spent \( 2\frac{5}{8} \) hours **less** after lunch than before lunch.

\section*{Question 87}
\textbf{Metadata}

\begin{itemize}
  \item Question ID: P5-FrMulImIm\_P3-FrSmp\_GPT4.1\_Services\_02
  \item Primary KC: FRACTIONS | Multiplication | multiplying two improper fractions
  \item Secondary KC: FRACTIONS | Simplifying | expressing a fraction in its simplest form
  \item Topic: Services such as installation, maintenance, repairing, cleaning, laundry, e-commerce
  \item Grade: Primary 5
\end{itemize}

\textbf{Question}

A wash-and-fold laundry service uses $\frac{7}{4}$ boxes of detergent each day. Over $\frac{5}{3}$ days, how many boxes of detergent will the service use in total? Express your answer in its simplest form.

\textbf{Solution}

To find the total number of boxes used, multiply the amount used per day by the number of days:

$$
\frac{7}{4} \times \frac{5}{3} = \frac{7 \times 5}{4 \times 3} = \frac{35}{12}
$$

$\frac{35}{12}$ cannot be simplified further as there are no common factors between $35$ and $12$ other than $1$. So,

**Answer:** The service will use $\frac{35}{12}$ boxes of detergent in total.

\section*{Question 88}
\textbf{Metadata}

\begin{itemize}
  \item Question ID: P5-FrMulImIm\_P5-FrCnv2Dc\_GPT4.1\_Services\_02
  \item Primary KC: FRACTIONS | Multiplication | multiplying two improper fractions
  \item Secondary KC: FRACTIONS | Conversion to decimals | expressing fractions as decimals
  \item Topic: Services such as installation, maintenance, repairing, cleaning, laundry, e-commerce
  \item Grade: Primary 5
\end{itemize}

\textbf{Question}

A cleaning company uses $\frac{9}{4}$ bottles of detergent for each big cleaning job. Last week, the company cleaned $\frac{7}{3}$ big houses. 

(a) How many bottles of detergent did they use in total? Express your answer as an improper fraction.

(b) Express the answer in part (a) in decimal form, correct to 2 decimal places.

\textbf{Solution}

Part (a):
They used $\frac{9}{4}$ bottles per job and cleaned $\frac{7}{3}$ jobs.

Total bottles used $= \frac{9}{4} \times \frac{7}{3} = \frac{9 \times 7}{4 \times 3} = \frac{63}{12}$

Simplify $\frac{63}{12}$:
$63 \div 3 = 21$, $12 \div 3 = 4$

So, $\frac{63}{12} = \frac{21}{4}$

Answer for (a): $\frac{21}{4}$ bottles

Part (b):
Now, express $\frac{21}{4}$ as a decimal.

$\frac{21}{4} = 5.25$

So, they used 5.25 bottles of detergent in total.

Final Answers:
(a) $\frac{21}{4}$ bottles

(b) $5.25$ bottles

\section*{Question 89}
\textbf{Metadata}

\begin{itemize}
  \item Question ID: P5-FrMulImIm\_P5-FrMul2nd\_GPT4.1\_Services\_02
  \item Primary KC: FRACTIONS | Multiplication | multiplying two improper fractions
  \item Secondary KC: FRACTIONS | Multiplication | fraction multiplication
  \item Topic: Services such as installation, maintenance, repairing, cleaning, laundry, e-commerce
  \item Grade: Primary 5
\end{itemize}

\textbf{Question}

A cleaning company uses $\frac{9}{4}$ litres of cleaning solution for each session to deep clean a classroom. If the company is hired to deep clean $\frac{8}{3}$ classrooms, how many litres of cleaning solution will be needed in total?

\textbf{Solution}

To find the total litres of cleaning solution required, multiply the number of litres used per classroom by the fraction of classrooms to be cleaned:

$\frac{9}{4} \times \frac{8}{3} = \frac{9 \times 8}{4 \times 3} = \frac{72}{12} = 6$

So, the company will need $6$ litres of cleaning solution in total.

\section*{Question 90}
\textbf{Metadata}

\begin{itemize}
  \item Question ID: P5-FrMulMixN\_P2-FrCmp\_GPT4.1\_Services\_02
  \item Primary KC: FRACTIONS | Multiplication | multiplying a mixed number and a whole number
  \item Secondary KC: FRACTIONS | Comparison and ordering | comparing and ordering fractions
  \item Topic: Services such as installation, maintenance, repairing, cleaning, laundry, e-commerce
  \item Grade: Primary 5
\end{itemize}

\textbf{Question}

A cleaning company is hired to clean several apartments. For each apartment, they use $2\frac{1}{4}$ litres of cleaning solution. If the company needs to clean $6$ apartments, how many litres of cleaning solution will they need in total? After cleaning, they compared the total amount used to the amount used by another company, which only used $13\frac{1}{2}$ litres to clean the same number of apartments. Which company used more cleaning solution and by how much?

\textbf{Solution}

First, we calculate the total cleaning solution used by the first company.

They use $2\frac{1}{4}$ litres per apartment, and there are $6$ apartments.

Convert the mixed number to an improper fraction:
$2\frac{1}{4} = \frac{9}{4}$

Multiply by $6$:
$6 \times \frac{9}{4} = \frac{54}{4} = 13\frac{1}{2}$ litres

So, the first company used $13\frac{1}{2}$ litres in total.

The second company used $13\frac{1}{2}$ litres as well.

Comparing $13\frac{1}{2}$ and $13\frac{1}{2}$:

Both companies used the same amount of cleaning solution.

\textbf{Answer:} Both companies used the same amount of cleaning solution, $13\frac{1}{2}$ litres.

\section*{Question 91}
\textbf{Metadata}

\begin{itemize}
  \item Question ID: P5-FrMulMixN\_P2-FrAdd2nd\_GPT4.1\_Services\_02
  \item Primary KC: FRACTIONS | Multiplication | multiplying a mixed number and a whole number
  \item Secondary KC: FRACTIONS | Addition | adding fractions
  \item Topic: Services such as installation, maintenance, repairing, cleaning, laundry, e-commerce
  \item Grade: Primary 5
\end{itemize}

\textbf{Question}

Sarah is helping her mother with laundry services during the school holidays. Each day, Sarah washes $2\dfrac{1}{2}$ baskets of clothes. On Monday, she worked for $4$ days. 

(a) How many baskets of clothes did Sarah wash in total over those $4$ days? 

(b) On the last day, her mother washed an additional $1\dfrac{1}{4}$ baskets. How many baskets did Sarah and her mother wash together on the last day?

\textbf{Solution}

(a) First, we multiply the number of baskets Sarah washed each day by the number of days:

$2\dfrac{1}{2} \times 4 = \dfrac{5}{2} \times 4 = \dfrac{5 \times 4}{2} = \dfrac{20}{2} = 10$

Sarah washed $10$ baskets of clothes in $4$ days.

(b) On the last day, Sarah washed $2\dfrac{1}{2}$ baskets and her mother washed $1\dfrac{1}{4}$ baskets. 
Convert the mixed numbers to improper fractions:

Sarah: $2\dfrac{1}{2} = \dfrac{5}{2}$
Mother: $1\dfrac{1}{4} = \dfrac{5}{4}$

To add these, find a common denominator:

$\dfrac{5}{2} + \dfrac{5}{4} = \dfrac{10}{4} + \dfrac{5}{4} = \dfrac{15}{4}$

$\dfrac{15}{4} = 3\dfrac{3}{4}$

Sarah and her mother washed $3\dfrac{3}{4}$ baskets together on the last day.

\section*{Question 92}
\textbf{Metadata}

\begin{itemize}
  \item Question ID: P5-FrMulMixN\_P2-FrSub2nd\_GPT4.1\_Services\_02
  \item Primary KC: FRACTIONS | Multiplication | multiplying a mixed number and a whole number
  \item Secondary KC: FRACTIONS | Subtraction | subtracting fractions
  \item Topic: Services such as installation, maintenance, repairing, cleaning, laundry, e-commerce
  \item Grade: Primary 5
\end{itemize}

\textbf{Question}

A cleaning company uses $1\frac{3}{4}$ litres of cleaning solution to clean one house. If they cleaned 6 houses in the morning, how many litres of cleaning solution did they use in total?

After finishing all 6 houses, they realised that $2\frac{1}{2}$ litres of solution remained in their tank. How much cleaning solution did they have at the start of the day?

\textbf{Solution}

To find the total cleaning solution used for 6 houses:

$1\frac{3}{4} \times 6 = \left(\frac{7}{4}\right) \times 6 = \frac{42}{4} = 10\frac{2}{4} = 10\frac{1}{2}$ litres used.

Let $x$ be the amount of cleaning solution in the tank at the start of the day. After cleaning 6 houses, $2\frac{1}{2}$ litres is left, so:

$x - 10\frac{1}{2} = 2\frac{1}{2}$

Therefore,

$x = 10\frac{1}{2} + 2\frac{1}{2} = 13$ litres.

So, they had 13 litres of cleaning solution at the start of the day.

\section*{Question 93}
\textbf{Metadata}

\begin{itemize}
  \item Question ID: P5-FrMulMixN\_P3-FrSmp\_GPT4.1\_Services\_02
  \item Primary KC: FRACTIONS | Multiplication | multiplying a mixed number and a whole number
  \item Secondary KC: FRACTIONS | Simplifying | expressing a fraction in its simplest form
  \item Topic: Services such as installation, maintenance, repairing, cleaning, laundry, e-commerce
  \item Grade: Primary 5
\end{itemize}

\textbf{Question}

A laundry shop washes tablecloths for restaurants. Each tablecloth takes $2\dfrac{1}{2}$ litres of water to wash. If a restaurant sends $8$ tablecloths to be cleaned, how many litres of water are needed in total? Express your answer in its simplest form.

\textbf{Solution}

First, convert the mixed number $2\dfrac{1}{2}$ to an improper fraction:  
$2\dfrac{1}{2} = \frac{5}{2}$.  
Now, multiply the amount of water per tablecloth by the number of tablecloths: 
$8 \times \frac{5}{2} = \frac{8 \times 5}{2} = \frac{40}{2} = 20$. 
So, the laundry shop needs $20$ litres of water in total.

\section*{Question 94}
\textbf{Metadata}

\begin{itemize}
  \item Question ID: P5-FrMulMixN\_P5-FrCnv2Dc\_GPT4.1\_Services\_02
  \item Primary KC: FRACTIONS | Multiplication | multiplying a mixed number and a whole number
  \item Secondary KC: FRACTIONS | Conversion to decimals | expressing fractions as decimals
  \item Topic: Services such as installation, maintenance, repairing, cleaning, laundry, e-commerce
  \item Grade: Primary 5
\end{itemize}

\textbf{Question}

A cleaning company charges $2\dfrac{1}{4}$ hours to clean one home. If the company has to clean 6 homes in a day, how many hours did the company spend cleaning in total? Express your answer as a decimal.

\textbf{Solution}

First, convert the mixed number to an improper fraction:

$2\dfrac{1}{4} = \frac{9}{4}$

Multiply the number of hours needed to clean one home by the number of homes:

$\frac{9}{4} \times 6 = \frac{9 \times 6}{4} = \frac{54}{4}$

$\frac{54}{4} = \frac{27}{2}$

Now, convert $\frac{27}{2}$ to decimal:

$\frac{27}{2} = 13.5$

So, the company spent a total of $13.5$ hours cleaning 6 homes in a day.

\section*{Question 95}
\textbf{Metadata}

\begin{itemize}
  \item Question ID: P5-FrMulMixN\_P5-FrMul2nd\_GPT4.1\_Services\_02
  \item Primary KC: FRACTIONS | Multiplication | multiplying a mixed number and a whole number
  \item Secondary KC: FRACTIONS | Multiplication | fraction multiplication
  \item Topic: Services such as installation, maintenance, repairing, cleaning, laundry, e-commerce
  \item Grade: Primary 5
\end{itemize}

\textbf{Question}

A cleaning company uses $2\frac{1}{2}$ litres of cleaning solution to clean one large house. During a promotion, each cleaner is assigned to clean $4$ large houses per day. 

(a) How many litres of cleaning solution does one cleaner use in a day? 

(b) The cleaning solution comes in bottles that hold $\frac{3}{4}$ litres each. How many bottles of cleaning solution does one cleaner need in a day?

\textbf{Solution}

(a) Amount of cleaning solution used for $1$ large house $= 2\frac{1}{2} = \frac{5}{2}$ litres. 

Number of houses cleaned in a day $= 4$

Total cleaning solution used $= \frac{5}{2} \times 4 = \frac{20}{2} = 10$ litres.

(b) Each bottle contains $\frac{3}{4}$ litres.

Number of bottles needed $= \frac{10}{\frac{3}{4}} = 10 \times \frac{4}{3} = \frac{40}{3} \approx 13.33$

So, the cleaner will need $14$ bottles (since you can't buy part of a bottle).

**Final answers:**

(a) $10$ litres

(b) $14$ bottles

\section*{Question 96}
\textbf{Metadata}

\begin{itemize}
  \item Question ID: P5-DcMul3dK\_P4-DcCmp3d\_GPT4.1\_Services\_02
  \item Primary KC: DECIMALS | Multiplication | multiplying decimals (up to 3 decimal places) by 10, 100, 1000 and their multiples
  \item Secondary KC: DECIMALS | Comparison and ordering | comparing and ordering decimals up to 3 decimal places
  \item Topic: Services such as installation, maintenance, repairing, cleaning, laundry, e-commerce
  \item Grade: Primary 5
\end{itemize}

\textbf{Question}

A laundry service charges $3.285$ per kilogram of clothes for washing. Mrs Tan sends three bags of clothes weighing $4.2\ \text{kg}$, $3.45\ \text{kg}$ and $5.12\ \text{kg}$ each. She wants to know which bag will cost her the most, and by how much more than the least expensive bag. Calculate the cost (rounded to the nearest cent) for each bag by multiplying the weight by the price per kilogram. Then, compare the costs and determine the difference between the highest and lowest costs.

\textbf{Solution}

First, calculate the cost for each bag by multiplying the weight (in kg) by the price per kg.

For the first bag:

Cost$_1 = 4.2 \times 3.285 $

$ = 13.797 $

Rounding to the nearest cent: $13.80$

For the second bag:

Cost$_2 = 3.45 \times 3.285 $

$ = 11.34825 $

Rounding to the nearest cent: $11.35$

For the third bag:

Cost$_3 = 5.12 \times 3.285 $

$ = 16.8192 $

Rounding to the nearest cent: $16.82$

Now, compare the costs:

- $13.80$ (first bag)
- $11.35$ (second bag)
- $16.82$ (third bag)

Ordering from least to greatest: $11.35 < 13.80 < 16.82$

The third bag costs the most. The difference between the most and least expensive is:

$16.82 - 11.35 = 5.47$

So, the third bag will cost Mrs Tan the most, and it is $5.47$ more than the least expensive bag.

\section*{Question 97}
\textbf{Metadata}

\begin{itemize}
  \item Question ID: P5-DcMul3dK\_P4-DcCnv2Fr\_GPT4.1\_Services\_02
  \item Primary KC: DECIMALS | Multiplication | multiplying decimals (up to 3 decimal places) by 10, 100, 1000 and their multiples
  \item Secondary KC: DECIMALS | Conversion from decimals to fraction | expressing decimals as fractions
  \item Topic: Services such as installation, maintenance, repairing, cleaning, laundry, e-commerce
  \item Grade: Primary 5
\end{itemize}

\textbf{Question}

A laundry shop charges $\$2.85$ per kilogram to clean clothes. Mrs Tan sends $12$ kg of clothes for cleaning. 

(a) How much does Mrs Tan have to pay in total?

(b) Express the price per kilogram, $2.85$, as a fraction in its simplest form.

\textbf{Solution}

(a) The total amount Mrs Tan has to pay is $2.85 \times 12 = 34.20$. 
So, Mrs Tan has to pay $\$34.20$ in total.

(b) $2.85$ can be written as $\dfrac{285}{100}$.

To simplify $\dfrac{285}{100}$, find the highest common factor:
 $285 = 5 \times 57$, $100 = 5 \times 20$. Divide numerator and denominator by $5$:

$\dfrac{285 \div 5}{100 \div 5} = \dfrac{57}{20}$.

Therefore, $2.85$ as a fraction in its simplest form is $\dfrac{57}{20}$.

\section*{Question 98}
\textbf{Metadata}

\begin{itemize}
  \item Question ID: P5-DcMul3dK\_P4-DcRnd3d\_GPT4.1\_Services\_02
  \item Primary KC: DECIMALS | Multiplication | multiplying decimals (up to 3 decimal places) by 10, 100, 1000 and their multiples
  \item Secondary KC: DECIMALS | Rounding | rounding decimals up to 3 decimal places to the nearest whole number, 1 decimal place and 2 decimal places 
  \item Topic: Services such as installation, maintenance, repairing, cleaning, laundry, e-commerce
  \item Grade: Primary 5
\end{itemize}

\textbf{Question}

A cleaning company charges $5.897$ per square metre to clean a carpet. A customer wants to clean $100$ square metres of carpet in his office. 

(a) Calculate the total cost before rounding by multiplying the rate per square metre by the total area to be cleaned.

(b) Round the total cost to the nearest dollar.

(c) Round the total cost to 2 decimal places, as would be shown on a typical invoice.

\textbf{Solution}

(a) Total cost $= 5.897 \times 100 = 589.7$

(b) Rounding $589.7$ to the nearest dollar: Since $0.7 > 0.5$, we round up.
So, the cost rounded to the nearest dollar is $590\textdollar$.

(c) Rounding $589.7$ to 2 decimal places: Since there are no digits beyond the tenths place, it remains $589.70$.
So, the cost to 2 decimal places is $589.70\textdollar$.

\section*{Question 99}
\textbf{Metadata}

\begin{itemize}
  \item Question ID: P5-DcMul3dK\_P4-DcAdd2nd\_GPT4.1\_Services\_02
  \item Primary KC: DECIMALS | Multiplication | multiplying decimals (up to 3 decimal places) by 10, 100, 1000 and their multiples
  \item Secondary KC: DECIMALS | Addition | adding decimals
  \item Topic: Services such as installation, maintenance, repairing, cleaning, laundry, e-commerce
  \item Grade: Primary 5
\end{itemize}

\textbf{Question}

Mr. Lim runs a cleaning service. He charges $3.75$ per square metre to clean carpets. On Monday, he cleaned an office that was $12.6$ square metres in area. The office manager also paid an extra $7.50$ for a stain removal service. 

What was the total amount in dollars that Mr. Lim charged for cleaning the office carpet and providing the stain removal service?

\textbf{Solution}

First, calculate the cleaning cost: 

Cost of cleaning $= 3.75 \times 12.6 = 47.25$

Add the additional stain removal service fee:

Total cost $= 47.25 + 7.50 = 54.75$

So, Mr. Lim charged $\textdollar54.75$ in total.

\section*{Question 100}
\textbf{Metadata}

\begin{itemize}
  \item Question ID: P5-DcMul3dK\_P4-DcSub2nd\_GPT4.1\_Services\_02
  \item Primary KC: DECIMALS | Multiplication | multiplying decimals (up to 3 decimal places) by 10, 100, 1000 and their multiples
  \item Secondary KC: DECIMALS | Subtraction | subtracting decimals
  \item Topic: Services such as installation, maintenance, repairing, cleaning, laundry, e-commerce
  \item Grade: Primary 5
\end{itemize}

\textbf{Question}

A laundry shop charges $2.80 per kilogram to wash and dry clothes. Mrs Tan brought in a basket of clothes weighing $4.725$ kg. She received a voucher and used it to get a discount of $5.60\textdollar$ off her total bill. 

How much must Mrs Tan pay after using her discount voucher?

\textbf{Solution}

First, calculate the total cost of washing and drying Mrs Tan's clothes without the discount:

\[
\text{Total cost} = 4.725\, \text{kg} \times 2.80\textdollar/\text{kg} = 13.23\textdollar
\]

Next, subtract the value of the discount voucher:

\[
13.23\textdollar - 5.60\textdollar = 7.63\textdollar
\]

\textbf{Final answer:} Mrs Tan must pay $7.63\textdollar$ after using her discount voucher.

\section*{Question 101}
\textbf{Metadata}

\begin{itemize}
  \item Question ID: P5-DcDiv3dK\_P4-DcCmp3d\_GPT4.1\_Services\_02
  \item Primary KC: DECIMALS | Division | dividing decimals (up to 3 decimal places) by 10, 100, 1000 and their multiples
  \item Secondary KC: DECIMALS | Comparison and ordering | comparing and ordering decimals up to 3 decimal places
  \item Topic: Services such as installation, maintenance, repairing, cleaning, laundry, e-commerce
  \item Grade: Primary 5
\end{itemize}

\textbf{Question}

A laundry service charges $\$4.850$ to wash $10$ kilograms of clothes. In a week, three friends each send a different amount of laundry: Ann sends $6$ kilograms, Ben sends $7.5$ kilograms, and Chloe sends $5$ kilograms. 

(a) Calculate the cost for each friend by first finding the price per kilogram, then multiplying accordingly. 

(b) List the amounts the three friends will pay in order from the lowest to the highest.

\textbf{Solution}

(a) The price for $10$ kg is $\$4.850$, so the price per kg is:

\[
\text{Price per kg} = \frac{4.850}{10} = 0.485
\]

\text{Ann's cost:} $\quad 6 \times 0.485 = 2.910$

\text{Ben's cost:} $\quad 7.5 \times 0.485 = 3.6375$

\text{Chloe's cost:} $\quad 5 \times 0.485 = 2.425$


(b) Comparing the amounts paid:
\[
2.425 < 2.910 < 3.6375
\]

So, in order from least to greatest: Chloe ($2.425$), Ann ($2.910$), Ben ($3.6375$).

\section*{Question 102}
\textbf{Metadata}

\begin{itemize}
  \item Question ID: P5-DcDiv3dK\_P4-DcCnv2Fr\_GPT4.1\_Services\_02
  \item Primary KC: DECIMALS | Division | dividing decimals (up to 3 decimal places) by 10, 100, 1000 and their multiples
  \item Secondary KC: DECIMALS | Conversion from decimals to fraction | expressing decimals as fractions
  \item Topic: Services such as installation, maintenance, repairing, cleaning, laundry, e-commerce
  \item Grade: Primary 5
\end{itemize}

\textbf{Question}

A dry cleaning shop is offering to clean and press a shirt for $\textdollar 12.500$. After a customer uses a promotional code, the price is divided by $100$. 

(a) What is the new price after the discount? Give your answer as a decimal.

(b) Express the new price as a fraction in its simplest form.

\textbf{Solution}

(a) To find the new price, divide $12.500$ by $100$:

$$12.500 \div 100 = 0.125$$

So, the new price is $0.125$ (or $\textdollar 0.125$).

(b) Express $0.125$ as a fraction:

$0.125 = \frac{125}{1000}$. 

Now, simplify the fraction. Divide both numerator and denominator by $125$:

$$\frac{125 \div 125}{1000 \div 125} = \frac{1}{8}$$

So, the new price as a fraction is $\frac{1}{8}$.

\section*{Question 103}
\textbf{Metadata}

\begin{itemize}
  \item Question ID: P5-DcDiv3dK\_P4-DcRnd3d\_GPT4.1\_Services\_02
  \item Primary KC: DECIMALS | Division | dividing decimals (up to 3 decimal places) by 10, 100, 1000 and their multiples
  \item Secondary KC: DECIMALS | Rounding | rounding decimals up to 3 decimal places to the nearest whole number, 1 decimal place and 2 decimal places 
  \item Topic: Services such as installation, maintenance, repairing, cleaning, laundry, e-commerce
  \item Grade: Primary 5
\end{itemize}

\textbf{Question}

A cleaning company charges \$378.000 to clean 1000 square metres of office space. 

(a) How much does it cost to clean 1 square metre of office space?

(b) Round your answer in part (a) to the nearest cent (2 decimal places).

(c) If a client has 37.6 square metres to clean, estimate the total cleaning cost by multiplying your answer in part (b) by 37.6, and then round your final answer to the nearest dollar.


\textbf{Solution}

(a) To find the cost to clean 1 square metre: 
\[
\frac{378.000}{1000} = 0.378
\]

(b) Rounding $0.378$ to the nearest cent (2 decimal places): 
\[
0.378 \approx 0.38
\]

(c) Estimated cost for 37.6 square metres:
\[
0.38 \times 37.6 = 14.288
\]
Rounding $14.288$ to the nearest dollar:
\[
14.288 \approx 14
\]

Thus, the total cost is \$14.

\section*{Question 104}
\textbf{Metadata}

\begin{itemize}
  \item Question ID: P5-DcDiv3dK\_P4-DcAdd2nd\_GPT4.1\_Services\_02
  \item Primary KC: DECIMALS | Division | dividing decimals (up to 3 decimal places) by 10, 100, 1000 and their multiples
  \item Secondary KC: DECIMALS | Addition | adding decimals
  \item Topic: Services such as installation, maintenance, repairing, cleaning, laundry, e-commerce
  \item Grade: Primary 5
\end{itemize}

\textbf{Question}

A laundry shop charges $\$5.200$ to wash 10 kg of clothes. Wei Ling has two bags of laundry: the first bag weighs $3.5$ kg and the second bag weighs $6.8$ kg. 

(a) What is the total weight of Wei Ling's laundry bags? 

(b) How much does it cost Wei Ling in total to wash both bags? Show your working clearly.

\textbf{Solution}

(a) To find the total weight, add the weights of both bags:

$3.5 + 6.8 = 10.3$ kg

Wei Ling's total laundry weighs $10.3$ kg.

(b) The cost to wash 10 kg is $\$5.200$. To find the cost per kg, divide $5.200$ by $10$:

$5.200 \div 10 = 0.520$

So, it costs $\$0.520$ per kg.

To find the cost of washing $10.3$ kg:

$10.3 \times 0.520 = 5.356$

Therefore, Wei Ling needs to pay $\$5.356$ in total to wash both bags.

\section*{Question 105}
\textbf{Metadata}

\begin{itemize}
  \item Question ID: P5-DcDiv3dK\_P4-DcSub2nd\_GPT4.1\_Services\_02
  \item Primary KC: DECIMALS | Division | dividing decimals (up to 3 decimal places) by 10, 100, 1000 and their multiples
  \item Secondary KC: DECIMALS | Subtraction | subtracting decimals
  \item Topic: Services such as installation, maintenance, repairing, cleaning, laundry, e-commerce
  \item Grade: Primary 5
\end{itemize}

\textbf{Question}

Jia Wei runs a laundry service. He charged a customer $18.750$ for washing $15$ shirts. To find out the charge for each shirt, Jia Wei divided the total cost by $15$. After finishing his work, Jia Wei realised he overcharged the customer by $0.125$ in total. 

What is the correct amount Jia Wei should have charged for each shirt after subtracting the overcharged amount and dividing by $15$ again? Give your answer correct to $3$ decimal places.

\textbf{Solution}

First, calculate the amount overcharged: $18.750 - 0.125 = 18.625$.

Now, divide the correct total amount by the number of shirts to find the correct charge for each shirt: 
\[ \frac{18.625}{15} = 1.241666... \]

Rounding off to three decimal places, the correct charge per shirt is $1.242$.

\section*{Question 106}
\textbf{Metadata}

\begin{itemize}
  \item Question ID: P5-PcRepWh\_P1-WNAdd2nd\_GPT4.1\_Services\_02
  \item Primary KC: PERCENTAGE | Representation and concept | expressing a part of a whole as a percentage
  \item Secondary KC: WHOLE NUMBERS | Addition | adding whole numbers
  \item Topic: Services such as installation, maintenance, repairing, cleaning, laundry, e-commerce
  \item Grade: Primary 5
\end{itemize}

\textbf{Question}

Aisha runs a cleaning company. On Monday, her team cleaned 15 offices, and on Tuesday, they cleaned 25 offices. Express the number of offices cleaned on Monday as a percentage of the total number of offices cleaned on both days.

\textbf{Solution}

First, find the total number of offices cleaned on both days:

$15 + 25 = 40$ offices.

Next, express the number of offices cleaned on Monday as a percentage of the total:

$\frac{15}{40} \times 100\% = 37.5\%$

So, the number of offices cleaned on Monday is $37.5\%$ of the total number of offices cleaned on both days.

\section*{Question 107}
\textbf{Metadata}

\begin{itemize}
  \item Question ID: P5-PcRepWh\_P1-WNSub2nd\_GPT4.1\_Services\_02
  \item Primary KC: PERCENTAGE | Representation and concept | expressing a part of a whole as a percentage
  \item Secondary KC: WHOLE NUMBERS | Subtraction | subtracting whole numbers
  \item Topic: Services such as installation, maintenance, repairing, cleaning, laundry, e-commerce
  \item Grade: Primary 5
\end{itemize}

\textbf{Question}

Muthu sent $30$ of his school uniforms to a laundry shop for cleaning. After washing, he found that $20\%$ of his uniforms still had stains and needed to be washed again. How many of his uniforms did not need to be washed again after the first wash?

\textbf{Solution}

First, calculate how many uniforms had stains after the wash:

$20\%$ of $30$ uniforms $= \frac{20}{100} \times 30 = 6$ uniforms.

This means $6$ uniforms had stains and needed to be washed again.

To find out how many uniforms did not need to be washed again, subtract the number of stained uniforms from the total:

$30 - 6 = 24$ uniforms.

\textbf{Answer:} $24$ uniforms did not need to be washed again after the first wash.

\section*{Question 108}
\textbf{Metadata}

\begin{itemize}
  \item Question ID: P5-PcRepWh\_P1-WNMul2nd\_GPT4.1\_Services\_02
  \item Primary KC: PERCENTAGE | Representation and concept | expressing a part of a whole as a percentage
  \item Secondary KC: WHOLE NUMBERS | Multiplication | multiplying whole numbers
  \item Topic: Services such as installation, maintenance, repairing, cleaning, laundry, e-commerce
  \item Grade: Primary 5
\end{itemize}

\textbf{Question}

A cleaning company cleaned $8$ offices in one day. At the end of the day, the supervisor checked and found that $6$ of the offices were cleaned satisfactorily. 

What percentage of the offices were cleaned satisfactorily?

\textbf{Solution}

Number of offices cleaned satisfactorily $= 6$

Total number of offices cleaned $= 8$

Fraction of offices cleaned satisfactorily $= \frac{6}{8}$

To express this as a percentage:

$= \frac{6}{8} \times 100\% = 0.75 \times 100\% = 75\%$

So, $75\%$ of the offices were cleaned satisfactorily.

\section*{Question 109}
\textbf{Metadata}

\begin{itemize}
  \item Question ID: P5-PcRepWh\_P1-WNDiv2nd\_GPT4.1\_Services\_02
  \item Primary KC: PERCENTAGE | Representation and concept | expressing a part of a whole as a percentage
  \item Secondary KC: WHOLE NUMBERS | Division | dividing whole numbers
  \item Topic: Services such as installation, maintenance, repairing, cleaning, laundry, e-commerce
  \item Grade: Primary 5
\end{itemize}

\textbf{Question}

A cleaning company was hired to clean $40$ offices in a building over a week. By Wednesday, the team had already cleaned $18$ offices. What percentage of all the offices had been cleaned by Wednesday? Express your answer as a percentage, showing all working clearly.

\textbf{Solution}

To find the percentage of offices cleaned by Wednesday, first divide the number of offices cleaned by Wednesday by the total number of offices: 

$\displaystyle\frac{18}{40} = 0.45$

Next, express $0.45$ as a percentage by multiplying by $100$:

$0.45 \times 100 = 45$%

Therefore, $45\%$ of all the offices had been cleaned by Wednesday.

\section*{Question 110}
\textbf{Metadata}

\begin{itemize}
  \item Question ID: P5-RtFndR\_P2-DcCnvN2D\_GPT4.1\_Services\_02
  \item Primary KC: RATE | Finding rate | finding rate given total amount and number of units
  \item Secondary KC: DECIMALS | Conversion to larger units | converting an amount or measurement from a smaller unit to a larger unit in decimal form
  \item Topic: Services such as installation, maintenance, repairing, cleaning, laundry, e-commerce
  \item Grade: Primary 5
\end{itemize}

\textbf{Question}

A cleaning company charges a total of $\textdollar234$ to clean 18 small apartments. Each apartment has a floor area of $36,000~\text{cm}^2$.\
\
(a) What is the cleaning rate per apartment?\
\
(b) Convert the floor area of each apartment from square centimetres to square metres in decimal form. (Note: $1~\text{m}^2 = 10,000~\text{cm}^2$)

\textbf{Solution}

(a) The cleaning rate per apartment: \
\[
\frac{\textdollar234}{18} = \textdollar13
\]
So, the cleaning rate is \textdollar13 per apartment.\
\
(b) Each apartment's area in square metres: \
\[
\frac{36,000~\text{cm}^2}{10,000} = 3.6~\text{m}^2
\]
Therefore, each apartment is $3.6~\text{m}^2$ in area.

\section*{Question 111}
\textbf{Metadata}

\begin{itemize}
  \item Question ID: P5-RtFndR\_P2-DcCnvD2N\_GPT4.1\_Services\_02
  \item Primary KC: RATE | Finding rate | finding rate given total amount and number of units
  \item Secondary KC: DECIMALS | Conversion to smaller units | converting an amount or measurement from a larger unit in decimal form to a smaller unit
  \item Topic: Services such as installation, maintenance, repairing, cleaning, laundry, e-commerce
  \item Grade: Primary 5
\end{itemize}

\textbf{Question}

A laundry shop charges a total of $18$ for cleaning $4.5$ kg of clothes. The shop records its cleaning rate in grams per dollar. What is the rate, in grams per dollar, that the shop charges? (Note: $1$ kg $= 1,000$ g)

\textbf{Solution}

First, convert $4.5$ kg of clothes to grams:

$4.5 \times 1,000 = 4,500$ g

Next, find the rate in grams per dollar:

\[
\text{Rate} = \frac{4,500 \text{ g}}{18\textdollar} = 250 \text{ g per } \textdollar
\]

So, the laundry shop charges at a rate of $250$ grams per dollar.

\section*{Question 112}
\textbf{Metadata}

\begin{itemize}
  \item Question ID: P5-RtFndT\_P2-DcCnvN2D\_GPT4.1\_Services\_02
  \item Primary KC: RATE | Finding total amount | finding total amount, given rate and number of units
  \item Secondary KC: DECIMALS | Conversion to larger units | converting an amount or measurement from a smaller unit to a larger unit in decimal form
  \item Topic: Services such as installation, maintenance, repairing, cleaning, laundry, e-commerce
  \item Grade: Primary 5
\end{itemize}

\textbf{Question}

A laundry shop charges $2.50$ per kilogram to wash clothes. Sarah brought her laundry and the total weight was $4500$ grams. 

How much does Sarah need to pay for her laundry? (Give your answer in dollars and cents.)

\textbf{Solution}

First, convert $4500$ grams to kilograms.

$4500$ grams $= 4500 \div 1000 = 4.5$ kilograms.

The rate is $\textdollar2.50$ per kilogram.

Total amount to pay $= 4.5 \times 2.50 = 11.25$

Sarah needs to pay $\textdollar11.25$ for her laundry.

\section*{Question 113}
\textbf{Metadata}

\begin{itemize}
  \item Question ID: P5-RtFndT\_P2-DcCnvD2N\_GPT4.1\_Services\_02
  \item Primary KC: RATE | Finding total amount | finding total amount, given rate and number of units
  \item Secondary KC: DECIMALS | Conversion to smaller units | converting an amount or measurement from a larger unit in decimal form to a smaller unit
  \item Topic: Services such as installation, maintenance, repairing, cleaning, laundry, e-commerce
  \item Grade: Primary 5
\end{itemize}

\textbf{Question}

Mrs Lee hired a cleaning company to clean her apartment. The company charges $3.75$ per square metre. If Mrs Lee’s apartment has a total area of $46.5$ square metres, what is the total cleaning cost in dollars? Express your answer to the nearest cent.

How many square centimetres is Mrs Lee’s apartment? (Note: $1$ square metre $= 10,000$ square centimetres.)

\textbf{Solution}

First, find the total cleaning cost:

Total cleaning cost $= 46.5 \times \textdollar3.75 = \textdollar174.375$

To the nearest cent, the total cost is $\textdollar174.38$.

Next, convert the area to square centimetres.

$46.5$ square metres $= 46.5 \times 10,000 = 465,000$ square centimetres.

Final answers:

- Total cleaning cost: $\textdollar174.38$
- Area in square centimetres: $465,000$ square centimetres.

\section*{Question 114}
\textbf{Metadata}

\begin{itemize}
  \item Question ID: P5-RtFndU\_P2-DcCnvN2D\_GPT4.1\_Services\_02
  \item Primary KC: RATE | Finding number of unit | finding number of units given rate and total amount
  \item Secondary KC: DECIMALS | Conversion to larger units | converting an amount or measurement from a smaller unit to a larger unit in decimal form
  \item Topic: Services such as installation, maintenance, repairing, cleaning, laundry, e-commerce
  \item Grade: Primary 5
\end{itemize}

\textbf{Question}

A laundry service charges $2.50$ for every $0.5$ kilograms of laundry washed. Mei Ling sent $3,200$ grams of clothes to be washed. 

(a) How many $0.5$ kilograms of laundry did she send?

(b) How much did Mei Ling have to pay for the laundry service?

\textbf{Solution}

(a) First, convert $3,200$ grams to kilograms:

$3,200 \text{ g} = 3,200 \div 1,000 = 3.2 \text{ kg}$

Next, find how many $0.5$ kilograms are there in $3.2$ kilograms:

$3.2 \div 0.5 = 6.4$

Mei Ling sent $6.4$ portions of $0.5$ kilograms.

(b) The laundry service charges $\textdollar2.50$ for each $0.5$ kg.

Total cost: $6.4 \times \textdollar2.50 = \textdollar16.00$

Thus, Mei Ling has to pay $\textdollar16.00$ for the laundry service.

\section*{Question 115}
\textbf{Metadata}

\begin{itemize}
  \item Question ID: P5-RtFndU\_P2-DcCnvD2N\_GPT4.1\_Services\_02
  \item Primary KC: RATE | Finding number of unit | finding number of units given rate and total amount
  \item Secondary KC: DECIMALS | Conversion to smaller units | converting an amount or measurement from a larger unit in decimal form to a smaller unit
  \item Topic: Services such as installation, maintenance, repairing, cleaning, laundry, e-commerce
  \item Grade: Primary 5
\end{itemize}

\textbf{Question}

A company charges \( \textdollar2.40 \) for every kilogram of laundry they wash. Mary sent \( 7.2 \) kg of clothes to the company. How many grams of clothes did Mary send for laundry, and how much must she pay for the service?

\textbf{Solution}

First, convert the amount of clothes from kilograms to grams: \( 1 \) kg = \( 1000 \) g, so \( 7.2 \) kg = \( 7.2 \times 1000 = 7200 \) g.

Next, find the total amount Mary must pay: Each kilogram costs \( \textdollar2.40 \). Thus, she must pay:
\[
7.2 \times 2.4 = 17.28
\]
So, Mary sent \( 7200 \) g of clothes and must pay \( \textdollar17.28 \) for the service.

\section*{Question 116}
\textbf{Metadata}

\begin{itemize}
  \item Question ID: P6-FrDivPN\_P2-FrCmp\_GPT4.1\_Services\_07
  \item Primary KC: FRACTIONS | Division | dividing a proper fraction by a whole number
  \item Secondary KC: FRACTIONS | Comparison and ordering | comparing and ordering fractions
  \item Topic: Services such as installation, maintenance, repairing, cleaning, laundry, e-commerce
  \item Grade: Primary 6
\end{itemize}

\textbf{Question}

A cleaning company was hired to clean 6 houses. The total amount of cleaning solution provided was $2\frac{1}{2}$ litres. The company decided to use the same amount of cleaning solution for each house. 

(a) How many litres of cleaning solution was used for each house?  
(b) After the cleaning jobs, another company cleaned 6 houses with $1\frac{2}{3}$ litres of solution, using the same amount per house. Which company used more cleaning solution per house? Arrange the two amounts in order from least to greatest.

\textbf{Solution}

(a) The amount of cleaning solution used per house is:

\[ 2\frac{1}{2} \div 6 = \frac{5}{2} \div 6 = \frac{5}{2} \times \frac{1}{6} = \frac{5}{12} \text{ litre per house} \]

(b) The second company: 

\[ 1\frac{2}{3} \div 6 = \frac{5}{3} \div 6 = \frac{5}{3} \times \frac{1}{6} = \frac{5}{18} \text{ litre per house} \]

To compare \( \frac{5}{12} \) and \( \frac{5}{18} \):

Find common denominators:

\[ \frac{5}{12} = \frac{15}{36} \text{; } \frac{5}{18} = \frac{10}{36} \]

Since \( 15 > 10 \), \( \frac{5}{12} > \frac{5}{18} \).

So, the first company used more cleaning solution per house.

Arranged from least to greatest: \( \frac{5}{18} \), \( \frac{5}{12} \).

\section*{Question 117}
\textbf{Metadata}

\begin{itemize}
  \item Question ID: P6-FrDivPN\_P2-FrAdd2nd\_GPT4.1\_Services\_02
  \item Primary KC: FRACTIONS | Division | dividing a proper fraction by a whole number
  \item Secondary KC: FRACTIONS | Addition | adding fractions
  \item Topic: Services such as installation, maintenance, repairing, cleaning, laundry, e-commerce
  \item Grade: Primary 6
\end{itemize}

\textbf{Question}

Mrs Lim is washing curtains for her clients. She uses $\frac{3}{4}$ of a litre of detergent to wash all the curtains in one client's house. This amount is divided equally to wash 2 different rooms in that house. After finishing with this client, she washes curtains for another house and uses $\frac{1}{2}$ of a litre of detergent for that house. How much detergent did Mrs Lim use in total for the two houses?

\textbf{Solution}

First, find the amount of detergent used for each room in the first house: 

\[
\frac{3}{4} \div 2 = \frac{3}{4} \times \frac{1}{2} = \frac{3}{8}
\]

So, each room in the first house uses $\frac{3}{8}$ of a litre of detergent. For the two rooms together, the total is still $\frac{3}{4}$ litre (as given).

For the second house, Mrs Lim uses $\frac{1}{2}$ litre of detergent.

Add the amount used for both houses:

\[
\frac{3}{4} + \frac{1}{2} = \frac{3}{4} + \frac{2}{4} = \frac{5}{4}
\]

Thus, Mrs Lim used a total of $\frac{5}{4}$ litres, or $1\frac{1}{4}$ litres of detergent for the two houses.

\section*{Question 118}
\textbf{Metadata}

\begin{itemize}
  \item Question ID: P6-FrDivPN\_P2-FrSub2nd\_GPT4.1\_Services\_02
  \item Primary KC: FRACTIONS | Division | dividing a proper fraction by a whole number
  \item Secondary KC: FRACTIONS | Subtraction | subtracting fractions
  \item Topic: Services such as installation, maintenance, repairing, cleaning, laundry, e-commerce
  \item Grade: Primary 6
\end{itemize}

\textbf{Question}

A washing service company handled $\frac{7}{8}$ of a load of laundry for a hotel and decided to distribute this evenly among $4$ industrial washing machines. 

(a) What fraction of the original load did each machine wash?

After the first round, one of the machines could not finish its part and was only able to wash $\frac{1}{16}$ load less than its share. 

(b) How much of the laundry did this machine actually wash?

\textbf{Solution}

Part (a):

Each machine washes $\frac{7}{8} \div 4 = \frac{7}{8} \times \frac{1}{4} = \frac{7}{32}$ of a load.

Part (b):

The machine completed $\frac{7}{32} - \frac{1}{16}$.

First, express $\frac{1}{16}$ with denominator 32: $\frac{1}{16} = \frac{2}{32}$.

So, $\frac{7}{32} - \frac{2}{32} = \frac{5}{32}$.

Therefore, the machine actually washed $\frac{5}{32}$ of the original load.

\section*{Question 119}
\textbf{Metadata}

\begin{itemize}
  \item Question ID: P6-FrDivPN\_P3-FrSmp\_GPT4.1\_Services\_07
  \item Primary KC: FRACTIONS | Division | dividing a proper fraction by a whole number
  \item Secondary KC: FRACTIONS | Simplifying | expressing a fraction in its simplest form
  \item Topic: Services such as installation, maintenance, repairing, cleaning, laundry, e-commerce
  \item Grade: Primary 6
\end{itemize}

\textbf{Question}

Mrs Tan is cleaning her house and she decides to divide \(\frac{3}{4}\) of a bottle of floor cleaner equally among 5 cleaning sessions. 

(a) What fraction of a bottle does she use for each cleaning session? 

(b) Express your answer in its simplest form.

\textbf{Solution}

(a) To find the amount of cleaner used per session, divide \(\frac{3}{4}\) by 5:
\[
\frac{3}{4} \div 5 = \frac{3}{4} \times \frac{1}{5} = \frac{3}{20}
\]

(b) \(\frac{3}{20}\) is already in its simplest form. 

So, she uses \(\frac{3}{20}\) of a bottle of floor cleaner for each session.

\section*{Question 120}
\textbf{Metadata}

\begin{itemize}
  \item Question ID: P6-FrDivPN\_P5-FrCnv2Dc\_GPT4.1\_Services\_07
  \item Primary KC: FRACTIONS | Division | dividing a proper fraction by a whole number
  \item Secondary KC: FRACTIONS | Conversion to decimals | expressing fractions as decimals
  \item Topic: Services such as installation, maintenance, repairing, cleaning, laundry, e-commerce
  \item Grade: Primary 6
\end{itemize}

\textbf{Question}

Mrs Yap hires a cleaning service to tidy up her house. The cleaners use $\frac{3}{4}$ litre of cleaning solution to clean the entire house. If the cleaning team consists of 5 people and they share the cleaning solution equally, 

(a) How much cleaning solution does each person get? 

(b) Express the amount each person gets as a decimal.

\textbf{Solution}

(a) Each person gets $\frac{3}{4} \div 5$ litres of cleaning solution.

$\frac{3}{4} \div 5 = \frac{3}{4} \times \frac{1}{5} = \frac{3}{20}$ litres per person.

(b) To express $\frac{3}{20}$ as a decimal, divide 3 by 20:

$\frac{3}{20} = 0.15$

So, each person gets $\frac{3}{20}$ litres or $0.15$ litres of cleaning solution.

\section*{Question 121}
\textbf{Metadata}

\begin{itemize}
  \item Question ID: P6-FrDivPN\_P5-FrMul2nd\_GPT4.1\_Services\_02
  \item Primary KC: FRACTIONS | Division | dividing a proper fraction by a whole number
  \item Secondary KC: FRACTIONS | Multiplication | fraction multiplication
  \item Topic: Services such as installation, maintenance, repairing, cleaning, laundry, e-commerce
  \item Grade: Primary 6
\end{itemize}

\textbf{Question}

A laundry shop uses $\frac{3}{4}$ litres of detergent to wash 5 loads of clothes. The shop receives an order to wash 3 loads. 

(a) How much detergent, in litres, does the shop use for 1 load of clothes? 

(b) How much detergent, in litres, is needed to wash the 3 loads?

\textbf{Solution}

(a) To find the amount of detergent used for 1 load, divide the total detergent by the number of loads: 

$\frac{3}{4} \div 5 = \frac{3}{4} \times \frac{1}{5} = \frac{3}{20}$ litres$.$ 

So, the shop uses $\frac{3}{20}$ litres for 1 load.

(b) For 3 loads: $\frac{3}{20} \times 3 = \frac{9}{20}$ litres$.$

Therefore, the shop needs $\frac{9}{20}$ litres of detergent to wash the 3 loads.

\section*{Question 122}
\textbf{Metadata}

\begin{itemize}
  \item Question ID: P6-FrDivPP\_P2-FrCmp\_GPT4.1\_Services\_07
  \item Primary KC: FRACTIONS | Division | dividing a whole number/proper fraction by a proper fraction
  \item Secondary KC: FRACTIONS | Comparison and ordering | comparing and ordering fractions
  \item Topic: Services such as installation, maintenance, repairing, cleaning, laundry, e-commerce
  \item Grade: Primary 6
\end{itemize}

\textbf{Question}

A cleaning company charges $\textdollar90$ for cleaning one house. Each team of cleaners can finish cleaning $\frac{2}{3}$ of a house in one hour.

(a) How many hours will it take for one team to finish cleaning one house?

(b) After checking their schedule, the company compared the time it takes to clean three houses: House A, House B, and House C. The team took $\frac{9}{5}$ hours for House A, $\frac{7}{4}$ hours for House B, and $\frac{11}{6}$ hours for House C. Arrange the houses in order from the shortest to the longest cleaning time.


\textbf{Solution}

(a) To find the number of hours needed to finish cleaning one house:

One team cleans $\frac{2}{3}$ of a house in $1$ hour.

To clean $1$ house, the time needed $= 1 \div \frac{2}{3}$

$= 1 \times \frac{3}{2} = \frac{3}{2}$ hours

So, it will take $\frac{3}{2}$ hours (or $1$ hour $30$ minutes) for one team to finish cleaning one house.

(b) To arrange the cleaning times from shortest to longest:

House A: $\frac{9}{5}$ hours

House B: $\frac{7}{4}$ hours

House C: $\frac{11}{6}$ hours

Let's convert all times to decimals for easier comparison:

$\frac{9}{5} = 1.8$

$\frac{7}{4} = 1.75$

$\frac{11}{6} \approx 1.83$

Arranged from shortest to longest:

House B ($\frac{7}{4}$ hours), House A ($\frac{9}{5}$ hours), House C ($\frac{11}{6}$ hours).


\section*{Question 123}
\textbf{Metadata}

\begin{itemize}
  \item Question ID: P6-FrDivPP\_P2-FrAdd2nd\_GPT4.1\_Services\_02
  \item Primary KC: FRACTIONS | Division | dividing a whole number/proper fraction by a proper fraction
  \item Secondary KC: FRACTIONS | Addition | adding fractions
  \item Topic: Services such as installation, maintenance, repairing, cleaning, laundry, e-commerce
  \item Grade: Primary 6
\end{itemize}

\textbf{Question}

A cleaning company uses $\frac{2}{3}$ of a bottle of cleaning solution for each apartment they clean. Last Saturday, they used $8$ bottles of cleaning solution in total. 

(a) How many apartments did the company clean last Saturday? 

On the same day, the company cleaned $3$ more offices, each requiring $\frac{3}{4}$ of a bottle of cleaning solution. 

(b) What is the total amount of cleaning solution, in bottles, used for both apartments and offices last Saturday? Give your answer as a mixed number if necessary.

\textbf{Solution}

(a) Number of apartments cleaned $= 8 \div \frac{2}{3} = 8 \times \frac{3}{2} = 12$.

The company cleaned $12$ apartments.

(b) Total solution used for offices $= 3 \times \frac{3}{4} = \frac{9}{4} = 2\frac{1}{4}$ bottles.

Total solution used $= 8 + 2\frac{1}{4} = 10\frac{1}{4}$ bottles.

The company used a total of $10\frac{1}{4}$ bottles of cleaning solution last Saturday.

\section*{Question 124}
\textbf{Metadata}

\begin{itemize}
  \item Question ID: P6-FrDivPP\_P2-FrSub2nd\_GPT4.1\_Services\_02
  \item Primary KC: FRACTIONS | Division | dividing a whole number/proper fraction by a proper fraction
  \item Secondary KC: FRACTIONS | Subtraction | subtracting fractions
  \item Topic: Services such as installation, maintenance, repairing, cleaning, laundry, e-commerce
  \item Grade: Primary 6
\end{itemize}

\textbf{Question}

A cleaning company uses $\frac{2}{3}$ litres of cleaning solution to clean one classroom. They have $8$ litres of cleaning solution in total. After cleaning some classrooms, they found that $\frac{1}{2}$ litre of solution had spilled and could not be used. 

(a) How many classrooms can they clean with the $8$ litres of solution before the spill? 

(b) After the $\frac{1}{2}$ litre was spilled, how many classrooms can they clean now?

\textbf{Solution}

First, find how many classrooms can be cleaned with $8$ litres:

(a) $8$ litres $\div \frac{2}{3}$ litre per classroom $= 8 \times \frac{3}{2} = 12$ classrooms.

(b) After $\frac{1}{2}$ litre is spilled, the remaining solution = $8 - \frac{1}{2} = \frac{16}{2} - \frac{1}{2} = \frac{15}{2}$ litres.

Number of classrooms that can now be cleaned:

$\frac{15}{2} \div \frac{2}{3} = \frac{15}{2} \times \frac{3}{2} = \frac{45}{4} = 11\frac{1}{4}$ classrooms.

Therefore, the company can clean $12$ classrooms before the spill, and $11\frac{1}{4}$ classrooms after the spill.

\section*{Question 125}
\textbf{Metadata}

\begin{itemize}
  \item Question ID: P6-FrDivPP\_P3-FrSmp\_GPT4.1\_Services\_07
  \item Primary KC: FRACTIONS | Division | dividing a whole number/proper fraction by a proper fraction
  \item Secondary KC: FRACTIONS | Simplifying | expressing a fraction in its simplest form
  \item Topic: Services such as installation, maintenance, repairing, cleaning, laundry, e-commerce
  \item Grade: Primary 6
\end{itemize}

\textbf{Question}

A cleaning company uses $\frac{3}{4}$ litres of detergent to clean one office room. If they have $6$ litres of detergent, how many office rooms can they clean? Express your answer as a fraction in its simplest form.

\textbf{Solution}

To find how many office rooms can be cleaned, divide the total amount of detergent by the amount needed per room:

\[
6 \div \frac{3}{4}
\]

Dividing by a fraction is the same as multiplying by its reciprocal:

\[
6 \times \frac{4}{3} = \frac{6 \times 4}{3} = \frac{24}{3} = 8
\]

So, the cleaning company can clean $8$ office rooms. Expressed as a fraction in its simplest form, the answer is:

\[
8
\]

Therefore, they can clean $8$ office rooms.

\section*{Question 126}
\textbf{Metadata}

\begin{itemize}
  \item Question ID: P6-FrDivPP\_P5-FrCnv2Dc\_GPT4.1\_Services\_07
  \item Primary KC: FRACTIONS | Division | dividing a whole number/proper fraction by a proper fraction
  \item Secondary KC: FRACTIONS | Conversion to decimals | expressing fractions as decimals
  \item Topic: Services such as installation, maintenance, repairing, cleaning, laundry, e-commerce
  \item Grade: Primary 6
\end{itemize}

\textbf{Question}

A cleaning company uses $\frac{3}{4}$ of a litre of detergent to clean one house. The company purchases $12$ litres of detergent for a day's work. 

(a) How many houses can the company clean with $12$ litres of detergent?

(b) If the company cleaned only $8$ houses that day, what fraction of the detergent did they use? Express your answer as a decimal.

\textbf{Solution}

(a) Number of houses that can be cleaned:

\[
\frac{12}{\frac{3}{4}} = 12 \times \frac{4}{3} = \frac{48}{3} = 16
\]
So, the company can clean $16$ houses with $12$ litres of detergent.

(b) Detergent used to clean $8$ houses:

\[
8 \times \frac{3}{4} = \frac{24}{4} = 6
\]
They used $6$ litres out of $12$ litres.

Fraction of detergent used:
\[
\frac{6}{12} = \frac{1}{2}
\]
Expressed as a decimal:
\[
\frac{1}{2} = 0.5
\]

So, they used $0.5$ (or $50\%$) of the detergent.

\section*{Question 127}
\textbf{Metadata}

\begin{itemize}
  \item Question ID: P6-FrDivPP\_P5-FrMul2nd\_GPT4.1\_Services\_02
  \item Primary KC: FRACTIONS | Division | dividing a whole number/proper fraction by a proper fraction
  \item Secondary KC: FRACTIONS | Multiplication | fraction multiplication
  \item Topic: Services such as installation, maintenance, repairing, cleaning, laundry, e-commerce
  \item Grade: Primary 6
\end{itemize}

\textbf{Question}

A cleaning company has a large bottle of detergent that contains $6$ litres of cleaning solution. Each cleaning job requires $\frac{3}{4}$ litre of the solution. 

(a) How many complete cleaning jobs can the company perform with the $6$ litres of detergent?

(b) If each cleaning job requires $\frac{2}{3}$ the amount of detergent used by the previous job (starting from $\frac{3}{4}$ litre for the first), what is the total amount of detergent used for the first two jobs?

\textbf{Solution}

(a) Number of complete cleaning jobs possible:
$$
\text{Number of jobs} = \frac{6}{\frac{3}{4}} = 6 \times \frac{4}{3} = 8
$$
So the company can perform $8$ complete cleaning jobs.

(b) Amount of detergent for the first job: $\frac{3}{4}$ litre

Amount of detergent for the second job:
$$
\frac{2}{3} \times \frac{3}{4} = \frac{2 \times 3}{3 \times 4} = \frac{6}{12} = \frac{1}{2}
$$
So, amount for the second job is $\frac{1}{2}$ litre.

Total for the first two jobs:
$$
\frac{3}{4} + \frac{1}{2} = \frac{3}{4} + \frac{2}{4} = \frac{5}{4}
$$
Therefore, the total amount of detergent used for the first two jobs is $\frac{5}{4}$ litres or $1\frac{1}{4}$ litres.

\section*{Question 128}
\textbf{Metadata}

\begin{itemize}
  \item Question ID: P6-PcFndWN\_P1-WNAdd2nd\_GPT4.1\_Services\_02
  \item Primary KC: PERCENTAGE | Finding the whole | finding the whole given a part and the percentage
  \item Secondary KC: WHOLE NUMBERS | Addition | adding whole numbers
  \item Topic: Services such as installation, maintenance, repairing, cleaning, laundry, e-commerce
  \item Grade: Primary 6
\end{itemize}

\textbf{Question}

A washing machine servicing company advertises a cleaning deal where customers pay for only $25\%$ of the total servicing cost upfront. Mr Tan paid $\textdollar120$ upfront for the cleaning of his two washing machines. Later, he had one more washing machine added to the service, and the total cost became $\textdollar180$ more than his previous total cost. What is the total servicing cost for all three washing machines?

\textbf{Solution}

Let the total servicing cost for the initial two washing machines be $x$. Mr Tan paid $25\%$ of $x$ as the upfront payment.

So, $0.25x = 120$, 
\[ x = \frac{120}{0.25} = 480. \]

When he adds one more washing machine, the total servicing cost increases by $\textdollar180$.

Therefore, the new total cost $= 480 + 180 = 660$.

\textbf{Answer:} The total servicing cost for all three washing machines is $\textdollar660$.

\section*{Question 129}
\textbf{Metadata}

\begin{itemize}
  \item Question ID: P6-PcFndWN\_P1-WNSub2nd\_GPT4.1\_Services\_02
  \item Primary KC: PERCENTAGE | Finding the whole | finding the whole given a part and the percentage
  \item Secondary KC: WHOLE NUMBERS | Subtraction | subtracting whole numbers
  \item Topic: Services such as installation, maintenance, repairing, cleaning, laundry, e-commerce
  \item Grade: Primary 6
\end{itemize}

\textbf{Question}

A cleaning company charged a customer \$180 to clean 30\% of an office building. After the cleaning, the customer realised that two rooms were missed out, so the company subtracted the cost of those missed rooms from the bill. If each of the missed rooms would have cost \$15 to clean, and there were 2 such rooms, what is the total cost to clean the whole office building? How much should the customer have paid after subtracting the cost of the missed rooms?

\textbf{Solution}

Let the total cost to clean the whole office building be $x$. 

Since 30\% of the building costs \$180:

\[
0.3x = 180 \\
x = \frac{180}{0.3} = 600
\]

Total cost to clean the whole office building is \$600.

The cost of the two missed rooms is \$15 \times 2 = \$30.$

So after subtracting the cost of the missed rooms:

\[
\text{Customer should pay} = 180 - 30 = 150\textdollar
\]

\textbf{Answers:}

1. Total cost to clean the whole office building = \$600
2. The customer should pay \$150 after subtracting the cost of the missed rooms.

\section*{Question 130}
\textbf{Metadata}

\begin{itemize}
  \item Question ID: P6-PcFndWN\_P1-WNMul2nd\_GPT4.1\_Services\_02
  \item Primary KC: PERCENTAGE | Finding the whole | finding the whole given a part and the percentage
  \item Secondary KC: WHOLE NUMBERS | Multiplication | multiplying whole numbers
  \item Topic: Services such as installation, maintenance, repairing, cleaning, laundry, e-commerce
  \item Grade: Primary 6
\end{itemize}

\textbf{Question}

A cleaning company completed $20$ jobs in a week. If $40\%$ of the jobs were for offices and the rest were for houses, how many office jobs did the company complete? If the company charges $\textdollar50$ for each office job, how much did they earn from all the office jobs?

\textbf{Solution}

First, we find the number of office jobs completed.\newline
$40\%$ of $20$ jobs is $\frac{40}{100} \times 20 = 0.4 \times 20 = 8$ jobs.\newline
So, the company completed $8$ office jobs.\newline
Next, we find out how much they earned from these office jobs.\newline
Since each office job costs $\textdollar50$:\newline
$8 \times 50 = 400$\newline
So, the company earned $\textdollar400$ from the office jobs.

\section*{Question 131}
\textbf{Metadata}

\begin{itemize}
  \item Question ID: P6-PcFndWN\_P1-WNDiv2nd\_GPT4.1\_Services\_02
  \item Primary KC: PERCENTAGE | Finding the whole | finding the whole given a part and the percentage
  \item Secondary KC: WHOLE NUMBERS | Division | dividing whole numbers
  \item Topic: Services such as installation, maintenance, repairing, cleaning, laundry, e-commerce
  \item Grade: Primary 6
\end{itemize}

\textbf{Question}

A cleaning company charges their customers based on the total cost of cleaning a house. During a special promotion, the company gives a $25\%$ discount to all customers. After applying the discount, Mrs Tan pays $120\textdollar$ for her house to be cleaned. 

What was the original cost of cleaning Mrs Tan’s house before the discount was given?

If the original cost was shared equally among 4 rooms in her house, how much did each room cost before any discount?

\textbf{Solution}

Let the original cost be $x$.

After a $25\%$ discount, Mrs Tan pays $75\%$ of the original cost:
$$
75\% \text{ of } x = 120\textdollar
$$
$$
0.75x = 120
$$
To find $x$:
$$
x = \frac{120}{0.75} = 160
$$
So, the original cost was $160\textdollar$.

Now, to find the cost per room when shared equally among 4 rooms:
$$
\text{Cost per room} = \frac{160}{4} = 40
$$

Final Answers:

- The original cost was $160\textdollar$.
- The cost per room was $40\textdollar$ before any discount.

\section*{Question 132}
\textbf{Metadata}

\begin{itemize}
  \item Question ID: P6-PcFndChg\_P1-WNAdd2nd\_GPT4.1\_Services\_02
  \item Primary KC: PERCENTAGE | Finding change | finding percentage increase/decrease
  \item Secondary KC: WHOLE NUMBERS | Addition | adding whole numbers
  \item Topic: Services such as installation, maintenance, repairing, cleaning, laundry, e-commerce
  \item Grade: Primary 6
\end{itemize}

\textbf{Question}

A washing machine repair service in Singapore charges $80$\textdollar for a basic repair job. Last month, the company completed $15$ basic repair jobs. This month, they completed $18$ basic repair jobs. What is the percentage increase in the number of basic repair jobs done by the company this month compared to last month?

\textbf{Solution}

First, find the change in the number of repair jobs:

$18 - 15 = 3$ jobs

Next, find the percentage increase:

\[
\text{Percentage Increase} = \frac{\text{Increase}}{\text{Original}} \times 100\%
\]
\[
= \frac{3}{15} \times 100\% = 20\%
\]

The percentage increase in the number of basic repair jobs this month compared to last month is $20\%$.

\section*{Question 133}
\textbf{Metadata}

\begin{itemize}
  \item Question ID: P6-PcFndChg\_P1-WNSub2nd\_GPT4.1\_Services\_02
  \item Primary KC: PERCENTAGE | Finding change | finding percentage increase/decrease
  \item Secondary KC: WHOLE NUMBERS | Subtraction | subtracting whole numbers
  \item Topic: Services such as installation, maintenance, repairing, cleaning, laundry, e-commerce
  \item Grade: Primary 6
\end{itemize}

\textbf{Question}

A washing machine repair company charged Ali $\textdollar180$ for fixing his washing machine last year. This year, the company increased its repair charges. If Ali is charged $\textdollar225$ for the same repair this year, by what percentage did the repair charges increase compared to last year?

\textbf{Solution}

First, find the increase in the repair charges: 

$\text{Increase} = 225 - 180 = 45$

Next, calculate the percentage increase based on last year's charge:

$\text{Percentage Increase} = \frac{\text{Increase}}{\text{Original amount}} \times 100\% = \frac{45}{180} \times 100\%$

$= 0.25 \times 100\% = 25\%$

\textbf{Answer:} The repair charges increased by $25\%$ compared to last year.

\section*{Question 134}
\textbf{Metadata}

\begin{itemize}
  \item Question ID: P6-PcFndChg\_P1-WNMul2nd\_GPT4.1\_Services\_02
  \item Primary KC: PERCENTAGE | Finding change | finding percentage increase/decrease
  \item Secondary KC: WHOLE NUMBERS | Multiplication | multiplying whole numbers
  \item Topic: Services such as installation, maintenance, repairing, cleaning, laundry, e-commerce
  \item Grade: Primary 6
\end{itemize}

\textbf{Question}

A washing machine repair service charges \textdollar80 for each standard repair. Last month, the service company completed 25 standard repairs. This month, the number of repairs increased by $20\%$ compared to last month. 

(a) How many standard repairs did the company do this month?

(b) What is the total amount collected from standard repairs this month?

\textbf{Solution}

(a) Last month, the company completed $25$ repairs. This month, the number of repairs increased by $20\%$.

Number of additional repairs: \[ 20\% \text{ of } 25 = \frac{20}{100} \times 25 = 5 \]

Total repairs this month: \[ 25 + 5 = 30 \]

(b) Each repair costs \textdollar80.

Total amount collected: \[ 30 \times 80 = 2400\textdollar \]

So, the total amount collected from standard repairs this month is $2400\textdollar$.

\section*{Question 135}
\textbf{Metadata}

\begin{itemize}
  \item Question ID: P6-PcFndChg\_P1-WNDiv2nd\_GPT4.1\_Services\_02
  \item Primary KC: PERCENTAGE | Finding change | finding percentage increase/decrease
  \item Secondary KC: WHOLE NUMBERS | Division | dividing whole numbers
  \item Topic: Services such as installation, maintenance, repairing, cleaning, laundry, e-commerce
  \item Grade: Primary 6
\end{itemize}

\textbf{Question}

A cleaning service company charged a total of $300$\textdollar\ for cleaning $5$ offices last month. This month, due to an increase in demand, the charge per office was raised by $20\%$. If the company cleaned the same number of offices this month, how much did the company charge per office this month?

\textbf{Solution}

First, we find the amount charged per office last month: 

\[ \text{Charge per office last month} = \frac{300\textdollar}{5} = 60\textdollar \]

Next, we calculate the new charge per office after the $20\%$ increase:

\[ \text{Increase} = 60\textdollar \times 20\% = 60\textdollar \times 0.2 = 12\textdollar \]

\[ \text{New charge per office} = 60\textdollar + 12\textdollar = 72\textdollar \]

Therefore, the company charged $72$\textdollar\ per office this month.

\section*{Question 136}
\textbf{Metadata}

\begin{itemize}
  \item Question ID: P6-RoFndDvqWN\_P6-RoFndTmWN\_GPT4.1\_Services\_07
  \item Primary KC: RATIO | Finding divided quantities | dividing a given quantity in a given ratio
  \item Secondary KC: RATIO | Finding a missing term | finding the missing term in a pair of equivalent ratios
  \item Topic: Services such as installation, maintenance, repairing, cleaning, laundry, e-commerce
  \item Grade: Primary 6
\end{itemize}

\textbf{Question}

A laundry shop received $90\textdollar$ for washing and ironing clothes in a day. The amount was to be shared between Ali, Ben, and Clara in the ratio $2:3:x$. If Ben received $36\textdollar$ and the amount Ali and Ben received together was $48\textdollar$, what is the value of $x$? How much did Clara receive?

\textbf{Solution}

Let the shares for Ali, Ben, and Clara be $2$ parts, $3$ parts, and $x$ parts, respectively. Their total share is $2 + 3 + x$. The amount Ben received is $3$ parts. Given Ben received $36\textdollar$, so $3$ parts $= 36\textdollar$, which means $1$ part $= 36\textdollar \div 3 = 12\textdollar$.

The combined amount of Ali and Ben is $48\textdollar$, so $2$ parts $+ 3$ parts $= 5$ parts $= 48\textdollar$. $1$ part $= 48\textdollar \div 5 = 9.6\textdollar$. However, from Ben's calculation, $1$ part $= 12\textdollar$, so there is an apparent conflict, but let us check:

From the total received:
- Total amount $= 90\textdollar$
- $1$ part $= 12\textdollar$ (from Ben's share)

Total parts $= 90\textdollar \div 12\textdollar = 7.5$ parts

But $2 + 3 + x = 7.5 \implies x = 2.5$

Check:
- Ali's share: $2$ parts $= 2 \times 12\textdollar = 24\textdollar$
- Ben's share: $3$ parts $= 36\textdollar$
- Clara's share: $x$ parts $= 2.5 \times 12\textdollar = 30\textdollar$

So, 
$x = 2.5$ 
Clara received $30\textdollar$.

\section*{Question 137}
\textbf{Metadata}

\begin{itemize}
  \item Question ID: P6-RoFndRoWN\_P1-WNAdd2nd\_GPT4.1\_Services\_02
  \item Primary KC: RATIO | Finding ratio | finding the ratio of two or three given whole numbers
  \item Secondary KC: WHOLE NUMBERS | Addition | adding whole numbers
  \item Topic: Services such as installation, maintenance, repairing, cleaning, laundry, e-commerce
  \item Grade: Primary 6
\end{itemize}

\textbf{Question}

A cleaning company received three requests for cleaning services from Shop A, Shop B, and Shop C in one day. Shop A requested cleaning for $18$ tables, Shop B requested cleaning for $24$ tables, and Shop C requested cleaning for $12$ tables. What is the ratio of the number of tables cleaned at Shop A to Shop B to Shop C? What is the total number of tables cleaned by the company that day?

\textbf{Solution}

Number of tables at Shop A: $18$. 

Number of tables at Shop B: $24$. 

Number of tables at Shop C: $12$. 

First, find the ratio of tables cleaned at Shop A : Shop B : Shop C.

The ratio is $18 : 24 : 12$.

Divide all terms by their greatest common divisor, which is $6$:

$\dfrac{18}{6} : \dfrac{24}{6} : \dfrac{12}{6} = 3 : 4 : 2$

So, the ratio is $3 : 4 : 2$.

Next, find the total number of tables cleaned:

$18 + 24 + 12 = 54$

The company cleaned $54$ tables in total.

\section*{Question 138}
\textbf{Metadata}

\begin{itemize}
  \item Question ID: P6-RoFndRoWN\_P1-WNSub2nd\_GPT4.1\_Services\_02
  \item Primary KC: RATIO | Finding ratio | finding the ratio of two or three given whole numbers
  \item Secondary KC: WHOLE NUMBERS | Subtraction | subtracting whole numbers
  \item Topic: Services such as installation, maintenance, repairing, cleaning, laundry, e-commerce
  \item Grade: Primary 6
\end{itemize}

\textbf{Question}

A cleaning company had $120$ cleaning cloths at the start of the week. During the week, $35$ cloths were used and could not be reused. The rest of the cloths were divided among three cleaners: Ahmad, Bala, and Cindy. Ahmad received $20$ cloths, Bala received $15$ cloths, and Cindy received the rest. 

(a) How many cloths did Cindy receive?
(b) What is the ratio of the number of cloths received by Ahmad, Bala, and Cindy? Give your answer in the simplest form.

\textbf{Solution}

(a) Number of cleaning cloths left after $35$ were used: $120 - 35 = 85$ cloths.

Cloths given to Ahmad and Bala: $20 + 15 = 35$ cloths.

Cloths received by Cindy: $85 - 35 = 50$ cloths.

(b) The number of cloths received by Ahmad, Bala, and Cindy is $20 : 15 : 50$.

To simplify this ratio, divide each number by $5$:
\[
20 : 15 : 50 = \frac{20}{5} : \frac{15}{5} : \frac{50}{5} = 4 : 3 : 10
\]

So, the simplest form of the ratio is $4 : 3 : 10$.

\section*{Question 139}
\textbf{Metadata}

\begin{itemize}
  \item Question ID: P6-RoFndRoWN\_P1-WNMul2nd\_GPT4.1\_Services\_02
  \item Primary KC: RATIO | Finding ratio | finding the ratio of two or three given whole numbers
  \item Secondary KC: WHOLE NUMBERS | Multiplication | multiplying whole numbers
  \item Topic: Services such as installation, maintenance, repairing, cleaning, laundry, e-commerce
  \item Grade: Primary 6
\end{itemize}

\textbf{Question}

A laundry shop received 24 shirts, 36 trousers, and 60 towels to wash in one day. 

(a) Find the ratio of shirts to trousers to towels laundered that day.

(b) If each towel takes 2 times as long to clean as a shirt, and each trousers takes 3 times as long as a shirt, what is the ratio of total time needed to wash all the shirts, trousers, and towels?

Give your ratios in their simplest form.

\textbf{Solution}

(a) Number of shirts : trousers : towels = $24 : 36 : 60$

Find the simplest form:
Divide each by 12:
$24 \div 12 = 2$
$36 \div 12 = 3$
$60 \div 12 = 5$
So, the ratio is $2 : 3 : 5$.

(b) Let the time needed to clean 1 shirt be $t$ minutes.

A towel takes $2t$ minutes, a trousers takes $3t$ minutes.

Total time for all shirts $= 24 \times t = 24t$
Total time for all trousers $= 36 \times 3t = 108t$
Total time for all towels $= 60 \times 2t = 120t$

So the ratio of total time for shirts : trousers : towels is $24t : 108t : 120t$
Divide each by $12t$:
$24t \div 12t = 2$
$108t \div 12t = 9$
$120t \div 12t = 10$

The simplest ratio is $2 : 9 : 10$.

\section*{Question 140}
\textbf{Metadata}

\begin{itemize}
  \item Question ID: P6-RoFndRoWN\_P1-WNDiv2nd\_GPT4.1\_Services\_02
  \item Primary KC: RATIO | Finding ratio | finding the ratio of two or three given whole numbers
  \item Secondary KC: WHOLE NUMBERS | Division | dividing whole numbers
  \item Topic: Services such as installation, maintenance, repairing, cleaning, laundry, e-commerce
  \item Grade: Primary 6
\end{itemize}

\textbf{Question}

A cleaning company sent $12$ cleaners to clean three office buildings. The number of cleaners sent to Building A, Building B, and Building C were in the ratio $2 : 3 : 1$ respectively. 

(a) Find the number of cleaners sent to each building. 

(b) If the cleaners were equally divided among $6$ floors in Building B, how many cleaners were on each floor?

\textbf{Solution}

(a) Let the number of cleaners sent to Building A, Building B, and Building C be $2x$, $3x$, and $1x$ respectively. 

Total ratio parts $= 2 + 3 + 1 = 6$.

So, $6x = 12$ 

$x = \dfrac{12}{6} = 2$

Thus,

- Building A: $2x = 2 \times 2 = 4$ cleaners
- Building B: $3x = 3 \times 2 = 6$ cleaners
- Building C: $1x = 1 \times 2 = 2$ cleaners

(b) Number of cleaners in Building B $= 6$

Number of floors in Building B $= 6$

Number of cleaners per floor $= \dfrac{6}{6} = 1$

So, there was $1$ cleaner on each floor in Building B.

\section*{Question 141}
\textbf{Metadata}

\begin{itemize}
  \item Question ID: P6-RoFndRoWN\_P6-RoSmpWN\_GPT4.1\_Services\_07
  \item Primary KC: RATIO | Finding ratio | finding the ratio of two or three given whole numbers
  \item Secondary KC: RATIO | Simplifying | expressing a ratio in its simplest form
  \item Topic: Services such as installation, maintenance, repairing, cleaning, laundry, e-commerce
  \item Grade: Primary 6
\end{itemize}

\textbf{Question}

A cleaning company has three types of cleaning teams: Team A, Team B, and Team C. In one week, Team A cleaned $36$ offices, Team B cleaned $18$ offices, and Team C cleaned $54$ offices.

(a) Find the ratio of the number of offices cleaned by Team A, Team B, and Team C.

(b) Express this ratio in its simplest form.

\textbf{Solution}

(a) The number of offices cleaned by Team A, Team B, and Team C are $36$, $18$, and $54$, respectively.

So, the ratio is $36:18:54$.

(b) To simplify, we find the greatest common divisor (GCD) of $36$, $18$, and $54$, which is $18$.

Divide each number by $18$:

$\dfrac{36}{18} : \dfrac{18}{18} : \dfrac{54}{18} = 2:1:3$

So, the simplest form of the ratio is $2:1:3$.

\section*{Question 142}
\textbf{Metadata}

\begin{itemize}
  \item Question ID: P6-AgRepLrEx\_P6-AgSmpLrEx\_GPT4.1\_Services\_07
  \item Primary KC: ALGEBRA | Representation and concept | translation of simple real-world situations into linear algebraic expressions
  \item Secondary KC: ALGEBRA | Simplifying | simplifying simple linear expressions
  \item Topic: Services such as installation, maintenance, repairing, cleaning, laundry, e-commerce
  \item Grade: Primary 6
\end{itemize}

\textbf{Question}

A cleaning company charges a fixed fee of $15$\textdollar for travelling to a customer's house and $8$\textdollar for each hour of cleaning. 

(a) Write an algebraic expression for the total amount, in \textdollar, a customer pays if the company spends $h$ hours cleaning the house.

(b) Simplify the expression.

(c) If a customer paid $47$\textdollar in total, how many hours did the company spend cleaning the house?

\textbf{Solution}

(a) The algebraic expression is $15 + 8h$.

(b) The expression $15 + 8h$ is already simplified, as it has one variable term and one constant term.

(c) Let the total amount paid be $47$\textdollar. Set up the equation:

$15 + 8h = 47$

Subtract $15$ from both sides:

$8h = 47 - 15$

$8h = 32$

Divide both sides by $8$:

$h = \dfrac{32}{8}$

$h = 4$

The company spent $4$ hours cleaning the house.

\section*{Question 143}
\textbf{Metadata}

\begin{itemize}
  \item Question ID: P6-AgRepLrEx\_P6-AgEvlLrEx\_GPT4.1\_Services\_07
  \item Primary KC: ALGEBRA | Representation and concept | translation of simple real-world situations into linear algebraic expressions
  \item Secondary KC: ALGEBRA | Evaluation | evaluating simple linear expressions by substitution
  \item Topic: Services such as installation, maintenance, repairing, cleaning, laundry, e-commerce
  \item Grade: Primary 6
\end{itemize}

\textbf{Question}

A washing machine repair service charges a fixed inspection fee of $x$ dollars and an additional repair fee of $y$ dollars for each part replaced. Sarah's washing machine needed 3 parts to be replaced during the repair. 

(a) Write down an expression, in terms of $x$ and $y$, for the total amount Sarah has to pay.

(b) If the inspection fee is $\textdollar40$ and each part replaced costs $\textdollar15$, calculate the total amount Sarah paid for the repair.

\textbf{Solution}

(a) The total amount Sarah has to pay is the inspection fee plus the cost for 3 parts replaced:

\[
\text{Total amount} = x + 3y
\]

(b) Substitute $x = 40$ and $y = 15$ into the expression:
\[
\text{Total amount} = 40 + 3 \times 15 = 40 + 45 = 85
\]
Therefore, Sarah paid $\textdollar85$ in total for the repair.

\section*{Question 144}
\textbf{Metadata}

\begin{itemize}
  \item Question ID: P6-AgSlvLrN\_P6-AgRepLrEx\_GPT4.1\_Services\_07
  \item Primary KC: ALGEBRA | Solving simple linear equations | solving simple linear equations involving whole number coefficient and one variable only
  \item Secondary KC: ALGEBRA | Representation and concept | translation of simple real-world situations into linear algebraic expressions
  \item Topic: Services such as installation, maintenance, repairing, cleaning, laundry, e-commerce
  \item Grade: Primary 6
\end{itemize}

\textbf{Question}

A cleaning company charges a fixed fee of $10\textdollar$ for transportation and $x\textdollar$ for every room cleaned. Avril hires the company to clean 3 rooms in her house and the total cost is $46\textdollar$. 

(a) Write an algebraic equation to represent this situation.

(b) Find the amount Avril is charged for cleaning each room.

\textbf{Solution}

(a) Let $x$ be the amount charged for cleaning each room.

Total cost $= 10 + 3x = 46$

So, the algebraic equation is:
$$
10 + 3x = 46
$$

(b) To find $x$:

Subtract $10$ from both sides:
$$
10 + 3x = 46 \\
3x = 46 - 10 \\
3x = 36
$$

Divide both sides by $3$:
$$
x = \frac{36}{3} = 12
$$

So, Avril is charged $12\textdollar$ for cleaning each room.

\section*{Question 145}
\textbf{Metadata}

\begin{itemize}
  \item Question ID: O1-FDSub\_O1-FDAdd\_GPT4.1\_Services\_03
  \item Primary KC: FRACTIONS AND DECIMALS | Subtraction | Subtraction involving one fraction and one decimal number
  \item Secondary KC: FRACTIONS AND DECIMALS | Addition | Addition involving one fraction and one decimal number
  \item Topic: Services such as installation, maintenance, repairing, cleaning, laundry, e-commerce
  \item Grade: Secondary O-level 1
\end{itemize}

\textbf{Question}

A washing machine repairman is paid for his services as follows: He charges $2.5$ hours for installation work. This includes $1\frac{1}{4}$ hours spent cleaning up after the installation.

Later, he is hired again for a maintenance job that takes $0.75$ hours. 

(a) What is the total time, in hours, the repairman spent on both jobs?

(b) Out of the $2.5$ installation hours, how many hours did he spend on installation work excluding cleaning up?

\textbf{Solution}

(a) The total time spent on both jobs is:

$2.5 + 0.75 = 3.25$ hours.

(b) The time spent on installation work excluding cleaning up is:

$2.5 - 1\frac{1}{4}$

First, convert $1\frac{1}{4}$ to decimal:
$1\frac{1}{4} = 1 + \frac{1}{4} = 1 + 0.25 = 1.25$

So, $2.5 - 1.25 = 1.25$ hours.

\textbf{Final Answers:}
(a) $3.25$ hours
(b) $1.25$ hours

\section*{Question 146}
\textbf{Metadata}

\begin{itemize}
  \item Question ID: O1-FDMul\_O1-FDAdd\_GPT4.1\_Services\_03
  \item Primary KC: FRACTIONS AND DECIMALS | Multiplication | Multiplication involving one fraction and one decimal number
  \item Secondary KC: FRACTIONS AND DECIMALS | Addition | Addition involving one fraction and one decimal number
  \item Topic: Services such as installation, maintenance, repairing, cleaning, laundry, e-commerce
  \item Grade: Secondary O-level 1
\end{itemize}

\textbf{Question}

A cleaning company charges $\$10.50$ per hour for basic cleaning. For a special service, they charge $\frac{3}{4}$ times as much per hour as the basic cleaning. Sarah hired the company for 2 hours of basic cleaning and 1.5 hours of the special service. What is the total amount Sarah needs to pay?

\textbf{Solution}

First, find the hourly rate for the special service:

\[
\text{Special rate} = \frac{3}{4} \times 10.50 = 0.75 \times 10.50 = \$7.88
\]

(The exact calculation: $0.75 \times 10.50 = 7.875$, so $\$7.88$ to the nearest cent.)

Next, calculate the cost for each service:

\[
\begin{align*}
\text{Cost for basic cleaning} &= 2 \times 10.50 = \$21.00 \\
\text{Cost for special service} &= 1.5 \times 7.88 = \$11.82
\end{align*}
\]

Now, add the two amounts (using addition with a decimal and another decimal that comes from a product with a fraction):

\[
\text{Total amount} = 21.00 + 11.82 = \$32.82
\]

Sarah needs to pay $\$32.82$ in total.


\section*{Question 147}
\textbf{Metadata}

\begin{itemize}
  \item Question ID: O1-FDMul\_O1-FDSub\_GPT4.1\_Services\_03
  \item Primary KC: FRACTIONS AND DECIMALS | Multiplication | Multiplication involving one fraction and one decimal number
  \item Secondary KC: FRACTIONS AND DECIMALS | Subtraction | Subtraction involving one fraction and one decimal number
  \item Topic: Services such as installation, maintenance, repairing, cleaning, laundry, e-commerce
  \item Grade: Secondary O-level 1
\end{itemize}

\textbf{Question}

A cleaning company charges $0.75\textdollar$ per square metre to clean carpets. A customer asks the company to clean $\frac{2}{3}$ of her living room, which is $30$ square metres in area. She has already paid a deposit of $8.50\textdollar$. How much more does she need to pay after the cleaning is done?

\textbf{Solution}

First, calculate the area to be cleaned: $\frac{2}{3} \times 30 = 20$ square metres. 

Next, find the total cleaning charge: $0.75\textdollar \times 20 = 15\textdollar$. 

She has already paid a deposit of $8.50\textdollar$, so the remaining amount is: $15\textdollar - 8.50\textdollar = 6.50\textdollar$.

\textbf{Final Answer:} She needs to pay $6.50\textdollar$ more after the cleaning is done.

\section*{Question 148}
\textbf{Metadata}

\begin{itemize}
  \item Question ID: O1-FDDiv\_O1-FDAdd\_GPT4.1\_Services\_03
  \item Primary KC: FRACTIONS AND DECIMALS | Division | Division involving one fraction and one decimal number
  \item Secondary KC: FRACTIONS AND DECIMALS | Addition | Addition involving one fraction and one decimal number
  \item Topic: Services such as installation, maintenance, repairing, cleaning, laundry, e-commerce
  \item Grade: Secondary O-level 1
\end{itemize}

\textbf{Question}

A laundry service charges $\$4.80$ to clean a fractional load of clothes. Sarah has a load that is $\frac{3}{5}$ of the full capacity of the washing machine. She also has an additional batch of clothes which weighs $0.25$ of the machine's full capacity. If she sends both batches at once (together as one combined load),

(a) What is the total fraction of the washing machine's capacity that will be cleaned?

(b) The laundry charges are applied to each full load. How many full loads is Sarah sending in total? (Give your answer as a decimal number.)

(c) How much does Sarah have to pay for this laundry service, if the shop charges for fractional amounts accordingly?


\textbf{Solution}

(a) Total fraction of the washing machine's capacity:
Sarah sends $\frac{3}{5}$ and $0.25$ of the capacity.
Convert $0.25$ to a fraction: $0.25 = \frac{1}{4}$.
Total fraction $= \frac{3}{5} + 0.25 = \frac{3}{5} + \frac{1}{4}$
Find common denominator:
$\frac{3}{5} = \frac{12}{20}$,
$\frac{1}{4} = \frac{5}{20}$
Total $= \frac{12}{20} + \frac{5}{20} = \frac{17}{20}$.

(b) Number of full loads Sarah is sending:
She is sending $\frac{17}{20}$ of a full load. In decimal, this is:
$\frac{17}{20} = 0.85$
So Sarah is sending $0.85$ full loads.

(c) Total cost Sarah has to pay:
Each full load costs $\$4.80$.
So, total cost $= 0.85 \times 4.80$
$= 4.08$

Sarah has to pay $\$4.08$ for the laundry service.

\section*{Question 149}
\textbf{Metadata}

\begin{itemize}
  \item Question ID: O1-FDDiv\_O1-FDSub\_GPT4.1\_Services\_03
  \item Primary KC: FRACTIONS AND DECIMALS | Division | Division involving one fraction and one decimal number
  \item Secondary KC: FRACTIONS AND DECIMALS | Subtraction | Subtraction involving one fraction and one decimal number
  \item Topic: Services such as installation, maintenance, repairing, cleaning, laundry, e-commerce
  \item Grade: Secondary O-level 1
\end{itemize}

\textbf{Question}

A cleaning company uses $3.6$ litres of detergent to clean $1\frac{1}{2}$ houses. 

(a) How many litres of detergent does the company use to clean one house? Give your answer in decimal form.

After cleaning one house, the company had $2.2$ litres of detergent left.

(b) How much detergent did the company use for that house? Give your answer as a decimal.

\textbf{Solution}

(a) To find the amount of detergent used for one house, divide the total detergent by the number of houses:

\[
\text{Amount per house} = \frac{3.6}{1\frac{1}{2}} = \frac{3.6}{\frac{3}{2}} = 3.6 \times \frac{2}{3} = \frac{7.2}{3} = 2.4
\]

So, the company uses $2.4$ litres for one house.

(b) The company had $2.2$ litres left after cleaning one house. The detergent used is:

\[
2.4 - 2.2 = 0.2
\]

So, the company used $0.2$ litres of detergent for that house.

\section*{Question 150}
\textbf{Metadata}

\begin{itemize}
  \item Question ID: O1-FDDiv\_O1-FDMul\_GPT4.1\_Services\_03
  \item Primary KC: FRACTIONS AND DECIMALS | Division | Division involving one fraction and one decimal number
  \item Secondary KC: FRACTIONS AND DECIMALS | Multiplication | Multiplication involving one fraction and one decimal number
  \item Topic: Services such as installation, maintenance, repairing, cleaning, laundry, e-commerce
  \item Grade: Secondary O-level 1
\end{itemize}

\textbf{Question}

A cleaning company charges $0.8$ dollars per $rac{2}{3}$ square metre to clean a carpet. Mrs Lim wants to clean a carpet that has an area of $6.4$ square metres. 

(a) How many groups of $rac{2}{3}$ square metre are there in $6.4$ square metres?

(b) What is the total cost for cleaning Mrs Lim’s carpet?

\textbf{Solution}

Let's solve part (a) first: 

(a) Number of groups of $\frac{2}{3}$ square metre in $6.4$ square metres:

To find this, divide $6.4$ by $\frac{2}{3}$:

\[
6.4 \div \frac{2}{3} = 6.4 \times \frac{3}{2} = (6.4 \times 3) \div 2 = 19.2 \div 2 = 9.6
\]

So, there are $9.6$ groups of $\frac{2}{3}$ square metre in $6.4$ square metres.

(b) Total cost for cleaning Mrs Lim’s carpet:

Each group costs $\textdollar0.8$, so multiply $9.6 \times 0.8$:

\[
9.6 \times 0.8 = 7.68
\]

Therefore, the total cost is $\textdollar7.68$.

\section*{Question 151}
\textbf{Metadata}

\begin{itemize}
  \item Question ID: O1-RoRepFr\_P2-FrAdd2nd\_GPT4.1\_Services\_02
  \item Primary KC: RATIO | Representation and concept | ratios involving fractions
  \item Secondary KC: FRACTIONS | Addition | adding fractions
  \item Topic: Services such as installation, maintenance, repairing, cleaning, laundry, e-commerce
  \item Grade: Secondary O-level 1
\end{itemize}

\textbf{Question}

A laundry service sorts clothes into three types of washes: regular, delicate, and heavy-duty. The ratio of the number of baskets for regular to delicate to heavy-duty washes is $\frac{1}{2} : \frac{1}{4} : \frac{1}{4}$. If there are a total of $32$ baskets, how many baskets are used for delicate and heavy-duty washes altogether?

\textbf{Solution}

Let the number of baskets for regular, delicate, and heavy-duty washes be in the ratio $\frac{1}{2} : \frac{1}{4} : \frac{1}{4}$. 

First, let $k$ be the common multiplying factor. So,

Regular: $\frac{1}{2}k$

Delicate: $\frac{1}{4}k$

Heavy-duty: $\frac{1}{4}k$

Sum of ratios:
$\frac{1}{2}k + \frac{1}{4}k + \frac{1}{4}k = k$

So, $k = 32$ (since there are $32$ baskets in total).

Number of baskets for delicate and heavy-duty washes together:
$\frac{1}{4}k + \frac{1}{4}k = \frac{2}{4}k = \frac{1}{2}k$

$\frac{1}{2} \times 32 = 16$

\textbf{Therefore, there are $16$ baskets used for delicate and heavy-duty washes altogether.}

\section*{Question 152}
\textbf{Metadata}

\begin{itemize}
  \item Question ID: O1-RoRepFr\_P2-FrSub2nd\_GPT4.1\_Services\_02
  \item Primary KC: RATIO | Representation and concept | ratios involving fractions
  \item Secondary KC: FRACTIONS | Subtraction | subtracting fractions
  \item Topic: Services such as installation, maintenance, repairing, cleaning, laundry, e-commerce
  \item Grade: Secondary O-level 1
\end{itemize}

\textbf{Question}

A laundry service has two types of detergent: blue and green. For every $\frac{3}{4}$ litre of blue detergent used, the service uses $\frac{1}{2}$ litre of green detergent to wash a batch of uniforms. Last week, they used $2$ litres of blue detergent in total.

(a) What is the ratio of the amount of blue detergent used to green detergent used for one batch, expressed in its simplest form?

(b) If the service had $2$ litres of green detergent at the start of the week and used some for washing uniforms, how much green detergent was left at the end of the week?

\textbf{Solution}

(a) For one batch, the amount of blue to green detergent used is $\frac{3}{4} : \frac{1}{2}$. 

To express this ratio in simplest form:

$\frac{3}{4} : \frac{1}{2} = \frac{3}{4} \div \frac{1}{2} = \frac{3}{4} \times \frac{2}{1} = \frac{6}{4} = \frac{3}{2}$

So the simplified ratio is $3:2$.

(b) In total, 2 litres of blue detergent were used. Each batch uses $\frac{3}{4}$ litre of blue detergent, so:

Number of batches = $2 \div \frac{3}{4} = 2 \times \frac{4}{3} = \frac{8}{3}$ batches.

For each batch, $\frac{1}{2}$ litre of green detergent is used. So, total green detergent used = $\frac{8}{3} \times \frac{1}{2} = \frac{8}{6} = \frac{4}{3}$ litres.

Amount of green detergent left = Initial amount $-$ Used amount

$2 - \frac{4}{3} = \frac{6}{3} - \frac{4}{3} = \frac{2}{3}$ litres

\textbf{Answers:}

(a) $3:2$

(b) $\frac{2}{3}$ litres of green detergent left.

\section*{Question 153}
\textbf{Metadata}

\begin{itemize}
  \item Question ID: O1-RoRepFr\_P5-FrMul2nd\_GPT4.1\_Services\_02
  \item Primary KC: RATIO | Representation and concept | ratios involving fractions
  \item Secondary KC: FRACTIONS | Multiplication | fraction multiplication
  \item Topic: Services such as installation, maintenance, repairing, cleaning, laundry, e-commerce
  \item Grade: Secondary O-level 1
\end{itemize}

\textbf{Question}

A laundry company uses a special detergent mix for cleaning towels and blankets. For every $\frac{2}{3}$ kg of detergent used for towels, $\frac{3}{5}$ kg is used for blankets. During a cleaning session, the company used $6$ kg of detergent in total, maintaining the same ratio. 

(a) How many kilograms of detergent were used for towels? 

(b) If the company uses $\frac{5}{8}$ of the detergent for towels to clean hotel towels, how many kilograms of detergent were used for hotel towels?

\textbf{Solution}

(a) The ratio of detergent used for towels to blankets is $\frac{2}{3} : \frac{3}{5}$. 

To simplify the ratio, express both terms with a common denominator:

$\frac{2}{3} = \frac{10}{15}$, $\frac{3}{5} = \frac{9}{15}$

So, the ratio is $10:9$.

Let the detergent used for towels be $10x$ kg and for blankets be $9x$ kg. Total detergent used:

$10x + 9x = 19x = 6$

$x = \frac{6}{19}$

Detergent for towels: $10x = 10 \times \frac{6}{19} = \frac{60}{19}$ kg $
\approx 3.16$ kg.

(b) Detergent used for hotel towels: $\frac{5}{8}$ of $\frac{60}{19}$ kg

$= \frac{5}{8} \times \frac{60}{19} = \frac{300}{152}$ kg $= \frac{75}{38}$ kg $
\approx 1.97$ kg.

**Answer:**

a) $\frac{60}{19}$ kg detergent was used for towels.
b) $\frac{75}{38}$ kg detergent was used to clean hotel towels.

\section*{Question 154}
\textbf{Metadata}

\begin{itemize}
  \item Question ID: O1-RoRepFr\_P6-FrDiv2nd\_GPT4.1\_Services\_02
  \item Primary KC: RATIO | Representation and concept | ratios involving fractions
  \item Secondary KC: FRACTIONS | Division | fraction division
  \item Topic: Services such as installation, maintenance, repairing, cleaning, laundry, e-commerce
  \item Grade: Secondary O-level 1
\end{itemize}

\textbf{Question}

A laundry service has a special offer on washing two types of blankets. For every $2$ kg of regular blankets washed, they wash $\frac{3}{4}$ kg of wool blankets. If a customer brings in $\frac{9}{2}$ kg of regular blankets and wants to maintain the same ratio, how many kilograms of wool blankets should be washed? How many times heavier is the amount of wool blankets compared to $1$ kg of regular blankets?

\textbf{Solution}

Let the amount of wool blankets to be washed be $x$ kg.

The ratio given is $2$ kg (regular) : $\frac{3}{4}$ kg (wool), which means for every $2$ kg of regular, the corresponding wool to wash is $\frac{3}{4}$ kg.

First, find how many groups of $2$ kg are in $\frac{9}{2}$ kg:

$\frac{9}{2} \div 2 = \frac{9}{2} \times \frac{1}{2} = \frac{9}{4}$

Each group of $2$ kg corresponds to $\frac{3}{4}$ kg of wool, so:

Amount of wool blankets $= \frac{9}{4} \times \frac{3}{4} = \frac{27}{16}$ kg

So, the customer should wash $\frac{27}{16}$ kg of wool blankets.

Next, to find how many times heavier this is compared to $1$ kg of regular blankets (at this ratio):

At the same ratio, for $1$ kg of regular blankets:

$2$ kg regular $: \frac{3}{4}$ kg wool

So, $1$ kg regular $: \frac{3}{4} \div 2 = \frac{3}{4} \times \frac{1}{2} = \frac{3}{8}$ kg wool

Now, $\frac{27}{16} \div \frac{3}{8} = \frac{27}{16} \times \frac{8}{3} = \frac{27 \times 8}{16 \times 3} = \frac{216}{48} = 4.5$

Answer:
The customer should wash $\frac{27}{16}$ kg of wool blankets. This amount is $4.5$ times heavier than the amount of wool blankets needed for $1$ kg of regular blankets at the same ratio.

\section*{Question 155}
\textbf{Metadata}

\begin{itemize}
  \item Question ID: O1-RoRepFr\_O1-RoSmpFr\_GPT4.1\_Services\_03
  \item Primary KC: RATIO | Representation and concept | ratios involving fractions
  \item Secondary KC: RATIO | Simplifying | converting a ratio involving fractions to its simplest form
  \item Topic: Services such as installation, maintenance, repairing, cleaning, laundry, e-commerce
  \item Grade: Secondary O-level 1
\end{itemize}

\textbf{Question}

A cleaner is washing the windows of two office buildings. The time she spends on each building is in the ratio $\frac{2}{3} : \frac{4}{5}$. If she wants to plan her schedule based on the simplest whole-number ratio of time spent on each building, what is the simplified ratio?

\textbf{Solution}

First, express the ratio $\frac{2}{3}:\frac{4}{5}$ as a single fraction by dividing:

$\frac{2}{3} \div \frac{4}{5} = \frac{2}{3} \times \frac{5}{4} = \frac{2 \times 5}{3 \times 4} = \frac{10}{12}$.

Therefore, the ratio can be rewritten as $\frac{10}{12}:1$ or $10:12$ (since dividing by 1 does not change the ratio).

Now, simplify $10:12$ by dividing both terms by their highest common factor, which is 2:

$10 \div 2 = 5$

$12 \div 2 = 6$

So, the simplest whole-number ratio is $5:6$.

**Final Answer:**

The simplified ratio of the time spent on the two buildings is $5:6$.

\section*{Question 156}
\textbf{Metadata}

\begin{itemize}
  \item Question ID: O1-RoRepDc\_P4-DcAdd2nd\_GPT4.1\_Services\_02
  \item Primary KC: RATIO | Representation and concept | ratios involving decimals
  \item Secondary KC: DECIMALS | Addition | adding decimals
  \item Topic: Services such as installation, maintenance, repairing, cleaning, laundry, e-commerce
  \item Grade: Secondary O-level 1
\end{itemize}

\textbf{Question}

A cleaning company offers two types of cleaning solutions: Solution X and Solution Y. The ratio of the amount of Solution X used to Solution Y used for cleaning a house is $2.5 : 1.2$. If the cleaners use $3.7\,\text{litres}$ of Solution X for a particular job, what is the total amount (in litres) of both solutions used?

\textbf{Solution}

Let the amount of Solution X used be $2.5$ parts and Solution Y be $1.2$ parts.

Given that $2.5$ parts $= 3.7\,\text{litres}$.
So, $1$ part $= \dfrac{3.7}{2.5} = 1.48\,\text{litres}$.

Thus, amount of Solution Y used $= 1.2$ parts $= 1.2 \times 1.48 = 1.776\,\text{litres}$.

Total amount used $= 3.7 + 1.776 = 5.476\,\text{litres}$.

\textbf{Answer:}
The total amount of both solutions used is $5.476\,\text{litres}$. 

\section*{Question 157}
\textbf{Metadata}

\begin{itemize}
  \item Question ID: O1-RoRepDc\_P4-DcSub2nd\_GPT4.1\_Services\_02
  \item Primary KC: RATIO | Representation and concept | ratios involving decimals
  \item Secondary KC: DECIMALS | Subtraction | subtracting decimals
  \item Topic: Services such as installation, maintenance, repairing, cleaning, laundry, e-commerce
  \item Grade: Secondary O-level 1
\end{itemize}

\textbf{Question}

A laundry service in Singapore charges customers based on the ratio of the weight of regular clothes to the weight of heavy items (such as curtains) in the laundry load. For a particular batch, the ratio of the weight of regular clothes to heavy items is $2.7 : 1.5$. The total weight of the laundry batch is $8.2$ kg. After the wash, $0.6$ kg of water is lost due to evaporation from the regular clothes.

(a) How many kilograms of regular clothes were there before the wash started?

(b) What is the new weight of regular clothes after the water is lost?

\textbf{Solution}

(a) Let the weight of regular clothes = $2.7x$ kg, and the weight of heavy items = $1.5x$ kg.

We are told that:
$2.7x + 1.5x = 8.2$
$4.2x = 8.2$
$x = \frac{8.2}{4.2} = 1.952$

Weight of regular clothes = $2.7 \times 1.952 = 5.2704$ kg (rounded to $2$ decimal places: $5.27$ kg)

(b) After $0.6$ kg of water is lost from the regular clothes, the weight becomes:
$5.27-0.6 = 4.67$ kg

**Answers:**

(a) The weight of regular clothes before the wash is $5.27$ kg.

(b) The new weight of regular clothes after water is lost is $4.67$ kg.

\section*{Question 158}
\textbf{Metadata}

\begin{itemize}
  \item Question ID: O1-RoRepDc\_O1-RoSmpDc\_GPT4.1\_Services\_03
  \item Primary KC: RATIO | Representation and concept | ratios involving decimals
  \item Secondary KC: RATIO | Simplifying | converting a ratio involving decimals to its simplest form
  \item Topic: Services such as installation, maintenance, repairing, cleaning, laundry, e-commerce
  \item Grade: Secondary O-level 1
\end{itemize}

\textbf{Question}

In a laundry service, the staff uses two types of detergent to clean clothes: Type A and Type B. For a particular washing cycle, they used $2.5$ liters of Type A detergent and $1.75$ liters of Type B detergent. 

(a) Express the ratio of the volume of Type A detergent to Type B detergent used in that washing cycle. 

(b) Write the ratio in its simplest whole number form.

\textbf{Solution}

(a) The ratio of Type A to Type B detergent used is $2.5 : 1.75$. 

(b) To simplify, divide both numbers by $0.25$ (the smallest value that gives whole numbers):

$2.5 \div 0.25 = 10$ 

$1.75 \div 0.25 = 7$ 

So, the simplest whole number ratio is $10:7$.

\section*{Question 159}
\textbf{Metadata}

\begin{itemize}
  \item Question ID: O1-PcRep2q\_O1-PcCnv2Fr\_GPT4.1\_Services\_03
  \item Primary KC: PERCENTAGE | Representation and concept | comparing two quantities by percentage
  \item Secondary KC: PERCENTAGE | Conversion to fraction | expressing percentage as a fraction
  \item Topic: Services such as installation, maintenance, repairing, cleaning, laundry, e-commerce
  \item Grade: Secondary O-level 1
\end{itemize}

\textbf{Question}

A dry cleaning service cleaned $200$ shirts in a week. The manager noticed that $15\%$ of the shirts needed special stain treatment before cleaning. 

(a) By how many percent is the number of shirts needing special stain treatment less than the total number of shirts cleaned?

(b) Express the percentage of shirts that needed special stain treatment as a fraction in its simplest form.

\textbf{Solution}

Let the number of shirts needing special stain treatment be $x$.

First, calculate $15\%$ of $200$:

$x = 15\% \times 200 = \frac{15}{100} \times 200 = 30$

(a) The number of shirts needing special stain treatment is $30$ out of a total of $200$. To find by how many percent it is less than the total shirts cleaned:

Difference = Total shirts $- $ Shirts needing special stain treatment $= 200 - 30 = 170$

To find this difference as a percentage of the total:

$\frac{170}{200} \times 100\% = 85\%$

So, the number of shirts needing special stain treatment is $85\%$ less than the total number of shirts cleaned.

(b) Express $15\%$ as a fraction in its simplest form:

$15\% = \frac{15}{100} = \frac{3}{20}$

Final answers:

(a) $85\%$

(b) $\frac{3}{20}$

\section*{Question 160}
\textbf{Metadata}

\begin{itemize}
  \item Question ID: O1-PcRep2q\_O1-PcCnv2Dc\_GPT4.1\_Services\_03
  \item Primary KC: PERCENTAGE | Representation and concept | comparing two quantities by percentage
  \item Secondary KC: PERCENTAGE | Conversion to decimals | expressing percentage as a decimal
  \item Topic: Services such as installation, maintenance, repairing, cleaning, laundry, e-commerce
  \item Grade: Secondary O-level 1
\end{itemize}

\textbf{Question}

A cleaning service company charges \$50 for basic house cleaning. They introduced a new premium package that costs \$62.50 per session.

(a) By what percentage is the premium package more expensive than the basic package?

(b) Express your answer in part (a) as a decimal.

\textbf{Solution}

Let the price of the basic package be \$50 and the premium package be \$62.50.

(a) Find how much more expensive the premium package is:

The difference is $62.50 - 50 = 12.50$.

Percentage increase $= \frac{12.50}{50} \times 100\% = 25\%$.

(b) Express $25\%$ as a decimal:

$25\% = \frac{25}{100} = 0.25$.

\section*{Question 161}
\textbf{Metadata}

\begin{itemize}
  \item Question ID: O1-PcFndRslt\_P1-WNAdd2nd\_GPT4.1\_Services\_02
  \item Primary KC: PERCENTAGE | Finding result after change | increasing/decreasing a quantity by a given percentage
  \item Secondary KC: WHOLE NUMBERS | Addition | adding whole numbers
  \item Topic: Services such as installation, maintenance, repairing, cleaning, laundry, e-commerce
  \item Grade: Secondary O-level 1
\end{itemize}

\textbf{Question}

A company charges \textdollar120 to clean an office. This month, they cleaned 5 offices. Next month, the company plans to increase their cleaning fee by $15\%$. 

(a) How much did the company earn from cleaning the offices *this* month?

(b) What will be the new cleaning fee per office next month after the increase?

(c) If the company cleans the same number of offices next month, how much will they earn then?

\textbf{Solution}

(a) This month, the earning from 5 offices is $120 \times 5 = 600\textdollar$.

(b) The increase is $15\%$ of $120 = 0.15 \times 120 = 18\textdollar$. 
New fee per office is $120 + 18 = 138\textdollar$.

(c) Next month, if they clean 5 offices, the earnings will be $138 \times 5 = 690\textdollar$.

\section*{Question 162}
\textbf{Metadata}

\begin{itemize}
  \item Question ID: O1-PcFndRslt\_P1-WNSub2nd\_GPT4.1\_Services\_02
  \item Primary KC: PERCENTAGE | Finding result after change | increasing/decreasing a quantity by a given percentage
  \item Secondary KC: WHOLE NUMBERS | Subtraction | subtracting whole numbers
  \item Topic: Services such as installation, maintenance, repairing, cleaning, laundry, e-commerce
  \item Grade: Secondary O-level 1
\end{itemize}

\textbf{Question}

A household cleaning service charges \textdollar120 for a full house cleaning. In June, the company increased its price by $15\%$. However, after listening to customer feedback, the company decided to reduce the new price by \textdollar18 for regular customers. \\ 

(a) What is the new price after the $15\%$ increase? \\ 
(b) What is the final price that regular customers pay after the \textdollar18 deduction?

\textbf{Solution}

(a) The original price is \textdollar120. Increase by $15\%$: \\ 
$15\%$ of \textdollar120 is $0.15 \times 120 = 18$. \\ 
New price after the increase $= 120 + 18 = \textdollar138$. \\ 
(b) The company gives a discount of \textdollar18, so: \\ 
Final price $= 138 - 18 = \textdollar120$. \\ 

\textbf{Answers:} \\ 
(a) \textdollar138 \\ 
(b) \textdollar120

\section*{Question 163}
\textbf{Metadata}

\begin{itemize}
  \item Question ID: O1-PcFndRslt\_P1-WNMul2nd\_GPT4.1\_Services\_02
  \item Primary KC: PERCENTAGE | Finding result after change | increasing/decreasing a quantity by a given percentage
  \item Secondary KC: WHOLE NUMBERS | Multiplication | multiplying whole numbers
  \item Topic: Services such as installation, maintenance, repairing, cleaning, laundry, e-commerce
  \item Grade: Secondary O-level 1
\end{itemize}

\textbf{Question}

A cleaning company charges $\textdollar{60}$ to clean one apartment. This year, they increased their cleaning fee by $15\%$. A condominium manager hired the company to clean $8$ apartments after the price increase. What is the new total amount the condominium manager has to pay for cleaning all $8$ apartments?

\textbf{Solution}

First, find the increased price for cleaning one apartment: 

Increase = $15\% \times \textdollar{60} = \frac{15}{100} \times 60 = \textdollar{9}$.

New price per apartment = $\textdollar{60} + \textdollar{9} = \textdollar{69}$.

The total amount for $8$ apartments = $8 \times \textdollar{69} = \textdollar{552}$.

So, the condominium manager has to pay $\boxed{\textdollar{552}}$ in total.

\section*{Question 164}
\textbf{Metadata}

\begin{itemize}
  \item Question ID: O1-PcFndRslt\_P1-WNDiv2nd\_GPT4.1\_Services\_02
  \item Primary KC: PERCENTAGE | Finding result after change | increasing/decreasing a quantity by a given percentage
  \item Secondary KC: WHOLE NUMBERS | Division | dividing whole numbers
  \item Topic: Services such as installation, maintenance, repairing, cleaning, laundry, e-commerce
  \item Grade: Secondary O-level 1
\end{itemize}

\textbf{Question}

A company charges a one-time installation fee of $400$ for setting up a new air-conditioning unit. After a promotional discount, the installation fee is decreased by $25\%$. If a resident association wants to install air-conditioning units in $12$ different units, how much does each unit owner have to pay if the total discounted installation cost is divided equally among all $12$ unit owners?

\textbf{Solution}

First, calculate the amount of discount: $25\%$ of $400$ is $400 \times 0.25 = 100$.\
\
So, the new installation fee after the discount is $400 - 100 = \textdollar300$.\
\
If $12$ unit owners share the cost equally, each person pays $\frac{300}{12} = \textdollar25$.\
\
\textbf{Each unit owner has to pay \textdollar25 for the installation.}

\section*{Question 165}
\textbf{Metadata}

\begin{itemize}
  \item Question ID: O1-PcRepRvs\_O1-PcCnv2Fr\_GPT4.1\_Services\_03
  \item Primary KC: PERCENTAGE | Representation and concept | reverse percentages
  \item Secondary KC: PERCENTAGE | Conversion to fraction | expressing percentage as a fraction
  \item Topic: Services such as installation, maintenance, repairing, cleaning, laundry, e-commerce
  \item Grade: Secondary O-level 1
\end{itemize}

\textbf{Question}

A washing machine cleaning service in Singapore advertises a special offer: after a 20\% discount, the customer pays \textdollar144 for the cleaning service. 

(a) What fraction does the 20\% represent? Express 20\% as a fraction in its simplest form.

(b) What was the original price of the cleaning service before the discount?

\textbf{Solution}

(a) 20\% means $\frac{20}{100} = \frac{1}{5}$. Thus, 20\% as a fraction in its simplest form is $\frac{1}{5}$.

(b) Let the original price be $x$ dollars. After a 20\% discount, the customer pays 80\% of the original price:

$0.8x = 144$

$x = \frac{144}{0.8}$

$x = 180$

So, the original price of the cleaning service before the discount was \textdollar180.

\section*{Question 166}
\textbf{Metadata}

\begin{itemize}
  \item Question ID: O1-PcRepRvs\_O1-PcCnv2Dc\_GPT4.1\_Services\_03
  \item Primary KC: PERCENTAGE | Representation and concept | reverse percentages
  \item Secondary KC: PERCENTAGE | Conversion to decimals | expressing percentage as a decimal
  \item Topic: Services such as installation, maintenance, repairing, cleaning, laundry, e-commerce
  \item Grade: Secondary O-level 1
\end{itemize}

\textbf{Question}

A company offers air-conditioning maintenance services. During a promotion, they announced that after giving a discount of $15\%$ off, customers only need to pay $\textdollar85$ for the standard service. The discounted price is calculated directly from the original price, which is then reduced by the percentage discount. 

(a) What is the original price of the standard service before the discount? 

(b) Express the $15\%$ discount as a decimal and use it to verify your answer for part (a).

\textbf{Solution}

(a) Let the original price be $x$ dollars. After a discount of $15\%$, the customer pays $85\%$ of the original price: 

$0.85x = 85$ 

$x = \frac{85}{0.85}$ 

$x = 100$ 

So, the original price was $\textdollar100$.

(b) Convert $15\%$ to decimal: $\frac{15}{100} = 0.15$

The amount paid after the discount is $(1 - 0.15)x = 0.85x$, so 

$0.85x = 85$ 

$x = \frac{85}{0.85} = 100$

This confirms that the original price is $\textdollar100$.

\section*{Question 167}
\textbf{Metadata}

\begin{itemize}
  \item Question ID: O1-AgRepExSq\_O1-AgEvlEx\_GPT4.1\_Services\_03
  \item Primary KC: ALGEBRA | Representation and concept | translation of simple real-world situations into quadratic algebraic expressions
  \item Secondary KC: ALGEBRA | Evaluation | evaluation of algebraic expressions and formulae
  \item Topic: Services such as installation, maintenance, repairing, cleaning, laundry, e-commerce
  \item Grade: Secondary O-level 1
\end{itemize}

\textbf{Question}

A cleaning company charges a fixed fee of $50$ for each job plus an extra charge based on the square of the number of hours worked. If the extra charge is $2h^2$, where $h$ is the number of hours spent on cleaning, write down an algebraic expression for the total cost of a job in terms of $h$. Then, evaluate the total cost if the cleaners worked for $3$ hours.

\textbf{Solution}

The total cost, $C$, in dollars, can be represented as:

\[
C = 50 + 2h^2
\]

For $h = 3$ hours:

\[
C = 50 + 2(3^2)
\]
\[
= 50 + 2 \times 9
\]
\[
= 50 + 18
\]
\[
= 68
\]

So, the total cost for $3$ hours of cleaning is \textdollar68.

\section*{Question 168}
\textbf{Metadata}

\begin{itemize}
  \item Question ID: O1-AgRepnth\_O1-AgEvlEx\_GPT4.1\_Services\_03
  \item Primary KC: ALGEBRA | Representation and concept | recognising and representing patterns/relationships by finding an algebraic expression for the nth term
  \item Secondary KC: ALGEBRA | Evaluation | evaluation of algebraic expressions and formulae
  \item Topic: Services such as installation, maintenance, repairing, cleaning, laundry, e-commerce
  \item Grade: Secondary O-level 1
\end{itemize}

\textbf{Question}

A laundry service charges a fixed fee of $3\textdollar$ for pick-up and delivery, plus $2\textdollar$ per kilogram of laundry washed. Let $n$ be the number of kilograms of laundry sent to the service in one order.

(a) Find an algebraic expression for the total cost, $C$, of sending $n$ kilograms of laundry to the service.

(b) Using your expression, calculate the total cost if you send $7$ kilograms of laundry.

\textbf{Solution}

(a) The fixed fee is $3\textdollar$ and the cost per kilogram is $2\textdollar$. The total cost, $C$, for $n$ kilograms is:

$$C = 3 + 2n$$

(b) For $n = 7$ kilograms:

$$C = 3 + 2 \times 7$$
$$C = 3 + 14$$
$$C = 17$$

The total cost is $17\textdollar$.

\section*{Question 169}
\textbf{Metadata}

\begin{itemize}
  \item Question ID: O1-AgSlvFrLr\_O1-AgRepEq\_GPT4.1\_Services\_03
  \item Primary KC: ALGEBRA | Solving | solving simple fractional equations that can be reduced to linear equations
  \item Secondary KC: ALGEBRA | Representation and concept | translation of simple real-world situations to equations
  \item Topic: Services such as installation, maintenance, repairing, cleaning, laundry, e-commerce
  \item Grade: Secondary O-level 1
\end{itemize}

\textbf{Question}

A cleaning company charges a fixed fee of $\textdollar 15$ plus $\frac{2}{3}$ of the amount it costs for cleaning materials per job. If Mr. Lim paid a total of $\textdollar 35$ for a cleaning job and the cost of the cleaning materials is $x$, write an equation using $x$ to represent this situation and solve for the cost of the cleaning materials.

\textbf{Solution}

Let the cost of the cleaning materials be $x$ dollars. The total cost Mr. Lim pays is the fixed fee plus $\frac{2}{3}$ of the cost of the cleaning materials, so:

\[
15 + \frac{2}{3}x = 35
\]

Subtract $15$ from both sides:

\[
\frac{2}{3}x = 35 - 15
\]
\[
\frac{2}{3}x = 20
\]

Multiply both sides by $\frac{3}{2}$ to solve for $x$:

\[
x = 20 \times \frac{3}{2}
\]
\[
x = 30
\]

So, the cost of the cleaning materials was $\textdollar 30$.

\section*{Question 170}
\textbf{Metadata}

\begin{itemize}
  \item Question ID: O2-RoRepDP\_P1-WNMul2nd\_GPT4.1\_Services\_02
  \item Primary KC: RATIO | Representation and concept | direct proportion
  \item Secondary KC: WHOLE NUMBERS | Multiplication | multiplying whole numbers
  \item Topic: Services such as installation, maintenance, repairing, cleaning, laundry, e-commerce
  \item Grade: Secondary O-level 2
\end{itemize}

\textbf{Question}

A professional cleaning company charges its customers in direct proportion to the number of cleaners sent for a job. If 3 cleaners cost $90$ to clean an office for one session, how much will it cost to hire 7 cleaners for the same job and session? Then, if a business needs to clean its office 4 times a month and always hires 7 cleaners for each session, how much will it pay in total for one month?

\textbf{Solution}

First, let’s find the cost per cleaner for one session. 

The cost for 3 cleaners for 1 session is $90$. 

Cost per cleaner per session $= \frac{90}{3} = 30$.

For 7 cleaners:

Cost for 7 cleaners $= 7 \times 30 = 210$.

So, 7 cleaners for 1 session costs $210$.

If the business cleans the office 4 times a month using 7 cleaners each time:

Total cost for the month $= 210 \times 4 = 840$.

So, the business will pay $840$ in total for one month.

\section*{Question 171}
\textbf{Metadata}

\begin{itemize}
  \item Question ID: O2-RoRepDP\_P1-WNDiv2nd\_GPT4.1\_Services\_02
  \item Primary KC: RATIO | Representation and concept | direct proportion
  \item Secondary KC: WHOLE NUMBERS | Division | dividing whole numbers
  \item Topic: Services such as installation, maintenance, repairing, cleaning, laundry, e-commerce
  \item Grade: Secondary O-level 2
\end{itemize}

\textbf{Question}

A cleaning service company charges its customers based on the amount of cleaning solution used. The company uses a ratio of $2:5$ between litres of cleaning solution and the number of rooms cleaned. Last week, the company cleaned 20 rooms using this ratio.

(a) How many litres of cleaning solution did the company use to clean 20 rooms?

(b) If the company wants to clean 35 rooms next week using the same ratio, how many litres of cleaning solution should they prepare?

After finding the amount from (b), if each bottle contains $7$ litres of cleaning solution, how many full bottles will be needed?

\textbf{Solution}

(a) The ratio of cleaning solution to rooms cleaned is $2:5$. That means 2 litres for every 5 rooms.

For 20 rooms:

Number of sets of 5 rooms in 20 rooms: $\frac{20}{5} = 4$

So, for 4 sets:

Number of litres needed $= 4 \times 2 = 8$

The company used 8 litres of cleaning solution for 20 rooms.

(b) For 35 rooms:

Number of sets of 5 rooms in 35 rooms: $\frac{35}{5} = 7$

Number of litres needed $= 7 \times 2 = 14$

The company should prepare 14 litres of cleaning solution for 35 rooms.

Now, if each bottle contains $7$ litres:

Number of full bottles needed $= \frac{14}{7} = 2$

They will need 2 full bottles of cleaning solution.

\section*{Question 172}
\textbf{Metadata}

\begin{itemize}
  \item Question ID: O2-RoRepIvP\_P1-WNMul2nd\_GPT4.1\_Services\_02
  \item Primary KC: RATIO | Representation and concept | inverse proportion
  \item Secondary KC: WHOLE NUMBERS | Multiplication | multiplying whole numbers
  \item Topic: Services such as installation, maintenance, repairing, cleaning, laundry, e-commerce
  \item Grade: Secondary O-level 2
\end{itemize}

\textbf{Question}

A cleaning company sends its staff to clean the hallways of an apartment building. If $4$ cleaners can finish the job in $9$ hours, how long will it take for $6$ cleaners to complete the same job, assuming they all work at the same rate and start at the same time? After you find the answer, calculate the total number of cleaning hours used in each situation (number of cleaners multiplied by the number of hours worked). Are the total cleaning hours the same in both cases?

\textbf{Solution}

Since the number of cleaners and the time taken are inversely proportional, we have:

$4 \times 9 = 6 \times t$

Where $t$ is the time in hours it takes $6$ cleaners to finish the job.

So,
$36 = 6t$
$t = \frac{36}{6} = 6$

It will take $6$ cleaners $6$ hours to complete the job.

Next, calculate the total number of cleaning hours in each situation:

1. For $4$ cleaners in $9$ hours: $4 \times 9 = 36$ cleaning hours.
2. For $6$ cleaners in $6$ hours: $6 \times 6 = 36$ cleaning hours.

Yes, the total cleaning hours are the same in both cases: $36$ cleaning hours.

\section*{Question 173}
\textbf{Metadata}

\begin{itemize}
  \item Question ID: O2-RoRepIvP\_P1-WNDiv2nd\_GPT4.1\_Services\_02
  \item Primary KC: RATIO | Representation and concept | inverse proportion
  \item Secondary KC: WHOLE NUMBERS | Division | dividing whole numbers
  \item Topic: Services such as installation, maintenance, repairing, cleaning, laundry, e-commerce
  \item Grade: Secondary O-level 2
\end{itemize}

\textbf{Question}

A cleaning company has a contract to clean an office building. If $6$ workers can clean all the windows in $8$ hours, how long will it take $4$ workers to finish cleaning all the windows, assuming they all work at the same rate? After that, if the $4$ workers receive a total of $\textdollar320$ for the job and they divide the money equally among themselves, how much does each worker get?

\textbf{Solution}

First, we use the concept of inverse proportion between the number of workers and the time required to complete the job. Let $t$ be the required number of hours for $4$ workers:

$$6 \times 8 = 4 \times t$$

$$48 = 4t$$

$$t = 12$$

So, $4$ workers will take $12$ hours to finish cleaning all the windows.

Next, we divide $\textdollar320$ equally among $4$ workers:

$$\frac{320}{4} = 80$$

Each worker receives $\textdollar80$.

**Final Answers:**

- It will take $4$ workers $12$ hours to clean all the windows.
- Each worker receives $\textdollar80$.

\section*{Question 174}
\textbf{Metadata}

\begin{itemize}
  \item Question ID: O2-AgSlvIneq\_O2-AgRepIneq\_GPT4.1\_Services\_02
  \item Primary KC: ALGEBRA | Solving | solving simple linear inequalities with one variable
  \item Secondary KC: ALGEBRA | Representation and concept | translation of simple real-world situations to simple linear inequalities with one variable
  \item Topic: Services such as installation, maintenance, repairing, cleaning, laundry, e-commerce
  \item Grade: Secondary O-level 2
\end{itemize}

\textbf{Question}

A cleaning company charges a fixed fee of $15\textdollar$ for transportation and $x\textdollar$ per hour for cleaning. If a customer has a budget of no more than $60\textdollar$ for a cleaning job that lasted 3 hours, write down an inequality to represent this situation and solve the inequality to find the maximum hourly rate, $x$, that the customer can afford?

\textbf{Solution}

The total cost of the cleaning service is $15\textdollar$ (fixed fee) plus $3x\textdollar$ ($x\textdollar$ per hour for 3 hours). 
Let $x$ be the hourly rate. 
The customer wants to spend no more than $60\textdollar$.

So, the inequality will be:

$$
15 + 3x \leq 60
$$

Subtract 15 from both sides:
$$
3x \leq 45
$$

Divide both sides by 3:
$$
x \leq 15
$$

Therefore, the maximum hourly rate the customer can afford is $15\textdollar$.

\section*{Question 175}
\textbf{Metadata}

\begin{itemize}
  \item Question ID: O2-AgSlvSq1v\_O1-AgRepEq\_GPT4.1\_Services\_02
  \item Primary KC: ALGEBRA | Solving | solving quadratic equations in one variable
  \item Secondary KC: ALGEBRA | Representation and concept | translation of simple real-world situations to equations
  \item Topic: Services such as installation, maintenance, repairing, cleaning, laundry, e-commerce
  \item Grade: Secondary O-level 2
\end{itemize}

\textbf{Question}

A washing machine repair service charges a fixed service fee and an additional charge that depends on the complexity of the repair. For certain complicated repairs, the total cost in dollars for a customer is modeled by the equation:

$$	ext{Total Cost} = 50 + x^2 - 8x$$

where $x$ is the number of hours the technician spends on the repair.

A customer paid $50 for a simple service, meaning $x = 0$, but on another visit, the total bill was $66$ dollars.

Form an equation to represent this situation and find how many hours, $x$, did the repair take for the $66\textdollar$ bill.

\textbf{Solution}

Let $x$ be the number of hours spent on the complicated repair.

Given the total cost:

$$66 = 50 + x^2 - 8x$$

Subtract $50$ from both sides:

$$66 - 50 = x^2 - 8x$$
$$16 = x^2 - 8x$$

Bring all terms to one side to form a quadratic equation:

$$x^2 - 8x - 16 = 0$$

Solve this quadratic equation using the quadratic formula:

$$x = \frac{-b \pm \sqrt{b^2 - 4ac}}{2a}$$
where $a = 1$, $b = -8$, $c = -16$.

$$x = \frac{-(-8) \pm \sqrt{(-8)^2 - 4 \times 1 \times (-16)}}{2 \times 1}$$
$$x = \frac{8 \pm \sqrt{64 + 64}}{2}$$
$$x = \frac{8 \pm \sqrt{128}}{2}$$
$$\sqrt{128} = 8\sqrt{2}$$

So,
$$x = \frac{8 \pm 8\sqrt{2}}{2} = 4 \pm 4\sqrt{2}$$

Only the positive value makes sense for time measured in hours:
$$x = 4 + 4\sqrt{2}$$

Approximating $\sqrt{2} \approx 1.41$:
$$x \approx 4 + 4(1.41) = 4 + 5.64 = 9.64$$

Therefore, the repair took approximately $9.64$ hours.

**Answer:** The repair took approximately $9.64$ hours.

\section*{Question 176}
\textbf{Metadata}

\begin{itemize}
  \item Question ID: O2-AgSlvLr2v\_O1-AgRepEq\_GPT4.1\_Services\_02
  \item Primary KC: ALGEBRA | Solving | solving linear equations in two variables
  \item Secondary KC: ALGEBRA | Representation and concept | translation of simple real-world situations to equations
  \item Topic: Services such as installation, maintenance, repairing, cleaning, laundry, e-commerce
  \item Grade: Secondary O-level 2
\end{itemize}

\textbf{Question}

A company charges a one-time installation fee and a fixed monthly maintenance fee for air-conditioning services. For Lina, who used the service for $5$ months, the total amount paid was $250\textdollar$. For Ahmad, who used the service for $8$ months, the total amount paid was $340\textdollar$. Let $x$ be the one-time installation fee and $y$ be the monthly maintenance fee.

(a) Write two equations based on the information above.

(b) Find the values of $x$ and $y$.

\textbf{Solution}

Let $x$ be the installation fee and $y$ be the monthly maintenance fee.

(a) From the information given: 

Lina: $x + 5y = 250$ 
Ahmad: $x + 8y = 340$

(b) Subtract the first equation from the second:

$[x + 8y] - [x + 5y] = 340 - 250$

$8y - 5y = 90$

$3y = 90$

$y = 30$

Substitute $y$ back into the first equation:

$x + 5(30) = 250$

$x + 150 = 250$

$x = 250 - 150 = 100$

So, the installation fee is $100\textdollar$ and the monthly maintenance fee is $30\textdollar$.

\section*{Question 177}
\textbf{Metadata}

\begin{itemize}
  \item Question ID: O2-SPFndmdn\_O2-SPFndmode\_GPT4.1\_Services\_02
  \item Primary KC: STATISTICS AND PROBABILITY | Finding median | Finding median for a set of data
  \item Secondary KC: STATISTICS AND PROBABILITY | Finding mode | Finding mode for a set of data
  \item Topic: Services such as installation, maintenance, repairing, cleaning, laundry, e-commerce
  \item Grade: Secondary O-level 2
\end{itemize}

\textbf{Question}

A cleaning company recorded the number of offices cleaned by each of its 7 staff members in a particular week. The numbers were: 12, 15, 15, 13, 17, 14, 13. 

(a) Find the median number of offices cleaned by the staff members. 

(b) Find the mode of the number of offices cleaned.

\textbf{Solution}

(a) First, arrange the numbers in ascending order: $12,\ 13,\ 13,\ 14,\ 15,\ 15,\ 17$

With 7 data points (an odd number), the median is the fourth value:

Median $= 14$

(b) The mode is the number that appears most frequently. Both $13$ and $15$ appear twice, while the others appear only once.

Mode $= 13$ and $15$

\section*{Question 178}
\textbf{Metadata}

\begin{itemize}
  \item Question ID: O2-SPFndmdn\_O3-SPFndPctl\_GPT4.1\_Services\_02
  \item Primary KC: STATISTICS AND PROBABILITY | Finding median | Finding median for a set of data
  \item Secondary KC: STATISTICS AND PROBABILITY | Finding percentiles | finding percentiles for a set of data
  \item Topic: Services such as installation, maintenance, repairing, cleaning, laundry, e-commerce
  \item Grade: Secondary O-level 2
\end{itemize}

\textbf{Question}

A home appliance repair company recorded the time (in minutes) it took to complete 11 different service jobs on one day. The times were: $18$, $22$, $26$, $30$, $34$, $35$, $36$, $38$, $41$, $44$, and $50$ minutes. 

(a) Find the median service time taken.

(b) Find the 75th percentile (also known as the third quartile, $Q_3$) of the service times.

\textbf{Solution}

(a) To find the median, first list the times in ascending order (already done): 
$18, 22, 26, 30, 34, 35, 36, 38, 41, 44, 50$

There are $11$ service times, which is an odd number.

The median is the value at position $\frac{n+1}{2} = \frac{11+1}{2} = 6$.

The 6th value is $35$ minutes.

\textbf{Median service time = $35$ minutes}

(b) To find the $75$th percentile (third quartile, $Q_3$), use the formula for the position:

Position $= \frac{75}{100} \times (n+1) = 0.75 \times 12 = 9$

So, $Q_3$ is at the 9th data value, which is $41$ minutes.

\textbf{The $75$th percentile of the service times is $41$ minutes}.

\section*{Question 179}
\textbf{Metadata}

\begin{itemize}
  \item Question ID: O2-SPFndmdn\_O3-SPFndrng\_GPT4.1\_Services\_02
  \item Primary KC: STATISTICS AND PROBABILITY | Finding median | Finding median for a set of data
  \item Secondary KC: STATISTICS AND PROBABILITY | Finding range | finding range as measures of spread for a set of data 
  \item Topic: Services such as installation, maintenance, repairing, cleaning, laundry, e-commerce
  \item Grade: Secondary O-level 2
\end{itemize}

\textbf{Question}

A cleaning company recorded the number of homes cleaned by 7 of its workers in a week: 14, 10, 16, 13, 18, 11, and 15.

(a) What is the median number of homes cleaned by the workers?

(b) What is the range of the number of homes cleaned?

\textbf{Solution}

(a) First, arrange the data in ascending order: $10, 11, 13, 14, 15, 16, 18$.

Since there are 7 workers (an odd number), the median is the 4th value:

Median $= 14$

(b) The range is the difference between the maximum and minimum values:

Range $= 18 - 10 = 8$

So, the median is $14$ homes, and the range is $8$ homes.

\section*{Question 180}
\textbf{Metadata}

\begin{itemize}
  \item Question ID: O2-SPFndmean\_O3-BPRepSN\_GPT4.1\_Services\_02
  \item Primary KC: STATISTICS AND PROBABILITY | Finding mean deviation | calculation of the mean for a set of data
  \item Secondary KC: BASE AND POWER | Representation and concept  | use of standard form Ax10^n , where n is an integer, and 1<= A<= 10
  \item Topic: Services such as installation, maintenance, repairing, cleaning, laundry, e-commerce
  \item Grade: Secondary O-level 2
\end{itemize}

\textbf{Question}

A cleaning company receives five requests for laundry services in a week. The weights of the laundry collected (in kilograms) are as follows: $3.2$, $4.5$, $2.8$, $5.1$, and $4.4$ kg. 

(a) Calculate the mean weight of the laundry collected, giving your answer in standard form $A \times 10^n$ where $1 \leq A < 10$ and $n$ is an integer.

(b) Find the mean deviation of the weights from the mean value (rounded to two decimal places).

\textbf{Solution}

(a) The mean weight is calculated as follows: 

\[ \text{Mean} = \frac{3.2 + 4.5 + 2.8 + 5.1 + 4.4}{5} = \frac{20.0}{5} = 4.0 \text{ kg} \]

In standard form: 

\[ 4.0 = 4.0 \times 10^0 \]

(b) To find the mean deviation:

First, find the absolute deviations from the mean:

\[
|3.2 - 4.0| = 0.8 \\
|4.5 - 4.0| = 0.5 \\
|2.8 - 4.0| = 1.2 \\
|5.1 - 4.0| = 1.1 \\
|4.4 - 4.0| = 0.4 
\]

Sum of the absolute deviations:

\[0.8 + 0.5 + 1.2 + 1.1 + 0.4 = 4.0\]

Mean deviation:

\[ \frac{4.0}{5} = 0.8 \]

So, the mean deviation is $0.80$ kg (rounded to two decimal places).

\section*{Question 181}
\textbf{Metadata}

\begin{itemize}
  \item Question ID: O3-BPRepPosI\_O3-BPRepNegI\_GPT4.1\_Services\_02
  \item Primary KC: BASE AND POWER | Representation and concept  | positive indices that is not 1
  \item Secondary KC: BASE AND POWER | Representation and concept  | negative indices
  \item Topic: Services such as installation, maintenance, repairing, cleaning, laundry, e-commerce
  \item Grade: Secondary O-level 3/4
\end{itemize}

\textbf{Question}

A company offers professional air-conditioning cleaning services. The number of bacteria remaining in an air-conditioner filter after $n$ rounds of cleaning can be modelled by $B = 800 	imes 4^{-n}$, where $B$ is the number of bacteria. If each cleaning costs \textdollar15,

(a) Find the number of bacteria remaining after 2 rounds of cleaning, expressing your answer using positive indices.

(b) After how many rounds of cleaning will the number of bacteria first fall below $13$?

(c) Suppose the bacteria count is accidentally expressed as $800 \times 4^{n}$ instead of $800 \times 4^{-n}$ for $n=3$. What is the correct way to write $800 \times 4^{-3}$ as a power with a positive index?

\textbf{Solution}

(a) Using $n=2$, $B = 800 \times 4^{-2} = 800 \times \dfrac{1}{4^2} = 800 \times \dfrac{1}{16} = 50$. So, there are $50$ bacteria remaining after $2$ rounds of cleaning. $4^{-2}$ can be expressed as $\dfrac{1}{4^2}$ or $4^{-2}$ (with $-2$ being a negative index).

(b) We want $800 \times 4^{-n} < 13$.
$4^{-n} < \dfrac{13}{800}$
Take logarithms (or trial with powers):
When $n = 3$, $B = 800 \times 4^{-3} = 800 \times \dfrac{1}{64} = 12.5$.
So, after $3$ rounds, the bacteria drops below $13$.

(c) $800 \times 4^{-3}$ (negative index) can be written as $800 \times \dfrac{1}{4^3}$, or $\dfrac{800}{4^3}$, where the exponent is positive in the denominator.

\section*{Question 182}
\textbf{Metadata}

\begin{itemize}
  \item Question ID: O3-BPRepPosI\_O3-BPRepFrI\_GPT4.1\_Services\_02
  \item Primary KC: BASE AND POWER | Representation and concept  | positive indices that is not 1
  \item Secondary KC: BASE AND POWER | Representation and concept  | fractional indices
  \item Topic: Services such as installation, maintenance, repairing, cleaning, laundry, e-commerce
  \item Grade: Secondary O-level 3/4
\end{itemize}

\textbf{Question}

A cleaning service company charges a base cleaning fee that increases by a factor of $2^n$ for every $n$ rooms cleaned in a house, where $n>1$ and $n$ is a positive integer. If a house has $4$ rooms, the total base cleaning fee is $2^4=16\textdollar$. After cleaning, the service also applies a special treatment to every room, costing $3$ times the square root of the total base fee per room. What is the total amount that the customer must pay for cleaning and special treatment of all $4$ rooms?

\textbf{Solution}

First, the cleaning fee for $4$ rooms is given as $2^4=16\textdollar$. 

The special treatment per room costs $3 \times \sqrt{16\textdollar}$. Since $\sqrt{16} = 16^{1/2} = 4$, the cost per room is $3 \times 4 = 12\textdollar$. 

The total special treatment for $4$ rooms is $12\textdollar \times 4 = 48\textdollar$. 

So, the total amount the customer must pay is:

$16\textdollar$ (cleaning) $+ 48\textdollar$ (special treatment) $= 64\textdollar$.

\section*{Question 183}
\textbf{Metadata}

\begin{itemize}
  \item Question ID: O3-BPRepNegI\_O3-BPRepPosI\_GPT4.1\_Services\_02
  \item Primary KC: BASE AND POWER | Representation and concept  | negative indices
  \item Secondary KC: BASE AND POWER | Representation and concept  | positive indices that is not 1
  \item Topic: Services such as installation, maintenance, repairing, cleaning, laundry, e-commerce
  \item Grade: Secondary O-level 3/4
\end{itemize}

\textbf{Question}

A company provides air-conditioner maintenance services. The number of air-conditioners they are able to service each week can be represented by $2^4$. However, due to a new efficiency software, each technician can now operate at $2^{-1}$ times the original servicing time per unit. 

Calculate:

(a) The total number of air-conditioners the company originally services in a week.

(b) Using the new efficiency (with the negative index), express the new rate as a power of $2$ and find out how many times more air-conditioners each technician can now service compared to before.

(Calculate each part and show your steps clearly.)

\textbf{Solution}

(a) The original number of air-conditioners serviced each week is $2^4 = 16$.

(b) The new efficiency is $2^{-1}$ times the original servicing time per unit, which means each technician can complete a servicing in half the time it took previously. In terms of servicing rate, this means each technician can now handle twice as many air-conditioners as before because $2^{-1}$ is the same as multiplying by $\frac{1}{2}$, so servicing time per unit halves, and rate doubles.

Expressed as a power of $2$, doubling the servicing rate corresponds to $2^1$ (since $2^{0} = 1$ and $2^{1} = 2$), so each technician can now service $2^1$ times as many air-conditioners as before.

Therefore, each technician can now service $2^1 = 2$ times more air-conditioners compared to before.

\section*{Question 184}
\textbf{Metadata}

\begin{itemize}
  \item Question ID: O3-BPRepNegI\_O3-BPRepFrI\_GPT4.1\_Services\_02
  \item Primary KC: BASE AND POWER | Representation and concept  | negative indices
  \item Secondary KC: BASE AND POWER | Representation and concept  | fractional indices
  \item Topic: Services such as installation, maintenance, repairing, cleaning, laundry, e-commerce
  \item Grade: Secondary O-level 3/4
\end{itemize}

\textbf{Question}

A laundry shop uses a special detergent, and the effectiveness of the detergent per wash is modelled by the function $E(n) = 8 \times 2^{-n/2}$, where $E(n)$ is the effectiveness (in points) after $n$ washes. 

(a) What is the effectiveness of the detergent after 4 washes? 

(b) The shop wants to know after how many washes the effectiveness will be exactly $1$ point. Find the value of $n$.

\textbf{Solution}

(a)
After 4 washes, substitute $n = 4$ into the formula:

$E(4) = 8 \times 2^{-4/2}$
$= 8 \times 2^{-2}$
$= 8 \times \frac{1}{2^2}$
$= 8 \times \frac{1}{4}$
$= 2$

So, the effectiveness after 4 washes is $2$ points.

(b)
We want to solve for $n$ when $E(n) = 1$:

$1 = 8 \times 2^{-n/2}$

Divide both sides by $8$:
$\frac{1}{8} = 2^{-n/2}$

Recall that $\frac{1}{8} = 2^{-3}$. So:
$2^{-3} = 2^{-n/2}$

Equate the exponents:
$-3 = -\frac{n}{2}$

Multiply both sides by $-1$:
$3 = \frac{n}{2}$

Multiply both sides by $2$:
$n = 6$

So, the effectiveness will reach $1$ point after $6$ washes.

\section*{Question 185}
\textbf{Metadata}

\begin{itemize}
  \item Question ID: O3-BPRepFrI\_O3-BPRepPosI\_GPT4.1\_Services\_02
  \item Primary KC: BASE AND POWER | Representation and concept  | fractional indices
  \item Secondary KC: BASE AND POWER | Representation and concept  | positive indices that is not 1
  \item Topic: Services such as installation, maintenance, repairing, cleaning, laundry, e-commerce
  \item Grade: Secondary O-level 3/4
\end{itemize}

\textbf{Question}

A company offers a cleaning service where the initial effectiveness of a new cleaning solution is represented by $E_0 = 100$ (in percentage). However, due to repeated daily use, the effectiveness decreases each day. The effectiveness after $n$ days can be represented by $E_n = E_0 \times \left(\frac{3}{4}\right)^{\frac{n}{2}}$. 

(a) Calculate the effectiveness of the cleaning solution after 4 days.

(b) By what percentage has the cleaning effectiveness decreased from the initial value after 4 days?

\textbf{Solution}

Given effectiveness after $n$ days: 

$$ E_n = 100 \times \left(\frac{3}{4}\right)^{\frac{n}{2}} $$

(a) For $n = 4$:

$$ E_4 = 100 \times \left(\frac{3}{4}\right)^{\frac{4}{2}} = 100 \times \left(\frac{3}{4}\right)^2 $$

Calculate $\left(\frac{3}{4}\right)^2$:

$$ \left(\frac{3}{4}\right)^2 = \frac{9}{16} $$

So,

$$ E_4 = 100 \times \frac{9}{16} = 56.25 $$

Thus, the effectiveness after 4 days is $56.25\%$.

(b) The decrease from the initial is:

$$ 100\% - 56.25\% = 43.75\% $$

So, the effectiveness has decreased by $43.75\%$ after 4 days.

\section*{Question 186}
\textbf{Metadata}

\begin{itemize}
  \item Question ID: O3-BPRepFrI\_O3-BPRepNegI\_GPT4.1\_Services\_02
  \item Primary KC: BASE AND POWER | Representation and concept  | fractional indices
  \item Secondary KC: BASE AND POWER | Representation and concept  | negative indices
  \item Topic: Services such as installation, maintenance, repairing, cleaning, laundry, e-commerce
  \item Grade: Secondary O-level 3/4
\end{itemize}

\textbf{Question}

A company provides air-conditioner cleaning services and charges a fee that changes each year according to a special formula. In the first year, the fee for cleaning one air-conditioner is $80\textdollar$. According to the company policy, the fee for each subsequent year is calculated as $80\textdollar \times \left(\frac{3}{4}\right)^{-n/2}$, where $n$ is the number of years after the first year (i.e., $n = 1$ for the second year, $n = 2$ for the third year, and so on).

(a) Express the fee for the fourth year as a single power using fractional and negative indices.

(b) Calculate the fee for the fourth year, giving your answer correct to the nearest cent.

\textbf{Solution}

(a) For the fourth year, $n = 3$ (since $n$ is the number of years after the first year).

The formula for the fee in the fourth year is:

$80\textdollar \times \left(\frac{3}{4}\right)^{-3/2}$

This is already expressed as a single power with a negative fractional index.

(b) Now calculate $\left(\frac{3}{4}\right)^{-3/2}$:

Recall that $a^{-m} = \frac{1}{a^m}$ and $a^{1/2} = \sqrt{a}$.
So,

$\left(\frac{3}{4}\right)^{-3/2} = \frac{1}{\left(\frac{3}{4}\right)^{3/2}}$

First, evaluate $\left(\frac{3}{4}\right)^{3/2}$:

$\left(\frac{3}{4}\right)^{3/2} = \left[\left(\frac{3}{4}\right)^{1/2}\right]^3 = (\sqrt{\frac{3}{4}})^3 = \left(\frac{\sqrt{3}}{2}\right)^3 = \frac{(\sqrt{3})^3}{8} = \frac{3\sqrt{3}}{8}$

So, $\left(\frac{3}{4}\right)^{-3/2} = \frac{1}{\frac{3\sqrt{3}}{8}} = \frac{8}{3\sqrt{3}}$

Now,

Fee for the fourth year $= 80\textdollar \times \frac{8}{3\sqrt{3}}$

$= 80 \times \frac{8}{3\sqrt{3}}\textdollar$

$= \frac{640}{3\sqrt{3}}\textdollar$

$= \frac{640}{3 \times 1.73205}\textdollar$

$= \frac{640}{5.19615}\textdollar$

$\approx 123.13\textdollar$

So, the fee for the fourth year is $123.13\textdollar$ (to the nearest cent).

\section*{Question 187}
\textbf{Metadata}

\begin{itemize}
  \item Question ID: O3-BPOpr\_O3-BPRepPosI\_GPT4.1\_Services\_02
  \item Primary KC: BASE AND POWER | Operations | laws of indices
  \item Secondary KC: BASE AND POWER | Representation and concept  | positive indices that is not 1
  \item Topic: Services such as installation, maintenance, repairing, cleaning, laundry, e-commerce
  \item Grade: Secondary O-level 3/4
\end{itemize}

\textbf{Question}

A company offers a special cleaning service where the amount of cleaning solution required per job increases exponentially with the number of rooms to be cleaned. For the first room, they use $2^2$ litres of solution. For each additional room, the amount required is multiplied by $2$ (i.e., the amount for the $n$-th room is $2^n$ litres). 

(a) How much solution, in total, will the company use to clean $3$ rooms?

(b) If the cost of cleaning is \textdollar5 per litre of solution, what is the total cost to clean these $3$ rooms?

Give all your answers in index notation where appropriate.

\textbf{Solution}

(a) The amount used for each room:
- Room 1: $2^2$ litres
- Room 2: $2^3$ litres
- Room 3: $2^4$ litres
Total amount used: $2^2 + 2^3 + 2^4 = 4 + 8 + 16 = 28$ litres

(b) Total cost: $28 \times 5 = 140$

Thus, the total cost is \textdollar140.

\section*{Question 188}
\textbf{Metadata}

\begin{itemize}
  \item Question ID: O3-BPOpr\_O3-BPRepNegI\_GPT4.1\_Services\_02
  \item Primary KC: BASE AND POWER | Operations | laws of indices
  \item Secondary KC: BASE AND POWER | Representation and concept  | negative indices
  \item Topic: Services such as installation, maintenance, repairing, cleaning, laundry, e-commerce
  \item Grade: Secondary O-level 3/4
\end{itemize}

\textbf{Question}

A company offers an air-conditioner maintenance service where the charge for each year after installation decreases by a specific pattern. In the first year after installation, the customer pays $\textdollar 80$. Each following year, the amount paid is \( \frac{1}{2} \) times the amount paid in the previous year.

(a) Express the payment for the $n$th year after installation as a power with a base and index, using the laws of indices.

(b) Given that the amount paid in the 5th year after installation is \( \textdollar x \), write down an expression for the amount the customer will pay in the $n$th year after installation, using a negative index. What is the amount paid in the 8th year after installation?

\textbf{Solution}

(a) The payment decreases by half each year. The pattern of annual payment forms a geometric sequence:

Year 1: \( 80 \)
Year 2: \( 80 \times \left(\frac{1}{2}\right) \)
Year 3: \( 80 \times \left(\frac{1}{2}\right)^2 \)
...
Year $n$: \( 80 \times \left(\frac{1}{2}\right)^{n-1} \)

So, the payment for the $n$th year after installation is \( 80 \times \left(\frac{1}{2}\right)^{n-1} \).

(b) The amount paid in the 5th year is \( 80 \times \left(\frac{1}{2}\right)^{5-1} = 80 \times \left(\frac{1}{2}\right)^4 = 80 \times \frac{1}{16} = 5 \), so \( x = 5 \).

Using the law of negative indices: \( \left(\frac{1}{2}\right)^{n-1} = 2^{-(n-1)} \).

Thus, the general expression becomes:

\[
80 \times 2^{-(n-1)}
\]

For the 8th year:
\[
\text{Amount paid} = 80 \times 2^{-(8-1)} = 80 \times 2^{-7} = 80 \times \frac{1}{128} = 0.625
\]

So, the customer will pay \textdollar 0.625 in the 8th year after installation.

\section*{Question 189}
\textbf{Metadata}

\begin{itemize}
  \item Question ID: O3-BPOpr\_O3-BPRepFrI\_GPT4.1\_Services\_02
  \item Primary KC: BASE AND POWER | Operations | laws of indices
  \item Secondary KC: BASE AND POWER | Representation and concept  | fractional indices
  \item Topic: Services such as installation, maintenance, repairing, cleaning, laundry, e-commerce
  \item Grade: Secondary O-level 3/4
\end{itemize}

\textbf{Question}

A cleaning company uses a special machine that needs to be serviced regularly. The instruction manual states that the number of hours, $H$, it can run before requiring maintenance after $n$ complete cleaning cycles is given by $H = 512^{1 - \frac{n}{6}}$. 

(a) Express $H$ in terms of a power of $2$, using the laws of indices and fractional indices.

(b) If the machine has completed $3$ cleaning cycles, how many hours can it run before it needs maintenance?

\textbf{Solution}

(a) Since $512 = 2^9$, we have
\[
H = 512^{1 - \frac{n}{6}} = (2^9)^{1 - \frac{n}{6}} = 2^{9(1 - \frac{n}{6})}.
\]

(b) Substituting $n = 3$:
\[
H = 2^{9(1 - \frac{3}{6})} = 2^{9(1 - 0.5)} = 2^{9 \times 0.5} = 2^{4.5}.
\]
Now,
\[
2^{4.5} = 2^{4} \times 2^{0.5} = 16 \times \sqrt{2} \approx 16 \times 1.4142 = 22.6272.
\]
So, the machine can run approximately $22.6$ hours before needing maintenance after $3$ cleaning cycles.

\section*{Question 190}
\textbf{Metadata}

\begin{itemize}
  \item Question ID: O3-STOprUn\_O3-STOprIns\_GPT4.1\_Services\_02
  \item Primary KC: SET | Set operations | union of two sets
  \item Secondary KC: SET | Set operations | intersection of two sets
  \item Topic: Services such as installation, maintenance, repairing, cleaning, laundry, e-commerce
  \item Grade: Secondary O-level 3/4
\end{itemize}

\textbf{Question}

A cleaning company keeps a record of its clients. Let $A$ be the set of clients who requested carpet cleaning this month, and $B$ be the set of clients who requested sofa cleaning this month. There are $50$ clients in set $A$, $40$ clients in set $B$, and $20$ clients requested both carpet and sofa cleaning.

(a) How many clients requested either carpet cleaning or sofa cleaning this month?

(b) If the company had $80$ clients this month in total, how many did not request any of these two services?

\textbf{Solution}

(a) The number of clients who requested either carpet cleaning or sofa cleaning is given by the union of sets $A$ and $B$:

$ n(A \cup B) = n(A) + n(B) - n(A \cap B) $

$ n(A \cup B) = 50 + 40 - 20 = 70 $

Hence, $70$ clients requested either carpet or sofa cleaning this month.

(b) The total number of clients is $80$. The number who did not request these services is:

$80 - 70 = 10$

So, $10$ clients did not request either of the two services.

\section*{Question 191}
\textbf{Metadata}

\begin{itemize}
  \item Question ID: O3-MXMulSM\_O3-MXAdd\_GPT4.1\_Services\_02
  \item Primary KC: MATRICES | Multiplication | product of a scalar quantity and a matrix
  \item Secondary KC: MATRICES | Addition | addition of matrices
  \item Topic: Services such as installation, maintenance, repairing, cleaning, laundry, e-commerce
  \item Grade: Secondary O-level 3/4
\end{itemize}

\textbf{Question}

An electronics company maintains records of installation and maintenance costs for two types of air conditioners (Type A and Type B) over two months (June and July) in the matrix 
\[
M = \begin{bmatrix} 300 & 250 \\ 280 & 220 \end{bmatrix}
\]
where the first row represents installation costs (in \textdollar) for June and July, and the second row represents maintenance costs (in \textdollar) for the same months. 

Due to increased demand in July, the company decided to offer a 20% discount on all maintenance costs for that month. Represent this change using a scalar multiplication on the corresponding entry in the matrix. 

Later, an additional fixed service fee matrix 
\[
F = \begin{bmatrix} 50 & 50 \\ 30 & 30 \end{bmatrix}
\]
was added to every category and month.

What is the final cost matrix after applying the discount and adding the fixed fee matrix?

\textbf{Solution}

First, apply the 20% discount to the maintenance cost for July (row 2, column 2 of $M$):

$280$ (June) and $220$ (July). Maintenance for July after 20% discount:

$220 \times 0.8 = 176$

So, the updated matrix $M'$ after the discount is:
\[
M' = \begin{bmatrix} 300 & 250 \\ 280 & 176 \end{bmatrix}
\]

Now, add the fixed fee matrix $F$:
\[
\begin{bmatrix} 300 & 250 \\ 280 & 176 \end{bmatrix}
+
\begin{bmatrix} 50 & 50 \\ 30 & 30 \end{bmatrix}
=
\begin{bmatrix} 350 & 300 \\ 310 & 206 \end{bmatrix}
\]

Therefore, the final cost matrix is:
\[
\begin{bmatrix} 350 & 300 \\ 310 & 206 \end{bmatrix}
\]

where each entry represents the final amount (in \textdollar) to be charged for installation and maintenance of Type A and B air conditioners for June and July, after all changes.

\section*{Question 192}
\textbf{Metadata}

\begin{itemize}
  \item Question ID: O3-MXMulSM\_O3-MXSub\_GPT4.1\_Services\_02
  \item Primary KC: MATRICES | Multiplication | product of a scalar quantity and a matrix
  \item Secondary KC: MATRICES | Subtraction | subtraction of matrices
  \item Topic: Services such as installation, maintenance, repairing, cleaning, laundry, e-commerce
  \item Grade: Secondary O-level 3/4
\end{itemize}

\textbf{Question}

A cleaning service company tracks the number of small, medium, and large rooms cleaned each day. On Monday, the company cleaned the following numbers of rooms, represented by Matrix $A$:

$$
A = \begin{bmatrix}
4 \\
6 \\
3
\end{bmatrix}
$$

On Tuesday, the company cleaned fewer rooms, represented as Matrix $B$:

$$
B = \begin{bmatrix}
2 \\
4 \\
2
\end{bmatrix}
$$

(a) The manager gives a bonus multiplier of $3$ for Monday’s results to project potential future performance. Calculate the new projected Monday results as a matrix.

(b) The manager wants to know the difference in the projected Monday results and the actual Tuesday results. Find the matrix that represents this difference.

\textbf{Solution}

(a) To find the projected Monday results, multiply matrix $A$ by scalar 3:

$$
3 \times A = 3 \times \begin{bmatrix} 4 \\ 6 \\ 3 \end{bmatrix} = \begin{bmatrix} 12 \\ 18 \\ 9 \end{bmatrix}
$$

(b) To find the difference between the projected Monday results and actual Tuesday results, subtract matrix $B$ from $3 \times A$:

$$
\begin{bmatrix} 12 \\ 18 \\ 9 \end{bmatrix} - \begin{bmatrix} 2 \\ 4 \\ 2 \end{bmatrix} = \begin{bmatrix} 12-2 \\ 18-4 \\ 9-2 \end{bmatrix} = \begin{bmatrix} 10 \\ 14 \\ 7 \end{bmatrix}
$$

So, the matrix representing the difference is:

$$
\begin{bmatrix} 10 \\ 14 \\ 7 \end{bmatrix}
$$

\section*{Question 193}
\textbf{Metadata}

\begin{itemize}
  \item Question ID: O3-MXMulSM\_O3-MXMul\_GPT4.1\_Services\_02
  \item Primary KC: MATRICES | Multiplication | product of a scalar quantity and a matrix
  \item Secondary KC: MATRICES | Multiplication | multiplication of matrices
  \item Topic: Services such as installation, maintenance, repairing, cleaning, laundry, e-commerce
  \item Grade: Secondary O-level 3/4
\end{itemize}

\textbf{Question}

A cleaning company manages different service packages for office cleaning. For a certain client, the company charges different rates based on the size of the office. The matrix below shows the number of times each type of cleaning service (basic, deep, disinfection) is performed in a month for two different office types (small and large):

\[
A = \begin{bmatrix}
5 & 2 \\
8 & 3 \\
1 & 2
\end{bmatrix}
\]

where each row represents a service (row 1: basic, row 2: deep, row 3: disinfection) and each column represents an office type (column 1: small, column 2: large).

The company increases the price of all services by a factor of $3$. Let $B = 3A$ represent the updated monthly schedule after the rate change.

Also, the company assigns the following charges (in \textdollar) per service type: basic: 10\textdollar, deep: 25\textdollar, disinfection: 40\textdollar. Express this as a column matrix: 
\[
C = \begin{bmatrix}
10 \\
25 \\
40
\end{bmatrix}
\]

(a) Find the new monthly schedule matrix $B$. 

(b) Find the total revenue (in \textdollar) from each office type based on the new schedule by multiplying $B$ with $C$.


\textbf{Solution}

(a) To find $B$, multiply every entry in $A$ by $3$:

\[
B = 3A = 3 \times \begin{bmatrix}
5 & 2 \\
8 & 3 \\
1 & 2
\end{bmatrix}
= \begin{bmatrix}
15 & 6 \\
24 & 9 \\
3 & 6
\end{bmatrix}
\]

(b) For each office type (column), we calculate the total revenue by multiplying $B$ by $C$.

Let office 1 be 'small' and office 2 be 'large'.

Revenue for small office:
\[
(15 \times 10) + (24 \times 25) + (3 \times 40) = 150 + 600 + 120 = 870\textdollar
\]

Revenue for large office:
\[
(6 \times 10) + (9 \times 25) + (6 \times 40) = 60 + 225 + 240 = 525\textdollar
\]

Therefore, the total revenue (in matrix form) is:
\[
\begin{bmatrix}
870 \\
525
\end{bmatrix}
\]

This means, after the changes, the company earns 870\textdollar from the small office type and 525\textdollar from the large office type in a month.

\section*{Question 194}
\textbf{Metadata}

\begin{itemize}
  \item Question ID: O3-MXSub\_O3-MXAdd\_GPT4.1\_Services\_02
  \item Primary KC: MATRICES | Subtraction | subtraction of matrices
  \item Secondary KC: MATRICES | Addition | addition of matrices
  \item Topic: Services such as installation, maintenance, repairing, cleaning, laundry, e-commerce
  \item Grade: Secondary O-level 3/4
\end{itemize}

\textbf{Question}

A cleaning company charges customers for three types of services: basic cleaning, deep cleaning, and window cleaning. The total number of each service provided in January and February is represented by the following matrices:

January: $A = \begin{bmatrix} 24 & 15 & 10 \\ 18 & 12 & 8 \end{bmatrix}$

February: $B = \begin{bmatrix} 20 & 10 & 12 \\ 21 & 9 & 14 \end{bmatrix}$

In these matrices, the first row shows the services provided for residential buildings, and the second row shows the services for commercial buildings, in the order: basic cleaning, deep cleaning, and window cleaning.

(a) Find the total number of services provided (of each type and building) in January and February combined.

(b) Find the difference in the number of services provided between February and January for each service type and building.

\textbf{Solution}

(a) To find the total number of services provided for each type and building in both months combined, add matrices $A$ and $B$:

$A + B = \begin{bmatrix} 24 & 15 & 10 \\ 18 & 12 & 8 \end{bmatrix} + \begin{bmatrix} 20 & 10 & 12 \\ 21 & 9 & 14 \end{bmatrix} = \begin{bmatrix} 24+20 & 15+10 & 10+12 \\ 18+21 & 12+9 & 8+14 \end{bmatrix} = \begin{bmatrix} 44 & 25 & 22 \\ 39 & 21 & 22 \end{bmatrix}$

So, the total services (basic, deep, window) for residential and commercial buildings in both months are:
- Residential: 44 (basic), 25 (deep), 22 (window)
- Commercial: 39 (basic), 21 (deep), 22 (window)

(b) To find the difference in number of services from February to January, subtract matrix $A$ from $B$:

$B - A = \begin{bmatrix} 20 & 10 & 12 \\ 21 & 9 & 14 \end{bmatrix} - \begin{bmatrix} 24 & 15 & 10 \\ 18 & 12 & 8 \end{bmatrix} = \begin{bmatrix} 20-24 & 10-15 & 12-10 \\ 21-18 & 9-12 & 14-8 \end{bmatrix} = \begin{bmatrix} -4 & -5 & 2 \\ 3 & -3 & 6 \end{bmatrix}$

So, compared to January, in February there were:
- Residential: 4 fewer basic, 5 fewer deep, 2 more window cleanings
- Commercial: 3 more basic, 3 fewer deep, 6 more window cleanings

\section*{Question 195}
\textbf{Metadata}

\begin{itemize}
  \item Question ID: O3-MXMul\_O3-MXMulSM\_GPT4.1\_Services\_02
  \item Primary KC: MATRICES | Multiplication | multiplication of matrices
  \item Secondary KC: MATRICES | Multiplication | product of a scalar quantity and a matrix
  \item Topic: Services such as installation, maintenance, repairing, cleaning, laundry, e-commerce
  \item Grade: Secondary O-level 3/4
\end{itemize}

\textbf{Question}

A laundry company records the number of shirts, trousers, and jackets they clean at three different locations in a week. The data is recorded in the following matrix $A$ (number of items):

\[
A = \begin{bmatrix}
12 & 8 & 5 \\
10 & 6 & 7 \\
14 & 9 & 4
\end{bmatrix}
\]
where each row represents a different location, and the columns represent the number of shirts, trousers, and jackets cleaned, respectively.

The company charges \textdollar3 to clean a shirt, \textdollar4 for a trouser, and \textdollar8 for a jacket. Express the charges as a column matrix $B$:
\[
B = \begin{bmatrix} 3 \\ 4 \\ 8 \end{bmatrix}
\]

(a) Calculate the matrix product $A \times B$ to find the total cleaning charges for each location.

(b) During a promotion week, the company offers a 20% discount on all cleaning charges. Represent this as multiplying the matrix found in (a) by a scalar. What is the total amount each location collects during the promotion week?

\textbf{Solution}

First, calculate the matrix product $A \times B$:

\[
A \times B = \begin{bmatrix}
12 & 8 & 5 \\
10 & 6 & 7 \\
14 & 9 & 4
\end{bmatrix}
\times
\begin{bmatrix} 3 \\ 4 \\ 8 \end{bmatrix}
\]

For each row:

- First location: $(12 \times 3) + (8 \times 4) + (5 \times 8) = 36 + 32 + 40 = 108$
- Second location: $(10 \times 3) + (6 \times 4) + (7 \times 8) = 30 + 24 + 56 = 110$
- Third location: $(14 \times 3) + (9 \times 4) + (4 \times 8) = 42 + 36 + 32 = 110$

So,
\[
A \times B = \begin{bmatrix} 108 \\ 110 \\ 110 \end{bmatrix}
\]
Each value represents the total cleaning charges (in dollars) collected at each location.

(b) A 20% discount means the company collects 80% of the total charges. Multiply each amount by the scalar $0.8$:
\[
0.8 \times \begin{bmatrix} 108 \\ 110 \\ 110 \end{bmatrix} = \begin{bmatrix} 86.4 \\ 88 \\ 88 \end{bmatrix}
\]

Thus, during the promotion week, the total amount collected at each location is \textdollar86.40, \textdollar88.00, and \textdollar88.00, respectively.

\section*{Question 196}
\textbf{Metadata}

\begin{itemize}
  \item Question ID: O3-MXMul\_O3-MXAdd\_GPT4.1\_Services\_02
  \item Primary KC: MATRICES | Multiplication | multiplication of matrices
  \item Secondary KC: MATRICES | Addition | addition of matrices
  \item Topic: Services such as installation, maintenance, repairing, cleaning, laundry, e-commerce
  \item Grade: Secondary O-level 3/4
\end{itemize}

\textbf{Question}

A cleaning company tracks the number of cleaning jobs completed by two teams, Team A and Team B, for three different services: office cleaning, house cleaning, and carpet cleaning over two weekdays, Monday and Tuesday. The results are shown in the matrices below, where the rows represent the teams, and the columns represent the services.

On Monday:
$M = \begin{pmatrix} 3 & 5 & 2 \\ 4 & 6 & 1 \end{pmatrix}$

On Tuesday:
$N = \begin{pmatrix} 2 & 7 & 3 \\ 5 & 3 & 2 \end{pmatrix}$

(a) Calculate the total number of jobs completed for each team and each service across both days by finding $M + N$.

The company charges different prices for each service:
$P = \begin{pmatrix} 30 \\ 40 \\ 50 \end{pmatrix}$
where each entry in $P$ represents the price (in \textdollar) for office cleaning, house cleaning, and carpet cleaning, respectively.

(b) Find the revenue matrix for each team over both days by multiplying the total jobs matrix ($M + N$) with the price matrix $P$.

\textbf{Solution}

(a) To find the total number of jobs for each team and service over both days, add $M$ and $N$:

$M + N = \begin{pmatrix} 3 & 5 & 2 \\ 4 & 6 & 1 \end{pmatrix} + \begin{pmatrix} 2 & 7 & 3 \\ 5 & 3 & 2 \end{pmatrix} = \begin{pmatrix} 3+2 & 5+7 & 2+3 \\ 4+5 & 6+3 & 1+2 \end{pmatrix} = \begin{pmatrix} 5 & 12 & 5 \\ 9 & 9 & 3 \end{pmatrix}$

(b) To find the total revenue for each team, multiply the total jobs matrix by the price matrix:

$R = (M + N) \times P$

Calculation for Team A:
$[5, 12, 5] \times \begin{pmatrix} 30 \\ 40 \\ 50 \end{pmatrix} = 5 \times 30 + 12 \times 40 + 5 \times 50 = 150 + 480 + 250 = \textdollar880$

Calculation for Team B:
$[9, 9, 3] \times \begin{pmatrix} 30 \\ 40 \\ 50 \end{pmatrix} = 9 \times 30 + 9 \times 40 + 3 \times 50 = 270 + 360 + 150 = \textdollar780$

The revenue matrix is:
$\begin{pmatrix} 880 \\ 780 \end{pmatrix}$

So, Team A earned \textdollar880 and Team B earned \textdollar780 in total over the two days.

\section*{Question 197}
\textbf{Metadata}

\begin{itemize}
  \item Question ID: O3-MXMul\_O3-MXSub\_GPT4.1\_Services\_02
  \item Primary KC: MATRICES | Multiplication | multiplication of matrices
  \item Secondary KC: MATRICES | Subtraction | subtraction of matrices
  \item Topic: Services such as installation, maintenance, repairing, cleaning, laundry, e-commerce
  \item Grade: Secondary O-level 3/4
\end{itemize}

\textbf{Question}

A cleaning company keeps records of the number of cleaning jobs completed by two teams, Team A and Team B, at two different types of locations: offices and hotels. In January, the number of jobs completed is represented by the matrix:

$$
M_1 = \begin{pmatrix}12 & 15 \\ 18 & 10\end{pmatrix}
$$

In February, the number of jobs completed is represented by the matrix:

$$
M_2 = \begin{pmatrix}14 & 13 \\ 17 & 12\end{pmatrix}
$$

Each office job earns the company $\textdollar80$ and each hotel job earns $\textdollar120$. This price information is represented as:

$$
P = \begin{pmatrix}80 & 120 \\ 80 & 120\end{pmatrix}
$$

(a) Calculate the change in the number of jobs from January to February for each team and location in a matrix form.

(b) Find the total earnings for February by multiplying $M_2$ by the price matrix $P$ using the appropriate matrix multiplication.

(c) Calculate the difference in total earnings between January and February for all teams and locations.

\textbf{Solution}

(a) The change in the number of jobs for each team and location is given by subtracting January's matrix from February's matrix:

$$
\Delta M = M_2 - M_1 = \begin{pmatrix}14 & 13 \\ 17 & 12\end{pmatrix} - \begin{pmatrix}12 & 15 \\ 18 & 10\end{pmatrix} = \begin{pmatrix}14-12 & 13-15 \\ 17-18 & 12-10\end{pmatrix} = \begin{pmatrix}2 & -2 \\ -1 & 2\end{pmatrix}
$$

(b) To find the total earnings for February, we multiply $M_2$ (jobs) by the price matrix $P$ (price per job):

$$M_2 = \begin{pmatrix}14 & 13 \\ 17 & 12\end{pmatrix}$$
$$P = \begin{pmatrix}80 \\ 120\end{pmatrix}$$

(Since each job type has the same price structure for both teams, we use the column price matrix)

So,
$$
\text{Earnings for February} = M_2 \times P =
\begin{pmatrix}14 & 13 \\ 17 & 12\end{pmatrix} \times \begin{pmatrix}80 \\ 120\end{pmatrix} = \begin{pmatrix}(14 \times 80) + (13 \times 120) \\ (17 \times 80) + (12 \times 120)\end{pmatrix}
$$

Calculate each entry:

First team:
$$
(14 \times 80) + (13 \times 120) = 1120 + 1560 = 2680\textdollar
$$

Second team:
$$
(17 \times 80) + (12 \times 120) = 1360 + 1440 = 2800\textdollar
$$

So, the earnings for February for each team are:
$$
\begin{pmatrix}2680 \\ 2800\end{pmatrix}\textdollar
$$

(c) Calculate the earnings for January in the same way:

$$
M_1 = \begin{pmatrix}12 & 15 \\ 18 & 10\end{pmatrix}
$$

So,
$$
\text{Earnings for January} = M_1 \times P = \begin{pmatrix}(12 \times 80) + (15 \times 120) \\ (18 \times 80) + (10 \times 120)\end{pmatrix}
$$

Calculate each entry:

First team:
$$
(12 \times 80) + (15 \times 120) = 960 + 1800 = 2760\textdollar
$$

Second team:
$$
(18 \times 80) + (10 \times 120) = 1440 + 1200 = 2640\textdollar
$$

So, the earnings for January for each team are:
$$
\begin{pmatrix}2760 \\ 2640\end{pmatrix}\textdollar
$$

The difference in total earnings between January and February is:

First team:
$$
2680\textdollar - 2760\textdollar = -80\textdollar
$$
Second team:
$$
2800\textdollar - 2640\textdollar = 160\textdollar
$$

So the matrix of changes in earnings is:
$$
\begin{pmatrix}-80 \\ 160\end{pmatrix}\textdollar
$$

\section*{Question 198}
\textbf{Metadata}

\begin{itemize}
  \item Question ID: O3-SPFndQtl\_O3-SPFndIQR\_GPT4.1\_Services\_02
  \item Primary KC: STATISTICS AND PROBABILITY | Finding quartiles | finding quartiles for a set of data
  \item Secondary KC: STATISTICS AND PROBABILITY | Finding range | finding interquartile range as measures of spread for a set of data 
  \item Topic: Services such as installation, maintenance, repairing, cleaning, laundry, e-commerce
  \item Grade: Secondary O-level 3/4
\end{itemize}

\textbf{Question}

A laundry company records the number of shirts cleaned each day for 11 days as follows:

$45, 52, 47, 49, 60, 55, 48, 50, 53, 54, 58$

(a) Find the lower quartile ($Q_1$), median ($Q_2$), and upper quartile ($Q_3$) for this set of data.

(b) Find the interquartile range (IQR) for the number of shirts cleaned per day.

\textbf{Solution}

**Step 1: Arrange the data in ascending order:**

$45, 47, 48, 49, 50, 52, 53, 54, 55, 58, 60$

There are 11 data values.

**Step 2: Find the median ($Q_2$):**

Median position = $(n+1)/2 = (11+1)/2 = 6$

$Q_2$ is the 6th value: $52$

**Step 3: Find the lower quartile ($Q_1$):**

The lower half (excluding the median) is:

$45, 47, 48, 49, 50$ (positions 1 to 5)

$Q_1$ is the median of these five values: the 3rd value, which is $48$

**Step 4: Find the upper quartile ($Q_3$):**

The upper half (excluding the median) is:

$53, 54, 55, 58, 60$ (positions 7 to 11)

$Q_3$ is the median of these five values: the 3rd value, which is $55$

**Step 5: Interquartile Range (IQR):**

$IQR = Q_3 - Q_1 = 55 - 48 = 7$

**Final Answers:**

(a) Lower quartile $= 48$, Median $= 52$, Upper quartile $= 55$

(b) Interquartile range $= 7$

\section*{Question 199}
\textbf{Metadata}

\begin{itemize}
  \item Question ID: O3-SPFndstd\_O2-SPFndmean\_GPT4.1\_Services\_02
  \item Primary KC: STATISTICS AND PROBABILITY | Finding standard deviation | calculation of the standard deviation for a set of data
  \item Secondary KC: STATISTICS AND PROBABILITY | Finding mean deviation | calculation of the mean for a set of data
  \item Topic: Services such as installation, maintenance, repairing, cleaning, laundry, e-commerce
  \item Grade: Secondary O-level 3/4
\end{itemize}

\textbf{Question}

A laundry service recorded the weights (in kg) of clothes collected from 6 different households in one day. The weights were: $8$, $12$, $10$, $7$, $9$, $14$. 

(a) Calculate the mean weight of the clothes collected.

(b) Find the standard deviation of the weights, correct to 2 decimal places.

\textbf{Solution}

Let the weights be $w_1 = 8$, $w_2 = 12$, $w_3 = 10$, $w_4 = 7$, $w_5 = 9$, $w_6 = 14$.

(a) Mean weight $= \dfrac{8+12+10+7+9+14}{6} = \dfrac{60}{6} = 10$ kg.

(b) The standard deviation $\sigma$ is given by

$\sigma = \sqrt{\dfrac{1}{n} \sum_{i=1}^{n} (w_i - \text{mean})^2}$

Compute deviations from the mean:
$8 - 10 = -2$
$12 - 10 = 2$
$10 - 10 = 0$
$7 - 10 = -3$
$9 - 10 = -1$
$14 - 10 = 4$

Squares of deviations:
$(-2)^2 = 4$
$2^2 = 4$
$0^2 = 0$
$(-3)^2 = 9$
$(-1)^2 = 1$
$4^2 = 16$

Sum of squared deviations:
$4 + 4 + 0 + 9 + 1 + 16 = 34$

Standard deviation:
$\sigma = \sqrt{\dfrac{34}{6}} = \sqrt{5.6667} \approx 2.38 \text{ kg}$ (to 2 decimal places)

**Answers:**
(a) Mean weight $= 10$ kg
(b) Standard deviation $\approx 2.38$ kg

\section*{Question 200}
\textbf{Metadata}

\begin{itemize}
  \item Question ID: O3-SPAddProb\_O2-SPRepPrSE\_GPT4.1\_Services\_02
  \item Primary KC: STATISTICS AND PROBABILITY | Addition | addition of probabilities
  \item Secondary KC: STATISTICS AND PROBABILITY | Representation and concept | probability of single events
  \item Topic: Services such as installation, maintenance, repairing, cleaning, laundry, e-commerce
  \item Grade: Secondary O-level 3/4
\end{itemize}

\textbf{Question}

A home repair company tracks the probability that a service call in a day will be for either air-con servicing or plumbing repair. The probability of receiving an air-con servicing request on a given day is $0.25$, and the probability of receiving a plumbing repair request is $0.15$. These two events are mutually exclusive because a single service call cannot involve both services at the same time.

What is the probability that on a given day, the company receives either an air-con servicing request or a plumbing repair request?

\textbf{Solution}

Let $A$ be the event 'receiving an air-con servicing request' and $P$ be the event 'receiving a plumbing repair request'. Since these two events are mutually exclusive,

\[
P(A \text{ or } P) = P(A) + P(P)
\]

Given:
$P(A) = 0.25$
$P(P) = 0.15$

So,
\[
P(A \text{ or } P) = 0.25 + 0.15 = 0.40
\]

Therefore, the probability that the company receives either an air-con servicing request or a plumbing repair request on a given day is $0.40$.

\section*{Question 201}
\textbf{Metadata}

\begin{itemize}
  \item Question ID: O3-SPAddProb\_O3-SPFndPrCE\_GPT4.1\_Services\_02
  \item Primary KC: STATISTICS AND PROBABILITY | Addition | addition of probabilities
  \item Secondary KC: STATISTICS AND PROBABILITY | Finding probability | probability of simple combined events
  \item Topic: Services such as installation, maintenance, repairing, cleaning, laundry, e-commerce
  \item Grade: Secondary O-level 3/4
\end{itemize}

\textbf{Question}

A laundry shop offers two special services: express washing and express ironing. The probability that a randomly chosen customer requests express washing is $0.45$, while the probability that a customer requests express ironing is $0.30$. The probability that a customer requests both services at the same visit is $0.15$.

(a) Find the probability that a customer requests at least one of the two services during their visit to the laundry shop.

(b) What is the probability that a customer requests neither express washing nor express ironing?

\textbf{Solution}

(a) To find the probability that a randomly selected customer requests at least one of the two services, we use the formula for the addition of probabilities:
\[
P(A \cup B) = P(A) + P(B) - P(A \cap B)
\]
where $A$ is the event 'requests express washing', and $B$ is the event 'requests express ironing'.

Given:
$P(A) = 0.45$
$P(B) = 0.30$
$P(A \cap B) = 0.15$

Substituting the values:
\[
P(A \cup B) = 0.45 + 0.30 - 0.15 = 0.60
\]
So, the probability that a customer requests at least one of the two services is $0.60$.

(b) The probability that a customer requests neither service is the complement of (a):
\[
P(\text{neither}) = 1 - P(A \cup B) = 1 - 0.60 = 0.40
\]
So, the probability that a customer requests neither express washing nor express ironing is $0.40$.

\section*{Question 202}
\textbf{Metadata}

\begin{itemize}
  \item Question ID: O3-SPMulProb\_O2-SPRepPrSE\_GPT4.1\_Services\_02
  \item Primary KC: STATISTICS AND PROBABILITY | Multiplication | multiplication of probabilities
  \item Secondary KC: STATISTICS AND PROBABILITY | Representation and concept | probability of single events
  \item Topic: Services such as installation, maintenance, repairing, cleaning, laundry, e-commerce
  \item Grade: Secondary O-level 3/4
\end{itemize}

\textbf{Question}

A cleaning company sends two workers, Roy and Mei, to clean houses. The probability that Roy will finish cleaning a house on time is $0.8$, while the probability that Mei will finish cleaning on time is $0.9$.

Assuming the two workers work independently, what is the probability that both Roy and Mei will finish cleaning on time for the same house?

Express your answer as a decimal.

\textbf{Solution}

Let the probability that Roy finishes on time be $P(R) = 0.8$ and Mei finishes on time be $P(M) = 0.9$.

Since they work independently,

$P(\text{both finish on time}) = P(R) \times P(M) = 0.8 \times 0.9 = 0.72$

So, the probability that both Roy and Mei will finish cleaning on time is $0.72$.

\section*{Question 203}
\textbf{Metadata}

\begin{itemize}
  \item Question ID: O3-SPMulProb\_O3-SPFndPrCE\_GPT4.1\_Services\_02
  \item Primary KC: STATISTICS AND PROBABILITY | Multiplication | multiplication of probabilities
  \item Secondary KC: STATISTICS AND PROBABILITY | Finding probability | probability of simple combined events
  \item Topic: Services such as installation, maintenance, repairing, cleaning, laundry, e-commerce
  \item Grade: Secondary O-level 3/4
\end{itemize}

\textbf{Question}

A home cleaning service offers two types of additional services: carpet shampooing and window cleaning. Each service can be added independent of the other when a customer books a regular cleaning session. On a particular day, the probability that a customer chooses carpet shampooing is $0.4$, and the probability that a customer chooses window cleaning is $0.3$.

(a) What is the probability that a randomly selected customer chooses both carpet shampooing and window cleaning for their cleaning session?

(b) What is the probability that a customer chooses at least one of these two additional services?


\textbf{Solution}

(a) Since the two events are independent, the probability that a customer chooses both carpet shampooing and window cleaning is:
\[
P(\text{Carpet and Window}) = P(\text{Carpet}) \times P(\text{Window}) = 0.4 \times 0.3 = 0.12
\]
So, the probability is $0.12$.

(b) The probability that a customer chooses at least one of the two services is given by:
\[
P(\text{at least one}) = P(\text{Carpet}) + P(\text{Window}) - P(\text{Carpet and Window})
= 0.4 + 0.3 - 0.12 = 0.58
\]
Thus, the probability that a customer chooses at least one additional service is $0.58$.

\end{document}
