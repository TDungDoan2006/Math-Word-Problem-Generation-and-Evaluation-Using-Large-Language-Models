\documentclass{article}
\usepackage[utf8]{inputenc}
\usepackage{amsmath}
\usepackage{amsfonts}
\usepackage{amssymb}
\usepackage{graphicx}
\usepackage{hyperref}
\title{20 Services MWPs by GPT4.1}
\author{Tien Dung Doan}
\begin{document}
\maketitle
\section*{Question 1}
\textbf{Metadata}

\begin{itemize}
  \item Question ID: P5-DcMul3dK\_P4-DcCmp3d\_GPT4.1\_Services\_02
  \item Primary KC: DECIMALS | Multiplication | multiplying decimals (up to 3 decimal places) by 10, 100, 1000 and their multiples
  \item Secondary KC: DECIMALS | Comparison and ordering | comparing and ordering decimals up to 3 decimal places
  \item Topic: Services such as installation, maintenance, repairing, cleaning, laundry, e-commerce
  \item Grade: Primary 5
\end{itemize}

\textbf{Solution}

First, calculate the cost for each bag by multiplying the weight (in kg) by the price per kg.

For the first bag:

Cost$_1 = 4.2 \times 3.285 $

$ = 13.797 $

Rounding to the nearest cent: $13.80$

For the second bag:

Cost$_2 = 3.45 \times 3.285 $

$ = 11.34825 $

Rounding to the nearest cent: $11.35$

For the third bag:

Cost$_3 = 5.12 \times 3.285 $

$ = 16.8192 $

Rounding to the nearest cent: $16.82$

Now, compare the costs:

- $13.80$ (first bag)
- $11.35$ (second bag)
- $16.82$ (third bag)

Ordering from least to greatest: $11.35 < 13.80 < 16.82$

The third bag costs the most. The difference between the most and least expensive is:

$16.82 - 11.35 = 5.47$

So, the third bag will cost Mrs Tan the most, and it is $5.47$ more than the least expensive bag.

\section*{Question 2}
\textbf{Metadata}

\begin{itemize}
  \item Question ID: P3-FrSubRl12\_P2-FrAdd2nd\_GPT4.1\_Services\_03
  \item Primary KC: FRACTIONS | Subtraction | subtracting two related fractions within one whole with denominators of given fractions not exceeding 12
  \item Secondary KC: FRACTIONS | Addition | adding fractions
  \item Topic: Services such as installation, maintenance, repairing, cleaning, laundry, e-commerce
  \item Grade: Primary 3
\end{itemize}

\textbf{Solution}

First, find out how much longer Siti spent washing than folding:

$\frac{5}{8} - \frac{1}{8} = \frac{4}{8} = \frac{1}{2}$ (hour)

So, Siti spent $\frac{1}{2}$ hour longer washing than folding.

Next, add all the time spent on laundry:

Washing: $\frac{5}{8}$ hour  
Folding: $\frac{1}{8}$ hour  
Ironing: $\frac{1}{4}$ hour

First, add folding and ironing. Convert $\frac{1}{4}$ to eighths:  
$\frac{1}{4} = \frac{2}{8}$

So, $\frac{1}{8} + \frac{2}{8} = \frac{3}{8}$

Now add washing:
$\frac{5}{8} + \frac{3}{8} = \frac{8}{8} = 1$

Siti spent a total of 1 hour on all the laundry tasks.

\section*{Question 3}
\textbf{Metadata}

\begin{itemize}
  \item Question ID: O2-RoRepDP\_P1-WNDiv2nd\_GPT4.1\_Services\_02
  \item Primary KC: RATIO | Representation and concept | direct proportion
  \item Secondary KC: WHOLE NUMBERS | Division | dividing whole numbers
  \item Topic: Services such as installation, maintenance, repairing, cleaning, laundry, e-commerce
  \item Grade: Secondary O-level 2
\end{itemize}

\textbf{Solution}

(a) The ratio of cleaning solution to rooms cleaned is $2:5$. That means 2 litres for every 5 rooms.

For 20 rooms:

Number of sets of 5 rooms in 20 rooms: $\frac{20}{5} = 4$

So, for 4 sets:

Number of litres needed $= 4 \times 2 = 8$

The company used 8 litres of cleaning solution for 20 rooms.

(b) For 35 rooms:

Number of sets of 5 rooms in 35 rooms: $\frac{35}{5} = 7$

Number of litres needed $= 7 \times 2 = 14$

The company should prepare 14 litres of cleaning solution for 35 rooms.

Now, if each bottle contains $7$ litres:

Number of full bottles needed $= \frac{14}{7} = 2$

They will need 2 full bottles of cleaning solution.

\section*{Question 4}
\textbf{Metadata}

\begin{itemize}
  \item Question ID: P5-DcMul3dK\_P4-DcAdd2nd\_GPT4.1\_Services\_02
  \item Primary KC: DECIMALS | Multiplication | multiplying decimals (up to 3 decimal places) by 10, 100, 1000 and their multiples
  \item Secondary KC: DECIMALS | Addition | adding decimals
  \item Topic: Services such as installation, maintenance, repairing, cleaning, laundry, e-commerce
  \item Grade: Primary 5
\end{itemize}

\textbf{Solution}

First, calculate the cleaning cost: 

Cost of cleaning $= 3.75 \times 12.6 = 47.25$

Add the additional stain removal service fee:

Total cost $= 47.25 + 7.50 = 54.75$

So, Mr. Lim charged $\textdollar54.75$ in total.

\section*{Question 5}
\textbf{Metadata}

\begin{itemize}
  \item Question ID: P5-PcRepWh\_P1-WNSub2nd\_GPT4.1\_Services\_02
  \item Primary KC: PERCENTAGE | Representation and concept | expressing a part of a whole as a percentage
  \item Secondary KC: WHOLE NUMBERS | Subtraction | subtracting whole numbers
  \item Topic: Services such as installation, maintenance, repairing, cleaning, laundry, e-commerce
  \item Grade: Primary 5
\end{itemize}

\textbf{Solution}

First, calculate how many uniforms had stains after the wash:

$20\%$ of $30$ uniforms $= \frac{20}{100} \times 30 = 6$ uniforms.

This means $6$ uniforms had stains and needed to be washed again.

To find out how many uniforms did not need to be washed again, subtract the number of stained uniforms from the total:

$30 - 6 = 24$ uniforms.

\textbf{Answer:} $24$ uniforms did not need to be washed again after the first wash.

\section*{Question 6}
\textbf{Metadata}

\begin{itemize}
  \item Question ID: P5-FrSubMix\_P2-FrAdd2nd\_GPT4.1\_Services\_02
  \item Primary KC: FRACTIONS | Subtraction | subtracting mixed numbers
  \item Secondary KC: FRACTIONS | Addition | adding fractions
  \item Topic: Services such as installation, maintenance, repairing, cleaning, laundry, e-commerce
  \item Grade: Primary 5
\end{itemize}

\textbf{Solution}

(a) Total number of hours spent cleaning on the first and second day is:

\[
2\dfrac{3}{4} + 1\dfrac{2}{3} = \frac{11}{4} + \frac{5}{3}
\]
To add these, find a common denominator (12):

\[
\frac{11}{4} = \frac{33}{12},\quad \frac{5}{3} = \frac{20}{12}
\]
\[
\frac{33}{12} + \frac{20}{12} = \frac{53}{12} = 4\dfrac{5}{12}
\]

So, Mrs Tan spent $4\dfrac{5}{12}$ hours cleaning on both days altogether.

(b) The total time spent cleaning over two days was $5\dfrac{1}{2}$ hours. She spent $2\dfrac{3}{4}$ hours on the first day.

\[
5\dfrac{1}{2} - 2\dfrac{3}{4} = \frac{11}{2} - \frac{11}{4}
\]
Find a common denominator (4):
\[
\frac{11}{2} = \frac{22}{4},\quad \frac{11}{4}
\]
\[
\frac{22}{4} - \frac{11}{4} = \frac{11}{4} = 2\dfrac{3}{4}
\]

She spent $2\dfrac{3}{4}$ hours more cleaning over two days than just on the first day.

\section*{Question 7}
\textbf{Metadata}

\begin{itemize}
  \item Question ID: P3-WNDiv3d1d\_P1-WNSub2nd\_GPT4.1\_Services\_03
  \item Primary KC: WHOLE NUMBERS | Division | dividing whole numbers up to 3 digits by 1 digit
  \item Secondary KC: WHOLE NUMBERS | Subtraction | subtracting whole numbers
  \item Topic: Services such as installation, maintenance, repairing, cleaning, laundry, e-commerce
  \item Grade: Primary 3
\end{itemize}

\textbf{Solution}

To find the number of full washing machine loads, we divide $156$ by $4$. 

$156 \div 4 = 39$ loads, with no remainder. 

That means Sam washed all $156$ pieces in full loads, and no clothing was left over.

(a) Number of full loads: $39$

(b) Number of pieces left unwashed: $156 - (4 \times 39) = 156 - 156 = 0$

So, (a) $39$ full loads, (b) $0$ pieces left unwashed.

\section*{Question 8}
\textbf{Metadata}

\begin{itemize}
  \item Question ID: P4-DcAdd2d\_P4-DcAdd2nd\_GPT4.1\_Services\_03
  \item Primary KC: DECIMALS | Addition | adding decimals (up to 2 decimal places)
  \item Secondary KC: DECIMALS | Addition | adding decimals
  \item Topic: Services such as installation, maintenance, repairing, cleaning, laundry, e-commerce
  \item Grade: Primary 4
\end{itemize}

\textbf{Solution}

First, we add the charges for the living room and the kitchen:

$42.75 + 37.30 = 80.05$

Next, we add the cost for cleaning the balcony:

$80.05 + 15.60 = 95.65$

So, Mrs Tan paid a total of $95.65$ for all the cleaning services.

\section*{Question 9}
\textbf{Metadata}

\begin{itemize}
  \item Question ID: P6-FrDivPN\_P5-FrMul2nd\_GPT4.1\_Services\_02
  \item Primary KC: FRACTIONS | Division | dividing a proper fraction by a whole number
  \item Secondary KC: FRACTIONS | Multiplication | fraction multiplication
  \item Topic: Services such as installation, maintenance, repairing, cleaning, laundry, e-commerce
  \item Grade: Primary 6
\end{itemize}

\textbf{Solution}

(a) To find the amount of detergent used for 1 load, divide the total detergent by the number of loads: 

$\frac{3}{4} \div 5 = \frac{3}{4} \times \frac{1}{5} = \frac{3}{20}$ litres$.$ 

So, the shop uses $\frac{3}{20}$ litres for 1 load.

(b) For 3 loads: $\frac{3}{20} \times 3 = \frac{9}{20}$ litres$.$

Therefore, the shop needs $\frac{9}{20}$ litres of detergent to wash the 3 loads.

\section*{Question 10}
\textbf{Metadata}

\begin{itemize}
  \item Question ID: P3-WNMul3d1d\_P1-WNCmp\_GPT4.1\_Services\_03
  \item Primary KC: WHOLE NUMBERS | Multiplication | multiplying whole numbers up to 3 digits by 1 digit
  \item Secondary KC: WHOLE NUMBERS | Comparison and ordering | comparing and ordering whole numbers
  \item Topic: Services such as installation, maintenance, repairing, cleaning, laundry, e-commerce
  \item Grade: Primary 3
\end{itemize}

\textbf{Solution}

(a) For Type A: $128 \times 4 = 512$

For Type B: $85 \times 4 = 340$

For Type C: $64 \times 4 = 256$

(b) The totals are: $512$, $340$, $256$. 
Arranged from least to greatest: $256$, $340$, $512$.

\section*{Question 11}
\textbf{Metadata}

\begin{itemize}
  \item Question ID: O3-MXSub\_O3-MXAdd\_GPT4.1\_Services\_02
  \item Primary KC: MATRICES | Subtraction | subtraction of matrices
  \item Secondary KC: MATRICES | Addition | addition of matrices
  \item Topic: Services such as installation, maintenance, repairing, cleaning, laundry, e-commerce
  \item Grade: Secondary O-level 3/4
\end{itemize}

\textbf{Solution}

(a) To find the total number of services provided for each type and building in both months combined, add matrices $A$ and $B$:

$A + B = \begin{bmatrix} 24 & 15 & 10 \\ 18 & 12 & 8 \end{bmatrix} + \begin{bmatrix} 20 & 10 & 12 \\ 21 & 9 & 14 \end{bmatrix} = \begin{bmatrix} 24+20 & 15+10 & 10+12 \\ 18+21 & 12+9 & 8+14 \end{bmatrix} = \begin{bmatrix} 44 & 25 & 22 \\ 39 & 21 & 22 \end{bmatrix}$

So, the total services (basic, deep, window) for residential and commercial buildings in both months are:
- Residential: 44 (basic), 25 (deep), 22 (window)
- Commercial: 39 (basic), 21 (deep), 22 (window)

(b) To find the difference in number of services from February to January, subtract matrix $A$ from $B$:

$B - A = \begin{bmatrix} 20 & 10 & 12 \\ 21 & 9 & 14 \end{bmatrix} - \begin{bmatrix} 24 & 15 & 10 \\ 18 & 12 & 8 \end{bmatrix} = \begin{bmatrix} 20-24 & 10-15 & 12-10 \\ 21-18 & 9-12 & 14-8 \end{bmatrix} = \begin{bmatrix} -4 & -5 & 2 \\ 3 & -3 & 6 \end{bmatrix}$

So, compared to January, in February there were:
- Residential: 4 fewer basic, 5 fewer deep, 2 more window cleanings
- Commercial: 3 more basic, 3 fewer deep, 6 more window cleanings

\section*{Question 12}
\textbf{Metadata}

\begin{itemize}
  \item Question ID: P4-DcSub2d\_P4-DcCnv2Fr\_GPT4.1\_Services\_03
  \item Primary KC: DECIMALS | Subtraction | subtracting decimals (up to 2 decimal places)
  \item Secondary KC: DECIMALS | Conversion from decimals to fraction | expressing decimals as fractions
  \item Topic: Services such as installation, maintenance, repairing, cleaning, laundry, e-commerce
  \item Grade: Primary 4
\end{itemize}

\textbf{Solution}

(a) Amount Jenny needed to pay after the discount:

$35.60 - 12.25 = 23.35$

Jenny needed to pay $23.35$ after the discount.

(b) To express $23.35$ as a fraction:

$23.35 = 23 + 0.35$

$0.35 = \frac{35}{100} = \frac{7}{20}$

So, $23.35 = 23\frac{7}{20}$ or as an improper fraction:

$23\frac{7}{20} = \frac{23 \times 20 + 7}{20} = \frac{460 + 7}{20} = \frac{467}{20}$

Therefore, the amount Jenny needed to pay as a fraction in simplest form is $\frac{467}{20}$.

\section*{Question 13}
\textbf{Metadata}

\begin{itemize}
  \item Question ID: P5-FrMulPIm\_P2-FrSub2nd\_GPT4.1\_Services\_02
  \item Primary KC: FRACTIONS | Multiplication | multiplying a proper fraction and a proper/improper fractions
  \item Secondary KC: FRACTIONS | Subtraction | subtracting fractions
  \item Topic: Services such as installation, maintenance, repairing, cleaning, laundry, e-commerce
  \item Grade: Primary 5
\end{itemize}

\textbf{Solution}

Let the total number of clothes be $120$.

(a) The number of clothes cleaned in the morning is $\frac{5}{8} \times 120 = 75$.

The number of uniforms among these is $\frac{3}{5} \times 75 = 45$.

**Answer to (a):** $45$ uniforms are cleaned in the morning.

(b) The number of casual clothes cleaned in the morning is $75 - 45 = 30$.

The fraction of all the clothes received that are casual clothes cleaned in the morning:

$\frac{30}{120} = \frac{1}{4}$.

**Answer to (b):** $\frac{1}{4}$ of all the clothes are casual clothes cleaned in the morning.

(c) Fraction of clothes left to be cleaned in the afternoon:
The fraction cleaned in the morning is $\frac{5}{8}$, so the fraction left is $1 - \frac{5}{8} = \frac{3}{8}$.

**Answer to (c):** $\frac{3}{8}$ of all the clothes are left to be cleaned in the afternoon.

\section*{Question 14}
\textbf{Metadata}

\begin{itemize}
  \item Question ID: P4-DcSub2d\_P4-DcSub2nd\_GPT4.1\_Services\_03
  \item Primary KC: DECIMALS | Subtraction | subtracting decimals (up to 2 decimal places)
  \item Secondary KC: DECIMALS | Subtraction | subtracting decimals
  \item Topic: Services such as installation, maintenance, repairing, cleaning, laundry, e-commerce
  \item Grade: Primary 4
\end{itemize}

\textbf{Solution}

To find how much Mrs Tan paid after using the coupon, subtract the value of the coupon from the total bill:

$24.80 - 6.55 = 18.25$

So, Mrs Tan paid $18.25$ after using the coupon.

\section*{Question 15}
\textbf{Metadata}

\begin{itemize}
  \item Question ID: P3-WNAdd4d\_P1-WNAdd2nd\_GPT4.1\_Services\_04
  \item Primary KC: WHOLE NUMBERS | Addition | adding whole numbers up to 4 digits
  \item Secondary KC: WHOLE NUMBERS | Addition | adding whole numbers
  \item Topic: Services such as installation, maintenance, repairing, cleaning, laundry, e-commerce
  \item Grade: Primary 3
\end{itemize}

\textbf{Solution}

To find the total number of apartments cleaned, we need to add the numbers for each day:

$1,250 + 2,346 = 3,596$

$3,596 + 1,578 = 5,174$

$5,174 + 2,424 = 7,598$

Therefore, Mrs Tan's team cleaned a total of 7,598 apartments from Monday to Thursday.

\section*{Question 16}
\textbf{Metadata}

\begin{itemize}
  \item Question ID: P5-FrMulPIm\_P5-FrCnv2Dc\_GPT4.1\_Services\_02
  \item Primary KC: FRACTIONS | Multiplication | multiplying a proper fraction and a proper/improper fractions
  \item Secondary KC: FRACTIONS | Conversion to decimals | expressing fractions as decimals
  \item Topic: Services such as installation, maintenance, repairing, cleaning, laundry, e-commerce
  \item Grade: Primary 5
\end{itemize}

\textbf{Solution}

First, find the total amount of detergent bottles used:

$\frac{3}{5} \times \frac{7}{2} = \frac{3 \times 7}{5 \times 2} = \frac{21}{10}$

Now, express $\frac{21}{10}$ as a decimal:

$\frac{21}{10} = 2.1$

So, the company used $2.1$ bottles of detergent that day.

\section*{Question 17}
\textbf{Metadata}

\begin{itemize}
  \item Question ID: P3-FrSubRl12\_P2-FrSub2nd\_GPT4.1\_Services\_03
  \item Primary KC: FRACTIONS | Subtraction | subtracting two related fractions within one whole with denominators of given fractions not exceeding 12
  \item Secondary KC: FRACTIONS | Subtraction | subtracting fractions
  \item Topic: Services such as installation, maintenance, repairing, cleaning, laundry, e-commerce
  \item Grade: Primary 3
\end{itemize}

\textbf{Solution}

To find out how many more washing machines the repairman fixed after lunch compared to before lunch, subtract the number fixed before lunch from the number fixed after lunch:

\[ \frac{5}{8} - \frac{3}{8} = \frac{2}{8} \]

\[ \frac{2}{8} = \frac{1}{4} \]

So, the repairman fixed \(\frac{1}{4}\) more of the washing machines before lunch compared to after lunch.

\section*{Question 18}
\textbf{Metadata}

\begin{itemize}
  \item Question ID: O3-BPRepFrI\_O3-BPRepPosI\_GPT4.1\_Services\_02
  \item Primary KC: BASE AND POWER | Representation and concept  | fractional indices
  \item Secondary KC: BASE AND POWER | Representation and concept  | positive indices that is not 1
  \item Topic: Services such as installation, maintenance, repairing, cleaning, laundry, e-commerce
  \item Grade: Secondary O-level 3/4
\end{itemize}

\textbf{Solution}

Given effectiveness after $n$ days: 

$$ E_n = 100 \times \left(\frac{3}{4}\right)^{\frac{n}{2}} $$

(a) For $n = 4$:

$$ E_4 = 100 \times \left(\frac{3}{4}\right)^{\frac{4}{2}} = 100 \times \left(\frac{3}{4}\right)^2 $$

Calculate $\left(\frac{3}{4}\right)^2$:

$$ \left(\frac{3}{4}\right)^2 = \frac{9}{16} $$

So,

$$ E_4 = 100 \times \frac{9}{16} = 56.25 $$

Thus, the effectiveness after 4 days is $56.25\%$.

(b) The decrease from the initial is:

$$ 100\% - 56.25\% = 43.75\% $$

So, the effectiveness has decreased by $43.75\%$ after 4 days.

\section*{Question 19}
\textbf{Metadata}

\begin{itemize}
  \item Question ID: P4-WNDiv4d1d\_P1-WNAdd2nd\_GPT4.1\_Services\_03
  \item Primary KC: WHOLE NUMBERS | Division | dividing whole numbers up to 4 digits by 1 digit
  \item Secondary KC: WHOLE NUMBERS | Addition | adding whole numbers
  \item Topic: Services such as installation, maintenance, repairing, cleaning, laundry, e-commerce
  \item Grade: Primary 4
\end{itemize}

\textbf{Solution}

First, divide the total number of towels by the number of vans: 

$2,376 \div 4 = 594$

Each van got $594$ towels. 

Next, add the extra towels for each van:

$594 + 28 = 622$

Each van delivered $622$ towels in total.

\section*{Question 20}
\textbf{Metadata}

\begin{itemize}
  \item Question ID: O1-FDSub\_O1-FDAdd\_GPT4.1\_Services\_03
  \item Primary KC: FRACTIONS AND DECIMALS | Subtraction | Subtraction involving one fraction and one decimal number
  \item Secondary KC: FRACTIONS AND DECIMALS | Addition | Addition involving one fraction and one decimal number
  \item Topic: Services such as installation, maintenance, repairing, cleaning, laundry, e-commerce
  \item Grade: Secondary O-level 1
\end{itemize}

\textbf{Solution}

(a) The total time spent on both jobs is:

$2.5 + 0.75 = 3.25$ hours.

(b) The time spent on installation work excluding cleaning up is:

$2.5 - 1\frac{1}{4}$

First, convert $1\frac{1}{4}$ to decimal:
$1\frac{1}{4} = 1 + \frac{1}{4} = 1 + 0.25 = 1.25$

So, $2.5 - 1.25 = 1.25$ hours.

\textbf{Final Answers:}
(a) $3.25$ hours
(b) $1.25$ hours

\end{document}
