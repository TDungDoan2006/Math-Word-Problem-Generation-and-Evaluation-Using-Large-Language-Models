\documentclass{article}
\usepackage[utf8]{inputenc}
\usepackage{amsmath}
\usepackage{amsfonts}
\usepackage{amssymb}
\usepackage{graphicx}
\usepackage{hyperref}
\title{'TD Solutions services v8 v1'}
\author{Tien Dung Doan}
\begin{document}
\maketitle
\section*{Question 1}
\textbf{Metadata}

\begin{itemize}
  \item Question ID: P3-WNAdd4d\_P1-WNCmp\_GPT4.1\_Services\_07
  \item Primary KC: WHOLE NUMBERS | Addition | adding whole numbers up to 4 digits
  \item Secondary KC: WHOLE NUMBERS | Comparison and ordering | comparing and ordering whole numbers
  \item Topic: Services such as installation, maintenance, repairing, cleaning, laundry, hotel, retail, e-commerce, streaming services, digital services etc.
  \item Grade: Primary 3
\end{itemize}

\textbf{Solution}

(a) Number of shirts cleaned by the first laundry shop on Monday and Tuesday altogether:\\
\\
$2,356 + 1,879 = 4,235$\\
\\
The first laundry shop cleaned $4,235$ shirts in total on Monday and Tuesday.\\
\\
(b) The first laundry shop cleaned $4,235$ shirts, and the other shop cleaned $4,180$ shirts.\\
\\
Since $4,235 > 4,180$, the first laundry shop cleaned more shirts.\\
\\
Difference: $4,235 - 4,180 = 55$\\
\\
The first laundry shop cleaned $55$ more shirts than the other shop.

\section*{Question 2}
\textbf{Metadata}

\begin{itemize}
  \item Question ID: P3-WNSub4d\_P1-WNAdd2nd\_GPT4.1\_Services\_05
  \item Primary KC: WHOLE NUMBERS | Subtraction | subtracting whole numbers up to 4 digits
  \item Secondary KC: WHOLE NUMBERS | Addition | adding whole numbers
  \item Topic: Services such as installation, maintenance, repairing, cleaning, laundry, hotel, retail, e-commerce, streaming services, digital services etc.
  \item Grade: Primary 3
\end{itemize}

\textbf{Solution}

First, find the total number of sponges the company had after buying more:

\[
2,145 + 356 = 2,501
\]

Next, subtract the number of sponges left at the end of the month to find out how many were used:

\[
2,501 - 1,789 = 712
\]

\textbf{The company used $712$ sponges during the month.}

\section*{Question 3}
\textbf{Metadata}

\begin{itemize}
  \item Question ID: P3-WNDivRmd3d\_P1-WNSub2nd\_GPT4.1\_Services\_05
  \item Primary KC: WHOLE NUMBERS | Division | dividing whole numbers up to 3 digits by 1 digit with remainder 
  \item Secondary KC: WHOLE NUMBERS | Subtraction | subtracting whole numbers
  \item Topic: Services such as installation, maintenance, repairing, cleaning, laundry, hotel, retail, e-commerce, streaming services, digital services etc.
  \item Grade: Primary 3
\end{itemize}

\textbf{Solution}

(a) To find how many washing machines Mr. Tan can completely fill:

$157 \div 4 = 39$ remainder $1$

So, he can completely fill $39$ washing machines.

(b) The remainder is $1$, so he will have $1$ towel left after filling all the washing machines.

Next, he washes $3$ towels by hand, but since he only has $1$ towel left, after washing it:

Number of towels still to wash $= 1 - 1 = 0$

Final answer: Mr. Tan will not have any towels left to wash after washing them by hand.

\section*{Question 4}
\textbf{Metadata}

\begin{itemize}
  \item Question ID: P3-WNMul3d1d\_P1-WNCmp\_GPT4.1\_Services\_05
  \item Primary KC: WHOLE NUMBERS | Multiplication | multiplying whole numbers up to 3 digits by 1 digit
  \item Secondary KC: WHOLE NUMBERS | Comparison and ordering | comparing and ordering whole numbers
  \item Topic: Services such as installation, maintenance, repairing, cleaning, laundry, hotel, retail, e-commerce, streaming services, digital services etc.
  \item Grade: Primary 3
\end{itemize}

\textbf{Solution}

(a) Number of rooms cleaned by staff in the first hotel: \\
$7 \times 128 = 896$ 

So, $896$ rooms were cleaned in the first hotel.

(b) Number of rooms cleaned by staff in the second hotel: \\
$5 \times 153 = 765$ 

So, $765$ rooms were cleaned in the second hotel.

(c) Comparing both numbers: $896$ rooms (first hotel) and $765$ rooms (second hotel)

$896 > 765$, so the first hotel had more rooms cleaned.

Difference: $896 - 765 = 131$ 

\textbf{Answer:} The first hotel had more rooms cleaned by $131$ rooms.

\section*{Question 5}
\textbf{Metadata}

\begin{itemize}
  \item Question ID: P3-WNDiv3d1d\_P1-WNCmp\_GPT4.1\_Services\_05
  \item Primary KC: WHOLE NUMBERS | Division | dividing whole numbers up to 3 digits by 1 digit
  \item Secondary KC: WHOLE NUMBERS | Comparison and ordering | comparing and ordering whole numbers
  \item Topic: Services such as installation, maintenance, repairing, cleaning, laundry, hotel, retail, e-commerce, streaming services, digital services etc.
  \item Grade: Primary 3
\end{itemize}

\textbf{Solution}

(a) To find the number of towels each machine will wash: \
\[ \frac{288}{6} = 48 \]\nSo, each machine will wash $48$ towels.\
\
(b) Comparing the number of towels: The first shop has $288$ towels, and the shop next door has $245$ towels. \
Since $288 > 245$, the first shop has more towels to wash.\
\
(c) To find the difference: \
\[ 288 - 245 = 43 \]\nTherefore, the first shop has $43$ more towels to wash than the shop next door.

\section*{Question 6}
\textbf{Metadata}

\begin{itemize}
  \item Question ID: P3-FrSubRl12\_P2-FrCmp\_GPT4.1\_Services\_05
  \item Primary KC: FRACTIONS | Subtraction | subtracting two related fractions within one whole with denominators of given fractions not exceeding 12
  \item Secondary KC: FRACTIONS | Comparison and ordering | comparing and ordering fractions
  \item Topic: Services such as installation, maintenance, repairing, cleaning, laundry, hotel, retail, e-commerce, streaming services, digital services etc.
  \item Grade: Primary 3
\end{itemize}

\textbf{Solution}

(a)

Aisha completed $\frac{3}{4}$ of her cleaning task in the morning and $\frac{1}{4}$ in the afternoon. The difference is:
\[
\frac{3}{4} - \frac{1}{4} = \frac{2}{4} = \frac{1}{2}
\]
So, Aisha completed $\frac{1}{2}$ more of her cleaning in the morning than in the afternoon.

(b)

The two fractions are $\frac{3}{4}$ (morning) and $\frac{1}{4}$ (afternoon).
\[
\frac{3}{4} > \frac{1}{4}
\]
So, in order from greatest to least: $\frac{3}{4},\ \frac{1}{4}$.

\textbf{Final answers:}

(a) $\frac{1}{2}$

(b) $\frac{3}{4},\ \frac{1}{4}$ (from greatest to least)

\section*{Question 7}
\textbf{Metadata}

\begin{itemize}
  \item Question ID: P4-WNMul4d1d\_P1-WNCmp\_GPT4.1\_Services\_05
  \item Primary KC: WHOLE NUMBERS | Multiplication | multiplying whole numbers up to 4 digits by 1 digit or up to 3 digits by 2 digits
  \item Secondary KC: WHOLE NUMBERS | Comparison and ordering | comparing and ordering whole numbers
  \item Topic: Services such as installation, maintenance, repairing, cleaning, laundry, hotel, retail, e-commerce, streaming services, digital services etc.
  \item Grade: Primary 4
\end{itemize}

\textbf{Solution}

(a) For the first company: \\[2mm]
Number of uniforms = $367 \times 8 = 2936$ uniforms.\\[2mm]
For the second company: \\[2mm]
Number of uniforms = $519 \times 5 = 2595$ uniforms.\\[2mm]
(b) Comparing $2936$ and $2595$, the first company washed more uniforms.\\[2mm]
Difference = $2936 - 2595 = 341$ uniforms.\\[2mm]
\textbf{Answer:} The first company washed 2936 uniforms, and the second company washed 2595 uniforms. The first company washed 341 more uniforms than the second company.

\section*{Question 8}
\textbf{Metadata}

\begin{itemize}
  \item Question ID: P4-WNMul4d1d\_P4-WNRnd5d\_GPT4.1\_Services\_05
  \item Primary KC: WHOLE NUMBERS | Multiplication | multiplying whole numbers up to 4 digits by 1 digit or up to 3 digits by 2 digits
  \item Secondary KC: WHOLE NUMBERS | Rounding | rounding whole numbers up to 100000 to the nearest 10, 100 or 1000 
  \item Topic: Services such as installation, maintenance, repairing, cleaning, laundry, hotel, retail, e-commerce, streaming services, digital services etc.
  \item Grade: Primary 4
\end{itemize}

\textbf{Solution}

(a) The company earned $38$ for each of the $67$ apartments cleaned.

\[
38 \times 67 = 2546
\]
So, the company earned \textdollar2546 last month.

(b) To round \textdollar2546 to the nearest hundred, we look at the tens digit, which is 4.

Since 4 is less than 5, we round down. Thus,

\[
2546 \approx 2500 \quad \text{(to the nearest hundred)}
\]

So, the answer to part (b) is \textdollar2500.

\section*{Question 9}
\textbf{Metadata}

\begin{itemize}
  \item Question ID: P4-WNDiv4d1d\_P1-WNSub2nd\_GPT4.1\_Services\_05
  \item Primary KC: WHOLE NUMBERS | Division | dividing whole numbers up to 4 digits by 1 digit
  \item Secondary KC: WHOLE NUMBERS | Subtraction | subtracting whole numbers
  \item Topic: Services such as installation, maintenance, repairing, cleaning, laundry, hotel, retail, e-commerce, streaming services, digital services etc.
  \item Grade: Primary 4
\end{itemize}

\textbf{Solution}

(a) To find how many chairs each cleaner needs to clean, divide 2960 by 8:
\[
\frac{2960}{8} = 370
\]
So, each cleaner will need to clean 370 chairs.

(b) To find out how many chairs the cleaner has left to clean, subtract 340 from 370:
\[
370 - 340 = 30
\]
The cleaner has 30 chairs left to clean.

\section*{Question 10}
\textbf{Metadata}

\begin{itemize}
  \item Question ID: P4-WNDiv4d1d\_P4-WNRnd5d\_GPT4.1\_Services\_05
  \item Primary KC: WHOLE NUMBERS | Division | dividing whole numbers up to 4 digits by 1 digit
  \item Secondary KC: WHOLE NUMBERS | Rounding | rounding whole numbers up to 100000 to the nearest 10, 100 or 1000 
  \item Topic: Services such as installation, maintenance, repairing, cleaning, laundry, hotel, retail, e-commerce, streaming services, digital services etc.
  \item Grade: Primary 4
\end{itemize}

\textbf{Solution}

First, divide the total number of rooms by the number of teams:

\[
2,872 \div 6 = 478 
\]

Each team will be assigned $478$ rooms.

Next, round $478$ to the nearest $10$.

The digit in the ones place is $8$, which is $5$ or more, so we round up.

So, $478$ rounded to the nearest $10$ is $480$.

\textbf{Each team will be assigned 478 rooms, which is about 480 rooms when rounded to the nearest 10.}

\section*{Question 11}
\textbf{Metadata}

\begin{itemize}
  \item Question ID: P4-FrAddU12\_P2-FrCmp\_GPT4.1\_Services\_05
  \item Primary KC: FRACTIONS | Addition | adding unlike fractions with two different denominators not exceeding 12
  \item Secondary KC: FRACTIONS | Comparison and ordering | comparing and ordering fractions
  \item Topic: Services such as installation, maintenance, repairing, cleaning, laundry, hotel, retail, e-commerce, streaming services, digital services etc.
  \item Grade: Primary 4
\end{itemize}

\textbf{Solution}

1. To find the total time taken to clean both carpets, add $\frac{3}{4}$ hour and $\frac{2}{3}$ hour. \newline 
First, find a common denominator. The denominators are $4$ and $3$. The lowest common multiple is $12$. \newline 
$\frac{3}{4} = \frac{3 \times 3}{4 \times 3} = \frac{9}{12}$ \newline 
$\frac{2}{3} = \frac{2 \times 4}{3 \times 4} = \frac{8}{12}$ \newline 
Total time $= \frac{9}{12} + \frac{8}{12} = \frac{17}{12}$ hours \newline 
$\frac{17}{12}$ hours is $1 \frac{5}{12}$ hours. \newline 
\textbf{Answer:} It took $1 \frac{5}{12}$ hours to clean both carpets in total. \newline 
2. To compare $\frac{3}{4}$ and $\frac{2}{3}$, convert them to the same denominator as above: $\frac{3}{4} = \frac{9}{12}$ and $\frac{2}{3} = \frac{8}{12}$. \newline 
Since $8 < 9$, $\frac{2}{3} < \frac{3}{4}$. \newline 
\textbf{Answer:} The order from shortest to longest cleaning time is $\frac{2}{3}$ hour, $\frac{3}{4}$ hour.

\section*{Question 12}
\textbf{Metadata}

\begin{itemize}
  \item Question ID: P4-FrSubU12\_P3-FrSmp\_GPT4.1\_Services\_05
  \item Primary KC: FRACTIONS | Subtraction | subtracting unlike fractions with two different denominators not exceeding 12
  \item Secondary KC: FRACTIONS | Simplifying | expressing a fraction in its simplest form
  \item Topic: Services such as installation, maintenance, repairing, cleaning, laundry, hotel, retail, e-commerce, streaming services, digital services etc.
  \item Grade: Primary 4
\end{itemize}

\textbf{Solution}

First, subtract the fraction spent on cleaning supplies from the fraction spent on the streaming service:

\[
\frac{5}{8} - \frac{1}{3}
\]

To subtract these fractions, find a common denominator. The least common multiple of $8$ and $3$ is $24$.

\[
\frac{5}{8} = \frac{5 \times 3}{8 \times 3} = \frac{15}{24}
\]
\[
\frac{1}{3} = \frac{1 \times 8}{3 \times 8} = \frac{8}{24}
\]

Now subtract:

\[
\frac{15}{24} - \frac{8}{24} = \frac{7}{24}
\]

$\frac{7}{24}$ is already in its simplest form.

\textbf{Answer: } Aisha spent $\frac{7}{24}$ more of her allowance on the streaming service compared to the cleaning supplies.

\section*{Question 13}
\textbf{Metadata}

\begin{itemize}
  \item Question ID: P4-DcAdd2d\_P4-DcCmp3d\_GPT4.1\_Services\_05
  \item Primary KC: DECIMALS | Addition | adding decimals (up to 2 decimal places)
  \item Secondary KC: DECIMALS | Comparison and ordering | comparing and ordering decimals up to 3 decimal places
  \item Topic: Services such as installation, maintenance, repairing, cleaning, laundry, hotel, retail, e-commerce, streaming services, digital services etc.
  \item Grade: Primary 4
\end{itemize}

\textbf{Solution}

First, add the costs of all the cleaning services to find the total amount Mrs Lee should pay.\
\
Total amount: \
\[\
\textdollar12.40 + \textdollar8.75 + \textdollar15.30 = \textdollar36.45\
\]\
\
Mrs Lee should pay \textdollar36.45.\
\
Now, compare the three receipt totals: $36.450$, $35.890$, and $36.550$.\
\
- $36.450$ matches \textdollar36.45 exactly (note that $36.450$ is the same as $36.45$).\
- So, the correct receipt is $36.450$.\
\
Next, arrange the amounts from smallest to largest: \
\[\
35.890 < 36.450 < 36.550\
\]\
\
\textbf{Answer:}\
\begin{itemize}\
  \item Mrs Lee should pay \textdollar36.45.\
  \item The receipt with $36.450$ matches the correct total amount.\
  \item The receipt totals in order from smallest to largest are: $35.890$, $36.450$, $36.550$.\
\end{itemize}

\section*{Question 14}
\textbf{Metadata}

\begin{itemize}
  \item Question ID: P4-DcSub2d\_P4-DcCnv2Fr\_GPT4.1\_Services\_05
  \item Primary KC: DECIMALS | Subtraction | subtracting decimals (up to 2 decimal places)
  \item Secondary KC: DECIMALS | Conversion from decimals to fraction | expressing decimals as fractions
  \item Topic: Services such as installation, maintenance, repairing, cleaning, laundry, hotel, retail, e-commerce, streaming services, digital services etc.
  \item Grade: Primary 4
\end{itemize}

\textbf{Solution}

First, subtract the two amounts:

\[
\textdollar12.80 - \textdollar9.45 = \textdollar3.35
\]

So, Megan paid \textdollar3.35 more than her friend.

Next, express 3.35 as a fraction:

\[
3.35 = \frac{335}{100}
\]

Now, simplify the fraction by dividing both numerator and denominator by 5:

\[
\frac{335}{100} = \frac{67}{20}
\]

Therefore, Megan paid \textdollar3.35 (or \( \frac{67}{20} \)) more than her friend.

\section*{Question 15}
\textbf{Metadata}

\begin{itemize}
  \item Question ID: P4-DcSub2d\_P4-DcRnd3d\_GPT4.1\_Services\_05
  \item Primary KC: DECIMALS | Subtraction | subtracting decimals (up to 2 decimal places)
  \item Secondary KC: DECIMALS | Rounding | rounding decimals up to 3 decimal places to the nearest whole number, 1 decimal place and 2 decimal places 
  \item Topic: Services such as installation, maintenance, repairing, cleaning, laundry, hotel, retail, e-commerce, streaming services, digital services etc.
  \item Grade: Primary 4
\end{itemize}

\textbf{Solution}

First, find the difference in cost:

$25.75 - 18.60 = 7.15$

Now, round $7.15$ to:

1. The nearest whole number: $7$
2. One decimal place: $7.2$
3. Two decimal places: $7.15$

So, the cleaning service cost \textdollar7.15 more than the laundry service. Rounded, this is \textdollar7 (nearest whole number), \textdollar7.2 (one decimal place), and \textdollar7.15 (two decimal places).

\section*{Question 16}
\textbf{Metadata}

\begin{itemize}
  \item Question ID: P4-DcMul2d1d\_P4-DcCnv2Fr\_GPT4.1\_Services\_04
  \item Primary KC: DECIMALS | Multiplication | multiplying decimals (up to 2 decimal places) by a 1-digit whole number
  \item Secondary KC: DECIMALS | Conversion from decimals to fraction | expressing decimals as fractions
  \item Topic: Services such as installation, maintenance, repairing, cleaning, laundry, hotel, retail, e-commerce, streaming services, digital services etc.
  \item Grade: Primary 4
\end{itemize}

\textbf{Solution}

(a) The total cost is $3$ hours $\times$ $\textdollar4.75$ per hour.

$3 \times 4.75 = 14.25$

So, Ms Tan pays $\textdollar14.25$ in total.

(b) To express $4.75$ as a fraction:

$4.75 = 4 + 0.75 = 4 + \frac{75}{100}$

Simplify $\frac{75}{100}$:

$\frac{75}{100} = \frac{3}{4}$

So, $4.75 = 4\dfrac{3}{4} = \frac{19}{4}$

Therefore, the hourly rate as a fraction in its simplest form is $\frac{19}{4}$.

\section*{Question 17}
\textbf{Metadata}

\begin{itemize}
  \item Question ID: P4-DcMul2d1d\_P4-DcAdd2nd\_GPT4.1\_Services\_04
  \item Primary KC: DECIMALS | Multiplication | multiplying decimals (up to 2 decimal places) by a 1-digit whole number
  \item Secondary KC: DECIMALS | Addition | adding decimals
  \item Topic: Services such as installation, maintenance, repairing, cleaning, laundry, hotel, retail, e-commerce, streaming services, digital services etc.
  \item Grade: Primary 4
\end{itemize}

\textbf{Solution}

First, we find the total cost of printing 6 photos.

The cost for one photo is $2.35$. For 6 photos, the total is:

$2.35 \times 6 = 14.10$

So, printing 6 photos costs \textdollar14.10.

Next, we add the price of the photo album which is \textdollar3.80:

$14.10 + 3.80 = 17.90$

Therefore, Sarah spent \textdollar17.90 in total.

\section*{Question 18}
\textbf{Metadata}

\begin{itemize}
  \item Question ID: P4-DcDiv2d1d\_P4-DcCmp3d\_GPT4.1\_Services\_04
  \item Primary KC: DECIMALS | Division | dividing decimals (up to 2 decimal places) by a 1-digit whole number
  \item Secondary KC: DECIMALS | Comparison and ordering | comparing and ordering decimals up to 3 decimal places
  \item Topic: Services such as installation, maintenance, repairing, cleaning, laundry, hotel, retail, e-commerce, streaming services, digital services etc.
  \item Grade: Primary 4
\end{itemize}

\textbf{Solution}

(a) To find the cost per room for Mrs. Lim:

\[
\text{Cost per room} = \frac{38.75}{5} = 7.75
\]

So, Mrs. Lim paid \textdollar7.75 per room.

(b) The cost per room for the other house is \textdollar9.725.

Comparing the two amounts:
- Mrs. Lim's house: \textdollar7.750 (up to 3 decimal places)
- Other house: \textdollar9.725

Arranged from the lowest to highest:

\[
\textdollar7.75 < \textdollar9.725
\]

Therefore, the cost per room for Mrs. Lim's house is lower than the cost per room for the other house.

\section*{Question 19}
\textbf{Metadata}

\begin{itemize}
  \item Question ID: P4-DcDiv2d1d\_P4-DcAdd2nd\_GPT4.1\_Services\_04
  \item Primary KC: DECIMALS | Division | dividing decimals (up to 2 decimal places) by a 1-digit whole number
  \item Secondary KC: DECIMALS | Addition | adding decimals
  \item Topic: Services such as installation, maintenance, repairing, cleaning, laundry, hotel, retail, e-commerce, streaming services, digital services etc.
  \item Grade: Primary 4
\end{itemize}

\textbf{Solution}

First, find the total cost for the dresses: 

$3 \times 7.60 = 22.80$

Next, find the total cost for the shirts:

$2 \times 3.90 = 7.80$

Now, add the two amounts to get the total cost:

$22.80 + 7.80 = 30.60$

Divide the total cost equally among 5 family members:

$30.60 \div 5 = 6.12$

So, each person pays \textdollar6.12.

\section*{Question 20}
\textbf{Metadata}

\begin{itemize}
  \item Question ID: P5-FrAddMix\_P2-FrCmp\_GPT4.1\_Services\_04
  \item Primary KC: FRACTIONS | Addition | adding mixed numbers
  \item Secondary KC: FRACTIONS | Comparison and ordering | comparing and ordering fractions
  \item Topic: Services such as installation, maintenance, repairing, cleaning, laundry, hotel, retail, e-commerce, streaming services, digital services etc.
  \item Grade: Primary 5
\end{itemize}

\textbf{Solution}

Part (a):
\[
2\frac{1}{3} + 1\frac{3}{4} = \frac{7}{3} + \frac{7}{4}
\]
Find a common denominator (12):
\[
\frac{7}{3} = \frac{28}{12}\quad \frac{7}{4} = \frac{21}{12}
\]
Add:
\[
\frac{28}{12} + \frac{21}{12} = \frac{49}{12}
\]
Convert back to mixed number:
\[
\frac{49}{12} = 4\frac{1}{12}
\]
So, the total time spent is $4\frac{1}{12}$ hours.

Part (b):
Compare $4\frac{1}{12}$ hours and $3\frac{5}{6}$ hours:
Convert $3\frac{5}{6}$ to an improper fraction:
\[
3\frac{5}{6} = \frac{18}{6} + \frac{5}{6} = \frac{23}{6}
\]
Rewrite with denominator 12:
\[
\frac{23}{6} = \frac{46}{12}
\]
Now, compare $\frac{49}{12}$ and $\frac{46}{12}$:
\[
\frac{49}{12} > \frac{46}{12}
\]
So, $4\frac{1}{12}$ hours is more than $3\frac{5}{6}$ hours. If we arrange the days from least to greatest total cleaning time, the day with $4\frac{1}{12}$ hours will be placed after the day with $3\frac{5}{6}$ hours.

\section*{Question 21}
\textbf{Metadata}

\begin{itemize}
  \item Question ID: P5-FrSubMix\_P2-FrCmp\_GPT4.1\_Services\_04
  \item Primary KC: FRACTIONS | Subtraction | subtracting mixed numbers
  \item Secondary KC: FRACTIONS | Comparison and ordering | comparing and ordering fractions
  \item Topic: Services such as installation, maintenance, repairing, cleaning, laundry, hotel, retail, e-commerce, streaming services, digital services etc.
  \item Grade: Primary 5
\end{itemize}

\textbf{Solution}

(a) Amount of clothes sent on Monday: \(3\frac{2}{5} = \frac{17}{5}\)

Amount of clothes sent on Tuesday: \(2\frac{4}{5} = \frac{14}{5}\)

Difference: \[
\frac{17}{5} - \frac{14}{5} = \frac{3}{5}
\]
So, Mrs Lee sent \(\frac{3}{5}\) kg more clothes on Monday than on Tuesday.

(b) Amount of clothes sent on Wednesday:
\[
3\frac{1}{2} = \frac{7}{2}
\]
Let's convert all to decimals for easier comparison:

\(\frac{17}{5} = 3.4\)
\(\frac{14}{5} = 2.8\)
\(\frac{7}{2} = 3.5\)

Arranging from least to greatest:
Monday: 3.4 kg
Tuesday: 2.8 kg
Wednesday: 3.5 kg

So, Tuesday < Monday < Wednesday.

\[
2\frac{4}{5} < 3\frac{2}{5} < 3\frac{1}{2}
\]

Therefore, the order from least to greatest is: \(2\frac{4}{5}\) kg, \(3\frac{2}{5}\) kg, \(3\frac{1}{2}\) kg.

\section*{Question 22}
\textbf{Metadata}

\begin{itemize}
  \item Question ID: P5-FrMulImN\_P2-FrCmp\_GPT4.1\_Services\_04
  \item Primary KC: FRACTIONS | Multiplication | multiplying a proper/improper fraction and a whole number
  \item Secondary KC: FRACTIONS | Comparison and ordering | comparing and ordering fractions
  \item Topic: Services such as installation, maintenance, repairing, cleaning, laundry, hotel, retail, e-commerce, streaming services, digital services etc.
  \item Grade: Primary 5
\end{itemize}

\textbf{Solution}

(a) Number of uniforms laundered in $1$ day $= \frac{3}{4} \times 20 = \frac{3 \times 20}{4} = \frac{60}{4} = 15$.\\
Therefore, Mrs Lim launders $15$ uniforms in $1$ day.\\

(b) Number of uniforms Mrs Lim laundered on Monday $= 15$.\\
Number of uniforms her friend laundered $= \frac{5}{8} \times 15 = \frac{75}{8} = 9\frac{3}{8}$.\\
Comparing $15$ and $9\frac{3}{8}$, $15 > 9\frac{3}{8}$.\\
Difference $= 15 - 9\frac{3}{8} = 5\frac{5}{8}$.\\
Therefore, Mrs Lim laundered more uniforms than her friend by $5\frac{5}{8}$ uniforms.

\section*{Question 23}
\textbf{Metadata}

\begin{itemize}
  \item Question ID: P5-FrMulPIm\_P2-FrCmp\_GPT4.1\_Services\_04
  \item Primary KC: FRACTIONS | Multiplication | multiplying a proper fraction and a proper/improper fractions
  \item Secondary KC: FRACTIONS | Comparison and ordering | comparing and ordering fractions
  \item Topic: Services such as installation, maintenance, repairing, cleaning, laundry, hotel, retail, e-commerce, streaming services, digital services etc.
  \item Grade: Primary 5
\end{itemize}

\textbf{Solution}

For Hotel Galaxy:

The fraction of bedsheets sent is $\frac{4}{5}$. Of these, $\frac{2}{3}$ are white.

So, the fraction of all Hotel Galaxy's bedsheets that are white and sent for laundry:

$$
\frac{4}{5} \times \frac{2}{3} = \frac{8}{15}
$$

For Hotel Star:

The fraction sent is $\frac{3}{4}$. Of these, $\frac{1}{2}$ are white:

$$
\frac{3}{4} \times \frac{1}{2} = \frac{3}{8}
$$

Now, compare $\frac{8}{15}$ and $\frac{3}{8}$:

Find a common denominator (lowest common multiple of 15 and 8 is 120):

$\frac{8}{15} = \frac{8 \times 8}{15 \times 8} = \frac{64}{120}$

$\frac{3}{8} = \frac{3 \times 15}{8 \times 15} = \frac{45}{120}$

So $\frac{8}{15} > \frac{3}{8}$.

Arranging in ascending order:

$\frac{3}{8}$ (Hotel Star), $\frac{8}{15}$ (Hotel Galaxy)

\textbf{Answer:}
Hotel Galaxy sends a greater fraction of white bedsheets to Laundry Clean each week. The fractions in ascending order are $\frac{3}{8}$ (Hotel Star), $\frac{8}{15}$ (Hotel Galaxy).

\section*{Question 24}
\textbf{Metadata}

\begin{itemize}
  \item Question ID: P5-FrMulPIm\_P2-FrAdd2nd\_GPT4.1\_Services\_04
  \item Primary KC: FRACTIONS | Multiplication | multiplying a proper fraction and a proper/improper fractions
  \item Secondary KC: FRACTIONS | Addition | adding fractions
  \item Topic: Services such as installation, maintenance, repairing, cleaning, laundry, hotel, retail, e-commerce, streaming services, digital services etc.
  \item Grade: Primary 5
\end{itemize}

\textbf{Solution}

On Monday, the company cleaned $\frac{3}{5}$ of the mansion.

On Tuesday, they cleaned $\frac{2}{3}$ of Monday's cleaned area:

\[
\text{Tuesday's area} = \frac{2}{3} \times \frac{3}{5} = \frac{2 \times 3}{3 \times 5} = \frac{6}{15} = \frac{2}{5}
\]

On Wednesday, they cleaned $\frac{1}{10}$ of the mansion.

Now, add all cleaned fractions:

\[
\text{Total cleaned} = \text{Monday} + \text{Tuesday} + \text{Wednesday} = \frac{3}{5} + \frac{2}{5} + \frac{1}{10}
\]

First, add $\frac{3}{5}$ and $\frac{2}{5}$:
\[
\frac{3}{5} + \frac{2}{5} = \frac{5}{5} = 1
\]

Now add $1$ to $\frac{1}{10}$:
\[
1 + \frac{1}{10} = \frac{10}{10} + \frac{1}{10} = \frac{11}{10}
\]

So, after three days, the company has cleaned $\frac{11}{10}$ of the mansion, which means the entire mansion and $\frac{1}{10}$ more (perhaps the staff did overlapping cleaning or an extra area, depending on the context).

\section*{Question 25}
\textbf{Metadata}

\begin{itemize}
  \item Question ID: P5-FrMulPIm\_P3-FrSmp\_GPT4.1\_Services\_04
  \item Primary KC: FRACTIONS | Multiplication | multiplying a proper fraction and a proper/improper fractions
  \item Secondary KC: FRACTIONS | Simplifying | expressing a fraction in its simplest form
  \item Topic: Services such as installation, maintenance, repairing, cleaning, laundry, hotel, retail, e-commerce, streaming services, digital services etc.
  \item Grade: Primary 5
\end{itemize}

\textbf{Solution}

To find the total amount of floor cleaner solution used, we multiply the amount needed for one room by the number of rooms:

\[
\text{Total floor cleaner used} = \dfrac{5}{8} \times \dfrac{9}{4}
\]

Multiply the numerators and denominators:

\[
\dfrac{5 \times 9}{8 \times 4} = \dfrac{45}{32}
\]

There are no common factors between 45 and 32, so the fraction is already in its simplest form.

\[
\boxed{\dfrac{45}{32}} \text{ litres}
\]

Thus, the company will use $\dfrac{45}{32}$ litres of floor cleaner solution in total.

\section*{Question 26}
\textbf{Metadata}

\begin{itemize}
  \item Question ID: P5-FrMulImIm\_P2-FrAdd2nd\_GPT4.1\_Services\_04
  \item Primary KC: FRACTIONS | Multiplication | multiplying two improper fractions
  \item Secondary KC: FRACTIONS | Addition | adding fractions
  \item Topic: Services such as installation, maintenance, repairing, cleaning, laundry, hotel, retail, e-commerce, streaming services, digital services etc.
  \item Grade: Primary 5
\end{itemize}

\textbf{Solution}

(a) Amount of cleaning solution needed for 3 office floors:

$\frac{9}{4} \times 3 = \frac{9 \times 3}{4} = \frac{27}{4}$ litres

$\frac{27}{4}$ litres can also be written as $6\frac{3}{4}$ litres.

(b) Amount of special rinse needed for 3 office floors:

$\frac{5}{6} \times 3 = \frac{15}{6} = \frac{5}{2}$ litres

$\frac{5}{2}$ litres can also be written as $2\frac{1}{2}$ litres.

Total amount of solution and rinse used:

$\frac{27}{4} + \frac{5}{2}$

First, convert $\frac{5}{2}$ to quarters: $\frac{5}{2} = \frac{10}{4}$

Now, add:

$\frac{27}{4} + \frac{10}{4} = \frac{37}{4}$ litres

$\frac{37}{4}$ litres is $9\frac{1}{4}$ litres.

\textbf{Answer:}
(a) $\frac{27}{4}$ litres or $6\frac{3}{4}$ litres of cleaning solution are needed.

(b) $\frac{37}{4}$ litres or $9\frac{1}{4}$ litres of cleaning solution and rinse are needed for all 3 office floors.

\section*{Question 27}
\textbf{Metadata}

\begin{itemize}
  \item Question ID: P5-FrMulImIm\_P2-FrSub2nd\_GPT4.1\_Services\_04
  \item Primary KC: FRACTIONS | Multiplication | multiplying two improper fractions
  \item Secondary KC: FRACTIONS | Subtraction | subtracting fractions
  \item Topic: Services such as installation, maintenance, repairing, cleaning, laundry, hotel, retail, e-commerce, streaming services, digital services etc.
  \item Grade: Primary 5
\end{itemize}

\textbf{Solution}

First, find the total number of loads washed:  
$\frac{9}{4} \times \frac{7}{3} = \frac{9 \times 7}{4 \times 3} = \frac{63}{12}$  
Simplify $\frac{63}{12}$. The greatest common divisor is 3:  
$\frac{63 \div 3}{12 \div 3} = \frac{21}{4}$  
So, $\frac{21}{4}$ loads are washed.  

Next, find how many loads are sent for ironing:  
$\frac{5}{6} \times \frac{21}{4} = \frac{5 \times 21}{6 \times 4} = \frac{105}{24}$  
Simplify $\frac{105}{24}$. The greatest common divisor is 3:  
$\frac{105 \div 3}{24 \div 3} = \frac{35}{8}$  
So, $\frac{35}{8}$ loads are sent for ironing.  

To find loads not sent for ironing:  
$\frac{21}{4} - \frac{35}{8}$  
Rewrite $\frac{21}{4}$ with denominator 8: $\frac{21}{4} = \frac{42}{8}$  
Now subtract: $\frac{42}{8} - \frac{35}{8} = \frac{7}{8}$  

\textbf{Answer:} $\frac{7}{8}$ load of laundry is not sent for ironing.

\section*{Question 28}
\textbf{Metadata}

\begin{itemize}
  \item Question ID: P5-FrMulImIm\_P3-FrSmp\_GPT4.1\_Services\_04
  \item Primary KC: FRACTIONS | Multiplication | multiplying two improper fractions
  \item Secondary KC: FRACTIONS | Simplifying | expressing a fraction in its simplest form
  \item Topic: Services such as installation, maintenance, repairing, cleaning, laundry, hotel, retail, e-commerce, streaming services, digital services etc.
  \item Grade: Primary 5
\end{itemize}

\textbf{Solution}

To find the total amount of cleaning solution used, multiply the amount used per room by the total number of rooms cleaned:

$\dfrac{7}{3} \times \dfrac{8}{5} = \dfrac{7 \times 8}{3 \times 5} = \dfrac{56}{15}$

$\dfrac{56}{15}$ cannot be simplified further because 56 and 15 have no common factors other than 1.

Therefore, the company uses $\dfrac{56}{15}$ litres of cleaning solution in total.

\section*{Question 29}
\textbf{Metadata}

\begin{itemize}
  \item Question ID: P5-FrMulMixN\_P2-FrSub2nd\_GPT4.1\_Services\_04
  \item Primary KC: FRACTIONS | Multiplication | multiplying a mixed number and a whole number
  \item Secondary KC: FRACTIONS | Subtraction | subtracting fractions
  \item Topic: Services such as installation, maintenance, repairing, cleaning, laundry, hotel, retail, e-commerce, streaming services, digital services etc.
  \item Grade: Primary 5
\end{itemize}

\textbf{Solution}

(a) To find the total hours Jenny expected to spend, multiply the time per room by the number of rooms:

\[
5 \times 2\dfrac{1}{4} = 5 \times \left(2 + \dfrac{1}{4}\right) = (5 \times 2) + \left(5 \times \dfrac{1}{4}\right) = 10 + \dfrac{5}{4} = 10 + 1\dfrac{1}{4} = 11\dfrac{1}{4}
\]

Jenny expected to spend $11\dfrac{1}{4}$ hours cleaning the 5 rooms.

(b) She spent $1\dfrac{1}{2}$ hours less than expected:

\[
11\dfrac{1}{4} - 1\dfrac{1}{2}
\]
First, express each as improper fractions:
\[
11\dfrac{1}{4} = \dfrac{45}{4}, \quad 1\dfrac{1}{2} = \dfrac{3}{2} = \dfrac{6}{4}
\]
Now subtract:
\[
\dfrac{45}{4} - \dfrac{6}{4} = \dfrac{39}{4} = 9\dfrac{3}{4}
\]
So, Jenny actually spent $9\dfrac{3}{4}$ hours cleaning the 5 rooms.

\section*{Question 30}
\textbf{Metadata}

\begin{itemize}
  \item Question ID: P5-DcMul3dK\_P4-DcCnv2Fr\_GPT4.1\_Services\_04
  \item Primary KC: DECIMALS | Multiplication | multiplying decimals (up to 3 decimal places) by 10, 100, 1000 and their multiples
  \item Secondary KC: DECIMALS | Conversion from decimals to fraction | expressing decimals as fractions
  \item Topic: Services such as installation, maintenance, repairing, cleaning, laundry, hotel, retail, e-commerce, streaming services, digital services etc.
  \item Grade: Primary 5
\end{itemize}

\textbf{Solution}

(a) The total cost of cleaning 12 kilograms of clothes is given by:

\[
3.75 \times 12 = 45
\]

Therefore, Mrs Lee needs to pay \textdollar45.

(b) To express 3.75 as a fraction:

\[
3.75 = \frac{375}{100}
\]

Reduce the fraction by dividing by the highest common factor, which is 25:

\[
\frac{375 \div 25}{100 \div 25} = \frac{15}{4}
\]

So, \textdollar3.75 as a fraction is \(\frac{15}{4}\).

\section*{Question 31}
\textbf{Metadata}

\begin{itemize}
  \item Question ID: P5-DcMul3dK\_P4-DcSub2nd\_GPT4.1\_Services\_04
  \item Primary KC: DECIMALS | Multiplication | multiplying decimals (up to 3 decimal places) by 10, 100, 1000 and their multiples
  \item Secondary KC: DECIMALS | Subtraction | subtracting decimals
  \item Topic: Services such as installation, maintenance, repairing, cleaning, laundry, hotel, retail, e-commerce, streaming services, digital services etc.
  \item Grade: Primary 5
\end{itemize}

\textbf{Solution}

(a) To find the cost for 140 square metres, multiply the rate per square metre by the total area:

\[
\text{Cost for 140 square metres} = 3.850 \times 140 = 539.0
\]

So, the cost for 140 square metres is \textdollar539.0.

(b) The area that was not cleaned is:
\[
140 - 138.6 = 1.4 \text{ square metres}
\]
The cost to be subtracted is:
\[
3.850 \times 1.4 = 5.39
\]
So, \textdollar5.39 should be subtracted.

(c) The final amount to be paid:
\[
539.0 - 5.39 = 533.61
\]

The manager needs to pay \textdollar533.61.

\section*{Question 32}
\textbf{Metadata}

\begin{itemize}
  \item Question ID: P5-DcDiv3dK\_P4-DcRnd3d\_GPT4.1\_Services\_04
  \item Primary KC: DECIMALS | Division | dividing decimals (up to 3 decimal places) by 10, 100, 1000 and their multiples
  \item Secondary KC: DECIMALS | Rounding | rounding decimals up to 3 decimal places to the nearest whole number, 1 decimal place and 2 decimal places 
  \item Topic: Services such as installation, maintenance, repairing, cleaning, laundry, hotel, retail, e-commerce, streaming services, digital services etc.
  \item Grade: Primary 5
\end{itemize}

\textbf{Solution}

(a) The cost to clean $1$ square metre is $\frac{123.456}{1000}$.

$123.456 \div 1000 = 0.123456$

Up to 3 decimal places, the cost per square metre is $0.123$ (since the 4th decimal digit is $4$, we keep the 3rd digit as is).

(b) Rounding $0.123$ to the nearest 2 decimal places:

- The third decimal digit is $3$, which is less than $5$.
- So we keep the second decimal.

$0.123$ rounded to 2 decimal places is $0.12$.

(c) For $32$ square metres, estimate the total cost using the rounded price:

$0.12 \times 32 = 3.84$

Rounding $3.84$ to the nearest whole number:

- Since $0.84 \geq 0.5$, round up to $4$.

So the estimated total cost is $\textdollar4$.

\section*{Question 33}
\textbf{Metadata}

\begin{itemize}
  \item Question ID: P5-DcDiv3dK\_P4-DcAdd2nd\_GPT4.1\_Services\_04
  \item Primary KC: DECIMALS | Division | dividing decimals (up to 3 decimal places) by 10, 100, 1000 and their multiples
  \item Secondary KC: DECIMALS | Addition | adding decimals
  \item Topic: Services such as installation, maintenance, repairing, cleaning, laundry, hotel, retail, e-commerce, streaming services, digital services etc.
  \item Grade: Primary 5
\end{itemize}

\textbf{Solution}

(a) The cost of cleaning 1 square metre is:

\[
\frac{12.500}{1000} = 0.0125
\]

So, it costs \textdollar0.0125 to clean 1 square metre.

(b) The total area cleaned is:
\[
1000 + 250 = 1250\text{ square metres}
\]

The total cost is:
\[
1250 \times 0.0125 = 15.625
\]

(c) The customer needs to pay \textdollar15.625 altogether.

\section*{Question 34}
\textbf{Metadata}

\begin{itemize}
  \item Question ID: P5-PcRepWh\_P1-WNMul2nd\_GPT4.1\_Services\_04
  \item Primary KC: PERCENTAGE | Representation and concept | expressing a part of a whole as a percentage
  \item Secondary KC: WHOLE NUMBERS | Multiplication | multiplying whole numbers
  \item Topic: Services such as installation, maintenance, repairing, cleaning, laundry, hotel, retail, e-commerce, streaming services, digital services etc.
  \item Grade: Primary 5
\end{itemize}

\textbf{Solution}

\textbf{(a) Number of subscribers using the family subscription plan:}

30% of 600 = $\frac{30}{100} \times 600 = 0.3 \times 600 = 180$

So, 180 subscribers use the family subscription plan.

\textbf{(b) Total monthly revenue from family plans:}

Each family subscription costs \textdollar24 per month.

Total revenue = $180 \times 24 = 4320$

So, the total monthly revenue from the family subscription plans is \textdollar4320.

\section*{Question 35}
\textbf{Metadata}

\begin{itemize}
  \item Question ID: P5-PcRepWh\_P1-WNDiv2nd\_GPT4.1\_Services\_04
  \item Primary KC: PERCENTAGE | Representation and concept | expressing a part of a whole as a percentage
  \item Secondary KC: WHOLE NUMBERS | Division | dividing whole numbers
  \item Topic: Services such as installation, maintenance, repairing, cleaning, laundry, hotel, retail, e-commerce, streaming services, digital services etc.
  \item Grade: Primary 5
\end{itemize}

\textbf{Solution}

First, find the total number of subscribers at the end of the month:

\[
\text{Total subscribers} = 3000 + 450 = 3450
\]

Next, find the percentage of the total subscribers that are the new subscribers:

\[
\text{Percentage of new subscribers} = \frac{450}{3450} \times 100\%
\]

Divide:

\[
\frac{450}{3450} = \frac{45}{345} = \frac{9}{69} \approx 0.1304
\]

Multiply by $100\%$:

\[
0.1304 \times 100\% \approx 13.04\% 
\]

\textbf{Answer:} The new subscribers make up approximately $13.04\%$ of the total subscribers at the end of the month.

\section*{Question 36}
\textbf{Metadata}

\begin{itemize}
  \item Question ID: P5-RtFndU\_P2-DcCnvN2D\_GPT4.1\_Services\_04
  \item Primary KC: RATE | Finding number of unit | finding number of units given rate and total amount
  \item Secondary KC: DECIMALS | Conversion to larger units | converting a measurement from a smaller unit to a larger unit in decimal form
  \item Topic: Services such as installation, maintenance, repairing, cleaning, laundry, hotel, retail, e-commerce, streaming services, digital services etc.
  \item Grade: Primary 5
\end{itemize}

\textbf{Solution}

First, we need to convert the weight from grams to kilograms, since the rate is per kilogram.\newline
$6500$ grams $= 6500 \div 1000 = 6.5$ kilograms.\newline
Next, we find the total amount paid by multiplying the rate per kilogram by the number of kilograms.\newline
Total amount $= 2.50 \times 6.5 = 16.25$\newline
Therefore, Clara had to pay \textdollar16.25.

\section*{Question 37}
\textbf{Metadata}

\begin{itemize}
  \item Question ID: P6-FrDivPN\_P2-FrAdd2nd\_GPT4.1\_Services\_04
  \item Primary KC: FRACTIONS | Division | dividing a proper fraction by a whole number
  \item Secondary KC: FRACTIONS | Addition | adding fractions
  \item Topic: Services such as installation, maintenance, repairing, cleaning, laundry, hotel, retail, e-commerce, streaming services, digital services etc.
  \item Grade: Primary 6
\end{itemize}

\textbf{Solution}

Part (a):

To find out how many loads are needed for the curtains, divide the total mass of curtains by the size of each load:

\[
\text{Number of loads} = \frac{3}{4} \div \frac{1}{8} = \frac{3}{4} \times \frac{8}{1} = \frac{3 \times 8}{4} = \frac{24}{4} = 6
\]

So, Mrs Lee needs 6 loads to wash all her curtains.

Part (b):

For the cushion covers:
\[
\text{Number of loads} = \frac{5}{8} \div \frac{1}{8} = \frac{5}{8} \times \frac{8}{1} = \frac{5 \times 8}{8} = 5
\]

Total number of laundry loads needed:
\[
6 + 5 = 11
\]

\textbf{Final Answers:} 

(a) 6 loads 

(b) 11 loads in total

\section*{Question 38}
\textbf{Metadata}

\begin{itemize}
  \item Question ID: P6-FrDivPN\_P5-FrCnv2Dc\_GPT4.1\_Services\_09
  \item Primary KC: FRACTIONS | Division | dividing a proper fraction by a whole number
  \item Secondary KC: FRACTIONS | Conversion to decimals | expressing fractions as decimals
  \item Topic: Services such as installation, maintenance, repairing, cleaning, laundry, hotel, retail, e-commerce, streaming services, digital services etc.
  \item Grade: Primary 6
\end{itemize}

\textbf{Solution}

(a) Amount of cleaning solution for one small room: 

$\frac{3}{4} \div 6 = \frac{3}{4} \times \frac{1}{6} = \frac{3}{24} = \frac{1}{8}$ litre.

Answer: $\frac{1}{8}$ litre.

(b) Express $\frac{1}{8}$ as a decimal:

$\frac{1}{8} = 0.125$

Answer: 0.125 litre.

\section*{Question 39}
\textbf{Metadata}

\begin{itemize}
  \item Question ID: P6-FrDivPP\_P2-FrSub2nd\_GPT4.1\_Services\_04
  \item Primary KC: FRACTIONS | Division | dividing a whole number/proper fraction by a proper fraction
  \item Secondary KC: FRACTIONS | Subtraction | subtracting fractions
  \item Topic: Services such as installation, maintenance, repairing, cleaning, laundry, hotel, retail, e-commerce, streaming services, digital services etc.
  \item Grade: Primary 6
\end{itemize}

\textbf{Solution}

(a) To find the number of full washes, we divide $12$ kilograms by $\frac{3}{4}$ kilogram per wash:
\[
12 \div \frac{3}{4} = 12 \times \frac{4}{3} = \frac{48}{3} = 16
\]
So, the shop can make $16$ full washes.

(b) The unwashed laundry is $\frac{1}{2}$ kilogram from before, plus $\frac{5}{6}$ kilogram sent newly by the hotel. The total amount to be washed:
\[
\frac{5}{6} + \frac{1}{2} = \frac{5}{6} + \frac{3}{6} = \frac{8}{6} = \frac{4}{3}
\]
The shop has $\frac{4}{3}$ kilograms of laundry to wash now.

\section*{Question 40}
\textbf{Metadata}

\begin{itemize}
  \item Question ID: P6-FrDivPP\_P5-FrCnv2Dc\_GPT4.1\_Services\_09
  \item Primary KC: FRACTIONS | Division | dividing a whole number/proper fraction by a proper fraction
  \item Secondary KC: FRACTIONS | Conversion to decimals | expressing fractions as decimals
  \item Topic: Services such as installation, maintenance, repairing, cleaning, laundry, hotel, retail, e-commerce, streaming services, digital services etc.
  \item Grade: Primary 6
\end{itemize}

\textbf{Solution}

(a) To find the number of cleaning sessions Maya can pay for, divide the total amount she has by the cost for each session:

\[
\text{Number of sessions} = \frac{60}{\frac{3}{4}}
\]

Dividing by a fraction is the same as multiplying by its reciprocal:

\[
= 60 \times \frac{4}{3}
= 80
\]

Maya can pay for 80 cleaning sessions.

(b) Expressing the result as a decimal:

Since 80 is a whole number, as a decimal it is:

\[
80.0
\]

\textbf{Answer:} 

(a) She can pay for 80 cleaning sessions.

(b) Expressed as a decimal: 80.0

\section*{Question 41}
\textbf{Metadata}

\begin{itemize}
  \item Question ID: P6-PcFndWN\_P1-WNAdd2nd\_GPT4.1\_Services\_04
  \item Primary KC: PERCENTAGE | Finding the whole | finding the whole given a part and the percentage
  \item Secondary KC: WHOLE NUMBERS | Addition | adding whole numbers
  \item Topic: Services such as installation, maintenance, repairing, cleaning, laundry, hotel, retail, e-commerce, streaming services, digital services etc.
  \item Grade: Primary 6
\end{itemize}

\textbf{Solution}

Let the total number of rooms booked on Saturday be $x$. \newline
Initially, 40\% of the rooms requested the cleaning service, so:
\[ 0.4x = 36 \]
\[ x = \frac{36}{0.4} = 90 \]
Next, 18 more rooms requested the service, so the new total is:
\[ 36 + 18 = 54 \]
Let the new percentage be $p\%$:
\[ p\% \times 90 = 54 \Rightarrow \frac{p}{100} \times 90 = 54 \\ p = \frac{54}{90} \times 100 = 60 \]
But the question asks for the total number of rooms booked, so the answer is:

\[ \boxed{90} \]

\section*{Question 42}
\textbf{Metadata}

\begin{itemize}
  \item Question ID: P6-PcFndChg\_P1-WNDiv2nd\_GPT4.1\_Services\_04
  \item Primary KC: PERCENTAGE | Finding change | finding percentage increase/decrease
  \item Secondary KC: WHOLE NUMBERS | Division | dividing whole numbers
  \item Topic: Services such as installation, maintenance, repairing, cleaning, laundry, hotel, retail, e-commerce, streaming services, digital services etc.
  \item Grade: Primary 6
\end{itemize}

\textbf{Solution}

Let the original number of subscribers be $600$. The new number of subscribers is $750$.

\textbf{Step 1: Find the increase in the number of subscribers.}

Increase $= 750 - 600 = 150$

\textbf{Step 2: Find the percentage increase.}

Percentage increase $= \frac{\text{Increase}}{\text{Original number}} \times 100\%$

$= \frac{150}{600} \times 100\%$

$= 0.25 \times 100\%$

$= 25\%$

\textbf{Step 3: Divide the increase equally among 5 regions.}

Number of new subscribers each region gets $= \frac{150}{5} = 30$

\textbf{Answers:}
(a) The percentage increase in the number of subscribers is $25\%$.
(b) Each region gets $30$ new subscribers.

\section*{Question 43}
\textbf{Metadata}

\begin{itemize}
  \item Question ID: P6-RoFndRoWN\_P1-WNAdd2nd\_GPT4.1\_Services\_04
  \item Primary KC: RATIO | Finding ratio | finding the ratio of two or three given whole numbers
  \item Secondary KC: WHOLE NUMBERS | Addition | adding whole numbers
  \item Topic: Services such as installation, maintenance, repairing, cleaning, laundry, hotel, retail, e-commerce, streaming services, digital services etc.
  \item Grade: Primary 6
\end{itemize}

\textbf{Solution}

(a) The total number of rooms cleaned by the three cleaners is:

$15 + 20 + 25 = 60$

(b) The ratio of the number of rooms cleaned by Cleaner A to Cleaner B to Cleaner C is:

$15 : 20 : 25$

To express the ratio in its simplest form, divide each number by their greatest common divisor, which is 5:

$\frac{15}{5} : \frac{20}{5} : \frac{25}{5} = 3 : 4 : 5$

\textbf{Final answer:}

(a) 60 rooms 

(b) $3 : 4 : 5$

\section*{Question 44}
\textbf{Metadata}

\begin{itemize}
  \item Question ID: P6-AgRepLrEx\_P6-AgSmpLrEx\_GPT4.1\_Services\_09
  \item Primary KC: ALGEBRA | Representation and concept | translation of real-world situations into linear algebraic expressions
  \item Secondary KC: ALGEBRA | Simplifying | simplifying linear expressions
  \item Topic: Services such as installation, maintenance, repairing, cleaning, laundry, hotel, retail, e-commerce, streaming services, digital services etc.
  \item Grade: Primary 6
\end{itemize}

\textbf{Solution}

(a) Let $d$ be the number of devices. The total monthly cost, $C$, can be represented by the expression:

\[
C = 8 + 2d
\]

(b) John uses the service on a total of $3 + 2 = 5$ devices.

Substitute $d = 5$ into the expression:

\[
C = 8 + 2 \times 5
\]
\[
C = 8 + 10
\]
\[
C = 18
\]

So, John pays \textdollar18 per month.

\section*{Question 45}
\textbf{Metadata}

\begin{itemize}
  \item Question ID: O1-PcFndRslt\_P1-WNSub2nd\_GPT4.1\_Services\_04
  \item Primary KC: PERCENTAGE | Finding result after change | increasing/decreasing a quantity by a given percentage
  \item Secondary KC: WHOLE NUMBERS | Subtraction | subtracting whole numbers
  \item Topic: Services such as installation, maintenance, repairing, cleaning, laundry, hotel, retail, e-commerce, streaming services, digital services etc.
  \item Grade: Secondary O-level 1
\end{itemize}

\textbf{Solution}

First, find the amount of the 15\% discount on the original monthly fee:

\[
15\% \text{ of } \textdollar30 = \frac{15}{100} \times 30 = 4.50
\]

Discounted monthly fee:

\[
\textdollar30 - \textdollar4.50 = \textdollar25.50
\]

Total amount paid at the discounted rate for two months:

\[
2 \times \textdollar25.50 = \textdollar51.00
\]

Total amount that would have been paid for two months at the original rate:

\[
2 \times \textdollar30 = \textdollar60
\]

The difference (how much less Sarah paid):

\[
\textdollar60 - \textdollar51.00 = \textdollar9.00
\]

\textbf{Sarah paid \textdollar9.00 less in total for two months because of the promotion.}

\section*{Question 46}
\textbf{Metadata}

\begin{itemize}
  \item Question ID: O1-PcRepRvs\_O1-PcCnv2Dc\_GPT4.1\_Services\_05
  \item Primary KC: PERCENTAGE | Representation and concept | reverse percentages
  \item Secondary KC: PERCENTAGE | Conversion to decimals | expressing percentage as a decimal
  \item Topic: Services such as installation, maintenance, repairing, cleaning, laundry, hotel, retail, e-commerce, streaming services, digital services etc.
  \item Grade: Secondary O-level 1
\end{itemize}

\textbf{Solution}

(a) Let the original price be $x$. After a 25\% discount, the new price is $90$.

25\% as a decimal is $0.25$.

The price after the discount is:
\[
\text{Original price} - \text{Discount} = x - 0.25x = 0.75x
\]
We know $0.75x = 90$.
\[
0.75x = 90 \\
x = \frac{90}{0.75} \\
x = 120
\]
So, the original price is \textdollar120.

(b) For 3 nights at the original price:
\[
3 \times 120 = 360
\]
Therefore, the customer will pay \textdollar360 in total for 3 nights at the original price.

\section*{Question 47}
\textbf{Metadata}

\begin{itemize}
  \item Question ID: O2-RoRepDP\_P1-WNMul2nd\_GPT4.1\_Services\_04
  \item Primary KC: RATIO | Representation and concept | direct proportion
  \item Secondary KC: WHOLE NUMBERS | Multiplication | multiplying whole numbers
  \item Topic: Services such as installation, maintenance, repairing, cleaning, laundry, hotel, retail, e-commerce, streaming services, digital services etc.
  \item Grade: Secondary O-level 2
\end{itemize}

\textbf{Solution}

Since the number of apartments cleaned is directly proportional to the number of workers:

Let $x$ be the number of apartments 5 workers can clean in 4 hours.

Set up the ratio:

\[
\frac{3}{12} = \frac{5}{x}
\]

Cross multiply:

\[
3x = 5 \times 12 = 60\\
x = \frac{60}{3} = 20
\]

Therefore, 5 workers can clean 20 apartments in 4 hours.

If the company earns \textdollar20 per apartment:

\[
20 \times 20 = 400
\]

So, the company earns \textdollar400 for the work done by the 5 workers in those 4 hours.

\section*{Question 48}
\textbf{Metadata}

\begin{itemize}
  \item Question ID: O3-MXMul\_O3-MXAdd\_GPT4.1\_Services\_04
  \item Primary KC: MATRICES | Multiplication | multiplication of matrices
  \item Secondary KC: MATRICES | Addition | addition of matrices
  \item Topic: Services such as installation, maintenance, repairing, cleaning, laundry, hotel, retail, e-commerce, streaming services, digital services etc.
  \item Grade: Secondary O-level 3/4
\end{itemize}

\textbf{Solution}

(a) Add matrices $A$ and $B$:

$A + B = \begin{bmatrix} 8+10 & 5+6 \\ 6+7 & 7+8 \end{bmatrix} = \begin{bmatrix} 18 & 11 \\ 13 & 15 \end{bmatrix}$

(b) Multiply the resulting matrix with $C$. For Team 1, their row is $[18\ 11]$:  

$[18\ 11] \times \begin{bmatrix} 2 \\ 3 \end{bmatrix} = 18\times2 + 11\times3 = 36 + 33 = 69$

For Team 2, their row is $[13\ 15]$:  

$[13\ 15] \times \begin{bmatrix} 2 \\ 3 \end{bmatrix} = 13\times2 + 15\times3 = 26 + 45 = 71$

**Final answers:**

Total effort matrix: $\boxed{\begin{bmatrix} 18 & 11 \\ 13 & 15 \end{bmatrix}}$

Impact factors:

- Team 1: $69$
- Team 2: $71$

\section*{Question 49}
\textbf{Metadata}

\begin{itemize}
  \item Question ID: O3-SPFndstd\_O2-SPFndmean\_GPT4.1\_Services\_04
  \item Primary KC: STATISTICS AND PROBABILITY | Finding standard deviation | calculation of the standard deviation for a set of data
  \item Secondary KC: STATISTICS AND PROBABILITY | Finding mean deviation | calculation of the mean for a set of data
  \item Topic: Services such as installation, maintenance, repairing, cleaning, laundry, hotel, retail, e-commerce, streaming services, digital services etc.
  \item Grade: Secondary O-level 3/4
\end{itemize}

\textbf{Solution}

(a) To find the mean, sum up all the numbers and divide by $6$:

\[
\text{Mean} = \frac{8 + 10 + 6 + 12 + 9 + 11}{6} = \frac{56}{6} \approx 9.33
\]

(b) To find the standard deviation:

First, subtract the mean from each value and square the result:

\[
\begin{align*}
(8 - 9.33)^2 &= 1.77 \\
(10 - 9.33)^2 &= 0.44 \\
(6 - 9.33)^2 &= 11.11 \\
(12 - 9.33)^2 &= 7.11 \\
(9 - 9.33)^2 &= 0.11 \\
(11 - 9.33)^2 &= 2.78 \\
\end{align*}
\]

Sum the squared differences:

\[
1.77 + 0.44 + 11.11 + 7.11 + 0.11 + 2.78 = 23.32
\]

Divide by $6$ (since it's a population standard deviation):

\[
\frac{23.32}{6} \approx 3.89
\]

Take the square root:

\[
\text{Standard deviation} = \sqrt{3.89} \approx 1.97
\]

\textbf{Answers:}

(a) The mean number of cleaning jobs per week is $9.33$.

(b) The standard deviation is $1.97$ (correct to 2 decimal places).

\end{document}
