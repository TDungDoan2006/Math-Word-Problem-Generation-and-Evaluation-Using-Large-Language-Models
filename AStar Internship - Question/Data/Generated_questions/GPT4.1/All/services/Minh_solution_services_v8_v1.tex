\documentclass{article}
\usepackage[utf8]{inputenc}
\usepackage{amsmath}
\usepackage{amsfonts}
\usepackage{amssymb}
\usepackage{graphicx}
\usepackage{hyperref}
\title{'Minh Solutions services v8 v1'}
\author{Tien Dung Doan}
\begin{document}
\maketitle
\section*{Question 1}
\textbf{Metadata}

\begin{itemize}
  \item Question ID: P3-WNSub4d\_P1-WNCmp\_GPT4.1\_Services\_06
  \item Primary KC: WHOLE NUMBERS | Subtraction | subtracting whole numbers up to 4 digits
  \item Secondary KC: WHOLE NUMBERS | Comparison and ordering | comparing and ordering whole numbers
  \item Topic: Services such as installation, maintenance, repairing, cleaning, laundry, hotel, retail, e-commerce, streaming services, digital services etc.
  \item Grade: Primary 3
\end{itemize}

\textbf{Solution}

(a) First, subtract the number of pieces washed on Tuesday from the number washed on Monday:

$3,576 - 2,843 = 733$

Amy washed 733 more pieces of clothing on Monday than on Tuesday.

(b) The numbers are:
- Monday: $3,576$
- Tuesday: $2,843$
- Wednesday: $3,612$

Arranged from least to greatest:

$2,843$,$3,576$,$3,612$

\section*{Question 2}
\textbf{Metadata}

\begin{itemize}
  \item Question ID: P3-WNDivRmd3d\_P1-WNAdd2nd\_GPT4.1\_Services\_05
  \item Primary KC: WHOLE NUMBERS | Division | dividing whole numbers up to 3 digits by 1 digit with remainder 
  \item Secondary KC: WHOLE NUMBERS | Addition | adding whole numbers
  \item Topic: Services such as installation, maintenance, repairing, cleaning, laundry, hotel, retail, e-commerce, streaming services, digital services etc.
  \item Grade: Primary 3
\end{itemize}

\textbf{Solution}

First, we divide $215$ by $6$ to find out how many full baskets can be used and how many shirts are left:

$215 \div 6 = 35$ remainder $5$

So, there are $5$ shirts left without a basket on the first day.

On the second day, $58$ more shirts are received. Total number of dirty shirts on the second day is:

$5 + 58 = 63$

Therefore, the laundry service has $5$ leftover shirts from the first day. On the second day, after receiving $58$ more shirts, the total number of dirty shirts is $63$.

\section*{Question 3}
\textbf{Metadata}

\begin{itemize}
  \item Question ID: P3-WNDivRmd3d\_P1-WNMul2nd\_GPT4.1\_Services\_05
  \item Primary KC: WHOLE NUMBERS | Division | dividing whole numbers up to 3 digits by 1 digit with remainder 
  \item Secondary KC: WHOLE NUMBERS | Multiplication | multiplying whole numbers
  \item Topic: Services such as installation, maintenance, repairing, cleaning, laundry, hotel, retail, e-commerce, streaming services, digital services etc.
  \item Grade: Primary 3
\end{itemize}

\textbf{Solution}

(a) We divide $245$ by $6$ to find how many baskets can be filled completely:

$245 \div 6 = 40$ remainder $5$.

So, Mrs Tan can fill $40$ baskets completely, with $5$ towels left over.

(b) Each washing machine takes $4$ baskets.
$5$ washing machines can wash:
$4 \times 5 = 20$ baskets in one round.

Therefore, $20$ baskets of towels are washed after running $5$ washing machines at the same time.

\section*{Question 4}
\textbf{Metadata}

\begin{itemize}
  \item Question ID: P3-WNDiv3d1d\_P1-WNAdd2nd\_GPT4.1\_Services\_05
  \item Primary KC: WHOLE NUMBERS | Division | dividing whole numbers up to 3 digits by 1 digit
  \item Secondary KC: WHOLE NUMBERS | Addition | adding whole numbers
  \item Topic: Services such as installation, maintenance, repairing, cleaning, laundry, hotel, retail, e-commerce, streaming services, digital services etc.
  \item Grade: Primary 3
\end{itemize}

\textbf{Solution}

First, find the total number of clothes Sarah received by adding the shirts and trousers: 

\[
156 + 84 = 240
\]

So, there are 240 clothes in total. Now, divide these clothes equally among 6 machines:

\[
240 \div 6 = 40
\]

Each machine will wash 40 clothes.

\section*{Question 5}
\textbf{Metadata}

\begin{itemize}
  \item Question ID: P3-WNDiv3d1d\_P1-WNSub2nd\_GPT4.1\_Services\_05
  \item Primary KC: WHOLE NUMBERS | Division | dividing whole numbers up to 3 digits by 1 digit
  \item Secondary KC: WHOLE NUMBERS | Subtraction | subtracting whole numbers
  \item Topic: Services such as installation, maintenance, repairing, cleaning, laundry, hotel, retail, e-commerce, streaming services, digital services etc.
  \item Grade: Primary 3
\end{itemize}

\textbf{Solution}

First, we divide the total towels among the 4 washing machines:  

$216 \div 4 = 54$ towels in each washing machine.

The total number of clean towels after removing the 19 dirty towels is:

$216 - 19 = 197$

Therefore, there are $197$ clean towels now.

\section*{Question 6}
\textbf{Metadata}

\begin{itemize}
  \item Question ID: P3-WNDiv3d1d\_P1-WNMul2nd\_GPT4.1\_Services\_05
  \item Primary KC: WHOLE NUMBERS | Division | dividing whole numbers up to 3 digits by 1 digit
  \item Secondary KC: WHOLE NUMBERS | Multiplication | multiplying whole numbers
  \item Topic: Services such as installation, maintenance, repairing, cleaning, laundry, hotel, retail, e-commerce, streaming services, digital services etc.
  \item Grade: Primary 3
\end{itemize}

\textbf{Solution}

(a) To find out how many towels each washing machine washes, we divide $168$ by $4$:\\
\[\frac{168}{4} = 42\]\
So, each washing machine will wash $42$ towels.\\
(b) Now, we need to find out how many bundles of $8$ towels can be made from $42$ towels. We divide $42$ by $8$:\\
\[\frac{42}{8} = 5\text{ remainder }2\]\
Therefore, $5$ bundles of $8$ towels can be made, and there will be $2$ towels left over.

\section*{Question 7}
\textbf{Metadata}

\begin{itemize}
  \item Question ID: P3-FrAddRl12\_P2-FrCmp\_GPT4.1\_Services\_05
  \item Primary KC: FRACTIONS | Addition | adding two related fractions within one whole with denominators of given fractions not exceeding 12
  \item Secondary KC: FRACTIONS | Comparison and ordering | comparing and ordering fractions
  \item Topic: Services such as installation, maintenance, repairing, cleaning, laundry, hotel, retail, e-commerce, streaming services, digital services etc.
  \item Grade: Primary 3
\end{itemize}

\textbf{Solution}

(a) To find the fraction of drinks that were either orange juice or apple juice, add the two fractions:
\[
\frac{3}{8} + \frac{1}{4}
\]
First, convert $\frac{1}{4}$ to have a denominator of 8:
\[
\frac{1}{4} = \frac{2}{8}
\]
Now add:
\[
\frac{3}{8} + \frac{2}{8} = \frac{5}{8}
\]
So, $\frac{5}{8}$ of the drinks were either orange juice or apple juice.

(b) To compare $\frac{3}{8}$ (orange juice) and $\frac{1}{4}$ (apple juice), convert $\frac{1}{4}$ to $\frac{2}{8}$. Since $\frac{3}{8} > \frac{2}{8}$, more orange juice was served than apple juice.

Arrange the fractions from smallest to largest:

$\frac{1}{4}$ (apple juice), $\frac{3}{8}$ (orange juice), $\frac{5}{8}$ (total).

Thus:
\[
\frac{1}{4} < \frac{3}{8} < \frac{5}{8}
\]

\section*{Question 8}
\textbf{Metadata}

\begin{itemize}
  \item Question ID: P3-FrSubRl12\_P2-FrAdd2nd\_GPT4.1\_Services\_05
  \item Primary KC: FRACTIONS | Subtraction | subtracting two related fractions within one whole with denominators of given fractions not exceeding 12
  \item Secondary KC: FRACTIONS | Addition | adding fractions
  \item Topic: Services such as installation, maintenance, repairing, cleaning, laundry, hotel, retail, e-commerce, streaming services, digital services etc.
  \item Grade: Primary 3
\end{itemize}

\textbf{Solution}

First, add the fractions of rooms cleaned in the morning and afternoon:  

$\frac{2}{3} + \frac{1}{6}$  

To add, use a common denominator of 6:  

$\frac{2}{3} = \frac{4}{6}$  

So, $\frac{4}{6} + \frac{1}{6} = \frac{5}{6}$  

This means $\frac{5}{6}$ of the rooms have been cleaned.  

To find the fraction of rooms not yet cleaned, subtract $\frac{5}{6}$ from 1 (which is $\frac{6}{6}$):  

$\frac{6}{6} - \frac{5}{6} = \frac{1}{6}$  

Answer: $\frac{1}{6}$ of the rooms still need to be cleaned.

\section*{Question 9}
\textbf{Metadata}

\begin{itemize}
  \item Question ID: P4-WNMul4d1d\_P1-WNAdd2nd\_GPT4.1\_Services\_05
  \item Primary KC: WHOLE NUMBERS | Multiplication | multiplying whole numbers up to 4 digits by 1 digit or up to 3 digits by 2 digits
  \item Secondary KC: WHOLE NUMBERS | Addition | adding whole numbers
  \item Topic: Services such as installation, maintenance, repairing, cleaning, laundry, hotel, retail, e-commerce, streaming services, digital services etc.
  \item Grade: Primary 4
\end{itemize}

\textbf{Solution}

First, calculate the amount earned from cleaning office rooms:

$25 \times 36 = 900$

So, the company earned \textdollar900 from cleaning office rooms.

Next, calculate the amount earned from cleaning meeting rooms:

$40 \times 18 = 720$

So, the company earned \textdollar720 from cleaning meeting rooms.

Now, add the two amounts to find the total earnings:

$900 + 720 = 1620$

The company earned \textdollar1620 in total from cleaning all the rooms on that day.

\section*{Question 10}
\textbf{Metadata}

\begin{itemize}
  \item Question ID: P4-WNDiv4d1d\_P1-WNCmp\_GPT4.1\_Services\_05
  \item Primary KC: WHOLE NUMBERS | Division | dividing whole numbers up to 4 digits by 1 digit
  \item Secondary KC: WHOLE NUMBERS | Comparison and ordering | comparing and ordering whole numbers
  \item Topic: Services such as installation, maintenance, repairing, cleaning, laundry, hotel, retail, e-commerce, streaming services, digital services etc.
  \item Grade: Primary 4
\end{itemize}

\textbf{Solution}

(a) To find the number of towels each washing machine has to wash, we divide $3,276$ by $6$:
\[
3,276 \div 6 = 546
\]
So, each washing machine should wash $546$ towels.

(b) If Machine A was given $50$ more towels, it has:
\[
546 + 50 = 596
\]
towels.

Since all the other machines still have $546$ towels, Machine A has $596$ towels, which is more than $546$.

Therefore, yes, Machine A has the most towels among all the machines.

\section*{Question 11}
\textbf{Metadata}

\begin{itemize}
  \item Question ID: P4-WNDiv4d1d\_P1-WNMul2nd\_GPT4.1\_Services\_05
  \item Primary KC: WHOLE NUMBERS | Division | dividing whole numbers up to 4 digits by 1 digit
  \item Secondary KC: WHOLE NUMBERS | Multiplication | multiplying whole numbers
  \item Topic: Services such as installation, maintenance, repairing, cleaning, laundry, hotel, retail, e-commerce, streaming services, digital services etc.
  \item Grade: Primary 4
\end{itemize}

\textbf{Solution}

First, we divide the total number of chairs by the number of workers:

\[
\frac{2448}{6} = 408
\]

So, each worker will clean 408 chairs.

Next, the hotel pays $\textdollar3$ for each chair cleaned by a worker:

\[
408 \times 3 = 1224
\]

Therefore, each worker will earn $\textdollar1224$ for the chairs they clean.

\section*{Question 12}
\textbf{Metadata}

\begin{itemize}
  \item Question ID: P4-FrAddU12\_P3-FrSmp\_GPT4.1\_Services\_05
  \item Primary KC: FRACTIONS | Addition | adding unlike fractions with two different denominators not exceeding 12
  \item Secondary KC: FRACTIONS | Simplifying | expressing a fraction in its simplest form
  \item Topic: Services such as installation, maintenance, repairing, cleaning, laundry, hotel, retail, e-commerce, streaming services, digital services etc.
  \item Grade: Primary 4
\end{itemize}

\textbf{Solution}

To find the total time, we add $\frac{3}{8}$ and $\frac{2}{3}$:

Find a common denominator for $8$ and $3$. The least common multiple is $24$.

$\frac{3}{8} = \frac{3 \times 3}{8 \times 3} = \frac{9}{24}$

$\frac{2}{3} = \frac{2 \times 8}{3 \times 8} = \frac{16}{24}$

Add:

$\frac{9}{24} + \frac{16}{24} = \frac{25}{24}$

$\frac{25}{24}$ is already in its simplest form.


Therefore, Mrs Lee spends a total of $\frac{25}{24}$ hours cleaning.

\section*{Question 13}
\textbf{Metadata}

\begin{itemize}
  \item Question ID: P4-FrSubU12\_P2-FrCmp\_GPT4.1\_Services\_05
  \item Primary KC: FRACTIONS | Subtraction | subtracting unlike fractions with two different denominators not exceeding 12
  \item Secondary KC: FRACTIONS | Comparison and ordering | comparing and ordering fractions
  \item Topic: Services such as installation, maintenance, repairing, cleaning, laundry, hotel, retail, e-commerce, streaming services, digital services etc.
  \item Grade: Primary 4
\end{itemize}

\textbf{Solution}

(a) To find out how much more storage Package A uses compared to Package B each month, subtract the fraction for Package B from Package A:

$\dfrac{3}{4} - \dfrac{5}{8}$

First, find a common denominator. The lowest common denominator between 4 and 8 is 8.

$\dfrac{3}{4} = \dfrac{6}{8}$

$\dfrac{6}{8} - \dfrac{5}{8} = \dfrac{1}{8}$

So, Package A uses $\dfrac{1}{8}$ more storage each month than Package B.

(b) To arrange the packages from least to most storage used, compare the fractions:

Package A: $\dfrac{3}{4} = \dfrac{6}{8}$

Package B: $\dfrac{5}{8}$

Package C: $\dfrac{2}{3}$. Find a common denominator for easy comparison. The lowest common denominator for 4, 8, and 3 is 24.

Package A: $\dfrac{3}{4} = \dfrac{18}{24}$

Package B: $\dfrac{5}{8} = \dfrac{15}{24}$

Package C: $\dfrac{2}{3} = \dfrac{16}{24}$

Arrange in order:

Package B ($\dfrac{5}{8}$ or $\dfrac{15}{24}$) < Package C ($\dfrac{2}{3}$ or $\dfrac{16}{24}$) < Package A ($\dfrac{3}{4}$ or $\dfrac{18}{24}$)

Thus, the order from least to most storage used is:

Package B, Package C, Package A.

\section*{Question 14}
\textbf{Metadata}

\begin{itemize}
  \item Question ID: P4-DcAdd2d\_P4-DcCnv2Fr\_GPT4.1\_Services\_05
  \item Primary KC: DECIMALS | Addition | adding decimals (up to 2 decimal places)
  \item Secondary KC: DECIMALS | Conversion from decimals to fraction | expressing decimals as fractions
  \item Topic: Services such as installation, maintenance, repairing, cleaning, laundry, hotel, retail, e-commerce, streaming services, digital services etc.
  \item Grade: Primary 4
\end{itemize}

\textbf{Solution}

To find the total amount Mrs Tan paid:

Add the two costs together:

\[
25.50 + 18.75 = 44.25
\]

So, the total amount paid is \textdollar44.25.

Next, express \$44.25 as a fraction:

44.25 can be written as \( \frac{4425}{100} \).

To simplify:

Both numerator and denominator can be divided by 25:

\[
\frac{4425 \div 25}{100 \div 25} = \frac{177}{4}
\]

Therefore, \$44.25 = \frac{177}{4}

**Final answers:**

- Decimal: \textdollar44.25
- Fraction: \$44.25 = \frac{177}{4}


\section*{Question 15}
\textbf{Metadata}

\begin{itemize}
  \item Question ID: P4-DcAdd2d\_P4-DcRnd3d\_GPT4.1\_Services\_05
  \item Primary KC: DECIMALS | Addition | adding decimals (up to 2 decimal places)
  \item Secondary KC: DECIMALS | Rounding | rounding decimals up to 3 decimal places to the nearest whole number, 1 decimal place and 2 decimal places 
  \item Topic: Services such as installation, maintenance, repairing, cleaning, laundry, hotel, retail, e-commerce, streaming services, digital services etc.
  \item Grade: Primary 4
\end{itemize}

\textbf{Solution}

(a) To find the total monthly cost, add the two subscription fees:

\[
8.95 + 13.299 = 22.249
\]

So, Aisha spends \textdollar22.249 each month in total.

(b) Now, round \(22.249\) as follows:

- To the nearest whole number: \(22.249 \approx 22\)
- To 1 decimal place: \(22.249 \approx 22.2\)
- To 2 decimal places: \(22.249 \approx 22.25\)

\textbf{Answers:}

(a) \textdollar22.249 per month.

(b)
- Nearest whole number: \textdollar22
- 1 decimal place: \textdollar22.2
- 2 decimal places: \textdollar22.25

\section*{Question 16}
\textbf{Metadata}

\begin{itemize}
  \item Question ID: P4-DcSub2d\_P4-DcAdd2nd\_GPT4.1\_Services\_05
  \item Primary KC: DECIMALS | Subtraction | subtracting decimals (up to 2 decimal places)
  \item Secondary KC: DECIMALS | Addition | adding decimals
  \item Topic: Services such as installation, maintenance, repairing, cleaning, laundry, hotel, retail, e-commerce, streaming services, digital services etc.
  \item Grade: Primary 4
\end{itemize}

\textbf{Solution}

(a) To find the total monthly bill before removing the kids' channel, add the two amounts:

$12.95 + 3.50 = 16.45$

So, Mrs Lee's total monthly bill before removing the kids' channel was \textdollar16.45.

(b) To find her new monthly bill after removing the kids' channel, subtract the price of the kids' channel from the original total:

$16.45 - 3.50 = 12.95$

So, Mrs Lee's new monthly bill is \textdollar12.95.

\section*{Question 17}
\textbf{Metadata}

\begin{itemize}
  \item Question ID: P4-DcDiv2d1d\_P4-DcCnv2Fr\_GPT4.1\_Services\_04
  \item Primary KC: DECIMALS | Division | dividing decimals (up to 2 decimal places) by a 1-digit whole number
  \item Secondary KC: DECIMALS | Conversion from decimals to fraction | expressing decimals as fractions
  \item Topic: Services such as installation, maintenance, repairing, cleaning, laundry, hotel, retail, e-commerce, streaming services, digital services etc.
  \item Grade: Primary 4
\end{itemize}

\textbf{Solution}

(a) To find how many shirts Peter sent, we divide the total amount paid by the cost per shirt:

\[
\text{Number of shirts} = \frac{24}{3.75}
\]

To divide by a decimal, we can first express $3.75$ as $\frac{375}{100}$, or multiply both numerator and denominator by $100$ to remove the decimal:

\[
\frac{24}{3.75} = \frac{24 \times 100}{3.75 \times 100} = \frac{2400}{375}
\]

Now, simplify:
\[
\frac{2400}{375} = \frac{2400 \div 75}{375 \div 75} = \frac{32}{5}
\]

$\frac{32}{5}$ is $6.4$, so Peter cleaned $6.4$ shirts. However, since you cannot have a fraction of a shirt, let us check by direct division:

\[
24 \div 3.75 = 6.4
\]

Therefore, Peter could have cleaned $6$ shirts (if only whole shirts are allowed) or $6.4$ shirts if partial pieces are allowed. According to the context (laundry per shirt), the answer is:

\[
\boxed{6\ \text{shirts}}
\]

(b) To express $3.75$ as a fraction:

\[
3.75 = \frac{375}{100}
\]

Divide numerator and denominator by $25$:
\[
\frac{375 \div 25}{100 \div 25} = \frac{15}{4}
\]

So,

\[
3.75 = \frac{15}{4}
\]

\textbf{Final answers:}

(a) Peter sent $6$ shirts for cleaning.

(b) $3.75$ expressed as a fraction in simplest form is $\frac{15}{4}$. 

\section*{Question 18}
\textbf{Metadata}

\begin{itemize}
  \item Question ID: P4-DcDiv2d1d\_P4-DcSub2nd\_GPT4.1\_Services\_04
  \item Primary KC: DECIMALS | Division | dividing decimals (up to 2 decimal places) by a 1-digit whole number
  \item Secondary KC: DECIMALS | Subtraction | subtracting decimals
  \item Topic: Services such as installation, maintenance, repairing, cleaning, laundry, hotel, retail, e-commerce, streaming services, digital services etc.
  \item Grade: Primary 4
\end{itemize}

\textbf{Solution}

First, we need to find the initial weight of the laundry in each machine by dividing the total weight by $4$:

\[
\text{Weight per machine before drying} = \frac{8.64}{4} = 2.16 \text{ kg}
\]

After washing and drying, the total weight decreased by $0.28$ kg due to water loss. So the total weight after drying is:

\[
8.64 - 0.28 = 8.36 \text{ kg}
\]

Now divide the new total weight by $4$ to find the weight in each machine after washing and drying:

\[
\text{Weight per machine after drying} = \frac{8.36}{4} = 2.09 \text{ kg}
\]

\textbf{Answer:} Each machine has $2.09$ kg of laundry after washing and drying.

\section*{Question 19}
\textbf{Metadata}

\begin{itemize}
  \item Question ID: P5-FrAddMix\_P3-FrSmp\_GPT4.1\_Services\_04
  \item Primary KC: FRACTIONS | Addition | adding mixed numbers
  \item Secondary KC: FRACTIONS | Simplifying | expressing a fraction in its simplest form
  \item Topic: Services such as installation, maintenance, repairing, cleaning, laundry, hotel, retail, e-commerce, streaming services, digital services etc.
  \item Grade: Primary 5
\end{itemize}

\textbf{Solution}

First, write the mixed numbers as improper fractions:

$2\frac{1}{4} = \frac{9}{4}$

$1\frac{2}{3} = \frac{5}{3}$

Find a common denominator for $4$ and $3$, which is $12$:

$\frac{9}{4} = \frac{9 \times 3}{4 \times 3} = \frac{27}{12}$

$\frac{5}{3} = \frac{5 \times 4}{3 \times 4} = \frac{20}{12}$

Add the fractions:

$\frac{27}{12} + \frac{20}{12} = \frac{47}{12}$

Now, express $\frac{47}{12}$ as a mixed number:

$47 \div 12 = 3$ remainder $11$, so:

$\frac{47}{12} = 3\frac{11}{12}$

Thus, Anna spent $3\frac{11}{12}$ hours in total cleaning her house.

\section*{Question 20}
\textbf{Metadata}

\begin{itemize}
  \item Question ID: P5-FrSubMix\_P2-FrAdd2nd\_GPT4.1\_Services\_04
  \item Primary KC: FRACTIONS | Subtraction | subtracting mixed numbers
  \item Secondary KC: FRACTIONS | Addition | adding fractions
  \item Topic: Services such as installation, maintenance, repairing, cleaning, laundry, hotel, retail, e-commerce, streaming services, digital services etc.
  \item Grade: Primary 5
\end{itemize}

\textbf{Solution}

Let us solve the problem step by step.

(a) First, add the hours cleaned in the morning and afternoon:

$4\dfrac{1}{2} + 1\dfrac{3}{4}$

Convert both to improper fractions:

$4\dfrac{1}{2} = \dfrac{9}{2}$

$1\dfrac{3}{4} = \dfrac{7}{4}$

Find a common denominator (4):

$\dfrac{9}{2} = \dfrac{18}{4}$

Add:

$\dfrac{18}{4} + \dfrac{7}{4} = \dfrac{25}{4}$

$\dfrac{25}{4}$ hours can be written as $6\dfrac{1}{4}$ hours.

So, total hours spent cleaning: \textbf{$6\dfrac{1}{4}$ hours}


(b) Mrs. Lee requested $7$ hours in total. So, hours remaining:

$7 - 6\dfrac{1}{4}$

Convert $7$ to a fraction with denominator 4:

$7 = \dfrac{28}{4}$

$6\dfrac{1}{4} = \dfrac{25}{4}$

Subtract:

$\dfrac{28}{4} - \dfrac{25}{4} = \dfrac{3}{4}$

So, $\boxed{\dfrac{3}{4}}$ more hours of cleaning remain for the day after the afternoon session.

\section*{Question 21}
\textbf{Metadata}

\begin{itemize}
  \item Question ID: P5-FrMulImN\_P2-FrSub2nd\_GPT4.1\_Services\_04
  \item Primary KC: FRACTIONS | Multiplication | multiplying a proper/improper fraction and a whole number
  \item Secondary KC: FRACTIONS | Subtraction | subtracting fractions
  \item Topic: Services such as installation, maintenance, repairing, cleaning, laundry, hotel, retail, e-commerce, streaming services, digital services etc.
  \item Grade: Primary 5
\end{itemize}

\textbf{Solution}

To find out how much disinfectant was used to clean 6 rooms, multiply the amount used per room by the number of rooms:

Amount of disinfectant used per room: $\dfrac{3}{5}\times10 = 6$ litres per room.

Disinfectant used for 6 rooms: $6\times6 = 36$ litres.

But this cannot be correct because $\dfrac{3}{5}$ of 10 litres per room for 6 rooms would exceed the total amount. Let’s check carefully:

Amount used per room: $\dfrac{3}{5}\times10 = 6$ litres for each room.
For 6 rooms: $6\times6 = 36$ litres, which is more than the bottle contains.
Let's try interpreting the context as using $\dfrac{3}{5}$ litres per room (instead of $\dfrac{3}{5}$ of the bottle per room).

Correct interpretation: Each room uses $\dfrac{3}{5}$ litres.
Thus, for 6 rooms: $6\times\dfrac{3}{5} = \dfrac{18}{5} = 3\dfrac{3}{5}$ litres.

Given last week, $\dfrac{2}{3}$ of the 10-litre bottle was already used:
$10\times\dfrac{2}{3} = \dfrac{20}{3} \approx 6.67$ litres used last week.

Start of Monday: $10 - 6.67 \approx 3.33$ litres left.

After cleaning 6 rooms:
Amount remaining $= 3.33 - 3.6 = -0.27$ litres (negative, indicating not enough left).

Try different values to keep the problem viable. Let's revise the per room usage to $\dfrac{1}{5}$ litres:

Amount used per room: $\dfrac{1}{5}$ litres.
For 6 rooms: $6\times\dfrac{1}{5} = \dfrac{6}{5} = 1\dfrac{1}{5}$ litres.

Given $\dfrac{2}{3}$ of the bottle was already used: $10\times\dfrac{2}{3} = \dfrac{20}{3} = 6\dfrac{2}{3}$ litres used last week.

So, remaining before Monday: $10 - 6\dfrac{2}{3} = 3\dfrac{1}{3}$ litres remain.

After 6 rooms: $3\dfrac{1}{3} - 1\dfrac{1}{5} = \dfrac{10}{3} - \dfrac{6}{5}$
Find common denominator (15): $\dfrac{50}{15} - \dfrac{18}{15} = \dfrac{32}{15}$ litres left.

\boxed{\dfrac{32}{15}} litres or $2\dfrac{2}{15}$ litres of disinfectant were left in the bottle after cleaning the 6 rooms on Monday.

\section*{Question 22}
\textbf{Metadata}

\begin{itemize}
  \item Question ID: P5-FrMulPIm\_P2-FrSub2nd\_GPT4.1\_Services\_04
  \item Primary KC: FRACTIONS | Multiplication | multiplying a proper fraction and a proper/improper fractions
  \item Secondary KC: FRACTIONS | Subtraction | subtracting fractions
  \item Topic: Services such as installation, maintenance, repairing, cleaning, laundry, hotel, retail, e-commerce, streaming services, digital services etc.
  \item Grade: Primary 5
\end{itemize}

\textbf{Solution}

First, we find the fraction of all the rooms that required special cleaning:

Let the total number of rooms be $1$ (as a whole).

The fraction of rooms that needed to be cleaned is $\frac{2}{3}$.

Out of these, $\frac{4}{5}$ needed special cleaning:

$\frac{2}{3} \times \frac{4}{5} = \frac{8}{15}$

So, $\frac{8}{15}$ of all rooms needed special cleaning.

After lunch, the staff finished $\frac{7}{15}$ of all rooms.

To find the fraction of rooms that needed to be cleaned but were not yet cleaned after lunch:

We know $\frac{2}{3}$ of all rooms needed to be cleaned, and $\frac{7}{15}$ of all rooms have been cleaned (including both normal and special cleaning rooms).

The fraction of rooms that needed to be cleaned but were not cleaned is:

$\frac{2}{3} - \frac{7}{15}$

To subtract, convert $\frac{2}{3}$ to fifteenths:

$\frac{2}{3} = \frac{10}{15}$

So

$\frac{10}{15} - \frac{7}{15} = \frac{3}{15} = \frac{1}{5}$

**Answer:**

$\boxed{\frac{1}{5}}$ of all the rooms in the hotel still needed to be cleaned after lunch.

\section*{Question 23}
\textbf{Metadata}

\begin{itemize}
  \item Question ID: P5-FrMulPIm\_P5-FrCnv2Dc\_GPT4.1\_Services\_04
  \item Primary KC: FRACTIONS | Multiplication | multiplying a proper fraction and a proper/improper fractions
  \item Secondary KC: FRACTIONS | Conversion to decimals | expressing fractions as decimals
  \item Topic: Services such as installation, maintenance, repairing, cleaning, laundry, hotel, retail, e-commerce, streaming services, digital services etc.
  \item Grade: Primary 5
\end{itemize}

\textbf{Solution}

(a) Each blanket originally costs \textdollar4.50. The price she pays after the $\frac{2}{3}$ discount is:

$= \frac{2}{3} \times 4.50$

$= \frac{2}{3} \times \frac{9}{2}$  (since $4.50 = \frac{9}{2}$)

$= \frac{2 \times 9}{3 \times 2} = \frac{18}{6} = 3$

So, after the $\frac{2}{3}$ discount, each blanket costs \textdollar3.

(b) Expressed as a decimal, the price per blanket after the discount is $3.00$.

\section*{Question 24}
\textbf{Metadata}

\begin{itemize}
  \item Question ID: P5-FrMulMixN\_P2-FrCmp\_GPT4.1\_Services\_04
  \item Primary KC: FRACTIONS | Multiplication | multiplying a mixed number and a whole number
  \item Secondary KC: FRACTIONS | Comparison and ordering | comparing and ordering fractions
  \item Topic: Services such as installation, maintenance, repairing, cleaning, laundry, hotel, retail, e-commerce, streaming services, digital services etc.
  \item Grade: Primary 5
\end{itemize}

\textbf{Solution}

First, multiply the allowed daily hours by 5 to find the total hours for each plan.\\
\\
For Plan A: \\ 
$3\frac{1}{2} \times 5 = \frac{7}{2} \times 5 = \frac{35}{2} = 17\frac{1}{2}$ hours \\ 
\\
For Plan B: \\ 
$2\frac{3}{4} \times 5 = \frac{11}{4} \times 5 = \frac{55}{4} = 13\frac{3}{4}$ hours \\ 
\\
For Plan C: \\ 
$4\frac{1}{2} \times 5 = \frac{9}{2} \times 5 = \frac{45}{2} = 22\frac{1}{2}$ hours \\ 
\\
Now, arrange the total hours from least to greatest: \\ 
Plan B: $13\frac{3}{4}$ hours \\ 
Plan A: $17\frac{1}{2}$ hours \\ 
Plan C: $22\frac{1}{2}$ hours \\ 
\\
\textbf{Final order:} Plan B, Plan A, Plan C.

\section*{Question 25}
\textbf{Metadata}

\begin{itemize}
  \item Question ID: P5-FrMulMixN\_P3-FrSmp\_GPT4.1\_Services\_04
  \item Primary KC: FRACTIONS | Multiplication | multiplying a mixed number and a whole number
  \item Secondary KC: FRACTIONS | Simplifying | expressing a fraction in its simplest form
  \item Topic: Services such as installation, maintenance, repairing, cleaning, laundry, hotel, retail, e-commerce, streaming services, digital services etc.
  \item Grade: Primary 5
\end{itemize}

\textbf{Solution}

First, write the mixed number as an improper fraction:

$2\dfrac{1}{4} = \dfrac{9}{4}$

Multiply the improper fraction by the whole number:

$6 \times \dfrac{9}{4} = \dfrac{6 \times 9}{4} = \dfrac{54}{4}$

Now, simplify $\dfrac{54}{4}$:
Find the greatest common divisor of $54$ and $4$, which is $2$.

$\dfrac{54 \div 2}{4 \div 2} = \dfrac{27}{2}$

So, it takes $\dfrac{27}{2}$ hours to clean all $6$ rooms.

Answer: $\boxed{\dfrac{27}{2}}$ hours

\section*{Question 26}
\textbf{Metadata}

\begin{itemize}
  \item Question ID: P5-FrMulMixN\_P5-FrCnv2Dc\_GPT4.1\_Services\_04
  \item Primary KC: FRACTIONS | Multiplication | multiplying a mixed number and a whole number
  \item Secondary KC: FRACTIONS | Conversion to decimals | expressing fractions as decimals
  \item Topic: Services such as installation, maintenance, repairing, cleaning, laundry, hotel, retail, e-commerce, streaming services, digital services etc.
  \item Grade: Primary 5
\end{itemize}

\textbf{Solution}

(a) To find the total number of hours, multiply the time per room by the number of rooms:

\[ 1\frac{3}{4} \times 4 = \frac{7}{4} \times 4 = \frac{7\times 4}{4} = 7 \]

So, the company spent 7 hours cleaning all 4 rooms.

(b) Since 7 is already a whole number, as a decimal it is $7.0$.

\textbf{Answer:}
(a) 7 hours
(b) $7.0$ hours

\section*{Question 27}
\textbf{Metadata}

\begin{itemize}
  \item Question ID: P5-DcMul3dK\_P4-DcRnd3d\_GPT4.1\_Services\_04
  \item Primary KC: DECIMALS | Multiplication | multiplying decimals (up to 3 decimal places) by 10, 100, 1000 and their multiples
  \item Secondary KC: DECIMALS | Rounding | rounding decimals up to 3 decimal places to the nearest whole number, 1 decimal place and 2 decimal places 
  \item Topic: Services such as installation, maintenance, repairing, cleaning, laundry, hotel, retail, e-commerce, streaming services, digital services etc.
  \item Grade: Primary 5
\end{itemize}

\textbf{Solution}

(a) Total cost for 100 months: 

\[
6.125 \times 100 = 612.5
\]

So, the total cost is \textdollar612.5.

(b) Rounding to the nearest whole number:
\[
612.5 \rightarrow 613
\]

Rounding to the nearest 1 decimal place:
\[
612.5\text{ is already at 1 decimal place: }612.5
\]

Rounding to the nearest 2 decimal places:
\[
612.5\text{ to 2 decimal places: }612.50
\]

Final answers:

- Nearest whole number: \textdollar613
- Nearest 1 decimal place: \textdollar612.5
- Nearest 2 decimal places: \textdollar612.50

\section*{Question 28}
\textbf{Metadata}

\begin{itemize}
  \item Question ID: P5-DcDiv3dK\_P4-DcCnv2Fr\_GPT4.1\_Services\_04
  \item Primary KC: DECIMALS | Division | dividing decimals (up to 3 decimal places) by 10, 100, 1000 and their multiples
  \item Secondary KC: DECIMALS | Conversion from decimals to fraction | expressing decimals as fractions
  \item Topic: Services such as installation, maintenance, repairing, cleaning, laundry, hotel, retail, e-commerce, streaming services, digital services etc.
  \item Grade: Primary 5
\end{itemize}

\textbf{Solution}

(a) To find the amount Sarah needs to pay:

She needs to pay $\frac{13.6}{100}$ dollars.

$13.6 \div 100 = 0.136$

So, Sarah needs to pay $\textdollar0.136$.

(b) To express $0.136$ as a fraction:

$0.136 = \frac{136}{1000}$

To simplify $\frac{136}{1000}$, we find the greatest common divisor (GCD) of 136 and 1000, which is 4.

Divide both numerator and denominator by 4:

$\frac{136 \div 4}{1000 \div 4} = \frac{34}{250}$

So, the amount $0.136$ in fraction in its simplest form is $\frac{34}{250}$.

\section*{Question 29}
\textbf{Metadata}

\begin{itemize}
  \item Question ID: P5-RtFndR\_P2-DcCnvN2D\_GPT4.1\_Services\_04
  \item Primary KC: RATE | Finding rate | finding rate given total amount and number of units
  \item Secondary KC: DECIMALS | Conversion to larger units | converting a measurement from a smaller unit to a larger unit in decimal form
  \item Topic: Services such as installation, maintenance, repairing, cleaning, laundry, hotel, retail, e-commerce, streaming services, digital services etc.
  \item Grade: Primary 5
\end{itemize}

\textbf{Solution}

To find the average cost per room:

\[
\text{Average cost per room} = \frac{\textdollar72.80}{16} = \textdollar4.55
\]

To convert this amount to hundreds of dollars:

\[
\textdollar4.55 = 4.55 \text{ dollars}
\]

One hundred dollars is \textdollar100. To convert dollars to hundreds of dollars:

\[
\text{Amount in hundreds of dollars} = \frac{4.55}{100} = 0.0455
\]

So, the average cost per room in hundreds of dollars, expressed in decimal form, is $0.0455$.

\section*{Question 30}
\textbf{Metadata}

\begin{itemize}
  \item Question ID: P5-RtFndR\_P2-DcCnvD2N\_GPT4.1\_Services\_04
  \item Primary KC: RATE | Finding rate | finding rate given total amount and number of units
  \item Secondary KC: DECIMALS | Conversion to smaller units | converting a measurement from a larger unit in decimal form to a smaller unit
  \item Topic: Services such as installation, maintenance, repairing, cleaning, laundry, hotel, retail, e-commerce, streaming services, digital services etc.
  \item Grade: Primary 5
\end{itemize}

\textbf{Solution}

First, we need to find the total number of hours used to clean all 8 rooms.\newline

Each room took $1.5$ hours, so total hours $= 1.5 \times 8 = 12$ hours.\newline

Next, we find the rate per hour.\newline

Rate per hour $= \dfrac{\textdollar72.50}{12}$\newline

$= \textdollar6.04$ (rounded to the nearest cent).\newline

So, the cleaning charge per hour for each room is \textdollar6.04.

\section*{Question 31}
\textbf{Metadata}

\begin{itemize}
  \item Question ID: P5-RtFndT\_P2-DcCnvN2D\_GPT4.1\_Services\_04
  \item Primary KC: RATE | Finding total amount | finding total amount, given rate and number of units
  \item Secondary KC: DECIMALS | Conversion to larger units | converting a measurement from a smaller unit to a larger unit in decimal form
  \item Topic: Services such as installation, maintenance, repairing, cleaning, laundry, hotel, retail, e-commerce, streaming services, digital services etc.
  \item Grade: Primary 5
\end{itemize}

\textbf{Solution}

First, we need to convert $8500$ square centimetres to square metres. $1$ square metre $= 10\,000$ square centimetres.\\
So, $\displaystyle 8500$ square centimetres $= \frac{8500}{10\,000} = 0.85$ square metres.\\
Next, calculate the total cost: $0.85$ square metres $\times$ $\textdollar0.75$ per square metre $= 0.85 \times 0.75 = 0.6375$ dollars.\\
So, Mrs Tan needs to pay $\textdollar0.64$ (rounded to the nearest cent) for cleaning her carpet.

\section*{Question 32}
\textbf{Metadata}

\begin{itemize}
  \item Question ID: P5-RtFndU\_P2-DcCnvD2N\_GPT4.1\_Services\_04
  \item Primary KC: RATE | Finding number of unit | finding number of units given rate and total amount
  \item Secondary KC: DECIMALS | Conversion to smaller units | converting a measurement from a larger unit in decimal form to a smaller unit
  \item Topic: Services such as installation, maintenance, repairing, cleaning, laundry, hotel, retail, e-commerce, streaming services, digital services etc.
  \item Grade: Primary 5
\end{itemize}

\textbf{Solution}

(a) Lina’s total earnings $= 18.5 \times 7.50 = 138.75$

So, Lina earned \textdollar138.75 that week.

(b) To find the total number of minutes Lina worked:

Number of minutes $= 18.5 \times 60 = 1110$

So, Lina worked $1,110$ minutes in total that week.

\section*{Question 33}
\textbf{Metadata}

\begin{itemize}
  \item Question ID: P6-FrDivPN\_P3-FrSmp\_GPT4.1\_Services\_09
  \item Primary KC: FRACTIONS | Division | dividing a proper fraction by a whole number
  \item Secondary KC: FRACTIONS | Simplifying | expressing a fraction in its simplest form
  \item Topic: Services such as installation, maintenance, repairing, cleaning, laundry, hotel, retail, e-commerce, streaming services, digital services etc.
  \item Grade: Primary 6
\end{itemize}

\textbf{Solution}

To find the amount of juice in each glass, we divide the total amount of juice by the number of glasses:

\[
\frac{3}{4} \div 3 = \frac{3}{4} \times \frac{1}{3} = \frac{3 \times 1}{4 \times 3} = \frac{3}{12}
\]

Now, simplify $\frac{3}{12}$ to its simplest form by dividing both numerator and denominator by their highest common factor (which is 3):

\[
\frac{3 \div 3}{12 \div 3} = \frac{1}{4}
\]

Each glass will contain $\frac{1}{4}$ litre of juice.

\section*{Question 34}
\textbf{Metadata}

\begin{itemize}
  \item Question ID: P6-FrDivPP\_P2-FrCmp\_GPT4.1\_Services\_09
  \item Primary KC: FRACTIONS | Division | dividing a whole number/proper fraction by a proper fraction
  \item Secondary KC: FRACTIONS | Comparison and ordering | comparing and ordering fractions
  \item Topic: Services such as installation, maintenance, repairing, cleaning, laundry, hotel, retail, e-commerce, streaming services, digital services etc.
  \item Grade: Primary 6
\end{itemize}

\textbf{Solution}

(a) One cleaner can clean $\frac{3}{4}$ of a room in 1 hour.\
\
To find out how many hours it takes to clean 1 room:\\
Number of hours to clean 1 room $= 1 \div \frac{3}{4} = 1 \times \frac{4}{3} = \frac{4}{3}$ hours.\\
\
To clean 5 rooms:\\
Number of hours $= 5 \div \frac{3}{4} = 5 \times \frac{4}{3} = \frac{20}{3}$ hours $= 6\frac{2}{3}$ hours.\
\
(b) The second hotel's cleaner can clean $\frac{2}{3}$ of a room in 1 hour.\
\
The first hotel's cleaner: $\frac{3}{4}$ of a room in 1 hour.\
The second hotel's cleaner: $\frac{2}{3}$ of a room in 1 hour.\
\
Comparing $\frac{3}{4}$ and $\frac{2}{3}$:\\
Find common denominator: $\frac{3}{4} = \frac{9}{12}$; $\frac{2}{3} = \frac{8}{12}$.\\
Since $\frac{9}{12} > \frac{8}{12}$, the first hotel's cleaner is faster at cleaning a room.

\section*{Question 35}
\textbf{Metadata}

\begin{itemize}
  \item Question ID: P6-FrDivPP\_P3-FrSmp\_GPT4.1\_Services\_09
  \item Primary KC: FRACTIONS | Division | dividing a whole number/proper fraction by a proper fraction
  \item Secondary KC: FRACTIONS | Simplifying | expressing a fraction in its simplest form
  \item Topic: Services such as installation, maintenance, repairing, cleaning, laundry, hotel, retail, e-commerce, streaming services, digital services etc.
  \item Grade: Primary 6
\end{itemize}

\textbf{Solution}

(a) To find the number of machine loads needed, divide the total weight of clothes by the amount each machine can wash:
\[
\text{Number of machine loads} = \frac{18}{\frac{3}{4}}
\]
Dividing by a fraction is the same as multiplying by its reciprocal:
\[
\frac{18}{\frac{3}{4}} = 18 \times \frac{4}{3}
\]
\[
18 \times \frac{4}{3} = \frac{18 \times 4}{3} = \frac{72}{3} = 24
\]
So, $24$ machine loads are needed.

(b) The answer, $24$, is already a whole number and cannot be simplified further.

\textbf{Final Answer:} $24$ machine loads.

\section*{Question 36}
\textbf{Metadata}

\begin{itemize}
  \item Question ID: P6-PcFndWN\_P1-WNSub2nd\_GPT4.1\_Services\_04
  \item Primary KC: PERCENTAGE | Finding the whole | finding the whole given a part and the percentage
  \item Secondary KC: WHOLE NUMBERS | Subtraction | subtracting whole numbers
  \item Topic: Services such as installation, maintenance, repairing, cleaning, laundry, hotel, retail, e-commerce, streaming services, digital services etc.
  \item Grade: Primary 6
\end{itemize}

\textbf{Solution}

Let the usual service charge be $x$.

The customer paid 15\% less than the usual charge, so the amount paid is 85\% of the usual charge:

\[
85\% \times x = 60
\]
\[
0.85x = 60
\]
\[
x = \frac{60}{0.85}
\]
\[
x = 70.588...
\]
So, the usual service charge is \textdollar70.59 (rounded to the nearest cent).

\textbf{Verification:}
\[
15\% \times 70.59 = 0.15 \times 70.59 = 10.5885
\]
\[
70.59 - 10.59 = 60.00
\]
Therefore, the usual service charge is \boxed{\textdollar70.59}.

\section*{Question 37}
\textbf{Metadata}

\begin{itemize}
  \item Question ID: P6-PcFndChg\_P1-WNSub2nd\_GPT4.1\_Services\_04
  \item Primary KC: PERCENTAGE | Finding change | finding percentage increase/decrease
  \item Secondary KC: WHOLE NUMBERS | Subtraction | subtracting whole numbers
  \item Topic: Services such as installation, maintenance, repairing, cleaning, laundry, hotel, retail, e-commerce, streaming services, digital services etc.
  \item Grade: Primary 6
\end{itemize}

\textbf{Solution}

(a) Increase in price:

The increase is $54 - 45 = 9$.

So, the price increased by \textdollar9.

(b) Percentage increase:

\[
\text{Percentage increase} = \left(\frac{9}{45}\right) \times 100\%
\]
\[
= 0.2 \times 100\% = 20\%
\]

So, the percentage increase in the price of the subscription is $20\%$.

\section*{Question 38}
\textbf{Metadata}

\begin{itemize}
  \item Question ID: P6-PcFndChg\_P1-WNMul2nd\_GPT4.1\_Services\_04
  \item Primary KC: PERCENTAGE | Finding change | finding percentage increase/decrease
  \item Secondary KC: WHOLE NUMBERS | Multiplication | multiplying whole numbers
  \item Topic: Services such as installation, maintenance, repairing, cleaning, laundry, hotel, retail, e-commerce, streaming services, digital services etc.
  \item Grade: Primary 6
\end{itemize}

\textbf{Solution}

(a) To find the new price per office cleaning:

Percentage increase $= 15\%$

Original price $= \textdollar60$

Amount of increase $= 15\% \times 60 = \dfrac{15}{100} \times 60 = 9$

New price per office cleaning $= 60 + 9 = \textdollar69$

(b) To find the total for 4 offices:

Total amount $= 69 \times 4 = \textdollar276$

\textbf{Final Answer:}

(a) The new price for cleaning one office is \textdollar69.

(b) The total amount for cleaning 4 offices is \textdollar276.

\section*{Question 39}
\textbf{Metadata}

\begin{itemize}
  \item Question ID: P6-RoFndDvqWN\_P1-WNAdd2nd\_GPT4.1\_Services\_02
  \item Primary KC: RATIO | Finding divided quantities | dividing a given quantity in a given ratio
  \item Secondary KC: WHOLE NUMBERS | Addition | adding whole numbers
  \item Topic: Services such as installation, maintenance, repairing, cleaning, laundry, hotel, retail, e-commerce, streaming services, digital services etc.
  \item Grade: Primary 6
\end{itemize}

\textbf{Solution}

(a) The total parts in the ratio are $2 + 3 + 5 = 10$ parts.

Each part is worth $\textdollar480 \div 10 = \textdollar48$.

Amir receives $2$ parts: $2 \times \textdollar48 = \textdollar96$

Bryan receives $3$ parts: $3 \times \textdollar48 = \textdollar144$

Claire receives $5$ parts: $5 \times \textdollar48 = \textdollar240$

(b) Adding the \textdollar20 bonus:

Amir: $\textdollar96 + \textdollar20 = \textdollar116$

Bryan: $\textdollar144 + \textdollar20 = \textdollar164$

Claire: $\textdollar240 + \textdollar20 = \textdollar260$

\section*{Question 40}
\textbf{Metadata}

\begin{itemize}
  \item Question ID: P6-RoFndRoWN\_P1-WNSub2nd\_GPT4.1\_Services\_04
  \item Primary KC: RATIO | Finding ratio | finding the ratio of two or three given whole numbers
  \item Secondary KC: WHOLE NUMBERS | Subtraction | subtracting whole numbers
  \item Topic: Services such as installation, maintenance, repairing, cleaning, laundry, hotel, retail, e-commerce, streaming services, digital services etc.
  \item Grade: Primary 6
\end{itemize}

\textbf{Solution}

After the renovation, the number of single rooms is $85 + 15 = 100$. 

The number of double rooms is $120 - 15 = 105$.

The ratio of single rooms to double rooms is $100 : 105$.

To express this ratio in its simplest form, divide both numbers by their highest common factor, which is $5$:
\[
100 \div 5 = 20 \\
105 \div 5 = 21
\]
So, the simplest form of the ratio is $20 : 21$.

\section*{Question 41}
\textbf{Metadata}

\begin{itemize}
  \item Question ID: P6-RoFndRoWN\_P1-WNDiv2nd\_GPT4.1\_Services\_04
  \item Primary KC: RATIO | Finding ratio | finding the ratio of two or three given whole numbers
  \item Secondary KC: WHOLE NUMBERS | Division | dividing whole numbers
  \item Topic: Services such as installation, maintenance, repairing, cleaning, laundry, hotel, retail, e-commerce, streaming services, digital services etc.
  \item Grade: Primary 6
\end{itemize}

\textbf{Solution}

(a) The numbers of rooms cleaned are 48 (A), 36 (B), and 60 (C).

First, find the ratio of 48 : 36 : 60.

Find the greatest common divisor (GCD) of 48, 36, and 60. The GCD is 12.

$48 \div 12 = 4$
$36 \div 12 = 3$
$60 \div 12 = 5$

So, the ratio is $4 : 3 : 5$.

(b) Each cleaner cleaned 12 rooms. Number of cleaners sent to Building C:

Number of rooms in Building C $ = 60 $
Number of cleaners $= 60 \div 12 = 5$

Therefore, 5 cleaners were sent to Building C.

\section*{Question 42}
\textbf{Metadata}

\begin{itemize}
  \item Question ID: P6-RoFndTmWN\_P1-WNAdd2nd\_GPT4.1\_Services\_02
  \item Primary KC: RATIO | Finding a missing term | finding the missing term in a pair of equivalent ratios
  \item Secondary KC: WHOLE NUMBERS | Addition | adding whole numbers
  \item Topic: Services such as installation, maintenance, repairing, cleaning, laundry, hotel, retail, e-commerce, streaming services, digital services etc.
  \item Grade: Primary 6
\end{itemize}

\textbf{Solution}

The ratio of workers to apartments is $3:5$.

Let $x$ be the number of workers needed for 12 apartments. Set up the equivalent ratio:

\[
3:5 = x:12
\]

That means
\[
\frac{3}{5} = \frac{x}{12}
\]

Cross-multiply:
\[
3 \times 12 = 5 \times x \\
36 = 5x \\
x = \frac{36}{5} = 7.2
\]

However, since the number of workers must be a whole number, the company will need 8 workers for 12 apartments.

After adding 2 more workers:
\[
8 + 2 = 10
\]

\textbf{Answers:}

1. \textbf{8 workers} are needed to clean 12 apartments.
2. After adding 2 more workers, there are a \textbf{total of 10 workers}.

\section*{Question 43}
\textbf{Metadata}

\begin{itemize}
  \item Question ID: P6-AgRepLrEx\_P6-AgEvlLrEx\_GPT4.1\_Services\_09
  \item Primary KC: ALGEBRA | Representation and concept | translation of real-world situations into linear algebraic expressions
  \item Secondary KC: ALGEBRA | Evaluation | evaluating linear expressions by substitution
  \item Topic: Services such as installation, maintenance, repairing, cleaning, laundry, hotel, retail, e-commerce, streaming services, digital services etc.
  \item Grade: Primary 6
\end{itemize}

\textbf{Solution}

(a) The total cost consists of a fixed fee plus a variable amount that depends on the number of rooms.

Let $r$ be the number of rooms to be cleaned.

The total cost, $C = 40 + 12r$.

(b) If $r = 5$, substitute into the expression:

$C = 40 + 12 \times 5$

$C = 40 + 60$

$C = 100$

So, the total cost is \textdollar100.

\section*{Question 44}
\textbf{Metadata}

\begin{itemize}
  \item Question ID: P6-AgSlvLrN\_P6-AgRepLrEx\_GPT4.1\_Services\_09
  \item Primary KC: ALGEBRA | Solving simple linear equations | solving linear equations involving whole number coefficient and one variable only
  \item Secondary KC: ALGEBRA | Representation and concept | translation of real-world situations into linear algebraic expressions
  \item Topic: Services such as installation, maintenance, repairing, cleaning, laundry, hotel, retail, e-commerce, streaming services, digital services etc.
  \item Grade: Primary 6
\end{itemize}

\textbf{Solution}

Let $x$ be the cost of each visit in dollars. The total amount paid for 4 visits is $4x$.\n\nGiven that $4x = 120$.\n\nTo find $x$, we solve the equation:\n\n\[4x = 120\]\n\nDivide both sides by 4:\n\n\[x = \frac{120}{4}\]\n\n\[x = 30\]\n\nTherefore, the cleaning company charges \textdollar30 for each visit.

\section*{Question 45}
\textbf{Metadata}

\begin{itemize}
  \item Question ID: O1-RoRepFr\_P2-FrSub2nd\_GPT4.1\_Services\_04
  \item Primary KC: RATIO | Representation and concept | ratios involving fractions
  \item Secondary KC: FRACTIONS | Subtraction | subtracting fractions
  \item Topic: Services such as installation, maintenance, repairing, cleaning, laundry, hotel, retail, e-commerce, streaming services, digital services etc.
  \item Grade: Secondary O-level 1
\end{itemize}

\textbf{Solution}

(a) Total fraction spent on vacuuming and mopping:

\[
\dfrac{2}{5} + \dfrac{1}{4} = \dfrac{8}{20} + \dfrac{5}{20} = \dfrac{13}{20}
\]

Fraction spent on dusting:
\[
1 - \dfrac{13}{20} = \dfrac{20}{20} - \dfrac{13}{20} = \dfrac{7}{20}
\]

So, \(\dfrac{7}{20}\) of the cleaning time is spent on dusting.

(b) The ratio of the time spent on vacuuming : mopping : dusting is:
\[
\left(\dfrac{2}{5}\right) : \left(\dfrac{1}{4}\right) : \left(\dfrac{7}{20}\right)
\]

To write this as $a : b : c$ with the same denominator (20):
\[
\dfrac{2}{5} = \dfrac{8}{20},\quad \dfrac{1}{4} = \dfrac{5}{20},\quad \dfrac{7}{20}
\]

So, the ratio is:
\[
8 : 5 : 7
\]
Thus, the ratio of time spent on vacuuming to mopping to dusting is $8 : 5 : 7$.

\section*{Question 46}
\textbf{Metadata}

\begin{itemize}
  \item Question ID: O1-RoRepFr\_P6-FrDiv2nd\_GPT4.1\_Services\_04
  \item Primary KC: RATIO | Representation and concept | ratios involving fractions
  \item Secondary KC: FRACTIONS | Division | fraction division
  \item Topic: Services such as installation, maintenance, repairing, cleaning, laundry, hotel, retail, e-commerce, streaming services, digital services etc.
  \item Grade: Secondary O-level 1
\end{itemize}

\textbf{Solution}

Let the amount of Eco-Wash used be $x$ litres.~

Given the ratio Eco-Wash : Bright-Clean = $\frac{3}{4}:1$.

We also know that Bright-Clean used = $\frac{5}{8}$ litres.

So,
\[
\frac{x}{\frac{5}{8}} = \frac{3}{4} \div 1 = \frac{3}{4}
\]

Now solve for $x$:
\[
x = \frac{3}{4} \times \frac{5}{8}
\]
\[
x = \frac{15}{32}
\]
So, $\frac{15}{32}$ litres of Eco-Wash are needed.

Next, each bottle contains $\frac{1}{2}$ litre. To find how many bottles are needed:
\[
\text{Number of bottles} = \frac{\frac{15}{32}}{\frac{1}{2}} = \frac{15}{32} \times \frac{2}{1} = \frac{30}{32} = \frac{15}{16}
\]

Since the laundry cannot use a fraction of a bottle, they will need to open 1 full bottle to have enough Eco-Wash for one session.

\section*{Question 47}
\textbf{Metadata}

\begin{itemize}
  \item Question ID: O1-RoRepDc\_P4-DcSub2nd\_GPT4.1\_Services\_04
  \item Primary KC: RATIO | Representation and concept | ratios involving decimals
  \item Secondary KC: DECIMALS | Subtraction | subtracting decimals
  \item Topic: Services such as installation, maintenance, repairing, cleaning, laundry, hotel, retail, e-commerce, streaming services, digital services etc.
  \item Grade: Secondary O-level 1
\end{itemize}

\textbf{Solution}

(a) Let the number of basic packages sold be $2.5x$ and the number of premium packages be $1x$.

So, $2.5x + x = 195$

$3.5x = 195$

$x = \frac{195}{3.5} = 55.714...$

But the number of packages must be whole numbers. Since the ratio uses decimals, let's write it as $5:2$ (since $2.5:1 = 5:2$ when multiplied by $2$).

Let the number of basic packages be $5y$ and the number of premium packages be $2y$.

$5y + 2y = 7y = 195$

$y = 27.857...$

Again, since the total number must be a whole number, but $195$ divided by $7$ gives $27.857...$, which means either the total package number should be a multiple of $7$, or the ratio must stay decimal. Let's continue with decimal values as per the knowledge component:

Let $x$ be the number of sets. Then, number of basic packages $= 2.5x$, number of premium packages $= 1x$, and total $= 3.5x = 195$, so $x = 55.714...$

Number of basic packages $= 2.5 \times 55.714 = 139.285$

Number of premium packages $= 1 \times 55.714 = 55.714$

Since we must use whole number of packages in practice, assuming calculation with decimals as per the knowledge component:

Number of basic packages is approximately $139.29$, and the number of premium packages is approximately $55.71$.

(b) The difference in price between the premium and basic package is:

$\$5.60 - \$4.80 = \$0.80$

\boxed{\textdollar0.80}$


\section*{Question 48}
\textbf{Metadata}

\begin{itemize}
  \item Question ID: O1-PcRep2q\_O1-PcCnv2Dc\_GPT4.1\_Services\_05
  \item Primary KC: PERCENTAGE | Representation and concept | comparing two quantities by percentage
  \item Secondary KC: PERCENTAGE | Conversion to decimals | expressing percentage as a decimal
  \item Topic: Services such as installation, maintenance, repairing, cleaning, laundry, hotel, retail, e-commerce, streaming services, digital services etc.
  \item Grade: Secondary O-level 1
\end{itemize}

\textbf{Solution}

(a) The difference in cost between Plan B and Plan A is:

$\textdollar15 - \textdollar12 = \textdollar3$

To find the percentage by which Plan B is more expensive than Plan A:

$\left( \frac{3}{12} \right) \times 100\% = 25\%$

So, Plan B is 25\% more expensive than Plan A.

(b) To express 25\% as a decimal:

$25\% = 0.25$

Answer:
(a) Plan B is 25\% more expensive than Plan A.
(b) 25\% expressed as a decimal is $0.25$.

\section*{Question 49}
\textbf{Metadata}

\begin{itemize}
  \item Question ID: O1-PcFndRslt\_P1-WNDiv2nd\_GPT4.1\_Services\_04
  \item Primary KC: PERCENTAGE | Finding result after change | increasing/decreasing a quantity by a given percentage
  \item Secondary KC: WHOLE NUMBERS | Division | dividing whole numbers
  \item Topic: Services such as installation, maintenance, repairing, cleaning, laundry, hotel, retail, e-commerce, streaming services, digital services etc.
  \item Grade: Secondary O-level 1
\end{itemize}

\textbf{Solution}

(a) The decrease in price is $25\% \times \textdollar48 = \frac{25}{100} \times 48 = \textdollar12$.

So, the new monthly subscription fee = $\textdollar48 - \textdollar12 = \textdollar36$.

(b) Number of full months Jason can pay for is $\frac{192}{36} = 5.333\ldots$, so he can pay for 5 full months.

\section*{Question 50}
\textbf{Metadata}

\begin{itemize}
  \item Question ID: O1-AgSlvFrLr\_O1-AgRepEq\_GPT4.1\_Services\_05
  \item Primary KC: ALGEBRA | Solving | solving simple fractional equations that can be reduced to linear equations
  \item Secondary KC: ALGEBRA | Representation and concept | translation of simple real-world situations to equations
  \item Topic: Services such as installation, maintenance, repairing, cleaning, laundry, hotel, retail, e-commerce, streaming services, digital services etc.
  \item Grade: Secondary O-level 1
\end{itemize}

\textbf{Solution}

Let $x$ be the regular monthly price of the Premium plan.

1 month of Basic plan costs: $\textdollar12$.
\newline
2 months of Basic plan cost: $12 \times 2 = \textdollar24$.

The first month of Premium plan has a 1/3 discount, so you pay $\frac{2}{3}x$ for that month.

Setting up the equation:
\[ \frac{2}{3}x = 24 \]

To solve for $x$:
\begin{align*}
\frac{2}{3}x &= 24 \\[6pt]
x &= 24 \times \frac{3}{2} \\[6pt]
x &= 36
\end{align*}

\textbf{Answer:} The regular monthly price of the Premium plan is $\textdollar36$. 

\textbf{Equation used:} $\frac{2}{3}x = 24$.

\section*{Question 51}
\textbf{Metadata}

\begin{itemize}
  \item Question ID: O2-RoRepIvP\_P1-WNMul2nd\_GPT4.1\_Services\_04
  \item Primary KC: RATIO | Representation and concept | inverse proportion
  \item Secondary KC: WHOLE NUMBERS | Multiplication | multiplying whole numbers
  \item Topic: Services such as installation, maintenance, repairing, cleaning, laundry, hotel, retail, e-commerce, streaming services, digital services etc.
  \item Grade: Secondary O-level 2
\end{itemize}

\textbf{Solution}

(a) Let $t$ be the number of hours and $n$ be the number of workers. Since time taken is inversely proportional to the number of workers:
$$ t \propto \frac{1}{n} $$
or
$$ t \times n = k $$
where $k$ is a constant.

Given:
$$ 4 \text{ workers} \rightarrow 6 \text{ hours} $$
So,
$$ 4 \times 6 = 24 $$
Thus, for 6 workers:
$$ t \times 6 = 24 $$
$$ t = \frac{24}{6} = 4 $$

\textbf{Answer:} 6 workers will take 4 hours to clean the office.

(b) Let the required number of workers be $n$ if the cleaning must be done in 2 hours:
$$ n \times 2 = 24 $$
$$ n = \frac{24}{2} = 12 $$

\textbf{Answer:} The company must send at least 12 workers to clean the office in 2 hours.

\section*{Question 52}
\textbf{Metadata}

\begin{itemize}
  \item Question ID: O2-RoRepIvP\_P1-WNDiv2nd\_GPT4.1\_Services\_04
  \item Primary KC: RATIO | Representation and concept | inverse proportion
  \item Secondary KC: WHOLE NUMBERS | Division | dividing whole numbers
  \item Topic: Services such as installation, maintenance, repairing, cleaning, laundry, hotel, retail, e-commerce, streaming services, digital services etc.
  \item Grade: Secondary O-level 2
\end{itemize}

\textbf{Solution}

Let $h$ be the number of hours for 4 cleaners. Since the total amount of cleaning work is constant, and the number of cleaners and time taken are in inverse proportion:

$6 \times 8 = 4 \times h$

$48 = 4h$

$h = \frac{48}{4} = 12$ hours.

So, it will take 4 cleaners 12 hours to finish the job.

Now, the total payment of \textdollar480 is divided equally among 4 cleaners:

Each cleaner receives $= \frac{480}{4} = \textdollar120$.

\section*{Question 53}
\textbf{Metadata}

\begin{itemize}
  \item Question ID: O2-AgSlvLr2v\_O1-AgRepEq\_GPT4.1\_Services\_04
  \item Primary KC: ALGEBRA | Solving | solving linear equations in two variables
  \item Secondary KC: ALGEBRA | Representation and concept | translation of simple real-world situations to equations
  \item Topic: Services such as installation, maintenance, repairing, cleaning, laundry, hotel, retail, e-commerce, streaming services, digital services etc.
  \item Grade: Secondary O-level 2
\end{itemize}

\textbf{Solution}

Let the total cost of Package A be $C_A$ and Package B be $C_B$.

$C_A = 50 + 20x$

$C_B = 30 + 25x$

We want to find the value of $x$ such that $C_A = C_B$:

$50 + 20x = 30 + 25x$

Bring like terms together:

$50 - 30 = 25x - 20x$

$20 = 5x$

$x = \frac{20}{5} = 4$

Therefore, when $x = 4$, or 4 hours of cleaning are needed, the total cost for both packages will be the same.

\section*{Question 54}
\textbf{Metadata}

\begin{itemize}
  \item Question ID: O2-SPFndmdn\_O2-SPFndmode\_GPT4.1\_Services\_04
  \item Primary KC: STATISTICS AND PROBABILITY | Finding median | Finding median for a set of data
  \item Secondary KC: STATISTICS AND PROBABILITY | Finding mode | Finding mode for a set of data
  \item Topic: Services such as installation, maintenance, repairing, cleaning, laundry, hotel, retail, e-commerce, streaming services, digital services etc.
  \item Grade: Secondary O-level 2
\end{itemize}

\textbf{Solution}

(a) First, arrange the data in ascending order:

$9,\ 9,\ 9,\ 10,\ 11,\ 12,\ 12,\ 12,\ 14,\ 15$

There are $10$ data values. The median will be the average of the 5th and 6th numbers.

The 5th value is $11$ and the 6th value is $12$.

So, \[
    \text{Median} = \frac{11 + 12}{2} = 11.5
\]

(b) The mode is the number that appears most frequently.

From the arranged data:

$9$ appears $3$ times

$12$ appears $3$ times

All other numbers appear fewer times.

So, the modes are $9$ and $12$ (bimodal).

\textbf{Summary:}

- Median = $11.5$
- Modes = $9$ and $12$

\section*{Question 55}
\textbf{Metadata}

\begin{itemize}
  \item Question ID: O2-SPFndmdn\_O3-SPFndPctl\_GPT4.1\_Services\_04
  \item Primary KC: STATISTICS AND PROBABILITY | Finding median | Finding median for a set of data
  \item Secondary KC: STATISTICS AND PROBABILITY | Finding percentiles | finding percentiles for a set of data
  \item Topic: Services such as installation, maintenance, repairing, cleaning, laundry, hotel, retail, e-commerce, streaming services, digital services etc.
  \item Grade: Secondary O-level 2
\end{itemize}

\textbf{Solution}

First, arrange the data in ascending order:

$12,\ 13,\ 14,\ 15,\ 16,\ 17,\ 18,\ 19,\ 20,\ 21$

(a) **Median**:

Since there are 10 values (even number), the median is the average of the 5th and 6th values.

The 5th value: $16$  
The 6th value: $17$

Median $= \frac{16 + 17}{2} = 16.5$

So, the median number of shirts cleaned per day is $16.5$.

(b) **75th Percentile (Third Quartile, $Q_3$):**

The 75th percentile can be calculated using the position:

$P = 0.75 \times (n + 1) = 0.75 \times 11 = 8.25$

So, the 75th percentile is between the 8th and 9th values.

8th value: $19$  
9th value: $20$

Calculate the value at the 8.25th position:

$Q_3 = 19 + 0.25 \times (20 - 19) = 19 + 0.25 = 19.25$

Therefore, the 75th percentile (third quartile) of the number of shirts cleaned per day is $19.25$.

\section*{Question 56}
\textbf{Metadata}

\begin{itemize}
  \item Question ID: O3-BPOpr\_O3-BPRepPosI\_GPT4.1\_Services\_04
  \item Primary KC: BASE AND POWER | Operations | laws of indices
  \item Secondary KC: BASE AND POWER | Representation and concept  | positive indices that is not 1
  \item Topic: Services such as installation, maintenance, repairing, cleaning, laundry, hotel, retail, e-commerce, streaming services, digital services etc.
  \item Grade: Secondary O-level 3/4
\end{itemize}

\textbf{Solution}

Let $M(n)$ be the number of movies a subscriber can watch after $n$ years.

Since the number of movies doubles each year starting from $8$:

$M(n) = 8 \times 2^{n-1}$

This expression uses indices (laws of indices for multiplication).

For $n = 4$ years:

$M(4) = 8 \times 2^{4-1} = 8 \times 2^{3} = 8 \times 8 = 64$

So, after $4$ years, a subscriber can watch $64$ movies in that year.

\section*{Question 57}
\textbf{Metadata}

\begin{itemize}
  \item Question ID: O3-BPOpr\_O3-BPRepFrI\_GPT4.1\_Services\_04
  \item Primary KC: BASE AND POWER | Operations | laws of indices
  \item Secondary KC: BASE AND POWER | Representation and concept  | fractional indices
  \item Topic: Services such as installation, maintenance, repairing, cleaning, laundry, hotel, retail, e-commerce, streaming services, digital services etc.
  \item Grade: Secondary O-level 3/4
\end{itemize}

\textbf{Solution}

(a)\
We have $n = 4 \left(2^{\frac{1}{2}} \right)^k = 4 \times 2^{\frac{k}{2}} = 4 \times 2^a$ where $a = \frac{k}{2}$.\
\\
(b)\
After $k = 6$ years: \\ $n = 4 \left(2^{\frac{1}{2}}\right)^6 = 4 \times 2^{\frac{6}{2}} = 4 \times 2^3 = 4 \times 8 = 32$.\\
So, the service allows 32 devices after 6 years.\
\\
(c)\
Let $n_1$ be the number of devices after 6 years, and $n_2$ after $m$ more years.\\
$n_2 = 4 \left(2^{\frac{1}{2}} \right)^{6 + m} = 4 \times 2^{\frac{6 + m}{2}}$.\\
We are told that $n_2 = 2n_1$.\\
$4 \times 2^{\frac{6 + m}{2}} = 2 \left(4 \times 2^{\frac{6}{2}}\right)$ \\ 
$4 \times 2^{\frac{6 + m}{2}} = 8 \times 2^3 = 8 \times 8 = 64$ (Wait: but $n_1 = 32$; so $2n_1 = 64$.)\\
So:\
$4 \times 2^{\frac{6 + m}{2}} = 64$ \\ 
$2^{\frac{6 + m}{2}} = 16$ \\ 
But $16 = 2^4$, so:\
$\frac{6 + m}{2} = 4$ \\ 
$6 + m = 8$ \\ 
$m = 2$\\
So, after 2 more years, the number of devices is doubled.

\section*{Question 58}
\textbf{Metadata}

\begin{itemize}
  \item Question ID: O3-STOprUn\_O3-STOprIns\_GPT4.1\_Services\_04
  \item Primary KC: SET | Set operations | union of two sets
  \item Secondary KC: SET | Set operations | intersection of two sets
  \item Topic: Services such as installation, maintenance, repairing, cleaning, laundry, hotel, retail, e-commerce, streaming services, digital services etc.
  \item Grade: Secondary O-level 3/4
\end{itemize}

\textbf{Solution}

Let $A$ be the set of customers who subscribed to music streaming, and $B$ be the set who subscribed to video streaming.

We are given:
\begin{align*}
|A| &= 75 \\
|B| &= 60 \\
|A \cup B| &= 120 \\
|A \cap B| &= 30
\end{align*}

The number of customers who subscribed to only one service is:
\[
(|A| - |A \cap B|) + (|B| - |A \cap B|) = (75 - 30) + (60 - 30) = 45 + 30 = 75.
\]

\textbf{Answer:} $75$ customers subscribed to only one type of service.

\section*{Question 59}
\textbf{Metadata}

\begin{itemize}
  \item Question ID: O3-MXMulSM\_O3-MXSub\_GPT4.1\_Services\_04
  \item Primary KC: MATRICES | Multiplication | product of a scalar quantity and a matrix
  \item Secondary KC: MATRICES | Subtraction | subtraction of matrices
  \item Topic: Services such as installation, maintenance, repairing, cleaning, laundry, hotel, retail, e-commerce, streaming services, digital services etc.
  \item Grade: Secondary O-level 3/4
\end{itemize}

\textbf{Solution}

(a) Multiplying the original staff numbers by the scalar $k = 3$:
\[
3 \times \mathbf{N} = 3 \times \begin{bmatrix} 6 \\ 8 \\ 5 \end{bmatrix} = \begin{bmatrix} 18 \\ 24 \\ 15 \end{bmatrix}
\]
So, after multiplication:
- Hotel A: 18 staff
- Hotel B: 24 staff
- Hotel C: 15 staff

(b) Subtract the reductions for Hotel B and Hotel C:
\[
\text{Final staff matrix} = \begin{bmatrix} 18 \\ 24 \\ 15 \end{bmatrix} - \begin{bmatrix} 0 \\ 2 \\ 3 \end{bmatrix} = \begin{bmatrix} 18 \\ 22 \\ 12 \end{bmatrix}
\]
So, the final number of cleaning staff at each hotel is:
- Hotel A: 18 staff
- Hotel B: 22 staff
- Hotel C: 12 staff


\section*{Question 60}
\textbf{Metadata}

\begin{itemize}
  \item Question ID: O3-MXMulSM\_O3-MXMul\_GPT4.1\_Services\_04
  \item Primary KC: MATRICES | Multiplication | product of a scalar quantity and a matrix
  \item Secondary KC: MATRICES | Multiplication | multiplication of matrices
  \item Topic: Services such as installation, maintenance, repairing, cleaning, laundry, hotel, retail, e-commerce, streaming services, digital services etc.
  \item Grade: Secondary O-level 3/4
\end{itemize}

\textbf{Solution}

(a) Since the customer gives $2$ times as many clothes as indicated, we multiply the matrix $Q$ by the scalar $2$:
\[
A = 2 \times Q = 2 \times \begin{bmatrix}2 & 1 \\ 1 & 3\end{bmatrix} = \begin{bmatrix}4 & 2 \\ 2 & 6\end{bmatrix}
\]
So,
\[
A = \begin{bmatrix}4 & 2 \\ 2 & 6\end{bmatrix}
\]

(b) The total amount to be paid is the sum of products of price and number of corresponding clothes:
\[
\text{Total Payment} = P \circ A 
\]
Here, since the price matrix $P$ and the clothes matrix $A$ match in dimensions, their entry-wise product matrix is:
\[
P \circ A = \begin{bmatrix} 3 & 5 \\ 4 & 6 \end{bmatrix} \circ \begin{bmatrix} 4 & 2 \\ 2 & 6 \end{bmatrix} = \begin{bmatrix} 3 \times 4 & 5 \times 2 \\ 4 \times 2 & 6 \times 6 \end{bmatrix} = \begin{bmatrix} 12 & 10 \\ 8 & 36 \end{bmatrix}
\]

Sum all the entries to get the total cost:
\[
\text{Total Payment} = 12 + 10 + 8 + 36 = 66
\]

Therefore, the total amount to be paid is \textdollar66.

\section*{Question 61}
\textbf{Metadata}

\begin{itemize}
  \item Question ID: O3-MXMul\_O3-MXSub\_GPT4.1\_Services\_04
  \item Primary KC: MATRICES | Multiplication | multiplication of matrices
  \item Secondary KC: MATRICES | Subtraction | subtraction of matrices
  \item Topic: Services such as installation, maintenance, repairing, cleaning, laundry, hotel, retail, e-commerce, streaming services, digital services etc.
  \item Grade: Secondary O-level 3/4
\end{itemize}

\textbf{Solution}

(a) To find the difference matrix $D = B - A$:

\[
D = B - A = \begin{pmatrix}12-10 & 20-18 \\ 5-8 & 15-14 \\ 7-6 & 13-12\end{pmatrix} = \begin{pmatrix}2 & 2 \\ -3 & 1 \\ 1 & 1\end{pmatrix}
\]

(b) To find the additional revenue, we need to multiply $D$ by the transpose of $C$.
First, write $C^T$:

\[
C^T = \begin{pmatrix}25 & 15 & 30 \\ 40 & 25 & 50\end{pmatrix}
\]

Now, calculate $D \times C^T$:

Since $D$ is $3 \times 2$ and $C^T$ is $2 \times 3$, the product will be a $3 \times 3$ matrix.

\[
D \times C^T =
\begin{pmatrix}
2 & 2 \\ -3 & 1 \\ 1 & 1
\end{pmatrix}
\begin{pmatrix}
25 & 15 & 30 \\ 40 & 25 & 50
\end{pmatrix}
\]

Calculate each element:

First row:
- $(2 \times 25) + (2 \times 40) = 50 + 80 = 130$
- $(2 \times 15) + (2 \times 25) = 30 + 50 = 80$
- $(2 \times 30) + (2 \times 50) = 60 + 100 = 160$

Second row:
- $(-3 \times 25) + (1 \times 40) = -75 + 40 = -35$
- $(-3 \times 15) + (1 \times 25) = -45 + 25 = -20$
- $(-3 \times 30) + (1 \times 50) = -90 + 50 = -40$

Third row:
- $(1 \times 25) + (1 \times 40) = 25 + 40 = 65$
- $(1 \times 15) + (1 \times 25) = 15 + 25 = 40$
- $(1 \times 30) + (1 \times 50) = 30 + 50 = 80$

So,
\[
D \times C^T = \begin{pmatrix}130 & 80 & 160 \\ -35 & -20 & -40 \\ 65 & 40 & 80\end{pmatrix}
\]

Each entry represents the additional revenue generated (in dollars) for each combination of task (row) and cost type (column). For example, $130$ is the additional revenue from installations at Basic plan cost.

Therefore, the matrix $D \times C^T$ shows the total additional (or reduced, if negative) revenue for each task and plan due to the change from January to February.

\section*{Question 62}
\textbf{Metadata}

\begin{itemize}
  \item Question ID: O3-SPFndQtl\_O3-SPFndIQR\_GPT4.1\_Services\_04
  \item Primary KC: STATISTICS AND PROBABILITY | Finding quartiles | finding quartiles for a set of data
  \item Secondary KC: STATISTICS AND PROBABILITY | Finding range | finding interquartile range as measures of spread for a set of data 
  \item Topic: Services such as installation, maintenance, repairing, cleaning, laundry, hotel, retail, e-commerce, streaming services, digital services etc.
  \item Grade: Secondary O-level 3/4
\end{itemize}

\textbf{Solution}

(a) First, order is already arranged from smallest to largest.

There are $n = 12$ data points.

Median position (Q$_2$): $\frac{12+1}{2} = 6.5$th value, so average the 6th and 7th values.

$\Rightarrow Q_2 = \frac{17 + 19}{2} = 18$

Lower quartile (Q$_1$) is the median of the lower half (first 6 values: $5, 8, 12, 14, 15, 17$):

Q$_1$ position: average of 3rd and 4th values: $\frac{12 + 14}{2} = 13$

Upper quartile (Q$_3$) is the median of the upper half (last 6 values: $19, 20, 22, 24, 28, 30$):

Q$_3$ position: average of 3rd and 4th values in upper half: $\frac{22 + 24}{2} = 23$

Thus, $Q_1 = 13$, $Q_2 = 18$, $Q_3 = 23$.

(b) The interquartile range (IQR) is:

$IQR = Q_3 - Q_1 = 23 - 13 = 10$.

\textbf{Answers:}
(a) $Q_1 = 13$, $Q_2 = 18$, $Q_3 = 23$

(b) Interquartile range is $10$.

\section*{Question 63}
\textbf{Metadata}

\begin{itemize}
  \item Question ID: O3-SPMulProb\_O2-SPRepPrSE\_GPT4.1\_Services\_04
  \item Primary KC: STATISTICS AND PROBABILITY | Multiplication | multiplication of probabilities
  \item Secondary KC: STATISTICS AND PROBABILITY | Representation and concept | probability of single events
  \item Topic: Services such as installation, maintenance, repairing, cleaning, laundry, hotel, retail, e-commerce, streaming services, digital services etc.
  \item Grade: Secondary O-level 3/4
\end{itemize}

\textbf{Solution}

(a) Since the two events are independent, the probability that both events happen for a single piece of clothing is:

\[
P(\text{damaged and late}) = P(\text{damaged}) \times P(\text{late}) = 0.02 \times 0.05 = 0.001
\]

So, the probability is $0.001$.

(b) The probability that a piece of clothing will either get damaged or the delivery will be late is:

\[
P(\text{damaged or late}) = P(\text{damaged}) + P(\text{late}) - P(\text{damaged and late})
\]
\[
= 0.02 + 0.05 - 0.001 = 0.069
\]

So, the probability is $0.069$. 

\end{document}
