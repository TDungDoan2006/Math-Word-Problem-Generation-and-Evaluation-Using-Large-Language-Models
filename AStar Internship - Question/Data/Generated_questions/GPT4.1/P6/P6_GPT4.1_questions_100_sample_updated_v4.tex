\documentclass{article}
\usepackage[utf8]{inputenc}
\usepackage{amsmath}
\usepackage{amsfonts}
\usepackage{amssymb}
\usepackage{graphicx}
\usepackage{hyperref}
\title{100 P6 samples generated by GPT-4.1 updated v4}
\author{Tien Dung Doan}
\begin{document}
\maketitle
\section*{Question 1}
\textbf{Metadata}

\begin{itemize}
  \item Question ID: P6-FrDivPN\_P5-FrSubMix\_GPT4.1\_Transporation\_05
  \item Primary KC: FRACTIONS | Division | dividing a proper fraction by a whole number
  \item Secondary KC: FRACTIONS | Subtraction | subtracting mixed numbers
  \item Topic: Transporation
  \item Grade: Primary 6
\end{itemize}

\textbf{Question}

A lorry has \( \frac{5}{6} \) of a tank of diesel. The driver needs to share this amount equally among 3 trucks for a delivery trip. 

(a) How much diesel does each truck receive?

After the delivery trip, one truck uses \(2 \frac{1}{4}\) litres of diesel, leaving \(1 \frac{5}{12}\) litres in its tank. 

(b) How much diesel did the truck use for the trip?

\textbf{Solution}

(a) $
\text{Amount of diesel each truck receives} = \frac{5}{6} \div 3 = \frac{5}{6} \times \frac{1}{3} = \frac{5}{18} \text{ of a tank}
$

(b)$
\text{Amount of diesel used} = 2\frac{1}{4} - 1\frac{5}{12}
$
First, convert to improper fractions:
$
2\frac{1}{4} = \frac{9}{4}
$
$
1\frac{5}{12} = \frac{17}{12}
$
Express both fractions with a common denominator (12):
$
\frac{9}{4} = \frac{27}{12}
$
Now subtract:
$
\frac{27}{12} - \frac{17}{12} = \frac{10}{12} = \frac{5}{6} \text{ litres}
$

\textbf{Answer:}
(a) Each truck receives \(\frac{5}{18}\) of a tank of diesel.

(b) The truck used \(\frac{5}{6}\) litres of diesel for the trip.

\section*{Question 2}
\textbf{Metadata}

\begin{itemize}
  \item Question ID: P6-RoFndRoWN\_P4-WNMul4d1d\_GPT4.1\_Transporation\_05
  \item Primary KC: RATIO | Finding ratio | finding the ratio of two or three given whole numbers
  \item Secondary KC: WHOLE NUMBERS | Multiplication | multiplication up to 4 digits by 1 digit or up to 3 digits by 2 digits
  \item Topic: Transporation
  \item Grade: Primary 6
\end{itemize}

\textbf{Question}

In a school, there are 3 shuttle buses used to transport students for an excursion. Bus A has 48 students, Bus B has 60 students, and Bus C has 108 students. 

(a) Find the ratio of the number of students in Bus A to Bus B to Bus C in its simplest form.

(b) If the number of students in Bus B increased by 3 times compared to the original number, what would be the new ratio of Bus A to Bus B to Bus C in simplest form?

\textbf{Solution}

(a) The number of students in Bus A : Bus B : Bus C are 48 : 60 : 108.

To simplify, divide each number by their greatest common factor, which is 12:

$48 \div 12 = 4$
$60 \div 12 = 5$
$108 \div 12 = 9$

So, the simplest ratio is $4:5:9$.

(b) The new number of students in Bus B is $60 \times 3 = 180$.

Now, the numbers are Bus A : Bus B : Bus C = $48 : 180 : 108$.

Divide each number by 12:
$48 \div 12 = 4$
$180 \div 12 = 15$
$108 \div 12 = 9$

So, the new simplest ratio is $4:15:9$.

\section*{Question 3}
\textbf{Metadata}

\begin{itemize}
  \item Question ID: P6-PcFndChg\_P3-WNAdd4d\_GPT4.1\_Manufacturing\_05
  \item Primary KC: PERCENTAGE | Finding change | finding percentage increase/decrease
  \item Secondary KC: WHOLE NUMBERS | Addition | addition up to 4 digits
  \item Topic: Manufacturing
  \item Grade: Primary 6
\end{itemize}

\textbf{Question}

In a factory, 1,250 toy cars were made in January. In February, the factory made 2,450 toy cars. 

(a) How many more toy cars were made in February than in January?

(b) What is the percentage increase in the number of toy cars produced from January to February? 

Give your answer to the nearest whole number.

\textbf{Solution}

(a) Number of toy cars made in February = 2,450

Number of toy cars made in January = 1,250

Number of additional cars made in February = 2,450 - 1,250 = 1,200

(b) Percentage increase = \( \frac{\text{increase}}{\text{original number}} \times 100\% \)

Increase = 1,200

Original number = 1,250 (January's production)

$ \text{Percentage increase} = \frac{1,200}{1,250} \times 100\% = 0.96 \times 100\% = 96\% $

So, the number of toy cars made in February was 1,200 more than in January, and this is a 96\% increase (to the nearest whole number) from January.

\section*{Question 4}
\textbf{Metadata}

\begin{itemize}
  \item Question ID: P6-RoFndRoWN\_P4-WNDiv4d1d\_GPT4.1\_Services\_05
  \item Primary KC: RATIO | Finding ratio | finding the ratio of two or three given whole numbers
  \item Secondary KC: WHOLE NUMBERS | Division | division up to 4 digits by 1 digit
  \item Topic: Services
  \item Grade: Primary 6
\end{itemize}

\textbf{Question}

In a library, there are 936 fiction books and 468 non-fiction books on the shelves. Each library trolley can carry 3 books at a time. How many trolleys are needed to carry all the fiction and non-fiction books together? What is the ratio of the number of fiction books to non-fiction books in its simplest form?

\textbf{Solution}

First, add the number of fiction and non-fiction books:

$
936 + 468 = 1404
$

Next, divide the total number of books by the number of books each trolley can carry:

$
\frac{1404}{3} = 468
$

So, 468 trolleys are needed to carry all the books.

Now, find the ratio of fiction books to non-fiction books:

$
936 : 468
$

Divide both numbers by their greatest common divisor, which is 468:

$
\frac{936}{468} : \frac{468}{468} = 2 : 1
$

Final answers:

- 468 trolleys are needed.
- The simplest ratio of fiction books to non-fiction books is \(2:1\).

\section*{Question 5}
\textbf{Metadata}

\begin{itemize}
  \item Question ID: P6-AgRepLrEx\_P6-AgSmpLrEx\_GPT4.1\_Household finance\_05
  \item Primary KC: ALGEBRA | Representation and concept | translation of simple real-world situations into linear algebraic expressions
  \item Secondary KC: ALGEBRA | Simplifying | simplifying simple linear expressions excluding brackets
  \item Topic: Household finance
  \item Grade: Primary 6
\end{itemize}

\textbf{Question}

Aisha wants to buy some packs of pencils for her home study desk. Each pack of pencils costs $x$ dollars. She has already spent $12$ dollars on erasers. \\ 
(a) Write a linear algebraic expression, in terms of $x$, that represents the total amount Aisha would spend if she buys $4$ packs of pencils. \\ 
(b) Simplify the expression you wrote in part (a).

\textbf{Solution}

(a) The amount Aisha spends on $4$ packs of pencils is $4x$ dollars. The total spent, including the erasers, is $4x + 12$. \\ 
\text{Linear algebraic expression: } 4x + 12 \\ 
(b) Since there are no like terms to combine, the simplified expression is $4x + 12$.

\section*{Question 6}
\textbf{Metadata}

\begin{itemize}
  \item Question ID: P6-FrDivPP\_P4-FrRepSet\_GPT4.1\_Digital ecconomy\_05
  \item Primary KC: FRACTIONS | Division | dividing a whole number/proper fraction by a proper fraction
  \item Secondary KC: FRACTIONS | Representation and concept | fraction as part of a set 
  \item Topic: Digital ecconomy
  \item Grade: Primary 6
\end{itemize}

\textbf{Question}

A group of friends is earning money by selling digital stickers online. They earned a total of $24 from selling a set of stickers. Each sticker costs $\dfrac{3}{4}$ dollars.\
\\
(a) How many digital stickers did they sell?\
\\
(b) If the group wants to split these digital stickers equally among 4 friends, what fraction of the total set does each friend receive?


\textbf{Solution}

(a) To find out how many stickers were sold, divide the total amount earned by the price per sticker:\\
\\
$
\text{Number of stickers} = 24 \div \dfrac{3}{4}
$
\\
Dividing by a fraction is the same as multiplying by its reciprocal:\\
$
24 \div \dfrac{3}{4} = 24 \times \dfrac{4}{3} = \dfrac{24 \times 4}{3} = \dfrac{96}{3} = 32
$
\\
So, they sold **32 digital stickers**.\\
\\
(b) If the 32 stickers are shared equally among 4 friends, each friend gets:\\
$
32 \div 4 = 8
$
\\
To express this as a fraction of the total set:\\
$
\dfrac{8}{32} = \dfrac{1}{4}
$
\\
**Each friend receives \( \dfrac{1}{4} \) of the total set of digital stickers.**


\section*{Question 7}
\textbf{Metadata}

\begin{itemize}
  \item Question ID: P6-FrDivPN\_P5-FrSubMix\_GPT4.1\_Manufacturing\_05
  \item Primary KC: FRACTIONS | Division | dividing a proper fraction by a whole number
  \item Secondary KC: FRACTIONS | Subtraction | subtracting mixed numbers
  \item Topic: Manufacturing
  \item Grade: Primary 6
\end{itemize}

\textbf{Question}


A factory produces \( 3\frac{1}{2} \) metres of fabric each hour. In one day, the factory needs to use \( \frac{3}{4} \) of a metre of fabric to make each bag.

(a) If the factory wants to pack all of today's fabric equally into 5 boxes, how many metres of fabric will each box have?

(b) After packing, the manager realises that \(1\frac{1}{4}\) metres has already been used for samples. How much fabric is left in each box after the samples have been subtracted?


\textbf{Solution}


(a) Total fabric produced in one hour is \( 3\frac{1}{2} = \frac{7}{2} \) metres.

If this fabric is packed equally into 5 boxes, the amount in each box is:

$
\frac{7}{2} \div 5 = \frac{7}{2} \times \frac{1}{5} = \frac{7}{10}
$

So, each box has \(\frac{7}{10}\) metres of fabric.

(b) If \(1\frac{1}{4}\) metres is used for samples, subtract this from the amount in each box:

First, convert \(1\frac{1}{4}\) to an improper fraction:
$
1\frac{1}{4} = \frac{5}{4}
$

Subtract from \(\frac{7}{10}\) (find a common denominator, which is 20):

$
\frac{7}{10} = \frac{14}{20}, \quad \frac{5}{4} = \frac{25}{20}
$
$
\frac{14}{20} - \frac{25}{20} = -\frac{11}{20}
$

Since \(-\frac{11}{20}\) is negative, it means there is not enough fabric in each box after the samples are taken; each box is short of \(\frac{11}{20}\) metre.


\section*{Question 8}
\textbf{Metadata}

\begin{itemize}
  \item Question ID: P6-RoFndRoWN\_P3-WNAdd4d\_GPT4.1\_Farming\_05
  \item Primary KC: RATIO | Finding ratio | finding the ratio of two or three given whole numbers
  \item Secondary KC: WHOLE NUMBERS | Addition | addition up to 4 digits
  \item Topic: Farming
  \item Grade: Primary 6
\end{itemize}

\textbf{Question}

A farmer harvested 1,250 carrots, 950 potatoes, and 800 radishes from his farm.

(a) Find the total number of vegetables the farmer harvested.

(b) Find the ratio of the number of carrots to the number of potatoes to the number of radishes. Express your answer in its simplest form.

\textbf{Solution}

\textbf{(a) Finding the total number of vegetables:}

$1,250 + 950 + 800 = 3,000$

\textbf{Answer:} The farmer harvested a total of $3,000$ vegetables.

\textbf{(b) Finding the required ratio:}

The numbers are:

\begin{align*}
\text{Carrots} &= 1,250 \\
\text{Potatoes} &= 950 \\
\text{Radishes} &= 800
\end{align*}

So, the ratio is $1,250 : 950 : 800$.

Find the highest common factor for all three numbers. Let's check:
- $1,250$ is divisible by $50$ ($1,250 \div 50 = 25$)
- $950$ is divisible by $50$ ($950 \div 50 = 19$)
- $800$ is divisible by $50$ ($800 \div 50 = 16$)

Divide all by $50$:

$1,250 \div 50 : 950 \div 50 : 800 \div 50 = 25 : 19 : 16$

\textbf{Answer:} The ratio of the number of carrots to potatoes to radishes is $25 : 19 : 16$.

\section*{Question 9}
\textbf{Metadata}

\begin{itemize}
  \item Question ID: P6-RoFndRoWN\_P6-RoSmpWN\_GPT4.1\_Transporation\_05
  \item Primary KC: RATIO | Finding ratio | finding the ratio of two or three given whole numbers
  \item Secondary KC: RATIO | Simplifying | expressing a ratio in its simplest form
  \item Topic: Transporation
  \item Grade: Primary 6
\end{itemize}

\textbf{Question}

In a bus, there are 24 adults, 16 children, and 12 senior citizens. 

(a) Find the ratio of adults to children to senior citizens on the bus.

(b) Express this ratio in its simplest form.

\textbf{Solution}

Let us list the numbers:

Number of adults = 24 

Number of children = 16 

Number of senior citizens = 12

(a) The ratio of adults to children to senior citizens is $24 : 16 : 12$.

(b) To simplify $24 : 16 : 12$, we find the highest common factor (HCF) of the three numbers:

The HCF of 24, 16, and 12 is 4.

Divide each term by 4:

$24 \div 4 = 6$

$16 \div 4 = 4$

$12 \div 4 = 3$

So, the ratio in simplest form is $6 : 4 : 3$.

\section*{Question 10}
\textbf{Metadata}

\begin{itemize}
  \item Question ID: P6-PcFndWN\_P3-WNSub4d\_GPT4.1\_Leisure\_05
  \item Primary KC: PERCENTAGE | Finding the whole | finding the whole given a part and the percentage
  \item Secondary KC: WHOLE NUMBERS | Subtraction | subtraction up to 4 digits
  \item Topic: Leisure
  \item Grade: Primary 6
\end{itemize}

\textbf{Question}

In a bowling competition, Sarah scored 25\% of the total points she could get in the first round, which was 130 points. After the second round, she subtracted 48 points from her total possible score, as she missed one game. If her final possible score is now 472 points, what was the original total number of points Sarah could have achieved before the score was reduced?

\textbf{Solution}

Let the original total number of points Sarah could have achieved be $x$.

From the first round:

Sarah scored 25\% of $x$ which equals 130.

So: 

$$
0.25x = 130
$$

$$
x = \frac{130}{0.25}
$$

$$
x = 520
$$

After the second round, she missed a game, so 48 points were subtracted:

$$
\text{Final possible score} = x - 48
$$

We are told the final possible score is now 472:

$$
x - 48 = 472
$$

$$
x = 472 + 48 = 520
$$

Therefore, Sarah's original total possible score was \boxed{520} points.

\section*{Question 11}
\textbf{Metadata}

\begin{itemize}
  \item Question ID: P6-FrDivPP\_P3-FrSmp\_GPT4.1\_Household finance\_05
  \item Primary KC: FRACTIONS | Division | dividing a whole number/proper fraction by a proper fraction
  \item Secondary KC: FRACTIONS | Simplifying | expressing a fraction in its simplest form
  \item Topic: Household finance
  \item Grade: Primary 6
\end{itemize}

\textbf{Question}

Auntie Siti has $3$ litres of orange juice. She wants to pour the juice equally into bottles, each that can hold $\dfrac{3}{4}$ litre. 

(a) How many bottles can she fill completely?

(b) Write your answer as a simplified fraction if necessary.

\textbf{Solution}

\textbf{Solution:}

(a) The number of bottles filled completely is:

$$
\text{Number of bottles} = \frac{3}{\frac{3}{4}} = 3 \div \frac{3}{4}
$$

Dividing by a fraction is the same as multiplying by its reciprocal:

$$
3 \div \frac{3}{4} = 3 \times \frac{4}{3} = \frac{12}{3} = 4
$$

\textbf{Answer:}\ Auntie Siti can fill 4 bottles completely.

(b) As the answer is $4$, it is already in its simplest form.

\textbf{Final Answer:} $4$ bottles.

\section*{Question 12}
\textbf{Metadata}

\begin{itemize}
  \item Question ID: P6-FrDivPN\_P5-FrAddMix\_GPT4.1\_Transporation\_05
  \item Primary KC: FRACTIONS | Division | dividing a proper fraction by a whole number
  \item Secondary KC: FRACTIONS | Addition | adding mixed numbers
  \item Topic: Transporation
  \item Grade: Primary 6
\end{itemize}

\textbf{Question}

A bus has to travel a total distance of $\dfrac{3}{4}$ kilometre each trip. On Monday, the bus makes 5 trips. The total distance travelled by the bus driver on Monday and Tuesday is $3\dfrac{1}{2}$ kilometres more than the distance travelled on Monday. If the total distance travelled on both days is equally shared among 7 drivers, how much distance does each driver travel?

\textbf{Solution}

Distance travelled on Monday for 5 trips: $\dfrac{3}{4} \times 5 = \dfrac{15}{4} = 3\dfrac{3}{4}$ km$
$
Distance travelled on both days: $3\dfrac{3}{4} + 3\dfrac{1}{2} = (3 + 3) + (\dfrac{3}{4} + \dfrac{1}{2}) = 6 + (\dfrac{3}{4} + \dfrac{2}{4}) = 6 + \dfrac{5}{4} = 7\dfrac{1}{4}$ km$
$
Total distance travelled on both days: $3\dfrac{3}{4} + 7\dfrac{1}{4} = (3 + 7) + (\dfrac{3}{4} + \dfrac{1}{4}) = 10 + 1 = 11$ km$
$
But the correct sum is: Monday: $3\dfrac{3}{4}$ km, Both days: $3\dfrac{3}{4} + 3\dfrac{1}{2} = 7\dfrac{1}{4}$ km

Now, the total distance $7\dfrac{1}{4}$ km is shared among 7 drivers:

$7\dfrac{1}{4} \div 7 = \dfrac{29}{4} \div 7 = \dfrac{29}{4} \times \dfrac{1}{7} = \dfrac{29}{28} = 1\dfrac{1}{28}$ km

**Each driver travels $1\dfrac{1}{28}$ km.**

\section*{Question 13}
\textbf{Metadata}

\begin{itemize}
  \item Question ID: P6-PcFndChg\_P4-WNDiv4d1d\_GPT4.1\_Leisure\_05
  \item Primary KC: PERCENTAGE | Finding change | finding percentage increase/decrease
  \item Secondary KC: WHOLE NUMBERS | Division | division up to 4 digits by 1 digit
  \item Topic: Leisure
  \item Grade: Primary 6
\end{itemize}

\textbf{Question}

A group of 2,496 visitors visited a science centre over the course of a week. Compared to the previous week, the number of visitors increased by 20\%. If the increase in the number of visitors was shared equally over 4 special exhibitions, how many more visitors did each exhibition have compared to last week?

\textbf{Solution}

First, let the number of visitors last week be $x$. This week, the number of visitors increased by 20\%, so:

$
2,496 = x + 0.2x = 1.2x
$
$
x = \frac{2,496}{1.2} = 2,080
$
So, the increase in the number of visitors is:
$
2,496 - 2,080 = 416
$
This increase is shared equally over 4 exhibitions:
$
\text{Number of additional visitors per exhibition} = \frac{416}{4} = 104
$

\textbf{Answer:} Each exhibition had 104 more visitors compared to last week.

\section*{Question 14}
\textbf{Metadata}

\begin{itemize}
  \item Question ID: P6-FrDivPP\_P3-FrSmp\_GPT4.1\_Leisure\_05
  \item Primary KC: FRACTIONS | Division | dividing a whole number/proper fraction by a proper fraction
  \item Secondary KC: FRACTIONS | Simplifying | expressing a fraction in its simplest form
  \item Topic: Leisure
  \item Grade: Primary 6
\end{itemize}

\textbf{Question}

Su Lin has $3$ hours to read as many books as she can. Each book takes her $\frac{3}{4}$ of an hour to finish. How many books can Su Lin read completely in $3$ hours? Express your answer as a simplified fraction.

\textbf{Solution}

To find out how many books Su Lin can read, divide the total time by the time needed for each book:
$
\text{Number of books} = \frac{3}{\frac{3}{4}} = 3 \times \frac{4}{3} = 4
$
Su Lin can read $4$ books completely in $3$ hours.

\section*{Question 15}
\textbf{Metadata}

\begin{itemize}
  \item Question ID: P6-PcFndChg\_P4-WNDiv4d1d\_GPT4.1\_Education\_05
  \item Primary KC: PERCENTAGE | Finding change | finding percentage increase/decrease
  \item Secondary KC: WHOLE NUMBERS | Division | division up to 4 digits by 1 digit
  \item Topic: Education
  \item Grade: Primary 6
\end{itemize}

\textbf{Question}

A school had 864 students in 2022. In 2023, the number of students increased to 972. 

(a) By how many students did the school population increase from 2022 to 2023? 

(b) What is the percentage increase in the number of students from 2022 to 2023? 

(c) If the students in 2023 were divided equally into 6 classes, how many students were there in each class?

\textbf{Solution}

Let us solve each part step by step.

(a) Number of students increased $= 972 - 864 = 108$

(b) Percentage increase $= \left(\dfrac{108}{864} \times 100\%\right) = \left(\dfrac{108 \times 100}{864}\right)\% = \left(\dfrac{10800}{864}\right)\% = 12.5\%$

(c) Number of students in each class $= \dfrac{972}{6} = 162$

\textbf{Answers:}

(a) 108 students

(b) 12.5\%

(c) 162 students in each class

\section*{Question 16}
\textbf{Metadata}

\begin{itemize}
  \item Question ID: P6-FrDivPN\_P3-FrSmp\_GPT4.1\_Farming\_05
  \item Primary KC: FRACTIONS | Division | dividing a proper fraction by a whole number
  \item Secondary KC: FRACTIONS | Simplifying | expressing a fraction in its simplest form
  \item Topic: Farming
  \item Grade: Primary 6
\end{itemize}

\textbf{Question}

A farmer harvested $\frac{5}{8}$ of a bag of corn and wants to share it equally among 4 workers. How much of the bag of corn does each worker get? Express your answer in its simplest form.

\textbf{Solution}

Each worker gets $\frac{5}{8} \div 4$ of a bag of corn.

To divide $\frac{5}{8}$ by 4:

$\frac{5}{8} \div 4 = \frac{5}{8} \times \frac{1}{4} = \frac{5}{32}$

$\frac{5}{32}$ is already in its simplest form.

\textbf{Answer:} Each worker gets $\frac{5}{32}$ of a bag of corn.

\section*{Question 17}
\textbf{Metadata}

\begin{itemize}
  \item Question ID: P6-PcFndWN\_P3-WNAdd4d\_GPT4.1\_Farming\_05
  \item Primary KC: PERCENTAGE | Finding the whole | finding the whole given a part and the percentage
  \item Secondary KC: WHOLE NUMBERS | Addition | addition up to 4 digits
  \item Topic: Farming
  \item Grade: Primary 6
\end{itemize}

\textbf{Question}

A farmer harvested some carrots in his farm. He sold $35\%$ of his carrots and had $420$ carrots left. Later, he harvested an additional $250$ carrots. What was the total number of carrots the farmer had after the second harvest?

\textbf{Solution}

Let the original number of carrots be $x$. After selling $35\%$, the farmer has $65\%$ left, which is $420$ carrots.

So,
$0.65x = 420$
$\Rightarrow x = \frac{420}{0.65}$
$\Rightarrow x = 646.15$

Since the number of carrots must be a whole number, $x = 646$ (to the nearest whole number).

After harvesting $250$ more carrots, the total number of carrots is:
$420$ (left after selling) $+ 250 = 670$

\textbf{Therefore, the farmer had a total of $670$ carrots after the second harvest.}

\section*{Question 18}
\textbf{Metadata}

\begin{itemize}
  \item Question ID: P6-RoFndRoWN\_P4-WNDiv4d1d\_GPT4.1\_Transporation\_05
  \item Primary KC: RATIO | Finding ratio | finding the ratio of two or three given whole numbers
  \item Secondary KC: WHOLE NUMBERS | Division | division up to 4 digits by 1 digit
  \item Topic: Transporation
  \item Grade: Primary 6
\end{itemize}

\textbf{Question}

In a car park, there are 2,184 cars and 546 motorcycles. The total number of vehicles is equally divided among 3 different entrances. What is the ratio of the number of cars to the number of motorcycles at one entrance? Express your answer in its simplest form.

\textbf{Solution}

First, find the total number of cars and motorcycles each at one entrance:

$
\text{Number of cars at one entrance} = \frac{2,184}{3} = 728
$
$
\text{Number of motorcycles at one entrance} = \frac{546}{3} = 182
$

Now, find the ratio of the number of cars to the number of motorcycles at one entrance:

$
\text{Ratio of cars to motorcycles} = 728 : 182
$

Next, simplify the ratio by dividing both numbers by their greatest common divisor (GCD). The GCD of 728 and 182 is 182.

$
\frac{728}{182} = 4, \qquad \frac{182}{182} = 1
$

Therefore, the ratio in its simplest form is:

$
\boxed{4:1}
$

So, at each entrance, the ratio of the number of cars to the number of motorcycles is \(4:1\).

\section*{Question 19}
\textbf{Metadata}

\begin{itemize}
  \item Question ID: P6-FrDivPN\_P2-FrCmp\_GPT4.1\_Food\_05
  \item Primary KC: FRACTIONS | Division | dividing a proper fraction by a whole number
  \item Secondary KC: FRACTIONS | Comparison and ordering | comparing and ordering fractions
  \item Topic: Food
  \item Grade: Primary 6
\end{itemize}

\textbf{Question}

Amy baked $3$ identical cakes and decided to share them with her friends. She divided the first cake into $\frac{3}{4}$ and gave it equally to $2$ friends. She divided the second cake into $\frac{5}{6}$ and gave it equally to $5$ friends. She divided the third cake into $\frac{4}{5}$ and gave it equally to $4$ friends. 

Among the three groups of friends, who received the largest share of cake from Amy? Arrange the shares each friend received in order from smallest to largest.

\textbf{Solution}

Let us calculate the share each friend received from their group:

\begin{itemize}
  \item First cake: $\frac{3}{4} \div 2 = \frac{3}{4} \times \frac{1}{2} = \frac{3}{8}$ per friend
  \item Second cake: $\frac{5}{6} \div 5 = \frac{5}{6} \times \frac{1}{5} = \frac{1}{6}$ per friend
  \item Third cake: $\frac{4}{5} \div 4 = \frac{4}{5} \times \frac{1}{4} = \frac{1}{5}$ per friend
\end{itemize}

Now, compare $\frac{3}{8}$, $\frac{1}{6}$ and $\frac{1}{5}$:

Find their decimal values:

\begin{itemize}
  \item $\frac{3}{8} = 0.375$
  \item $\frac{1}{6} \approx 0.1667$
  \item $\frac{1}{5} = 0.2$
\end{itemize}

Arranging from smallest to largest:

$
\frac{1}{6} < \frac{1}{5} < \frac{3}{8}
$

Thus, each friend in the first group (who shared $\frac{3}{4}$ between $2$ friends) received the largest share: $\frac{3}{8}$.

\textbf{Order from smallest to largest:} $\frac{1}{6}$, $\frac{1}{5}$, $\frac{3}{8}$.

\textbf{The friends who received the largest share are those who shared the first cake.}

\section*{Question 20}
\textbf{Metadata}

\begin{itemize}
  \item Question ID: P6-FrDivPN\_P3-FrSmp\_GPT4.1\_Manufacturing\_05
  \item Primary KC: FRACTIONS | Division | dividing a proper fraction by a whole number
  \item Secondary KC: FRACTIONS | Simplifying | expressing a fraction in its simplest form
  \item Topic: Manufacturing
  \item Grade: Primary 6
\end{itemize}

\textbf{Question}

In a bakery, Mdm Tan has \(\frac{5}{6}\) of a tray of cookies left. She wants to pack this amount equally into 3 gift boxes for her customers. 

How much of a tray of cookies will each gift box contain? Express your answer in its simplest form.

\textbf{Solution}

Total amount of cookies to be divided: \(\frac{5}{6}\) of a tray.

Number of gift boxes: 3.

Each box will contain: 
$
\frac{5}{6} \div 3 = \frac{5}{6} \times \frac{1}{3} = \frac{5}{18}
$

Each gift box will contain \(\frac{5}{18}\) of a tray of cookies.

\section*{Question 21}
\textbf{Metadata}

\begin{itemize}
  \item Question ID: P6-PcFndWN\_P4-WNMul4d1d\_GPT4.1\_Transporation\_05
  \item Primary KC: PERCENTAGE | Finding the whole | finding the whole given a part and the percentage
  \item Secondary KC: WHOLE NUMBERS | Multiplication | multiplication up to 4 digits by 1 digit or up to 3 digits by 2 digits
  \item Topic: Transporation
  \item Grade: Primary 6
\end{itemize}

\textbf{Question}

A bus can carry a certain number of passengers when it is full. On Monday morning, the bus was 40\% full, and there were 48 passengers onboard. If there are 7 such buses operating together and they are all 40\% full, 
\begin{enumerate}
    \item What is the maximum number of passengers one bus can carry when it is full?
    \item How many passengers are there in total on all 7 buses?
\end{enumerate}

\textbf{Solution}

Let the maximum number of passengers one bus can carry be $x$. 

From the problem, 40\% of $x$ is 48:

$0.4x = 48$

To find $x$:
$
x = \frac{48}{0.4} = 120
$

So, the maximum number of passengers one bus can carry is \textbf{120}. 

If there are 7 buses, each carrying 48 passengers:
$
48 \times 7 = 336
$

The total number of passengers on all 7 buses is \textbf{336}.

\section*{Question 22}
\textbf{Metadata}

\begin{itemize}
  \item Question ID: P6-PcFndChg\_P4-WNDiv4d1d\_GPT4.1\_Transporation\_05
  \item Primary KC: PERCENTAGE | Finding change | finding percentage increase/decrease
  \item Secondary KC: WHOLE NUMBERS | Division | division up to 4 digits by 1 digit
  \item Topic: Transporation
  \item Grade: Primary 6
\end{itemize}

\textbf{Question}

A train from Singapore to Kuala Lumpur had 2,376 passengers last month. This month, the number of passengers increased by 25\%. The passengers are equally distributed among 4 train carriages. 

(a) How many passengers took the train this month?

(b) How many passengers were seated in each carriage this month?

\textbf{Solution}

(a) Number of passengers this month:

$
\text{Increase} = 25\% \times 2,376 = \frac{25}{100} \times 2,376 = 594
$
$
\text{Total passengers this month} = 2,376 + 594 = 2,970
$

(b) Number of passengers in each carriage:

$
\text{Number per carriage} = \frac{2,970}{4} = 742\text{ R }2
$

Each carriage had 742 passengers, and there were 2 extra passengers.

\section*{Question 23}
\textbf{Metadata}

\begin{itemize}
  \item Question ID: P6-FrDivPN\_P5-FrCnv2Dc\_GPT4.1\_Digital ecconomy\_05
  \item Primary KC: FRACTIONS | Division | dividing a proper fraction by a whole number
  \item Secondary KC: FRACTIONS | Conversion to decimals | expressing fractions as decimals
  \item Topic: Digital ecconomy
  \item Grade: Primary 6
\end{itemize}

\textbf{Question}

A digital artist earns \(\frac{3}{4}\) of a dollar each time someone downloads her artwork. If 5 people download her artwork, how much money does she earn per person, in decimal form?

\textbf{Solution}

To find how much the artist earns per person, we need to divide \(\frac{3}{4}\) by 5.\\

$
\frac{3}{4} \div 5 = \frac{3}{4} \times \frac{1}{5} = \frac{3}{20}
$

Now, express \(\frac{3}{20}\) as a decimal:\\

$
\frac{3}{20} = 0.15
$

\textbf{Answer:} The artist earns \(0.15\) dollars, or 15 cents, per person.

\section*{Question 24}
\textbf{Metadata}

\begin{itemize}
  \item Question ID: P6-FrDivPN\_P5-FrMulMixN\_GPT4.1\_Services\_05
  \item Primary KC: FRACTIONS | Division | dividing a proper fraction by a whole number
  \item Secondary KC: FRACTIONS | Multiplication | multiplying a mixed number and a whole number
  \item Topic: Services
  \item Grade: Primary 6
\end{itemize}

\textbf{Question}

A bakery has $\frac{3}{4}$ of a cake left after a busy day. The baker wants to give this amount equally to 3 staff. 

(a) How much of the cake does each staff receive?

On another day, the bakery prepares $2\frac{1}{2}$ kg of batter for each cake and bakes 4 cakes. 

(b) What is the total amount of batter used for 4 cakes?

\textbf{Solution}

a) Amount of cake each staff receives:

Amount of cake to share: $\frac{3}{4}$
Number of staff: $3$

Amount for each staff:
$\frac{3}{4} \div 3 = \frac{3}{4} \times \frac{1}{3} = \frac{3}{12} = \frac{1}{4}$

Each staff receives $\boxed{\frac{1}{4}}$ of the cake.

b) Total amount of batter for 4 cakes:

Amount of batter for each cake: $2\frac{1}{2} = \frac{5}{2}$ kg
Number of cakes: $4$

Total batter used:
$\frac{5}{2} \times 4 = \frac{5 \times 4}{2} = \frac{20}{2} = 10$ kg

So, the total amount of batter used for 4 cakes is $\boxed{10}$ kg.

\section*{Question 25}
\textbf{Metadata}

\begin{itemize}
  \item Question ID: P6-FrDivPN\_P2-FrCmp\_GPT4.1\_Digital ecconomy\_05
  \item Primary KC: FRACTIONS | Division | dividing a proper fraction by a whole number
  \item Secondary KC: FRACTIONS | Comparison and ordering | comparing and ordering fractions
  \item Topic: Digital ecconomy
  \item Grade: Primary 6
\end{itemize}

\textbf{Question}

A group of friends decided to share 2/3 of a cryptocurrency token equally among 4 people. 

After dividing the token, each friend compares their share with another group who split 3/5 of a similar token equally among 5 people. 

(a) What fraction of a token does each friend in the first group receive?  
(b) What fraction does each friend in the second group receive? 
(c) Who receives a bigger share, and by how much? Write your answers in the simplest form.

\textbf{Solution}

Let us solve each part step-by-step.  

(a) Each friend in the first group receives:  
$ \frac{2}{3} \div 4 = \frac{2}{3} \times \frac{1}{4} = \frac{2}{12} = \frac{1}{6} $

(b) Each friend in the second group receives:  
$ \frac{3}{5} \div 5 = \frac{3}{5} \times \frac{1}{5} = \frac{3}{25} $

(c) To compare \( \frac{1}{6} \) and \( \frac{3}{25} \), express both with a common denominator:

$ \text{LCM of 6 and 25 is 150} $
$ \frac{1}{6} = \frac{25}{150} $
$ \frac{3}{25} = \frac{18}{150} $

\( \frac{1}{6} = \frac{25}{150} \) is greater than \( \frac{3}{25} = \frac{18}{150} \).

Difference:  
$ \frac{25}{150} - \frac{18}{150} = \frac{7}{150} $

\textbf{Answer:}

(a) Each friend in the first group receives \( \frac{1}{6} \) of a token.  
(b) Each friend in the second group receives \( \frac{3}{25} \) of a token.  
(c) The first group receives a bigger share by \( \frac{7}{150} \) of a token.

\section*{Question 26}
\textbf{Metadata}

\begin{itemize}
  \item Question ID: P6-FrDivPP\_P5-FrCnv2Dc\_GPT4.1\_Household finance\_05
  \item Primary KC: FRACTIONS | Division | dividing a whole number/proper fraction by a proper fraction
  \item Secondary KC: FRACTIONS | Conversion to decimals | expressing fractions as decimals
  \item Topic: Household finance
  \item Grade: Primary 6
\end{itemize}

\textbf{Question}

\textbf{Word Problem:} 

Adeline has $3$ litres of orange juice. She wants to pour the orange juice equally into bottles, each holding $\frac{3}{5}$ litre. 

(a) How many bottles can she fill completely? 

(b) Express the amount of orange juice in each bottle as a decimal, correct to $2$ decimal places.

\textbf{Solution}

\textbf{Solution:}

(a) To find how many bottles she can fill, divide the total volume by the volume of each bottle:
\$
\text{Number of bottles} = 3 \div \frac{3}{5}
\$
This is the same as:
\$
= 3 \times \frac{5}{3} = 1 \times 5 = 5
\$
So, Adeline can fill \boxed{5} bottles completely.

(b) The capacity of each bottle is $\frac{3}{5}$ litre.

To express $\frac{3}{5}$ as a decimal:
\$
\frac{3}{5} = 0.6
\$
Correct to $2$ decimal places, $0.60$.

\textbf{Answer:}
(a) She can fill 5 bottles completely.

(b) Each bottle holds $0.60$ litre (correct to 2 decimal places).

\section*{Question 27}
\textbf{Metadata}

\begin{itemize}
  \item Question ID: P6-FrDivPN\_P6-FrDivPP\_GPT4.1\_Leisure\_05
  \item Primary KC: FRACTIONS | Division | dividing a proper fraction by a whole number
  \item Secondary KC: FRACTIONS | Division | dividing a whole number/proper fraction by a proper fraction
  \item Topic: Leisure
  \item Grade: Primary 6
\end{itemize}

\textbf{Question}

A group of friends decided to share some chocolate bars during a movie night. Each chocolate bar can be split equally into $\frac{3}{4}$ of a bar portions for snacks.

(a) If they have $2$ whole chocolate bars and want to divide them so that each portion is $\frac{3}{4}$ of a bar, how many portions can they make?

(b) Later, they decide to further divide each $\frac{3}{4}$ portion equally among $3$ people. How much of a chocolate bar does each person get?


\textbf{Solution}

(a) To find how many $\frac{3}{4}$ bar portions are in $2$ whole bars, we divide:

$$
2 \div \frac{3}{4} = 2 \times \frac{4}{3} = \frac{8}{3}
$$

This means they can make $\frac{8}{3}$ portions. Changing this to mixed number:

$
\frac{8}{3} = 2\frac{2}{3}
$

So they can make $2$ full portions and $\frac{2}{3}$ of another portion.

(b) Now, each $\frac{3}{4}$ portion is divided equally among $3$ people, so each person gets:

$$
\frac{3}{4} \div 3 = \frac{3}{4} \times \frac{1}{3} = \frac{1}{4}
$$

Therefore, each person gets $\frac{1}{4}$ of a chocolate bar from each $\frac{3}{4}$ portion that is shared.


\section*{Question 28}
\textbf{Metadata}

\begin{itemize}
  \item Question ID: P6-RoFndRoWN\_P4-WNDiv4d1d\_GPT4.1\_Leisure\_05
  \item Primary KC: RATIO | Finding ratio | finding the ratio of two or three given whole numbers
  \item Secondary KC: WHOLE NUMBERS | Division | division up to 4 digits by 1 digit
  \item Topic: Leisure
  \item Grade: Primary 6
\end{itemize}

\textbf{Question}

A group of 864 children went to a swimming carnival. There were 3 children in every team. The organiser separated the teams into three different pools: Pool A, Pool B, and Pool C. Pool A had 24 teams, Pool B had twice as many teams as Pool A, and the rest of the teams were in Pool C. Find the ratio of the number of children in Pool A to Pool B to Pool C.

\textbf{Solution}

Total number of children = 864
Number of children in each team = 3
Total number of teams = 864 \div 3 = 288
Number of teams in Pool A = 24
Number of teams in Pool B = 2 \times 24 = 48
Number of teams in Pool C = 288 - 24 - 48 = 216
Number of children in Pool A = 24 \times 3 = 72
Number of children in Pool B = 48 \times 3 = 144
Number of children in Pool C = 216 \times 3 = 648
The ratio of children in Pool A : Pool B : Pool C = 72 : 144 : 648
Divide all numbers by 72 to simplify:
72 \div 72 : 144 \div 72 : 648 \div 72 = 1 : 2 : 9
\textbf{Final answer:} The ratio of the number of children in Pool A to Pool B to Pool C is $1 : 2 : 9$.

\section*{Question 29}
\textbf{Metadata}

\begin{itemize}
  \item Question ID: P6-AgRepLrEx\_P6-AgSmpLrEx\_GPT4.1\_Sports\_05
  \item Primary KC: ALGEBRA | Representation and concept | translation of simple real-world situations into linear algebraic expressions
  \item Secondary KC: ALGEBRA | Simplifying | simplifying simple linear expressions excluding brackets
  \item Topic: Sports
  \item Grade: Primary 6
\end{itemize}

\textbf{Question}

In a sports club, Eugene pays \$3 for every badminton session he attends. Let $n$ represent the number of badminton sessions he attends in one month. He also pays a one-time registration fee of \$12. 

Write down an algebraic expression to represent the total amount, in dollars, Eugene pays in one month if he attends $n$ badminton sessions. Then, simplify your expression.

\textbf{Solution}

Let the total amount Eugene pays in one month be represented by the expression:

$$
3n + 12
$$

This shows that for every session ($n$), Eugene pays \$3, and there is a one-time registration fee of \$12. 

Since there are no like terms to combine, the expression $3n + 12$ is already simplified.

\section*{Question 30}
\textbf{Metadata}

\begin{itemize}
  \item Question ID: P6-FrDivPP\_P5-FrSubMix\_GPT4.1\_Food\_05
  \item Primary KC: FRACTIONS | Division | dividing a whole number/proper fraction by a proper fraction
  \item Secondary KC: FRACTIONS | Subtraction | subtracting mixed numbers
  \item Topic: Food
  \item Grade: Primary 6
\end{itemize}

\textbf{Question}

A baker has $5$ loaves of bread. He wants to give each customer $\frac{2}{3}$ of a loaf. 

(a) How many customers can he serve with his $5$ loaves of bread?

After serving these customers, he decides to use $2\dfrac{1}{2}$ loaves to make sandwiches. 

(b) How many loaves of bread does he have left after using some for sandwiches?

\textbf{Solution}

(a) To find how many customers he can serve:

Number of customers $= \dfrac{5}{\frac{2}{3}} = 5 \times \dfrac{3}{2} = \dfrac{15}{2} = 7\dfrac{1}{2}$.

So, the baker can serve $7$ customers fully, and the $8$th customer will get half of the required amount.

(b) To find how many loaves are left after using some for sandwiches:

After giving bread to customers, the baker uses $2\dfrac{1}{2}$ loaves for sandwiches.
He started with $5$ loaves. He subtracts $2\dfrac{1}{2}$ loaves:

$5 - 2\dfrac{1}{2} = \dfrac{10}{2} - \dfrac{5}{2} = \dfrac{5}{2}$ loaves left.

Thus, the baker has $2\dfrac{1}{2}$ loaves of bread left.

\section*{Question 31}
\textbf{Metadata}

\begin{itemize}
  \item Question ID: P6-PcFndChg\_P3-WNSub4d\_GPT4.1\_Food\_05
  \item Primary KC: PERCENTAGE | Finding change | finding percentage increase/decrease
  \item Secondary KC: WHOLE NUMBERS | Subtraction | subtraction up to 4 digits
  \item Topic: Food
  \item Grade: Primary 6
\end{itemize}

\textbf{Question}

Sarah went to the supermarket and bought a packet of rice last month for \$2450. This month, the same packet of rice costs \$2100. What is the percentage decrease in the price of the rice packet?

\textbf{Solution}

First, we find the change in price:

$
\text{Change in price} = 2450 - 2100 = 350
$

Next, we calculate the percentage decrease:

$
\text{Percentage decrease} = \frac{\text{Change in price}}{\text{Original price}} \times 100\% = \frac{350}{2450} \times 100\% 
$

$
= \frac{350 \times 100}{2450}\%
$

$
= \frac{35000}{2450}\% = 14.3\% \text{ (rounded to 1 decimal place)}
$

\text{The percentage decrease in the price of the rice packet is } 14.3\%.

\section*{Question 32}
\textbf{Metadata}

\begin{itemize}
  \item Question ID: P6-AgSlvLrN\_P6-AgRepLrEx\_GPT4.1\_Manufacturing\_05
  \item Primary KC: ALGEBRA | Solving simple linear equations | solving simple linear equations involving whole number coefficient only
  \item Secondary KC: ALGEBRA | Representation and concept | translation of simple real-world situations into linear algebraic expressions
  \item Topic: Manufacturing
  \item Grade: Primary 6
\end{itemize}

\textbf{Question}

A factory packs toys into boxes. Each box contains $x$ toys. If the factory packs 5 boxes, there are a total of 60 toys.

Write an equation to represent this situation. Then, find the value of $x$.

\textbf{Solution}

Let $x$ be the number of toys in each box.

Total toys packed in 5 boxes $= 5x$.

We are told $5x = 60$.

To find $x$, divide both sides by 5:
$
5x = 60 \\
x = \frac{60}{5} \\
x = 12
$

So, each box contains $12$ toys.

\section*{Question 33}
\textbf{Metadata}

\begin{itemize}
  \item Question ID: P6-FrDivPN\_P6-FrDivPP\_GPT4.1\_Services\_05
  \item Primary KC: FRACTIONS | Division | dividing a proper fraction by a whole number
  \item Secondary KC: FRACTIONS | Division | dividing a whole number/proper fraction by a proper fraction
  \item Topic: Services
  \item Grade: Primary 6
\end{itemize}

\textbf{Question}

A printing company received an order to produce $4$ batches of flyers. Each batch requires $\frac{3}{4}$ of a ream of paper. The company wants to divide these reams equally among $2$ printing machines, so that each machine handles the same total amount of paper. 

(a) How many reams of paper does one printing machine use?

(b) If each printing machine is able to process only $\frac{1}{2}$ a ream at a time, how many times must one machine run to use up all its allocated paper?

\textbf{Solution}

(a) Total paper required = $4 \times \frac{3}{4} = 3$ reams. 

Paper given to each machine = $\frac{3}{2}$ reams (since $3 \div 2 = \frac{3}{2}$).

(b) Number of times a machine must run = $\frac{\frac{3}{2}}{\frac{1}{2}} = \frac{3}{2} \times \frac{2}{1}=3$ times.

\textbf{Final answers:}

(a) Each machine uses $\frac{3}{2}$ reams of paper.

(b) Each machine must run $3$ times.

\section*{Question 34}
\textbf{Metadata}

\begin{itemize}
  \item Question ID: P6-AgRepLrEx\_P6-AgEvlLrEx\_GPT4.1\_Services\_05
  \item Primary KC: ALGEBRA | Representation and concept | translation of simple real-world situations into linear algebraic expressions
  \item Secondary KC: ALGEBRA | Evaluation | evaluating simple linear expressions by substitution
  \item Topic: Services
  \item Grade: Primary 6
\end{itemize}

\textbf{Question}

A mobile phone repair shop charges a fixed service fee of \$15 and an additional \$8 for each phone repaired. If Samuel brings \( n \) phones for repair, write an algebraic expression to represent the total amount he has to pay. If Samuel brings 4 phones for repair, how much does he pay in total?


\textbf{Solution}

Let the number of phones Samuel brings be \( n \).

The total amount he pays can be represented by the expression:

$ 15 + 8n $

When Samuel brings 4 phones, substitute \( n = 4 \) into the expression:

$
15 + 8 \times 4 = 15 + 32 = 47
$

Thus, Samuel pays \$47 in total.

\section*{Question 35}
\textbf{Metadata}

\begin{itemize}
  \item Question ID: P6-PcFndWN\_P4-WNDiv4d1d\_GPT4.1\_Food\_05
  \item Primary KC: PERCENTAGE | Finding the whole | finding the whole given a part and the percentage
  \item Secondary KC: WHOLE NUMBERS | Division | division up to 4 digits by 1 digit
  \item Topic: Food
  \item Grade: Primary 6
\end{itemize}

\textbf{Question}

Siti baked some muffins for a charity sale. She managed to sell $35\%$ of the muffins, which is a total of 224 muffins. If she plans to divide the remaining muffins equally among 4 friends to take home, how many muffins will each friend receive?

\textbf{Solution}

Let the total number of muffins Siti baked be $x$. 

Since $35\%$ of the muffins is 224, 

$0.35x = 224$ 

So, $x = \dfrac{224}{0.35} = 640$ 

Number of muffins left after selling is $640 - 224 = 416$ 

She wants to divide the remaining muffins equally among 4 friends. 

So, each friend will get: $\dfrac{416}{4} = 104$ 

\boxed{104} muffins.

\section*{Question 36}
\textbf{Metadata}

\begin{itemize}
  \item Question ID: P6-RoFndRoWN\_P4-WNDiv4d1d\_GPT4.1\_Education\_05
  \item Primary KC: RATIO | Finding ratio | finding the ratio of two or three given whole numbers
  \item Secondary KC: WHOLE NUMBERS | Division | division up to 4 digits by 1 digit
  \item Topic: Education
  \item Grade: Primary 6
\end{itemize}

\textbf{Question}

A teacher has $1\,224$ pencils, $816$ erasers, and $408$ rulers to give to her students. She wants to pack them into sets so that each set has an equal number of pencils, erasers, and rulers, and no items are left over. 

Find the largest number of sets she can make, and write the ratio of pencils to erasers to rulers in each set in its simplest form.

\textbf{Solution}

First, we need to find the largest number that can divide $1\,224$, $816$, and $408$ exactly. 

Let us find the greatest common divisor (GCD) of these three numbers.

Let's find the GCD step by step:

$1\,224 \div 408 = 3$, so $408$ is a divisor of $1\,224$ ($408 \times 3 = 1\,224$)

$816 \div 408 = 2$, so $408$ is also a divisor of $816$.

$408 \div 408 = 1$

Therefore, the GCD of $1\,224$, $816$, and $408$ is $408$.

So, the largest number of sets is $408$.

Now, find the number of each item in 1 set:
- Number of pencils in each set: $1\,224 \div 408 = 3$
- Number of erasers in each set: $816 \div 408 = 2$
- Number of rulers in each set: $408 \div 408 = 1$

Therefore, the ratio of pencils to erasers to rulers in each set is $3:2:1$ (in simplest form).

\section*{Question 37}
\textbf{Metadata}

\begin{itemize}
  \item Question ID: P6-PcFndWN\_P4-WNMul4d1d\_GPT4.1\_Manufacturing\_05
  \item Primary KC: PERCENTAGE | Finding the whole | finding the whole given a part and the percentage
  \item Secondary KC: WHOLE NUMBERS | Multiplication | multiplication up to 4 digits by 1 digit or up to 3 digits by 2 digits
  \item Topic: Manufacturing
  \item Grade: Primary 6
\end{itemize}

\textbf{Question}

A factory produces a certain number of electronic parts every week. Last week, the factory reported that 180 parts were found to be defective, which made up 12\% of the total number of parts produced last week. 

(a) How many parts did the factory produce last week in total?

(b) If each electronic part is packed into boxes containing 24 parts each, how many boxes can be packed with all the parts that were produced last week?

\textbf{Solution}

Let the total number of parts produced last week be $x$.

12\% of $x$ equals 180:

$0.12x = 180$

$x = \frac{180}{0.12}$

$x = 1500$

So, the factory produced $1,500$ parts last week.

(b) Number of boxes $= \frac{\text{Total number of parts}}{\text{Number of parts per box}}$

$= \frac{1500}{24}$

Now, $1500 \div 24 = 62\ \text{R}\ 12$

So, 62 full boxes can be packed, and there will be 12 parts left over.

\section*{Question 38}
\textbf{Metadata}

\begin{itemize}
  \item Question ID: P6-AgRepLrEx\_P6-AgSmpLrEx\_GPT4.1\_Food\_05
  \item Primary KC: ALGEBRA | Representation and concept | translation of simple real-world situations into linear algebraic expressions
  \item Secondary KC: ALGEBRA | Simplifying | simplifying simple linear expressions excluding brackets
  \item Topic: Food
  \item Grade: Primary 6
\end{itemize}

\textbf{Question}

A bakery sells muffins and cupcakes. Each muffin costs \(x\) cents and each cupcake costs \(y\) cents. Sara bought 3 muffins and 2 cupcakes. Write a linear algebraic expression that represents the total amount, in cents, that Sara paid for the muffins and cupcakes. Simplify your answer as much as possible.

\textbf{Solution}

The cost of 3 muffins is \(3x\) cents, and the cost of 2 cupcakes is \(2y\) cents.\newline

Total cost that Sara paid = \(3x + 2y\) cents.\newline

This is already a simplified linear algebraic expression because there are no like terms to combine and there are no brackets.\newline

\textbf{Final answer: } \(3x + 2y\)

\section*{Question 39}
\textbf{Metadata}

\begin{itemize}
  \item Question ID: P6-FrDivPP\_P3-FrSmp\_GPT4.1\_Education\_05
  \item Primary KC: FRACTIONS | Division | dividing a whole number/proper fraction by a proper fraction
  \item Secondary KC: FRACTIONS | Simplifying | expressing a fraction in its simplest form
  \item Topic: Education
  \item Grade: Primary 6
\end{itemize}

\textbf{Question}

In a science class, a teacher has $3$ litres of coloured liquid. She wants to pour the liquid equally into bottles, with each bottle containing $\frac{3}{4}$ litre. How many bottles can she fill completely? Express your answer as a fraction in its simplest form.

\textbf{Solution}

To find the number of bottles the teacher can fill, we divide the total amount of liquid by the amount each bottle holds:

$
\text{Number of bottles} = \frac{3}{\frac{3}{4}}
$

Recall that dividing by a fraction is the same as multiplying by its reciprocal:

$
\frac{3}{\frac{3}{4}} = 3 \times \frac{4}{3} = \frac{12}{3} = 4
$

Therefore, the teacher can fill $\boxed{4}$ bottles completely.

\section*{Question 40}
\textbf{Metadata}

\begin{itemize}
  \item Question ID: P6-AgSlvLrN\_P6-AgRepLrEx\_GPT4.1\_Sports\_05
  \item Primary KC: ALGEBRA | Solving simple linear equations | solving simple linear equations involving whole number coefficient only
  \item Secondary KC: ALGEBRA | Representation and concept | translation of simple real-world situations into linear algebraic expressions
  \item Topic: Sports
  \item Grade: Primary 6
\end{itemize}

\textbf{Question}

In a basketball game, Jia Wei scored $x$ points in the first half and 8 points in the second half. He scored a total of 24 points in the game. Write down an equation to represent this situation and find the number of points Jia Wei scored in the first half.

\textbf{Solution}

Let $x$ be the number of points Jia Wei scored in the first half.

Jia Wei scored a total of 24 points, so the equation is:

$ x + 8 = 24 $

Solving for $x$:

$
\begin{align*}
x + 8 &= 24 \\
x &= 24 - 8 \\
x &= 16
\end{align*}
$

Jia Wei scored \( 16 \) points in the first half.

\section*{Question 41}
\textbf{Metadata}

\begin{itemize}
  \item Question ID: P6-PcFndChg\_P3-WNAdd4d\_GPT4.1\_Household finance\_05
  \item Primary KC: PERCENTAGE | Finding change | finding percentage increase/decrease
  \item Secondary KC: WHOLE NUMBERS | Addition | addition up to 4 digits
  \item Topic: Household finance
  \item Grade: Primary 6
\end{itemize}

\textbf{Question}

Alicia spends $1250 last month on groceries. This month, she spent $1375 on groceries after buying more snacks for a family gathering. 

(a) What is the total amount Alicia spent over the two months?

(b) What was the percentage increase in Alicia's grocery spending from last month to this month? (Round your answer to the nearest whole number.)

\textbf{Solution}

Let last month's spending be $1250$ and this month's spending be $1375$.

(a) The total amount Alicia spent over the two months:

$
1250 + 1375 = 2625
$

Alicia spent a total of $2625$ over the two months.

(b) The increase in spending from last month to this month:

$
1375 - 1250 = 125
$

To find the percentage increase:

$
\text{Percentage Increase} = \frac{\text{Increase}}{\text{Original Amount}} \times 100\%
$

$
= \frac{125}{1250} \times 100\% = 10\%
$

So, Alicia's grocery spending increased by \(10\%\) from last month to this month.

\section*{Question 42}
\textbf{Metadata}

\begin{itemize}
  \item Question ID: P6-RoFndDvqWN\_P6-RoFndTmWN\_GPT4.1\_Leisure\_05
  \item Primary KC: RATIO | Finding divided quantities | dividing a quantity in a given ratio
  \item Secondary KC: RATIO | Finding a missing term | finding the missing term in a pair of equivalent ratios
  \item Topic: Leisure
  \item Grade: Primary 6
\end{itemize}

\textbf{Question}

\textbf{Word Problem:}  

Sarah and Marcus are planning to buy movie tickets. They agree to share the total cost of $84 in the ratio $2:5$. However, Marcus only brings enough money to pay $45$. What is the least amount Sarah must pay so that the ratio of their payments remains the same as $2:5$? Also, if Sarah has already paid $x$ dollars, and the ratio of Sarah's payment to Marcus's payment is $x:45 = 2:5$, what is the value of $x$?

\textbf{Solution}

\textbf{Solution:}  

First, we divide the total cost in the ratio $2:5$.  
Sum of parts $= 2+5 = 7$ parts.  

Sarah's share $= \frac{2}{7} \times 84 = 24$ \\  
Marcus's share $= \frac{5}{7} \times 84 = 60$  

But Marcus only pays $45$, not $60$. To keep the ratio the same, the total amount paid must be in the ratio $2:5$.  
So, let Marcus's payment $=45$, then let Sarah's payment be $x$.  

We set up the ratio:  

$ \frac{x}{45} = \frac{2}{5} $

Solving for $x$:

$ x = 45 \times \frac{2}{5} = 18 $

So the least amount Sarah must pay is $18$ so that the ratio remains the same.

Sarah pays $18$, Marcus pays $45$, and the ratio is $18:45 = 2:5$.  

\boxed{18}$ is the value of $x$.

\section*{Question 43}
\textbf{Metadata}

\begin{itemize}
  \item Question ID: P6-FrDivPN\_P6-FrDivPP\_GPT4.1\_Food\_05
  \item Primary KC: FRACTIONS | Division | dividing a proper fraction by a whole number
  \item Secondary KC: FRACTIONS | Division | dividing a whole number/proper fraction by a proper fraction
  \item Topic: Food
  \item Grade: Primary 6
\end{itemize}

\textbf{Question}

\textbf{Word Problem:} \\ 
Amy made $\dfrac{3}{4}$ of a chocolate cake for her family's picnic. She wants to share the cake equally among 3 of her cousins. 

(a) How much of the cake will each cousin get?  \\ 
Later, Amy's family brings another $1$ chocolate cake, making a total of $\dfrac{7}{4}$ cakes. At the picnic, each person receives $\dfrac{1}{4}$ of a cake. 

(b) How many people can Amy serve with the total amount of cake?

\textbf{Solution}

\textbf{Solution:} \\ 
\text{(a) Amount for each cousin:} \\ 
\dfrac{3}{4} \div 3 = \dfrac{3}{4} \div \dfrac{3}{1} = \dfrac{3}{4} \times \dfrac{1}{3} = \dfrac{3 \times 1}{4 \times 3} = \dfrac{3}{12} = \dfrac{1}{4} \\ 
\text{Each cousin gets } \boxed{\dfrac{1}{4}} \text{ of the cake.} \\ 
\text{(b) Number of people served:} \\ 
\text{Total cake: } \dfrac{3}{4} + 1 = \dfrac{3}{4} + \dfrac{4}{4} = \dfrac{7}{4} \\ 
\dfrac{7}{4} \div \dfrac{1}{4} = \dfrac{7}{4} \times \dfrac{4}{1} = \dfrac{7 \times 4}{4 \times 1} = \dfrac{28}{4} = 7 \\ 
\text{Amy can serve } \boxed{7} \text{ people with the total amount of cake.}

\section*{Question 44}
\textbf{Metadata}

\begin{itemize}
  \item Question ID: P6-FrDivPP\_P5-FrMulMixN\_GPT4.1\_Farming\_05
  \item Primary KC: FRACTIONS | Division | dividing a whole number/proper fraction by a proper fraction
  \item Secondary KC: FRACTIONS | Multiplication | multiplying a mixed number and a whole number
  \item Topic: Farming
  \item Grade: Primary 6
\end{itemize}

\textbf{Question}

\textbf{Word Problem:}

Mr. Tan has a farm with $24$ bags of fertilizer. Each vegetable plot requires $\frac{3}{4}$ bag of fertilizer. After fertilizing his vegetable plots, he wants to plant fruit trees. If each fruit tree requires $2\frac{1}{2}$ times the amount of fertilizer used for a single vegetable plot, how many fruit trees can Mr. Tan fertilize with the remaining fertilizer after all the vegetable plots are done?

\textbf{Solution}

\textbf{Solution:}

\begin{align*}
&\text{Step 1: Find the number of vegetable plots Mr. Tan can fertilize.}\\
&\quad \text{Number of plots} = \frac{24}{\frac{3}{4}} = 24 \times \frac{4}{3} = 32\\

&\text{Therefore, he can fertilize 32 vegetable plots.}\\

&\text{Step 2: Total amount of fertilizer used for the vegetable plots.}\\
&\quad \text{Fertilizer used} = 32 \times \frac{3}{4} = 24 \text{ bags}\\

&\text{So, all 24 bags are used for the vegetable plots. There is no fertilizer left.}\\
&\text{Therefore, Mr. Tan cannot fertilize any fruit trees with the remaining fertilizer.}\\
\end{align*}

\boxed{0}\text{ fruit trees}

\textbf{If however, the question intended for Mr. Tan to use only part of his fertilizer for vegetable plots, update as needed according to additional instructions.}


\section*{Question 45}
\textbf{Metadata}

\begin{itemize}
  \item Question ID: P6-FrDivPP\_P2-FrCmp\_GPT4.1\_Education\_05
  \item Primary KC: FRACTIONS | Division | dividing a whole number/proper fraction by a proper fraction
  \item Secondary KC: FRACTIONS | Comparison and ordering | comparing and ordering fractions
  \item Topic: Education
  \item Grade: Primary 6
\end{itemize}

\textbf{Question}

In a mathematics class, Mrs. Tan has prepared \(\frac{3}{4}\) of a cake for a fun learning activity. She wants to divide the cake equally among groups of students, with each group receiving \(\frac{1}{8}\) of a cake. 

(a) How many groups can Mrs. Tan form with the cake she has?

(b) Another teacher, Mr. Lee, does a similar activity with \(\frac{2}{3}\) of a cake, dividing it into portions of \(\frac{1}{6}\) of a cake each. 

Arrange the number of groups each teacher can form, in order from smallest to largest.

\textbf{Solution}

(a) Number of groups Mrs. Tan can form:

$
\frac{3}{4} \div \frac{1}{8} = \frac{3}{4} \times \frac{8}{1} = \frac{24}{4} = 6
$

So, Mrs. Tan can form 6 groups.

(b) Number of groups Mr. Lee can form:

$
\frac{2}{3} \div \frac{1}{6} = \frac{2}{3} \times \frac{6}{1} = \frac{12}{3} = 4
$

So, Mr. Lee can form 4 groups.

Comparing the numbers: 6 (Mrs. Tan) and 4 (Mr. Lee). 4 < 6.

\textbf{Final Answer:} Mr. Lee can form fewer groups than Mrs. Tan. The arrangement in order from smallest to largest is: 4 (Mr. Lee), 6 (Mrs. Tan).

\section*{Question 46}
\textbf{Metadata}

\begin{itemize}
  \item Question ID: P6-AgRepLrEx\_P6-AgEvlLrEx\_GPT4.1\_Food\_05
  \item Primary KC: ALGEBRA | Representation and concept | translation of simple real-world situations into linear algebraic expressions
  \item Secondary KC: ALGEBRA | Evaluation | evaluating simple linear expressions by substitution
  \item Topic: Food
  \item Grade: Primary 6
\end{itemize}

\textbf{Question}

A hawker centre sells plates of chicken rice at \$x\ per plate and cups of teh tarik at \$y\ per cup. Sarah buys 3 plates of chicken rice and 2 cups of teh tarik. \\ 
(a) Write an algebraic expression to show the total amount Sarah spends. \\ 
(b) If \( x = 4 \) and \( y = 1.50 \), how much does Sarah spend in total?

\textbf{Solution}

(a) The total amount Sarah spends is given by: \\ 
$ 3x + 2y $ \\ 
(b) Substituting \( x = 4 \) and \( y = 1.50 \): \\ 
$ 3(4) + 2(1.50) = 12 + 3 = 15 $ \\ 
Sarah spends \$15 in total.

\section*{Question 47}
\textbf{Metadata}

\begin{itemize}
  \item Question ID: P6-RoFndRoWN\_P4-WNMul4d1d\_GPT4.1\_Manufacturing\_05
  \item Primary KC: RATIO | Finding ratio | finding the ratio of two or three given whole numbers
  \item Secondary KC: WHOLE NUMBERS | Multiplication | multiplication up to 4 digits by 1 digit or up to 3 digits by 2 digits
  \item Topic: Manufacturing
  \item Grade: Primary 6
\end{itemize}

\textbf{Question}

\textbf{Word Problem:} 

A factory produces blue pens and red pens every day. On Monday, the factory made 1,236 blue pens and 864 red pens. Each box can contain only 12 pens of the same colour. 

(a) How many boxes of blue pens did the factory pack on Monday?  
(b) How many boxes of red pens did the factory pack on Monday?  
(c) What is the ratio of the number of boxes of blue pens to the number of boxes of red pens packed on Monday? Express your answer in its simplest form.

\textbf{Solution}

\textbf{Solution:} 

(a) To find the number of boxes of blue pens: 

Number of blue pens $= 1,236$

Each box holds $12$ pens.

Number of boxes of blue pens $= \dfrac{1,236}{12} = 103$

(b) To find the number of boxes of red pens: 

Number of red pens $= 864$

Each box holds $12$ pens.

Number of boxes of red pens $= \dfrac{864}{12} = 72$

(c) The ratio of the number of boxes of blue pens to red pens is $103:72$.

This ratio is already in its simplest form because $103$ and $72$ have no common factors other than $1$.

\textbf{Answer:} 

(a) $103$ boxes of blue pens 

(b) $72$ boxes of red pens 

(c) The ratio is $103:72$.

\section*{Question 48}
\textbf{Metadata}

\begin{itemize}
  \item Question ID: P6-FrDivPP\_P3-FrSmp\_GPT4.1\_Manufacturing\_05
  \item Primary KC: FRACTIONS | Division | dividing a whole number/proper fraction by a proper fraction
  \item Secondary KC: FRACTIONS | Simplifying | expressing a fraction in its simplest form
  \item Topic: Manufacturing
  \item Grade: Primary 6
\end{itemize}

\textbf{Question}

In a factory, a machine uses \( \frac{3}{4} \) kilogram of plastic to make one toy. One day, the machine receives \( 6 \) kilograms of plastic. How many toys can the machine make with the plastic it receives? Express your answer as a whole number or a fraction in its simplest form.

\textbf{Solution}

To find how many toys the machine can make, divide the total amount of plastic by the amount needed for one toy:

$
\text{Number of toys} = \frac{6}{\frac{3}{4}} = 6 \times \frac{4}{3} = \frac{24}{3} = 8
$

So, the machine can make \( 8 \) toys with \( 6 \) kilograms of plastic.

\section*{Question 49}
\textbf{Metadata}

\begin{itemize}
  \item Question ID: P6-RoFndRoWN\_P3-WNSub4d\_GPT4.1\_Education\_05
  \item Primary KC: RATIO | Finding ratio | finding the ratio of two or three given whole numbers
  \item Secondary KC: WHOLE NUMBERS | Subtraction | subtraction up to 4 digits
  \item Topic: Education
  \item Grade: Primary 6
\end{itemize}

\textbf{Question}

A school library has 2,560 English books and 1,480 Science books. After 450 English books and 320 Science books were borrowed by students, what is the ratio of the remaining number of English books to the remaining number of Science books? Give your answer in its simplest form.

\textbf{Solution}

First, calculate the remaining number of English books:
$
\text{Remaining English books} = 2,560 - 450 = 2,110
$
Calculate the remaining number of Science books:
$
\text{Remaining Science books} = 1,480 - 320 = 1,160
$
Now, find the ratio of the remaining number of English books to Science books:
$
\text{Ratio} = 2,110 : 1,160
$
Simplify the ratio by dividing both numbers by their greatest common divisor (GCD). The GCD of 2,110 and 1,160 is 10.
$
2,110 \div 10 = 211
$
$
1,160 \div 10 = 116
$
So, the simplest form of the ratio is:
$
\boxed{211 : 116}
$

\section*{Question 50}
\textbf{Metadata}

\begin{itemize}
  \item Question ID: P6-PcFndChg\_P3-WNSub4d\_GPT4.1\_Digital ecconomy\_05
  \item Primary KC: PERCENTAGE | Finding change | finding percentage increase/decrease
  \item Secondary KC: WHOLE NUMBERS | Subtraction | subtraction up to 4 digits
  \item Topic: Digital ecconomy
  \item Grade: Primary 6
\end{itemize}

\textbf{Question}

A local online store sold 2,150 smartphones last year. This year, the store sold 1,750 smartphones.

(a) How many fewer smartphones did the store sell this year compared to last year?

(b) What is the percentage decrease in the number of smartphones sold this year compared to last year?

\textbf{Solution}

(a) Number of smartphones sold last year = 2,150\\
Number of smartphones sold this year = 1,750\\
\\
Number of smartphones fewer sold = 2,150 - 1,750 = 400\\
\\
\textbf{Answer to (a):} 400 fewer smartphones were sold this year.
\\
(b) Percentage decrease = \frac{\text{Decrease in number}}{\text{Original number}} \times 100\%\\
\\
Decrease in number = 2,150 - 1,750 = 400\\
\\
Percentage decrease = \frac{400}{2,150} \times 100\% = 18.60\%\text{ (rounded to 2 decimal places)}\\
\\
\textbf{Answer to (b):} The percentage decrease is 18.60\%.

\section*{Question 51}
\textbf{Metadata}

\begin{itemize}
  \item Question ID: P6-RoFndRoWN\_P6-RoSmpWN\_GPT4.1\_Sports\_05
  \item Primary KC: RATIO | Finding ratio | finding the ratio of two or three given whole numbers
  \item Secondary KC: RATIO | Simplifying | expressing a ratio in its simplest form
  \item Topic: Sports
  \item Grade: Primary 6
\end{itemize}

\textbf{Question}

In a sports day event, there are \(24\) boys, \(18\) girls and \(12\) teachers helping to organise a relay race. \\ 
(a) What is the ratio of boys to girls to teachers? \\ 
(b) Express this ratio in its simplest form.

\textbf{Solution}

(a) The ratio of boys to girls to teachers is:
\\
\(24 : 18 : 12\)
\\
(b) To express this ratio in its simplest form, find the greatest common divisor (GCD) of \(24\), \(18\) and \(12\).
\\
The GCD is \(6\).
\\
Divide each number by \(6\):
\\
\(
\frac{24}{6} : \frac{18}{6} : \frac{12}{6} = 4 : 3 : 2
\)
\\
So, the simplest form of the ratio is \(4:3:2\).

\section*{Question 52}
\textbf{Metadata}

\begin{itemize}
  \item Question ID: P6-AgSlvLrN\_P6-AgRepLrEx\_GPT4.1\_Transporation\_05
  \item Primary KC: ALGEBRA | Solving simple linear equations | solving simple linear equations involving whole number coefficient only
  \item Secondary KC: ALGEBRA | Representation and concept | translation of simple real-world situations into linear algebraic expressions
  \item Topic: Transporation
  \item Grade: Primary 6
\end{itemize}

\textbf{Question}

A taxi company in Singapore charges a base fare of \$4, plus \$2 for every kilometre travelled. If Tom took a taxi and the total fare was \$18, how many kilometres did Tom travel? Represent the total fare as a linear equation and solve for the distance travelled.

\textbf{Solution}

Let $x$ be the number of kilometres Tom travelled. 

The total fare can be represented by the equation:

$
4 + 2x = 18
$

Subtract 4 from both sides:
$
2x = 18 - 4 \\
2x = 14
$
Divide both sides by 2:
$
x = \frac{14}{2} \\
x = 7
$

Tom travelled 7 kilometres.

\section*{Question 53}
\textbf{Metadata}

\begin{itemize}
  \item Question ID: P6-FrDivPN\_P5-FrSubMix\_GPT4.1\_Services\_05
  \item Primary KC: FRACTIONS | Division | dividing a proper fraction by a whole number
  \item Secondary KC: FRACTIONS | Subtraction | subtracting mixed numbers
  \item Topic: Services
  \item Grade: Primary 6
\end{itemize}

\textbf{Question}

A bakery received an order to deliver $3\dfrac{1}{2}$ kg of flour to a nearby cafe every day for 5 days. The bakery has $7\dfrac{2}{3}$ kg of flour in total. \newline
(a) How much flour does the cafe receive each day? Express your answer as a proper fraction. \newline
(b) After making deliveries for 5 days, how much flour does the bakery have left? Show your working clearly.

\textbf{Solution}

(a) \textbf{Flour delivered each day:} \newline
Total flour delivered in 5 days = $3\dfrac{1}{2}$ kg $\times$ 5 = $\dfrac{7}{2} \times 5 = \dfrac{35}{2}$ kg. \newline
Amount delivered each day = $\dfrac{35}{2} \div 5 = \dfrac{35}{2} \times \dfrac{1}{5} = \dfrac{35}{10} = \dfrac{7}{2}$ kg$ \div 5 = \dfrac{7}{10}$ kg$.$ So the cafe receives $\dfrac{7}{10}$ kg each day.\newline
(b) \textbf{Flour left after 5 days:} \newline
Total flour used = $3\dfrac{1}{2} \times 5 = 17\dfrac{1}{2}$ kg$.$ \newline
Available flour = $7\dfrac{2}{3}$ kg$.$ \newline
Flour left = $7\dfrac{2}{3} - 17\dfrac{1}{2}$ \newline
First, convert both to improper fractions:\newline
$7\dfrac{2}{3} = \dfrac{23}{3}$ \newline
$17\dfrac{1}{2} = \dfrac{35}{2}$ \newline
Find common denominator (6): \newline
$\dfrac{23}{3} = \dfrac{46}{6}$ \newline
$\dfrac{35}{2} = \dfrac{105}{6}$ \newline
So, $\dfrac{46}{6} - \dfrac{105}{6} = -\dfrac{59}{6}$.\newline
The bakery will need $\dfrac{59}{6}$ kg more flour, which means it does not have enough flour and will be short of $9\dfrac{5}{6}$ kg.

\section*{Question 54}
\textbf{Metadata}

\begin{itemize}
  \item Question ID: P6-FrDivPP\_P5-FrCnv2Dc\_GPT4.1\_Farming\_05
  \item Primary KC: FRACTIONS | Division | dividing a whole number/proper fraction by a proper fraction
  \item Secondary KC: FRACTIONS | Conversion to decimals | expressing fractions as decimals
  \item Topic: Farming
  \item Grade: Primary 6
\end{itemize}

\textbf{Question}

A farmer harvested $4$ sacks of rice. He wants to pack the rice into smaller bags, each containing $\frac{2}{5}$ sack of rice.

(a) How many smaller bags can he fill?

(b) Express the amount of rice in each small bag as a decimal.

\textbf{Solution}

(a) To find the number of smaller bags:

Number of bags $= 4 \div \frac{2}{5}$

$= 4 \times \frac{5}{2}$

$= \frac{4 \times 5}{2}$

$= \frac{20}{2}$

$= 10$

So, the farmer can fill $10$ smaller bags.

(b) Each small bag contains $\frac{2}{5}$ sack of rice. To express $\frac{2}{5}$ as a decimal:

$\frac{2}{5} = 0.4$

So, each small bag contains $0.4$ sack of rice.

\section*{Question 55}
\textbf{Metadata}

\begin{itemize}
  \item Question ID: P6-FrDivPN\_P6-FrDivPP\_GPT4.1\_Manufacturing\_05
  \item Primary KC: FRACTIONS | Division | dividing a proper fraction by a whole number
  \item Secondary KC: FRACTIONS | Division | dividing a whole number/proper fraction by a proper fraction
  \item Topic: Manufacturing
  \item Grade: Primary 6
\end{itemize}

\textbf{Question}

In a factory, a machine pours \( \dfrac{2}{3} \) litre of paint into a container for every batch. The supervisor wants to divide the paint equally into 4 smaller jars for testing. 

(a) How much paint will be in each jar?

(b) If each test requires \( \dfrac{1}{6} \) litre of paint, how many complete tests can be done with the paint from 1 jar?

\textbf{Solution}

(a) Total paint to be divided into 4 jars:

$
\frac{2}{3} \div 4 = \frac{2}{3} \div \frac{4}{1} = \frac{2}{3} \times \frac{1}{4} = \frac{2}{12} = \frac{1}{6}
$

So, each jar will contain \( \frac{1}{6} \) litre of paint.

(b) Number of complete tests possible with \( \frac{1}{6} \) litre of paint, using \( \frac{1}{6} \) litre per test:

$
\frac{1}{6} \div \frac{1}{6} = 1
$

So, only 1 complete test can be done with the paint from 1 jar.

\section*{Question 56}
\textbf{Metadata}

\begin{itemize}
  \item Question ID: P6-FrDivPP\_P2-FrCmp\_GPT4.1\_Digital ecconomy\_05
  \item Primary KC: FRACTIONS | Division | dividing a whole number/proper fraction by a proper fraction
  \item Secondary KC: FRACTIONS | Comparison and ordering | comparing and ordering fractions
  \item Topic: Digital ecconomy
  \item Grade: Primary 6
\end{itemize}

\textbf{Question}

A group of friends earned $12 from selling digital art online. They want to split the money equally among themselves, with each person getting $\frac{2}{3}$ of a dollar. 

(a) How many friends can share the money equally?

(b) After dividing the money, the friends decided to compare what would happen if each person gets $\frac{1}{2}$ of a dollar instead. Would more friends or fewer friends be able to share the money equally this time? Show your working.

\textbf{Solution}

(a) To find out how many friends can share the $12 equally if each gets $\frac{2}{3}$ of a dollar, we divide:

$$
12 \div \frac{2}{3} = 12 \times \frac{3}{2} = 18
$$

So, **18 friends** can share the money equally.

(b) For $\frac{1}{2}$ of a dollar per friend:
$$
12 \div \frac{1}{2} = 12 \times 2 = 24
$$
So, **24 friends** can share the money if each gets $\frac{1}{2}$ of a dollar.

Comparing $\frac{1}{2}$ and $\frac{2}{3}$:
$$
\frac{1}{2} = \frac{3}{6};\ \frac{2}{3} = \frac{4}{6}
$$
$\frac{1}{2}$ is less than $\frac{2}{3}$.

Therefore, **more friends** can share the money if each gets $\frac{1}{2}$ of a dollar.

\section*{Question 57}
\textbf{Metadata}

\begin{itemize}
  \item Question ID: P6-FrDivPN\_P2-FrCmp\_GPT4.1\_Education\_05
  \item Primary KC: FRACTIONS | Division | dividing a proper fraction by a whole number
  \item Secondary KC: FRACTIONS | Comparison and ordering | comparing and ordering fractions
  \item Topic: Education
  \item Grade: Primary 6
\end{itemize}

\textbf{Question}

A teacher has $\dfrac{3}{4}$ of a cake left after a class party. She wants to divide the remaining cake equally among 6 students.

(a) How much of the cake does each student get?

(b) Another teacher has $\dfrac{5}{6}$ of a cake left and also wants to share it equally among 6 students. Compare the amount of cake each student receives from both teachers. Who gets more cake per student?

\textbf{Solution}

(a) The amount of cake each student gets is:

$
\frac{3}{4} \div 6 = \frac{3}{4} \times \frac{1}{6} = \frac{3}{24} = \frac{1}{8}
$

Each student gets $\dfrac{1}{8}$ of the cake.

(b) The amount of cake each student gets from the second teacher is:

$
\frac{5}{6} \div 6 = \frac{5}{6} \times \frac{1}{6} = \frac{5}{36}
$

To compare $\dfrac{1}{8}$ and $\dfrac{5}{36}$, find a common denominator:

The lowest common denominator of 8 and 36 is 72.

$
\frac{1}{8} = \frac{9}{72}
$
$
\frac{5}{36} = \frac{10}{72}
$

So,

$
\frac{10}{72} > \frac{9}{72}
$

Therefore, each student from the second teacher gets more cake. 

\textbf{Answer:} Each student from the second teacher gets more cake.

\section*{Question 58}
\textbf{Metadata}

\begin{itemize}
  \item Question ID: P6-PcFndWN\_P4-WNDiv4d1d\_GPT4.1\_Transporation\_05
  \item Primary KC: PERCENTAGE | Finding the whole | finding the whole given a part and the percentage
  \item Secondary KC: WHOLE NUMBERS | Division | division up to 4 digits by 1 digit
  \item Topic: Transporation
  \item Grade: Primary 6
\end{itemize}

\textbf{Question}

A bus company reported that 25\% of its buses, which is a total of 324 buses, have been sent for annual maintenance. If each depot in Singapore manages 9 buses, how many depots does the company need in total to manage all of its buses?

\textbf{Solution}

Let the total number of buses be $x$. Since 25\% of the buses is 324, we have:

$
0.25x = 324
$

To find the total number of buses:
$
x = \frac{324}{0.25} = 1296
$

Each depot manages 9 buses. So, the total number of depots needed is:
$
\text{Number of depots} = \frac{1296}{9} = 144
$

\textbf{Answer:} The company needs 144 depots to manage all of its buses.

\section*{Question 59}
\textbf{Metadata}

\begin{itemize}
  \item Question ID: P6-PcFndChg\_P3-WNAdd4d\_GPT4.1\_Transporation\_05
  \item Primary KC: PERCENTAGE | Finding change | finding percentage increase/decrease
  \item Secondary KC: WHOLE NUMBERS | Addition | addition up to 4 digits
  \item Topic: Transporation
  \item Grade: Primary 6
\end{itemize}

\textbf{Question}

A bus company recorded the number of passengers on a popular route during two different months. In March, there were 1,860 passengers. In April, there were 2,325 passengers. 

(a) By how many passengers did the number increase from March to April?

(b) What was the percentage increase in the number of passengers from March to April? Give your answer correct to 1 decimal place.

\textbf{Solution}

(a) Number of passengers increased = 2,325 - 1,860 = 465 passengers.

(b) Percentage increase = \frac{\text{Increase}}{\text{Original}} \times 100\%

= \frac{465}{1,860} \times 100\%

= 25\%

\frac{465}{1,860} = 0.25

0.25 \times 100\% = 25\%

\boxed{25.0\%} (to 1 decimal place)

\section*{Question 60}
\textbf{Metadata}

\begin{itemize}
  \item Question ID: P6-FrDivPN\_P2-FrCmp\_GPT4.1\_Sports\_05
  \item Primary KC: FRACTIONS | Division | dividing a proper fraction by a whole number
  \item Secondary KC: FRACTIONS | Comparison and ordering | comparing and ordering fractions
  \item Topic: Sports
  \item Grade: Primary 6
\end{itemize}

\textbf{Question}

A basketball coach has $\dfrac{3}{4}$ of a litre of sports drink. He wants to share the drink equally among his 5 team members after a training session. 

(a) How much sports drink will each player receive?

(b) Arrange the following amounts in descending order:

$ \dfrac{3}{4} \div 5, \quad \dfrac{1}{8}, \quad \dfrac{2}{15} $

\textbf{Solution}

(a) To find the amount each player receives:

$
\dfrac{3}{4} \div 5 = \dfrac{3}{4} \times \dfrac{1}{5} = \dfrac{3}{20}\text{ litres}
$

So, each player receives $\dfrac{3}{20}$ litres of sports drink.

(b) Convert all fractions to have a common denominator to compare:

$
\dfrac{3}{20} = \dfrac{3}{20}
$

$
\dfrac{1}{8} = \dfrac{1 \times 5}{8 \times 5} = \dfrac{5}{40} = \dfrac{10}{20}
$

$
\dfrac{2}{15} = \dfrac{2 \times 4}{15 \times 4} = \dfrac{8}{60} = \dfrac{8}{20} \quad \text{(simplified as much as possible, but the common denominator with 20 is 60, so:)}
$

$
\dfrac{3}{20} = \dfrac{9}{60} \qquad \dfrac{1}{8} = \dfrac{7.5}{60}\qquad \dfrac{2}{15} = \dfrac{8}{60}
$
Alternatively, let us use decimals:

$
\dfrac{3}{20} = 0.15
$
$
\dfrac{1}{8} = 0.125
$
$
\dfrac{2}{15} \approx 0.133
$

Arranging in descending order:

$
\dfrac{3}{20},\ \dfrac{2}{15},\ \dfrac{1}{8}
$

So, the order from greatest to least is:

$
\dfrac{3}{20},\ \dfrac{2}{15},\ \dfrac{1}{8}
$

\section*{Question 61}
\textbf{Metadata}

\begin{itemize}
  \item Question ID: P6-FrDivPP\_P3-FrSmp\_GPT4.1\_Services\_05
  \item Primary KC: FRACTIONS | Division | dividing a whole number/proper fraction by a proper fraction
  \item Secondary KC: FRACTIONS | Simplifying | expressing a fraction in its simplest form
  \item Topic: Services
  \item Grade: Primary 6
\end{itemize}

\textbf{Question}


A cleaning company has $3$ litres of disinfectant. Each classroom in a school requires $\dfrac{2}{5}$ litre of disinfectant to be cleaned thoroughly.

(a) How many classrooms can be cleaned with $3$ litres of disinfectant?

(b) If the answer is a fraction, give your answer in its simplest form.


\textbf{Solution}


(a) To find how many classrooms can be cleaned, divide the total amount of disinfectant by the amount needed for each classroom:

$
\text{Number of classrooms} = \frac{3}{\dfrac{2}{5}}
$

To divide a whole number by a fraction, multiply by the reciprocal:
$
\frac{3}{1} \times \frac{5}{2} = \frac{15}{2}
$

So, $\frac{15}{2}$ classrooms can be cleaned.

(b) Expressing the answer in its simplest form:

$\frac{15}{2}$ is already in its simplest form. As a mixed number, it is $7\dfrac{1}{2}$ classrooms.


\section*{Question 62}
\textbf{Metadata}

\begin{itemize}
  \item Question ID: P6-RoFndRoWN\_P3-WNAdd4d\_GPT4.1\_Leisure\_05
  \item Primary KC: RATIO | Finding ratio | finding the ratio of two or three given whole numbers
  \item Secondary KC: WHOLE NUMBERS | Addition | addition up to 4 digits
  \item Topic: Leisure
  \item Grade: Primary 6
\end{itemize}

\textbf{Question}

Sarah attended a board game tournament. On her table, there were 1,285 red tiles, 1,760 blue tiles, and 955 green tiles. She wanted to find the ratio of red tiles to blue tiles to green tiles.

What is the ratio of red tiles to blue tiles to green tiles?

\textbf{Solution}

Sarah has $1,285$ red tiles, $1,760$ blue tiles, and $955$ green tiles.

First, write the numbers as a ratio:

$1,285 : 1,760 : 955$

To express this ratio in simplest form, find the greatest common divisor (GCD) of the three numbers.

$\gcd(1285,1760,955) = 5$

Now, divide all numbers by $5$:

$1,285 \div 5 = 257$

$1,760 \div 5 = 352$

$955 \div 5 = 191$

So, the simplest ratio is:

$257 : 352 : 191$

\section*{Question 63}
\textbf{Metadata}

\begin{itemize}
  \item Question ID: P6-PcFndWN\_P4-WNDiv4d1d\_GPT4.1\_Sports\_05
  \item Primary KC: PERCENTAGE | Finding the whole | finding the whole given a part and the percentage
  \item Secondary KC: WHOLE NUMBERS | Division | division up to 4 digits by 1 digit
  \item Topic: Sports
  \item Grade: Primary 6
\end{itemize}

\textbf{Question}

\textbf{In a basketball tournament, \(32\) players from Team Tiger scored a total of \(768\) points. If these points represent \(40\%\) of the total points scored by all the teams in the tournament, how many points did all the teams score altogether?}

\textbf{Give your answer in the simplest form.}

\textbf{Solution}

\textbf{Let the total points scored by all the teams be \(x\).}

Given that \(768\) points represent \(40\%\) of the total points:

$
40\% \text{ of } x = 768
$
$
\frac{40}{100} \times x = 768
$
$
\frac{2}{5}x = 768
$

To find \(x\), divide both sides by \(\frac{2}{5}\):
$
x = 768 \div \frac{2}{5}
$
$
x = 768 \times \frac{5}{2}
$
$
x = 384 \times 5
$
$
x = 1920
$

\textbf{Answer:}

All the teams scored \(1920\) points altogether.

\section*{Question 64}
\textbf{Metadata}

\begin{itemize}
  \item Question ID: P6-FrDivPP\_P5-FrMulMixN\_GPT4.1\_Services\_05
  \item Primary KC: FRACTIONS | Division | dividing a whole number/proper fraction by a proper fraction
  \item Secondary KC: FRACTIONS | Multiplication | multiplying a mixed number and a whole number
  \item Topic: Services
  \item Grade: Primary 6
\end{itemize}

\textbf{Question}

\textbf{Problem:} 

Andrew owns a cleaning service. He has 12 litres of cleaning solution. For each task, he uses $\dfrac{3}{4}$ litres of the solution. 

(a) How many tasks can Andrew complete with 12 litres of cleaning solution? 

Later, Andrew gets a large contract where each area to be cleaned requires $2\dfrac{1}{2}$ times more cleaning solution than a normal task, and he needs to clean 5 such areas. 

(b) How much cleaning solution in total does Andrew need for all 5 areas?

\textbf{Solution}

\textbf{Solution:}

\textbf{(a) Number of tasks Andrew can complete:}

Number of tasks $= \dfrac{12}{\frac{3}{4}}$

Divide 12 by $\frac{3}{4}$ by multiplying 12 by the reciprocal of $\frac{3}{4}$:

$= 12 \times \dfrac{4}{3}$

$= (12 \div 3) \times 4$

$= 4 \times 4$

$= 16$

\textbf{Andrew can complete 16 tasks with 12 litres of cleaning solution.}

\\

\textbf{(b) Total cleaning solution needed for 5 areas:}

Cleaning solution for one area $= 2\dfrac{1}{2} \times \dfrac{3}{4}$ litres

Convert mixed number to improper fraction:
$2\dfrac{1}{2} = \dfrac{5}{2}$

So:
$\dfrac{5}{2} \times \dfrac{3}{4} = \dfrac{15}{8}$ litres per area.

For 5 areas:
$5 \times \dfrac{15}{8} = \dfrac{75}{8}$ litres

Change to mixed number:
$\dfrac{75}{8} = 9\dfrac{3}{8}$ litres

\textbf{Andrew needs $9\dfrac{3}{8}$ litres of cleaning solution for all 5 areas.}

\section*{Question 65}
\textbf{Metadata}

\begin{itemize}
  \item Question ID: P6-FrDivPP\_P5-FrCnv2Dc\_GPT4.1\_Education\_05
  \item Primary KC: FRACTIONS | Division | dividing a whole number/proper fraction by a proper fraction
  \item Secondary KC: FRACTIONS | Conversion to decimals | expressing fractions as decimals
  \item Topic: Education
  \item Grade: Primary 6
\end{itemize}

\textbf{Question}

A classroom has a long whiteboard that is 7 metres in length. Mrs Tan wants to divide the whiteboard into equal sections for different subjects, with each section being $\frac{2}{5}$ metres long. 

(a) How many sections can Mrs Tan make on the whiteboard?

(b) What is the length of each section in decimal form, correct to 2 decimal places?

\textbf{Solution}

Let the total length of the whiteboard be 7 metres, and the length of each section be $\frac{2}{5}$ metres.

(a) The number of sections = $\dfrac{7}{\frac{2}{5}} = 7 \times \dfrac{5}{2} = \dfrac{35}{2} = 17.5$

Since we cannot have half a section in reality, Mrs Tan can make 17 full sections.

(b) Each section is $\frac{2}{5}$ metres long. As a decimal, $\frac{2}{5} = 0.4$. 

Thus, the length of each section in decimal form, correct to 2 decimal places, is $0.40$ metres.

\section*{Question 66}
\textbf{Metadata}

\begin{itemize}
  \item Question ID: P6-FrDivPP\_P5-FrAddMix\_GPT4.1\_Education\_05
  \item Primary KC: FRACTIONS | Division | dividing a whole number/proper fraction by a proper fraction
  \item Secondary KC: FRACTIONS | Addition | adding mixed numbers
  \item Topic: Education
  \item Grade: Primary 6
\end{itemize}

\textbf{Question}

\textbf{Mrs Tan is preparing teaching resources for her Primary 6 class.}\newline
She bought 3 large stacks of colour paper. Each stack contains $2\dfrac{1}{2}$ packs of paper. She wants to divide all the paper equally among folders, and each folder uses $\dfrac{3}{4}$ pack of paper.}\newline
(a) How many folders can Mrs Tan prepare using all the paper?\newline
(b) If she finds 1 more stack of colour paper containing $1\dfrac{2}{3}$ packs and adds it to the total, how many folders can she prepare now?

\textbf{Solution}

\textbf{(a) Find the total amount of packs of paper from 3 stacks:}\newline
Each stack = $2\dfrac{1}{2} = \frac{5}{2}$\newline
Total packs = $3 \times \frac{5}{2} = \frac{15}{2}$\newline
\textbf{Number of folders possible:}\newline
Each folder uses $\frac{3}{4}$ pack.\newline
Number of folders $= \frac{15}{2} \div \frac{3}{4} = \frac{15}{2} \times \frac{4}{3} = \frac{15 \times 4}{2 \times 3} = \frac{60}{6} = 10$\newline
\boxed{10\text{ folders}}\newline
\textbf{(b) She finds 1 more stack containing $1\dfrac{2}{3} = \frac{5}{3}$ packs.}\newline
Add to previous total: $\frac{15}{2} + \frac{5}{3}$\newline
Find common denominator (6):\newline
$\frac{15}{2} = \frac{45}{6}$, $\frac{5}{3} = \frac{10}{6}$\newline
Total = $\frac{45}{6} + \frac{10}{6} = \frac{55}{6}$\newline
Folders $= \frac{55}{6} \div \frac{3}{4} = \frac{55}{6} \times \frac{4}{3} = \frac{55 \times 4}{6 \times 3} = \frac{220}{18} = \frac{110}{9}\approx 12.22$\newline
She can prepare $12$ full folders.\newline
\boxed{12\text{ folders}}

\section*{Question 67}
\textbf{Metadata}

\begin{itemize}
  \item Question ID: P6-FrDivPN\_P6-FrDivPP\_GPT4.1\_Household finance\_05
  \item Primary KC: FRACTIONS | Division | dividing a proper fraction by a whole number
  \item Secondary KC: FRACTIONS | Division | dividing a whole number/proper fraction by a proper fraction
  \item Topic: Household finance
  \item Grade: Primary 6
\end{itemize}

\textbf{Question}

\textbf{Word Problem:} 

Sarah has $4$ litres of fruit juice in her kitchen. She wants to pour the juice equally into bottles, each holding $\dfrac{2}{5}$ of a litre. 

(a) How many bottles can she fill completely? 

(b) If she further shares one such bottle equally among $3$ friends, what fraction of a litre of juice does each friend get?

\textbf{Solution}

\textbf{Solution:}

(a) To find out how many bottles Sarah can fill completely:

Number of bottles $= \dfrac{4}{\dfrac{2}{5}} = 4 \times \dfrac{5}{2} = \dfrac{20}{2} = 10$

So, Sarah can fill $\boxed{10}$ bottles completely.

(b) One bottle contains $\dfrac{2}{5}$ litre. To share this equally among 3 friends:

Amount each friend gets $= \dfrac{2}{5} \div 3 = \dfrac{2}{5} \times \dfrac{1}{3} = \dfrac{2}{15}$ litre

Each friend receives $\boxed{\dfrac{2}{15}}$ litre of juice.

\section*{Question 68}
\textbf{Metadata}

\begin{itemize}
  \item Question ID: P6-FrDivPP\_P5-FrMulMixN\_GPT4.1\_Manufacturing\_05
  \item Primary KC: FRACTIONS | Division | dividing a whole number/proper fraction by a proper fraction
  \item Secondary KC: FRACTIONS | Multiplication | multiplying a mixed number and a whole number
  \item Topic: Manufacturing
  \item Grade: Primary 6
\end{itemize}

\textbf{Question}

A furniture factory uses $\dfrac{3}{4}$ metres of wood to make one table. The factory has a long plank that is $6$ metres long.

(a) How many such tables can the factory make from the $6$-metre plank?

Later, the factory decides to use $2\dfrac{1}{2}$ metres of wood to assemble one large bench. If the workers want to assemble $3$ such benches, how much wood will they need in total?

\textbf{Solution}

(a) Number of tables that can be made from the $6$-metre plank:

$6 \div \dfrac{3}{4} = 6 \times \dfrac{4}{3} = \dfrac{24}{3} = 8$

So, the factory can make $\boxed{8}$ tables.

(b) Amount of wood needed for $3$ benches:

$2\dfrac{1}{2} = \dfrac{5}{2}$

Total wood needed $= 3 \times \dfrac{5}{2} = \dfrac{15}{2} = 7\dfrac{1}{2}$ metres

So, they will need $\boxed{7\dfrac{1}{2}}$ metres of wood to assemble $3$ benches.

\section*{Question 69}
\textbf{Metadata}

\begin{itemize}
  \item Question ID: P6-FrDivPP\_P5-FrMulMixN\_GPT4.1\_Sports\_05
  \item Primary KC: FRACTIONS | Division | dividing a whole number/proper fraction by a proper fraction
  \item Secondary KC: FRACTIONS | Multiplication | multiplying a mixed number and a whole number
  \item Topic: Sports
  \item Grade: Primary 6
\end{itemize}

\textbf{Question}

A basketball coach has $15$ litres of sports drink to be shared equally among his team. If each player receives $\dfrac{3}{4}$ litre per match, how many players can he serve for one match? 

After distributing the drinks, the coach realizes he needs to prepare more for the next tournament. He decides to prepare $2\dfrac{1}{2}$ times as much sports drink for $4$ teams of the same size. How many litres of sports drink will he need in total for the next tournament?

\textbf{Solution}

First, find out how many players can be served with $15$ litres if each player gets $\dfrac{3}{4}$ litre:

Number of players $= 15 \div \dfrac{3}{4}$ 

To divide by a fraction, multiply by its reciprocal:

$15 \div \dfrac{3}{4} = 15 \times \dfrac{4}{3} = \dfrac{60}{3} = 20$

So, the coach can serve $20$ players for one match.

For the next tournament, he wants to make $2\dfrac{1}{2}$ times as much sports drink for $4$ teams of the same size.

First, convert $2\dfrac{1}{2}$ to an improper fraction:

$2\dfrac{1}{2} = \dfrac{5}{2}$

Amount needed for one team $= 15$ litres. 

Total for $4$ teams $= 15 \times 4 = 60$ litres.

Amount to prepare $= 60 \times \dfrac{5}{2} = 30 \times 5 = 150$ litres.

Therefore, the coach will need $\boxed{150}$ litres of sports drink for the next tournament.

\section*{Question 70}
\textbf{Metadata}

\begin{itemize}
  \item Question ID: P6-RoFndRoWN\_P3-WNSub4d\_GPT4.1\_Household finance\_05
  \item Primary KC: RATIO | Finding ratio | finding the ratio of two or three given whole numbers
  \item Secondary KC: WHOLE NUMBERS | Subtraction | subtraction up to 4 digits
  \item Topic: Household finance
  \item Grade: Primary 6
\end{itemize}

\textbf{Question}

A family spent \$2,150 on groceries and \$1,360 on utility bills in June. If their total household expenses for June were \$4,500, find the ratio of the amount spent on groceries to the amount spent on utility bills to the amount left after these two expenses. Express your answer in its simplest form.

\textbf{Solution}

Let us first find the amount left after groceries and utility bills.
$
\text{Amount left} = \$4,500 - (\$2,150 + \$1,360) = \$4,500 - \$3,510 = \$990
$

Now, the three amounts are:
- Groceries: \$2,150
- Utility Bills: \$1,360
- Amount Left: \$990

So, the required ratio is:
$
2,150 : 1,360 : 990
$

To simplify, find the highest common factor (HCF) for the three numbers.

Let us check the HCF:
- HCF of 2,150 and 1,360 is 10 (since 2,150 = 215 × 10, 1,360 = 136 × 10)
- Check if 10 divides 990: 990 ÷ 10 = 99

So, 10 is common.
Dividing each by 10:
$
\frac{2,150}{10} : \frac{1,360}{10} : \frac{990}{10} = 215 : 136 : 99
$

The simplest ratio is:
$
\boxed{215 : 136 : 99}
$

\section*{Question 71}
\textbf{Metadata}

\begin{itemize}
  \item Question ID: P6-PcFndWN\_P3-WNSub4d\_GPT4.1\_Farming\_05
  \item Primary KC: PERCENTAGE | Finding the whole | finding the whole given a part and the percentage
  \item Secondary KC: WHOLE NUMBERS | Subtraction | subtraction up to 4 digits
  \item Topic: Farming
  \item Grade: Primary 6
\end{itemize}

\textbf{Question}

A farm harvested some eggs in a week. On Monday, the farm sold 35\% of the eggs, which was 2100 eggs. After selling the eggs, there were 4800 eggs left. How many eggs were harvested on the farm that week?

\textbf{Solution}

Let the total number of eggs harvested be $x$.

35\% of $x$ were sold: $0.35x = 2100$

So, $x = \frac{2100}{0.35} = 6000$

After selling, eggs left: $x - 2100 = 4800$
Check: $6000 - 2100 = 3900$. However, the eggs left given is 4800; this suggests a mistake. Let's review:

Let $x$ be the total eggs.
Eggs left after selling $= x - 2100 = 4800$
So, $x = 4800 + 2100 = 6900$

Let's check if 35\% of 6900 is 2100:
$6900 \times 0.35 = 2415$, not 2100. So, let's try using a different method:

Let total eggs be $x$.
Eggs sold: $0.35x$
Eggs left: $x - 0.35x = 0.65x = 4800$

So, $0.65x = 4800$

$x = \frac{4800}{0.65} = 7384.62$, since whole eggs, $x = 7385$ (rounded to nearest whole egg).

Now, find 35\% of $7385$: $0.35 \times 7385 = 2584.75 \approx 2585$ eggs.

Update word problem slightly for correctness:

\textbf{A farm harvested some eggs in a week. On Monday, the farm sold 35\% of the eggs. After selling the eggs, there were 4800 eggs left. How many eggs were harvested on the farm that week?}

\textbf{Solution:}
Let the total number of eggs harvested be $x$.

After selling 35\%, eggs left $= 100\% - 35\% = 65\%$ of the eggs.
So, $0.65x = 4800$

Solve for $x$:
$x = \frac{4800}{0.65} = 7384.62$
Since the number of eggs must be a whole number, the farm harvested $7385$ eggs that week.

\section*{Question 72}
\textbf{Metadata}

\begin{itemize}
  \item Question ID: P6-PcFndChg\_P3-WNAdd4d\_GPT4.1\_Services\_05
  \item Primary KC: PERCENTAGE | Finding change | finding percentage increase/decrease
  \item Secondary KC: WHOLE NUMBERS | Addition | addition up to 4 digits
  \item Topic: Services
  \item Grade: Primary 6
\end{itemize}

\textbf{Question}

A taxi company increased the price of a ride from \$850 to \$1020 due to rising fuel costs.\
\\
(a) By how much did the price increase?\\
(b) What is the percentage increase in the price?\\
\\
If Simon took 3 rides after the price increase, what was the total amount he paid?

\textbf{Solution}

(a) Price increase = New price $-$ Old price = $1020 - 850 = 170$\\
(b) Percentage increase $= \left( \dfrac{170}{850} \right) \times 100\% = 20\%$\\
For 3 rides, total amount paid $= 1020 \times 3 = 3060$\\
\\
\textbf{Answers:}\\
(a) $170$\\
(b) $20\%$\\
(c) $3060$

\section*{Question 73}
\textbf{Metadata}

\begin{itemize}
  \item Question ID: P6-PcFndChg\_P3-WNSub4d\_GPT4.1\_Sports\_05
  \item Primary KC: PERCENTAGE | Finding change | finding percentage increase/decrease
  \item Secondary KC: WHOLE NUMBERS | Subtraction | subtraction up to 4 digits
  \item Topic: Sports
  \item Grade: Primary 6
\end{itemize}

\textbf{Question}

A sports shop sold $1\ 250$ badminton rackets in January. In February, the shop sold $1\ 850$ badminton rackets.\
\
(a) How many more badminton rackets did the shop sell in February than in January?\\
(b) What was the percentage increase in the number of badminton rackets sold from January to February?

\textbf{Solution}

(a) Number of additional badminton rackets sold: \\ 
$1\ 850 - 1\ 250 = 600$ \\ 
The shop sold $600$ more badminton rackets in February than in January.\$1em]
(b) Percentage increase = $\dfrac{\text{Increase}}{\text{Original amount}} \times 100\%$ \\ 
$= \dfrac{600}{1\ 250} \times 100\%$ \\ 
$= 0.48 \times 100\%$ \\ 
$= 48\%$ \\ 
The number of badminton rackets sold increased by $48\%$ from January to February.

\section*{Question 74}
\textbf{Metadata}

\begin{itemize}
  \item Question ID: P6-FrDivPN\_P4-FrRepSet\_GPT4.1\_Farming\_05
  \item Primary KC: FRACTIONS | Division | dividing a proper fraction by a whole number
  \item Secondary KC: FRACTIONS | Representation and concept | fraction as part of a set 
  \item Topic: Farming
  \item Grade: Primary 6
\end{itemize}

\textbf{Question}

A farmer has a field that is divided into $\frac{3}{4}$ of an acre of land for planting carrots. He wants to divide this carrot plot equally among 5 workers to take care of. 

(a) What fraction of an acre will each worker take care of? 

(b) Represent the answer as the fraction of the total field each worker is responsible for.

\textbf{Solution}

Let the total area for carrots be $\frac{3}{4}$ acre, to be shared equally among 5 workers. 

(a) The amount each worker gets is $\frac{3}{4} \div 5 = \frac{3}{4} \times \frac{1}{5} = \frac{3}{20}$ acre.

(b) Therefore, each worker is responsible for $\frac{3}{20}$ of an acre. Since the field for carrots is $\frac{3}{4}$ of the whole field, the proportion of the entire carrot plot covered by each worker is $\frac{3}{20}\div \frac{3}{4} = \frac{3}{20} \times \frac{4}{3} = \frac{4}{20} = \frac{1}{5}$. So, each worker is responsible for $\frac{1}{5}$ of the carrot plot.

\section*{Question 75}
\textbf{Metadata}

\begin{itemize}
  \item Question ID: P6-FrDivPN\_P4-FrRepSet\_GPT4.1\_Education\_05
  \item Primary KC: FRACTIONS | Division | dividing a proper fraction by a whole number
  \item Secondary KC: FRACTIONS | Representation and concept | fraction as part of a set 
  \item Topic: Education
  \item Grade: Primary 6
\end{itemize}

\textbf{Question}

\textbf{Mathematics Project Sharing}

A class of Primary 6 students worked on mathematics projects. Their teacher had a folder containing $\frac{3}{4}$ of all the project reports from the class. She wanted to distribute these reports equally among 5 groups for a group activity.

What fraction of the total class's project reports did each group receive? Express your answer in its simplest form.

\textbf{Solution}

Let the total number of project reports from the class be 1 (as a whole).

The teacher has $\frac{3}{4}$ of all the reports.

She wants to divide $\frac{3}{4}$ equally among 5 groups.

So, each group receives:

$
\frac{3}{4} \div 5 = \frac{3}{4} \times \frac{1}{5} = \frac{3}{20}
$

\textbf{Answer:}

Each group received $\frac{3}{20}$ of the total class's project reports.

\section*{Question 76}
\textbf{Metadata}

\begin{itemize}
  \item Question ID: P6-FrDivPN\_P5-FrCnv2Dc\_GPT4.1\_Manufacturing\_05
  \item Primary KC: FRACTIONS | Division | dividing a proper fraction by a whole number
  \item Secondary KC: FRACTIONS | Conversion to decimals | expressing fractions as decimals
  \item Topic: Manufacturing
  \item Grade: Primary 6
\end{itemize}

\textbf{Question}

A machine in a factory can pack \( \frac{3}{4} \) of a box of cookies in 1 minute. If 5 workers take turns to operate the machine, with each worker working for an equal amount of time, how much of a box of cookies does each worker pack in 1 minute? Express your answer as a fraction in simplest form, and then as a decimal.

\textbf{Solution}

Each worker operates the machine for equal periods, so the amount each worker packs in 1 minute is obtained by dividing the total amount packed by 5. 

Amount packed by one worker in 1 minute:
$
\frac{3}{4} \div 5 = \frac{3}{4} \times \frac{1}{5} = \frac{3}{20}
$
So, each worker packs \(\frac{3}{20}\) of a box of cookies in 1 minute.

Expressing \(\frac{3}{20}\) as a decimal:
$
\frac{3}{20} = 0.15
$

\textbf{Answer:} Each worker packs \(\frac{3}{20}\) of a box of cookies in 1 minute, which is 0.15 as a decimal.

\section*{Question 77}
\textbf{Metadata}

\begin{itemize}
  \item Question ID: P6-FrDivPN\_P5-FrSubMix\_GPT4.1\_Food\_05
  \item Primary KC: FRACTIONS | Division | dividing a proper fraction by a whole number
  \item Secondary KC: FRACTIONS | Subtraction | subtracting mixed numbers
  \item Topic: Food
  \item Grade: Primary 6
\end{itemize}

\textbf{Question}

Sarah baked a loaf of banana bread and cut it into $\frac{8}{9}$ of a loaf to share with friends. She wants to divide this amount equally among 4 friends. 

(a) How much of the loaf does each friend get?

Later, one friend ate $1\frac{1}{9}$ loaves of banana bread in total, including what she got from Sarah. 

(b) How much more banana bread did she eat besides the piece Sarah gave her?


\textbf{Solution}

(a) Amount each friend receives:
\\
$\frac{8}{9} \div 4 = \frac{8}{9} \times \frac{1}{4} = \frac{8}{36} = \frac{2}{9}$

So, each friend gets $\frac{2}{9}$ of a loaf.

(b) Amount of banana bread the friend ate besides Sarah's piece:
\\
$1\frac{1}{9} - \frac{2}{9} = \frac{10}{9} - \frac{2}{9} = \frac{8}{9}$

So, the friend ate $\frac{8}{9}$ of a loaf in addition to what Sarah gave.

\section*{Question 78}
\textbf{Metadata}

\begin{itemize}
  \item Question ID: P6-PcFndWN\_P4-WNMul4d1d\_GPT4.1\_Household finance\_05
  \item Primary KC: PERCENTAGE | Finding the whole | finding the whole given a part and the percentage
  \item Secondary KC: WHOLE NUMBERS | Multiplication | multiplication up to 4 digits by 1 digit or up to 3 digits by 2 digits
  \item Topic: Household finance
  \item Grade: Primary 6
\end{itemize}

\textbf{Question}

Aisha's family spent $25\%$ of their monthly income on groceries, which amounted to \$720. \\ 
If Aisha's family also spends 3 times as much on rent as they do on groceries each month, what is their total monthly income?

\textbf{Solution}

Let the total monthly income be \( x \).\\
Given that 25% of the income is spent on groceries: \\ 
$ \frac{25}{100} \times x = 720 $
$ 0.25x = 720 $
$ x = \frac{720}{0.25} $
$ x = 2880 $
So, their monthly income is \$2880.\\
\\
Aisha's family spends 3 times as much on rent as on groceries: \\ 
Rent = \( 720 \times 3 = 2160 \).\\
\\
You were asked for the total monthly income, which is \$2880. \\

\section*{Question 79}
\textbf{Metadata}

\begin{itemize}
  \item Question ID: P6-AgRepLrEx\_P6-AgEvlLrEx\_GPT4.1\_Household finance\_05
  \item Primary KC: ALGEBRA | Representation and concept | translation of simple real-world situations into linear algebraic expressions
  \item Secondary KC: ALGEBRA | Evaluation | evaluating simple linear expressions by substitution
  \item Topic: Household finance
  \item Grade: Primary 6
\end{itemize}

\textbf{Question}

Aisha wants to buy some packet drinks for a picnic. Each packet drink costs \$p. She also wants to buy a sandwich that costs \$3. \\ 
(a) Write a linear expression in terms of \( p \) for the total cost if she buys 4 packet drinks and 1 sandwich. \\ 
(b) If each packet drink costs \$2, how much does Aisha need to pay in total?

\textbf{Solution}

(a) The total cost for 4 packet drinks and 1 sandwich is: \\ 
\( 4p + 3 \) \\ 
(b) Substitute \( p = 2 \): \\ 
Total cost = \( 4\times2 + 3 = 8 + 3 = 11 \) \\ 
Aisha needs to pay \$11 in total.

\section*{Question 80}
\textbf{Metadata}

\begin{itemize}
  \item Question ID: P6-PcFndWN\_P3-WNSub4d\_GPT4.1\_Food\_05
  \item Primary KC: PERCENTAGE | Finding the whole | finding the whole given a part and the percentage
  \item Secondary KC: WHOLE NUMBERS | Subtraction | subtraction up to 4 digits
  \item Topic: Food
  \item Grade: Primary 6
\end{itemize}

\textbf{Question}

Sarah bought a box of cookies. She ate 15\% of the cookies and had 85 cookies left. Later, her brother ate 28 cookies from what was left. How many cookies were in the box at first?

\textbf{Solution}

Let the total number of cookies in the box at first be $x$. \\ 
Sarah ate 15\% of the cookies, so 85 cookies represent the remaining 85\% of the cookies.\\ 
Thus, $85\%$ of $x$ is 85 cookies:
$
0.85x = 85 
$
$
x = \frac{85}{0.85} = 100
$
So there were 100 cookies at first. \\ 

To check, Sarah ate $0.15 \times 100 = 15$ cookies, leaving $100 - 15 = 85$ cookies. 

After her brother ate 28 cookies, there are $85 - 28 = 57$ cookies left. 

Therefore, the answer is $\boxed{100}$ cookies.

\section*{Question 81}
\textbf{Metadata}

\begin{itemize}
  \item Question ID: P6-FrDivPN\_P5-FrMulMixN\_GPT4.1\_Digital ecconomy\_05
  \item Primary KC: FRACTIONS | Division | dividing a proper fraction by a whole number
  \item Secondary KC: FRACTIONS | Multiplication | multiplying a mixed number and a whole number
  \item Topic: Digital ecconomy
  \item Grade: Primary 6
\end{itemize}

\textbf{Question}

\textbf{Word Problem:} \\ 
Ahmad has $2\frac{1}{2}$ gigabytes (GB) of mobile data each day and wants to use the same amount of data for each of his 4 devices every day. \\ 
(a) How much mobile data does each device get to use per day? (Give your answer in GB) \\ 
If Ahmad uses the same daily amount for 6 days, \\ 
(b) How much total data does he use during the 6 days? (Give your answer in GB)

\textbf{Solution}

\textbf{Solution:} \\ 
\underline{(a)}\\ 
Total data per day = $2\frac{1}{2}$ GB \\ 
Number of devices = 4 \\ 
Amount of data per device per day $= 2\frac{1}{2} \div 4$ \\ 
First, convert $2\frac{1}{2}$ to improper fraction: \\ 
$2\frac{1}{2} = \frac{5}{2}$ \\ 
So, $\frac{5}{2} \div 4 = \frac{5}{2} \times \frac{1}{4} = \frac{5}{8}$ \\ 
Each device gets $\frac{5}{8}$ GB per day. \\ 
\underline{(b)} \\ 
Total data per day = $2\frac{1}{2}$ GB \\ 
Number of days = 6 \\ 
Total data used $= 2\frac{1}{2} \times 6$ \\ 
$2\frac{1}{2} = \frac{5}{2}$ \\ 
$\frac{5}{2} \times 6 = \frac{5 \times 6}{2} = \frac{30}{2} = 15$ \\ 
Ahmad uses $15$ GB in 6 days.

\section*{Question 82}
\textbf{Metadata}

\begin{itemize}
  \item Question ID: P6-FrDivPN\_P2-FrCmp\_GPT4.1\_Transporation\_05
  \item Primary KC: FRACTIONS | Division | dividing a proper fraction by a whole number
  \item Secondary KC: FRACTIONS | Comparison and ordering | comparing and ordering fractions
  \item Topic: Transporation
  \item Grade: Primary 6
\end{itemize}

\textbf{Question}

A bus driver has $\frac{3}{5}$ of a tank of fuel left. She plans to divide this remaining fuel equally among 3 identical shorter bus trips for the day.

(a) How much fuel, as a fraction, does each trip receive?

(b) After dividing the fuel equally, she realises another bus driver has $\frac{2}{5}$ of a tank left to use for his trips. Write the amount of fuel each bus trip receives in order from least to greatest.

\textbf{Solution}

(a) To find how much fuel each trip receives, divide $\frac{3}{5}$ by $3$:

$
\frac{3}{5} \div 3 = \frac{3}{5} \times \frac{1}{3} = \frac{3 \times 1}{5 \times 3} = \frac{3}{15} = \frac{1}{5}
$

So, each trip receives $\frac{1}{5}$ of a tank of fuel.

(b) The other bus driver has $\frac{2}{5}$ of a tank to use for one trip.

Now compare $\frac{1}{5}$ (fuel per trip for the first bus driver) and $\frac{2}{5}$ (fuel for the second bus driver):

Since the denominators are the same, $\frac{1}{5} < \frac{2}{5}$.

Therefore, in order from least to greatest, the amounts are:

$
\frac{1}{5},\ \frac{2}{5}
$

\section*{Question 83}
\textbf{Metadata}

\begin{itemize}
  \item Question ID: P6-FrDivPN\_P2-FrCmp\_GPT4.1\_Manufacturing\_05
  \item Primary KC: FRACTIONS | Division | dividing a proper fraction by a whole number
  \item Secondary KC: FRACTIONS | Comparison and ordering | comparing and ordering fractions
  \item Topic: Manufacturing
  \item Grade: Primary 6
\end{itemize}

\textbf{Question}

\textbf{Problem:}

A factory produces 3\frac{1}{2} kg of chocolate, which is to be packed equally into 4 boxes. 

(a) What is the weight of chocolate in each box?

(b) The manager also has another type of chocolate where he divides 5\frac{1}{4} kg equally into 7 boxes. 

Which type of chocolate box contains more chocolate, and by how much?

\textbf{Solution}

\textbf{Solution:}

(a) Total chocolate weight: $3\frac{1}{2} = \frac{7}{2}$ kg.

Divided into 4 boxes:

$\text{Weight in each box} = \frac{7}{2} \div 4 = \frac{7}{2} \times \frac{1}{4} = \frac{7}{8}$ kg.$

(b) Second type: $5\frac{1}{4} = \frac{21}{4}$ kg.

Divided into 7 boxes:

$\text{Weight in each box} = \frac{21}{4} \div 7 = \frac{21}{4} \times \frac{1}{7} = \frac{21}{28} = \frac{3}{4}$ kg.$

\text{Comparing:}

$\frac{7}{8}$ kg (first type) and $\frac{3}{4}$ kg (second type).$

Convert $\frac{3}{4}$ to eighths: $\frac{3}{4} = \frac{6}{8}$.

So $\frac{7}{8} > \frac{6}{8}$.

$\boxed{\frac{7}{8} - \frac{6}{8} = \frac{1}{8}}$ kg

\textbf{Answer:} The first type of chocolate box contains more chocolate by $\frac{1}{8}$ kg.

\section*{Question 84}
\textbf{Metadata}

\begin{itemize}
  \item Question ID: P6-FrDivPN\_P6-FrDivPP\_GPT4.1\_Sports\_05
  \item Primary KC: FRACTIONS | Division | dividing a proper fraction by a whole number
  \item Secondary KC: FRACTIONS | Division | dividing a whole number/proper fraction by a proper fraction
  \item Topic: Sports
  \item Grade: Primary 6
\end{itemize}

\textbf{Question}

A sports club organises a 3-hour basketball training session. \frac{3}{4} of the session is used for practice drills, and the remaining time is for playing games.\

(a) If this practice drills time is divided equally among 5 teams, how many hours does each team get to practise drills?\

(b) During the games segment, each game lasts for \frac{3}{8} hour. How many games can be played during the games segment?

\textbf{Solution}

(a) The practice drills time is \frac{3}{4} of 3 hours: \\
\frac{3}{4} \times 3 = \frac{9}{4} = 2\frac{1}{4}\ \text{hours}.

This time is divided equally among 5 teams:

\frac{2\frac{1}{4}}{5} = \frac{9}{4} \div 5 = \frac{9}{4} \times \frac{1}{5} = \frac{9}{20}\ \text{hour for each team.}

(b) The games segment takes up the remaining time:

Total session: 3 hours \\
Practice drills: 2\frac{1}{4} hours \\
Games segment: 3 - 2\frac{1}{4} = \frac{12}{4} - \frac{9}{4} = \frac{3}{4}\ \text{hour}

Each game lasts \frac{3}{8} hour. Number of games possible:

\frac{3}{4} \div \frac{3}{8} = \frac{3}{4} \times \frac{8}{3} = 2

\text{So 2 games can be played during the games segment.}

\section*{Question 85}
\textbf{Metadata}

\begin{itemize}
  \item Question ID: P6-PcFndChg\_P4-WNDiv4d1d\_GPT4.1\_Household finance\_05
  \item Primary KC: PERCENTAGE | Finding change | finding percentage increase/decrease
  \item Secondary KC: WHOLE NUMBERS | Division | division up to 4 digits by 1 digit
  \item Topic: Household finance
  \item Grade: Primary 6
\end{itemize}

\textbf{Question}

\textbf{A family used to spend \$2880 on electricity every year. This year, they decided to use less electricity and managed to reduce their spending by 25\%.} \\
\textbf{(a) How much did the family save this year?} \\
\textbf{(b) The family decided to divide the total amount saved equally among 3 children to deposit into their savings accounts. How much did each child receive?}

\textbf{Solution}

\textbf{(a) Amount saved:} \\
\text{Percentage saved} = 25\% \\
\text{Amount saved} = 25\% \times \$2880 = \frac{25}{100} \times 2880 = \$720 \\

\textbf{(b) Amount each child received:} \\
\text{Total saved} = \$720 \\
\text{Number of children} = 3 \\
\text{Amount per child} = \frac{720}{3} = \$240 \\

\boxed{\text{The family saved \$720, and each child received \$240.}}

\section*{Question 86}
\textbf{Metadata}

\begin{itemize}
  \item Question ID: P6-PcFndChg\_P3-WNSub4d\_GPT4.1\_Education\_05
  \item Primary KC: PERCENTAGE | Finding change | finding percentage increase/decrease
  \item Secondary KC: WHOLE NUMBERS | Subtraction | subtraction up to 4 digits
  \item Topic: Education
  \item Grade: Primary 6
\end{itemize}

\textbf{Question}

A primary school had 1240 students last year. This year, the number of students decreased to 1120 students. 

(a) How many fewer students are there this year compared to last year?

(b) What is the percentage decrease in the number of students from last year to this year?

\textbf{Solution}

(a) Number of students last year = 1240

Number of students this year = 1120

Number of students decreased = 1240 - 1120 = 120

So, there are 120 fewer students this year.

(b) Percentage decrease $= \frac{\text{amount of decrease}}{\text{original number}} \times 100\%$

$= \frac{120}{1240} \times 100\%$

$= 0.09677 \times 100\%$

$= 9.677\%$

Rounded to 1 decimal place, the percentage decrease is $9.7\%$.

\section*{Question 87}
\textbf{Metadata}

\begin{itemize}
  \item Question ID: P6-FrDivPN\_P5-FrMulMixN\_GPT4.1\_Leisure\_05
  \item Primary KC: FRACTIONS | Division | dividing a proper fraction by a whole number
  \item Secondary KC: FRACTIONS | Multiplication | multiplying a mixed number and a whole number
  \item Topic: Leisure
  \item Grade: Primary 6
\end{itemize}

\textbf{Question}

During the June holidays, Sarah is making lemonade. She has $\frac{3}{4}$ litres of syrup to use each day. She wants to share this amount evenly among her 3 friends to make individual drinks for them.

(a) How much syrup will each friend get?

On another day, Sarah makes lemonade for a party using $2\frac{1}{2}$ litres of syrup each day for 4 days.

(b) How much syrup does she use in total for the 4 days?

\textbf{Solution}

(a) Amount of syrup each friend gets:

Sarah has $\frac{3}{4}$ litres of syrup for the day and needs to divide it equally among 3 friends:

$\frac{3}{4} \div 3 = \frac{3}{4} \times \frac{1}{3} = \frac{3 \times 1}{4 \times 3} = \frac{3}{12} = \frac{1}{4}$

So each friend gets $\frac{1}{4}$ litre of syrup.

(b) Total amount of syrup used over 4 days:

$2\frac{1}{2}$ litres $= \frac{5}{2}$ litres

$\frac{5}{2} \times 4 = \frac{5 \times 4}{2} = \frac{20}{2} = 10$

So Sarah uses 10 litres of syrup in total for the 4 days.

\section*{Question 88}
\textbf{Metadata}

\begin{itemize}
  \item Question ID: P6-PcFndChg\_P4-WNMul4d1d\_GPT4.1\_Manufacturing\_05
  \item Primary KC: PERCENTAGE | Finding change | finding percentage increase/decrease
  \item Secondary KC: WHOLE NUMBERS | Multiplication | multiplication up to 4 digits by 1 digit or up to 3 digits by 2 digits
  \item Topic: Manufacturing
  \item Grade: Primary 6
\end{itemize}

\textbf{Question}

A factory produces 2,350 toy cars in January. In February, the factory increases its production by 24\%. \\ 
(a) How many more toy cars did the factory produce in February than in January? \\ 
(b) What is the total number of toy cars produced in February?

\textbf{Solution}

(a) To find how many more toy cars were produced in February, first find 24\% of 2,350: \\ 
24\% \times 2,350 = \frac{24}{100} \times 2,350 = 0.24 \times 2,350 = 564 \\ 
So, the factory produced 564 more toy cars in February. \\ 
(b) Total toy cars produced in February: \\ 
2,350 + 564 = 2,914 \\ 
Answer: \\ 
(a) 564 more toy cars \\ 
(b) 2,914 toy cars

\section*{Question 89}
\textbf{Metadata}

\begin{itemize}
  \item Question ID: P6-FrDivPP\_P2-FrCmp\_GPT4.1\_Sports\_05
  \item Primary KC: FRACTIONS | Division | dividing a whole number/proper fraction by a proper fraction
  \item Secondary KC: FRACTIONS | Comparison and ordering | comparing and ordering fractions
  \item Topic: Sports
  \item Grade: Primary 6
\end{itemize}

\textbf{Question}

Sarah is organising a long-distance running event. She has 4 litres of isotonic drink to provide to the participants. Each runner gets $\frac{2}{5}$ litre of the drink. 

(a) How many runners can she serve with 4 litres of the isotonic drink?

(b) After the race, Sarah has 3 types of isotonic drinks left: $\frac{3}{4}$ litre of Brand A, $\frac{5}{8}$ litre of Brand B, and $\frac{2}{3}$ litre of Brand C. List the brands in order from the most to the least leftover drink.

\textbf{Solution}

(a) Number of runners Sarah can serve:

$4 \div \dfrac{2}{5} = 4 \times \dfrac{5}{2} = \dfrac{20}{2} = 10$

So, Sarah can serve \boxed{10} runners.

(b) Compare the leftover amounts:

Convert all to a common denominator, 24:

- Brand A: $\dfrac{3}{4} = \dfrac{18}{24}$
- Brand B: $\dfrac{5}{8} = \dfrac{15}{24}$
- Brand C: $\dfrac{2}{3} = \dfrac{16}{24}$

So, ordering from the most to the least:

Brand A ($18/24$), Brand C ($16/24$), Brand B ($15/24$).

Final answer: \boxed{\text{Brand A, Brand C, Brand B}}

\section*{Question 90}
\textbf{Metadata}

\begin{itemize}
  \item Question ID: P6-AgSlvLrN\_P6-AgRepLrEx\_GPT4.1\_Digital ecconomy\_05
  \item Primary KC: ALGEBRA | Solving simple linear equations | solving simple linear equations involving whole number coefficient only
  \item Secondary KC: ALGEBRA | Representation and concept | translation of simple real-world situations into linear algebraic expressions
  \item Topic: Digital ecconomy
  \item Grade: Primary 6
\end{itemize}

\textbf{Question}

A student buys e-books for $5$ dollars each from an online store. She wants to buy enough e-books for herself and two friends, so that everyone gets the same number of e-books. If she spends a total of $45$ dollars on e-books, how many e-books does each person get? \\ 
Let $x$ be the number of e-books each person receives. Write a linear equation to represent this situation and solve for $x$.

\textbf{Solution}

Let $x$ be the number of e-books each person gets.\\
There are the student and two friends, so $3$ persons in total.\\ 
Each e-book costs $5$ dollars. The total number of e-books bought is $3x$.\\
The total cost is $5 \times 3x = 45$.\\
So, $15x = 45$.\\
Divide both sides by $15$:\\
$x = \frac{45}{15}$\\
$x = 3$\\
Therefore, each person gets $3$ e-books.

\section*{Question 91}
\textbf{Metadata}

\begin{itemize}
  \item Question ID: P6-FrDivPN\_P5-FrSubMix\_GPT4.1\_Household finance\_05
  \item Primary KC: FRACTIONS | Division | dividing a proper fraction by a whole number
  \item Secondary KC: FRACTIONS | Subtraction | subtracting mixed numbers
  \item Topic: Household finance
  \item Grade: Primary 6
\end{itemize}

\textbf{Question}

Sarah had $\frac{3}{4}$ of a loaf of bread left at home. She decided to share this equally among 3 family members for breakfast. After breakfast, her mother brought home another $1\frac{1}{2}$ loaves of bread. Sarah and her brother then used $1\frac{1}{4}$ loaves of bread to make sandwiches for a picnic. 
\\
(a) How much bread did each family member get for breakfast?
\\
(b) After all these activities, how much bread is left at home?


\textbf{Solution}

Let us solve each part step by step.
\\
(a) Sarah divides $\frac{3}{4}$ of a loaf equally among 3 family members:
\\
$ 
\text{Amount each gets} = \frac{3}{4} \div 3 = \frac{3}{4} \times \frac{1}{3} = \frac{3}{12} = \frac{1}{4} 
$
Each family member gets $\frac{1}{4}$ of a loaf.
\\
(b) After breakfast, the bread left at home is $0$ (since all $\frac{3}{4}$ was consumed), but then her mother brings home $1\frac{1}{2}$ loaves:
\\
$ \text{Bread at home after mother returns} = 1\frac{1}{2} = \frac{3}{2} \text{ loaves} $
Next, Sarah and her brother use $1\frac{1}{4}$ loaves for sandwiches:
\\
$ \text{Bread left} = \frac{3}{2} - 1\frac{1}{4} 
= \frac{3}{2} - \frac{5}{4}
$
Express both fractions with common denominator:
$ \frac{3}{2} = \frac{6}{4} $
So,
$ \frac{6}{4} - \frac{5}{4} = \frac{1}{4} $
After all these activities, there is $\frac{1}{4}$ of a loaf of bread left at home.

\section*{Question 92}
\textbf{Metadata}

\begin{itemize}
  \item Question ID: P6-FrDivPN\_P3-FrSmp\_GPT4.1\_Household finance\_05
  \item Primary KC: FRACTIONS | Division | dividing a proper fraction by a whole number
  \item Secondary KC: FRACTIONS | Simplifying | expressing a fraction in its simplest form
  \item Topic: Household finance
  \item Grade: Primary 6
\end{itemize}

\textbf{Question}

Sarah has $\frac{5}{6}$ of a kilogram of flour. She wants to pack this flour equally into 4 containers for her family to use during the week. How much flour will be in each container? Express your answer in its simplest form.

\textbf{Solution}

To find how much flour will be in each container, divide the total amount of flour by the number of containers:

$
\text{Amount in each container} = \frac{5}{6} \div 4
$

Recall that dividing by a whole number is the same as multiplying by its reciprocal:

$
\frac{5}{6} \div 4 = \frac{5}{6} \times \frac{1}{4} = \frac{5 \times 1}{6 \times 4} = \frac{5}{24}
$

So, each container will have \( \frac{5}{24} \) of a kilogram of flour.

\section*{Question 93}
\textbf{Metadata}

\begin{itemize}
  \item Question ID: P6-RoFndDvqWN\_P6-RoFndTmWN\_GPT4.1\_Farming\_05
  \item Primary KC: RATIO | Finding divided quantities | dividing a quantity in a given ratio
  \item Secondary KC: RATIO | Finding a missing term | finding the missing term in a pair of equivalent ratios
  \item Topic: Farming
  \item Grade: Primary 6
\end{itemize}

\textbf{Question}

A farmer has $96$ kg of fertiliser to be shared between his two vegetable patches, Patch A and Patch B, in the ratio $5:7$.

(a) How many kilograms of fertiliser should Patch B receive?

Later, the farmer realised he needed to prepare another batch of fertiliser for his fruit trees, this time using a ratio equivalent to $5:7$. If the new ratio is $x:14$ and the farmer wants Patch A to receive $35$ kg, what is the value of $x$?


\textbf{Solution}

(a) The total number of parts is $5 + 7 = 12$ parts.

One part = $96 \div 12 = 8$ kg.

Patch B receives $7$ parts:

$7 \times 8 = 56$ kg.

\boxed{56 \text{ kg}}$

(b) The ratio for the new batch is $x:14$. This is equivalent to $5:7$.

Set up the proportion:

$\dfrac{x}{14} = \dfrac{5}{7}$

Cross-multiply:

$7x = 5 \times 14$

$7x = 70$

$x = 10$

Therefore, the value of $x$ is $\boxed{10}$.


\section*{Question 94}
\textbf{Metadata}

\begin{itemize}
  \item Question ID: P6-PcFndWN\_P4-WNMul4d1d\_GPT4.1\_Farming\_05
  \item Primary KC: PERCENTAGE | Finding the whole | finding the whole given a part and the percentage
  \item Secondary KC: WHOLE NUMBERS | Multiplication | multiplication up to 4 digits by 1 digit or up to 3 digits by 2 digits
  \item Topic: Farming
  \item Grade: Primary 6
\end{itemize}

\textbf{Question}

A farmer harvested a certain number of oranges from his orchard. He sold 35\% of the oranges to a fruit market, which totaled 1,260 oranges. Later, he packed the remaining oranges equally into 28 boxes. 

(a) How many oranges did the farmer harvest in total?

(b) How many oranges were packed into each box?

\textbf{Solution}

(a) Let the total number of oranges harvested be $x$.

He sold 35\% of $x$ = 1,260 oranges.

$0.35x = 1,260$

$x = \frac{1,260}{0.35}$

$x = 3,600$

So, the farmer harvested \boxed{3,600} oranges in total.

(b) Number of oranges left = Total harvested $-$ Sold = $3,600 - 1,260 = 2,340$

Number of boxes = 28

Number of oranges in each box = $2,340 \div 28$

Multiply to make division easier:

$28 \times 80 = 2,240$

$2,340 - 2,240 = 100$

$100 \div 28 = 3$ remainder 16

So $2,340 \div 28 = 83$ remainder 16, or $83\frac{16}{28} = 83\frac{4}{7}$

Thus, each box contained 83 oranges, with 16 oranges left over.

\boxed{83} oranges were packed into each box.

\section*{Question 95}
\textbf{Metadata}

\begin{itemize}
  \item Question ID: P6-FrDivPP\_P5-FrAddMix\_GPT4.1\_Farming\_05
  \item Primary KC: FRACTIONS | Division | dividing a whole number/proper fraction by a proper fraction
  \item Secondary KC: FRACTIONS | Addition | adding mixed numbers
  \item Topic: Farming
  \item Grade: Primary 6
\end{itemize}

\textbf{Question}

A farmer harvested $\frac{7}{8}$ of a basket of tomatoes in the morning. In the afternoon, he harvested $2\frac{1}{4}$ baskets of tomatoes. If he wants to place all the tomatoes into baskets that can each hold $\frac{3}{4}$ of a basket, how many such baskets will he need to use in total? (Give your answer as a whole number, rounding up if necessary.)

\textbf{Solution}

First, add the total amount of tomatoes harvested:

$$
\text{Total harvested} = \frac{7}{8} + 2\frac{1}{4}
$$

Convert $2\frac{1}{4}$ to an improper fraction:
$$
2\frac{1}{4} = \frac{9}{4}
$$

Find a common denominator:
$$
\frac{7}{8} + \frac{9}{4} = \frac{7}{8} + \frac{18}{8} = \frac{25}{8}
$$

So, the farmer harvested $\frac{25}{8}$ baskets of tomatoes in total.

He wants to divide all these tomatoes equally into baskets that each hold $\frac{3}{4}$ of a basket:
$$
\text{Number of baskets} = \frac{\frac{25}{8}}{\frac{3}{4}} = \frac{25}{8} \times \frac{4}{3} = \frac{25 \times 4}{8 \times 3} = \frac{100}{24} = \frac{25}{6} \approx 4.17
$$

Since he cannot have a fraction of a basket, he will need $5$ baskets to hold all the tomatoes.

\textbf{Answer:} He will need 5 baskets.

\section*{Question 96}
\textbf{Metadata}

\begin{itemize}
  \item Question ID: P6-AgRepLrEx\_P6-AgSmpLrEx\_GPT4.1\_Education\_05
  \item Primary KC: ALGEBRA | Representation and concept | translation of simple real-world situations into linear algebraic expressions
  \item Secondary KC: ALGEBRA | Simplifying | simplifying simple linear expressions excluding brackets
  \item Topic: Education
  \item Grade: Primary 6
\end{itemize}

\textbf{Question}

A teacher is preparing to give out pens as rewards. She gives $x$ pens to each of her $4$ students and keeps $3$ pens for herself. \\ 
Write an algebraic expression to represent the total number of pens the teacher needs, and simplify your expression.

\textbf{Solution}

Let $x$ represent the number of pens given to each student. \\ 
She gives out $x$ pens to each student, so she gives $4 \times x = 4x$ pens to all her students. \\ 
She also keeps $3$ pens for herself. \\ 
The total number of pens needed is $4x + 3$. \\ 
This expression is already simplified since it contains no brackets or like terms to combine.

\section*{Question 97}
\textbf{Metadata}

\begin{itemize}
  \item Question ID: P6-AgRepLrEx\_P6-AgEvlLrEx\_GPT4.1\_Farming\_05
  \item Primary KC: ALGEBRA | Representation and concept | translation of simple real-world situations into linear algebraic expressions
  \item Secondary KC: ALGEBRA | Evaluation | evaluating simple linear expressions by substitution
  \item Topic: Farming
  \item Grade: Primary 6
\end{itemize}

\textbf{Question}

A farmer plants $x$ rows of carrot plants on her farm. In each row, she plants 8 carrot plants. She also plants 15 tomato plants in total on the same farm. \\ 
(a) Write an algebraic expression to represent the total number of plants on the farm. \\ 
(b) If there are 6 rows of carrot plants, how many plants are there altogether on the farm?

\textbf{Solution}

(a) Let $x$ represent the number of rows of carrot plants. \\ 
Number of carrot plants = $8x$ \\ 
Number of tomato plants = $15$ \\ 
Total number of plants = $8x + 15$ \\ 
(b) If $x = 6$, then \\ 
Total number of plants $= 8(6) + 15 = 48 + 15 = 63$ \\ 
\\ 
Therefore, there are 63 plants altogether on the farm.

\section*{Question 98}
\textbf{Metadata}

\begin{itemize}
  \item Question ID: P6-PcFndChg\_P4-WNMul4d1d\_GPT4.1\_Digital ecconomy\_05
  \item Primary KC: PERCENTAGE | Finding change | finding percentage increase/decrease
  \item Secondary KC: WHOLE NUMBERS | Multiplication | multiplication up to 4 digits by 1 digit or up to 3 digits by 2 digits
  \item Topic: Digital ecconomy
  \item Grade: Primary 6
\end{itemize}

\textbf{Question}

A shop owner in Singapore sold 350 digital headphones online in January. In February, the number of headphones sold increased by 24\%. \\ 
(a) How many headphones were sold in February? \\ 
(b) If each headphone is sold at $89, what is the total amount earned from the headphones sold in February? \\ 
Give your answers in full sentences.

\textbf{Solution}

(a) 
Number of headphones sold in January = 350 \\ 
Increase = 24\% \\ 
Number of headphones increased = 24\% \times 350 \\ 
= \frac{24}{100} \times 350 \\ 
= 84 \\ 
Number of headphones sold in February = 350 + 84 = 434 \\ 
Answer: 434 headphones were sold in February. \\ 
 
(b) 
Price per headphone = \$89 \\ 
Total amount earned = 434 \times 89 \\ 
Calculating: \\ 
434 \times 89 = (434 \times 80) + (434 \times 9) \\ 
434 \times 80 = 34,720 \\ 
434 \times 9 = 3,906 \\ 
Total = 34,720 + 3,906 = 38,626 \\ 
Answer: The total amount earned from the headphones sold in February is \$38,626.

\section*{Question 99}
\textbf{Metadata}

\begin{itemize}
  \item Question ID: P6-FrDivPP\_P3-FrSmp\_GPT4.1\_Food\_05
  \item Primary KC: FRACTIONS | Division | dividing a whole number/proper fraction by a proper fraction
  \item Secondary KC: FRACTIONS | Simplifying | expressing a fraction in its simplest form
  \item Topic: Food
  \item Grade: Primary 6
\end{itemize}

\textbf{Question}

A baker has $3$ litres of chocolate sauce. He wants to pour the sauce equally into jars, with each jar holding $\frac{3}{4}$ litre. How many jars can he fill? Give your answer in its simplest form.

\textbf{Solution}

To find out how many jars can be filled, divide the total amount of chocolate sauce by the amount each jar can hold:

$
\text{Number of jars} = \frac{3}{\frac{3}{4}}
$

To divide by a fraction, multiply by its reciprocal:

$
\frac{3}{\frac{3}{4}} = 3 \times \frac{4}{3}
$

$
3 \times \frac{4}{3} = \frac{12}{3} = 4
$

So, the baker can fill $4$ jars.

\section*{Question 100}
\textbf{Metadata}

\begin{itemize}
  \item Question ID: P6-RoFndRoWN\_P4-WNMul4d1d\_GPT4.1\_Leisure\_05
  \item Primary KC: RATIO | Finding ratio | finding the ratio of two or three given whole numbers
  \item Secondary KC: WHOLE NUMBERS | Multiplication | multiplication up to 4 digits by 1 digit or up to 3 digits by 2 digits
  \item Topic: Leisure
  \item Grade: Primary 6
\end{itemize}

\textbf{Question}

In a cycling club, there are 36 boys, 24 girls, and 60 adults. Each member is given a badge. If each badge costs \$2, what is the ratio of the number of boys to girls to adults? What is the total cost of all the badges given out? 

\textbf{Solution}

First, to find the ratio of the number of boys to girls to adults, we write the numbers: 36 : 24 : 60. \newline \newline Find the highest common factor (HCF) of 36, 24, and 60. The HCF is 12. \newline \newline Divide each number by 12: \newline 36 \div 12 = 3 \newline 24 \div 12 = 2 \newline 60 \div 12 = 5 \newline \newline So, the ratio is 3 : 2 : 5. \newline \newline Next, find the total number of badges: \newline 36 + 24 + 60 = 120 \newline \newline The total cost is: \newline 120 \times 2 = \$240 \newline \newline \textbf{Answer:} The ratio of boys to girls to adults is 3 : 2 : 5, and the total cost of the badges is \$240.

\end{document}
