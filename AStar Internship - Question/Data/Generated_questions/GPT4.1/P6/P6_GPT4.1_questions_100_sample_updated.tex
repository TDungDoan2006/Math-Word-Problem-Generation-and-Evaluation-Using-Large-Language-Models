\documentclass{article}
\usepackage[utf8]{inputenc}
\usepackage{amsmath}
\usepackage{amsfonts}
\usepackage{amssymb}
\usepackage{graphicx}
\usepackage{hyperref}
\title{100 P6 samples generated by GPT-4.1}
\author{Tien Dung Doan}
\begin{document}
\maketitle
\section*{P6-FrDivPP\_P5-FrMulMixN\_GPT4.1\_Household finance\_02}
\textbf{Metadata}

\begin{itemize}
  \item Primary KC: FRACTIONS | Division | dividing a whole number/proper fraction by a proper fraction
  \item Secondary KC: FRACTIONS | Multiplication | multiplying a mixed number and a whole number
  \item Topic: Household finance
  \item Grade: Primary 6
\end{itemize}

\textbf{Question}

\textbf{Ali is helping his mother prepare bags of rice for a neighbourhood charity drive.} \\ 
Each bag must contain $\frac{2}{3}$ kg of rice. Ali has $6~\text{kg}$ of rice at home.} \\ 
(1) How many complete bags of rice can Ali prepare? \\ 
After preparing the bags, his mother asks him to pack the remaining rice into smaller packets. Each packet must contain $1\dfrac{1}{2}$ times as much rice as one full bag. \\ 
(2) How much rice will each packet contain? \\ 
(3) How many such packets can Ali prepare with the leftover rice?

\textbf{Solution}

\textbf{Part (1):} \\ 
Number of full bags $=\dfrac{6}{\frac{2}{3}} = 6 \times \dfrac{3}{2} = 9$ full bags. \\ 
\textbf{Part (2):} \\ 
Amount of rice in one packet $= 1\dfrac{1}{2} \times \frac{2}{3} = \frac{3}{2} \times \frac{2}{3} = 1~\text{kg}$. \\ 
\textbf{Part (3):} \\ 
Rice used for full bags $= 9 \times \frac{2}{3} = 6~\text{kg}$. So, no rice is left. \\ 
Thus, Ali cannot prepare any extra packets with the leftover rice. \\ 
\textbf{Answers:} \\ 
(1) 9 full bags \\ 
(2) 1 kg per packet \\ 
(3) 0 packets

\section*{P6-PcFndWN\_P4-WNMul4d1d\_GPT4.1\_Education\_02}
\textbf{Metadata}

\begin{itemize}
  \item Primary KC: PERCENTAGE | Finding the whole | finding the whole given a part and the percentage
  \item Secondary KC: WHOLE NUMBERS | Multiplication | multiplication up to 4 digits by 1 digit or up to 3 digits by 2 digits
  \item Topic: Education
  \item Grade: Primary 6
\end{itemize}

\textbf{Question}

A school organised a Science Fair. 35\% of the students participated in the fair, which is a total of 315 students. Each participant received a set of 7 worksheets to complete. 

(a) How many students are there in the school? 

(b) How many worksheets were distributed in total during the fair?

\textbf{Solution}

Let the total number of students in the school be $x$. 

(a) 35\% of the students participated, so $0.35x = 315$. 

To find $x$:
\[
0.35x = 315 \\
x = \frac{315}{0.35} = 900
\]
There are 900 students in the school.  
  
(b) Each participant received 7 worksheets. The number of participants is 315.
\[
315 \times 7 = 2205
\]
A total of 2,205 worksheets were distributed during the fair.

\section*{P6-FrDivPP\_P2-FrCmp\_GPT4.1\_Transporation\_02}
\textbf{Metadata}

\begin{itemize}
  \item Primary KC: FRACTIONS | Division | dividing a whole number/proper fraction by a proper fraction
  \item Secondary KC: FRACTIONS | Comparison and ordering | comparing and ordering fractions
  \item Topic: Transporation
  \item Grade: Primary 6
\end{itemize}

\textbf{Question}

\textbf{Question:}\\
Sarah has a container with $6$ litres of petrol. She wants to pour the petrol equally into bottles, where each bottle can hold $\frac{3}{4}$ litre of petrol.\\
(a) How many bottles can she fill completely?\\
(b) Sarah notices that the bottles come in three different sizes: $\frac{1}{2}$ litre, $\frac{2}{3}$ litre, and $\frac{3}{4}$ litre. Which bottle size allows her to fill the greatest number of bottles completely from $6$ litres of petrol? Order the bottle sizes from the one that fills the most bottles to the least.

\textbf{Solution}

\textbf{Solution:}\\
(a) To find out how many bottles of $\frac{3}{4}$ litre she can fill, divide $6$ by $\frac{3}{4}$:
\[ 6 \div \frac{3}{4} = 6 \times \frac{4}{3} = \frac{24}{3} = 8. \]
So, she can fill 8 bottles completely.\\
(b) To compare the number of bottles for each size: \\ 
- For $\frac{1}{2}$ litre bottles: $6 \div \frac{1}{2} = 6 \times 2 = 12$.\\
- For $\frac{2}{3}$ litre bottles: $6 \div \frac{2}{3} = 6 \times \frac{3}{2} = 9$.\\
- For $\frac{3}{4}$ litre bottles: $6 \div \frac{3}{4} = 8$ (from part a).\\
\\
Thus,
\[
\frac{1}{2} \text{ litre} : 12 ~\text{bottles} \\
\frac{2}{3} \text{ litre} : 9 ~\text{bottles} \\
\frac{3}{4} \text{ litre} : 8 ~\text{bottles} 
\]
Ordering from most to least: $\frac{1}{2}$ litre, $\frac{2}{3}$ litre, $\frac{3}{4}$ litre.\\
So, the $\frac{1}{2}$ litre bottles allow her to fill the greatest number of bottles completely.

\section*{P6-PcFndChg\_P3-WNAdd4d\_GPT4.1\_Food\_02}
\textbf{Metadata}

\begin{itemize}
  \item Primary KC: PERCENTAGE | Finding change | finding percentage increase/decrease
  \item Secondary KC: WHOLE NUMBERS | Addition | addition up to 4 digits
  \item Topic: Food
  \item Grade: Primary 6
\end{itemize}

\textbf{Question}

A bakery sold 1,250 muffins on Monday. On Tuesday, the bakery sold 1,780 muffins. 

(a) By how many muffins did the sales increase from Monday to Tuesday?

(b) What is the percentage increase in the number of muffins sold from Monday to Tuesday? (Round your answer to 1 decimal place)

\textbf{Solution}

Let the number of muffins sold on Monday be 1,250 and on Tuesday be 1,780.

(a) Number of muffins increased $= 1,780 - 1,250 = 530$ muffins.

(b) Percentage increase $= \frac{\text{Increase}}{\text{Original amount}} \times 100\%$

$= \frac{530}{1,250} \times 100\%$

$= 0.424 \times 100\% = 42.4\%$

\textbf{Answer:}
\begin{enumerate}
    \item 530 muffins
    \item 42.4\%
\end{enumerate}

\section*{P6-PcFndChg\_P4-WNDiv4d1d\_GPT4.1\_Education\_02}
\textbf{Metadata}

\begin{itemize}
  \item Primary KC: PERCENTAGE | Finding change | finding percentage increase/decrease
  \item Secondary KC: WHOLE NUMBERS | Division | division up to 4 digits by 1 digit
  \item Topic: Education
  \item Grade: Primary 6
\end{itemize}

\textbf{Question}

A group of students took a Mathematics test. Last year, 480 students passed the test. This year, the number of students who passed increased by 25%. After the test this year, the passing students were divided equally into 6 classes. 

(a) How many students passed the test this year?

(b) How many students were in each class?

\textbf{Solution}

(a) Number of students who passed last year = 480. 

Percentage increase = 25%. 

Increase in students = $\frac{25}{100} \times 480 = 0.25 \times 480 = 120$.

Number of students who passed this year = $480 + 120 = 600$.

(b) The students were divided equally into 6 classes.

Number of students in each class = $\frac{600}{6} = 100$.

**Answer:**
(a) 600 students passed the test this year.
(b) There were 100 students in each class.

\section*{P6-FrDivPN\_P2-FrCmp\_GPT4.1\_Transporation\_02}
\textbf{Metadata}

\begin{itemize}
  \item Primary KC: FRACTIONS | Division | dividing a proper fraction by a whole number
  \item Secondary KC: FRACTIONS | Comparison and ordering | comparing and ordering fractions
  \item Topic: Transporation
  \item Grade: Primary 6
\end{itemize}

\textbf{Question}

A delivery lorry is carrying $\frac{5}{6}$ of a tonne of oranges. The driver needs to unload these oranges equally at 4 different stores along his route. 

(a) How many tonnes of oranges does each store receive? 

(b) After unloading the oranges at the first 2 stores, the driver compares the amount each remaining store will receive with $\frac{1}{8}$ of a tonne of oranges. Is the amount each remaining store receives more or less than $\frac{1}{8}$ of a tonne? Explain your answer.

\textbf{Solution}

Let us solve each part step by step.

(a) The lorry carries $\frac{5}{6}$ of a tonne of oranges to be shared equally among 4 stores.

Amount each store receives:

\[
\frac{5}{6} \div 4 = \frac{5}{6} \times \frac{1}{4} = \frac{5}{24}
\]

Each store receives $\frac{5}{24}$ of a tonne of oranges.

(b) After the first 2 stores, 2 stores remain to receive their oranges. Since the unloading is equal, each still receives $\frac{5}{24}$ of a tonne.

Now, compare $\frac{5}{24}$ with $\frac{1}{8}$. Make common denominators:

\[
\frac{1}{8} = \frac{3}{24}
\]

Since $\frac{5}{24} > \frac{3}{24}$, each remaining store receives more than $\frac{1}{8}$ of a tonne of oranges.

\section*{P6-FrDivPN\_P5-FrAddMix\_GPT4.1\_Transporation\_02}
\textbf{Metadata}

\begin{itemize}
  \item Primary KC: FRACTIONS | Division | dividing a proper fraction by a whole number
  \item Secondary KC: FRACTIONS | Addition | adding mixed numbers
  \item Topic: Transporation
  \item Grade: Primary 6
\end{itemize}

\textbf{Question}

A delivery van travels $\dfrac{3}{4}$ of a kilometre to its first stop. The remaining distance to its final destination is then divided equally among 4 other stops. \\ 
(a) What is the distance, in kilometres, between each of the 4 stops? \\ 
If another delivery route covers $2\dfrac{1}{2}$ km and $1\dfrac{3}{4}$ km in two segments, \\ 
(b) What is the total distance for the other delivery route?

\textbf{Solution}

Part (a): \\ 
Since the total journey after the first stop is $\dfrac{3}{4}$ km, and this is divided equally among 4 stops: \\ 
Distance between each stop $= \dfrac{3}{4} \div 4 = \dfrac{3}{4} \times \dfrac{1}{4} = \dfrac{3}{16}$ km \\ 
\\ 
Part (b): \\ 
Finding the sum of $2\dfrac{1}{2}$ km and $1\dfrac{3}{4}$ km. \\ 
First, convert to improper fractions: \\ 
$2\dfrac{1}{2} = \dfrac{5}{2}$, $1\dfrac{3}{4} = \dfrac{7}{4}$ \\ 
Find a common denominator (4): \\ 
$\dfrac{5}{2} = \dfrac{10}{4}$ \\ 
Add: $\dfrac{10}{4} + \dfrac{7}{4} = \dfrac{17}{4} = 4\dfrac{1}{4}$ \\ 
\\ 
So, the total distance for the other route is $4\dfrac{1}{4}$ km.

\section*{P6-PcFndChg\_P3-WNAdd4d\_GPT4.1\_Sports\_02}
\textbf{Metadata}

\begin{itemize}
  \item Primary KC: PERCENTAGE | Finding change | finding percentage increase/decrease
  \item Secondary KC: WHOLE NUMBERS | Addition | addition up to 4 digits
  \item Topic: Sports
  \item Grade: Primary 6
\end{itemize}

\textbf{Question}

\textbf{At the annual school sports day, the Red House team scored 2,350 points last year. This year, the Red House team scored 1,470 more points than last year. What is the percentage increase in the number of points scored by the Red House team this year, correct to 1 decimal place?}

\textbf{Solution}

\begin{align*}
\text{Points scored last year} &= 2,350 \\
\text{Increase in points} &= 1,470 \\
\text{Points scored this year} &= 2,350 + 1,470 = 3,820 \\
\text{Percentage increase} &= \frac{\text{Increase}}{\text{Original amount}} \times 100\\
&= \frac{1,470}{2,350} \times 100\\
&= 0.6255 \times 100\\
&= 62.55\%\\
\text{Correct to 1 decimal place:} &\,\boxed{62.6\%}
\end{align*}

\section*{P6-PcFndChg\_P3-WNAdd4d\_GPT4.1\_Household finance\_02}
\textbf{Metadata}

\begin{itemize}
  \item Primary KC: PERCENTAGE | Finding change | finding percentage increase/decrease
  \item Secondary KC: WHOLE NUMBERS | Addition | addition up to 4 digits
  \item Topic: Household finance
  \item Grade: Primary 6
\end{itemize}

\textbf{Question}

\textbf{Question:} \\ Sarah's family spent \$1,200 on groceries in January. In February, they bought additional household items, so they spent \$1,500 on groceries and items altogether. What is the percentage increase in their spending from January to February? \\ \\ \text{Give your answer to the nearest whole percent.}

\textbf{Solution}

\textbf{Solution:}\\ 
\begin{align*}
\text{Amount spent in January} &= \$1,200 \\
\text{Amount spent in February} &= \$1,500 \\
\text{Increase in spending} &= 1,500 - 1,200 = 300 \\
\text{Percentage increase} &= \frac{\text{Increase}}{\text{Original amount}} \times 100\% \\
&= \frac{300}{1,200} \times 100\% \\
&= 0.25 \times 100\% \\
&= 25\% \\
\end{align*}\\ 
\text{\textbf{Answer:}} \\ 
\text{The percentage increase in spending is } \boxed{25\%}.

\section*{P6-PcFndWN\_P4-WNDiv4d1d\_GPT4.1\_Digital ecconomy\_02}
\textbf{Metadata}

\begin{itemize}
  \item Primary KC: PERCENTAGE | Finding the whole | finding the whole given a part and the percentage
  \item Secondary KC: WHOLE NUMBERS | Division | division up to 4 digits by 1 digit
  \item Topic: Digital ecconomy
  \item Grade: Primary 6
\end{itemize}

\textbf{Question}

Question: 

A mobile app developer noticed that 35\% of his app's total downloads last month came from users in Singapore, which was 5600 downloads. 

(a) How many downloads did the app have in total last month?  

(b) If the developer wants to distribute the total downloads equally among 8 servers, how many downloads will each server need to handle?


\textbf{Solution}

Solution: 

(a) Let the total number of downloads be \( x \).

Given that 35\% of the total is 5600:

\[
35\% \times x = 5600
\]
\[
\frac{35}{100} \times x = 5600
\]
\[
x = \frac{5600 \times 100}{35}
\]
\[
x = \frac{560000}{35}
\]
\[
x = 16000
\]

So, the total number of downloads last month was \(16\,000\).

(b) The total downloads are to be distributed equally among 8 servers:

\[
\text{Downloads per server} = \frac{16\,000}{8} = 2\,000
\]

Each server will need to handle \(2\,000\) downloads.


\section*{P6-AgRepLrEx\_P6-AgSmpLrEx\_GPT4.1\_Farming\_02}
\textbf{Metadata}

\begin{itemize}
  \item Primary KC: ALGEBRA | Representation and concept | translation of simple real-world situations into linear algebraic expressions
  \item Secondary KC: ALGEBRA | Simplifying | simplifying simple linear expressions excluding brackets
  \item Topic: Farming
  \item Grade: Primary 6
\end{itemize}

\textbf{Question}

\textbf{Question:}\\
Mr. Lim has a farm where he grows carrots and potatoes. He harvested $n$ bags of carrots and $k$ bags of potatoes this week. Each bag of carrots contains 6 kg and each bag of potatoes contains 4 kg.\\
(a) Translate the total weight of carrots and potatoes harvested into a linear algebraic expression in terms of $n$ and $k$.\\
(b) Simplify the expression for the total weight if Mr. Lim harvested 2 more bags of carrots than potatoes, that is, $n = k + 2$.

\textbf{Solution}

\textbf{Solution:}\\
(a) The weight of carrots is $6n$\,kg. The weight of potatoes is $4k$\,kg.\\
\\
So, the total weight = $6n + 4k$\,kg.\\
(b) Substitute $n = k + 2$ into the expression: \\
$6n + 4k = 6(k + 2) + 4k$\\
$= 6k + 12 + 4k$\\
$= (6k + 4k) + 12$\\
$= 10k + 12$\\
\\
\textbf{Final answer:} The simplified expression is $10k + 12$ kg.

\section*{P6-FrDivPN\_P5-FrSubMix\_GPT4.1\_Household finance\_02}
\textbf{Metadata}

\begin{itemize}
  \item Primary KC: FRACTIONS | Division | dividing a proper fraction by a whole number
  \item Secondary KC: FRACTIONS | Subtraction | subtracting mixed numbers
  \item Topic: Household finance
  \item Grade: Primary 6
\end{itemize}

\textbf{Question}

Auntie Tan baked \( 3 \frac{1}{2} \) trays of pineapple tarts for Chinese New Year. She wanted to share \( \frac{3}{4} \) of a tray equally among her 2 nieces. After giving the tarts to her nieces, how many trays of pineapple tarts does Auntie Tan have left?

\textbf{Solution}

First, divide \( \frac{3}{4} \) by 2 to find out how much each niece gets:

\[
\frac{3}{4} \div 2 = \frac{3}{4} \times \frac{1}{2} = \frac{3}{8}
\]

Both nieces together get \( \frac{3}{4} \) of a tray, so after giving away \( \frac{3}{4} \) tray, the number of trays Auntie Tan has left is:

\[
3 \frac{1}{2} - \frac{3}{4}
\]

First, convert \( 3 \frac{1}{2} \) to an improper fraction:
\[
3 \frac{1}{2} = \frac{7}{2}
\]

Now subtract:
\[
\frac{7}{2} - \frac{3}{4}
\]

Find a common denominator (which is 4):
\[
\frac{14}{4} - \frac{3}{4} = \frac{11}{4}
\]

Convert \( \frac{11}{4} \) back to a mixed number:
\[
\frac{11}{4} = 2 \frac{3}{4}
\]

So, Auntie Tan has \( 2 \frac{3}{4} \) trays of pineapple tarts left.

\section*{P6-FrDivPP\_P3-FrSmp\_GPT4.1\_Education\_02}
\textbf{Metadata}

\begin{itemize}
  \item Primary KC: FRACTIONS | Division | dividing a whole number/proper fraction by a proper fraction
  \item Secondary KC: FRACTIONS | Simplifying | expressing a fraction in its simplest form
  \item Topic: Education
  \item Grade: Primary 6
\end{itemize}

\textbf{Question}

\textbf{Question:} \\ 
In a school, a group of students are sharing boxes of coloured pencils equally among themselves. Each box has $\frac{3}{4}$ of the pencils needed for one student's art project. If the Art teacher has 6 boxes, how many students can receive enough pencils for their projects? Express your answer as a simplified fraction.

\textbf{Solution}

\textbf{Solution:} \\ 
Each student needs 1 whole's worth of pencils for their project. Each box provides $\frac{3}{4}$ of what one student needs. \\ 
Total amount of pencils (in terms of 1 student's project requirement):
$6 \div \frac{3}{4}$ \\ 
Recall that dividing by a fraction is the same as multiplying by its reciprocal:
$6 \div \frac{3}{4} = 6 \times \frac{4}{3}$ \\ 
$= \frac{6 \times 4}{3} = \frac{24}{3}$ \\ 
$= 8$ \\ 
\textbf{Answer:} 8 students can receive enough pencils for their projects.

\section*{P6-FrDivPP\_P2-FrCmp\_GPT4.1\_Digital ecconomy\_02}
\textbf{Metadata}

\begin{itemize}
  \item Primary KC: FRACTIONS | Division | dividing a whole number/proper fraction by a proper fraction
  \item Secondary KC: FRACTIONS | Comparison and ordering | comparing and ordering fractions
  \item Topic: Digital ecconomy
  \item Grade: Primary 6
\end{itemize}

\textbf{Question}

A website developer manages 3 different online stores. She wants to divide 12 gigabytes (GB) of storage equally among her 3 stores, giving each store a fraction of the total storage. 

(a) How much storage (in GB) does each store receive? Express your answer as a simplified fraction.

Later, she sets aside \( \frac{2}{3} \) of the storage she received for Store A to save images, and \( \frac{4}{5} \) of the storage for Store B to save videos. 

(b) Which store has more space set aside for special files, Store A or Store B? Arrange the two amounts in ascending order.

\textbf{Solution}

Let’s solve each part step by step:

(a) Each store receives: \[
\frac{12}{3} = 4 \text{ GB}
\]

(b) Storage for special files:

- Store A: \( 4 \times \frac{2}{3} = \frac{8}{3} \) GB
- Store B: \( 4 \times \frac{4}{5} = \frac{16}{5} \) GB

To compare \( \frac{8}{3} \) and \( \frac{16}{5} \), we can convert them to decimals or find a common denominator:

\[
\frac{8}{3} = 2.666\ldots \text{ GB}
\frac{16}{5} = 3.2 \text{ GB}
\]
So, Store A: 2.666... GB, Store B: 3.2 GB.

Arranging in ascending order:
\[
\frac{8}{3} < \frac{16}{5}
\]

\textbf{Store A has less space set aside for special files than Store B.}

\textbf{Answer:} \( \frac{8}{3} \) GB (Store A), \( \frac{16}{5} \) GB (Store B); so \( \frac{8}{3} < \frac{16}{5} \).

\section*{P6-FrDivPP\_P5-FrCnv2Dc\_GPT4.1\_Household finance\_02}
\textbf{Metadata}

\begin{itemize}
  \item Primary KC: FRACTIONS | Division | dividing a whole number/proper fraction by a proper fraction
  \item Secondary KC: FRACTIONS | Conversion to decimals | expressing fractions as decimals
  \item Topic: Household finance
  \item Grade: Primary 6
\end{itemize}

\textbf{Question}

Auntie Lee has $8$ litres of orange juice. She plans to pour the juice equally into bottles, each bottle holding $\frac{2}{5}$ litre of juice. 

(a) How many bottles can she fill completely?

(b) If Auntie Lee decides to sell each bottle for the amount equal to its volume in decimal litres, how much will she charge per bottle?

(give your answer for part (b) correct to $2$ decimal places)

\textbf{Solution}

Part (a):

Number of bottles $= 8 \div \frac{2}{5}$

Reciprocal of $\frac{2}{5}$ is $\frac{5}{2}$, so:

$8 \div \frac{2}{5} = 8 \times \frac{5}{2} = 4 \times 5 = 20$

Therefore, she can fill $20$ bottles.

Part (b):

The volume in each bottle is $\frac{2}{5}$ litre.

To convert $\frac{2}{5}$ to decimals: $\frac{2}{5} = 0.4$

Therefore, she will charge $0.40$ per bottle.

\section*{P6-PcFndWN\_P3-WNSub4d\_GPT4.1\_Household finance\_02}
\textbf{Metadata}

\begin{itemize}
  \item Primary KC: PERCENTAGE | Finding the whole | finding the whole given a part and the percentage
  \item Secondary KC: WHOLE NUMBERS | Subtraction | subtraction up to 4 digits
  \item Topic: Household finance
  \item Grade: Primary 6
\end{itemize}

\textbf{Question}

Anna saved $120, which was 30\% of the total amount she wanted to save to buy a new washing machine. Later, she used $45 from her savings to buy groceries for her family. 

(a) How much was the total amount Anna wanted to save?

(b) How much money did Anna have left after buying groceries?

\textbf{Solution}

Let the total amount Anna wanted to save be $x$.

(a) Anna's savings is 30\% of the total amount:

\[
120 = 0.3 \times x
\]
\[
x = \frac{120}{0.3} = 400
\]

So the total amount Anna wanted to save was $\$400$.

(b) Anna spent $\$45$ on groceries:

Money left $= 120 - 45 = 75$

So Anna had $\$75$ left after buying groceries.

\section*{P6-FrDivPN\_P6-FrDivPP\_GPT4.1\_Digital ecconomy\_02}
\textbf{Metadata}

\begin{itemize}
  \item Primary KC: FRACTIONS | Division | dividing a proper fraction by a whole number
  \item Secondary KC: FRACTIONS | Division | dividing a whole number/proper fraction by a proper fraction
  \item Topic: Digital ecconomy
  \item Grade: Primary 6
\end{itemize}

\textbf{Question}

A digital content creator has $\frac{3}{4}$ GB of data remaining in his monthly plan. He wants to upload a series of short videos, each video requiring $\frac{1}{8}$ GB of data. He plans to split the $\frac{3}{4}$ GB equally among 3 days. 

(a) How much data does he allocate for uploading videos each day? 

(b) How many videos can he upload per day if each video requires $\frac{1}{8}$ GB?

\textbf{Solution}

\textbf{(a) Amount of data allocated per day:} 

The total data remaining is $\frac{3}{4}$ GB. He splits this equally over 3 days.

Amount per day $= \frac{3}{4} \div 3 = \frac{3}{4} \times \frac{1}{3} = \frac{3}{12} = \frac{1}{4}$ GB.$\newline$

\textbf{(b) Number of videos he can upload per day:}

Amount of data per day $= \frac{1}{4}$ GB.

Each video requires $\frac{1}{8}$ GB. To find the number of videos:

Number of videos $= \frac{1}{4} \div \frac{1}{8} = \frac{1}{4} \times \frac{8}{1} = \frac{8}{4} = 2$

$\boxed{2}$ videos can be uploaded per day.

\section*{P6-RoFndRoWN\_P4-WNMul4d1d\_GPT4.1\_Digital ecconomy\_02}
\textbf{Metadata}

\begin{itemize}
  \item Primary KC: RATIO | Finding ratio | finding the ratio of two or three given whole numbers
  \item Secondary KC: WHOLE NUMBERS | Multiplication | multiplication up to 4 digits by 1 digit or up to 3 digits by 2 digits
  \item Topic: Digital ecconomy
  \item Grade: Primary 6
\end{itemize}

\textbf{Question}

A digital store in Singapore sold 480 mobile apps, 360 e-books, and 240 digital music albums last month. The price of one mobile app is $3, one e-book is $5, and one digital music album is $8. 

(a) What is the ratio of the number of mobile apps to e-books to digital music albums sold? 

(b) How much money did the store earn from selling all the mobile apps? 

(c) What is the ratio of the money earned from selling mobile apps to the money earned from selling digital music albums?

\textbf{Solution}

\textbf{(a) Finding the ratio of the numbers sold:}  \\ 
\text{Number of mobile apps sold} = 480  \\ 
\text{Number of e-books sold} = 360  \\ 
\text{Number of digital music albums sold} = 240  \\ 
\text{Greatest common divisor (GCD) of 480, 360, 240 is 120.}  \\ 
\text{So, the ratio is:}  \\ 
\frac{480}{120} : \frac{360}{120} : \frac{240}{120} = 4 : 3 : 2 \\ 
\textbf{Answer: } 4 : 3 : 2 \\ 
\\ 
\textbf{(b) Money earned from selling all the mobile apps:}  \\ 
\text{Number of mobile apps} = 480  \\ 
\text{Price per mobile app} = \$3  \\ 
\text{Total money earned from mobile apps} = 480 \times 3 = \$1440 \\ 
\\ 
\textbf{(c) Ratio of money earned from mobile apps to digital music albums:}  \\ 
\text{Money earned from digital music albums} = 240 \times 8 = \$1920 \\ 
\text{Ratio} = 1440 : 1920 \\ 
\text{Divide both numbers by 480:} \\ 
\frac{1440}{480} : \frac{1920}{480} = 3 : 4 \\ 
\textbf{Final Answers:}  \\ 
(\text{a})\ 4 : 3 : 2 \\ 
(\text{b})\ \$1440 \\ 
(\text{c})\ 3 : 4

\section*{P6-AgSlvLrN\_P6-AgRepLrEx\_GPT4.1\_Services\_02}
\textbf{Metadata}

\begin{itemize}
  \item Primary KC: ALGEBRA | Solving simple linear equations | solving simple linear equations involving whole number coefficient only
  \item Secondary KC: ALGEBRA | Representation and concept | translation of simple real-world situations into linear algebraic expressions
  \item Topic: Services
  \item Grade: Primary 6
\end{itemize}

\textbf{Question}

\textbf{Question:} \\ 

Amir wants to book a badminton court. The booking service charges a fixed fee of \$12 plus \$4 for every hour he uses the court. Amir pays a total of \$32. \\

Let $x$ be the number of hours Amir used the court.
\\
\textbf{(a)} Write an equation to represent the total amount paid by Amir. \\
\textbf{(b)} Solve the equation to find the number of hours Amir used the court.

\textbf{Solution}

\textbf{Solution:} \\ 
\textbf{(a)} The equation representing the total amount paid by Amir is:
\[
12 + 4x = 32
\]
where $x$ is the number of hours Amir used the court. \\ 
\textbf{(b)} Solving for $x$: \\ 
Subtract 12 from both sides:
\[
4x = 32 - 12
\]
\[
4x = 20
\]
Divide both sides by 4:
\[
x = \frac{20}{4}
\]
\[
x = 5
\]
Therefore, Amir used the court for \textbf{5 hours}.

\section*{P6-RoFndRoWN\_P4-WNDiv4d1d\_GPT4.1\_Education\_02}
\textbf{Metadata}

\begin{itemize}
  \item Primary KC: RATIO | Finding ratio | finding the ratio of two or three given whole numbers
  \item Secondary KC: WHOLE NUMBERS | Division | division up to 4 digits by 1 digit
  \item Topic: Education
  \item Grade: Primary 6
\end{itemize}

\textbf{Question}

A box contains a total of 1,215 pencils to be distributed among three classes: Class 6A, Class 6B, and Class 6C. Class 6A receives 315 pencils, Class 6B receives 405 pencils, and the rest of the pencils go to Class 6C. Find the ratio of pencils received by Class 6A to Class 6B to Class 6C in its simplest form.

\textbf{Solution}

First, find the number of pencils received by Class 6C:

\[
\text{Total pencils for 6A and 6B} = 315 + 405 = 720 \\
\text{Pencils for 6C} = 1215 - 720 = 495
\]

Now, write the ratio of pencils 6A : 6B : 6C:

\[315 : 405 : 495\]

To simplify, divide all terms by their highest common factor (HCF):

The HCF of 315, 405, and 495 is 45.

\[\frac{315}{45} : \frac{405}{45} : \frac{495}{45} = 7 : 9 : 11\]

\textbf{Answer:}

The ratio of pencils received by Class 6A to Class 6B to Class 6C is \(7 : 9 : 11\).

\section*{P6-FrDivPP\_P2-FrCmp\_GPT4.1\_Household finance\_02}
\textbf{Metadata}

\begin{itemize}
  \item Primary KC: FRACTIONS | Division | dividing a whole number/proper fraction by a proper fraction
  \item Secondary KC: FRACTIONS | Comparison and ordering | comparing and ordering fractions
  \item Topic: Household finance
  \item Grade: Primary 6
\end{itemize}

\textbf{Question}

Mrs Lim has $4$ litres of orange juice. She wants to pour the orange juice equally into bottles, each containing $\frac{2}{3}$ litre.

(a) How many bottles can she fill?

(b) After filling the bottles, Mrs Lim wants to arrange the bottles in increasing order of fullness by pouring $\frac{1}{4}$ litre of orange juice into some empty bottles, $\frac{1}{2}$ litre of orange juice into some other empty bottles, and leave one bottle completely full.

Arrange the following amounts of orange juice in the bottles in ascending order: $\frac{1}{4}$ litre, $\frac{1}{2}$ litre, $\frac{2}{3}$ litre, $1$ litre.

\textbf{Solution}

Solution:

(a) To find the number of bottles Mrs Lim can fill, we divide $4$ litres by $\frac{2}{3}$ litre per bottle:

\[
\text{Number of bottles} = 4 \div \frac{2}{3} = 4 \times \frac{3}{2} = \frac{12}{2} = 6
\]

She can fill $6$ bottles.

(b) To arrange the different amounts, we compare $\frac{1}{4}$ litre, $\frac{1}{2}$ litre, $\frac{2}{3}$ litre, and $1$ litre.

Let’s write all fractions with the same denominator (for comparison):

$\frac{1}{4} = \frac{3}{12}$

$\frac{1}{2} = \frac{6}{12}$

$\frac{2}{3} = \frac{8}{12}$

$1 = \frac{12}{12}$

So, in ascending order:

\[
\frac{1}{4},\ \frac{1}{2},\ \frac{2}{3},\ 1
\]

Thus, the order of the bottles by fullness is: $\frac{1}{4}$ litre, $\frac{1}{2}$ litre, $\frac{2}{3}$ litre, $1$ litre.

\section*{P6-PcFndWN\_P3-WNAdd4d\_GPT4.1\_Food\_02}
\textbf{Metadata}

\begin{itemize}
  \item Primary KC: PERCENTAGE | Finding the whole | finding the whole given a part and the percentage
  \item Secondary KC: WHOLE NUMBERS | Addition | addition up to 4 digits
  \item Topic: Food
  \item Grade: Primary 6
\end{itemize}

\textbf{Question}

\textbf{Question:}\\
Sarah bought mangoes, oranges, and apples for a fruit stall. \\She found that she used 560 mangoes, which was 35\% of the total fruits she bought.\\
She also bought 780 oranges and 1,360 apples.\\
What is the total number of fruits Sarah bought altogether?

\textbf{Solution}

\textbf{Solution:}\\
Let the total number of fruits Sarah bought be $x$.\\
35\% of $x$ = 560 (number of mangoes)\\
\begin{align*}
35\% \times x &= 560 \\
\frac{35}{100} \times x &= 560 \\
x &= \frac{560 \times 100}{35} \\
x &= \frac{56,000}{35} \\
x &= 1,600
\end{align*}\\
So, Sarah bought \textbf{1,600} fruits in total.\\
Now, to find the total number of fruits she bought altogether, including mangoes, oranges, and apples:\\
\begin{align*}
\text{Total} &= \text{mangoes} + \text{oranges} + \text{apples} \\
&= 560 + 780 + 1,360 \\
&= 1,340 + 1,360 \\
&= 2,700
\end{align*}\\
Sarah bought \textbf{2,700} fruits altogether.

\section*{P6-RoFndRoWN\_P4-WNDiv4d1d\_GPT4.1\_Leisure\_02}
\textbf{Metadata}

\begin{itemize}
  \item Primary KC: RATIO | Finding ratio | finding the ratio of two or three given whole numbers
  \item Secondary KC: WHOLE NUMBERS | Division | division up to 4 digits by 1 digit
  \item Topic: Leisure
  \item Grade: Primary 6
\end{itemize}

\textbf{Question}

Amy, Ben, and Clara went to a board game café. Together, they played a total of 144 games over the weekend. Amy played 48 games, Ben played 36 games, and Clara played the rest of the games. 

(a) What is the ratio of the number of games played by Amy to Ben to Clara? 

(b) If Clara played her games equally over 3 days, how many games did she play each day?

\textbf{Solution}

Let us solve part (a) first:

Amy played 48 games. 
Ben played 36 games. 
Clara played: $144 - 48 - 36 = 60$ games.

So, the number of games played by Amy : Ben : Clara is $48 : 36 : 60$.

To find the simplest ratio, divide each by 12:

$\frac{48}{12} : \frac{36}{12} : \frac{60}{12} = 4 : 3 : 5$

So, the ratio is $\boxed{4:3:5}$.

Now, for part (b):

Clara played 60 games in total, over 3 days.
The number of games she played each day is:

$\dfrac{60}{3} = 20$

So, Clara played $\boxed{20}$ games each day.

\section*{P6-RoFndRoWN\_P3-WNSub4d\_GPT4.1\_Transporation\_02}
\textbf{Metadata}

\begin{itemize}
  \item Primary KC: RATIO | Finding ratio | finding the ratio of two or three given whole numbers
  \item Secondary KC: WHOLE NUMBERS | Subtraction | subtraction up to 4 digits
  \item Topic: Transporation
  \item Grade: Primary 6
\end{itemize}

\textbf{Question}

A bus leaves a station with 485 passengers on board. At the next stop, 186 passengers get off the bus. There are now two groups of passengers on the bus: 130 children and the rest are adults. \newline \newline Find the ratio of children to adults to total passengers now on the bus. Express your answer in its simplest form.

\textbf{Solution}

Number of passengers remaining after some get off: \[ 485 - 186 = 299 \] Number of children: 130 \newline Number of adults: \[ 299 - 130 = 169 \] \newline \text{So the ratio of children : adults : total passengers} = 130 : 169 : 299. \newline \text{Since these numbers have no common factor, this is the simplest form.}

\section*{P6-FrDivPP\_P5-FrCnv2Dc\_GPT4.1\_Services\_02}
\textbf{Metadata}

\begin{itemize}
  \item Primary KC: FRACTIONS | Division | dividing a whole number/proper fraction by a proper fraction
  \item Secondary KC: FRACTIONS | Conversion to decimals | expressing fractions as decimals
  \item Topic: Services
  \item Grade: Primary 6
\end{itemize}

\textbf{Question}

A cleaning company offers a service where each worker can clean \( \frac{3}{4} \) of a house in an hour. If a team cleaned 6 houses in total, how many hours did it take altogether? Express your answer as a fraction and as a decimal.

\textbf{Solution}

Let the total number of houses cleaned be 6.

Each worker cleans \( \frac{3}{4} \) of a house in 1 hour. To find out how many hours were needed to clean 6 houses, we divide the total number of houses by how much one worker can clean in 1 hour:

\[
\frac{6}{\frac{3}{4}} = 6 \times \frac{4}{3} = \frac{24}{3} = 8
\]

So, it took 8 hours to clean 6 houses.

Express 8 as a decimal:

8 hours = 8.0 hours

\textbf{Final answers:} \( 8 \) hours (fraction), \( 8.0 \) hours (decimal)

\section*{P6-FrDivPP\_P5-FrMulMixN\_GPT4.1\_Manufacturing\_02}
\textbf{Metadata}

\begin{itemize}
  \item Primary KC: FRACTIONS | Division | dividing a whole number/proper fraction by a proper fraction
  \item Secondary KC: FRACTIONS | Multiplication | multiplying a mixed number and a whole number
  \item Topic: Manufacturing
  \item Grade: Primary 6
\end{itemize}

\textbf{Question}

\textbf{A worker in a factory has a wooden plank of length 4 metres. He wants to cut this plank into smaller pieces, each piece being }\frac{2}{3} \textbf{ of a metre long.}\newline

\textbf{(a) How many pieces can he get from the 4-metre plank?}\newline

\textbf{After making the pieces, he paints each piece with two coats of paint. If each coat requires }1\frac{1}{2} \textbf{ litres of paint,}\newline
\textbf{(b) How many litres of paint does he need in total for all the pieces?}


\textbf{Solution}


\textbf{(a) Number of pieces obtained:}

4 \div \frac{2}{3} = 4 \times \frac{3}{2} = \frac{12}{2} = 6

\text{So, the worker can cut out 6 pieces from the 4-metre plank.}

\textbf{(b) Total amount of paint needed:}

\text{Each piece is painted with 2 coats. One coat for one piece requires } 1\frac{1}{2} = \frac{3}{2} \text{ litres.}

\text{Total amount for one piece for 2 coats:}
2 \times \frac{3}{2} = 3 \text{ litres}

\text{He has 6 pieces, so he needs:}
6 \times 3 = 18 \text{ litres}

\boxed{\text{He will need 18 litres of paint in total.}}


\section*{P6-RoFndRoWN\_P4-WNDiv4d1d\_GPT4.1\_Digital ecconomy\_02}
\textbf{Metadata}

\begin{itemize}
  \item Primary KC: RATIO | Finding ratio | finding the ratio of two or three given whole numbers
  \item Secondary KC: WHOLE NUMBERS | Division | division up to 4 digits by 1 digit
  \item Topic: Digital ecconomy
  \item Grade: Primary 6
\end{itemize}

\textbf{Question}

\textbf{Question:} \\ 

In a digital economy class, there are 96 students. The number of students using laptops, tablets, and smartphones are in the ratio 3 : 2 : 1. \\ 
(a) How many students use each type of device? \\ 
(b) If the students using tablets are grouped equally into teams of 4, how many teams can be formed?

\textbf{Solution}

\textbf{Solution:} \\ 

\textbf{(a)} Let the number of students using laptops, tablets, and smartphones be $3x$, $2x$, and $x$ respectively. \\ 
Total students: $3x + 2x + x = 6x$ \\ 
Given $6x = 96$, so $x = \frac{96}{6} = 16$. \\ 
- Number using laptops: $3x = 3 \times 16 = 48$ \\ 
- Number using tablets: $2x = 2 \times 16 = 32$ \\ 
- Number using smartphones: $x = 16$ \\ 

\textbf{(b)} Number of students using tablets = $32$ \\ 
Grouping into teams of 4: $\frac{32}{4} = 8$ teams can be formed.

\section*{P6-FrDivPP\_P2-FrCmp\_GPT4.1\_Farming\_02}
\textbf{Metadata}

\begin{itemize}
  \item Primary KC: FRACTIONS | Division | dividing a whole number/proper fraction by a proper fraction
  \item Secondary KC: FRACTIONS | Comparison and ordering | comparing and ordering fractions
  \item Topic: Farming
  \item Grade: Primary 6
\end{itemize}

\textbf{Question}

A farmer has $6$ litres of organic fertiliser to use for his vegetable plots. He needs to use $\frac{3}{4}$ litre of fertiliser for each plot. 

(a) How many vegetable plots can he fertilise with $6$ litres of fertiliser? 

After fertilising all the plots, the farmer decided to compare the fraction of fertiliser used for $4$ plots to the fraction of fertiliser used for $5$ plots.

(b) Which fraction is larger: the fraction of fertiliser used for $4$ plots or for $5$ plots? Arrange both fractions in order from smallest to largest.

\textbf{Solution}

a) To find the number of plots he can fertilise, divide the total amount of fertiliser by the amount needed for each plot:

\[
6 \div \frac{3}{4} = 6 \times \frac{4}{3} = \frac{24}{3} = 8
\]

So, the farmer can fertilise \(8\) vegetable plots.

(b) Fraction of fertiliser used for $4$ plots: 

\[
 4 \times \frac{3}{4} = 3 \text{ litres}
\]
So, fraction of 6 litres used: \(\frac{3}{6} = \frac{1}{2}\).

Fraction of fertiliser used for $5$ plots:

\[
 5 \times \frac{3}{4} = \frac{15}{4} = 3.75 \text{ litres}
\]
Fraction of 6 litres used: \(\frac{3.75}{6} = \frac{375}{600} = \frac{5}{8}\).

Now, compare $\frac{1}{2}$ and $\frac{5}{8}$:

$\frac{1}{2} = \frac{4}{8}$, which is less than $\frac{5}{8}$.

\[
\boxed{\frac{1}{2} < \frac{5}{8}}
\]

So, the fraction of fertiliser used for $5$ plots ($\frac{5}{8}$) is larger. Arranged from smallest to largest: $\frac{1}{2},\ \frac{5}{8}$.

\section*{P6-FrDivPN\_P6-FrDivPP\_GPT4.1\_Household finance\_02}
\textbf{Metadata}

\begin{itemize}
  \item Primary KC: FRACTIONS | Division | dividing a proper fraction by a whole number
  \item Secondary KC: FRACTIONS | Division | dividing a whole number/proper fraction by a proper fraction
  \item Topic: Household finance
  \item Grade: Primary 6
\end{itemize}

\textbf{Question}

Question:

Aunt Mary has $\frac{3}{4}$ of a cake. She wants to share it equally among 3 of her neighbours. 

(a) How much of the cake will each neighbour get?

Later, she bakes 2 more identical cakes and decides to cut each of the 2 cakes into pieces that are $\frac{1}{6}$ of a cake each.

(b) How many pieces of $\frac{1}{6}$-sized cake can she cut from the 2 cakes in total?

\textbf{Solution}

Solution:

(a) Aunt Mary has $\frac{3}{4}$ of a cake and wants to divide it equally among 3 neighbours.

Each neighbour will get:

$\frac{3}{4} \div 3 = \frac{3}{4} \times \frac{1}{3} = \frac{3}{12} = \frac{1}{4}$

So, each neighbour will get $\frac{1}{4}$ of a cake.

(b) She bakes 2 whole cakes and wants to cut each cake into pieces that are $\frac{1}{6}$ of a cake each.

Number of $\frac{1}{6}$-sized pieces from 2 cakes:

$2 \div \frac{1}{6} = 2 \times 6 = 12$

So, she can cut 12 pieces of $\frac{1}{6}$-sized cake from the 2 cakes.

\section*{P6-RoFndRoWN\_P6-RoSmpWN\_GPT4.1\_Household finance\_02}
\textbf{Metadata}

\begin{itemize}
  \item Primary KC: RATIO | Finding ratio | finding the ratio of two or three given whole numbers
  \item Secondary KC: RATIO | Simplifying | expressing a ratio in its simplest form
  \item Topic: Household finance
  \item Grade: Primary 6
\end{itemize}

\textbf{Question}

In a household, the amount of money that Steven, his sister Clara, and their father saved last month are as follows: Steven saved $120, Clara saved $180, and their father saved $300. 

(a) Find the ratio of the amount of money saved by Steven to Clara to their father. 

(b) Express this ratio in its simplest form.

\textbf{Solution}

(a) The amounts saved are $120$ (Steven), $180$ (Clara), and $300$ (Father).

The ratio is $120:180:300$.

(b) To express the ratio in its simplest form, find the highest common factor (HCF) of the three numbers.

HCF of $120, 180,$ and $300$ is $60$.

So, divide each number by $60$:

$120 \div 60 = 2$

$180 \div 60 = 3$

$300 \div 60 = 5$

The simplest form of the ratio is $2:3:5$.

\section*{P6-AgRepLrEx\_P6-AgSmpLrEx\_GPT4.1\_Education\_02}
\textbf{Metadata}

\begin{itemize}
  \item Primary KC: ALGEBRA | Representation and concept | translation of simple real-world situations into linear algebraic expressions
  \item Secondary KC: ALGEBRA | Simplifying | simplifying simple linear expressions excluding brackets
  \item Topic: Education
  \item Grade: Primary 6
\end{itemize}

\textbf{Question}

A school is ordering new notebooks for its students. Each notebook costs \(x\) dollars. The school plans to buy 8 notebooks for each student. If there are 35 students in the class, write down an expression that represents the total cost of all the notebooks. Simplify the expression.

\textbf{Solution}

Let the cost of each notebook be \(x\) dollars. Each student gets 8 notebooks. So, the total number of notebooks needed is \(35 \times 8 = 280\). The total cost is \(280 \times x = 280x\). The simplified expression for the total cost is \(280x\) dollars.

\section*{P6-FrDivPP\_P2-FrCmp\_GPT4.1\_Education\_02}
\textbf{Metadata}

\begin{itemize}
  \item Primary KC: FRACTIONS | Division | dividing a whole number/proper fraction by a proper fraction
  \item Secondary KC: FRACTIONS | Comparison and ordering | comparing and ordering fractions
  \item Topic: Education
  \item Grade: Primary 6
\end{itemize}

\textbf{Question}

Sarah has a box of 12 coloured pencils. She wants to divide them equally among her friends, giving each friend $\frac{3}{4}$ of a pencil. \\ 
(a) How many friends can she give the pencils to? \\ 
(b) If each friend also receives $\frac{5}{8}$ of a pencil of another colour, compare $\frac{3}{4}$ and $\frac{5}{8}$ and state who gets a larger share of coloured pencil. \\ 
Arrange $\frac{3}{4}$ and $\frac{5}{8}$ in order from smallest to largest.

\textbf{Solution}

(a) Number of friends $= 12 \div \frac{3}{4} = 12 \times \frac{4}{3} = \frac{48}{3} = 16$. \newline Sarah can give the pencils to 16 friends. \\ 
(b) To compare $\frac{3}{4}$ and $\frac{5}{8}$, express them with the same denominator: \\ 
$\frac{3}{4} = \frac{6}{8}$ and $\frac{5}{8}$ remains as $\frac{5}{8}$. \newline $\frac{6}{8} > \frac{5}{8}$. \newline 
So, each friend receives a larger share when they get $\frac{3}{4}$ of a pencil than when they receive $\frac{5}{8}$ of a pencil. \\ 
Arranged from smallest to largest: $\frac{5}{8}$, $\frac{3}{4}$.

\section*{P6-RoFndRoWN\_P3-WNSub4d\_GPT4.1\_Food\_02}
\textbf{Metadata}

\begin{itemize}
  \item Primary KC: RATIO | Finding ratio | finding the ratio of two or three given whole numbers
  \item Secondary KC: WHOLE NUMBERS | Subtraction | subtraction up to 4 digits
  \item Topic: Food
  \item Grade: Primary 6
\end{itemize}

\textbf{Question}

A bakery sold 154 chocolate muffins, 207 vanilla muffins, and 98 blueberry muffins on Friday. On Saturday, the bakery sold 62 fewer chocolate muffins than on Friday, but the sales of vanilla and blueberry muffins remained the same. What is the ratio of chocolate muffins to vanilla muffins to blueberry muffins sold on Saturday?

\textbf{Solution}

First, find the number of chocolate muffins sold on Saturday: 

154 - 62 = 92

The number of vanilla muffins sold on Saturday = 207
The number of blueberry muffins sold on Saturday = 98

Now, find the ratio of chocolate muffins : vanilla muffins : blueberry muffins sold on Saturday:

92 : 207 : 98

Since 92, 207, and 98 have no common factors other than 1, the ratio in its simplest form is:

\boxed{92 : 207 : 98}

\section*{P6-FrDivPP\_P3-FrSmp\_GPT4.1\_Sports\_02}
\textbf{Metadata}

\begin{itemize}
  \item Primary KC: FRACTIONS | Division | dividing a whole number/proper fraction by a proper fraction
  \item Secondary KC: FRACTIONS | Simplifying | expressing a fraction in its simplest form
  \item Topic: Sports
  \item Grade: Primary 6
\end{itemize}

\textbf{Question}

Sarah has $3$ litres of sports drink. She wants to pour the drink equally into bottles, with each bottle containing $\frac{3}{4}$ litre of the drink. 

How many bottles can Sarah fill completely? Express your answer as a whole number and in simplest fractional form if necessary.

\textbf{Solution}

To find how many bottles Sarah can fill, divide the total amount of sports drink by the amount each bottle holds: 

\[
\text{Number of bottles} = \frac{3}{\frac{3}{4}}
\]

Dividing by a fraction is the same as multiplying by its reciprocal: 

\[
\frac{3}{\frac{3}{4}} = 3 \times \frac{4}{3} = \frac{12}{3} = 4
\]

So, Sarah can fill \boxed{4} bottles completely.

\section*{P6-FrDivPN\_P4-FrRepSet\_GPT4.1\_Education\_02}
\textbf{Metadata}

\begin{itemize}
  \item Primary KC: FRACTIONS | Division | dividing a proper fraction by a whole number
  \item Secondary KC: FRACTIONS | Representation and concept | fraction as part of a set 
  \item Topic: Education
  \item Grade: Primary 6
\end{itemize}

\textbf{Question}

\textbf{Question:}\\
A group of 8 students are sharing some coloured pencils equally. There are \( \frac{3}{4} \) of a box of pencils to be shared.\\

(a) How much of the box does each student receive?\\
(b) If each pencil box originally has 12 pencils, how many pencils does each student get?\\

\textbf{Solution}

\textbf{Solution:}\\
(a) Each student receives: \\ 
\[
\frac{3}{4} \div 8 = \frac{3}{4} \times \frac{1}{8} = \frac{3}{32}
\]
So, each student receives \( \frac{3}{32} \) of a box of pencils.\\

(b) Since each box has 12 pencils, \( \frac{3}{32} \) of a box is:\\
\[
\frac{3}{32} \times 12 = \frac{36}{32} = 1\frac{1}{8}
\]
So, each student gets 1 whole pencil and \( \frac{1}{8} \) of a pencil, which can be shown as part of a set of pencils.\\

\section*{P6-RoFndRoWN\_P4-WNDiv4d1d\_GPT4.1\_Farming\_02}
\textbf{Metadata}

\begin{itemize}
  \item Primary KC: RATIO | Finding ratio | finding the ratio of two or three given whole numbers
  \item Secondary KC: WHOLE NUMBERS | Division | division up to 4 digits by 1 digit
  \item Topic: Farming
  \item Grade: Primary 6
\end{itemize}

\textbf{Question}

\textbf{Question:} \\ A farmer harvested 3,672 oranges and 2,448 apples in a week. He wants to pack all the fruits into boxes so that each box contains the same number of oranges and the same number of apples, with no fruit left over. \\ (a) How many oranges and how many apples will there be in each box if he packs them into the greatest possible number of identical boxes? \\ (b) What is the ratio of oranges to apples in each box? \\

\textbf{Solution}

\textbf{Solution:} \\ \textbf{(a) Find the greatest possible number of boxes:} \\ To pack all the fruits evenly, find the greatest common divisor (GCD) of 3,672 and 2,448. \\ \begin{align*} 3,672 \div 2,448 &= 1\text{ remainder }1,224 \\ 2,448 \div 1,224 &= 2\text{ remainder }0 \\ \end{align*} So, GCD is 1,224. \\ Therefore, he can pack them into \textbf{1,224 boxes}. \\ \textbf{Number of oranges in each box:} \\ $3,672 \div 1,224 = 3$ \\ \textbf{Number of apples in each box:} \\ $2,448 \div 1,224 = 2$ \\ \textbf{(b) Ratio of oranges to apples in each box:} \\ Ratio = $3 : 2$ \\ \textbf{Final Answer:} \\ (a) There will be 3 oranges and 2 apples in each box. \\ (b) The ratio of oranges to apples in each box is $3 : 2$. \\

\section*{P6-PcFndChg\_P4-WNDiv4d1d\_GPT4.1\_Farming\_02}
\textbf{Metadata}

\begin{itemize}
  \item Primary KC: PERCENTAGE | Finding change | finding percentage increase/decrease
  \item Secondary KC: WHOLE NUMBERS | Division | division up to 4 digits by 1 digit
  \item Topic: Farming
  \item Grade: Primary 6
\end{itemize}

\textbf{Question}

A farmer harvested 3,600 kilograms of tomatoes last month. This month, the harvest decreased by 25\% compared to last month. The farmer wants to divide the tomatoes harvested this month equally among 4 markets. 

(a) How many kilograms of tomatoes did the farmer harvest this month?

(b) How many kilograms of tomatoes did each market receive?

\textbf{Solution}

Let the amount of tomatoes harvested this month be $x$ kg.

(a) The harvest decreased by 25\%:
\\
\text{Decrease} = 25\% \times 3,600 = \frac{25}{100} \times 3,600 = 900 \\
\text{Tomatoes harvested this month} = 3,600 - 900 = \boxed{2,700 ~\text{kg}}
\\
(b) The tomatoes are divided equally among 4 markets:
\\
\text{Tomatoes each market received} = \frac{2,700}{4} = 675 \\
\boxed{675 ~\text{kg}}

\textbf{Final Answers:}

(a) 2,700 kg

(b) 675 kg

\section*{P6-FrDivPP\_P5-FrMulMixN\_GPT4.1\_Leisure\_02}
\textbf{Metadata}

\begin{itemize}
  \item Primary KC: FRACTIONS | Division | dividing a whole number/proper fraction by a proper fraction
  \item Secondary KC: FRACTIONS | Multiplication | multiplying a mixed number and a whole number
  \item Topic: Leisure
  \item Grade: Primary 6
\end{itemize}

\textbf{Question}

A group of friends rented bicycles to ride around East Coast Park. If the group has 8 bicycles and each ride lasts $\dfrac{2}{3}$ of an hour, how many $\dfrac{2}{5}$-hour rides can they make in total before all their rental time is used up?

After their rides, 3 friends decide to each ride for 1$\dfrac{1}{4}$ more hours. What is the total extra time, in hours, spent by these 3 friends?

\textbf{Solution}

First, we find the total time available:
Each bicycle can be used for $\dfrac{2}{3}$ hour. There are 8 bicycles, so total riding time:
$8 \times \dfrac{2}{3} = \dfrac{16}{3}$ hours.

Now, to find out how many $\dfrac{2}{5}$-hour rides can be made in total:
Total number of $\dfrac{2}{5}$-hour rides = $\dfrac{16}{3} \div \dfrac{2}{5}$
$= \dfrac{16}{3} \times \dfrac{5}{2}$
$= \dfrac{16 \times 5}{3 \times 2}$
$= \dfrac{80}{6} = \dfrac{40}{3} = 13\dfrac{1}{3}$ rides

So, they can make 13 full $\dfrac{2}{5}$-hour rides, with some time left over.

For the second part:
Each of 3 friends rides for $1\dfrac{1}{4}$ more hours.
First, $1\dfrac{1}{4} = \dfrac{5}{4}$ hours.
Total extra time = $3 \times \dfrac{5}{4} = \dfrac{15}{4} = 3\dfrac{3}{4}$ hours.

Summary:

Number of $\dfrac{2}{5}$-hour rides: $13\dfrac{1}{3}$
Total extra time spent by 3 friends: $3\dfrac{3}{4}$ hours.

\section*{P6-PcFndWN\_P4-WNMul4d1d\_GPT4.1\_Sports\_02}
\textbf{Metadata}

\begin{itemize}
  \item Primary KC: PERCENTAGE | Finding the whole | finding the whole given a part and the percentage
  \item Secondary KC: WHOLE NUMBERS | Multiplication | multiplication up to 4 digits by 1 digit or up to 3 digits by 2 digits
  \item Topic: Sports
  \item Grade: Primary 6
\end{itemize}

\textbf{Question}

\textbf{Question:}\\
During Sports Day, 35\% of the students in Primary 6 took part in the relay race. If 84 students took part in the relay race, how many Primary 6 students are there altogether? After the race, each participant received 4 medals. How many medals were given out in total?

\textbf{Solution}

\textbf{Solution:}\\
Let the total number of Primary 6 students be $x$.\\
35\% of $x$ took part in the relay race, and this equals 84 students.\\
So, $35\% \times x = 84$\\
$0.35x = 84$\\
$x = \dfrac{84}{0.35}$\\
$x = 240$\\
There are 240 Primary 6 students altogether.\\
\\
Each participant received 4 medals. Total medals given out $= 84 \times 4 = 336$ medals.\\
\\
\textbf{Answer:} There are 240 Primary 6 students altogether, and 336 medals were given out.

\section*{P6-FrDivPN\_P3-FrSmp\_GPT4.1\_Digital ecconomy\_02}
\textbf{Metadata}

\begin{itemize}
  \item Primary KC: FRACTIONS | Division | dividing a proper fraction by a whole number
  \item Secondary KC: FRACTIONS | Simplifying | expressing a fraction in its simplest form
  \item Topic: Digital ecconomy
  \item Grade: Primary 6
\end{itemize}

\textbf{Question}

A digital artist earns \(\frac{3}{4}\) of a cryptocurrency for completing a project. She decides to equally share this amount among 5 friends who contributed to the project. How much cryptocurrency does each friend receive? Give your answer in its simplest form.

\textbf{Solution}

Each friend receives \(\frac{3}{4} \div 5\) of a cryptocurrency.

\[
\frac{3}{4} \div 5 = \frac{3}{4} \times \frac{1}{5} = \frac{3}{20}
\]

So, each friend receives \(\frac{3}{20}\) of a cryptocurrency.

\section*{P6-RoFndRoWN\_P3-WNAdd4d\_GPT4.1\_Services\_02}
\textbf{Metadata}

\begin{itemize}
  \item Primary KC: RATIO | Finding ratio | finding the ratio of two or three given whole numbers
  \item Secondary KC: WHOLE NUMBERS | Addition | addition up to 4 digits
  \item Topic: Services
  \item Grade: Primary 6
\end{itemize}

\textbf{Question}

\textbf{Question:} \\ 
A cleaning company received 1,250 service requests for cleaning offices, 1,800 service requests for cleaning schools, and 950 service requests for cleaning hospitals in the month of June. \\ 
(a) What is the total number of service requests the company received?  \\ 
(b) Find the ratio of the number of office cleaning requests to school cleaning requests to hospital cleaning requests, expressing your answer in its simplest form.

\textbf{Solution}

\textbf{Solution:} \\ 
(a) Total number of service requests $= 1250 + 1800 + 950$ \\ 
$= 3050 + 950$ \\ 
$= 4000$ \\ 
\textit{Total number of service requests is 4,000.} \\ 
(b) Ratio of office : school : hospital requests $= 1250 : 1800 : 950$ \\ 
Let's find the greatest common divisor (GCD) for all three numbers. \\ 
- For $1250$, $1800$, and $950$, let's try dividing by 50 first: \\ 
$1250 \div 50 = 25$ \\ 
$1800 \div 50 = 36$ \\ 
$950 \div 50 = 19$ \\ 
So, $1250 : 1800 : 950 = 25 : 36 : 19$ \\ 
\textit{The simplest ratio of office : school : hospital service requests is $25 : 36 : 19$.}

\section*{P6-FrDivPN\_P5-FrSubMix\_GPT4.1\_Manufacturing\_02}
\textbf{Metadata}

\begin{itemize}
  \item Primary KC: FRACTIONS | Division | dividing a proper fraction by a whole number
  \item Secondary KC: FRACTIONS | Subtraction | subtracting mixed numbers
  \item Topic: Manufacturing
  \item Grade: Primary 6
\end{itemize}

\textbf{Question}

\textbf{Question:} \\ 
A factory produces \( \frac{7}{8} \) of a metre of cloth each day. The manager wants to distribute this total amount equally among 4 workers. Each worker then uses their share of the cloth to make dresses. If each worker uses \( 1 \frac{1}{8} \) metres less cloth than what they receive, how much cloth does each worker have left after making the dresses? \\ 
\textit{Give your answer in metres.}

\textbf{Solution}

\textbf{Solution:} \\ 
Total length of cloth produced by the factory = \( \frac{7}{8} \) m.\\ 
If this is divided equally among 4 workers: \\ 
\[ \text{Cloth per worker} = \frac{7}{8} \div 4 = \frac{7}{8} \times \frac{1}{4} = \frac{7}{32} \text{ metres} \] \\ 
Each worker uses \( 1 \frac{1}{8} = \frac{9}{8} \) metres less than what they receive.\\ 
Cloth left for each worker: \\ 
\[
\frac{7}{32} - \frac{9}{8} = \frac{7}{32} - \frac{36}{32} = -\frac{29}{32}
\] 
Since the answer is negative, this means each worker does not have enough cloth to use as planned; in fact, they are short of \(\frac{29}{32}\) metres.\\
So, each worker needs \(\frac{29}{32}\) metres more to reach their planned use.

\section*{P6-PcFndChg\_P3-WNAdd4d\_GPT4.1\_Education\_02}
\textbf{Metadata}

\begin{itemize}
  \item Primary KC: PERCENTAGE | Finding change | finding percentage increase/decrease
  \item Secondary KC: WHOLE NUMBERS | Addition | addition up to 4 digits
  \item Topic: Education
  \item Grade: Primary 6
\end{itemize}

\textbf{Question}

\textbf{Question:} \\ 
At Raffles Primary School, there were 1,250 students last year. This year, the school enrolled 375 more students. \\ 
(a) How many students are there in the school this year? \\ 
(b) What is the percentage increase in the number of students from last year to this year? \\

\textbf{Solution}

\textbf{Solution:} \\ 
(a) \text{Number of students this year} = 1,250 + 375 = 1,625 \\ 
\text{There are 1,625 students in the school this year.} \\ 
(b) \text{Increase in number of students} = 1,625 - 1,250 = 375 \\ 
\text{Percentage increase} = \dfrac{375}{1,250} \times 100\% = 30\% \\ 
\text{The percentage increase in the number of students is } 30\%. \\

\section*{P6-FrDivPN\_P6-FrDivPP\_GPT4.1\_Services\_02}
\textbf{Metadata}

\begin{itemize}
  \item Primary KC: FRACTIONS | Division | dividing a proper fraction by a whole number
  \item Secondary KC: FRACTIONS | Division | dividing a whole number/proper fraction by a proper fraction
  \item Topic: Services
  \item Grade: Primary 6
\end{itemize}

\textbf{Question}

A group of friends decided to share a pizza equally. \( \frac{3}{4} \) of a pizza is left, and they want to divide it equally among 3 people. Later, a new friend joins, and now they have to share another \( 2 \) full pizzas equally among all 4 friends.\
\
(a) How much pizza does each of the first 3 friends get from the \( \frac{3}{4} \) pizza?\
\
(b) How much pizza does each friend get from the 2 pizzas, if the 2 pizzas are shared equally among 4 friends?\
\
(c) If each friend ends up eating only half as much as their final share from (b), how many times more pizza did they get in part (a) compared to this new amount from (b)?

\textbf{Solution}

\textbf{(a)} \quad Each friend gets \( \frac{3}{4} \div 3 = \frac{3}{4} \times \frac{1}{3} = \frac{1}{4} \) of a pizza. \\ \\ 
\textbf{(b)} \quad Each friend gets \( 2 \div 4 = \frac{2}{4} = \frac{1}{2} \) of a pizza.
\\
\textbf{(c)} \quad If each friend eats half of what they get in (b), they eat \( \frac{1}{2} \div 2 = \frac{1}{4} \) of a pizza.\
Compare this to (a): They got \( \frac{1}{4} \) of a pizza in both cases.\
Thus, the number of times more pizza from (a) compared to the new (b) amount is \( \frac{\frac{1}{4}}{\frac{1}{4}} = 1 \) time. So, they got the same amount in both cases.

\section*{P6-RoFndRoWN\_P4-WNMul4d1d\_GPT4.1\_Farming\_02}
\textbf{Metadata}

\begin{itemize}
  \item Primary KC: RATIO | Finding ratio | finding the ratio of two or three given whole numbers
  \item Secondary KC: WHOLE NUMBERS | Multiplication | multiplication up to 4 digits by 1 digit or up to 3 digits by 2 digits
  \item Topic: Farming
  \item Grade: Primary 6
\end{itemize}

\textbf{Question}

In a farm, there are 256 chickens, 384 ducks, and 160 goats. 

(a) Find the ratio of the number of chickens to ducks to goats in its simplest form.

(b) Each duck lays 12 eggs in a week, while each goat produces 9 litres of milk in a week. Find the total number of eggs laid by all the ducks in a week, and the total amount of milk produced by all the goats in a week. What is the ratio of the total number of eggs to the total amount of milk produced?

\textbf{Solution}

Let us solve each part.

(a) To find the ratio of chickens : ducks : goats, we write the numbers:

Chickens : Ducks : Goats = 256 : 384 : 160

First, find the highest common factor (HCF) of 256, 384, and 160.
The HCF of 256, 384, and 160 is 32.

Divide each number by 32:
256 ÷ 32 = 8
384 ÷ 32 = 12
160 ÷ 32 = 5

So, the simplest ratio is \( 8 : 12 : 5 \).

(b) Number of ducks = 384
Each duck lays 12 eggs in a week.
Total eggs in a week = 384 × 12 = 4608

Number of goats = 160
Each goat produces 9 litres of milk in a week.
Total milk in a week = 160 × 9 = 1440 litres

The ratio of eggs to milk = 4608 : 1440
Find HCF of 4608 and 1440.
The HCF is  288.
4608 ÷ 288 = 16
1440 ÷ 288 = 5

Final ratio = \( 16 : 5 \)

Answer:
(a) The ratio of chickens to ducks to goats is \( 8:12:5 \).
(b) The total number of eggs laid is 4608, the total amount of milk is 1440 litres, and the ratio of eggs to milk is \( 16:5 \).

\section*{P6-RoFndRoWN\_P4-WNMul4d1d\_GPT4.1\_Education\_02}
\textbf{Metadata}

\begin{itemize}
  \item Primary KC: RATIO | Finding ratio | finding the ratio of two or three given whole numbers
  \item Secondary KC: WHOLE NUMBERS | Multiplication | multiplication up to 4 digits by 1 digit or up to 3 digits by 2 digits
  \item Topic: Education
  \item Grade: Primary 6
\end{itemize}

\textbf{Question}

In a school, there are three classes: Primary 6A, Primary 6B, and Primary 6C. 

Primary 6A has 28 students. The number of students in Primary 6B is 3 times the number of students in Primary 6A. Primary 6C has 54 students. 

Find the ratio of the number of students in Primary 6A to Primary 6B to Primary 6C. Express your answer in the simplest form.

\textbf{Solution}

Number of students in Primary 6A = 28.

Number of students in Primary 6B = $3 \times 28 = 84$.

Number of students in Primary 6C = 54.

So, the ratio of students in Primary 6A : Primary 6B : Primary 6C is $28 : 84 : 54$.

To simplify, divide all numbers by 2: $14 : 42 : 27$.

Since 14, 42, and 27 have no common factors, $14 : 42 : 27$ is the simplest form.

\textbf{Final Answer:} The ratio is $14 : 42 : 27$.

\section*{P6-AgRepLrEx\_P6-AgEvlLrEx\_GPT4.1\_Leisure\_02}
\textbf{Metadata}

\begin{itemize}
  \item Primary KC: ALGEBRA | Representation and concept | translation of simple real-world situations into linear algebraic expressions
  \item Secondary KC: ALGEBRA | Evaluation | evaluating simple linear expressions by substitution
  \item Topic: Leisure
  \item Grade: Primary 6
\end{itemize}

\textbf{Question}

Question: 

A dance class charges $x$ dollars for the registration fee and $5$ dollars for each session attended. Jenny attends $n$ sessions. 

(a) Write down an algebraic expression to represent the total amount Jenny has to pay for $n$ sessions.

(b) If the registration fee is $20$ and Jenny attends $8$ sessions, how much does she have to pay in total?

\textbf{Solution}

Solution: 

(a) The total amount Jenny has to pay can be represented by:

\[ \text{Total amount} = x + 5n \]

(b) Substituting $x = 20$ and $n = 8$ into the expression:

\[
\text{Total amount} = 20 + 5 \times 8 \\
= 20 + 40 \\
= 60
\]

Therefore, Jenny has to pay $60 in total.

\section*{P6-PcFndWN\_P4-WNDiv4d1d\_GPT4.1\_Education\_02}
\textbf{Metadata}

\begin{itemize}
  \item Primary KC: PERCENTAGE | Finding the whole | finding the whole given a part and the percentage
  \item Secondary KC: WHOLE NUMBERS | Division | division up to 4 digits by 1 digit
  \item Topic: Education
  \item Grade: Primary 6
\end{itemize}

\textbf{Question}

A school organised a Mathematics contest. \( 24\% \) of the students in Primary 6 participated in the contest. If \( 84 \) students took part in the contest, how many students are there in Primary 6 altogether? After the contest, the school decided to divide all the Primary 6 students equally into 7 groups for a class activity. How many students will there be in each group?

\textbf{Solution}

Let the total number of Primary 6 students be \( x \).

Given that \( 24\% \) of the students is \( 84 \):

\[
24\% \times x = 84
\]
\[
\frac{24}{100} \times x = 84
\]
\[
x = \frac{84 \times 100}{24}
\]
\[
x = \frac{8400}{24} = 350
\]

So, there are \( 350 \) Primary 6 students.

Next, to find the number of students in each group when they are divided into 7 groups:

\[
\frac{350}{7} = 50
\]

**Final Answer:**
There are \( 350 \) Primary 6 students. Each group will have \( 50 \) students.

\section*{P6-PcFndWN\_P3-WNSub4d\_GPT4.1\_Digital ecconomy\_02}
\textbf{Metadata}

\begin{itemize}
  \item Primary KC: PERCENTAGE | Finding the whole | finding the whole given a part and the percentage
  \item Secondary KC: WHOLE NUMBERS | Subtraction | subtraction up to 4 digits
  \item Topic: Digital ecconomy
  \item Grade: Primary 6
\end{itemize}

\textbf{Question}

Alya sold some digital art pieces online. She received 15\% of the earnings as her commission, which amounted to \$270. After paying for advertising expenses of \$98, she calculated her profit from the commission. 

(a) What was the total amount of earnings from the digital art sales?

(b) After subtracting the advertising expenses, how much profit did Alya make from her commission?

\textbf{Solution}

Let's solve each part step by step.

(a) We know that 15\% of the total earnings is \$270.

Let the total earnings be \( x \).

\[
15\% \times x = 270 \\
\frac{15}{100} \times x = 270 \\
x = \frac{270 \times 100}{15} \\
x = \frac{27000}{15} \\
x = 1800
\]
So, the total earnings from digital art sales is \$1,800.

(b) Alya's profit from her commission, after advertising expenses:

\[
\text{Profit} = \text{Commission} - \text{Advertising Expenses} \\
\text{Profit} = 270 - 98 = 172
\]

So, after subtracting the advertising expenses, Alya made a profit of \$172 from her commission.

\section*{P6-PcFndWN\_P3-WNSub4d\_GPT4.1\_Farming\_02}
\textbf{Metadata}

\begin{itemize}
  \item Primary KC: PERCENTAGE | Finding the whole | finding the whole given a part and the percentage
  \item Secondary KC: WHOLE NUMBERS | Subtraction | subtraction up to 4 digits
  \item Topic: Farming
  \item Grade: Primary 6
\end{itemize}

\textbf{Question}

\textbf{Amy has a farm with a number of chickens. She sold 84 chickens, which was 30\% of all the chickens she had at first. After selling the 84 chickens, she found that she had 196 chickens left. How many chickens did Amy have at first?}

\textbf{Solution}

\textbf{Let the total number of chickens Amy had at first be } x. \\ 
After selling 84 chickens, she had 196 left. \\ 
So, the number of chickens at first: \\ 
 x = 196 + 84 = 280 \\ 
But 84 chickens is 30\% of the total. \\ 
\Rightarrow 30\% \times x = 84 \\ 
\Rightarrow 0.3x = 84 \\ 
\Rightarrow x = \frac{84}{0.3} = 280 \\ 
\textbf{Amy had 280 chickens at first.}

\section*{P6-PcFndChg\_P3-WNSub4d\_GPT4.1\_Sports\_02}
\textbf{Metadata}

\begin{itemize}
  \item Primary KC: PERCENTAGE | Finding change | finding percentage increase/decrease
  \item Secondary KC: WHOLE NUMBERS | Subtraction | subtraction up to 4 digits
  \item Topic: Sports
  \item Grade: Primary 6
\end{itemize}

\textbf{Question}

A sports equipment shop sold 1,250 badminton rackets in January. In February, the shop sold only 950 rackets. 

(a) How many fewer rackets were sold in February compared to January? 

(b) What was the percentage decrease in the number of badminton rackets sold from January to February?


\textbf{Solution}

(a) Number of rackets sold in January = 1,250 

Number of rackets sold in February = 950 

Number of rackets less in February = 1,250 - 950 = 300

Answer (a): 300 fewer rackets were sold in February.

(b) Percentage decrease = \( \frac{\text{Decrease}}{\text{Original amount}} \times 100\% \)

Decrease = 300

Original amount = 1,250

\[
\text{Percentage decrease} = \frac{300}{1250} \times 100\% = 24\%
\]

Answer (b): The percentage decrease in the number of rackets sold from January to February is 24\%.


\section*{P6-FrDivPP\_P3-FrSmp\_GPT4.1\_Food\_02}
\textbf{Metadata}

\begin{itemize}
  \item Primary KC: FRACTIONS | Division | dividing a whole number/proper fraction by a proper fraction
  \item Secondary KC: FRACTIONS | Simplifying | expressing a fraction in its simplest form
  \item Topic: Food
  \item Grade: Primary 6
\end{itemize}

\textbf{Question}

\textbf{Anna baked a large chocolate cake and decided to share it equally among her friends. She has } 3 \text{ cakes, and she wants to give } \frac{2}{5} \text{ of a cake to each friend.}

\text{(a) How many friends can Anna share the cakes with?}

\text{(b) If there is any cake left after sharing, express the remaining amount of cake in its simplest form.}

\textbf{Solution}

\textbf{(a) To find out how many friends Anna can share the cakes with, we divide the total amount of cakes she has by the amount each friend will receive:}

\[ 3 \div \frac{2}{5} \]

\text{Dividing by a fraction is equivalent to multiplying by its reciprocal:}

\[ 3 \div \frac{2}{5} = 3 \times \frac{5}{2} = \frac{3 \times 5}{2} = \frac{15}{2} \]

\text{This means Anna can give } \frac{2}{5} \text{ of a cake to } \frac{15}{2} \text{ friends. Since she cannot have half a friend, she can serve 7 friends completely.}

\textbf{(b) To find out if any cake is left:}

\text{Amount given to 7 friends: } 7 \times \frac{2}{5} = \frac{14}{5}

\text{Cakes leftover: } 3 - \frac{14}{5} = \frac{15}{5} - \frac{14}{5} = \frac{1}{5}

\text{Anna will have } \frac{1}{5} \text{ of a cake left. This is already in its simplest form.}

\section*{P6-PcFndWN\_P4-WNDiv4d1d\_GPT4.1\_Sports\_02}
\textbf{Metadata}

\begin{itemize}
  \item Primary KC: PERCENTAGE | Finding the whole | finding the whole given a part and the percentage
  \item Secondary KC: WHOLE NUMBERS | Division | division up to 4 digits by 1 digit
  \item Topic: Sports
  \item Grade: Primary 6
\end{itemize}

\textbf{Question}

A sports club had a number of badminton rackets. During a yearly check, the club found that 25\% of the rackets, which is 48 rackets, needed to be re-strung. The club decided to divide all the rackets equally among 6 teams. 

(a) How many badminton rackets did the sports club have in total?

(b) How many rackets did each team receive?

\textbf{Solution}

Let the total number of badminton rackets be $x$.

(a) 25\% of the rackets is 48:

\[
25\% \times x = 48 \\
0.25x = 48 \\
x = \frac{48}{0.25} \\
x = 48 \times 4 \\
x = 192
\]

So, the sports club had 192 badminton rackets in total.

(b) Number of rackets each team receives:

\[
\text{Number of teams} = 6 \\
\text{Rackets per team} = \frac{192}{6} = 32
\]

Each team received 32 badminton rackets.

\section*{P6-RoFndRoWN\_P6-RoSmpWN\_GPT4.1\_Sports\_02}
\textbf{Metadata}

\begin{itemize}
  \item Primary KC: RATIO | Finding ratio | finding the ratio of two or three given whole numbers
  \item Secondary KC: RATIO | Simplifying | expressing a ratio in its simplest form
  \item Topic: Sports
  \item Grade: Primary 6
\end{itemize}

\textbf{Question}

In a sports competition, the number of boys, girls, and teachers participating are 24, 36, and 12 respectively. 

(a) Find the ratio of boys to girls to teachers.

(b) Express this ratio in its simplest form.

\textbf{Solution}

Let's consider the numbers:
- Number of boys = 24
- Number of girls = 36
- Number of teachers = 12

(a) The ratio of boys to girls to teachers is:
\[
24:36:12
\]

(b) To express this ratio in its simplest form, find the highest common factor (HCF) of the three numbers. 

The HCF of 24, 36, and 12 is 12.

Now divide each number by 12:
- Boys: $24 \div 12 = 2$
- Girls: $36 \div 12 = 3$
- Teachers: $12 \div 12 = 1$

So the simplest form of the ratio is:
\[
2:3:1
\]

\section*{P6-FrDivPP\_P5-FrAddMix\_GPT4.1\_Household finance\_02}
\textbf{Metadata}

\begin{itemize}
  \item Primary KC: FRACTIONS | Division | dividing a whole number/proper fraction by a proper fraction
  \item Secondary KC: FRACTIONS | Addition | adding mixed numbers
  \item Topic: Household finance
  \item Grade: Primary 6
\end{itemize}

\textbf{Question}

Question: 

In a household, Mrs Lim is preparing orange juice for a family gathering. She has $3\dfrac{1}{2}$ litres of orange juice and wants to pour it equally into bottles, each holding $\dfrac{3}{4}$ litre. After filling all the bottles, she realises that she had already prepared $1\dfrac{1}{4}$ litres earlier and adds that to her total supply before pouring. How many bottles can she completely fill?


\textbf{Solution}

Solution: 

Step 1: Find the total amount of orange juice Mrs Lim has after adding the extra juice.

$3\dfrac{1}{2} + 1\dfrac{1}{4} = \dfrac{7}{2} + \dfrac{5}{4}$

Convert $\dfrac{7}{2}$ to quarters: $\dfrac{7}{2} = \dfrac{14}{4}$

Total = $\dfrac{14}{4} + \dfrac{5}{4} = \dfrac{19}{4}$ litres

Step 2: Find the number of bottles she can fill by dividing the total by the capacity of each bottle.

Number of bottles $= \dfrac{19}{4} \div \dfrac{3}{4}$

Recall $a \div b = a \times \dfrac{1}{b}$:

$= \dfrac{19}{4} \times \dfrac{4}{3}$

$= \dfrac{19 \times 4}{4 \times 3}$

$= \dfrac{76}{12}$

$= \dfrac{19}{3}$

$= 6\dfrac{1}{3}$

So, Mrs Lim can completely fill \(6\) bottles with orange juice, and there will be some juice left over.


\section*{P6-RoFndRoWN\_P6-RoSmpWN\_GPT4.1\_Manufacturing\_02}
\textbf{Metadata}

\begin{itemize}
  \item Primary KC: RATIO | Finding ratio | finding the ratio of two or three given whole numbers
  \item Secondary KC: RATIO | Simplifying | expressing a ratio in its simplest form
  \item Topic: Manufacturing
  \item Grade: Primary 6
\end{itemize}

\textbf{Question}

In a factory, there are three types of machines: Type A, Type B, and Type C. One section of the factory has 36 Type A machines, 48 Type B machines, and 60 Type C machines.  

(a) Find the ratio of Type A machines to Type B machines to Type C machines in that section.

(b) Express this ratio in its simplest form.

\textbf{Solution}

Let us look at the numbers: 36 Type A, 48 Type B, 60 Type C.

(a) The ratio of Type A : Type B : Type C is 36 : 48 : 60.

(b) To simplify the ratio, find the highest common factor (HCF) of 36, 48, and 60.

- The factors of 36 are: 1, 2, 3, 4, 6, 9, 12, 18, 36 
- The factors of 48 are: 1, 2, 3, 4, 6, 8, 12, 16, 24, 48 
- The factors of 60 are: 1, 2, 3, 4, 5, 6, 10, 12, 15, 20, 30, 60

The highest common factor is 12.

Divide each number by 12:

$$\frac{36}{12} : \frac{48}{12} : \frac{60}{12} = 3:4:5$$

So the simplest form of the ratio is 3 : 4 : 5.

\section*{P6-RoFndRoWN\_P6-RoSmpWN\_GPT4.1\_Digital ecconomy\_02}
\textbf{Metadata}

\begin{itemize}
  \item Primary KC: RATIO | Finding ratio | finding the ratio of two or three given whole numbers
  \item Secondary KC: RATIO | Simplifying | expressing a ratio in its simplest form
  \item Topic: Digital ecconomy
  \item Grade: Primary 6
\end{itemize}

\textbf{Question}

A group of friends took part in a coding competition. 12 of them used laptops, 8 used tablets, and 16 used smartphones to participate. 

(a) Find the ratio of the number of participants who used laptops to those who used tablets to those who used smartphones. 

(b) Express this ratio in its simplest form.

\textbf{Solution}

Let us write the numbers as a ratio: 
Laptops : Tablets : Smartphones = 12 : 8 : 16

To express this ratio in its simplest form, we find the highest common factor (HCF) of 12, 8, and 16.

Prime factors:
12 = 2 × 2 × 3
8 = 2 × 2 × 2
16 = 2 × 2 × 2 × 2
The HCF is 4.

Divide all numbers by 4:
12 ÷ 4 = 3
8 ÷ 4 = 2
16 ÷ 4 = 4

So, the simplest form is 3 : 2 : 4.

Final answers: 
(a) 12 : 8 : 16
(b) 3 : 2 : 4

\section*{P6-FrDivPP\_P4-FrRepSet\_GPT4.1\_Education\_02}
\textbf{Metadata}

\begin{itemize}
  \item Primary KC: FRACTIONS | Division | dividing a whole number/proper fraction by a proper fraction
  \item Secondary KC: FRACTIONS | Representation and concept | fraction as part of a set 
  \item Topic: Education
  \item Grade: Primary 6
\end{itemize}

\textbf{Question}

\textbf{In Mrs Tan's class, there are 24 pupils. She wants to divide the class into groups such that each group has $\frac{3}{4}$ of the number of pupils that are usually placed in one group for a class activity. If the usual number of pupils in one group is 6, how many groups will Mrs Tan form?}

\textit{Express your answer as a fraction if necessary.}

\textbf{Solution}

\textbf{Solution:}

The usual number of pupils in one group $= 6$.

$\frac{3}{4}$ of the usual number of pupils in one group $= 6 \times \frac{3}{4} = \frac{18}{4} = 4\frac{1}{2}$ pupils in each group.

Total number of pupils $= 24$

Number of groups Mrs Tan will form:

$= \dfrac{24}{4\frac{1}{2}}$

First, convert $4\frac{1}{2}$ to an improper fraction:

$4\frac{1}{2} = \dfrac{9}{2}$

So,

$\dfrac{24}{4\frac{1}{2}} = \dfrac{24}{\frac{9}{2}}$

$= 24 \times \dfrac{2}{9}$

$= \dfrac{48}{9}$

$= 5\dfrac{1}{3}$

\textbf{Mrs Tan will be able to form $5\dfrac{1}{3}$ groups if she divides the class this way.}

\section*{P6-RoFndRoWN\_P3-WNSub4d\_GPT4.1\_Education\_02}
\textbf{Metadata}

\begin{itemize}
  \item Primary KC: RATIO | Finding ratio | finding the ratio of two or three given whole numbers
  \item Secondary KC: WHOLE NUMBERS | Subtraction | subtraction up to 4 digits
  \item Topic: Education
  \item Grade: Primary 6
\end{itemize}

\textbf{Question}

In a school, the Science Club has 536 members and the Art Club has 478 members. After 186 students left the Science Club and 124 students left the Art Club at the end of the year, what is the ratio of the number of Science Club members to the number of Art Club members, in its simplest form?

\textbf{Solution}

Number of Science Club members left: $536 - 186 = 350$.

Number of Art Club members left: $478 - 124 = 354$.

The required ratio is $350:354$.

To simplify, divide both numbers by 2: $350 \div 2 = 175$, $354 \div 2 = 177$.

Thus, the ratio in simplest form is $175:177$.

\section*{P6-FrDivPN\_P5-FrCnv2Dc\_GPT4.1\_Farming\_02}
\textbf{Metadata}

\begin{itemize}
  \item Primary KC: FRACTIONS | Division | dividing a proper fraction by a whole number
  \item Secondary KC: FRACTIONS | Conversion to decimals | expressing fractions as decimals
  \item Topic: Farming
  \item Grade: Primary 6
\end{itemize}

\textbf{Question}

A farmer has $\dfrac{3}{4}$ kilogram of fertilizer. She wants to divide this amount equally among 5 garden plots. 

(a) How many kilograms of fertilizer does each garden plot get? Express your answer as a fraction.

(b) Express your answer in part (a) as a decimal.

\textbf{Solution}

Let the total amount of fertilizer be $\dfrac{3}{4}$ kg, to be divided by 5.

(a) 
\[
\text{Amount per garden plot} = \dfrac{3}{4} \div 5 = \dfrac{3}{4} \times \dfrac{1}{5} = \dfrac{3}{20}
\]

Each garden plot gets $\dfrac{3}{20}$ kg of fertilizer.

(b) To express $\dfrac{3}{20}$ as a decimal:
\[
\dfrac{3}{20} = \dfrac{3 \times 5}{20 \times 5} = \dfrac{15}{100} = 0.15
\]

Each garden plot gets 0.15 kg of fertilizer.

\section*{P6-RoFndDvqWN\_P6-RoFndTmWN\_GPT4.1\_Leisure\_02}
\textbf{Metadata}

\begin{itemize}
  \item Primary KC: RATIO | Finding divided quantities | dividing a quantity in a given ratio
  \item Secondary KC: RATIO | Finding a missing term | finding the missing term in a pair of equivalent ratios
  \item Topic: Leisure
  \item Grade: Primary 6
\end{itemize}

\textbf{Question}

A group of friends decided to share the cost of renting a badminton court. The total cost is divided between Ali and Ben in the ratio \(2:3\). If Ben paid \$45, 

(a) How much did Ali pay?

(b) If the total number of hours the court was booked is 10 hours, and the number of hours Ali used it compared to Ben is in the same ratio as the money they paid, how many hours did Ali use?

\textbf{Solution}

Let's denote the ratio of money Ali paid to Ben as $2:3$.

(a) Since Ben paid $\$45$, which represents the '3' parts of the ratio, we first find the value of one part:

\[ 3 \rightarrow 45 \\
1 \rightarrow 45 \div 3 = 15 \]

Ali's share (2 parts):

\[ 2 \times 15 = 30 \]

Ali paid $\$30$.

(b) The number of hours Ali used compared to Ben is also $2:3$.

Total parts = $2 + 3 = 5$

Each part represents:

\[ 10 \div 5 = 2~\text{hours} \]

Ali used:

\[ 2 \times 2 = 4~\text{hours} \]

\textbf{Answers:}
(a) Ali paid $\$30$.\newline
(b) Ali used $4$ hours.

\section*{P6-FrDivPP\_P5-FrSubMix\_GPT4.1\_Manufacturing\_02}
\textbf{Metadata}

\begin{itemize}
  \item Primary KC: FRACTIONS | Division | dividing a whole number/proper fraction by a proper fraction
  \item Secondary KC: FRACTIONS | Subtraction | subtracting mixed numbers
  \item Topic: Manufacturing
  \item Grade: Primary 6
\end{itemize}

\textbf{Question}

\textbf{Question:}\\
A factory produces metal rods that are each $\frac{3}{4}$ metre long. There is a piece of metal measuring $9$ metres.\\
(a) How many such metal rods of $\frac{3}{4}$ metre each can be cut from the $9$-metre piece?\\
After making the rods, the leftover metal is further cut so that $2\frac{1}{2}$ metres are used for another project.\\
(b) How much metal is left after using $2\frac{1}{2}$ metres for the other project?

\textbf{Solution}

\textbf{Solution:}\\
(a) \text{Number of rods} = 9 \div \frac{3}{4} = 9 \times \frac{4}{3} = \frac{36}{3} = 12.\\
So, 12 rods of $\frac{3}{4}$ metre each can be cut from the $9$-metre piece.\\
\text{Total length used for rods} = 12 \times \frac{3}{4} = 12 \times 0.75 = 9$ metres$.\\
So, there is no leftover from this part, but let's suppose the last rod is never incomplete and all $9$ metres are used (if not, you could cut $12$ full rods and have $0$ left).\\
\\
(b) After making the rods, if there is any leftover, we subtract $2\frac{1}{2}$ metres for another project.\newline
Suppose instead that out of a total of $9$ metres of metal, only $8\frac{1}{2}$ metres were used for rods (for variety) and the remainder is $9 - 8\frac{1}{2} = \frac{1}{2}$ metre. Then we subtract $2\frac{1}{2}$ metres: \\ 
$9 - 2\frac{1}{2} = 9 - \frac{5}{2} = \frac{18}{2} - \frac{5}{2} = \frac{13}{2} = 6\frac{1}{2}$ metres left.\\
\text{Therefore, after using $2\frac{1}{2}$ metres, $6\frac{1}{2}$ metres of metal is left.}

\section*{P6-PcFndWN\_P3-WNSub4d\_GPT4.1\_Manufacturing\_02}
\textbf{Metadata}

\begin{itemize}
  \item Primary KC: PERCENTAGE | Finding the whole | finding the whole given a part and the percentage
  \item Secondary KC: WHOLE NUMBERS | Subtraction | subtraction up to 4 digits
  \item Topic: Manufacturing
  \item Grade: Primary 6
\end{itemize}

\textbf{Question}

Question: 

A factory produced some pens in a day. 30\% of the pens, which is 540 pens, were found to be in perfect condition. The rest had minor defects. After removing 192 pens with severe defects from the defective lot, how many pens with minor defects were left?


\textbf{Solution}

Solution:

Let the total number of pens produced be $x$.

Since 30\% of the pens were in perfect condition and that is 540 pens,

$30\%$ of $x = 540$

$\frac{30}{100} \times x = 540$

$\Rightarrow x = \frac{540 \times 100}{30}$

$= \frac{54000}{30}$

$= 1800$

So, the factory produced 1800 pens in total.

The number of pens with minor defects $= 1800-540=1260$

After removing 192 pens with severe defects, the number of pens with minor defects left $= 1260-192=1068$

Answer: 1068 pens with minor defects were left.

\section*{P6-PcFndChg\_P4-WNDiv4d1d\_GPT4.1\_Food\_02}
\textbf{Metadata}

\begin{itemize}
  \item Primary KC: PERCENTAGE | Finding change | finding percentage increase/decrease
  \item Secondary KC: WHOLE NUMBERS | Division | division up to 4 digits by 1 digit
  \item Topic: Food
  \item Grade: Primary 6
\end{itemize}

\textbf{Question}

\textbf{Problem:}\\
Ben bought a pack of 560 cookies to sell at a school food fair. On the first day, he sold 35\% of the cookies. On the second day, he sold the remaining cookies equally among 5 groups of students.\\
(a) How many cookies did Ben sell on the first day?\\
(b) How many cookies did each group of students get on the second day?

\textbf{Solution}

\textbf{Solution:}\\
(a)\quad \text{Number of cookies Ben sold on the first day} = 35\% \times 560 \\ 
\quad = \dfrac{35}{100} \times 560 \\ 
\quad = 0.35 \times 560 \\ 
\quad = 196 \\ 
\text{Ben sold 196 cookies on the first day.}\\
(b)\quad \text{Number of cookies left} = 560 - 196 = 364 \\ 
\quad \text{Number of cookies each group received} = \dfrac{364}{5} = 72.8 \\ 
Since the cookies cannot be divided into decimal pieces, Ben can give 72 cookies to each group and have 4 cookies left over, or he may split the remaining cookies as he chooses.\\ 
\text{Each group of students received 72 cookies, with 4 cookies remaining.}

\section*{P6-FrDivPN\_P4-FrRepSet\_GPT4.1\_Food\_02}
\textbf{Metadata}

\begin{itemize}
  \item Primary KC: FRACTIONS | Division | dividing a proper fraction by a whole number
  \item Secondary KC: FRACTIONS | Representation and concept | fraction as part of a set 
  \item Topic: Food
  \item Grade: Primary 6
\end{itemize}

\textbf{Question}

Question:

A baker made \( \frac{3}{4} \) of a tray of brownies for a party. He wants to share the brownies equally among 5 friends. What fraction of a whole tray of brownies does each friend get? Give your answer in its simplest form.

\textbf{Solution}

Solution:

First, we are asked to divide \( \frac{3}{4} \) of a tray by 5 friends. 

\[
\text{Each friend gets: } \frac{3}{4} \div 5 = \frac{3}{4} \times \frac{1}{5} = \frac{3}{20}
\]

So, each friend will receive \( \frac{3}{20} \) of a whole tray of brownies.

Each fraction represents the part (\( \frac{3}{20} \)) of the entire set (one whole tray) that each friend will receive.

\section*{P6-RoFndDvqWN\_P6-RoFndTmWN\_GPT4.1\_Food\_02}
\textbf{Metadata}

\begin{itemize}
  \item Primary KC: RATIO | Finding divided quantities | dividing a quantity in a given ratio
  \item Secondary KC: RATIO | Finding a missing term | finding the missing term in a pair of equivalent ratios
  \item Topic: Food
  \item Grade: Primary 6
\end{itemize}

\textbf{Question}

\textbf{Question:}\\
A baker has 210 g of chocolate, which she wants to divide between dark and milk chocolate in the ratio $3:4$.\\
(a) How many grams of each type of chocolate should she use?\\
Later, she makes another batch using the same ratio of dark to milk chocolate, but this time, she uses 90 g of dark chocolate.\\
(b) How many grams of milk chocolate does she use to keep the ratio the same?

\textbf{Solution}

\textbf{Solution:}\\
(a) The ratio of dark : milk chocolate is $3:4$.\\
Total parts $= 3 + 4 = 7$.\\
Amount of dark chocolate $= \frac{3}{7} \times 210 = 90~\text{g}$.\\
Amount of milk chocolate $= \frac{4}{7} \times 210 = 120~\text{g}$.\\[1em]
(b) Let the amount of milk chocolate be $x$ g.\\
To keep the same ratio $3:4$,\\
\[
\frac{90}{x} = \frac{3}{4}
\]
Cross multiply: $90 \times 4 = 3 \times x$\\
$360 = 3x$\\
$x = 120$\\
So, she uses $120$ g of milk chocolate.

\section*{P6-PcFndChg\_P3-WNAdd4d\_GPT4.1\_Farming\_02}
\textbf{Metadata}

\begin{itemize}
  \item Primary KC: PERCENTAGE | Finding change | finding percentage increase/decrease
  \item Secondary KC: WHOLE NUMBERS | Addition | addition up to 4 digits
  \item Topic: Farming
  \item Grade: Primary 6
\end{itemize}

\textbf{Question}

A farmer harvested 1,350 kg of tomatoes last year. This year, he harvested 1,710 kg of tomatoes. 

(a) What is the total amount of tomatoes harvested over the two years?

(b) By what percentage did the tomato harvest increase from last year to this year?

\textbf{Solution}

Let the amount of tomatoes harvested last year be $1,350$ kg.
Let the amount harvested this year be $1,710$ kg.

\textbf{(a) Total harvested over two years:}

$1,350 + 1,710 = 3,060$ kg

\textbf{Answer to (a):} $3,060$ kg

\textbf{(b) Percentage increase from last year:}

$\text{Increase} = 1,710 - 1,350 = 360$ kg

$\text{Percentage increase} = \frac{360}{1,350} \times 100\%$

$= 0.2667 \times 100\%$

$= 26.67\%$

\textbf{Answer to (b):} The tomato harvest increased by approximately $26.7\%$ from last year to this year.

\section*{P6-AgRepLrEx\_P6-AgEvlLrEx\_GPT4.1\_Digital ecconomy\_02}
\textbf{Metadata}

\begin{itemize}
  \item Primary KC: ALGEBRA | Representation and concept | translation of simple real-world situations into linear algebraic expressions
  \item Secondary KC: ALGEBRA | Evaluation | evaluating simple linear expressions by substitution
  \item Topic: Digital ecconomy
  \item Grade: Primary 6
\end{itemize}

\textbf{Question}

\textbf{Question:} \\ 
Janet is planning to start an online business selling e-books. She finds that for every e-book she sells, she earns \$8, and she has to pay a monthly platform fee of \$20. Let $n$ be the number of e-books Janet sells in a month. \\ 
(a) Write an algebraic expression for the total profit Janet makes in a month in terms of $n$. \\ 
(b) If Janet sells 15 e-books in a month, calculate her total profit for that month.

\textbf{Solution}

\textbf{Solution:} \\ 
(a) Janet earns \$8 for each e-book, so her income is $8n$. She pays a fixed fee of \$20, so her profit is: \\ 
\[ \text{Total Profit} = 8n - 20 \] \\ 
(b) If $n = 15$, substitute 15 into the expression: \\ 
\[ \text{Total Profit} = 8 \times 15 - 20 = 120 - 20 = 100 \] \\ 
Janet’s total profit for that month is \$100.

\section*{P6-RoFndRoWN\_P4-WNDiv4d1d\_GPT4.1\_Transporation\_02}
\textbf{Metadata}

\begin{itemize}
  \item Primary KC: RATIO | Finding ratio | finding the ratio of two or three given whole numbers
  \item Secondary KC: WHOLE NUMBERS | Division | division up to 4 digits by 1 digit
  \item Topic: Transporation
  \item Grade: Primary 6
\end{itemize}

\textbf{Question}

A bus company has 1,296 adult passengers and 432 child passengers taking a trip from Singapore to Johor Bahru. The company wants to arrange the passengers equally onto buses that can each carry only adult or only child passengers. If each bus can carry 36 adult passengers or 36 child passengers, 

(a) How many buses are needed for the adults and how many for the children?

(b) What is the ratio of buses used for adults to buses used for children? Give your answer in its simplest form.

\textbf{Solution}

Let us calculate the number of buses needed for adults:

Number of adults = 1,296
Capacity per adult bus = 36
Number of adult buses $= \frac{1,296}{36} = 36$

Number of children = 432
Capacity per child bus = 36
Number of child buses $= \frac{432}{36} = 12$

Now, finding the ratio of adult buses to child buses:

Ratio = 36 : 12
To simplify, divide both by 12:
$\frac{36}{12} : \frac{12}{12} = 3 : 1$

\textbf{Answer:}
(a) 36 buses are needed for adults and 12 buses are needed for children.

(b) The ratio of buses used for adults to buses used for children is $3 : 1$.

\section*{P6-PcFndChg\_P4-WNMul4d1d\_GPT4.1\_Digital ecconomy\_02}
\textbf{Metadata}

\begin{itemize}
  \item Primary KC: PERCENTAGE | Finding change | finding percentage increase/decrease
  \item Secondary KC: WHOLE NUMBERS | Multiplication | multiplication up to 4 digits by 1 digit or up to 3 digits by 2 digits
  \item Topic: Digital ecconomy
  \item Grade: Primary 6
\end{itemize}

\textbf{Question}

\textbf{Question:} \\ 
A shop in Singapore sold 150 digital tablets in January. In February, the shop increased its sales by 28\% compared to January. Each tablet is priced at \$790. \\ 
(a) How many digital tablets did the shop sell in February? \\ 
(b) What is the total revenue from the sale of tablets in February?

\textbf{Solution}

\textbf{Solution:} \\ 
(a) \text{Number of tablets sold in January} = 150 \\ 
\text{Percentage increase} = 28\% \\ 
\text{Increase in number of tablets} = 28\% \times 150 = \frac{28}{100} \times 150 = 42 \\ 
\text{Total tablets sold in February} = 150 + 42 = 192 \\ 
(b) \text{Price per tablet} = \$790 \\ 
\text{Total revenue in February} = 192 \times 790 \\ 
\quad = 151{,}680 \\ 
\text{Therefore, the total revenue from the sale of tablets in February is } \$151{,}680.}

\section*{P6-RoFndRoWN\_P3-WNSub4d\_GPT4.1\_Household finance\_02}
\textbf{Metadata}

\begin{itemize}
  \item Primary KC: RATIO | Finding ratio | finding the ratio of two or three given whole numbers
  \item Secondary KC: WHOLE NUMBERS | Subtraction | subtraction up to 4 digits
  \item Topic: Household finance
  \item Grade: Primary 6
\end{itemize}

\textbf{Question}

\textbf{Ben and his sister Lisa are saving money for their household expenses. Ben saved \$2,350 while Lisa saved \$1,480. However, Ben decided to use \$350 from his savings to buy groceries. After buying the groceries, what is the ratio of Ben's remaining savings to Lisa's savings? Write your answer in its simplest form.}

\textbf{Solution}

\textbf{Step 1: Calculate Ben's remaining savings.} \\ Ben's initial savings = \$2,350 \\ Amount spent by Ben = \$350 \\ Ben's remaining savings = 2,350 - 350 = \$2,000 \\ \\ Lisa's savings = \$1,480 \\ \\ Step 2: Find the ratio of Ben's remaining savings to Lisa's savings. \\ Ben : Lisa = 2,000 : 1,480 \\ \\ Step 3: Simplify the ratio. \\ Find the greatest common divisor (GCD) of 2,000 and 1,480. \\ 2,000 \div 40 = 50 \\ 1,480 \div 40 = 37 \\ \\ So, the simplest ratio is \boxed{50:37}.

\section*{P6-PcFndWN\_P3-WNAdd4d\_GPT4.1\_Farming\_02}
\textbf{Metadata}

\begin{itemize}
  \item Primary KC: PERCENTAGE | Finding the whole | finding the whole given a part and the percentage
  \item Secondary KC: WHOLE NUMBERS | Addition | addition up to 4 digits
  \item Topic: Farming
  \item Grade: Primary 6
\end{itemize}

\textbf{Question}

A farmer harvested some tomatoes from his farm. He packed 672 tomatoes and gave them to a grocery store, which was 28\% of his total harvest. On the next day, he harvested and packed another 1,248 tomatoes from the same farm. What was the total number of tomatoes he harvested in these two days?

\textbf{Solution}

Let the total number of tomatoes harvested on the first day be $x$.

Given that $672$ is $28\%$ of $x$:

$672 = 28\% \times x$

$672 = \frac{28}{100} \times x$

$\Rightarrow x = \frac{672 \times 100}{28}$

$\Rightarrow x = \frac{67200}{28}$

$\Rightarrow x = 2,400$

So, the farmer harvested $2,400$ tomatoes on the first day.

On the next day, he harvested $1,248$ tomatoes.

Total number of tomatoes harvested in two days:
$2,400 + 1,248 = 3,648$

\textbf{Answer:} The farmer harvested a total of $3,648$ tomatoes in these two days.

\section*{P6-FrDivPP\_P5-FrSubMix\_GPT4.1\_Education\_02}
\textbf{Metadata}

\begin{itemize}
  \item Primary KC: FRACTIONS | Division | dividing a whole number/proper fraction by a proper fraction
  \item Secondary KC: FRACTIONS | Subtraction | subtracting mixed numbers
  \item Topic: Education
  \item Grade: Primary 6
\end{itemize}

\textbf{Question}

\textbf{Word Problem}

Ali has $3$ books. He wants to share them equally among his friends so that each friend gets $\frac{2}{3}$ of a book.}

(a) How many friends can Ali share his books with if each friend gets $\frac{2}{3}$ of a book?

Later, he decides to keep $1\dfrac{1}{2}$ books for himself before sharing the rest with his friends.}

(b) After keeping $1\dfrac{1}{2}$ books, how many books does Ali have left to share with his friends?

\textbf{Solution}

\textbf{Solution}

(a)

Number of friends $=$ $3 \div \frac{2}{3}$

$= 3 \times \frac{3}{2}$

$= \frac{9}{2}$

$= 4\dfrac{1}{2}$

So, Ali can share his $3$ books equally among $4$ friends (with each getting $\frac{2}{3}$ of a book), and there will be enough for half the amount for a $5$th friend.

(b)

Ali keeps $1\dfrac{1}{2}$ books for himself.

Books left to share $= 3 - 1\dfrac{1}{2}$

$= 3 - \frac{3}{2}$

Convert $3$ to $\frac{6}{2}$:

$= \frac{6}{2} - \frac{3}{2}$

$= \frac{3}{2}$

\textbf{Answer:}

(a) Ali can share his $3$ books among $4\dfrac{1}{2}$ friends if each gets $\frac{2}{3}$ of a book.

(b) After keeping $1\dfrac{1}{2}$ books, Ali has $\frac{3}{2}$ books left to share with his friends.

\section*{P6-FrDivPP\_P5-FrMulMixN\_GPT4.1\_Education\_02}
\textbf{Metadata}

\begin{itemize}
  \item Primary KC: FRACTIONS | Division | dividing a whole number/proper fraction by a proper fraction
  \item Secondary KC: FRACTIONS | Multiplication | multiplying a mixed number and a whole number
  \item Topic: Education
  \item Grade: Primary 6
\end{itemize}

\textbf{Question}

A mathematics teacher has $\displaystyle 12$ pieces of coloured paper. She wants to give each student $1\frac{1}{2}$ pieces of coloured paper for a class art activity. 

(a) How many students can she give coloured paper to? 

(b) If there are 4 groups in the class, and each group receives an equal number of total coloured papers as above, how many pieces of coloured paper does each group get?

\textbf{Solution}

(a) To find out how many students can get $1\frac{1}{2}$ pieces of coloured paper each from $12$ pieces, we divide:

$$
12 \div 1\frac{1}{2} = 12 \div \frac{3}{2} = 12 \times \frac{2}{3} = 8.
$$

So, the teacher can give coloured paper to **8 students**.

(b) Total coloured paper is $12$ pieces to be shared equally among 4 groups:

Each group gets:
$$
12 \div 4 = 3 \text{ pieces}
$$

Each group therefore receives **3 pieces of coloured paper**.

\section*{P6-FrDivPN\_P5-FrSubMix\_GPT4.1\_Transporation\_02}
\textbf{Metadata}

\begin{itemize}
  \item Primary KC: FRACTIONS | Division | dividing a proper fraction by a whole number
  \item Secondary KC: FRACTIONS | Subtraction | subtracting mixed numbers
  \item Topic: Transporation
  \item Grade: Primary 6
\end{itemize}

\textbf{Question}

In a race, Sarah cycled a distance of $\frac{3}{4}$ km each day for 4 days. After completing her journey, she realised that her friend, Mei Lin, cycled $2\frac{1}{2}$ km in total. 

(a) How far did Sarah cycle each day? Express your answer in km. 

(b) By how much did Sarah cycle less than Mei Lin? Give your answer in km.

\textbf{Solution}

(a) To find how far Sarah cycled each day, divide the total distance Sarah cycled by the number of days:
Sarah's total distance = $\frac{3}{4}$ km
Number of days = 4

Distance per day = $\frac{3}{4} \div 4 = \frac{3}{4} \times \frac{1}{4} = \frac{3}{16}$ km$

(b) To find by how much Sarah cycled less than Mei Lin, subtract Sarah's total distance from Mei Lin's total distance:
Mei Lin's distance = $2\frac{1}{2} = \frac{5}{2}$ km
Sarah's distance = $\frac{3}{4}$ km

Difference = $\frac{5}{2} - \frac{3}{4}$
To subtract, convert to a common denominator:
$\frac{5}{2} = \frac{10}{4}$

So,
Difference = $\frac{10}{4} - \frac{3}{4} = \frac{7}{4}$ km = $1\frac{3}{4}$ km

Final Answers:
(a) $\frac{3}{16}$ km
(b) $1\frac{3}{4}$ km

\section*{P6-AgRepLrEx\_P6-AgEvlLrEx\_GPT4.1\_Food\_02}
\textbf{Metadata}

\begin{itemize}
  \item Primary KC: ALGEBRA | Representation and concept | translation of simple real-world situations into linear algebraic expressions
  \item Secondary KC: ALGEBRA | Evaluation | evaluating simple linear expressions by substitution
  \item Topic: Food
  \item Grade: Primary 6
\end{itemize}

\textbf{Question}

A bakery sells cupcakes at $x$ cents each and cookies at $y$ cents each. Rachel buys 5 cupcakes and 3 cookies. \\ (a) Write down an algebraic expression, in terms of $x$ and $y$, for the total amount (in cents) Rachel pays. \\ (b) If $x = 60$ and $y = 40$, find the total amount that Rachel pays.

\textbf{Solution}

(a) The total amount Rachel pays in cents is $5x + 3y$. \\ (b) If $x = 60$ and $y = 40$, then the total amount is $5 \times 60 + 3 \times 40 = 300 + 120 = 420$ cents.

\section*{P6-PcFndWN\_P3-WNAdd4d\_GPT4.1\_Manufacturing\_02}
\textbf{Metadata}

\begin{itemize}
  \item Primary KC: PERCENTAGE | Finding the whole | finding the whole given a part and the percentage
  \item Secondary KC: WHOLE NUMBERS | Addition | addition up to 4 digits
  \item Topic: Manufacturing
  \item Grade: Primary 6
\end{itemize}

\textbf{Question}

A factory produced some toy cars in one day. 35\% of the toy cars produced were painted red, and there were 560 red toy cars. Later, the factory produced 1,325 more toy cars the next day. 

(a) How many toy cars did the factory produce on the first day? 
(b) What is the total number of toy cars produced by the factory after the two days?

\textbf{Solution}

(a) Let the total number of toy cars produced on the first day be $x$. 35\% of $x$ is 560.

So, \( 0.35x = 560 \)

\( x = \frac{560}{0.35} = 1600 \)

Thus, the factory produced \textbf{1,600} toy cars on the first day.

(b) Total number of toy cars after two days = 1,600 + 1,325 = \boxed{2,925}.

\section*{P6-FrDivPN\_P3-FrSmp\_GPT4.1\_Manufacturing\_02}
\textbf{Metadata}

\begin{itemize}
  \item Primary KC: FRACTIONS | Division | dividing a proper fraction by a whole number
  \item Secondary KC: FRACTIONS | Simplifying | expressing a fraction in its simplest form
  \item Topic: Manufacturing
  \item Grade: Primary 6
\end{itemize}

\textbf{Question}

\textbf{Manufacturing Fraction Problem} \\ \\ A factory produces \(\dfrac{5}{6}\) metres of ribbon for each dress it makes. The factory plans to divide this total ribbon equally among 3 new dresses. How much ribbon does each new dress get? Express your answer in its simplest form.

\textbf{Solution}

\textbf{Solution} \\ \\ The total length of ribbon to be divided is \(\dfrac{5}{6}\) metres. \\ We are dividing this ribbon equally among 3 dresses. \\ \\ Amount of ribbon per new dress = \(\dfrac{5}{6} \div 3\) \\ \\ To divide a fraction by a whole number, we multiply the fraction by the reciprocal of the whole number: \\ \[ \dfrac{5}{6} \div 3 = \dfrac{5}{6} \times \dfrac{1}{3} = \dfrac{5 \times 1}{6 \times 3} = \dfrac{5}{18} \] \\ \(\dfrac{5}{18}\) is already in its simplest form. \\ \\ \textbf{Answer:} Each new dress gets \(\dfrac{5}{18}\) metres of ribbon.

\section*{P6-FrDivPP\_P3-FrSmp\_GPT4.1\_Services\_02}
\textbf{Metadata}

\begin{itemize}
  \item Primary KC: FRACTIONS | Division | dividing a whole number/proper fraction by a proper fraction
  \item Secondary KC: FRACTIONS | Simplifying | expressing a fraction in its simplest form
  \item Topic: Services
  \item Grade: Primary 6
\end{itemize}

\textbf{Question}

A cleaning company needs to distribute 5 litres of cleaning solution equally into bottles, each bottle holding $\frac{2}{5}$ litre. 

(a) How many bottles can be filled?

(b) Express your answer as a whole number or a fraction in its simplest form, if there is any solution left over.

\textbf{Solution}

First, we divide the total amount by the volume of each bottle:

$5 \div \frac{2}{5} = 5 \times \frac{5}{2} = \frac{25}{2}$

$\frac{25}{2} = 12\frac{1}{2}$

So, (a) 12 bottles can be completely filled.

(b) There is $\frac{1}{2}$ bottle worth of cleaning solution left.

Final answer: $12\frac{1}{2}$ bottles, or twelve and a half bottles.

\section*{P6-FrDivPN\_P3-FrSmp\_GPT4.1\_Education\_02}
\textbf{Metadata}

\begin{itemize}
  \item Primary KC: FRACTIONS | Division | dividing a proper fraction by a whole number
  \item Secondary KC: FRACTIONS | Simplifying | expressing a fraction in its simplest form
  \item Topic: Education
  \item Grade: Primary 6
\end{itemize}

\textbf{Question}

\textbf{Word Problem:} \\ 
A class project requires students to share some coloured paper for their craftwork. Mrs Lim bought $\dfrac{7}{8}$ of a sheet of coloured paper and wants to divide it equally among 4 students. \\ 
(a) How much coloured paper does each student get? \\ 
(b) Give your answer in its simplest form.

\textbf{Solution}

\textbf{Solution:} \\ 
To find how much coloured paper each student gets, divide $\dfrac{7}{8}$ by 4. \\ 
(a) \[ \frac{7}{8} \div 4 = \frac{7}{8} \times \frac{1}{4} = \frac{7}{32} \] \\ 
(b) $\frac{7}{32}$ is already in its simplest form, since 7 and 32 have no common factor other than 1. \\ 
\textbf{Answer:} Each student gets $\dfrac{7}{32}$ of a sheet of coloured paper.

\section*{P6-FrDivPN\_P5-FrAddMix\_GPT4.1\_Household finance\_02}
\textbf{Metadata}

\begin{itemize}
  \item Primary KC: FRACTIONS | Division | dividing a proper fraction by a whole number
  \item Secondary KC: FRACTIONS | Addition | adding mixed numbers
  \item Topic: Household finance
  \item Grade: Primary 6
\end{itemize}

\textbf{Question}

\textbf{Sarah is baking cakes for a family gathering. She has}~\frac{3}{4}~\text{kg of flour that she wants to share equally among 3 cake recipes.} \\ 
\text{After dividing the flour, she realizes she also has 1}\frac{1}{2}\text{kg of sugar and 2}\frac{1}{4}\text{kg of butter left.} \\ 
\text{How much flour will Sarah use for each recipe? What is the total weight (in kg) of sugar and butter left?}

\textbf{Solution}

\textbf{Step 1: Divide the flour equally among 3 recipes.} \\ 
\text{Flour per recipe:}~\frac{3}{4} \div 3 = \frac{3}{4} \times \frac{1}{3} = \frac{3}{12} = \frac{1}{4}~\text{kg} \\ 
\textbf{Step 2: Add the mixed numbers to find the total weight of sugar and butter.} \\ 
1\frac{1}{2} + 2\frac{1}{4} = \left(1 + \frac{1}{2}\right) + \left(2 + \frac{1}{4}\right) \\ 
= (1 + 2) + \left(\frac{1}{2} + \frac{1}{4}\right) \\ 
= 3 + \left(\frac{2}{4} + \frac{1}{4}\right) \\ 
= 3 + \frac{3}{4} \\ 
= 3\frac{3}{4}~\text{kg} \\ 
\boxed{\text{Sarah will use}\ \frac{1}{4}\ \text{kg of flour for each recipe, and she has}\ 3\frac{3}{4}\ \text{kg of sugar and butter left in total.}}

\section*{P6-FrDivPN\_P2-FrCmp\_GPT4.1\_Services\_02}
\textbf{Metadata}

\begin{itemize}
  \item Primary KC: FRACTIONS | Division | dividing a proper fraction by a whole number
  \item Secondary KC: FRACTIONS | Comparison and ordering | comparing and ordering fractions
  \item Topic: Services
  \item Grade: Primary 6
\end{itemize}

\textbf{Question}

A cleaning company received an order to clean several office buildings. For each building, they use $\frac{3}{4}$ litre of cleaning detergent. The company has a container with $2$ litres of detergent. 

(a) How many buildings can they clean if they divide the detergent equally, using $\frac{3}{4}$ litre for each building?

(b) After cleaning as many buildings as possible, compare the fraction of detergent used to the fraction left. Which fraction is greater?

\textbf{Solution}

(a) To find how many buildings can be cleaned, divide the total detergent by the amount needed for each building:

$2 \div \frac{3}{4} = 2 \times \frac{4}{3} = \frac{8}{3}$

$\frac{8}{3} = 2\frac{2}{3}$

So, the company can clean $2$ buildings completely, and $\frac{2}{3}$ of the detergent required for the third building is available (not enough for a complete third building).

(b) The amount of detergent used for 2 buildings: $2 \times \frac{3}{4} = \frac{6}{4} = \frac{3}{2}$ litres.

Amount of detergent left: $2 - \frac{3}{2} = \frac{4}{2} - \frac{3}{2} = \frac{1}{2}$ litre.

Compare $\frac{3}{2}$ (used) and $\frac{1}{2}$ (left):

$\frac{3}{2} > \frac{1}{2}$

So, the fraction of detergent used is greater than the fraction left.

\section*{P6-FrDivPP\_P5-FrMulMixN\_GPT4.1\_Transporation\_02}
\textbf{Metadata}

\begin{itemize}
  \item Primary KC: FRACTIONS | Division | dividing a whole number/proper fraction by a proper fraction
  \item Secondary KC: FRACTIONS | Multiplication | multiplying a mixed number and a whole number
  \item Topic: Transporation
  \item Grade: Primary 6
\end{itemize}

\textbf{Question}

A train journey from Singapore to Kuala Lumpur takes $\frac{3}{4}$ of an hour. There are 12 passengers in one cabin, and each passenger will take turns to drive a small toy train that lasts $\frac{3}{4}$ of an hour per turn.

(a) How many passengers can drive the toy train in 6 hours if each turn lasts $\frac{3}{4}$ of an hour?

(b) If every passenger gets 2 turns to drive the toy train, and each turn lasts $1\frac{1}{2}$ hours, how many hours in total are needed if there are 5 passengers?

\textbf{Solution}

(a) Number of passengers who can drive = Total time \div Time per turn = $6 \div \frac{3}{4}$

$= 6 \times \frac{4}{3} = 8$

So, 8 passengers can drive the toy train in 6 hours.

(b) Each passenger gets 2 turns, so total number of turns = $5 \times 2 = 10$
Total time for each turn = $1\frac{1}{2}$ hours = $\frac{3}{2}$ hours

Total time needed = $10 \times \frac{3}{2} = 15$ hours.

So, 15 hours are needed in total if there are 5 passengers each getting 2 turns, and each turn lasts $1\frac{1}{2}$ hours.

\section*{P6-PcFndChg\_P4-WNDiv4d1d\_GPT4.1\_Leisure\_02}
\textbf{Metadata}

\begin{itemize}
  \item Primary KC: PERCENTAGE | Finding change | finding percentage increase/decrease
  \item Secondary KC: WHOLE NUMBERS | Division | division up to 4 digits by 1 digit
  \item Topic: Leisure
  \item Grade: Primary 6
\end{itemize}

\textbf{Question}

A group of 648 students from six schools participated in a sports carnival last year. This year, the number of students who participated increased by 25\%. Each school sent the same number of students again. 

(a) How many students participated in the carnival this year?

(b) How many students did each school send this year?

\textbf{Solution}

Let the number of students last year be 648.

(a) Increase = 25\% of 648 = \frac{25}{100} \times 648 = 162

Total students this year = 648 + 162 = 810

So, 810 students participated this year.

(b) Number of schools = 6

Each school sent = 810 \div 6 = 135 students this year.

\section*{P6-FrDivPP\_P4-FrRepSet\_GPT4.1\_Services\_02}
\textbf{Metadata}

\begin{itemize}
  \item Primary KC: FRACTIONS | Division | dividing a whole number/proper fraction by a proper fraction
  \item Secondary KC: FRACTIONS | Representation and concept | fraction as part of a set 
  \item Topic: Services
  \item Grade: Primary 6
\end{itemize}

\textbf{Question}

\textbf{Mrs Lim's Bakery}\\
Mrs Lim is packing cupcakes into boxes for a charity event. She has 36 cupcakes. Each box can hold $\frac{3}{4}$ of a dozen cupcakes.\\

(a) How many boxes does Mrs Lim need to pack all the cupcakes?\\
(b) What fraction of the total cupcakes does each box contain, in terms of the whole set of 36 cupcakes?\\

\textbf{Solution}

\textbf{Solution}\\
(a) Each box holds $\frac{3}{4}$ of a dozen cupcakes. Since 1 dozen = 12 cupcakes,\\
$\frac{3}{4} \times 12 = 9$ cupcakes per box.\\
Total cupcakes = 36.\\
Number of boxes = $\frac{36}{9} = 4$.\\
\textbf{Mrs Lim needs 4 boxes.}\\

(b) Each box contains 9 cupcakes. \\Total cupcakes = 36.\\
Fraction per box = $\frac{9}{36} = \frac{1}{4}$.\\
\textbf{Each box contains $\frac{1}{4}$ of the entire set of cupcakes.}

\section*{P6-FrDivPN\_P4-FrRepSet\_GPT4.1\_Farming\_02}
\textbf{Metadata}

\begin{itemize}
  \item Primary KC: FRACTIONS | Division | dividing a proper fraction by a whole number
  \item Secondary KC: FRACTIONS | Representation and concept | fraction as part of a set 
  \item Topic: Farming
  \item Grade: Primary 6
\end{itemize}

\textbf{Question}

\textbf{Sarah owns a vegetable farm.} \newline 
She harvested $\frac{3}{4}$ of a crate of tomatoes. She wants to divide these tomatoes equally among her 5 friends.}

\begin{enumerate}
    \item How much of the crate of tomatoes does each friend receive?
    \item If each friend represents one part of a set of 5, what fraction of the original crate does each friend get in terms of the set?
\end{enumerate}

\textbf{Solution}

\textbf{Solution} 

\begin{enumerate}
    \item Each friend receives: \[ \frac{3}{4} \div 5 = \frac{3}{4} \times \frac{1}{5} = \frac{3}{20} \] 
    Each friend receives $\frac{3}{20}$ of a crate.
    
    \item Since the crate is divided into 5 equal parts for the 5 friends, each friend receives 1 part out of 5. But Sarah only has $\frac{3}{4}$ of a crate to share. Each friend's share is $\frac{3}{20}$ of the set (the whole crate), as calculated above. Therefore, in terms of the set, each friend receives $\frac{3}{20}$ of the original crate.
\end{enumerate}

\section*{P6-FrDivPP\_P3-FrSmp\_GPT4.1\_Leisure\_02}
\textbf{Metadata}

\begin{itemize}
  \item Primary KC: FRACTIONS | Division | dividing a whole number/proper fraction by a proper fraction
  \item Secondary KC: FRACTIONS | Simplifying | expressing a fraction in its simplest form
  \item Topic: Leisure
  \item Grade: Primary 6
\end{itemize}

\textbf{Question}

\textbf{Question} \\ 
Eileen bought a roll of ribbon that is $3$ metres long for her art project. She wants to cut the ribbon into pieces, each $\frac{3}{4}$ metre long.  \\ 
(a) How many pieces of ribbon can she cut? \\ 
(b) Express the length of each piece as a fraction of the original roll, in its simplest form.

\textbf{Solution}

\textbf{Solution} \\ 
(a) To find out how many pieces of ribbon Eileen can cut, divide the total length by the length of each piece: \\ 

$3 \div \frac{3}{4} = 3 \times \frac{4}{3} = \frac{12}{3} = 4$ \\ 
So, Eileen can cut $4$ pieces of ribbon. \\ 

(b) Each piece is $\frac{3}{4}$ metre long. The original ribbon is $3$ metres. The fraction of the original roll that one piece represents is: \\ 
$\frac{3}{4} \div 3 = \frac{3}{4} \times \frac{1}{3} = \frac{3}{12} = \frac{1}{4}$ \\ 
So, each piece represents $\frac{1}{4}$ of the original roll.

\section*{P6-FrDivPN\_P5-FrSubMix\_GPT4.1\_Leisure\_02}
\textbf{Metadata}

\begin{itemize}
  \item Primary KC: FRACTIONS | Division | dividing a proper fraction by a whole number
  \item Secondary KC: FRACTIONS | Subtraction | subtracting mixed numbers
  \item Topic: Leisure
  \item Grade: Primary 6
\end{itemize}

\textbf{Question}

\textbf{Nicole made a batch of cookies for a picnic. She used }\frac{3}{4}\text{ kg of chocolate chips, and divided them equally among her 5 friends.}\newline
\text{After giving out the chocolate chips, she realised she needed }2\frac{1}{4}\text{ kg more for another recipe. She only had }3\frac{2}{3}\text{ kg left in her pantry.}\newline
\text{How much chocolate chips will she have left after preparing for the new recipe?}

\textbf{Solution}

\textbf{Step 1: Find the amount of chocolate chips each friend gets.}\newline
\frac{3}{4} \div 5 = \frac{3}{4} \times \frac{1}{5} = \frac{3}{20}\ \text{kg per friend.}\newline
\textbf{Step 2: Subtract the amount needed for the new recipe from what is left.}\newline
\text{Nicole has }3\frac{2}{3}\text{ kg left. She needs }2\frac{1}{4}\text{ kg.}\newline
\text{Subtract the mixed numbers:}\newline
3\frac{2}{3} - 2\frac{1}{4} = \frac{11}{3} - \frac{9}{4}\newline
\text{Find a common denominator (12):}\newline
\frac{11}{3} = \frac{44}{12}\ ,\ \frac{9}{4} = \frac{27}{12}\newline
\frac{44}{12} - \frac{27}{12} = \frac{17}{12}\newline
\frac{17}{12} = 1\frac{5}{12}\ \text{kg}\newline
\boxed{\text{Nicole will have }1\frac{5}{12}\text{ kg of chocolate chips left after preparing for the new recipe.}}

\section*{P6-FrDivPP\_P3-FrSmp\_GPT4.1\_Farming\_02}
\textbf{Metadata}

\begin{itemize}
  \item Primary KC: FRACTIONS | Division | dividing a whole number/proper fraction by a proper fraction
  \item Secondary KC: FRACTIONS | Simplifying | expressing a fraction in its simplest form
  \item Topic: Farming
  \item Grade: Primary 6
\end{itemize}

\textbf{Question}

A farmer has $4$ kg of seeds. He wants to pack the seeds equally into bags, each bag containing $\frac{2}{3}$ kg of seeds. 

(a) How many bags can he fill completely? 

(b) Express your answer in its simplest form.

\textbf{Solution}

To find how many bags the farmer can fill, we divide the total amount of seeds by the amount in each bag:

Number of bags $= \frac{4}{\frac{2}{3}}$

To divide by a fraction, multiply by its reciprocal:

$= 4 \times \frac{3}{2}$

$= \frac{4 \times 3}{2}$

$= \frac{12}{2}$

$= 6$

So, the farmer can fill $6$ bags completely. The answer is already in its simplest form.

\section*{P6-FrDivPN\_P5-FrMulMixN\_GPT4.1\_Education\_02}
\textbf{Metadata}

\begin{itemize}
  \item Primary KC: FRACTIONS | Division | dividing a proper fraction by a whole number
  \item Secondary KC: FRACTIONS | Multiplication | multiplying a mixed number and a whole number
  \item Topic: Education
  \item Grade: Primary 6
\end{itemize}

\textbf{Question}

In a class, Miss Tan is preparing art supplies for a group activity. She has $\frac{3}{4}$ litres of blue paint to be shared equally among 6 groups of students. 

(a) How much blue paint will each group receive?

For another activity, Miss Tan also prepares $2\frac{1}{3}$ kilograms of modelling clay for each group, and there are 4 groups. 

(b) What is the total mass of modelling clay needed for all 4 groups?

\textbf{Solution}

\textbf{(a) Amount of blue paint each group receives:}

Miss Tan divides $\frac{3}{4}$ litres of blue paint equally among 6 groups:

$$
\frac{3}{4} \div 6 = \frac{3}{4} \times \frac{1}{6} = \frac{3}{24} = \frac{1}{8}
$$

Each group receives $\frac{1}{8}$ litre of blue paint.

\textbf{(b) Total mass of modelling clay:}

Each group needs $2\frac{1}{3}$ kg of clay. For 4 groups:

First, convert $2\frac{1}{3}$ to improper fraction:
$$
2\frac{1}{3} = \frac{7}{3}
$$

Multiply by 4:
$$
\frac{7}{3} \times 4 = \frac{28}{3} \text{ kg}
$$

$\frac{28}{3}$ kg can be written as $9\frac{1}{3}$ kg.

So, the total mass of modelling clay needed is $9\frac{1}{3}$ kg.

\section*{P6-FrDivPP\_P5-FrMulMixN\_GPT4.1\_Services\_02}
\textbf{Metadata}

\begin{itemize}
  \item Primary KC: FRACTIONS | Division | dividing a whole number/proper fraction by a proper fraction
  \item Secondary KC: FRACTIONS | Multiplication | multiplying a mixed number and a whole number
  \item Topic: Services
  \item Grade: Primary 6
\end{itemize}

\textbf{Question}

A cleaning company offers a special service to wash the windows of HDB flats. Each team can wash $\frac{3}{4}$ of a block in 1 day. 

(a) If the company has 12 blocks to wash, how many days will it take 1 team to finish washing all the blocks?

(b) On another day, the company assigns 3 teams to do the same job, and each team manages to wash $1\frac{1}{2}$ times as many blocks per day as they did before. How many blocks do all 3 teams wash together in 2 days?


\textbf{Solution}

Let us solve each part step-by-step:

(a) 
Each team can wash $\frac{3}{4}$ of a block in 1 day. Let $x$ be the number of days needed to wash 12 blocks.

So, $x \times \frac{3}{4} = 12$

To find $x$, we divide 12 by $\frac{3}{4}$:

\[
x = 12 \div \frac{3}{4} = 12 \times \frac{4}{3} = 16
\]

So, it takes 16 days for 1 team to wash all 12 blocks.

(b)
Each team can now wash $1\frac{1}{2}$ times as many blocks per day as before:

$\frac{3}{4} \times 1\frac{1}{2} = \frac{3}{4} \times \frac{3}{2} = \frac{9}{8}$ blocks per day per team.

Total number of blocks washed by 3 teams in 2 days is:

\[
\text{Total} = 3 \times 2 \times \frac{9}{8} = 6 \times \frac{9}{8} = \frac{54}{8} = 6\frac{6}{8} = 6\frac{3}{4}
\]

So, in 2 days, all 3 teams together can wash $6\frac{3}{4}$ blocks.

\section*{P6-FrDivPP\_P5-FrSubMix\_GPT4.1\_Leisure\_02}
\textbf{Metadata}

\begin{itemize}
  \item Primary KC: FRACTIONS | Division | dividing a whole number/proper fraction by a proper fraction
  \item Secondary KC: FRACTIONS | Subtraction | subtracting mixed numbers
  \item Topic: Leisure
  \item Grade: Primary 6
\end{itemize}

\textbf{Question}

\textbf{Sarah is making fruit tarts for a party. She has 6 cups of flour.} \\ 
\textbf{For each tart, she uses } \frac{3}{4} \textbf{ cup of flour.}} \\ 
\textbf{(a) How many tarts can Sarah make with 6 cups of flour?} \\ 
\textbf{Later, her friend Jason also bakes some tarts and uses } 2\frac{1}{2} \textbf{ cups of flour.} \\ 
\textbf{(b) After both have baked, how much flour is left if they started with 8 cups in total?}

\textbf{Solution}

\textbf{(a) Number of tarts:} \\ 
\text{To find the number of tarts, divide the total flour by the flour used for each tart:} \\ 
6 \div \frac{3}{4} = 6 \times \frac{4}{3} = \frac{24}{3} = 8 \\ 
\textbf{Sarah can make 8 tarts.} \\ 
\\ 
\textbf{(b) Flour left after both have baked:} \\ 
\text{Total flour used:} \\ 
\text{Sarah used:} 6 \text{ cups} \\ 
\text{Jason used:} 2\frac{1}{2} = \frac{5}{2} \text{ cups} \\ 
\text{Total used} = 6 + \frac{5}{2} = \frac{12}{2} + \frac{5}{2} = \frac{17}{2} \text{ cups} \\ 
\text{Flour left:} 8 - \frac{17}{2} = \frac{16}{2} - \frac{17}{2} = -\frac{1}{2} \\ 
\text{Since the answer is negative, this means that 1/2 cup more flour was needed.} \\ 
\textbf{Therefore, they are short of } \frac{1}{2} \textbf{ cup of flour.}

\section*{P6-PcFndWN\_P4-WNDiv4d1d\_GPT4.1\_Leisure\_02}
\textbf{Metadata}

\begin{itemize}
  \item Primary KC: PERCENTAGE | Finding the whole | finding the whole given a part and the percentage
  \item Secondary KC: WHOLE NUMBERS | Division | division up to 4 digits by 1 digit
  \item Topic: Leisure
  \item Grade: Primary 6
\end{itemize}

\textbf{Question}


A group of friends went to an amusement park. They bought tickets to ride a roller coaster. If 36 tickets represent 60\% of all the tickets available for the roller coaster, how many tickets were available in total? After all the tickets were sold, the park manager divided the total number of tickets equally among 4 lucky winners for a prize. How many tickets did each winner get?


\textbf{Solution}


Let the total number of tickets be $x$.

Given that 36 tickets represent 60\% of all the tickets:

\[
60\% \times x = 36
\]
\[
\frac{60}{100} \times x = 36
\]
\[
\frac{3}{5}x = 36
\]
\[
x = 36 \div \frac{3}{5}
\]
\[
x = 36 \times \frac{5}{3} = 12 \times 5 = \boxed{60}
\]

So, there were 60 tickets in total.

Now, the manager divided 60 tickets equally among 4 winners:

\[
60 \div 4 = \boxed{15}
\]

Each winner received 15 tickets.


\section*{P6-PcFndChg\_P3-WNSub4d\_GPT4.1\_Transporation\_02}
\textbf{Metadata}

\begin{itemize}
  \item Primary KC: PERCENTAGE | Finding change | finding percentage increase/decrease
  \item Secondary KC: WHOLE NUMBERS | Subtraction | subtraction up to 4 digits
  \item Topic: Transporation
  \item Grade: Primary 6
\end{itemize}

\textbf{Question}

\textbf{Question:} \\ 
A bus company had 3,250 passengers in January. In February, the number of passengers increased to 3,900 because more people took the bus to work. \\ 
(a) By how many passengers did the number increase? \\ 
(b) What is the percentage increase in the number of passengers from January to February? \\ 
Give your answers correct to 1 decimal place.

\textbf{Solution}

\textbf{Solution:} \\ 
(a) Number of passengers increased = 3,900 - 3,250 = 650. \\ 
(b) Percentage increase = \frac{\text{Increase}}{\text{Original number}} \times 100\% = \frac{650}{3,250} \times 100\% = 20\%. \\ 
So, the percentage increase is 20.0\%.

\section*{P6-PcFndWN\_P4-WNMul4d1d\_GPT4.1\_Services\_02}
\textbf{Metadata}

\begin{itemize}
  \item Primary KC: PERCENTAGE | Finding the whole | finding the whole given a part and the percentage
  \item Secondary KC: WHOLE NUMBERS | Multiplication | multiplication up to 4 digits by 1 digit or up to 3 digits by 2 digits
  \item Topic: Services
  \item Grade: Primary 6
\end{itemize}

\textbf{Question}

A taxi company charges a booking fee and a fare based on distance. Samantha paid $\$18.40$ for a trip. This amount was $80\%$ of her total monthly transport expenses. 

(a) How much did Samantha spend in total on transport that month?

(b) If Samantha took 12 such trips that month, and the fare for each trip (excluding the booking fee) was $\$15$, what is the total amount Samantha spent on booking fees alone for these 12 trips?

\textbf{Solution}

Given $\$18.40$ is $80\%$ of Samantha's total transport expenses for the month.

(a) Let the total amount spent be $x$. 

$80\%$ of $x = 18.40$

$0.8x = 18.40$

$x = \frac{18.40}{0.8}$

$x = 23$ 

So, Samantha spent $\$23$ in total on transport that month.

(b) Samantha took 12 trips. 

Total spent for 12 trips: $\$18.40 \times 12 = \$220.80$

Total spent on fares (excluding booking fees): $\$15 \times 12 = \$180$

Total spent on booking fees: $220.80 - 180 = \$40.80$

Final answers:
(a) $\boxed{23}$

(b) $\boxed{40.80}$

\section*{P6-PcFndChg\_P3-WNSub4d\_GPT4.1\_Food\_02}
\textbf{Metadata}

\begin{itemize}
  \item Primary KC: PERCENTAGE | Finding change | finding percentage increase/decrease
  \item Secondary KC: WHOLE NUMBERS | Subtraction | subtraction up to 4 digits
  \item Topic: Food
  \item Grade: Primary 6
\end{itemize}

\textbf{Question}

A packet of rice used to cost \$480 for 20 kg at a supermarket. After a price revision, the same packet now costs \$432 for 20 kg.\

(a) By how much did the price decrease?\

(b) What is the percentage decrease in the price of the rice packet?\


\textbf{Solution}

(a) Subtract the new price from the old price: \\ 
Old price = \$480 \\ 
New price = \$432 \\ 
Price decrease = 480 - 432 = \$48 \\ 
\\
(b) To find the percentage decrease: \\ 
Percentage decrease = \frac{\text{Decrease}}{\text{Original price}} \times 100\% \\ 
= \frac{48}{480} \times 100\% \\ 
= \frac{1}{10} \times 100\% = 10\% \\ 
\\
So, the price decreased by \$48, which is a 10\% decrease.

\section*{P6-FrDivPN\_P5-FrMulMixN\_GPT4.1\_Manufacturing\_02}
\textbf{Metadata}

\begin{itemize}
  \item Primary KC: FRACTIONS | Division | dividing a proper fraction by a whole number
  \item Secondary KC: FRACTIONS | Multiplication | multiplying a mixed number and a whole number
  \item Topic: Manufacturing
  \item Grade: Primary 6
\end{itemize}

\textbf{Question}

A factory produces $4\dfrac{1}{2}$ metres of cloth for each uniform. The factory uses $3$ metres of cloth to make a sash for each uniform. If the total cloth for making $5$ uniforms is shared equally among $2$ tailors, how many metres of cloth does each tailor receive?

\textbf{Solution}

First, find the total amount of cloth needed for $5$ uniforms:

\[
\text{Cloth for 1 uniform} = 4\dfrac{1}{2} = \dfrac{9}{2}\,\text{m}
\]
\[
\text{For 5 uniforms:}
\quad 5 \times \dfrac{9}{2} = \dfrac{45}{2}\,
\]

Next, account for the sash for each uniform:

\[
\text{Sash for 1 uniform} = 3\,\text{m}
\]
\[
\text{For 5 uniforms:}\ 5 \times 3 = 15\,\text{m}
\]

Total cloth needed:

\[
\dfrac{45}{2} + 15 = \dfrac{45}{2} + \dfrac{30}{2} = \dfrac{75}{2}\,\text{m}
\]

Now, divide the total equally among $2$ tailors:

\[
\dfrac{75}{2} \div 2 = \dfrac{75}{2} \times \dfrac{1}{2} = \dfrac{75}{4} = 18\dfrac{3}{4}\,\text{m}
\]

\textbf{Answer:} Each tailor receives $18\dfrac{3}{4}$ metres of cloth.

\section*{P6-FrDivPN\_P5-FrAddMix\_GPT4.1\_Sports\_02}
\textbf{Metadata}

\begin{itemize}
  \item Primary KC: FRACTIONS | Division | dividing a proper fraction by a whole number
  \item Secondary KC: FRACTIONS | Addition | adding mixed numbers
  \item Topic: Sports
  \item Grade: Primary 6
\end{itemize}

\textbf{Question}

\textbf{Question:}\\

During a sports day, Mary has $\frac{3}{4}$ litres of isotonic drink, which she wants to share equally among 5 athletes, including herself. After dividing the drink equally, each athlete receives the same amount. Later, each athlete drinks their share and then refills their bottles with $1 \frac{1}{2}$ litres of water each.\\

(a) How many litres of isotonic drink does each athlete receive?\\
(b) What is the total volume of drink (each athlete's isotonic drink share combined with their water refill) that each athlete now has?\\

\textbf{Solution}

\textbf{Solution:}\\

(a) To find out how much isotonic drink each athlete receives, divide $\frac{3}{4}$ litres by 5.
\[
\frac{3}{4} \div 5 = \frac{3}{4} \times \frac{1}{5} = \frac{3}{20}
\]
So, each athlete receives $\frac{3}{20}$ litres of isotonic drink.

(b) Each athlete then refills with $1 \frac{1}{2}$ litres of water.
Convert $1 \frac{1}{2}$ to an improper fraction:
\[
1 \frac{1}{2} = \frac{3}{2}
\]
Now, add the isotonic drink share and water refill:
\[
\frac{3}{20} + \frac{3}{2}
\]
Find a common denominator (20):
\[
\frac{3}{2} = \frac{30}{20}
\]
So,
\[
\frac{3}{20} + \frac{30}{20} = \frac{33}{20}
\]
Express $\frac{33}{20}$ as a mixed number:
\[
\frac{33}{20} = 1 \frac{13}{20}
\]
\textbf{Therefore, each athlete now has $1 \frac{13}{20}$ litres of drink in total.}


\section*{P6-FrDivPP\_P5-FrAddMix\_GPT4.1\_Digital ecconomy\_02}
\textbf{Metadata}

\begin{itemize}
  \item Primary KC: FRACTIONS | Division | dividing a whole number/proper fraction by a proper fraction
  \item Secondary KC: FRACTIONS | Addition | adding mixed numbers
  \item Topic: Digital ecconomy
  \item Grade: Primary 6
\end{itemize}

\textbf{Question}

In a digital economy workshop, Sarah earned 3 whole digital tokens. She decides to share these equally among a group of friends. If each friend receives $\dfrac{3}{4}$ of a token, how many friends are there? 

Later, Sarah and her friend John combine their token earnings for a project. John has $2\dfrac{1}{2}$ tokens. How many tokens do Sarah and John have together?


\textbf{Solution}

First, find out how many friends Sarah can share her tokens with:

Sarah's tokens = 3
Each friend receives = $\dfrac{3}{4}$ token

Number of friends $= \dfrac{3}{\dfrac{3}{4}} = 3 \times \dfrac{4}{3} = 4$

So, Sarah can share her tokens equally among **4 friends**.

For the second part,
Sarah's tokens = 3
John's tokens = $2\dfrac{1}{2} = 2 + \dfrac{1}{2} = \dfrac{5}{2}$

Total tokens together $= 3 + 2\dfrac{1}{2} = 3 + 2.5 = 5.5$
Or as a mixed number: $5\dfrac{1}{2}$

So, together they have **$5\dfrac{1}{2}$ tokens**.

\section*{P6-FrDivPN\_P5-FrMulMixN\_GPT4.1\_Farming\_02}
\textbf{Metadata}

\begin{itemize}
  \item Primary KC: FRACTIONS | Division | dividing a proper fraction by a whole number
  \item Secondary KC: FRACTIONS | Multiplication | multiplying a mixed number and a whole number
  \item Topic: Farming
  \item Grade: Primary 6
\end{itemize}

\textbf{Question}

A farmer harvested $4\frac{1}{2}$ baskets of apples. She wants to pack the apples equally into 6 boxes. 

(a) How many baskets of apples will each box contain? 

(b) If each box is then shared equally between 3 friends, how many baskets of apples does each friend get?

\textbf{Solution}

(a) To find out how many baskets of apples will be packed into each box, we divide $4\frac{1}{2}$ by $6$.

First, convert the mixed number $4\frac{1}{2}$ to an improper fraction:
\[
4\frac{1}{2} = \frac{9}{2}
\]

Now divide by $6$:
\[
\frac{9}{2} \div 6 = \frac{9}{2} \times \frac{1}{6} = \frac{9}{12} = \frac{3}{4}
\]

So, each box contains \( \frac{3}{4} \) baskets of apples.

(b) Now, each box (\( \frac{3}{4} \) baskets) is shared among 3 friends. We divide \( \frac{3}{4} \) by 3:
\[
\frac{3}{4} \div 3 = \frac{3}{4} \times \frac{1}{3} = \frac{3}{12} = \frac{1}{4}
\]

Each friend receives \( \frac{1}{4} \) baskets of apples.

\end{document}
