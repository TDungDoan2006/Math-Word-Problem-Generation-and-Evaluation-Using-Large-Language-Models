\documentclass{article}
\usepackage[utf8]{inputenc}
\usepackage{amsmath}
\usepackage{amsfonts}
\usepackage{amssymb}
\usepackage{graphicx}
\usepackage{hyperref}
\title{20 O1 samples generated by GPT-4.1 updated v5}
\author{Tien Dung Doan}
\begin{document}
\maketitle
\section*{Question 1}
\textbf{Metadata}

\begin{itemize}
  \item Question ID: O1-RoRepDc\_O1-RoSmpDc\_GPT4.1\_Farming\_01
  \item Primary KC: RATIO | Representation and concept | ratios involving decimals
  \item Secondary KC: RATIO | Simplifying | converting a ratio involving decimals to its simplest form
  \item Topic: Farming – Scenarios involving agriculture, livestock, crop production, or farm work
  \item Grade: Secondary O-level 1
\end{itemize}

\textbf{Question}

A vegetable farm in Singapore harvested 36.5 kg of carrots and 14.6 kg of radishes one morning.

(a) What is the ratio of the mass of carrots to the mass of radishes, in decimal form?

(b) Express this ratio in its simplest form.

\section*{Question 2}
\textbf{Metadata}

\begin{itemize}
  \item Question ID: O1-AgRepExSq\_O1-AgEvlEx\_GPT4.1\_Services\_01
  \item Primary KC: ALGEBRA | Representation and concept | translation of simple real-world situations into quadratic algebraic expressions
  \item Secondary KC: ALGEBRA | Evaluation | evaluation of algebraic expressions and formulae
  \item Topic: Services – Situations involving service jobs or industries like healthcare, retail, hospitality, or repair
  \item Grade: Secondary O-level 1
\end{itemize}

\textbf{Question}

A hotel charges $x$ dollars per night for a standard room. During the school holidays, they offer a special deal: if a guest books $n$ consecutive nights, the total cost includes a one-time discount of $10n^2$ dollars to attract families. 

(a) Write an algebraic expression in terms of $x$ and $n$ to represent the total cost for booking $n$ consecutive nights during the holiday promotion.

(b) If the standard rate is $80$ dollars per night and a family stays for $5$ nights, evaluate the total cost using your expression from part (a).

\section*{Question 3}
\textbf{Metadata}

\begin{itemize}
  \item Question ID: O1-RoRepFr\_P5-FrMulMixN\_GPT4.1\_Household Finance\_01
  \item Primary KC: RATIO | Representation and concept | ratios involving fractions
  \item Secondary KC: FRACTIONS | Multiplication | multiplying a mixed number and a whole number
  \item Topic: Household Finance – Contexts related to saving, spending, budgeting, bills, or personal/family expenses
  \item Grade: Secondary O-level 1
\end{itemize}

\textbf{Question}

A family spends $\frac{3}{4}$ of their monthly food budget on groceries and the rest on eating out. Last month, the ratio of the money spent on groceries to the money spent on eating out was $\frac{3}{4}:1$. If the family’s total food budget for last month was $2\frac{1}{2}$ times larger than $\$200$, how much money did the family spend on eating out?

\section*{Question 4}
\textbf{Metadata}

\begin{itemize}
  \item Question ID: O1-FDMul\_O1-FDSub\_GPT4.1\_Manufacturing\_01
  \item Primary KC: FRACTIONS AND DECIMALS | Multiplication | Multiplication involving one fraction and one decimal number
  \item Secondary KC: FRACTIONS AND DECIMALS | Subtraction | Subtraction involving one fraction and one decimal number
  \item Topic: Manufacturing – Contexts related to production, factories, machinery, or goods creation
  \item Grade: Secondary O-level 1
\end{itemize}

\textbf{Question}

A factory produces 0.75 metres of fabric for each shirt. In one day, they use $\frac{2}{3}$ of the fabric produced for making a new type of jacket. 

(a) How many metres of fabric are used for making jackets in one day?

(b) After making the jackets, the factory has 0.4 metres of fabric left unused. How much fabric did the factory have at the start of the day?

\section*{Question 5}
\textbf{Metadata}

\begin{itemize}
  \item Question ID: O1-AgSlvFrLr\_O1-AgRepEq\_GPT4.1\_Services\_01
  \item Primary KC: ALGEBRA | Solving | solving simple fractional equations that can be reduced to linear equations
  \item Secondary KC: ALGEBRA | Representation and concept | concept of equation
  \item Topic: Services – Situations involving service jobs or industries like healthcare, retail, hospitality, or repair
  \item Grade: Secondary O-level 1
\end{itemize}

\textbf{Question}

A mobile phone repair shop charges a fixed service fee and a fraction of the total repair cost as a handling charge. For a particular repair, the total cost (in dollars) is given by the equation:

\[
\frac{1}{3}x + 18 = 50
\]

where $x$ is the cost of the parts used in the repair and $18$ is the fixed service fee. What is the cost of the parts used in this repair?

\section*{Question 6}
\textbf{Metadata}

\begin{itemize}
  \item Question ID: O1-PcFndRslt\_P3-WNAdd4d\_GPT4.1\_Food\_01
  \item Primary KC: PERCENTAGE | Finding result after change | increasing/decreasing a quantity by a given percentage (including concept of percentage point)
  \item Secondary KC: WHOLE NUMBERS | Addition | addition up to 4 digits
  \item Topic: Food – Situations involving meals, groceries, ingredients, or food preparation and consumption
  \item Grade: Secondary O-level 1
\end{itemize}

\textbf{Question}

A bakery sold 2,150 loaves of bread in January. In February, the bakery increased its sales by 12\%. 

(a) How many loaves of bread did the bakery sell in February?

In March, the bakery sold 1,365 more loaves of bread than in February.

(b) How many loaves of bread did the bakery sell altogether in January, February, and March?

Give your answers in whole numbers.

\section*{Question 7}
\textbf{Metadata}

\begin{itemize}
  \item Question ID: O1-RoRepFr\_P6-FrDivPP\_GPT4.1\_Sports\_01
  \item Primary KC: RATIO | Representation and concept | ratios involving fractions
  \item Secondary KC: FRACTIONS | Division | dividing a whole number/proper fraction by a proper fraction
  \item Topic: Sports – Activities involving games, training, competitions, or physical performance
  \item Grade: Secondary O-level 1
\end{itemize}

\textbf{Question}

During a basketball training session, Coach Lim prepares a sports drink by mixing two types of fruit juice. The ratio of apple juice to orange juice used is $\dfrac{3}{4} : 1$. If he has 6 litres of orange juice and uses all of it,

(a) How many litres of apple juice does Coach Lim need to use?

(b) If the total amount of sports drink is then divided equally among bottles, and each bottle holds $\dfrac{3}{5}$ litres, how many bottles can be filled completely?

\section*{Question 8}
\textbf{Metadata}

\begin{itemize}
  \item Question ID: O1-RoRepFr\_P6-FrDivPP\_GPT4.1\_Education\_01
  \item Primary KC: RATIO | Representation and concept | ratios involving fractions
  \item Secondary KC: FRACTIONS | Division | dividing a whole number/proper fraction by a proper fraction
  \item Topic: Education – Scenarios related to schools, learning, teachers, students, exams, or classroom settings
  \item Grade: Secondary O-level 1
\end{itemize}

\textbf{Question}

A class is preparing materials for an art project. The teacher distributes glitter into containers such that the ratio of blue glitter to silver glitter is $\frac{3}{4} : \frac{5}{6}$. If there are $24$ grams of blue glitter given out in total, 

(a) What is the total mass of silver glitter distributed?

(b) If each container of silver glitter must have $\frac{5}{12}$ grams, how many containers of silver glitter did the class prepare?

\section*{Question 9}
\textbf{Metadata}

\begin{itemize}
  \item Question ID: O1-FDMul\_O1-FDAdd\_GPT4.1\_Farming\_01
  \item Primary KC: FRACTIONS AND DECIMALS | Multiplication | Multiplication involving one fraction and one decimal number
  \item Secondary KC: FRACTIONS AND DECIMALS | Addition | Addition involving one fraction and one decimal number
  \item Topic: Farming – Scenarios involving agriculture, livestock, crop production, or farm work
  \item Grade: Secondary O-level 1
\end{itemize}

\textbf{Question}

On a farm, a farmer uses $\frac{3}{4}$ of a field to plant corn. Each square metre of the field produces $2.5$ kg of corn. What is the total mass of corn produced from the corn-planted area if the field is $1200$ square metres in size? 

After harvesting, $140.5$ kg of corn from another section is added to the harvested corn. What is the total mass of corn the farmer has now?

\section*{Question 10}
\textbf{Metadata}

\begin{itemize}
  \item Question ID: O1-AgRepExSq\_O1-AgEvlEx\_GPT4.1\_Transportation\_01
  \item Primary KC: ALGEBRA | Representation and concept | translation of simple real-world situations into quadratic algebraic expressions
  \item Secondary KC: ALGEBRA | Evaluation | evaluation of algebraic expressions and formulae
  \item Topic: Transportation – Problems involving travel, vehicles, routes, distances, or commuting
  \item Grade: Secondary O-level 1
\end{itemize}

\textbf{Question}

A car travels from Jurong to Tampines. The distance between the two locations is 60 km. Let the speed of the car be $x$ km/h. On the way back, due to traffic, the speed decreases by 10 km/h. 

(a) Write an algebraic expression, in terms of $x$, for the total time (in hours) taken for the round trip.

(b) If $x = 40$, evaluate the total time taken for the round trip. 

Provide your answers correct to 2 decimal places where necessary.

\section*{Question 11}
\textbf{Metadata}

\begin{itemize}
  \item Question ID: O1-PcFndRslt\_P3-WNAdd4d\_GPT4.1\_Digital Economy\_01
  \item Primary KC: PERCENTAGE | Finding result after change | increasing/decreasing a quantity by a given percentage (including concept of percentage point)
  \item Secondary KC: WHOLE NUMBERS | Addition | addition up to 4 digits
  \item Topic: Digital Economy – Scenarios involving online platforms, digital products, virtual services, or e-commerce
  \item Grade: Secondary O-level 1
\end{itemize}

\textbf{Question}

An online store had 1,250 customers purchasing its digital subscription last month. This month, the number of customers increased by 12\%. After the increase, the store also sold an additional 260 single-use digital passes. 

What is the total number of digital products (subscriptions and single-use passes) that the store sold this month?

\section*{Question 12}
\textbf{Metadata}

\begin{itemize}
  \item Question ID: O1-PcFndRslt\_P3-WNSub4d\_GPT4.1\_Sports\_01
  \item Primary KC: PERCENTAGE | Finding result after change | increasing/decreasing a quantity by a given percentage (including concept of percentage point)
  \item Secondary KC: WHOLE NUMBERS | Subtraction | subtraction up to 4 digits
  \item Topic: Sports – Activities involving games, training, competitions, or physical performance
  \item Grade: Secondary O-level 1
\end{itemize}

\textbf{Question}

A school's basketball team scored 1,200 points in the last season. This year, their coach set a target to increase their total score by 15\%. However, due to some tough matches, the team only managed to improve their score by 12\% from last year's total. 

(a) How many points did the team actually score this year?

(b) By how many points did the team's actual score fall short of the coach's target?

(Express your answers as whole numbers.)

\section*{Question 13}
\textbf{Metadata}

\begin{itemize}
  \item Question ID: O1-RoRepFr\_P5-FrAddMix\_GPT4.1\_Farming\_01
  \item Primary KC: RATIO | Representation and concept | ratios involving fractions
  \item Secondary KC: FRACTIONS | Addition | adding mixed numbers
  \item Topic: Farming – Scenarios involving agriculture, livestock, crop production, or farm work
  \item Grade: Secondary O-level 1
\end{itemize}

\textbf{Question}

On a farm, the ratio of the amount of wheat harvested to the amount of corn harvested is $\dfrac{3}{4} : 2$. If the farmer harvested $3\dfrac{1}{2}$ tonnes of wheat, how much corn did the farmer harvest? Later, the farmer harvested an additional $1\dfrac{1}{4}$ tonnes of corn. Find the new total amount of corn harvested, giving your answer as a mixed number.

\section*{Question 14}
\textbf{Metadata}

\begin{itemize}
  \item Question ID: O1-PcRepRvs\_O1-PcCnv2Dc\_GPT4.1\_Farming\_01
  \item Primary KC: PERCENTAGE | Representation and concept | reverse percentages
  \item Secondary KC: PERCENTAGE | Conversion to decimals | expressing percentage as a decimal
  \item Topic: Farming – Scenarios involving agriculture, livestock, crop production, or farm work
  \item Grade: Secondary O-level 1
\end{itemize}

\textbf{Question}

A farmer harvested a field of corn and later found that, due to a counting error, he had recorded 20\% less corn than he actually harvested. The incorrect record shows that he harvested 1,200 kilograms of corn. Express 20\% as a decimal, and calculate the actual amount of corn, in kilograms, that he really harvested.

\section*{Question 15}
\textbf{Metadata}

\begin{itemize}
  \item Question ID: O1-PcFndRslt\_P4-WNMul4d1d\_GPT4.1\_Digital Economy\_01
  \item Primary KC: PERCENTAGE | Finding result after change | increasing/decreasing a quantity by a given percentage (including concept of percentage point)
  \item Secondary KC: WHOLE NUMBERS | Multiplication | multiplication up to 4 digits by 1 digit or up to 3 digits by 2 digits
  \item Topic: Digital Economy – Scenarios involving online platforms, digital products, virtual services, or e-commerce
  \item Grade: Secondary O-level 1
\end{itemize}

\textbf{Question}

A local business sells digital art prints online. In January, it sold 135 digital prints each at a price of \$12. In February, the company decided to increase the price of each digital print by 15% to keep up with rising costs. 

(a) What is the new price of each digital print in February, after the 15% increase?

(b) If the business sells the same number of prints (135) in February at the new price, what is the total revenue for February?

(c) By how much did the total revenue increase from January to February?

\section*{Question 16}
\textbf{Metadata}

\begin{itemize}
  \item Question ID: O1-RoRepDc\_P4-DcAdd2d\_GPT4.1\_Transportation\_01
  \item Primary KC: RATIO | Representation and concept | ratios involving decimals
  \item Secondary KC: DECIMALS | Addition | adding decimals (up to 2 decimal places)
  \item Topic: Transportation – Problems involving travel, vehicles, routes, distances, or commuting
  \item Grade: Secondary O-level 1
\end{itemize}

\textbf{Question}

Amir and Siti decided to take a taxi to the airport. The taxi fare is calculated based on the distance travelled. Amir’s journey to the airport is 8.40 km, while Siti's journey is 12.60 km. 

(a) The ratio of the distance travelled by Amir to the distance travelled by Siti can be written in the form of decimals. What is the ratio of Amir’s distance to Siti’s distance, in its simplest form using decimals?

(b) If the taxi charges $0.65 per kilometre, what is the total fare (in dollars) for both Amir and Siti, if they travel separately to the airport? (Give your answer correct to 2 decimal places.)

\section*{Question 17}
\textbf{Metadata}

\begin{itemize}
  \item Question ID: O1-PcFndRslt\_P4-WNDiv4d1d\_GPT4.1\_Food\_01
  \item Primary KC: PERCENTAGE | Finding result after change | increasing/decreasing a quantity by a given percentage (including concept of percentage point)
  \item Secondary KC: WHOLE NUMBERS | Division | division up to 4 digits by 1 digit
  \item Topic: Food – Situations involving meals, groceries, ingredients, or food preparation and consumption
  \item Grade: Secondary O-level 1
\end{itemize}

\textbf{Question}

A restaurant had 1,824 bowls of noodles in stock at the beginning of the month. Due to an ongoing food festival, the manager decided to increase the number of noodle bowls by 25\%. After the increase, the restaurant wants to divide the total number of bowls equally among 8 outlets. 

(a) What is the new number of noodle bowls after the 25\% increase?

(b) How many bowls will each outlet receive?


\section*{Question 18}
\textbf{Metadata}

\begin{itemize}
  \item Question ID: O1-PcFndRslt\_P4-WNDiv4d1d\_GPT4.1\_Sports\_01
  \item Primary KC: PERCENTAGE | Finding result after change | increasing/decreasing a quantity by a given percentage (including concept of percentage point)
  \item Secondary KC: WHOLE NUMBERS | Division | division up to 4 digits by 1 digit
  \item Topic: Sports – Activities involving games, training, competitions, or physical performance
  \item Grade: Secondary O-level 1
\end{itemize}

\textbf{Question}

A basketball coach had 1248 basketballs for training last year. This year, the coach increased the number of basketballs by 15\%. After the increase, the coach wants to divide the new total number of basketballs equally among 8 teams for their training sessions.

(a) How many basketballs does the coach have this year after the increase?

(b) How many basketballs will each team get, and how many basketballs will be left undistributed, if any?

\section*{Question 19}
\textbf{Metadata}

\begin{itemize}
  \item Question ID: O1-AgRepExSq\_O1-AgEvlEx\_GPT4.1\_Food\_01
  \item Primary KC: ALGEBRA | Representation and concept | translation of simple real-world situations into quadratic algebraic expressions
  \item Secondary KC: ALGEBRA | Evaluation | evaluation of algebraic expressions and formulae
  \item Topic: Food – Situations involving meals, groceries, ingredients, or food preparation and consumption
  \item Grade: Secondary O-level 1
\end{itemize}

\textbf{Question}

James is preparing a fruit punch for a school event. He knows that the amount of punch, $V$ (in litres), he can make depends on the number of fruit juice cartons $n$ he uses, according to the expression $V = n^2 + 4n$. 

(a) Write a quadratic algebraic expression to represent the amount of punch James can make if he uses $n$ cartons.

(b) If James wants to prepare punch using 3 cartons of fruit juice, evaluate the amount of punch he can make.

\section*{Question 20}
\textbf{Metadata}

\begin{itemize}
  \item Question ID: O1-FDDiv\_O1-FDAdd\_GPT4.1\_Digital Economy\_01
  \item Primary KC: FRACTIONS AND DECIMALS | Division | Division involving one fraction and one decimal number
  \item Secondary KC: FRACTIONS AND DECIMALS | Addition | Addition involving one fraction and one decimal number
  \item Topic: Digital Economy – Scenarios involving online platforms, digital products, virtual services, or e-commerce
  \item Grade: Secondary O-level 1
\end{itemize}

\textbf{Question}

Sarah is shopping for e-books on an online platform. She purchases one e-book for $\frac{3}{4}$ of the regular price, which is $6.8. She also adds another e-book costing $2.5$ to her cart.\
\
(a) What is the regular price of the first e-book?\
\
(b) What is the total cost (in dollars) of both e-books that Sarah has to pay?\
\
Give your answer in decimals up to 2 decimal places where necessary.

\end{document}
