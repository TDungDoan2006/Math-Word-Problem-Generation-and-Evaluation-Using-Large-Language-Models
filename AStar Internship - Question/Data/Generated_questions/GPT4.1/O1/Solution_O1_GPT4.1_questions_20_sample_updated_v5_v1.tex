\documentclass{article}
\usepackage[utf8]{inputenc}
\usepackage{amsmath}
\usepackage{amsfonts}
\usepackage{amssymb}
\usepackage{graphicx}
\usepackage{hyperref}
\title{20 O1 samples generated by GPT-4.1 updated v5}
\author{Tien Dung Doan}
\begin{document}
\maketitle
\section*{Question 1}
\textbf{Metadata}

\begin{itemize}
  \item Question ID: O1-RoRepDc\_O1-RoSmpDc\_GPT4.1\_Farming\_01
  \item Primary KC: RATIO | Representation and concept | ratios involving decimals
  \item Secondary KC: RATIO | Simplifying | converting a ratio involving decimals to its simplest form
  \item Topic: Farming – Scenarios involving agriculture, livestock, crop production, or farm work
  \item Grade: Secondary O-level 1
\end{itemize}

\textbf{Solution}

Let the mass of carrots be 36.5 kg and the mass of radishes be 14.6 kg.

(a) The ratio of carrots to radishes in decimal form is:

$$
\text{Ratio} = 36.5 : 14.6
$$

(b) To simplify, divide both numbers by 14.6:

$$
36.5 \div 14.6 = 2.5 \\
14.6 \div 14.6 = 1
$$

So the simplest ratio is:

$$
2.5 : 1
$$

**Final Answer:**
(a) 36.5 : 14.6

(b) 2.5 : 1

\section*{Question 2}
\textbf{Metadata}

\begin{itemize}
  \item Question ID: O1-AgRepExSq\_O1-AgEvlEx\_GPT4.1\_Services\_01
  \item Primary KC: ALGEBRA | Representation and concept | translation of simple real-world situations into quadratic algebraic expressions
  \item Secondary KC: ALGEBRA | Evaluation | evaluation of algebraic expressions and formulae
  \item Topic: Services – Situations involving service jobs or industries like healthcare, retail, hospitality, or repair
  \item Grade: Secondary O-level 1
\end{itemize}

\textbf{Solution}

(a) The total cost before the discount is $x \times n$ dollars. The special discount is $10n^2$ dollars. So, the total cost after the discount is:

\[ \text{Total Cost} = x \times n - 10n^2 \]

(b) Substitute $x = 80$ and $n = 5$ into the expression:

\[
\begin{align*}
\text{Total Cost} &= 80 \times 5 - 10 \times 5^2 \\
&= 400 - 10 \times 25 \\
&= 400 - 250 \\
&= 150
\end{align*}
\]

So, the total cost for the family is \$150.

\section*{Question 3}
\textbf{Metadata}

\begin{itemize}
  \item Question ID: O1-RoRepFr\_P5-FrMulMixN\_GPT4.1\_Household Finance\_01
  \item Primary KC: RATIO | Representation and concept | ratios involving fractions
  \item Secondary KC: FRACTIONS | Multiplication | multiplying a mixed number and a whole number
  \item Topic: Household Finance – Contexts related to saving, spending, budgeting, bills, or personal/family expenses
  \item Grade: Secondary O-level 1
\end{itemize}

\textbf{Solution}

First, let's calculate the total food budget for last month:

$2\frac{1}{2} = \frac{5}{2}$

Total food budget $= \frac{5}{2} \times 200 = 5 \times 100 = \$500$

Let the amount spent on groceries be $G$ and the amount spent on eating out be $E$.

It is given that the ratio is $\frac{3}{4}:1$, so:

$\frac{G}{E} = \frac{3}{4}:1 = \frac{3}{4} \div 1 = \frac{3}{4}$

So,

$G = \frac{3}{4}E$

But:

$G + E = 500$

Substitute $G$:

$\frac{3}{4}E + E = 500$

$\left(\frac{3}{4} + 1\right)E = 500$

$\left(\frac{3}{4} + \frac{4}{4}\right)E = 500$

$\frac{7}{4}E = 500$

Multiply both sides by 4:

$7E = 2000$

$E = \frac{2000}{7} \approx \$285.71$

Therefore, the family spent roughly $\$285.71$ on eating out last month.

\section*{Question 4}
\textbf{Metadata}

\begin{itemize}
  \item Question ID: O1-FDMul\_O1-FDSub\_GPT4.1\_Manufacturing\_01
  \item Primary KC: FRACTIONS AND DECIMALS | Multiplication | Multiplication involving one fraction and one decimal number
  \item Secondary KC: FRACTIONS AND DECIMALS | Subtraction | Subtraction involving one fraction and one decimal number
  \item Topic: Manufacturing – Contexts related to production, factories, machinery, or goods creation
  \item Grade: Secondary O-level 1
\end{itemize}

\textbf{Solution}

(a) Amount of fabric used for jackets = $0.75 \times \frac{2}{3}$

First, convert $0.75$ to a fraction: $0.75 = \frac{3}{4}$.

So, $\frac{3}{4} \times \frac{2}{3} = \frac{6}{12} = 0.5$ metres.

(b) Let $F$ be the total fabric at the start of the day.

After making jackets and leaving 0.4 metres unused:

$F - 0.5 = 0.4$

$\Rightarrow F = 0.4 + 0.5 = 0.9$ metres.

**Answers:**
(a) $0.5$ metres of fabric are used for making jackets in one day.

(b) The factory had $0.9$ metres of fabric at the start of the day.

\section*{Question 5}
\textbf{Metadata}

\begin{itemize}
  \item Question ID: O1-AgSlvFrLr\_O1-AgRepEq\_GPT4.1\_Services\_01
  \item Primary KC: ALGEBRA | Solving | solving simple fractional equations that can be reduced to linear equations
  \item Secondary KC: ALGEBRA | Representation and concept | concept of equation
  \item Topic: Services – Situations involving service jobs or industries like healthcare, retail, hospitality, or repair
  \item Grade: Secondary O-level 1
\end{itemize}

\textbf{Solution}

To find the cost of the parts $x$, we start with the equation:
\[
\frac{1}{3}x + 18 = 50
\]
Subtract $18$ from both sides:
\[
\frac{1}{3}x = 50 - 18
\]
\[
\frac{1}{3}x = 32
\]
Multiply both sides by $3$ to solve for $x$:
\[
x = 32 \times 3
\]
\[
x = 96
\]

**Therefore, the cost of the parts used in this repair is $\boxed{96}$ dollars.**

\section*{Question 6}
\textbf{Metadata}

\begin{itemize}
  \item Question ID: O1-PcFndRslt\_P3-WNAdd4d\_GPT4.1\_Food\_01
  \item Primary KC: PERCENTAGE | Finding result after change | increasing/decreasing a quantity by a given percentage (including concept of percentage point)
  \item Secondary KC: WHOLE NUMBERS | Addition | addition up to 4 digits
  \item Topic: Food – Situations involving meals, groceries, ingredients, or food preparation and consumption
  \item Grade: Secondary O-level 1
\end{itemize}

\textbf{Solution}

Let us solve each part step by step.

\textbf{(a) Number of loaves sold in February}

The bakery increased its January sales (2,150 loaves) by 12\%.

Increase = 12\% of 2,150 = \frac{12}{100} \times 2,150 = 0.12 \times 2,150 = 258.

Number of loaves sold in February = 2,150 + 258 = 2,408.

\textbf{(b) Total number of loaves sold in January, February, and March}

The number of loaves sold in March = Number sold in February + 1,365 = 2,408 + 1,365 = 3,773

Total number of loaves sold in all three months = January + February + March

= 2,150 + 2,408 + 3,773

= 2,150 + 2,408 = 4,558

= 4,558 + 3,773 = 8,331

\boxed{2,408} loaves were sold in February. \quad \boxed{8,331} loaves were sold altogether in January, February, and March.

\section*{Question 7}
\textbf{Metadata}

\begin{itemize}
  \item Question ID: O1-RoRepFr\_P6-FrDivPP\_GPT4.1\_Sports\_01
  \item Primary KC: RATIO | Representation and concept | ratios involving fractions
  \item Secondary KC: FRACTIONS | Division | dividing a whole number/proper fraction by a proper fraction
  \item Topic: Sports – Activities involving games, training, competitions, or physical performance
  \item Grade: Secondary O-level 1
\end{itemize}

\textbf{Solution}

(a) Let the quantity of apple juice be $x$ litres. According to the ratio $\dfrac{3}{4} : 1$, for every $1$ part of orange juice, there are $\dfrac{3}{4}$ parts of apple juice.

Given: Orange juice = 6 litres

So,
$\dfrac{x}{6} = \dfrac{3}{4} \implies x = 6 \times \dfrac{3}{4} = \dfrac{18}{4} = 4.5$ litres

Coach Lim needs to use $4.5$ litres of apple juice.

(b) Total sports drink = $6$ litres (orange juice) + $4.5$ litres (apple juice) = $10.5$ litres

Each bottle holds $\dfrac{3}{5}$ litres.

Number of bottles $= \dfrac{10.5}{\dfrac{3}{5}} = 10.5 \times \dfrac{5}{3} = \dfrac{52.5}{3} = 17.5$

Coach Lim can fill $\boxed{17}$ bottles completely.


\section*{Question 8}
\textbf{Metadata}

\begin{itemize}
  \item Question ID: O1-RoRepFr\_P6-FrDivPP\_GPT4.1\_Education\_01
  \item Primary KC: RATIO | Representation and concept | ratios involving fractions
  \item Secondary KC: FRACTIONS | Division | dividing a whole number/proper fraction by a proper fraction
  \item Topic: Education – Scenarios related to schools, learning, teachers, students, exams, or classroom settings
  \item Grade: Secondary O-level 1
\end{itemize}

\textbf{Solution}

(a) First, express the ratio as fractions:
\[
\text{Blue} : \text{Silver} = \frac{3}{4} : \frac{5}{6}
\]
To work with the ratio, divide both terms by $\frac{3}{4}$, setting blue as 1 part:
\[
1 : \frac{5}{6} \div \frac{3}{4} = 1 : \left(\frac{5}{6} \times \frac{4}{3}\right) = 1 : \frac{20}{18} = 1 : \frac{10}{9}
\]
Alternate approach: express both parts over a common denominator.

Let the ratio be $\frac{3}{4} : \frac{5}{6}$. The common denominator is 12.
\[
\frac{3}{4}=\frac{9}{12},\qquad \frac{5}{6}=\frac{10}{12}
\]
So blue : silver = 9 : 10.

Given 24 grams of blue glitter corresponds to '9 parts', so each part is:
\[
24\div 9 = \frac{24}{9}=2\frac{2}{3} 
\]
Silver (10 parts):
\[
10\times \frac{24}{9} = \frac{240}{9} = 26\frac{6}{9} = 26\frac{2}{3}\text{ grams}
\]

**Answer to (a):** The total mass of silver glitter distributed is $26 \frac{2}{3}$ grams.

(b) Each container has $\frac{5}{12}$ grams. Number of containers:
\[
\text{Number of containers} = \frac{26\frac{2}{3}}{\frac{5}{12}}
\]
Convert $26\frac{2}{3}$ to improper fraction:
\[
26\frac{2}{3} = \frac{80}{3}
\]
\[
\text{Number} = \frac{80}{3} \div \frac{5}{12} = \frac{80}{3} \times \frac{12}{5}=\frac{80\times 12}{3\times 5}=\frac{960}{15}=64
\]

**Answer to (b):** The class prepared \boxed{64} containers of silver glitter.

\section*{Question 9}
\textbf{Metadata}

\begin{itemize}
  \item Question ID: O1-FDMul\_O1-FDAdd\_GPT4.1\_Farming\_01
  \item Primary KC: FRACTIONS AND DECIMALS | Multiplication | Multiplication involving one fraction and one decimal number
  \item Secondary KC: FRACTIONS AND DECIMALS | Addition | Addition involving one fraction and one decimal number
  \item Topic: Farming – Scenarios involving agriculture, livestock, crop production, or farm work
  \item Grade: Secondary O-level 1
\end{itemize}

\textbf{Solution}

First, we find the area used to plant corn:

$$
\text{Corn-planted area} = \frac{3}{4} \times 1200 = 900\ \text{square metres}
$$

Each square metre gives $2.5$ kg of corn:

$$
\text{Mass of corn from main area} = 900 \times 2.5 = 2250\ \text{kg}
$$

Next, $140.5$ kg of corn from another section is added:

$$
\text{Total mass of corn} = 2250 + 140.5 = 2390.5\ \text{kg}
$$

\textbf{Final Answer:} The farmer has $2390.5$ kg of corn in total.

\section*{Question 10}
\textbf{Metadata}

\begin{itemize}
  \item Question ID: O1-AgRepExSq\_O1-AgEvlEx\_GPT4.1\_Transportation\_01
  \item Primary KC: ALGEBRA | Representation and concept | translation of simple real-world situations into quadratic algebraic expressions
  \item Secondary KC: ALGEBRA | Evaluation | evaluation of algebraic expressions and formulae
  \item Topic: Transportation – Problems involving travel, vehicles, routes, distances, or commuting
  \item Grade: Secondary O-level 1
\end{itemize}

\textbf{Solution}

(a) Time taken from Jurong to Tampines at $x$ km/h = $\dfrac{60}{x}$ hours. 

Time taken from Tampines to Jurong at $(x - 10)$ km/h = $\dfrac{60}{x-10}$ hours. 

Total time for the round trip = $\dfrac{60}{x} + \dfrac{60}{x-10}$ hours.

(b) When $x = 40$:

Total time $= \dfrac{60}{40} + \dfrac{60}{30} = 1.5 + 2 = 3.5$ hours.

Therefore, the total time taken for the round trip at $x = 40$ km/h is $3.5$ hours.

\section*{Question 11}
\textbf{Metadata}

\begin{itemize}
  \item Question ID: O1-PcFndRslt\_P3-WNAdd4d\_GPT4.1\_Digital Economy\_01
  \item Primary KC: PERCENTAGE | Finding result after change | increasing/decreasing a quantity by a given percentage (including concept of percentage point)
  \item Secondary KC: WHOLE NUMBERS | Addition | addition up to 4 digits
  \item Topic: Digital Economy – Scenarios involving online platforms, digital products, virtual services, or e-commerce
  \item Grade: Secondary O-level 1
\end{itemize}

\textbf{Solution}

First, calculate the number of customers after a 12\% increase:

Number of customers increased = 1,250 \times \frac{12}{100} = 1,250 \times 0.12 = 150

Total number of subscription customers this month = 1,250 + 150 = 1,400

Now, add the number of single-use digital passes sold:

Total digital products sold = 1,400 + 260 = 1,660

\boxed{1,660}

Thus, the total number of digital products sold this month is 1,660.

\section*{Question 12}
\textbf{Metadata}

\begin{itemize}
  \item Question ID: O1-PcFndRslt\_P3-WNSub4d\_GPT4.1\_Sports\_01
  \item Primary KC: PERCENTAGE | Finding result after change | increasing/decreasing a quantity by a given percentage (including concept of percentage point)
  \item Secondary KC: WHOLE NUMBERS | Subtraction | subtraction up to 4 digits
  \item Topic: Sports – Activities involving games, training, competitions, or physical performance
  \item Grade: Secondary O-level 1
\end{itemize}

\textbf{Solution}

Let last year's score = 1,200 points.

(a) Actual percentage increase = 12\%.

Increase in points = 12\% of 1,200 = \( \frac{12}{100} \times 1,200 = 144 \)

Actual score this year = 1,200 + 144 = 1,344 points

(b) Target percentage increase = 15\%.
Target score = 1,200 + (15\% of 1,200) 
= 1,200 + 0.15 \times 1,200 
= 1,200 + 180 
= 1,380 points

Shortfall = Target score - Actual score = 1,380 - 1,344 = 36 points

\textbf{Answers:}

(a) The team actually scored 1,344 points this year.

(b) The team's score fell short of the target by 36 points.

\section*{Question 13}
\textbf{Metadata}

\begin{itemize}
  \item Question ID: O1-RoRepFr\_P5-FrAddMix\_GPT4.1\_Farming\_01
  \item Primary KC: RATIO | Representation and concept | ratios involving fractions
  \item Secondary KC: FRACTIONS | Addition | adding mixed numbers
  \item Topic: Farming – Scenarios involving agriculture, livestock, crop production, or farm work
  \item Grade: Secondary O-level 1
\end{itemize}

\textbf{Solution}

Let the amount of wheat be $W$ tonnes and the amount of corn be $C$ tonnes.

The ratio is $\dfrac{3}{4} : 2$,
so:

\[\frac{W}{C} = \frac{3/4}{2} = \frac{3/4}{2/1} = \frac{3}{4} \times \frac{1}{2} = \frac{3}{8}\]

Given $W = 3\dfrac{1}{2} = \frac{7}{2}$ tonnes.

Using the ratio:
\[\frac{W}{C} = \frac{3}{8}\]
\[\frac{7}{2C} = \frac{3}{8}\]
\[7 \times 8 = 2C \times 3\]
\[56 = 6C\]
\[C = \frac{56}{6} = \frac{28}{3} = 9\dfrac{1}{3}\text{ tonnes}\]

When the farmer harvested an extra $1\dfrac{1}{4} = \frac{5}{4}$ tonnes of corn,

New total corn = $9\dfrac{1}{3} + 1\dfrac{1}{4}$
Convert to improper fractions:
$9\dfrac{1}{3} = \frac{28}{3}$
$1\dfrac{1}{4} = \frac{5}{4}$

Find common denominator (LCM of 3 and 4 is 12):
$\frac{28}{3} = \frac{112}{12}$
$\frac{5}{4} = \frac{15}{12}$

Add:
$\frac{112}{12} + \frac{15}{12} = \frac{127}{12}$

Convert to mixed number:
$127 \div 12 = 10$ remainder 7
So, $\frac{127}{12} = 10\dfrac{7}{12}$

**Final answers:**
- The farmer originally harvested $9\dfrac{1}{3}$ tonnes of corn.
- After harvesting more, the total amount of corn is $10\dfrac{7}{12}$ tonnes.

\section*{Question 14}
\textbf{Metadata}

\begin{itemize}
  \item Question ID: O1-PcRepRvs\_O1-PcCnv2Dc\_GPT4.1\_Farming\_01
  \item Primary KC: PERCENTAGE | Representation and concept | reverse percentages
  \item Secondary KC: PERCENTAGE | Conversion to decimals | expressing percentage as a decimal
  \item Topic: Farming – Scenarios involving agriculture, livestock, crop production, or farm work
  \item Grade: Secondary O-level 1
\end{itemize}

\textbf{Solution}

First, we express 20\% as a decimal:\\
20\% = \frac{20}{100} = 0.2\\
Let the actual harvested corn be $x$ kg.\\
Since the farmer recorded 20\% less than what was actually harvested, the recorded amount is 80\% of the actual amount.\\
Expressing 80\% as a decimal: 80\% = 0.8\\
So,\\
0.8x = 1,200\\
x = \frac{1,200}{0.8} = 1,500\\
Therefore, the actual amount of corn harvested was 1,500 kilograms.

\section*{Question 15}
\textbf{Metadata}

\begin{itemize}
  \item Question ID: O1-PcFndRslt\_P4-WNMul4d1d\_GPT4.1\_Digital Economy\_01
  \item Primary KC: PERCENTAGE | Finding result after change | increasing/decreasing a quantity by a given percentage (including concept of percentage point)
  \item Secondary KC: WHOLE NUMBERS | Multiplication | multiplication up to 4 digits by 1 digit or up to 3 digits by 2 digits
  \item Topic: Digital Economy – Scenarios involving online platforms, digital products, virtual services, or e-commerce
  \item Grade: Secondary O-level 1
\end{itemize}

\textbf{Solution}

(a) The price increase is 15% of \$12.

Increase = $0.15 \times 12 = \$1.80$

New price = $12 + 1.80 = \$13.80$

(b) Total revenue in February = number of prints \( \times \) new price

= $135 \times 13.80$

= $1,863$

(c) Total revenue in January = $135 \times 12 =\$1,620$

Increase in revenue = $1,863 - 1,620 = \$243$

Thus:

(a) The new price of each digital print is \$13.80.

(b) The total revenue for February is \$1,863.

(c) The total revenue increased by \$243 from January to February.

\section*{Question 16}
\textbf{Metadata}

\begin{itemize}
  \item Question ID: O1-RoRepDc\_P4-DcAdd2d\_GPT4.1\_Transportation\_01
  \item Primary KC: RATIO | Representation and concept | ratios involving decimals
  \item Secondary KC: DECIMALS | Addition | adding decimals (up to 2 decimal places)
  \item Topic: Transportation – Problems involving travel, vehicles, routes, distances, or commuting
  \item Grade: Secondary O-level 1
\end{itemize}

\textbf{Solution}

(a) Amir’s distance = 8.40 km
Siti’s distance = 12.60 km

The ratio (in decimals) = 8.40 : 12.60
To simplify, divide both numbers by 8.40:

8.40 / 8.40 = 1  
12.60 / 8.40 = 1.5

So, the ratio in decimals is 1 : 1.5.

Alternatively, express as decimals: 8.40 : 12.60 = 0.67 : 1 (dividing both by 12.60),
which is also correct, but '1 : 1.5' matches the direct decimal form.

(b) Total fare for Amir = 8.40 × $0.65 = $5.46  
Total fare for Siti = 12.60 × $0.65 = $8.19

Total taxi fare = $5.46 + $8.19 = $13.65

Final Answers:
(a) Ratio of Amir’s distance to Siti’s distance is 1 : 1.5 in decimal form.
(b) The total taxi fare for both is $13.65.

\section*{Question 17}
\textbf{Metadata}

\begin{itemize}
  \item Question ID: O1-PcFndRslt\_P4-WNDiv4d1d\_GPT4.1\_Food\_01
  \item Primary KC: PERCENTAGE | Finding result after change | increasing/decreasing a quantity by a given percentage (including concept of percentage point)
  \item Secondary KC: WHOLE NUMBERS | Division | division up to 4 digits by 1 digit
  \item Topic: Food – Situations involving meals, groceries, ingredients, or food preparation and consumption
  \item Grade: Secondary O-level 1
\end{itemize}

\textbf{Solution}

(a) First, calculate the 25\% increase:

Number of bowls increased = 25\% of 1,824 = \frac{25}{100} \times 1,824 = 456

New number of bowls = 1,824 + 456 = 2,280

(b) Divide the new total number of bowls equally among 8 outlets:

Number of bowls per outlet = 2,280 \div 8 = 285

\boxed{285} bowls will be allocated to each outlet.

\section*{Question 18}
\textbf{Metadata}

\begin{itemize}
  \item Question ID: O1-PcFndRslt\_P4-WNDiv4d1d\_GPT4.1\_Sports\_01
  \item Primary KC: PERCENTAGE | Finding result after change | increasing/decreasing a quantity by a given percentage (including concept of percentage point)
  \item Secondary KC: WHOLE NUMBERS | Division | division up to 4 digits by 1 digit
  \item Topic: Sports – Activities involving games, training, competitions, or physical performance
  \item Grade: Secondary O-level 1
\end{itemize}

\textbf{Solution}

(a) First, find 15\% of 1248:

15\% of 1248 = \frac{15}{100} \times 1248 = 0.15 \times 1248 = 187.2

So, the increase is 187.2 basketballs. Add this to the original number:

1248 + 187.2 = 1435.2

Since the number of basketballs must be a whole number, the coach now has 1435 basketballs (if we round down to the nearest whole number).

(b) Divide 1435 basketballs equally among 8 teams:

1435 \div 8 = 179 \;\text{R}\;3

So, each team gets 179 basketballs. 3 basketballs will be left undistributed.

\section*{Question 19}
\textbf{Metadata}

\begin{itemize}
  \item Question ID: O1-AgRepExSq\_O1-AgEvlEx\_GPT4.1\_Food\_01
  \item Primary KC: ALGEBRA | Representation and concept | translation of simple real-world situations into quadratic algebraic expressions
  \item Secondary KC: ALGEBRA | Evaluation | evaluation of algebraic expressions and formulae
  \item Topic: Food – Situations involving meals, groceries, ingredients, or food preparation and consumption
  \item Grade: Secondary O-level 1
\end{itemize}

\textbf{Solution}

(a) The amount of punch James can make is given by the quadratic expression: $V = n^2 + 4n$, where $n$ is the number of fruit juice cartons.

(b) By substituting $n = 3$ into the expression:

$V = (3)^2 + 4 \times 3 = 9 + 12 = 21$

Therefore, if James uses 3 cartons of fruit juice, he can make 21 litres of punch.

\section*{Question 20}
\textbf{Metadata}

\begin{itemize}
  \item Question ID: O1-FDDiv\_O1-FDAdd\_GPT4.1\_Digital Economy\_01
  \item Primary KC: FRACTIONS AND DECIMALS | Division | Division involving one fraction and one decimal number
  \item Secondary KC: FRACTIONS AND DECIMALS | Addition | Addition involving one fraction and one decimal number
  \item Topic: Digital Economy – Scenarios involving online platforms, digital products, virtual services, or e-commerce
  \item Grade: Secondary O-level 1
\end{itemize}

\textbf{Solution}

(a) Let the regular price be $x$.\
\
Sarah paid $\frac{3}{4}$ of this regular price, which equals $6.8.$\
\[
\frac{3}{4}x = 6.8
\]
To find $x$, divide both sides by $\frac{3}{4}$ (which is equivalent to multiplying by $\frac{4}{3}$):\
\[
x = 6.8 \div \frac{3}{4} = 6.8 \times \frac{4}{3} = \frac{6.8 \times 4}{3} = \frac{27.2}{3} = 9.07\
\]
So, the regular price of the first e-book is $9.07.\$
\
(b) The amount Sarah pays for the first e-book is $6.8. She also buys another e-book for $2.5$.\
\[
\text{Total cost} = 6.8 + 2.5 = 9.3
\]
Sarah has to pay $9.30 in total for both e-books.

\end{document}
